\part{Calculs de limites}

\begin{rap}
	Soient $f$ et $g$ deux fonctions et $a \in \R \cup \{-\infty, +\infty\}$.

	On ne connaît pas, à l'avance, les limites de
	\begin{itemize}
		\item $f(x) - g(x)$ si $\begin{cases}
				f(x) \tendsto{x \to a} +\infty\\
				g(x) \tendsto{x \to a} +\infty
			\end{cases}$ \hfill (``$\infty-\infty$'')\\
		\item $\frac{f(x)}{g(x)}$ si $\begin{cases}
				f(x) \tendsto{x \to a} 0\\
				g(x) \tendsto{x \to a} 0
			\end{cases}$ \hfill (``$\textstyle\frac{0}{0}$'')\\
		\item $\frac{f(x)}{g(x)}$ si $\begin{cases}
				f(x) \tendsto{x \to a} \pm\,\infty\\
				g(x) \tendsto{x \to a} \pm\,\infty
			\end{cases}$ \hfill (``$\textstyle\frac{\infty}{\infty}$'')\\
		\item $f(x) \times g(x)$ si $\begin{cases}
				f(x) \tendsto{x \to a} 0\\
				g(x) \tendsto{x \to a} \pm\,\infty\\
			\end{cases}$ \hfill (``$0\times \infty$'')\\
	\end{itemize}
\end{rap}

\begin{exm}
	$\lim_{n \to +\infty}\left( 1 + \frac{1}{n} \right)^n$ ?

	\[
		\forall n \in \N, \left( 1 + \frac{1}{n} \right)^n = e^{n\ln\left(1 + \frac{1}{n}\right)}
	\] et \[
		\begin{cases}
			\ln\left( 1 + \frac{1}{n} \right) \tendsto{n \to +\infty} 0\\
			n \tendsto{n\to +\infty} +\infty
		\end{cases}
	\] donc c'est une forme indéterminée.
\end{exm}

\begin{prop}~\\
	Si $\begin{cases}
		f(x)\tendsto{x \to a} 1\\
		g(x)\tendsto{x \to a} \pm \infty
	\end{cases}$ alors, on ne sait pas à l'avance calculer $\lim_{x\to a} \big(f(x)\big)^{g(x)}$.\qed
\end{prop}

\begin{defn}
	Soient $f$ et $g$ deux fonctions et $a \in \overline{\R}$ où $\overline{\R} = \R\cup \{+\infty, -\infty\}$.

	On dit que $f$ et $g$ sont \underline{équivalentes au voisinage de $a$} (ou éqivalentes en $a$) s'il existe une fonction $u$ telle que \[
		\begin{cases}
			f = g \times u\\
			u(x) \tendsto{x \to a} 1
		\end{cases}
	\] On note alors $f\simi_a g$ ou $f(x)\simi_{x\to a}g(x)$.
	\index{équivalence (fonctions réelles)}
\end{defn}

\begin{prop}
	Un polynôme est équivalent en $\pm\infty$ à son terme de plus haut degré.
\end{prop}

\begin{prv}
	Soit $P: x \mapsto \sum_{k=0}^n a_kx^k$ avec $a_n \neq 0$.
	On pose $Q: x \mapsto a_nx^n$.

	\begin{align*}
		\forall x \in \R, P(x) &= a_nx^n \left( \sum_{k=0}^n \frac{a_kx^k}{a_nx^n} \right) \\
		&= Q(x)\Bigg( 1 + \sum_{k=0}^{n-1} \frac{a_k}{a_n} \times \underbrace{\left( \frac{1}{x^{n-k}} \right)}_{u(x)}\Bigg) \\
		&= Q(x)\,u(x) \\
	\end{align*}

	On a $u(x)\tendsto{x\to \pm\infty}$ 1 donc $P(x)\simi_{x\to +\infty}Q(x)$.
\end{prv}

\begin{prop}
	Un polynôme est équivalent en $0$ à sont terme de plus bas degré.
\end{prop}

\begin{prv}
	À faire
\end{prv}

\begin{rmk}
	Soient $f$ et $g$ deux fonctions définies au voisinage de $a$ tel que \[
		\forall x \in I, g(x) \neq 0
	\] où $I$ est un intervalle
	\begin{itemize}
		\item qui contient $a$ si $a \in \R$,
		\item dont une borne est $a$ si $a = \pm\infty$.
	\end{itemize}

	Alors, \[
		f\simi_ag \iff \frac{f(x)}{g(x)}\tendsto{\substack{x\to a\\\neq}} 1.
	\]
\end{rmk}

\begin{exm}
	$x^2+x^3\simi_{x\to 0}x^2$ car $\frac{x^2+x^3}{x^2} = 1+x \tendsto{x\to 0}1$.
\end{exm}

\begin{exm}
	Soit $f$ une fonction. \[
		f \simi_0 0 \iff \exists I \text{ voisinage de } a, \forall x \in I, f(x) = 0.
	\]
\end{exm}
