\part{Repère affine}

\begin{defn}
	Soit $F$ un sous-espaca affine de $E$. Un \underline{repère} de $F$ est la donnée d'un point $A \in F$ (``l'origine du repère'') et d'une base $\mathcal{B} = (\vec{e_i})_{i \in I}$ de $\vec{F}$ (``vecteurs direction''). \index{repère (espace affine)}
\end{defn}

\begin{exm}[mécanique -- physique]
	On pose $E = (\R^3, \R^3, +)$, un $\R$-espace affine. $(O, \vec{u_x}, \vec{u_y}, \vec{u_z})$ est une base de $E$ : la base carthésienne. Il existe plusieurs bases de $E$ : $(O, \vec{u_r}, \vec{u_\theta}, \vec{u_z})$ (base cylindrique) et $(O, \vec{u_r}, \vec{u_\theta}, \vec{u_\varphi})$ (base sphérique) sont deux autres bases de $E$.

	\begin{figure}[H]
		\centering
		\begin{asy}
			import three;

			size(5cm);

			texpreamble("\usepackage[f]{esvect}");

			draw(-2.5X -- 2.5X);
			draw(-2.5Y -- 2.5Y);
			draw(-2.5Z -- 2.5Z);

			draw(O -- X, magenta, Arrow3(TeXHead2));
			draw(O -- Y, magenta, Arrow3(TeXHead2));
			draw(O -- Z, magenta, Arrow3(TeXHead2));

			label("$\vv{u_x}$", X/2, magenta, align=S);
			label("$\vv{u_y}$", Y/2, magenta, align=S);
			label("$\vv{u_z}$", Z/2, magenta, align=W);

			dot("$O$", O, deepcyan, align=1.5N + E);
		\end{asy}
	\end{figure}
\end{exm}

\begin{prop-defn}
	Soit $F$ un sous-espace affine de $E$, et $\mathcal{R} = (A, \vec{e_1}, \ldots, \vec{e_n})$ un repère de $F$. Alors, pour tout $B \in F$, \[
		\exists ! (\lambda_1, \ldots, \lambda_n) \in \mathbbm{K}^n,\;B = A + \sum_{i=1}^n \lambda_i \vec{e_i}
	.\] On dit que $(\lambda_1, \ldots, \lambda_n)$ sont les \underline{coordonées} de $B$.\index{coordonnées (espace affine)}
\end{prop-defn}


\begin{rmk}
	Les solutions d'un problème linéaire forment un espace sous-affine de direction les solutions du système homogène associé.
\end{rmk}
