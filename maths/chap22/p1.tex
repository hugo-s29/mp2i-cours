\part{Topologie de $\R^2$}

\begin{defn}
	La \underline{norme (euclidienne)} de $\R^2$ est l'application définie par \[
		\forall (x,y) \in \R^2, \|(x,y)\| = \sqrt{x^2 + y^2}.
	\]

	\begin{figure}[H]
		\centering
		\begin{asy}
			import graph;
			axes(EndArrow);
			size(4cm);
			pair A = (3,2);
			dot(A);
			draw((3,0)--A, dashed);
			draw((0,2)--A, dashed);
			label("$x$", (3,0), align=S);
			label("$y$", (0,2), align=W);
			draw((0,0)--A);
			dot((4,3), white+0);
		\end{asy}
	\end{figure}
\end{defn}

\begin{prop}
	La norme euclidienne vérifie:
	\begin{enumerate}
		\item (séparation) \[
			\forall (x,y) \in \R^2, \|(x,y)\| = 0 \iff x = y = 0,
			\]
		\item (homogénéité positive) \[
				\forall \lambda \in \R, \forall (x,y) \in \R^2, \|\lambda(x,y)\|= \left| \lambda \right| \|(x,y)\|
			\]
		\item (inégalité triangulaire) \[
			\forall (x,y), (a,b) \in \R^2,
			\|(x,y)+(a,b)\|\le \|(x,y)\|+\|(a,b)\|.
		\]
	\end{enumerate}
\end{prop}

\begin{prv}
	Déjà vue en replaçant $(x,y)$ par $x+iy \in \C$ et $\|(x,y)\|$ par |x+iy|
\end{prv}

\begin{defn}
	Soit $(a,b) \in \R^2$ et $r \in \R_+$.

	La \underline{boule ouverte} (ou \underline{disque ouvert}) de centre $(a,b)$ et de rayon $r$ est \[
		B_{(a,b)}(r) = \big\{ (x,y) \in \R^2  \mid \|(x,y) - (a,b)\| < r \big\}.
	\]

	La \underline{boule fermée} (ou \underline{disque fermé}) de centre $(a,b)$ et de rayon $r$ est \[
		\overline{B_{(a,b)}}(r) = \big\{ (x,y)\in \R^2  \mid \|(x,y) - (a,b)\| \le r \big\}.
	\]

	La \underline{sphère} (ou \underline{boule}) de centre $(a,b)$ et de rayon $r$ est \[
		S_{(a,b)}(r) = \partial \overline{B_{(a,b)}}(r) = \big\{ (x,y) \in \R^2  \mid \|(x,y) - (a,b)\| = r \big\}.
	\]
\end{defn}

\begin{figure}[H]
		\centering
		\incfig{boule}
\end{figure}

\begin{rmk}
	On parle de boule en dimension quelconque.
\end{rmk}

\begin{defn}
	Une \underline{partie ouverte} $O$ de $\R^2$ (ou \underline{un ouvert}) si \[
		\forall (x,y) \in O, \exists r > 0, B_{(a,b)}(r) \subset O.
	\]
	Une partie $F$ est \underline{fermée} su $\R^2\setminus F$ est ouverte.
\end{defn}

\begin{figure}[H]
	\centering
	\incfig{partie-ouverte}
\end{figure}

\begin{prop}
	Une boule ouverte est ouverte. Une boule fermée est fermée.
\end{prop}

\begin{figure}[H]
	\centering
	\begin{subfigure}{4cm}
		\centering
		\begin{asy}
			import patterns;

			pair n(pair a) {return a / length(a);}

			add("hatch",hatch(2mm, SW, red));
			size(4cm);

			draw(circle((0,0), 1));
			dot('$(a_0, b_0)$', (0,0), align=S);

			draw((0,0) -- n((-1, 1)), dashed);
			label("$r$", n((-1, 1)) / 2, align=1.5S);

			pair A = n((1,3)) * (2/3);
			real rho = (1 - length(A)) * (2 / 3);

			dot("$(a,b)$", A, red, align=3SE);
			filldraw(circle(A, rho), pattern("hatch"), red);

			label("$O$", n((1,-1))*2.5/3);
		\end{asy}
	\end{subfigure}
	\begin{subfigure}{1cm}
		\centering~\\
	\end{subfigure}
	\begin{subfigure}{5cm}
		\centering
		\begin{asy}
			import patterns;

			pair n(pair a) {return a / length(a);}

			add("hatch",hatch(1mm, SW, red));
			add("hatch2",hatch(3mm, SE, blue));
			size(5cm);

			guide around = (-1.5, -1.5) -- (-1.5, 1.5) -- (2.5, 1.5) -- (2.5, -1.5) -- cycle;

			pair A = n((3, 1)) * 5/3; 
			real rho = 2 / 9;

			picture inter;
			fill(inter, around, pattern("hatch2"));
			fill(inter, circle((0,0), 1), white);
			add(inter);

			draw(circle((0,0), 1));
			dot('$(a_0, b_0)$', (0,0), align=S);

			draw((0,0) -- n((-1, 1)), dashed);
			label("$r$", n((-1, 1)) / 2, align=1.5S);

			dot("$(a,b)$", A, red, align=2SE);
			filldraw(circle(A, rho), pattern("hatch"), red);

			label("$F$", n((1,-1))*2.5/3);
		\end{asy}
	\end{subfigure}
\end{figure}

\begin{prv}
	$\O$ est un ouvert.

	Soit $B$ la boule ouverte de centre $(a_0, b_0) \in \R^2$ et de rayon $r > 0$.

	On pose $\rho = \frac{1}{2}\big(r - \|(a,b) - (a_0,b_0)\|\big)$.
	Montrons que \[
		B_{(a,b)}(\rho) \subset  B_{(a,b)}(r).
	\]

	Soit $(x,y) \in B_{(a,b)}(\rho)$.
	\begin{align*}
		\|(x,y) - (a_0,b_0)\|&= \|(x,y)- (a,b) + (a,b) - (a_0,b_0)\| \\
		&\le \|(x,y) - (a,b)\| + \|(a,b) - (a_0, b_0)\|\\
		&< \rho + \|(a,b) - (a_0, b_0)\| = \frac{1}{2}r + \frac{1}{2} \|(a,b) - (a_0, b_0)\|\\
		&< r
	\end{align*}
	
	Soit $F$ la boule fermée de centre $(a_0, b_0)$ et de rayon $r \ge 0$.

	Soit $(a,b) \not\in F$. On pose \[
		\rho = \frac{1}{2}\big(\|(a,b) - (a_0, b_0)\| - r\big) > 0.
	\]

	Montrons que $B_{(a,b)}(\rho) \subset \R^2\setminus F$.

	Soit $(x,y) \in B_{(a,b)}(\rho)$.

	\begin{align*}
		\|(x,y) - (a_0, b_0)\| &= \|(x,y) - (a,b) + (a,b) - (a_0, b_0)\| \\
		&\ge \big| \underbrace{\|(x,y) - (a,b)\|}_{\le \rho} - \underbrace{\|(a,b) - (a_0, b_0)\|}_{> r} \big|\\
		&\ge \|(a,b) - (a_0, b_0)\|- \|(x,y) - (a,b)\|\\
		&> \|(a,b) - (a_0, b_0)\|- \rho\\
		&> \frac{1}{2} \|(a,b) - (a_0, b_0)\| + \frac{1}{2}r\\
		&> r
	\end{align*}

	donc $(x,y) \not\in F$.
\end{prv}

\begin{exm}
	\begin{enumerate}
		\item $\O$ est ouvert.\\
			$\R^2$ est ouvert.
		\item $\O$ est fermé.\\
			$\R^2$ est fermé.\\
		\item $\big\{(x,0)  \mid x > 0\big\}$ n'est ni ouverte ni fermé.
	\end{enumerate}
\end{exm}

\begin{figure}[H]
	\centering
	\begin{asy}
		size(3cm);

		draw((0, -1) -- (0, 3), Arrow(TeXHead));
		draw((-1, 0) -- (3, 0), Arrow(TeXHead));
		
		draw((0,0) -- (0, 2.97), red);
		draw(circle((0,1.5), 0.5), deepred);
		draw(circle((0,0.5), 0.1), deepred);
	\end{asy}
\end{figure}

\begin{defn}
	Soit $(a,b) \in \R^2$ et $V \in \mathcal{P}(\R^2)$.

	On dit que $V$ est un voisinage de $(a,b)$ s'il existe $r > 0$ tel que \[
		B_{(a,b)}(r) \subset V.
	\]
\end{defn}

\begin{prop}
	Un ouvert non vide est un voisinage en chacun de ces points. \qed
\end{prop}

\begin{defn}
	Soit $D \subset \R^2$. Un \underline{point intérieur} de $D$ est un couple $(a,b) \in D$ tel que \[
		\exists r > 0, B_{(a,b)}(r) \subset D.
	\] en d'autres termes, si $D$ est un voisinage de $(a,b)$.

	On note $\mathring D$ l'ensemble des points interieurs à $D$. C'est \underline{l'intérieur} de $D$.
\end{defn}

\begin{prop}
	$\mathring D$ est le plus grand ouvert $O$ de $\R^2$ tel que $O \subset D$.
\end{prop}

\begin{figure}[H]
	\centering
	\incfig{interieur-plus-grand-ouvert}
\end{figure}


\begin{prv}
	Soit $(a,b) \in \mathring D$.

	Par définition, il existe $r > 0$ tel que \[
		B_{(a,b)}(r) \subset D.
	\] Montrons que $B_{(a,b)}(r) \subset \mathring D$.

	Soit $(x,y) \in B_{(a,b)}(r)$. Comme $B_{(a,b)}(r)$ est un ouvert de $\R^2$, il existe $\rho > 0$ tel que \[
		B_{(x,y)}(\rho) \subset B_{(a,b)}(r)
	\] donc $(x,y) \in \mathring D$.

	Donc $\mathring D$ est ouvert, $\mathring D \subset D$.

	Soit $O$ un ouvert de $\R^2$ tel que $O \subset D$. Montrons que $O \subset \mathring D$.

	Soit $(x,y) \in O$. Soit $r > 0$ tel que \[
		B_{(x,y)}(r) \subset O \subset D
	\] donc $(x,y) \in \mathring D$.
\end{prv}

\begin{defn}
	Soit $f: D \subset \R^2 \to \R$, $\ell \in \R$, $(a,b) \in \mathring D$.

	On dit que \underline{$f(x,y)$ tend vers $\ell$ quand $(x,y)$ tend vers $(a,b)$} ou que $\ell$ est \underline{une limite} de $f$ en $(a,b)$ si \[
		\forall \varepsilon > 0, \exists r > 0, \forall (x,y) \in D, \|(x,y) - (a,b)\| < r \implies \left| f(x,y) - \ell \right| \le \varepsilon.
	\] en d'autres termes si \[
		\forall V \in \mathcal{V}_{\ell}, \exists W \in \mathcal{V}_{(a,b)}, \forall (x,y) \in W \cap D, f(x,y) \in V.
	\]
\end{defn}

\begin{prop}
	[unicité de la limite]
	Soit $f: D \to \R$, $(a,b) \in \mathring D$, $\ell_1, \ell_2 \in \R$ telles que $\ell_1$ et $\ell_2$ sont des limites de $f$ en $(a,b)$.

	Alors $\ell_1 = \ell_2$.
\end{prop}

\begin{figure}[H]
	\centering
	\incfig{preuve-unicité-de-la-limite}
\end{figure}

\begin{prv}
	On suppose $\ell_1 < \ell_2$. On pose $\varepsilon = \frac{\ell_2 - \ell_1}{2} > 0$.

	Soit $r_1 > 0$ tel que \[
		f\big(B_{(a,b)}(r_1)\big) \subset ]\ell_1 - \varepsilon, \ell_1 + \varepsilon[.
	\] Soit $r_2 > 0$ tel que \[
		f\big(B_{(a,b)}(r_2)\big) \subset ]\ell_2 - \varepsilon, \ell_2 + \varepsilon [.
	\] On pose $r = \min(r_1, r_2)$ donc \[
		B_{(a,b)}(r_1) \cap B_{(a,b)}(r_2) = B_{(a,b)}(r) \neq \O.
	\] Soit $(x,y) \in B_{(a,b)}(r)$. Alors, \[
		f(x,y) \in ]\ell_1 - \varepsilon, \ell_1 + \varepsilon[ \cap ]\ell_2 - \varepsilon, \ell_2 + \varepsilon[ = \O.
	\] $\lightning$
\end{prv}

\begin{defn}
	Soit $f : D \to \R$, $(a,b) \in \mathring D$.

	On dit que $f$ est \underline{continue} en $(a,b)$ si \[
		f(x,y) \tendsto{(x,y) \to (a,b)}f(a,b).
	\]
\end{defn}

\begin{prop}
	\underline{Si} $f(x,y) \tendsto{(x,y) \to (a,b)} \ell$ \\
	\underline{alors} $\begin{cases}
		f(x,b) \tendsto{x \to a} \ell\\
		f(a,y) \tendsto{y \to b} \ell.\\
	\end{cases}$
\end{prop}

\begin{prv}~\\
	\begin{figure}[H]
		\centering
		\incfig{limite-x-en-a-et-y-en-b}
	\end{figure}
\end{prv}

\underline{Contre-exemple} : exercice 3.

\begin{exm}
	\begin{enumerate}
		\item $f : \begin{array}{rcl}
				\R^2 &\longrightarrow& \R \\
				(x,y) &\longmapsto& x
			\end{array}$ limite en $(0,0)$ ?

			Soit $\varepsilon > 0$. On pose $r = \varepsilon$. \[
				\forall (x,y) \in B_{(0,0)}(r),
				\left| f(x,y) \right| = \left| x \right| \le \|(x,y)\| < r = \varepsilon
			\] Donc $f(x,y) \tendsto{(x,y) \to (a,b)} 0$.
		\item limite $f : \begin{array}{rcl}
				\R^2 &\longrightarrow& \R \\
				(x,y) &\longmapsto& x^3
			\end{array}$ en $(0,0)$ ?

			Soit $\varepsilon > 0$. On pose $r = \sqrt[3]{r} > 0$. \[
				\forall (x,y) \in B_{(0,0)}(r),
				\left| f(x,y) \right| = \left| x^3 \right| \le \|(x,y)\|^3 < r^3 = \varepsilon.
			\]
		\item limite de $f : \begin{array}{rcl}
			\R^2 &\longrightarrow& \R \\
			(x,y) &\longmapsto& x^3y^2
		\end{array}$ en $(0,0)$ ?

		Soit $\varepsilon > 0$. On pose $r = \sqrt[5]{\varepsilon} > 0$. \[
			\forall (x,y) \in B_{(0,0)}(r), \left| f(x,y) \right| = \left| x^3 y^2 \right| \le \|(x,y)\|^3 \|(x,y)\|^2 < r^5 = \varepsilon.
		\]
	\end{enumerate}
\end{exm}

\begin{defn}
	Soient $D \subset \R^2$ et $(x,y) \in \R^2$.

	\begin{figure}[H]
    \centering
    \incfig{point-adhérent}
	\end{figure}
	
	On dit que $(x,y)$ est \underline{adhérent} à $D$ si \[
		\forall r > 0, B_{(x,y)}(r) \cap D \neq \O.
	\] L'ensemble des points adhérents à $D$ est noté $\overline{D}$. On dit que $\overline{D}$ est \underline{l'adhérence} de $D$.
\end{defn}

\begin{defn}
	Soit $f: D \subset \R^2 \to \R$ et $(a,b) \in \overline{D}$, $\ell \in \R$. On dit que $f$ tend vers $\ell$ quand $(x,y)$ tend vers $(a,b)$ si \[
		\forall \varepsilon > 0, \exists r > 0, \forall (x,y) \in B_{(a,b)}(r) \cap D,
		\left| f(x,y) - \ell \right| \le \varepsilon.
	\]
\end{defn}

\begin{prop}
	\begin{enumerate}
		\item Dans ce contexte, il y a unicité de la limite
		\item La limite d'une somme, d'un produit, d'un quotien, d'une composée se comporte comme dans le cas d'une seule variable.
		\item Soit $f: D \to \R$ continue. Soient $g: I \to \R$ et $h: I \to \R$ continues telles que \[
			\forall t \in I, \big(g(t), h(t)\big) \in D.
		\] Alors \[
			t \in I \mapsto f\big(g(t), h(t)\big) \in \R
		\] est continue.
	\end{enumerate}
\end{prop}

\begin{figure}[H]
	\centering
	\begin{asy}
		import three;
		import graph3;
		size(5cm);

		settings.render = 0;
		settings.prc = false;
		currentprojection = obliqueX;

		draw(O -- X, Arrow3(TeXHead2));
		draw(O -- Y, Arrow3(TeXHead2));
		draw(O -- Z, Arrow3(TeXHead2));

		triple f(real x, real y, real z = 0) { return (x,y,cos(x - 0.5) * cos(y - 0.5)/1.2 + 0.15); }

		real inc = 1 / 5;

		for(real x = 0; x <= 1; x += inc) {
			draw(graph(
				new real(real t) { return x; }, // x
				new real(real y) { return y; }, // y
				new real(real y) { return f(x,y).z; }, // z
				0, 1
			), gray);
		}

		for(real y = 0; y <= 1; y += inc) {
			draw(graph(
				new real(real x) { return x; }, // x
				new real(real t) { return y; }, // y
				new real(real x) { return f(x,y).z; }, // z
				0, 1
			), gray);
		}

		path3 path1 = (0.3, 0.2, 0) .. (0.5, 0.5, 0) .. (0.6, 0.7, 0) .. (0.9, 0.8, 0);
		path3 path2 = (0.3, 0.8, 0) .. (0.5, 0.5, 0) .. (0.6, 0.3, 0) .. (0.9, 0.2, 0);
		path3 pathA = f(0.3, 0.2, 0) .. f(0.5, 0.5, 0) .. f(0.6, 0.7, 0) .. f(0.9, 0.8, 0);
		path3 pathB = f(0.3, 0.8, 0) .. f(0.5, 0.5, 0) .. f(0.6, 0.3, 0) .. f(0.9, 0.2, 0);

		draw(path1, red, Arrow3(TeXHead2, position=0.5));
		draw(pathA, red, Arrow3(TeXHead2, position=0.5));
		draw(path2, deepcyan, Arrow3(TeXHead2, position=0.3));
		draw(pathB, deepcyan, Arrow3(TeXHead2, position=0.3));

		dot((0.5, 0.5, 0));
		dot(f(0.5, 0.5, 0));
		draw((0.5, 0.5, 0) -- f(0.5, 0.5, 0), dashed);
	\end{asy}
\end{figure}

