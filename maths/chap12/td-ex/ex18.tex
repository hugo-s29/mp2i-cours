\part{Exercice 18}

Montrons que 
\begin{itemize}
	\item $(\R,\oplus)$ est un groupe abélien
	\item $(\R,\otimes)$ est un monoïde commutatif
	\item $\forall x \in \R \setminus \{0_\R\}, \exists y \in \R, x \otimes y = 1_\R$
	\item $1_\R \neq  0_\R$
\end{itemize}\vspace{2mm}

Trouvons $1_{\R}$ et $0_\R$ :\\
Soit $a \in \R$
\begin{align*}
	0_\R \oplus a = a &\iff 0_\R + a - 1 = a\\
										&\iff 0_\R = 1
\end{align*}
\begin{align*}
	1_\R \otimes a = a &\iff 1_\R + a - 1_\R a = a\\
										 &\iff 1_\R (1-a) = 0\\
										 &\iff a = 1 \text{ ou } 1_\R = 0
\end{align*}

Or, $1_\R \otimes a = a$ est vrai pour toutes valeures de $a$ donc $1_\R = 0$ \\

\fbox{$1_\R \neq  0_\R$}\\

Soient $(x,y,z) \in \R^3$.
Montrons que $(x \oplus y) \oplus z = x \oplus (y \oplus z)$
\[
	\begin{array}{c r c l}
		&(x \oplus y) \oplus z &=& x \oplus (y \oplus z) \\
		\iff& (x + y - 1) \oplus z &=& x \oplus (y + z - 1)\\
		\iff& (x+y-1) + z - 1 &=& x + (y + z -1) -1 \\
		\iff& x+y+z-2 &=& x+y+z-2
	\end{array}
\]
Donc, $\oplus$ est associative.

Montrons que $x \oplus y = y \oplus x$ i.e. $x + y -1 = y + x - 1$
Donc  $\oplus$ est commutative.

On sait que $0_\R = 1 \in \R$

Soit $x \in \R$.
Trouvons $y \in \R$ tel que $x \oplus y = 0_{\R} = 1$ \\
\begin{align*}
	x \oplus y = 1 &\iff x + y - 1 = 1\\
								 &\iff y = 2-x \in \R
\end{align*}

\fbox{Donc $(\R,\oplus)$ est un groupe abélien}
