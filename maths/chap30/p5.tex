\part{Fonctions continues par morceaux}

\begin{defn}
	Soit $f : [a,b] \longrightarrow \R$. On dit que $f$ est \underline{continue par morceaux}\index{continuité par morceaux} si,
	\[
		\exists \sigma = (x_0,\ldots,x_n) \in \mathfrak{S}_{[a,b]}, \begin{cases}
			\forall i \in \left\llbracket 0,n-1 \right\rrbracket,\,f_{\big|]x_i,x_{i+1}[\big.} \text{ est continue};\\[2mm]
			\forall i \in \left\llbracket 1,n-1 \right\rrbracket,\,\lim_{\substack{x \to x_i\\<}}f(x) \in \R \text{ et } \lim_{\substack{x \to x_i\\>}}f(x) \in \R;\\
			\lim_{\substack{x \to a\\>}}f(x) \in \R;
			\lim_{\substack{x \to b\\<}}f(x) \in \R.
		\end{cases}
	\]
\end{defn}

\begin{thm}
	Toute fonction continue par morceaux sur un segment est Riemann-intégrable sur ce segment.
	\qed
\end{thm}

\begin{defn}
	Soit $f : I \subset \R\longrightarrow \R$. On dit que $f$ est \underline{continue par morceaux sur $I$}\index{continuité par morceaux sur un intervalle} si $f$ est continue par morceaux sur tout segment inclus dans $I$.
\end{defn}

\begin{rmk}~\\
	\begin{figure}[H]
		\centering
		\begin{asy}
			import graph;

			size(5cm);

			label("$\delta$", (4, 0), magenta, align=SE);
			draw((-5.3, 0) -- (5.3, 0), Arrow(TeXHead));
			draw((-5,0) -- (5,0), magenta);
			dot((0,0), white);
			draw((0, -0.3) -- (0, 1.5), Arrow(TeXHead));
			dot((0,1), magenta);
		\end{asy}
	\end{figure}

	\begin{align*}
		\delta: \R &\longrightarrow \R \\
		x &\longmapsto \begin{cases}
			0 &\text{ si } x \neq 0,\\
			1 &\text{ si } x = 0.
		\end{cases}
	\end{align*}

	$\delta$ est continue par morceaux sur $\R$. On a $\int_{[0,1]} \delta = 0$, et $\forall x \in [0,1],\;f(x) \ge 0$. Mais, $\delta \neq 0$.
\end{rmk}

