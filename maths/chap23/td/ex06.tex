\part{Exercice 6}

On numérote les pays de $1$ à $N$. On pose \begin{align*}
	f: \left\llbracket 1,N \right\rrbracket &\longrightarrow \left\llbracket 0, N - 1 \right\rrbracket \\
	p &\longmapsto \text{ le nombre de voisins du pays numéro } p
\end{align*}

On suppose $f$ injective. Alors $f$ est surjective : $\exists p, f(p) = 0$ et $\exists q, f(q) = N - 1$. Donc $p$ est isolé est $q$ est voisin de tous les pays. Donc $p$ et $q$ sont voisins. $\lightning$.

Donc $f$ n'est pas injective donc \[
	\exists p \neq q, f(p) = f(q).
\]

