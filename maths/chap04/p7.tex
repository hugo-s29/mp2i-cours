\part{Trigonométrie hyperbolique}

\begin{defn}
	Pour tout $x \in \R$, on pose \[
		\begin{cases}
			\ch x = \frac{e^x + e^{-x}}{2},\\
			\sh x = \frac{e^x - e^{-x}}{2},\\
			\th x = \frac{\sh x}{\ch x}.
		\end{cases}
	\]

	$\ch$ est appelé \underline{cosinus hyperbolique}, $\sh$ est appelé \underline{sinus hyperbolique} et $\th$ est appelé \underline{tangeante hyperbolique}.
	\index{cosinus hyperbolique}
	\index{sinus hyperbolique}
	\index{tangente hyperbolique}
\end{defn}

\begin{rmk}
	Ces formules rappèlent les formules d'Euler : pour tout $x \in \R$,
	\begin{align*}
		\cos x = \frac{e^{ix} + e^{-ix}}{2}\quad\longleftrightarrow\quad\ch x = \frac{e^x + e^{-x}}{2}\\
		\sin x = \frac{e^{ix} - e^{-ix}}{2i}\quad\longleftrightarrow\quad\sh x = \frac{e^x - e^{-x}}{2}\\
	\end{align*}
\end{rmk}

\begin{figure}[H]
	\centering
	\begin{asy}
		import graph;

		size(12cm);

		pair A = (-2, 0);
		pair B = (2, 0);

		real eps = 0.05;

		draw(shift(A) * ((0, -1.3) -- (0, 1.3)), Arrow(TeXHead));
		draw(shift(A) * ((-1.3, 0) -- (1.3, 0)), Arrow(TeXHead));

		draw(circle(A, 1), magenta);

		real theta = 38;
		pair M = dir(theta) + A;
		draw(A -- M, red);
		draw(arc(A, 0.35, 0, theta), red, Arrow(TeXHead));
		draw(M -- (A.x-eps, M.y), dashed);
		draw(M -- (M.x, A.y-eps), dashed);
		label("\small$\theta$", 0.5dir(theta/2) + A, red);
		label("\small$\cos\theta$", (M.x, A.y), align=S);
		label("\small$\sin\theta$", (A.x, M.y), align=1.2W);
		dot("\small$M$", M);

		label("\small$x^2 + y^2 = 1$", A + 1.5dir(45+180));

		draw(shift(B) * ((0, -1.3) -- (0, 1.3)), Arrow(TeXHead));
		draw(shift(B) * ((-1.3, 0) -- (1.3, 0)), Arrow(TeXHead));

		real ch(real x) { return (exp(x) + exp(-x)) / 2.; }
		real sh(real x) { return (exp(x) - exp(-x)) / 2.; }
		real nch(real x) { return -ch(x); }

		real k = 1.9; real r = 1.2;
		real t = 1.4;

		draw(shift(B) * scale(0.35) * graph(ch, sh, -k, k), magenta);
		draw(shift(B) * scale(0.35) * graph(nch, sh, -k, k), magenta);

		label("\small$x^2 - y^2 = 1$", B + 1.5dir(45+180) + (0, -0.2));

		M = B + 0.35(ch(t), sh(t));

		draw(M -- (B.x-eps, M.y), dashed);
		draw(M -- (M.x, B.y-eps), dashed);
		dot("\small$M$", M);
		label("\small$\ch x$", (M.x, B.y), align=S);
		label("\small$\sh x$", (B.x, M.y), align=1.2W);

		draw(shift(B) * ((-r, -r)--(r,r)), gray + dashed);
		draw(shift(B) * ((r, -r)--(-r,r)), gray + dashed);
	\end{asy}
\end{figure}

\begin{prop}
	\[
		\forall t \in \R, \ch^2 t - \sh^2 t = 1.
	\]
\end{prop}

\begin{prv}
	Soit $t \in \R$.
	\begin{align*}
		\ch^2 t - \sh^2 t &= \left( \frac{e^{t}+e^{-t}}{2} \right)^2 - \left( \frac{e^{t}-e^{-t}}{2} \right)^2 \\
		&= \frac{\cancel{e^t} + e^{-t} -\cancel{-e^t} + e^{-t}}{2} \times \frac{e^t + \cancel{e^{-t}} + e^t - \cancel{e^{-t}}}{2} \\
		&= \frac{2e^{-t}}{2} \times \frac{2e^{t}}{2} \\
		&= e^{-t} \times e^t \\
		&= 1 \\
	\end{align*}
\end{prv}

\begin{exo}\relax
	{\itshape Expliciter $\ch(a+b)$ avec $(a,b) \in \R^2$.}

	\begin{align*}
		\ch(a) \ch(b) + \sh(a) \sh(b) &= \frac{2e^{a+b} + e^{b-a} + e^{a-b}}{4} + \frac{2e^{-(a+b)} - e^{a-b} - e^{b-a}}{4}\\
		&= \frac{1}{4} \left( 2e^{a+b} + 2e^{-(a+b)} \right) \\
		&= \frac{e^{a+b} + e^{-(a+b)}}{2} \\
		&= \ch(a+b). \\
	\end{align*}

	{\itshape Expliciter $\sh(a+b)$ avec $(a,b) \in \R^2$.}

	\begin{align*}
		\sh(a)\ch(b) + \sh(b)\ch(a) &= \frac{1}{4} \left( 2e^{a+b} + \cancel{e^{a-b}} - \cancel{e^{b-a}} + \cancel{e^{b-a}} - \cancel{e^{a-b}} - 2e^{-(a+b)} \right) \\
		&= \frac{e^{a+b} - e^{-(a+b)}}{2} \\
		&= \sh(a+b) \\
	\end{align*}
\end{exo}

\begin{prop}
	\begin{enumerate}
		\item $\ch$ est paire, $\sh$ est impaire;
		\item $\ch$ et $\sh$ sont dérivables sur $\R$ et $\begin{cases}
				\ch' = \sh,\\
				\sh' = \ch;
			\end{cases}$
		\item $\sh$ est strictement croissante sur $\R$, $\ch$ est strictement croissante sur $\R^+$;
		\item $\lim_{x\to +\infty} \ch x = +\infty$, $\lim_{x\to +\infty} \sh x = +\infty$ et $\lim_{x\to -\infty} \sh x = -\infty$;
		\item $\ch x \simi_{x\to +\infty} \frac{1}{2} e^{x}$ et $\sh \simi_{x\to +\infty} \frac{1}{2} e^{x}$.
	\end{enumerate}
\end{prop}

\begin{figure}[H]
	\centering
	\begin{asy}
		import graph;

		size(8cm);

		real f(real x) { return cosh(x); }
		real g(real x) { return sinh(x); }

		draw(graph(f, -2.5, 2.5, 300), magenta);
		draw(graph(g, -2.5, 2.5, 300), deepcyan);

		label("\small$1$", (0, 1), align=SW);

		real eps = 0.3;
		draw((-eps, 1)--(eps, 1));

		draw((-2.7,0) -- (2.7,0), Arrow(TeXHead));
		draw((0,g(-2.4)) -- (0,f(2.4)), Arrow(TeXHead));

		draw(scale(0.8) * (dir(45) -- dir(45 + 180)), red, Arrows(TeXHead));
		draw((-0.8, 1) -- (0.8, 1), red, Arrows(TeXHead));

		label("\small$\sh$", (-2.4, g(-2.4)), deepcyan, align=E);
		label("\small$\ch$", (-2.4, f(-2.4)), magenta, align=W);
	\end{asy}
\end{figure}

\begin{prv}
	\begin{enumerate}
		\item \[
				\begin{cases}
					\ch(-x) = \frac{e^{-x} + e^{-(-x)}}{2} = \frac{e^{x} + e^{-x}}{2} = \ch x;\\
					\sh(-x) = \frac{e^{-x} - e^{-(-x)}}{2} = \frac{e^{-x} - e^{x}}{2} = -\frac{e^{x}-e^{-x}}{2} = -\sh x.
				\end{cases}
			\] 
		\item $x \mapsto e^x$ et $x \mapsto e^{-x}$ sont dérivables sur $\R$ donc $\ch$ et $\sh$ aussi. On a donc \[
				\forall x \in \R, \begin{cases}
					\ch'(x) = \frac{e^{x} - e^{-x}}{2} = \sh(x);\\
					\sh'(x) = \frac{e^{x}+e^{-x}}{2} = \ch(x).
				\end{cases}
			\]
		\item $\forall x \in \R, \sh' x = \ch x = \frac{e^{x}+e^{-x}}{2} > 0$ donc $\sh$ strictement croissante sur $\R$.\\
			$\forall x \in \R^+_*, \ch' x = \sh x > \sh 0 = 0$ (car $\sh$ strictement croissante) et $\ch' 0 = \sh 0 = 0$. Donc, $\ch$ est strictement croissante sur $\R^+$.
		\item~\kern-2.5mm et 5.  \[
			\forall x \in \R, \begin{cases}
				\ch x = \frac{1}{2}\left( e^{x} + e^{-x} \right) = \frac{e^{x}}{2}\left( 1 + e^{-2x} \right) \simi_{x\to +\infty} \frac{e^x}{2} \longrightarrow +\infty;\\
				\sh x = \frac{1}{2}\left( e^{x} - e^{-x} \right) = \frac{e^{x}}{2}\left( 1 - e^{-2x} \right) \simi_{x\to +\infty} \frac{e^x}{2} \longrightarrow +\infty;\\
			\end{cases}
		\] 
	\end{enumerate}
\end{prv}

\begin{rmk}
	La courbe représentative de $\ch$ est appelée ``chaînette''.
\end{rmk}

\begin{prop}
	\begin{enumerate}
		\item $\th$ est imapire;
		\item $\th$ est dérivable sur $\R$ et \[
				\forall x \in \R, \th' x = \frac{1}{\ch^2 x} = 1-\th^2 x;
			\]
		\item $\th$ est strictement croissante sur $\R$;
		\item $\lim_{x\to +\infty} \th x = 1$ et $\lim_{x\to -\infty} \th x = -1$.
	\end{enumerate}
\end{prop}

\begin{figure}[H]
	\centering
	\begin{asy}
		import graph;

		size(8cm);

		real f(real x) { return tanh(x); }

		draw(graph(f, -2.5, 2.5, 300), magenta);

		label("\small$1$", (0, 1), align=SW);
		label("\small$-1$", (0, -1), align=SW);

		real eps = 0.03;
		draw((-eps, 1)--(eps, 1));
		draw((-eps, -1)--(eps, -1));

		draw((-2.7,0) -- (2.7,0), Arrow(TeXHead));
		draw((0,-1.2) -- (0,1.2), Arrow(TeXHead));

		draw(scale(0.8) * (dir(45) -- dir(45 + 180)), red, Arrows(TeXHead));

		draw((-2.5, -1) -- (-1, -1), dashed + gray);
		draw((2.5, 1) -- (1, 1), dashed + gray);

		label("\small$\th$", (2.5, 1), magenta, align=SW);
	\end{asy}
\end{figure}

\begin{prv}
	\begin{enumerate}
		\item Soit $x \in \R$. \[
				\th(-x ) = \frac{\sh(-x)}{\ch(-x)} = -\frac{\sh x}{\ch x} = -\th x.
			\]
		\item $\sh$ et $\ch$ sont dérivables sur $\R$ et $\forall x \in \R,\,\ch x \neq 0$ donc $\th$ est dérivable sur $\R$ et
			\begin{align*}
				\forall x \in \R, \th' x = \frac{\ch^2 x - \sh^2x}{2} &= \frac{1}{\ch^2 x} \\
				&= 1-\frac{\sh^2 x}{\ch^2 x} = 1-\th^2 x. \\
			\end{align*}
		\item $\forall x \in \R, \th' x = \frac{1}{\ch^2 x} > 0$ donc $\th$ est strictement croissante sur $\R$.
		\item On a \[
				\forall x \in \R, \th x = \frac{\sh x}{\ch x} \sim \frac{e^{x} / 2}{e^x / 2} = 1 \tendsto{x\to +\infty} 1
			\] et, comme $\th$ est impaire, \[
				\lim_{x\to -\infty} \th x = - \lim_{x\to +\infty} \th x = -1.
			\]
	\end{enumerate}
\end{prv}

\begin{exo}\relax
	{\itshape Résoudre $\sh x = y$ d'inconnue $x$ et $y$ étant un paramètre réel.}

	Soit $x \in \R$.
	\begin{align*}
		\sh x = y \iff& \frac{e^x - e^{-x}}{2} = y\\
		\iff& \left( e^x \right)^2 - 1 = 2ye^x\\
		\iff& \left(e^x - y\right)^2  - \overbrace{(y^2 + 1)}^{> 0} = 0\\
		\iff& \left( e^x - y - \sqrt{y^2 + 1} \right) \left( e^x -y + \sqrt{y^2+1} \right)\\
		\iff& e^x = y + \sqrt{y^2 + 1} \ou e^x = y - \sqrt{y^2 + 1}\\
	\end{align*}
	
	Or, $\left( y + \sqrt{1+y^2} \right)\left( y - \sqrt{1+y^2} \right) = -1 < 0$ donc $y + \sqrt{1+y^2}$ et $y - \sqrt{1+y^2}$ sont de signes opposés.

	Comme $y + \sqrt{1+y^2} > y - \sqrt{1+y^2}$, on a $\begin{cases}
		y + \sqrt{1+y^2} > 0\\
		y - \sqrt{1 + y^2} < 0
	\end{cases}$ d'où \[
		\sh x = y \iff e^x = y + \sqrt{1+y^2} \iff x = \ln\left( y + \sqrt{1+y^2} \right).
	\]

	On a trouvé la réciproque de $\sh$ : \begin{align*}
		\Argsh: \R &\longrightarrow \R \\
		x &\longmapsto \ln\left(x + \sqrt{1+x^2}\right)
	\end{align*}
\end{exo}

