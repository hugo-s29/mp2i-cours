\part{Décomposition en éléments simples}

\marginpar{\bf Définition\\[-5mm]}
\begin{lem}
	\[
		\forall F \in \mathbbm{K}(X), \exists! (E,G) \in \mathbbm{K}[X] \times \mathbbm{K}(X),
		\begin{cases}
			F = E + G\\
			\deg(G) < 0
		\end{cases}
	\]
	On dit que $E$ est la \underline{partie entière} de $F$.
\end{lem}

\begin{prv}
	On pose $F = \frac{P}{Q}$ avec $P \in \mathbbm{K}[X], Q \in \mathbbm{K}[X]\setminus \{0\}$.
	\begin{itemize}
		\item[\underline{\sc Analyse}] Soit $E \in \mathbbm{K}[X]$ et $G \in \mathbbm{K}(X)$ tels que \[
				F = E + G\\
				\deg(G) < 0.
			\]

			On pose $G = \frac{A}{Q}$ avec $A \in \mathbbm{K}[X]$.
			\begin{align*}
				F = E + G &\iff \frac{P}{Q} = E + \frac{A}{Q}\\
				&\iff P = EQ + A.
			\end{align*}
			$\deg G < 0, \deg A < \deg Q$.

			Don $E$ est le quotient de la division euclidienne de $P$ par $Q$ et $A$ sont reste.
		\item[\underline{\sc Synthèse}] Soient $E$ le quotient et $A$ le reste de la division euclidienne de $P$ par $Q$. On a alors \[
			\begin{cases}
				P = EQ + A\\
				\deg(A) < \deg(Q)
			\end{cases}
		\] et donc \[
			F = E + \frac{A}{Q}\\
			\deg\left( \frac{A}{Q} \right) < 0.
		\]
	\end{itemize}
\end{prv}

\begin{exm}
	$F = \frac{X^3 + X - 1}{X^2+2}$, $\deg F = 1$.\\
	\begin{center}
		\begin{tabular}{D|D}
			X^3 + X - 1 & X^2 + 2\\ \cline{2-2}
			\raisesign{-}
			X^3 + 2X & X\\ \cline{1-1} \\[\dimexpr-\normalbaselineskip+\jot]
			-X - 1\\
		\end{tabular}
	\end{center}
	Donc, \[
		F = \frac{X(X^2+2) - (X+1)}{X^2+2} = X - \frac{X+1}{X^2+2}.
	\]
\end{exm}

\begin{lem}
	Soit $F = \frac{P}{AB}$ avec \[
		\begin{cases}
			(P, A, B) \in \mathbbm{K}[X]^3;\\
			A\neq 0; B\neq 0;\\
			A\wedge B = 1;
			\deg F < 0.
		\end{cases}
	\] Alors, \[
		\exists! (U,V) \in \mathbbm{K}[X]^2, \begin{cases}
			F = \frac{U}{A} + \frac{V}{B}\\
			\deg\left( \frac{U}{A} \right) < 0 \et \deg\left( \frac{V}{R} \right).
		\end{cases}
	\]
\end{lem}

\begin{prv}
	\begin{itemize}
		\item[\underline{\sc Analyse}] On suppose que \[
				\begin{cases}
					F = \frac{U}{A} + \frac{V}{B}\\
					U \in \mathbbm{K}[X], \deg U < \deg A\\
					V \in \mathbbm{K}[X], \deg V < \deg B.
				\end{cases}
			\] D'où \[
				\frac{P}{AB} = \frac{UB+VA}{AB}
			\] et donc $P = UB + VA$. D'après le théorème de Bézout, il existe $(R,S) \in \mathbbm{K}[X]^2$ tel que \[
				RB+SA = P.
			\] On a alors \[
				0 = B(R-U) + A(S-V)
			\] donc \[
				A (S-V) = B(U-R)
			\] donc \[
				A  \mid  B(U-R) \qquad \text{ dans }\mathbbm{K}[X].
			\] D'après le théorème de \Gauss, \[
				A  \mid U - R
			\] 
			Donc $U-R = AT$ avec $T \in \mathbbm{K}[X]$ donc \[
				A(S-V) = BAT
			\] donc \[
				S-V = BT
			\] donc \[
				\begin{cases}
					R = -AT + U\\
					S = BT + V
				\end{cases}.
			\]
			On a \[
				\begin{cases}
					S = BT+V\\
					\deg V < \deg B
				\end{cases}
			\] donc $V$ est le reste de la division euclidienne de $S$ par $B$ et $U$ est le reste de la division euclidienne de $R$ par $A$.

		\item[\underline{\sc Synthèse}] Soit $(R,S) \in \mathbbm{K}[X]^2$ tel que \[
				P = RB + SA.
			\] Soit $V$ le reste de la division euclidienne de $S$ par $B$. \[
				\begin{cases}
					S = BT + V, T \in \mathbbm{K}[X]\\
					\deg V < \deg B
				\end{cases}
			\] D'où, \[
				\frac{P}{AB} = \frac{R}{A} + T + \frac{V}{B} = \frac{R+AT}{A}+\frac{V}{B}
			\] et \[
				\deg\left( \frac{V}{B} \right) = \deg(V) - \deg(B) < 0.
			\]
			Si  $\deg\left( \frac{R+AT}{A} \right) \ge 0$, alors \[
				\deg\left( \frac{R+AT}{A} \right) > \deg\left( \frac{V}{B} \right)
			\] et alors \[
				\deg\left( \frac{P}{AB} \right) = \deg\left( \frac{R+AT}{A} \right) \ge 0.
			\] Or, \[
				\deg\left( \frac{P}{AB} \right) < 0.
			\] Donc \[
				\deg\left( \frac{R+AT}{A} \right) < 0.
			\] On pose $U = R + AT$. On a bien \[
				\frac{P}{AB} = \frac{U}{A} + \frac{V}{B} \text{ avec } \deg\left( \frac{U}{V} \right) < 0 \et \deg\left( \frac{V}{R} \right) < 0.
			\]
	\end{itemize}
\end{prv}

\begin{lem}
	Soit $H \in \mathbbm{K}[X]$ irréductible, $n \in \N_*$, $P \in \mathbbm{K}[X]$, $F = \frac{P}{H^n}$ et $\deg F < 0$. Alors, \[
		\begin{cases}
			\exists ! (U,V) \in \mathbbm{K}[X]^2, F = \frac{U}{H^n} + \frac{V}{H^{n-1}};\\
			\deg U < \deg H;\\
			\deg\left( \frac{V}{H^{n+1}} \right) < 0.
		\end{cases}
	\]
\end{lem}

\begin{prv}
	\begin{itemize}
		\item[\underline{\sc Analyse}] $F = \frac{U}{H^n} + \frac{V}{H^{n-1}}$ avec \[
				\begin{cases}
					\deg U < \deg H,\\
					U \in \mathbbm{K}[X], V \in \mathbbm{K}[X],\\
					\deg\left( \frac{V}{H^{n-1}} \right) < 0.
				\end{cases}
			\] D'où \[
				P = U + VH,\\
				\deg U < \deg H.
			\] Donc $V$ est le quotient et $U$ le reste de la division euclidienne de $P$ par $H$.
		\item[\underline{\sc Synthèse}] Soient $V$ et $U$ le quotient et le reste de la division euclidienne de $P$ par $H$ : \[
				\begin{cases}
					P = U + VH\\
					\deg U < \deg H.
				\end{cases}
			\] D'où \[
				F = \frac{U}{H^n} + \frac{V}{H^{n-1}}
			\] \[
				\deg U < \deg H
			\] Si $\deg\left( \frac{V}{H^{n-1}} \right) \ge 0$, alors $\deg F = \deg\left( \frac{V}{H^{n-1}} \right)\ge 0$ : une contradiction. Donc $\deg\left( \frac{V}{H^{n-1}} \right) < 0$.
	\end{itemize}
\end{prv}

\begin{thm}
	[Théorème de décomposition en éléments simples sur $\mathbf{\C(X)}$]
	Soit $F \in \mathbbm{K}(X)$, $F = \frac{P}{Q}$ la forme irréductible de $F$. On note $(z_1, \ldots, z_p)$ les racines complexes de $Q$ et $(\mu_1, \ldots, \mu_p)$ leur multiplicité.

	Alors,
	\begin{align*}
		&\exists!(E, a_{1,1}, \ldots, a_{1, \mu_1}, a_{2,1}, \ldots, a_{2, \mu_2}, \ldots, a_{p, 1}, \ldots, a_{p, \mu_p}) \in \C[X] \times \C^{\deg Q},\\
		&F = E + \sum_{i=1}^p \left( \sum_{j=1}^{\mu_i} \frac{a_{i,j}}{(X- z_i)^j} \right).
	\end{align*}
\end{thm}

\begin{exm}
	$F = \frac{X^7 - 1}{X^5 + 2X^3 + X} \in \C(X)$

	\begin{center}
		\begin{tabular}{D|D}
			X^7 - 1 & X^5 + 2X^3 + X\\ \cline{2-2}
			\raisesign{-}
			X^7 + 2X^5 + X^3 & X^2 - 2\\ \cline{1-1} \\[\dimexpr-\normalbaselineskip+\jot]
			-2X^5 - X^3 - 1\\
			\raisesign{-}
			-2X^5 -4X^3 -2X\\ \cline{1-1} \\[\dimexpr-\normalbaselineskip+\jot]
			3X^3 + 2X - 1
		\end{tabular}
	\end{center}
	\[
		F = (X^2-2) + \frac{3X^3 + 2X - 1}{X^5 + 2X^3 + X}
	\]
	\begin{align*}
		X^5 + 2X^3 + X &= X(X^4 + 2X^2+1) \\
		&= X(X^2 + 1)^2 \\
		&= X(X-i)^2(X+i)^2 \\
	\end{align*}
	D'après le 2\eme lemme, \[
		\frac{3X^3 + 2X - 1}{X^5 + 2X^3 + X} = \frac{a}{X} + \frac{bX+c}{(X-i)^2}
	\] D'après le 3\eme lemme,
	\begin{align*}
		\frac{bX + c}{(X-i)^2} = \frac{f}{(X-i)^2} + \frac{g}{X-i}\\
		\frac{dX +e}{(X+i)^2} = \frac{h}{(X+i)^2}+\frac{h}{X+i}
	\end{align*}
	\[
		F = (X^2-2) + \frac{a}{X} + \frac{f}{(X-i)^2}+\frac{g}{X-i}+\frac{h}{(X-i)^2}+\frac{k}{X+i}.
	\] 
	On multiplie par $X$ : \[
		\frac{X^7 - 1}{X^4 + 2X^2+1} = a + X\left( X^2 - 2 + \frac{f}{(X-i)^2} + \frac{g}{X-i} + \frac{h}{(X-i)^2}+ \frac{k}{X+1} \right).
	\] En rempla\c cant $X$ par $0$, on obtient \[
		a = \frac{-1}{1} = -1.
	\] On multiplie par $(X-i)^2$ et on remplace $X$ par $i$ : \[
		\frac{3i^2+2i - 1}{i(2i)^2} = f.
	\]Donc, \[
		f = \frac{-i - 1}{-4i} = \frac{1}{4} - \frac{i}{4}
	\] De même, \[
		h = \frac{3(-i)^3 + 2(-i) - 1}{-i (-2i)^2} = \overline{f} = \frac{1}{4} + \frac{i}{4}
	\]
	\begin{landscape}
		\[
			\frac{3X^2 + 2X - 1}{X^5 + 2X^3} +\frac{1}{X} - \left( \frac{1}{4} - \frac{i}{4} \right) \frac{1}{(X-i)^2} - \left( \frac{1}{4} + \frac{i}{4} \right) \frac{1}{(X+i)^2}
			 = \frac{g}{X-i} + \frac{h}{X+i}
		\]
		\begin{align*}
			\frac{g}{X-i} + \frac{h}{X+i} &= \frac{3X^3 + 2X - 1 + X^4 + 2X^2 + 1 + X(X+i)^2\left( -\frac{1}{4}+\frac{i}{4} \right) - X(X-i)^2\left( \frac{1}{4} -\frac{i}{4} \right)}{X^5 + 2X^3+ X}\\
			&= \frac{12X^3 + 8X - 4 + 4X^4 + 8X^2 + 4 + X^3 (-1 + i) + (1-i)X - X^3(1+i)  - 2X^2(1-i) + X(1+i)}{4(X^5 + 2X^3 + X)} \\
			&= \frac{4X^4 + 10X^3 + 8X^2 + 10X}{4\left( X^5 + 2X^3 + X \right)} \\
			&= \frac{2X^3 + 5X^2 + 2X + 5}{2(X^4 + 2X^2 + 1)} \\
			&= \frac{(X-i)(X+i)(2X+5)}{2(X-i)^2(X+i)^2} \\
			&= \frac{2X+5}{2(X-i)(X+i)} \\
		\end{align*}

		Donc, \[
			\begin{cases}
				g = \frac{2i + 5}{2\times 2i} = \frac{(2i+5)i}{-4} = \frac{2 - 5i}{4}\\
				~\\
				h = \frac{2(-i) + 5}{2(-2i)} = \frac{2 + 5i}{4}
			\end{cases}
		\]
	\end{landscape}
\end{exm}

\begin{prv}
	On suppose $\frac{P}{Q}\not\in \C[X]$. On peut supposer $Q$ unitaire.

	\begin{itemize}
		\item[\underline{\sc Existence}] D'après le lemme 1, il existe $E \in \C[X]$, $G \in \C(X)$ tels que \[
				\begin{cases}
					\frac{P}{Q} = E + G\\
					\deg(G) < 0
				\end{cases}
			\] Soit $\frac{A}{B}$ la forme irréductible de $G$ avec $B$ unitaire ($A\wedge B=1$ et $A\neq 0$).
			\[
				\frac{P}{Q} = E + \frac{A}{B}
			\] donc \[
				PB = EBQ + AQ \qquad (*)
			\] donc \[
				AQ = PB - EBQ
			\] donc \[
				A  \mid B (P-EQ)
			\] D'après le théorème de \Gauss, \[
				A  \mid P - EQ
			\] Soit $R \in \C[X]$ tel que \[
				AR = P - EQ
			\] D'où \[
				\frac{AR}{Q} = \frac{P}{Q}-E = \frac{A}{B}
			\] D'où \[
				\frac{R}{Q} = \frac{1}{B}
			\] donc $B \mid Q$.\\
			De $(*)$, on a aussi \[
				P  \mid Q (EB+A)
			\] Or, $P\wedge Q = 1$. Donc \[
				P \mid EB + A
			\] Soit $S \in \C[X]$ tel que \[
				PS = EB + A
			\] Donc \[
				\frac{PS}{B} = E + \frac{A}{B} = \frac{P}{Q}
			\] Donc \[
				\frac{S}{B} = \frac{1}{Q}
			\] et donc $Q \mid B$ \\
			Donc $Q = B$ (car ils sont unitaires).\\
			Donc $G = \frac{A}{Q}$.\\
			Or, \[
				Q = \prod_{j=1}^{p} (X - z_j)^{\mu_j}
			\] \[
				\forall j \neq k, (X - z_j)^{\mu_j} \wedge (X - z_k)^{\mu_k} = 1
			\] D'après le lemme 2, il existe $(A_1, \ldots, A_p) \in \C[X]^p$ tel que \[
				\begin{cases}
					\frac{A}{Q} = \sum_{j=1}^p \frac{A_j}{(X-z_j)^{\mu_j}}\\
					\forall j\in \left\llbracket 1, p \right\rrbracket, \deg(A_j) < \mu_j
				\end{cases}
			\] Soit $j \in \left\llbracket 1,p \right\rrbracket$. D'après le lemme 3, \[
				\frac{A_j}{(X-z_j)^{\mu_j}} = \frac{a_{j, \mu_j}}{(X-z_j)^{\mu_j}}
				+ \frac{A_{j,1}}{(X-z_j)^{\mu_j - 1}}
			\] avec \[
				\begin{cases}
					a_{j, \mu_j} \in \C\\
					A_{j,1} \in \C_{\mu_j-2}[X]
				\end{cases}
			\] En itérant ce procédé, on trouve $(a_{j, \mu_j}, \ldots, a_{j,1}) \in \C^{\mu_j}$ tel que \[
				\frac{A_j}{(X-z_j)^{\mu_j}} = \sum_{k=1}^{\mu_j} \frac{a_{j, k}}{(X-z_j)^k}
			\] D'où \[
				\frac{P}{Q} = E + \underbrace{\sum_{j=1}^p \sum_{k=1}^{\mu_j} \frac{a_{j,k}}{(X-z_j)^k}}_{\deg(\quad) < 0}
			\] 
		\item[\underline{\sc Unicité}] Soit $E_1 \in \C[X]$ et $(b_{j,k})_{\substack{1\le j\le p\\1\le k\le \mu_j}} \in \C^{\sum_{j=1}^p \mu_j}$ tels que \[
				\frac{P}{Q} = E_1 + \underbrace{\sum_{j=1}^p \sum_{k=1}^{\mu_j} \frac{b_{j,k}}{(X-z_j)^k}}_{\deg(\quad) < 0}
			\]
			D'après le lemme 1, \[
				E = E_1 \et \sum_{j=1}^p \sum_{k=1}^{\mu_j} \frac{a_{j,k}}{(X - z_j)^k} = \sum_{j=1}^p \sum_{k=1}^{\mu_j} \frac{b_{j,k}}{(X- z_j)^k}
			\]
			\begin{align*}
				\forall j \in \left\llbracket 1,p \right\rrbracket,\\
				\sum_{k=0}^{\mu_j}& \frac{b_{j,k}}{(X-z_j)^k} = \frac{\sum_{k=1}^{\mu_j} b_{j,k}(X-z_j)^{\mu_j-k}}{(X-z_j)^{\mu_j}}\\
				&\qquad\vrt=\\
				\sum_{k=1}^{\mu_j}& \frac{a_{j,k}}{(X-z_j)^k} = \frac{\sum_{k=1}^{\mu_j}a_{j,k} (X - z_j)^{\mu_j- k}}{(X-z_j)^{\mu_j}}
			\end{align*}
			Donc, \[
				\sum_{k=1}^{\mu_j} a_{j,k} (X-z_j)^k = \sum_{k=1}^{\mu_j} b_{j,k}(X-z_j)^k
			\] Comme $\left( X - z_j, (X-z_j)^2, \ldots, (X - z_j)^{\mu_j} \right)$ est libre dans $\C[X]$, \[
				\forall k \in \left\llbracket 1,\mu_j \right\rrbracket, b_{j,k} = a_{j,k}
			\]
	\end{itemize}
\end{prv}

\begin{thm}
	[Théorème de décomposition en éléments simples sur $\R(X)$]
	Soit $(P,Q) \in \R[X]^2,~P\wedge Q = 1,~Q$ unitaire, $Q \not\in \{0,1\}$. On pose \[
		Q = \prod_{i=1}^p (X - a_i)^{\mu_i} \prod_{k=1}^q (X^2 + \alpha_k X + \beta_k)^{\nu_k}
	\] avec \[
		\begin{cases}
			p \in \N, q \in \N,\\
			(a_1, \ldots, a_p) \in \R^p\\
			(\alpha_1, \ldots, \alpha_q, \beta_1, \ldots, \beta_q) \in \R^{2q}\\
			(\mu_1, \ldots, \mu_p, \nu_1, \ldots, \nu_p) \in \N^{p+q}\\
			\forall j \in \left\llbracket 1, q \right\rrbracket, \alpha_k^2 - 4\beta_k < 0
		\end{cases}
	\] Alors
	\begin{align*}
		\exists! (E, &\gamma_{1,1}, \ldots, \gamma_{1,\mu_1}, \gamma_{2,1}, \ldots, \gamma_{2, \mu_2}, \ldots, \gamma_{p,1}, \ldots, \gamma_{p, \mu_p},\\
		&\delta_{1,1}, \ldots, \delta_{1,\nu_1}, \delta_{2,1}, \ldots, \delta_{2, \nu_2}, \ldots, \delta_{q,1}, \ldots, \delta_{q, \nu_q},\\
		&\varepsilon_{1,1}, \ldots, \varepsilon_{1,\nu_1}, \varepsilon_{2,1}, \ldots, \varepsilon_{2, \nu_2}, \ldots, \varepsilon_{q,1}, \ldots, \varepsilon_{q, \nu_q})\\
		\!\!\!&\in \R[X] \times \R^{\mu_1 + \cdots + \mu_p} \times \R^{2(\nu_1 + \cdots + \nu_q)}
	\end{align*}
	\begin{align*}
		\frac{P}{Q} = E + & \sum_{i=1}^p \sum_{j=1}^{\mu_i} \frac{\gamma_{i,j}}{(X-a_i)^j}\\
		&+ \sum_{k=1}^{q} \sum_{j=1}^{\nu_k} \frac{\delta_{k,j}X + \varepsilon_{k,j}}{\left( X^2 + \alpha_k X + \beta_k \right)^j}
	\end{align*}
	\qed
\end{thm}

\begin{defn}
	Soit $F \in \C(X)$. Soient $(P,Q) \in \C[X]^2$ tels que $\begin{cases}
		P\wedge Q = 1\\
		F = \frac{P}{Q}
	\end{cases}$.\\
	Les racines de $P$ sont appelées \underline{zéros de $F$}\\
	Les racines de $Q$ sont appelées \underline{pôles de $F$}\\
	\index{zéros (fraction rationnelle)}
	\index{pôles (fraction rationnelle)}
\end{defn}

\begin{prop}
	Soit $F \in \C(X)$ et $z \in \C$ un pôle simple de $F$. Le coefficient devant $\frac{1}{X - z}$ dans la décomposition en éléments simples de $F$ est $\frac{P(z)}{Q'(z)}$.
\end{prop}

\begin{prv}
	Soit $(P,Q) \in \C[X]^2$ tels que \[
		\begin{cases}
			F = \frac{P}{Q}\\
			P\wedge Q = 1\\
			Q \text{ unitaire}
		\end{cases}
	\] On pose \[
		Q = (X - z) \prod_{i = 1}^q (X - z_i)^{\mu_i}
	\] où $z_1, \ldots, z_q$ sont les racines distinctes de $z$ du polynôme $Q$. Donc, \[
		\frac{P}{Q} = F = \frac{a}{X-z} + \sum_{i=1}^{q} \sum_{k=1}^{\mu_i} \frac{b_{i,k}}{(X-z_i)^k}
	\] avec $a, \left( b_{i,k} \right)_{\substack{1\le i\le q\\1\le k \le \mu_i}}$ des nombres complexes.\\
	On muliplie par $X-z$. \[
		\frac{P}{\displaystyle\prod_{i=1}^q (X-z_i)^{\mu_i}} = a + (X-z) \sum_{i=1}^q \sum_{k=1}^{\mu_i} \frac{b_{i,k}}{(X-z_i)^k}
	\] On remplace $X$ par $z$. \[
		\frac{P(z)}{\displaystyle\prod_{i=1}^q (z - z_i)^{\mu_i}} = a
	\] Or, \[
		Q = (X-z) \prod_{i=1}^{q} (X-z_i)^{\mu_i}
	\] D'où \[
		Q' = \prod_{i=1}^q (X-z_i)^{\mu_i} + (X-z)\left( \prod_{i=1}^q (X-z_i)^{\mu_i} \right)'
	\] Don \[
		Q'(z) = \prod_{i=1}^q (z - z_i)^{\mu_i}
	\] Donc \[
		a = \frac{P(z)}{Q'(z)}
	\]
\end{prv}

\begin{prop}
	Soit $P \in \C[X]$ avec $\deg(P)\ge 1$, $(z_1, \ldots, z_p)$ les racines de $P$, $\mu_1, \ldots, \mu_p$ leur multiplicité. Alors \[
		\frac{P'}{P} = \sum_{i=1}^p \frac{\mu_i}{X-z_i}
	\]
\end{prop}

\begin{prv}
	On pose \[
		P = \alpha \prod_{i=1}^p (X-z_i)^{\mu_i}
	\] Donc \[
		P' = \alpha \sum_{i=1}^p \left( \prod_{j\neq i} (X - z_j)^{\mu_j} \right)\mu_i (X-z_i)^{\mu_i-1}
	\] D'où
	\begin{align*}
		\frac{P'}{P} &= \frac{\cancel\alpha \sum_{i=1}^p \mu_i (X-z_i)^{\mu_i-1} \prod_{j\neq i}(X-z_j)^{\mu_j}}{\cancel\alpha \prod_{i=1}^p (X-z_i)^{\mu_i}}\\
		&= \sum_{i=1}^p \mu_i \frac{(X-z_i)^{\mu_i - 1} \prod{j\neq i}) (X - z_j)^{\mu_j}}{\prod_{j=1}^p (X - z_j)^{\mu_j}} \\
		&= \sum_{i=1}^p \mu_i \frac{1}{X-z_i} \\
	\end{align*}
\end{prv}

\begin{rmk}
	Il existe un ``truc'' pour retenir cette formule : \[
		\frac{P'}{P} = \underbrace{\ln(P)'}_{\mathclap{\text{n'existe pas !}}}
		 = \left(\ln\left( \alpha \prod_{i=1}^p (X-z_i)^{\mu_i} \right)\right)'
		 = \left( \sum_{i=1}^p \mu_i \ln(X-z_i) \right)'
		 = \sum_{i=1}^p \mu_i \frac{1}{X-z_i}
	\]
\end{rmk}

















