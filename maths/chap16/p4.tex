\part{Fonctions à valeurs complexes}

\begin{defn}
	Soient $f: I \to \C$, ($I$ intervalle de $\R$) et $a \in I$.\\
	$f$ est \underline{dérivable en $a$} si $\lim_{\substack{x\to a\\\neq}} \frac{f(x)-f(a)}{x-a}\in \C$
\end{defn}

\begin{prop}
	\begin{align*}
		f \text{ est dérivable en } a
		\iff \Re(f) \text{ et } \Im(f) \text{ sont dérivables en } a
	\end{align*}
	Dans ce cas, $f'(a) = \Re(f)'(a) + i\Im(f)'(a)$
	\qed
\end{prop}

\begin{prop}
	La somme, le produit, de fonctions dérivables sont dérivables; le quotient également si le dénominateur ne s'annule pas.
	\qed
\end{prop}

\begin{prop}
	idem avec les dérivées $n$-ièmes\qed
\end{prop}

\begin{rmk}
	[Attention \danger]
	Le théorème de Rolle n'est plus vraie.

	\begin{center}
		\begin{asy}
			import graph;
			size(6cm);
			
			axes("$\Re$", "$\Im$", EndArrow);

			draw(circle((0,0), 1));
			draw(circle((0,0), 1.5), white+0);

			dot("1", (1,0), align=SE);

			real t = 1;
			pair z = (cos(t), sin(t));
			pair zp = z * (0,1);
			pair z2 = z + zp / 3;

			dot("$f(t)$", z, align=SW);
			draw(z--z2, red + 0.5, Arrow(TeXHead));
			label("$f'(t)$", (z+z2)/2, red, align=NE);

			draw(arc((0,0), 1, 0, 10), red, Arrow(TeXHead));
		\end{asy}
	\end{center}

	\begin{align*}
		f: \R &\longrightarrow \C \\
		t &\longmapsto e^{it}
	\end{align*}

	$f(0) = f(2\pi) = 1$\\
	$f$ est continue sur $[0,2\pi]$ et dérivable sur $]0,2\pi[$ \\
	$\forall t, f'(t) = ie^{it} \neq 0$ \\
\end{rmk}

\begin{prop}
	La formule de Taylor avec reste intégral et l'inégalité de Taylor-Lagrange sont aussi vrais dans $\C$.\qed
\end{prop}
