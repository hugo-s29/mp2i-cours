\part{Exercice 9}

\begin{enumerate}
	\item $E^* = \mathcal{L}(E, \mathbbm{K})$
	\item Soit $i \in \left\llbracket 1,n \right\rrbracket$. Soient $(x,y) \in E^2$ et $(\lambda, \mu) \in \mathbbm{K}^2$.\\
		On pose $x = \sum_{j=1}^n x_j e_j$ avec $(x_1, \ldots, x_n) \in \mathbbm{K}^n$ et $y = \sum_{j=1}^n y_j e_j$ avec $(y_1, \ldots, y_n) \in \mathbbm{K}^n$.\\
		D'où, 
		\begin{align*}
			\lambda x + \mu y &= \lambda \sum_{j=1}^n x_j e_j + \mu \sum_{j=1}^n y_j e_j \\
			&= \sum_{j=1}^n (\lambda x_j + \mu y_j) e_j \\
		\end{align*}
		\begin{align*}
			e_i^*(\lambda x + \mu y) &= \lambda x_i + \mu y_i \\
			&= \lambda e_i^*(x) + \mu e_i^*(y) \\
		\end{align*}
		Donc, $e_i^* \in E^*$.\\
		Soit $(\lambda_1, \ldots, \lambda_n) \in \mathbbm{K}^n$. On suppose que \[
			\sum_{i=1}^n \lambda_i e_i^* = 0_{E^*}
		\] Donc \[
			\forall x \in E, \sum_{i=1}^n \lambda_i e_i^*(x) = 0_\mathbbm{K}
		\] \[
			\forall j \in \left\llbracket 1,n \right\rrbracket, \sum_{i=1}^n \lambda_i \underbrace{e^*_i(e_j)}_{\delta_{i,j}} = 0_\mathbbm{K} \text{ donc } \lambda_j = 0_\mathbbm{K}
		\] Donc $(e_1^*, \ldots, e_n^*)$ est libre.
	\item Soit $f \in E^*$. On va montrer que \[
			f = \sum_{i=1}^n f(e_i) e_i^*
		\] En effet, pour tout $x \in E$ :
		\begin{align*}
			f(x) = f\left( \sum_{i=1}^n e_i^*(x) e_i \right) 
			&= \sum_{i=1}^n f(e_i) e_i^*(x) \\
			&= \left(\sum_{i=1}^n f(e_i) e_i^*\right)(x) \\
		\end{align*}
		Donc $E^* = \Vect(e_1^*, \ldots, e_n^*)$. Donc $\dim(E^*) = \dim(E)$.
	\item \begin{align*}
			\Phi(u): E^* &\longrightarrow \mathbbm{K} \\
			f &\longmapsto f(u)
		\end{align*}
		Soient $f,g \in E^*$ et $\lambda,\mu \in \mathbbm{K}$.
		\begin{align*}
			\Phi(u) (\lambda f + \mu g) &= (\lambda f + \mu g)(u) \\
			&= \lambda f(u) + \mu g(u) \\
			&= \lambda \Phi(f) + \mu \Phi(g) \\
		\end{align*}
		Donc $\Phi(u) = \mathcal{L}(E^*, \mathbbm{K}) = \left( E^* \right)^* = E^{**}$.
	\item \begin{align*}
			\Phi: E &\longrightarrow E^{**} \\
			u &\longmapsto \Phi(u)
		\end{align*}
		Soient $(u,v) \in E^2$ et $(\lambda, \mu) \in \mathbbm{K}^2$.
		\begin{align*}
			\Phi(\lambda u + \mu v) &: E^* \longrightarrow \mathbbm{K}\\
			\lambda \Phi(u) + \mu \Phi(v) &: E^* \longrightarrow \mathbbm{K}\\
		\end{align*}
		Soit $f \in E^*$.
		\begin{align*}
			\Phi(\lambda u + \mu v)(f) &= f(\lambda u + \mu v) \\
			&= \lambda f(u) + \mu f(v) \\
			&= \lambda \Phi(u)(f) + \mu \Phi(v)(f) \\
			&= \big(\lambda \Phi(u) + \mu \Phi(v)\big)(f) \\
		\end{align*}
		Donc $\Phi \in \mathcal{L}(E, E^{**})$.\\[2mm]
		Soit $u \in E$.
		\begin{align*}
			\Phi(u) = 0_{E^{**}} \iff& \forall f \in E^*, \Phi(u)(f) = 0_\mathbbm{K}\\
			\iff& \forall f \in E^*, f(u) = 0_\mathbbm{K}
			\implies& \forall i \in \left\llbracket 1,n \right\rrbracket, e_i^*(u) = 0_\mathbbm{K}\\
			\implies& u = \sum_{i=1}^n e_i^*(u) e_i = 0_E\\
		\end{align*}
		Donc, $\Ker \Phi = \{0_E\}$ donc $\Phi$ est injective.
	\item \[
			\dim(E^{**}) = \dim(E^*) = \dim(E)
		\] donc \[
			\Phi \in \mathrm{GL}(E, E^{**})
		\]
\end{enumerate}

