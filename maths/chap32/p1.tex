\part{Familles sommables positives}

Dans ce paragraphe, toutes les familles considérées sont positives.

\begin{defn}
	Soit $(u_i)_{i\in I} \in \left( \R^+ \right)^I$.

	Si $\Bigg\{\:\;\sum_{i \in J} u_i \:\;\Bigg|\Bigg.\:\;\substack{\displaystyle J \subset I\\[1mm]\displaystyle J \text{ fini}}\:\; \Bigg\}$ est majorée, alors on dit qu'elle est \underline{sommable}\index{famille sommable (positive)} et on définit la \underline{somme}\index{somme (familles sommables)} des $(u_i)_{i\in I}$ par \[
		\sum_{i \in I} u_i = \sup_{\substack{J \subset I\\J \text{ fini}}}\: \sum_{j \in J}\:u_j
	.\]

	Sinon, la famille \underline{n'est pas sommable}\index{famille non sommable} et on définit \[
		\sum_{i \in I} u_i = +\infty
	.\]
\end{defn}

\begin{exm}
	\begin{enumerate}
		\item Toute famille finie positive est sommable : avec $I = \left\llbracket 1,n \right\rrbracket$, on a \[
				\sum_{i \in I} u_i = \sum_{i=1}^n u_i
			.\]
		\item Soit $(u_n)_{n\in\N} \in \left( \R^+ \right)^\N$.
			\begin{itemize}
				\item[\underline{\sc Cas 1}] $\Sigma u_n$ converge. Soit $J \subset I = \N$ finie et \[
						\sum_{j \in J} u_j \le \sum_{i=0}^{\max(J)}u_i \le \sum_{i=0}^{+\infty} u_i = S
					.\]

					Or, \[
						\forall \varepsilon > 0\, \exists N \in \N\,\forall n \ge N,\,S - \varepsilon \le \sum_{i=1}^{n} u_i
					\]
					donc $S - \varepsilon$ ne majore pas $\Bigg\{\:\;\sum_{i \in J} u_i \:\;\Bigg|\Bigg.\:\;\substack{\displaystyle J \subset I\\[1mm]\displaystyle J \text{ fini}}\:\; \Bigg\}$. Ainsi, \[
						S = \sup_{\substack{J \subset I\\J\text{ finie}}} \sum_{j \in J} u_j = \sum_{i \in \N} u_i
					.\]
				\item[\underline{\sc Cas 2}] $\Sigma u_n$ diverge. Alors $\lim_{N\to +\infty} \sum_{i=0}^Nu_i = +\infty$ et donc $\Bigg\{\:\;\sum_{i \in J} u_i \:\;\Bigg|\Bigg.\:\;\substack{\displaystyle J \subset I\\[1mm]\displaystyle J \text{ fini}}\:\; \Bigg\}$ n'est pas majorée et donc $(u_n)_{n\in\N}$ n'est pas sommable et \[
						\sum_{n \in \N} u_n = +\infty
					.\]
			\end{itemize}
	\end{enumerate}
\end{exm}

\begin{thm}
	Soit $(u_i)_{i\in I} \in \left( \R^+ \right)^{I}$ sommable et soit $\sigma : I \to I$ une bijection. Alors \[
		\sum_{i \in I} u_{\sigma(I)} = \sum_{i \in I} u_i
	.\]
\end{thm}

\begin{prv}
	\begin{align*}
		\Bigg\{\:\;\sum_{j \in J} u_{\sigma(j)} \:\;\Bigg|\Bigg.\:\;\substack{\displaystyle J \subset I\\[1mm]\displaystyle J \text{ fini}}\:\; \Bigg\} &= \Bigg\{\:\;\sum_{k \in \sigma(J)} u_i \:\;\Bigg|\Bigg.\:\;\substack{\displaystyle J \subset I\\[1mm]\displaystyle J \text{ fini}}\:\; \Bigg\} \\
		&= \Bigg\{\:\;\sum_{k \in K} u_k \:\;\Bigg|\Bigg.\:\;\substack{\displaystyle K \subset I\\[1mm]\displaystyle K \text{ fini}}\:\; \Bigg\} \\
	\end{align*}
\end{prv}

\begin{crlr}[sommation par paquets]
	Soit $(u_i)_{i\in I} \in \left( \R^+ \right)^{I}$ et $(I_j)_{j\in J}$ une partition de $I$ : \[
		I = \bigcupdot_{j \in J} I_j
	.\]

	\[
		(u_i)_{i\in I} \text{ est sommable } \iff \begin{cases}
			\forall j \in J,\,(u_i)_{i\in I_j} \text{ est sommable},\\
			\left( \Sigma_{i \in I_j} u_i \right)_{j \in J} \text{ est sommable}.
		\end{cases}
	.\]

	Dans ce cas, \[
		\sum_{i \in I} u_i = \sum_{j \in J} \sum_{i \in I_j} u_i
	.\]
	\qed
\end{crlr}

\begin{crlr}[Fubini positif]
	Soit $(u_{i,j})_{\substack{i \in I\\j \in J}} \in \left( \R^+ \right)^{I\times J}$.
	\[
		\sum_{i \in I} \sum_{j \in J} u_{i,j} = \sum_{j \in J} \sum_{i \in I} u_{i,j}
	.\]
\end{crlr}

\begin{prv}
	\begin{itemize}
		\item On suppose $(u_{i,j})_{\substack{i \in I\\j\in J}}$ sommable. On pose $I \times J = \bigcupdot_{i \in I} K_i$ où $K_i = \{i\} \times J = \{(i,j)  \mid j \in J\}$.
			
			Donc, $\forall i \in I,\,(u_{i,j})_{j \in J}$ est sommable, $\left( \Sigma_{j \in J} u_{i,j} \right)_{i \in I}$ est sommable et \[
				\sum_{(i,j) \in I\times J} u_{i,j} = \sum_{i \in I} \sum_{j \in J} u_{i,j}
			.\]

			On a aussi \[
				I \times J = \bigcup_{j \in J} L_j \qquad \text{ où } L_j = I \times \{j\}
			.\]

			$\forall j \in J,\,(u_{i,j})_{i\in I}$ est sommable, $\left( \Sigma_{i \in I}u_{i,j} \right)_{j \in J}$ est sommable et \[
				\sum_{(i,j) \in I\times J} u_{i,j} = \sum_{j \in J} \sum_{i \in I} u_{i,j}
			.\]
		\item On suppose $(u_i)_{\substack{i \in I\\j \in J}}$ non sommable et donc \[
				\sum_{i \in I} \sum_{j \in I} u_{i,j} = +\infty = \sum_{j \in J} \sum_{i \in I} u_{i,j}
			.\]
	\end{itemize}
\end{prv}

\begin{crlr}
	Soient $(a_i)_{i\in I}$ et $(b_j)_{j\in J}$ deux familles sommables de réels positifs.
	\[
		\sum_{\substack{i \in I\\j \in J}} a_i b_j = \left( \sum_{i \in I} a_i \right) \left( \sum_{j \in J} b_j \right)
	.\]
\end{crlr}

\begin{prv}
	On pose, pour $i \in I$, pour $j \in J$, $u_{i,j} = a_i b_j$.

	\begin{align*}
		\sum_{\substack{i \in I\\j \in J}} a_i b_j &= \sum_{(i,j) \in I\times J} u_{i,j} \\
		&= \sum_{i \in I} \sum_{j \in J} u_{i,j} \\
		&= \sum_{i \in I} \sum_{j \in J} a_i b_j \\
		&= \sum_{i \in I} a_i \sum_{j \in J} b_j \\
	\end{align*}
\end{prv}

Dans la preuve précédente, on a utilisé la linéarité de la somme :

\begin{prop}
	Soient $(a_i)_{i\in I} \in \left( \R^+ \right)^I$, $(b_i)_{i\in I} \in \left( \R^+ \right)^I$ et $\lambda \in \R^+$.
	\begin{enumerate}
		\item $\sum_{i \in I}(a_i + b_i) = \sum_{i \in I} a_i + \sum_{i \in I} b_i$ ;
		\item $\sum_{i \in I}(\lambda a_i) = \lambda \sum_{i \in I} a_i$.
	\end{enumerate}
\end{prop}

\begin{prv}
	\begin{itemize}
		\item[\underline{\sc Cas 1}] On suppose $(a_i)$ et $(b_i)$ sommables. Soit $J \subset I$ finie. \[
				\sum_{i \in J} (a_i + b_i) = \sum_{i \in J} a_i + \sum_{i \in J} b_i \le \sum_{i \in I} a_i + \sum_{i \in I} b_i
			.\]

			Donc, $S = \sum_{i \in I} a_i + \sum_{i \in I} b_i$ majore $\Bigg\{\:\;\sum_{i \in J} (a_i + b_i) \:\;\Bigg|\Bigg.\:\;\substack{\displaystyle J \subset I\\[1mm]\displaystyle J \text{ fini}}\:\; \Bigg\}$ et donc $(a_i + b_i)_{i \in I}$ est sommable.

			Soit $\varepsilon>0$. Il existe $J \subset I$ finie telle que \[
				\sum_{j \in J} a_j \ge \sum_{i \in I} a_i - \frac{\varepsilon}{2}
			.\]
			Également, il existe $J \subset I$ finie telle que \[
				\sum_{i \in K} b_i \ge \sum_{i \in J} b_i - \frac{\varepsilon}{2}
			.\]

			On pose $L = J \cup K$. $L \subset I$ et $L$ est un ensemble fini.

			\begin{align*}
				\sum_{i \in L}(a_i + bi) &= \sum_{i \in L} a_i  + \sum_{i \in L} b_i \\
				&\ge \sum_{i \in J} a_i + \sum_{i \in K} b_i\\
				&\ge \sum_{i \in I} a_i - \frac{\varepsilon}{2} + \sum_{i \in I} b_i - \frac{\varepsilon}{2}\\
				&\ge S - \varepsilon.
			\end{align*}

			Donc, \[
				S = \sup_{\substack{J \subset I\\J\text{ finie}}} \sum_{i \in J}(a_i + b_i) = \sum_{i \in I}(a_i + b_i)
			.\]
	\end{itemize}
\end{prv}

\begin{prop}
	Soient $(a_i)_{i \in I}$ et $(b_i)_{i\in I}$ deux familles de réels positifs telles que  \[
		\forall i \in I,\,0 \le a_i \le b_i
	.\] Si $(b_i)_{i\in I}$ est sommable, alors $(a_i)_{i\in I}$ aussi, et, dans ce cas, \[
		\sum_{i \in I} a_i \le \sum_{i \in I} b_i
	.\]
\end{prop}

\begin{prv}
	Soit $J \subset I$ finie. 
	\begin{align*}
		&\sum_{i \in J} a_i \le \sum_{i \in J} b_i \le \sum_{i \in I} b_i\\
		\text{donc }& \sum_{i \in I} b_i \text{ majore } \Bigg\{\:\;\sum_{i \in J} a_i \:\;\Bigg|\Bigg.\:\;\substack{\displaystyle J \subset I\\[1mm]\displaystyle J \text{ fini}}\:\; \Bigg\}\\
		\text{donc }& (a_i) \text{ est sommable et}
	\end{align*}
	\[
		\sum_{i \in I} a_i \le \sum_{i \in I} b_i
	.\]
\end{prv}


