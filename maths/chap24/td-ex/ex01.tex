\part{Exercice 1}

On a
\begin{itemize}
	\item $\sigma_1 = \begin{pmatrix}
			2&4&5&3
		\end{pmatrix} = \begin{pmatrix}
			2&3
		\end{pmatrix} \begin{pmatrix}
			2&5
		\end{pmatrix} \begin{pmatrix}
			2&4
		\end{pmatrix} \begin{pmatrix}
			7&6
		\end{pmatrix}$
	\item $\sigma_2 = \begin{pmatrix}
			2&4&1
		\end{pmatrix} \begin{pmatrix}
			3&7&5&6
		\end{pmatrix} = \begin{pmatrix}
			2&1
		\end{pmatrix} \begin{pmatrix}
			2&4
		\end{pmatrix} \begin{pmatrix}
			3&6
		\end{pmatrix} \begin{pmatrix}
			3&5
		\end{pmatrix} \begin{pmatrix}
			3&7
		\end{pmatrix}$
	\item $\sigma_3 = \begin{pmatrix}
			1&3&6&4
		\end{pmatrix} = \begin{pmatrix}
			1&4
		\end{pmatrix} \begin{pmatrix}
			1&6
		\end{pmatrix} \begin{pmatrix}
			1&3
		\end{pmatrix}$
\end{itemize}

On en déduit que \[
	\begin{cases}
		\varepsilon(\sigma_1) = 1,\\
		\varepsilon(\sigma_2) = -1\\
		\varepsilon(\sigma_3) = 1.
	\end{cases}
\]
