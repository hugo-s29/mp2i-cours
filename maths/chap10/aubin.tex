
	\pagebreak
	\begin{mdframed}
		La suite du cours provient d'Aubin. Je ne suis pas responsable pour les éventuelles bêtises qu'il a pu taper.
	\end{mdframed}
	\pagebreak


\let\cross\times
\let\gt\ge
\let\lt\le
\let\exist\exists



\part{Axiomatique de $\N$}


\begin{axm}[Axiomatique de Von Neumann]

		$(\N, \leq)$ est un ensemble totalement ordonné vérifiant\\
				Toute partie non vide de $\N$ a un plus petit élément\\
				Toute partie non vide majorée de $\N$ a un plus grand élément\\
				$\N$ n’est pas majoré\\

\end{axm}

\begin{defn}[$0$]

		$0 = \text{min}(\N)$\\

\end{defn}

\begin{defn}[$1$]

		$1 = \text{min}(\N \text{\\} \{0\})$\\

\end{defn}

\begin{defn}[$n+1$]

		Soit $n \in \N$\\
		On pose $n+1 = \text{min}(\{k \in \N|k>n\})$\\
		On dit que $n+1$ est le successeur de $n$\\

\end{defn}

\begin{prop}[$+1-1$]

		$\forall n \in \N, (n+1)-1 = n$\\
		$\forall n \in \N, (n-1)+1 = n$\\

\end{prop}

\begin{prv}

		Soient $n \in \N,\ p = n+1,\ q = p-1$ \\

		$n < p$ et $q < p$\\
		Donc $n \leq q$ car $q = \text{max}(\{k \in \N|k<p\})$\\
		Si $q > n$, alors $q \ge p$ car $p = \text{min}(\{k \in \N|k >n\})$\\
		Donc $q = n$\\

\end{prv}

\begin{prop}[Ensemble Ouvert Vide]

		$\forall n \in \N, ]n, n+1[ = \varnothing$\\

\end{prop}

\begin{prv}

		Soit $n \in \N$, on sait que $n+1>n$\\
		Soit $p>n$, on suppose $n < p < n+1$\\
		Comme $p >n, p \ge n+1$		Contradiction\\

\end{prv}

\begin{prop}[Théorème de Récurrence]

		Soit $P$ un prédicat sur $\N$ et $n \in \N$\\
		Si $\left\{\begin{array}{l c r}P(n_0) \text{ est vrai} \\ \forall n \ge n_0n P(n) \Longrightarrow P(n+1) \end{array}\right.$\\
		Alors $ \forall n \ge n_0, P(n)$ est vrai\\

\end{prop}

\begin{prv}

		Soit $A = \{n \in \N|n \ge n_0 \}$ et $P(n)$ faux\\
		Supposons $A \neq \varnothing$\\
		$A$ a donc un plus petit élement, soit $N = \text{min}(A)$\\

		Cas 1 : $N = 0$\\
				Alors, comme $N \in A$, on a $n_0 \leq 0$ et $P(0)$ est faux\\
				Alors $n_0 = 0$		Contradiction avec “$P(n)$ est vrai”\\

		Cas 2 : $N \neq 0$\\
				Alors $N-1 \in \N$\\
				$N-1 \notin A$ car $N-1 < N$\\
				Donc $N-1 < n_0$ ou $P(n)$ vrai\\

				Supposons $N-1 < n_0$\\
				$N \in A$ donc $N \ge n_0$\\
				$N-1 < n_0 < N$\\
				Donc $N = n_0$\\
				Or, $N \in A$ donc $P(n)$ est faux alors que $P(n)$ est vrai\\

				Supposons $\left\{\begin{array}{l c r}P(n-1) \text{ vrai} \\ N-1 \ge n_0\end{array}\right.$		Comme $N-1 \ge n_0, P(N-1) \Longrightarrow P(N)$\\

				Donc $P(N)$ est vrai\\
				Or, $N \in A$ donc $P(N)$ est faux\\

				Donc $A = \varnothing$\\



\end{prv}


\part{Récurrences}


\begin{prop}[Récurrence Double]

		Soient $P$ un prédicat sur $\N$ et $n_0 \in N$\\
		Si $\left\{\begin{array}{l c r} P(n_0)\text{ est vrai} \\ P(n_0 + 1) \text{ est vrai} \\ \forall n > n_0, P(n)\text{ et } P(n+1) \Longrightarrow P(n+2) \end{array}\right.$\\
		Alors $\forall n \ge n_0, P(n)$ est vrai\\

\end{prop}

\begin{prv}

		On pose $\forall n \in \N, Q(n) :$ “$P(n)\text{ et } P(n+1) \text{ vrais}$ ”\\
		$Q(n_0)$ est vrai\\

		Soit $n \ge n_0$, on suppose $Q(n)$ vrai\\
		On sait alors que $P(n+2)$ est vrai\\
		On sait par hypothèse de récurrence que $P(n+1)$ est vrai\\
		Donc $Q(n+1)$ est vrai\\

\end{prv}

\begin{prop}[Récurrence Multiple]

		Soient $P$ un prédicat sur $\N$ et $(p, n_0) \in \N^2$\\
		Si $\left\{\begin{array}{l c r} \forall k \in [\![0,p]\!], P(n_0-k)\text{ est vrai} \\ \forall n \ge n_0, (P(n) \text{ et ... } P(n+p-1)) \Longrightarrow P(n+p) \end{array}\right.$\\
		Alors $\forall n \ge n_0, P(n)$ est vrai\\

\end{prop}

\begin{prop}[Récurrence Forte]

		Soient $P$ un prédicat sur $\N$ et $n_0 \in \N$\\
		Si $\left\{\begin{array}{l c r} P(n_0)\text{ est vrai} \\ \forall n \ge n_0, (P(n_0) \text{ et ... } P(n-1)) \Longrightarrow P(n) \end{array}\right.$\\
		Alors $\forall n \ge n_0, P(n)$ est vrai\\

\end{prop}

\begin{prv}

		On pose $\forall n \in N, Q(n) :$ “$\forall k \in [\![n_0,n]\!], P(k) \text{ vrai}$”\\
		$Q(n_0)$ est vrai car $P(n_0)$ est vrai\\

		Soit $n \ge n_0$, on suppose $Q(n)$ vrai\\
		On sait que $\forall k \in [\![n_0,n]\!], P(k)$ est vrai\\
		Alors $P(n+1)$ est vrai\\
		Donc $\forall k \in [\![n_0,n+1]\!], P(k)$ est vrai\\
		Donc $Q(n+1)$ est vrai\\



\end{prv}


\part{Divisibilité}


\begin{defn}[Divisibilité]

		Soient $(a,b) \in \Z^2$\\
		On dit que $a$ divise $b$ si il existe $k \in \Z$ tel que $b = ka$\\
		On écrit $a|b$ et on dit que $\left\{\begin{array}{l c r}a \text{ est un diviseur de }b \\b \text{ est un multiple de }a\end{array}\right.$\\

\end{defn}

\begin{prop}[Caractéristiques de la Divisibilité]

		$|$ est une relation d’ordre sur $\Z$\\
		Ce n’est pas une relation totale\\

\end{prop}

\begin{prop}[Ordonnancement et Divisibilité]

		Soient $(a,b) \in \Z\cross \Z^*$\\
		Si $a|b, |a| \leq |b|  $\\

\end{prop}

\begin{prop}[Divisibilité et Combinaison Linéaire]

		Soient $(a,b,c) \in (\Z^*)^3$\\
		$\left\{\begin{array}{l c r}a|b\\a|c\end{array}\right. \Longrightarrow \forall (k,l) \in \Z^2, a|(bk+cl)$\\

\end{prop}

\begin{prv}

		$\left\{\begin{array}{l c r}b=au \text{ avec }u \in \Z \\ c = av \text{ avec } v \in \Z \end{array}\right.$\\
		Soient $(k,l) \in \Z^2$\\
		$bk + cl = auk+avl = a(uk+vl)$\\
		Donc $a|(bk+cl)$\\

\end{prv}

\begin{defn}[Nombres Associés]

		Soient $(a,b) \in \Z^2$\\

		$a$ et $b$ sont associés si $a=b$ ou $a=-b$\\

\end{defn}

\begin{prop}[Nombres Associés et Divisibilité]

		Soient $(a,b) \in \Z^2$\\
		$a|b \iff -a|b \iff a |-b \iff-a|-b$\\



\end{prop}


\part{Division Euclidienne}


\begin{prop}[Division Euclidienne dans $\N$]

		Soient $a \in \N$ et $b \in \N^*$\\
		$\exist!(q,r)\in \N^2, \left\{\begin{array}{l c r}a = bq+r \\r \in [0,b[\end{array}\right.$\\

\end{prop}

\begin{prv}

		Existence : On considère| $A = \{q \in \N|qb\leq a\}$ $A$ est non vide car $0 \in A$\\
				$A$ est majoré : $\forall q \in A, q \leq a$ car $a \ge qb \ge q$\\

				Soit $q = \text{max}(A)$, on pose $r = a-bq$\\
				Comme $a,b$ et $q \in \N, r \in \Z$\\
				$q \in A$ donc $qb \leq a$ donc $r \ge 0$\\
				$q+1 > \text{max}(A)$ donc $q+1 \notin A$ donc $(q+1)b > a$\\
				Donc $r < b$\\

		Unicité : Soit $(q',r') \in \N^2$ tel que $\left\{\begin{array}{l c r}a+bq'+r'\\0\leq r'<b\end{array}\right.$\\
				On sait aussi que $a = bq+r$\\
				Donc $0 = b(q'-q) + r'-r$\\
						$-r'+r=b(q'-q)$\\

				De plus, $\left\{\begin{array}{l c r}0\leq r < b \\ -b < -r \leq 0\end{array}\right.$\\
				Donc $-b < r'-r<b$\\

				Le seul multiple de $b$ dans $]\!]-b,b[\![$ est $0$\\
				Donc $r'-r=0$, donc $r=r'$\\
				et $b(q'-q)=0$\\
				Or, $b \neq 0$ donc $q'-q=0$ donc $q=q'$\\

\end{prv}

\begin{prop}[Division Euclidienne dans $\Z$]

		Soient $a \in \Z$ et $b \in \Z^*$\\
		$\exist! (q,r) \in \Z^2, \left\{\begin{array}{l c r}a=bq+r \\ 0 \leq r < |b| \end{array}\right.$\\

\end{prop}

\begin{prv}

		Existence :\\
				Cas 1 : $a \in \N, b \in \N^*$\\
						D’après ce qui précède, $\exist!(q,r) \in \N^2, \left\{\begin{array}{l c r}a=bq+r \\ 0 \leq r <b\end{array}\right.$\\
						Comme $b \in \N^*$, on a bien $0 \leq r < |b| $\\
						$q \in \N \subset \Z$\\

				Cas 2 : $a \in \Z, b \in \N^*$\\
						Comme $-a \in \N, \exist! (q',r') \in \N^2, \left\{\begin{array}{l c r}-a=bq'+r' \\ 0 \leq r' < b\end{array}\right.$\\

						Donc $a = b(-q') - r'$\\
								$=b(-q'-1) - r'+b$\\

						On pose $q = \left\{\begin{array}{l c r}-q'-1 \text{ si } r \neq 0 \\-q' \text{ si }r = 0 \end{array}\right.$	et	$r = \left\{\begin{array}{l c r}-r'+b \text{ si }r' \neq 0 \\ r' \text{ si } r'= 0\end{array}\right.$\\
						On a bien $\left\{\begin{array}{l c r}a = bq+r \\q \in \Z \\ 0 \leq r < |b| \end{array}\right.$\\

				Cas 3 : $a \in \N, b \in \Z^*_-$\\
						$\exist! (q',r') \in \N^2, \left\{\begin{array}{l c r}a=(-b)q'+r' \\0 \leq r' < -b\end{array}\right.$\\

						On pose $\left\{\begin{array}{l c r}q=-q'\\r=r'\end{array}\right.$\\
						Et on a bien $\left\{\begin{array}{l c r}a = bq+r \\0 \leq r < |b| \end{array}\right.$\\

				Cas 4 : $a \in \Z^-, b \in \Z^*_-$\\
						$\exist! (q',r') \in \N^2, \left\{\begin{array}{l c r}-a=-bq'+r' \\ 0 \leq r' < -b\end{array}\right.$\\

						Donc $a = bq' - r'$\\
								$=b(q'+1) - r'-b$\\

						On pose $q = \left\{\begin{array}{l c r}q'+1 \text{ si } r \neq 0 \\q' \text{ si }r = 0 \end{array}\right.$	et	$r = \left\{\begin{array}{l c r}-r-b' \text{ si }r' \neq 0 \\ r' \text{ si } r'= 0\end{array}\right.$\\
						On a bien $\left\{\begin{array}{l c r}a = bq+r \\q \in \Z \\ 0 \leq r < |b| \end{array}\right.$\\

		Unicité :\\
				Soit $(q',r') \in \Z^2$ tel que $\left\{\begin{array}{l c r}a=bq'+r' \\ 0 \leq r' < |b| \end{array}\right.$ et $\left\{\begin{array}{l c r}a = bq+r \\0 \leq r < |b|  \end{array}\right.$\\

				D’où $\left\{\begin{array}{l c r}b(q'-q) = r'-r \\ -|b| < r -r' < |b|   \end{array}\right.$\\
				Donc $r-r'=0$\\

				Donc $r'=r$ et $q'=q$\\

\end{prv}

\begin{defn}[Quotient et Reste]

		Soient $a \in \Z$ et $b \in \Z^*$\\
		D’après le théorème précédent, $\exist! (q,r) \in \Z\cross \N, \left\{\begin{array}{l c r}a=bq+r \\ 0 \leq r < |b| \end{array}\right.$\\
		On dit que $q$ est le quotient et $r$ le reste dans la division (euclidienne) de $a$ par $b$\\

\end{defn}

\begin{prop}[Reste et Divisibilité]

		Soient $a \in \Z, b \in \Z^*$\\
		On note $r$ le reste de la division de $a$ par $b$\\
		$r = 0 \iff b|a$\\

\end{prop}

\begin{prv}

		On pose $a = bq+r, q \in \Z$\\

		“$\Longrightarrow$” : Si $r = 0$, alors $\left\{\begin{array}{l c r}a = bq\\q \in \Z\end{array}\right.$	donc $b|a$\\

		“$\Longleftarrow$” : Si $b|a$, $\exist k \in \Z, a =bk$\\
				Donc $\left\{\begin{array}{l c r}a = bk+0\\0 \leq 0 < |b| \end{array}\right.$\\
				Par unicité du reste, $r=0$\\



\end{prv}


\part{Arithmétique Modulaire}


\begin{defn}[Congruences]

		Soient $(a,b) \in \Z^2, c \in \N^*$\\
		On dit que $a$ et $b$ sont congrus modulo $c$ si $a$ et $b$ ont le même reste dans ma division par $c$\\
		On note $a = b[c]$\\

\end{defn}

\begin{prop}[Congruence et Relation D’Equivalence]

		La relation de congruence modulo $c$ est une relation d’équivalence\\

\end{prop}

\begin{rmk}[Classes d’Equivalence Modulo $c$]

		On note $\Z/c \Z$ l’ensemble des classes d’équivalence modulo $c$\\
		$\Z/5 \Z = \{\bar0, \bar1, \bar2, \bar3, \bar4\}$\\

\end{rmk}

\begin{prop}[Modulo et Divisibilité]

		Soient $(a,b) \in \Z^2$ et $c \in \N^*$\\
		$a \equiv b [c] \iff c | b-a$\\

\end{prop}

\begin{prv}

		“$\Longrightarrow$” : On pose $\left\{\begin{array}{l c r}a = cq+r, q \in \Z, 0\ \leq r < c \\ b = cq' + r, q' \in \Z\end{array}\right.$\\
				Donc $b-a = c(q'-q)$\\
				Donc $c|b-a$\\

		“$\Longleftarrow$” : On pose $\left\{\begin{array}{l c r}a = cq+r, q \in \Z, 0\ \leq r < c \\ b = cq' + r', q' \in \Z, 0\ \leq r' < c\end{array}\right.$\\
				$b-a = c(q'-q) + r'-r$\\
				$-c \leq r'-r \leq c$\\

				Si $r'-r \ge 0, r'-r$ est le reste de la division de $a-b$ par $c$\\
				Donc $r'=r$ donc $a \equiv b [c]$\\

				Si $r'-r < 0, r-r'$ est le reste de la division de $a-b$ par $c$\\
				Donc $r'=r$ donc $a \equiv b[c]$\\

\end{prv}

\begin{prop}[Addition et Multiplication de Congruences]

		Soient $(a,b,x,y) \in \Z^4$ et $c \in \N^*$\\
		On suppose $\left\{\begin{array}{l c r}a \equiv b[c]\\x \equiv y [c]\end{array}\right.$\\

		Alors $\left\{\begin{array}{l c r}a+x \equiv b+ y [c] \\ ax \equiv by [c] \end{array}\right.$\\

\end{prop}

\begin{prv}

		$c|b-a$ et $c|y-x$\\
		Donc $c|(b-a+y-x)$\\
		Donc $c|(b+y-(a+x))$\\
		Donc $a+x \equiv b+y [c]$\\

		On pose $\left\{\begin{array}{l c r}a = ck+b, k \in \Z \\ x = cl+y, l \in \Z\end{array}\right.$\\
		$ax = (ck+b)(cl+y)$\\
				$=by+cky+clk+c^2kl$\\
				$=by+c(ky+bl+clk)$\\
		Donc $ax \equiv by [c]$\\

\end{prv}

\begin{prop}[Critères de Divisibilité en Base 10]

		Soit $N \in \N$, on notera ses chiffres $a_0...a_n$\\
		$N = \overset{n}{\underset{k=0}{\sum}}10^ka_k$\\

		Divisibilité par $2$ :\\
				$N$ pair $\iff N \equiv 0[2]$\\
						$\iff \overset{n}{\underset{k=0}{\sum}} 10^ka_k \equiv 0 [2]$\\
						$\iff a_0 \equiv 0 [2]$ car	$\forall k \ge 1, 10^k \equiv 0 [2] \\10^0 = 1 \equiv 1 [2]\quad\ $\\

		Divisibilité par $3$ :\\
				$\forall k \in \N, 10^k \equiv 1^k \equiv1 [3]$ car $10 \equiv 1 [3]$\\
				$3|N \iff N \equiv 0 [3]$\\
						$\iff \overset{n}{\underset{k=0}{\sum}}10^ka_k \equiv 0 [3]$\\

		Divisibilité par $9$ : \\
				$\forall k \in N, 10^k \equiv 1 [9]$\\
				$9|N \iff N \equiv 0 [9]$\\
						$\iff \overset{n}{\underset{k=0}{\sum}}10^ka_k \equiv 0 [9]$\\

		Divisibilité par $5$ :\\
				$\left\{\begin{array}{l c r}10^0 \equiv 1 [5] \\ \forall k \in \N^*, 10^k \equiv 0 [5] \end{array}\right.$\\
				$5|N \iff a_0 \equiv 0 [5]$\\
						$\iff a_0 \in \{0,5\}$\\

		Divisibilité par $11$ :\\
				$10^0 \equiv -1 [11]$\\
				Donc $\forall k \in \N, 10^k \equiv (-1)^k[11]$\\
				$N \equiv 0 [11] \iff \overset{n}{\underset{k=0}{\sum}}(-1)^ka_k \equiv 0 [11]$\\
						$\iff a_0-a_1+a_2...+(-1)^na_n \equiv 0 [11]$\\

\end{prop}

\begin{rmk}[Réécriture en Classes d’Equivalence]

		On peut réecrire le calcul précédent dans $\Z/11 \Z$ \\
		$\overline N = \overline{\overset{n}{\underset{k=0}{\sum}}10^ka_k} = \overset{n}{\underset{k=0}{\sum}}\overline{10^k}\ \overline{a_k} = \overset{n}{\underset{k=0}{\sum}}\overline{(-1)^k}\ \overline{a_k}$\\

\end{rmk}

\begin{rmk}[Opération dans $\Z / n \Z$]

		Dans $\Z / n \Z$, on dispose $\left\{\begin{array}{l c r}\text{d’une addition} : \quad\quad\ \ \overline{a} + \overline{ b} = \overline{a+b} \\ \text{d'une multiplication} : \overline{a} * \overline{b} = \overline{a*b}\end{array}\right.$\\

		L’addition est commutative, associative, n’élément neutre $\overline{0}$ et l’opposé de $\overline{a}$ est $\overline{-a}$\\
		La multiplication est commutative, associative, d’élément neutre $\overline{1}$ et distributive par rapport à $+$\\



\end{rmk}


\part{PCGD et PPCM}


\begin{defn}[PGCD]

		Soient $(a,b) \in \Z^2$\\
		Le PGCD de $a$ et $b$ est le plus grand diviseur commun à $a$ et $b$\\
		Il existe car $\mathcal{D} = \{d \in \N|d|a \text{ et } d|b\}$ est non vide car $a \in \mathcal{D}$\\
		$\mathcal{D}$ est majoré par $|a| $\\
		On le note $\text{PGCD}(a,b)$ ou $a\wedge b$\\

\end{defn}

\begin{prop}[Théorème d’Euclide]

		Soient $a \in \Z, b \in \N^*$\\
		Soit $r$ le reste de la division de $a$ par $b$\\
		$a \wedge b = b \wedge r$\\

\end{prop}

\begin{prv}

		On pose $\left\{\begin{array}{l c r}d = a \wedge b\\ \varsigma = b \wedge r \\ a = bq+r\end{array}\right.$\\

		$\left\{\begin{array}{l c r}d|a\\d|b\end{array}\right. \Longrightarrow \left\{\begin{array}{l c r}d|a-bq\\d|b\end{array}\right. \Longrightarrow \left\{\begin{array}{l c r}d|r\\d|b\end{array}\right. \Longrightarrow  d \leq \varsigma$\\

		$\left\{\begin{array}{l c r}\varsigma|b\\\varsigma|r\end{array}\right. \Longrightarrow \left\{\begin{array}{l c r}\varsigma|bq+r\\\varsigma|b\end{array}\right. \Longrightarrow \left\{\begin{array}{l c r}\varsigma|a\\\varsigma|b\end{array}\right. \Longrightarrow \varsigma \leq d$\\

		Donc $d = \varsigma$\\

\end{prv}

\begin{prop}[PGCD et Diviseurs]

		Soient $(a,b) \in \Z^2$ et $d = a \wedge b$\\
		$\mathcal{D} = \{k \in \Z|\ k|a, k|b \}$\\
		$\forall k \in \Z, k \in \mathcal{D} \iff k|d$\\

\end{prop}

\begin{prv}

		“$\Longleftarrow$” : Soit $k \in \Z$, on suppose $k|d$\\
				$d|a$ donc $k|a$\\
				$d|b$ donc $k|b$\\

				Donc $k \in \mathcal{D}$\\

		“$\Longrightarrow$” : Soit $k \in \mathcal{D}$\\
				On pose $r_0$ le reste de la division de $a$ par $b$,\\
				$r_1$ le reste de la division de $b$ par $r_0$\\
				et $\forall n \in \N, r_{n+1}$ le reste de la division de $r_{n-1}$ par $r_n$ si $r_n \neq 0$\\

				La suite $(r_n)$ est décroissante, minorée par $0$ et à valeurs entières\\
				Soit $N \in \N$ tel que $r_N = 0$			(si $N=0$, on pose $r_{-1} = b$)\\

				D’après la poposition précédente,\\
				$d =a \wedge b = b \wedge r_0 = r_0 \wedge r_1...r_{N-1} \wedge r_N$\\
						$= r_{N-1} \wedge 0 = r_N$\\

				On pose aussi $\forall n \in [\![1, N-1]\!], r_{n-1} = r_nq_n + r_{n+1}$\\
				On en déduit que $\exists(\alpha_n, \beta_n) \in \Z^2, r_{N-1} = a \alpha_N + b \beta_N$\\
				$\left\{\begin{array}{l c r}k|a\\k|b\end{array}\right. \Longrightarrow k|a \alpha_N + b \beta_N \Longrightarrow k|r_{N-1} \Longrightarrow k|d$\\

\end{prv}

\begin{defn}[PPCM]

		Soit $(a,b) \in \Z^2$, on pose $M = \{k \in \N|\ a|k, b|k\}$\\
		$M \neq \varnothing$ car $ab \in M$\\
		$M \neq \varnothing$ donc admet un plus petit élément noté $\text{PPCM}(a,b)$ ou $a \vee b$\\

\end{defn}

\begin{prop}[Produit PGCD PPCM]

		$\forall (a,b) \in \Z^2, (a \wedge b)(a \vee b) = ab$\\

\end{prop}

\begin{prv}

		Voir paragraphe Facteurs Premiers\\

\end{prv}

\begin{prop}[Propriétés de $\wedge$ et $\vee$]

		$\wedge$ est commutative, associative sur $\Z^*$\\
		$\vee$ est commutative, associative sur $\Z^*$\\

\end{prop}

\begin{prv}

		Soient $(a,b,c) \in (\Z^*)^3, d = (a \wedge b) \wedge c, \varsigma = a \wedge (b \wedge c)$ \\

		$\left\{\begin{array}{l c r}d|c\\d|a \wedge b\end{array}\right. \Longrightarrow \left\{\begin{array}{l c r}d|c\\d|a\\d|b\end{array}\right.$\\

		$\left\{\begin{array}{l c r}\varsigma|a\\\varsigma|b \wedge c\end{array}\right. \Longrightarrow \left\{\begin{array}{l c r}\varsigma|a\\\varsigma|b\\\varsigma|c\end{array}\right.$\\

		On pose $\varepsilon = \text{PGCD}(a,b,c)$\\
		On a $\left\{\begin{array}{l c r}d \leq \varepsilon \\ \varsigma \leq \varepsilon\end{array}\right.$\\

		$\left\{\begin{array}{l c r}\varepsilon|a\\\varepsilon|b\\\varepsilon|c\end{array}\right. \Longrightarrow \left\{\begin{array}{l c r}\varepsilon|a\\\\\varepsilon|b \wedge c\end{array}\right. \Longrightarrow \varepsilon|a \wedge (b \wedge c) \Longrightarrow \varepsilon \leq d$\\

		De même, on $\varepsilon\leq \varsigma$\\
		Donc $d = \varepsilon = \varsigma$\\

\end{prv}

\begin{prop}[Théorème de Bézout]

		Soient $(a,b) \in \Z \cross \Z^*, d = a \wedge b$\\
		$\exists(u,v) \in \Z^2, d = au+bv$\\

\end{prop}

\begin{prv}

		On pose $A = \{au+bv|(u,v) \in \Z^2\}$\\
		On veut montrer que $d \in A$\\
				$a = a*1+b*0$ donc $a \in A$\\
				$b = a*0+b*1$ donc $b \in A$\\
				$0 = a*0+b*0$ donc $0 \in A$\\

		Soit $(x,y) \in A^2$\\
				$x = au_1+bv_1, (u_1,v_1) \in \Z^2$\\
				$y = au_2 + bv_2, (u_2,v_2) \in \Z^2$\\
				$x+y = a(u_1+u_2) + b(v_1+v_2) \in A$\\

		Soit $x \in A, k \in \Z$\\
				$x = au+bv, (u,v) \in \Z^2$\\
				$kx = aku+bkv \in A$\\

		Soit $n = \text{min}(A \cap \N^*)$		$(|b| \in A \cap \N^*)$\\
		Soit $x \in A$\\

		Par division euclidienne de $x$ par $n$ :\\
				$\left\{\begin{array}{l c r}x = nq+r \\ q \in A, 0 \leq r<n \end{array}\right.$\\

		$\left\{\begin{array}{l c r}x \in A\\n \in A\end{array}\right. \Longrightarrow \left\{\begin{array}{l c r}x \in A\\-qn \in A\end{array}\right. \Longrightarrow x-qn \in A \Longrightarrow r \in A$\\

		$\left\{\begin{array}{l c r}r < n \\r \in A\end{array}\right. \Longrightarrow r \leq 0$\\

		Donc $r = 0$\\
		Donc $n|x$\\

		D’où $A = n \Z$\\
		Or, $\left\{\begin{array}{l c r}a \in A\\b \in A\end{array}\right. \Longrightarrow \left\{\begin{array}{l c r}n|a\\n|b\end{array}\right.$\\

		Cas particulier : $a \wedge b = d = 1$, alors $1$ est le seul diviseur positif de $a$ et $b$\\
		Donc $n=1$ donc $A = \Z$ donc $1 \in \Z$\\

		Cas général : On pose $a' = \frac{a}{d} \in \Z,\ b' = \frac{b}{d} \in \Z,\ a' \wedge b' = 1$\\
		D’après le cas particulier, $\exist (u,v) \in \Z^2, a'u+b'v = 1$\\

		D’où $au+bv=d$\\

\end{prv}

\begin{prop}[Réciproque du Théorème de Bézout]

		Soient $(a,b) \in \Z\cross \Z^*$\\
		On suppose qu’il existe $(u,v) \in \Z^2$ tel que $au+bv=1$\\
		Alors $a \wedge b = 1$\\

\end{prop}

\begin{prv}

		On pose $d = a \wedge b$\\
		$\left\{\begin{array}{l c r}d|a\\d|b\end{array}\right. \Longrightarrow d|au+bv \Longrightarrow d|1 \Longrightarrow d=1$\\

\end{prv}

\begin{prop}[Théorème de \Gauss]

		Soient $(a,b,c) \in \Z^3$ tels que $\left\{\begin{array}{l c r}a \wedge b = 1\\a|bc\end{array}\right.$\\
		Alors $a|c$\\

\end{prop}

\begin{prv}

		D’après le théorème de Bézout,\\
		$au+bv = 1$ avec $(u,v) \in \Z^2$\\
		D’où $acu+bcv = c$\\

		$\left\{\begin{array}{l c r}a|acu\\a|bcv\end{array}\right. $ \\
		Donc $a|acu+bcv $\\
		Donc $a|c$\\

\end{prv}

\begin{rmk}[Inversion Modulo $n$]

		Soit $x \in \Z$\\
		$\exist?y \in \Z, xy \equiv 1 [n]$\\
		$(\iff$dans $\Z/n \Z, \overline{x} + \overline{y} = \overline{1} ?)$\\

		Avec $n=4 :$\\
				$\begin{matrix}x\ 0123\\0\ 0000\\1\ 0123 \\2 \ 0202 \\3\ 0321\end{matrix}$		$\left\{\begin{array}{l c r}1 \text{ et }3 \text{ sont inversibles modulo } 4 \\ 2 \text{ n'est pas inversible modulo } 4\end{array}\right.$ \\

		$3x \equiv 2[4]$\\
		$\iff 3 * 3x \equiv 3*2[4]$\\
		$\iff x \equiv 2[4]$\\

		$2x \equiv 1[4]$\\
		$\Longrightarrow 2*2x \equiv 2[4]$\\
		$\Longrightarrow 0 \equiv 2 [4]$\\

\end{rmk}

\begin{prop}[Congruences et Nombres Premiers]

		Soit $p$ un nombre premier\\
		Alors $\forall x \in \Z, x \not \equiv 0[p] \Longrightarrow \exists y \in \Z, xy \equiv 1 [p]$\\

\end{prop}

\begin{prv}

		Soit $x \in \Z$ tel que $x \not \equiv 0[p]$\\
		Soit $y \in \Z$\\
		$xy \equiv 1[p] \iff \exists u \in \Z, xy = 1+pu$\\
				$\iff \exists u \in \Z, xy -pu = 1$\\

		$y$ existe $\iff x \wedge p = 1$\\
				$\iff p \nmid x$\\

\end{prv}

\begin{prop}[Inversibilité Modulo $n$]

		Soit $n \in \N^*, x \in \Z$\\
		$x$ inversible modulo $n$ $\iff x \wedge n = 1$\\

\end{prop}

\begin{prv}

		Voir précédent\\

\end{prv}

\begin{prop}[Petit Théorème de Fermat]

		Soit $p$ premier, $a \in \Z$\\
		$a^p \equiv a[p]$\\

\end{prop}

\begin{prv}

		Cas 1 : $a \equiv 0[p]$\\
				$a^p \equiv 0^p[p]$\\
				$a^p \equiv 0[p] \equiv a [p]$\\

		Cas 2 : $a \not \equiv 0 [p]$\\
				Alors $a \wedge p = 1$\\

				On pose $\forall i \in \N^*, r_i$ le reste de la division de $ia$ par $p$\\
				Soit $i \in [\![1,p-1]\!]$\\
						$r_i = 0 \Longrightarrow p|ia \Longrightarrow p|i$		Contradiction\\
						$\forall i \in \N^*, r_i \neq 0$\\

				Soit $(i,j) \in [\![1,p-1]\!]^2, i \neq j$\\
						On suppose $r_i = r_j$, alors $ia \equiv ja [p]$\\
						Or, $a \wedge p = 1$ donc $a$ est inversible modulo $p$\\
						Donc $a \equiv j[p]$ donc $i = j$		Contradiction\\

				Ainsi, $r_1...r_{p-1} \in [\![1,p-1]\!]$ distincts donc ils prennent toutes les valeurs de $[\![1,p-1]\!]$\\
				$i \longmapsto r_i$ est injective\\
				$\{r_1...r_{p-1}\} = [\![1,p-1]\!]$\\

				Donc $\overset{p-1}{\underset{k=1}{\prod}} r_i = (p-1)!$\\
				Donc $\overset{p-1}{\underset{k=1}{\prod}} ia \equiv (p-1)![p]$\\
				Donc $(p-1)! a^{p-1} \equiv (p-1)![p]$\\

				D’où $(p-1)! \equiv 0[p] \iff p |1*2*3\dots*(p-1)$\\
						$\iff\exists i \in [\![1,p-1]\!], p|i$\\
				Donc $(p-1)! \not \equiv 0[p]$\\
				Donc $(p-1)!$ est inversible modulo $p$\\
				Donc $a^p \equiv 1[p]$\\
				Donc $a^p \equiv a[p]$\\



\end{prv}


\part{Décomposition en Facteurs Premiers}


\begin{defn}[Nombre Premier]

		Soit $n \in \N$\\
		On dit que $n$ est premier si\\
				$n \ge 2$\\
				les seuls diviseurs entiers de $n$ sont $1$ et $n$\\

\end{defn}

\begin{prop}[Infinité de Nombres Premiers]

		Il y a une infinité de nombres premiers\\

\end{prop}

\begin{prv}

		On suppose qu’il n’y a qu’un nombre fini de nombres premiers\\
				$p_1<...<p_n$\\

		On pose $N = p_n*...*p_1+1$\\
		$N>p_n$ donc $N$ n’est pas premier\\
		$N$ a d’autres diviseurs positifs que $1$ et $N$\\
		$N$ est divisible par un nombre entre $2$ et $N-1$\\

		Soit $p = \text{min}(\{k \in [\![2,N-1]\!] | k |N\})$\\
		$p$ est premier		(Tout diviseur de $p$ divise aussi $N$)\\
		$\exists i \in [\![1,n]\!], p_i=p$\\
		$p_i|N$\\
		$p_i|N-p_1...p_n$\\
		$p_i|1$		Contradiction\\

		Donc il y a une infinité de nombres premiers\\

\end{prv}

\begin{prop}[Théorème Fondamental de l’Arithmétique]

		“Tout entier se décompose en un unique produit de nombres premiers”\\

		Soient $n \in \N$ tel que $n \ge 2$ et $\mathcal{P}$ l’ensemble des nombres premiers\\
		$\exist! \nu : \mathcal{P} \longrightarrow \N$ telle que $\left\{\begin{array}{l c r}\{p \in \mathcal{P}|\nu(p) \neq 0\} \text{ est fini} \\ n = {\underset{p \in \mathcal{P}}{\prod}}p^{\nu(p)} \end{array}\right.$\\

\end{prop}

\begin{prv}

		Existence : Déjà vue : Récurrence Forte (Chapitre 9)\\

		Unicité : Soit $n \ge 2$ et $\nu : \mathcal{P} \longrightarrow \N$ telle que\\
		${(*)\underset{p \in \mathcal{ P}}{\prod}}p^{\mu(p)} = {\underset{p \in \mathcal{ P}}{\prod}}p^{\nu(p)}$	avec $\left\{\begin{array}{l c r}\mu\neq \nu \\ M = \{p \in \mathcal{P}|\mu(p)\neq 0\} \text{ fini} \\ M = \{p \in \mathcal{P}|\nu(p)\neq 0\} \text{ fini} \\ n \text{ minimale pour cette propriété} \end{array}\right.$\\

		Soit $p \in M, \mu(p) \neq 0$ donc $p|n$\\
		Si $\nu(p) = 0, \forall q \in \N, p \wedge q = 1$ donc $p|1$		Contradiction avec le théorème de \Gauss\\

		Donc on peut simplifier $(*)$ par $p$\\
		On a alors $2$ décompositions de $\frac{n}{p}<n$		Contradiction\\
		Donc $n$ n’existe pas\\

\end{prv}

\begin{prop}[Divisibilité et Nombres Premiers]

		Soient $(,b,c) \in \N^3$ supérieurs à $2$\\
		On pose $a = {\underset{p \in \mathcal{ P}}{\prod}} p^{\alpha(p)}$ et $b = {\underset{p \in \mathcal{ P}}{\prod}}p^{\beta(p)}$\\
		$a|b \iff \forall p \in \mathcal{P}, \alpha(p) \leq \beta(p)$\\

\end{prop}

\begin{prv}

		“$\Longrightarrow$” : On suppose $a|b, \exist k, b = ak$\\
				On pose $k = {\underset{p \in \mathcal{ P}}{\prod}}p^{\kappa(p)}$\\
				et donc $b = {\underset{p \in \mathcal{ P}}{\prod}}p^{\alpha(p)}{\underset{p \in \mathcal{ P}}{\prod}}p^{\kappa(p)} = {\underset{p \in \mathcal{ P}}{\prod}}p^{\alpha(p) + \kappa(p)}$\\

				Par unicité de la décomposition en facteurs premiers,\\
				$\forall p \in \mathcal{P}, \beta(p) = \alpha(p) + \kappa(p) \ge \alpha(p)$\\

		“$\Longleftarrow$” : On suppose $\forall p \in \mathcal{P}, \beta(p) \ge \alpha(p)$\\
				On pose $\forall p \in \mathcal{P}, \kappa(p) = \beta(p) - \alpha(p) \in \N$\\

				Tous les $\alpha(p)$ et $\beta(p)$ sont nuls à partir d’un certain rang\\
				C'est donc le cas aussi pour les $\kappa(p)$\\
				Donc on a le droit de former le produit\\
				${\underset{p \in \mathcal{ P}}{\prod}}p^{\kappa(p)} \in \N$\\

				On pose $k = {\underset{p \in \mathcal{ P}}{\prod}}p^{\kappa(p)}$ et $ak = {\underset{p \in \mathcal{ P}}{\prod}}p^{\alpha(p)}{\underset{p \in \mathcal{ P}}{\prod}}p^{\kappa(p)} = {\underset{p \in \mathcal{ P}}{\prod}}p^{\alpha(p) + \kappa(p)} = {\underset{p \in \mathcal{ P}}{\prod}}p^{\beta(p)}=b$\\

\end{prv}

\begin{prop}[Produit de Facteurs Premiers, PGCD et PPCM]

		Avec les notations précédentes,\\

		$a \wedge b = {\underset{p \in \mathcal{ P}}{\prod}}p^{\text{min}(\alpha(p), \beta(p))}$	et	$a \vee b = {\underset{p \in \mathcal{ P}}{\prod}}p^{\text{max}(\alpha(p), \beta(p))}$\\
		\\

\end{prop}

\begin{crlr}

$(a \wedge b)(a \vee b) = ab$\\

\end{crlr}

\begin{prv}

		$(a \wedge b)(a \vee b) = {\underset{p \in \mathcal{ P}}{\prod}}p^{\text{min}(\alpha(p), \beta(p))}{\underset{p \in \mathcal{ P}}{\prod}}p^{\text{max}(\alpha(p), \beta(p))}$\\
				$= {\underset{p \in \mathcal{ P}}{\prod}}p^{\text{min}(\alpha(p),\beta(p)) + \text{max}(\alpha(p), \beta(p))}$\\
				$={\underset{p \in \mathcal{ P}}{\prod}}p^{\alpha(p) + \beta(p)} = ab$\\

\end{prv}

