\part{Comparaison avec une intégrale}

\begin{thm}
	Soit $\alpha \in \R$. \[
		\sum \frac{1}{n^\alpha} \text{ converge } \iff \alpha > 1
	\] Dans ce cas, on note \[
		\zeta(\alpha) = \sum_{n = 1}^{+\infty} \frac{1}{n^\alpha}.
	\]
\end{thm}

\begin{prv}
	\[
		\frac{1}{n^\alpha} \tendsto{n\to +\infty} \begin{cases}
			0 &\text{ si } \alpha > 0\\
			1 &\text{ si } \alpha = 0\\
			+\infty &\text{ si } \alpha < 0
		\end{cases}
	\] donc $\sum \frac{1}{n^\alpha}$ diverge pour $\alpha \le 0$.

	On suppose $\alpha > 0$. Soit $f_\alpha : \begin{array}{rcl}
		\R^+_* &\longrightarrow& \R \\
		x &\longmapsto& \frac{1}{x^\alpha} = x^{-\alpha}
	\end{array}$

	\begin{figure}[H]
		\centering
		\begin{asy}
			import graph;
			import math;

			size(5cm);

			draw((-1, 0) -- (4, 0), Arrow(TeXHead));
			draw((0, -1) -- (0, 3), Arrow(TeXHead));

			real alpha = 1.3;
			real f(real x) {return 1/ x^alpha; }
			bool3 check(real x) { return x != 0 && f(x) < 3; };

			draw(graph(f, 0, 3.8, check), red);

			real k = 1.5;
			real ak = 0.25;
			real bk = 2.75;

			dot("\small$k$", (k, 0), align=S);
			dot("\small$k - 1$", (ak, 0), align=S);
			dot("\small$k + 1$", (bk, 0), align=S);

			draw(box((ak, f(k)), (bk, 0)));
			draw((k, 0) -- (k, f(k)));
			draw((ak, f(k)) -- (-0.05, f(k)), dashed);
			label("\small$\frac{1}{k^\alpha}$", (0, f(k)), align=W);
		\end{asy}
	\end{figure}

	Soit $k \in \N$ avec $k \ge 2$. comme $f_\alpha$ est décroissante, \[
		\forall k \in [k, k+1], f_{\alpha}(x) \le f_{\alpha}(k)
	\] et donc \[
		\int_{k}^{k+1} \frac{1}{x^\alpha}~\mathrm{d}x \le \int_{k}^{k+1} \frac{1}{k^\alpha}~\mathrm{d}x = \frac{1}{k^\alpha}.
	\] De même, \[
		\forall k \in [k - 1, k], f_\alpha(x) \ge f_\alpha(k)
	\] et donc \[
		\int_{k-1}^{k} \frac{1}{x^\alpha}~\mathrm{d}x \ge \int_{k-1}^{k} \frac{1}{k^\alpha}~\mathrm{d}x = \frac{1}{k^\alpha}.
	\] Soit $n \in \N$ avec $n \ge 2$. \[
		\forall k \in \left\llbracket 2,n \right\rrbracket, 
		\int_{k}^{k+1} \frac{1}{x^\alpha}~\mathrm{d}x \le \frac{1}{k^\alpha} \le \int_{k-1}^{k} \frac{1}{x^\alpha}~\mathrm{d}x
	\] donc \[
		\int_{2}^{n+1} x^\alpha~\mathrm{d}x \le \sum_{k=2}^n \frac{1}{k^\alpha} \le \int_{1}^{n} \frac{1}{x^\alpha}~\mathrm{d}x
	\]

	\begin{itemize}
		\item[\underline{\sc Cas 1}] On suppose $\alpha > 1$. Alors
			\begin{align*}
				\forall n \ge 2, \sum_{k=1}^n \frac{1}{k^\alpha} &\le 1 + \int_{1}^{n} \frac{1}{x^\alpha}~\mathrm{d}x = 1 + \left[ \frac{x^{-\alpha + 1}}{-\alpha + 1} \right]^n_1\\
				&\le 1 + \frac{1}{1-\alpha}\left( \frac{1}{n^{\alpha - 1}} - 1 \right)\\
				&\le 1 + \frac{1}{\alpha - 1}\left( 1 - \frac{1}{n^{\alpha - 1}} \right)\\
				&\le 1 + \frac{1}{\alpha - 1}.
			\end{align*}
			La suite $\left( \sum_{k=1}^n \frac{1}{k^\alpha} \right)_n$ est croissante et majorée donc elle converge.
		\item[\underline{\sc Cas 2}] On suppose $\alpha = 1$. \[
				\forall n \ge 2, \sum_{k=1}^n \frac{1}{k} \ge 1 + \int_{2}^{n+1} \frac{1}{x}~\mathrm{d}x \ge \underbrace{1 + \ln(n+1) - \ln 2}_{\tendsto{n\to +\infty} +\infty}
			\] Par comparaison, $\sum_{k=1}^n \frac{1}{k} \tendsto{n\to +\infty} +\infty$.
		\item[\underline{\sc Cas 3}] On suppose $\alpha > 1$.
			\begin{align*}
				\forall n \ge 2, \sum_{k=1}^n \frac{1}{k^\alpha} &\ge 1 + \int_{2}^{n+1} \frac{1}{x^\alpha}~\mathrm{d}x\\
				&\ge 1 + \left[ \frac{x^{-\alpha+1}}{-\alpha + 1} \right]_2^{n+1}\\
				&\ge \underbrace{1 + \frac{1}{1-\alpha}\left( \frac{1}{(n+1)^{\alpha - 1}} - \frac{1}{2^{\alpha - 1}} \right)}_{\mathclap{\tendsto{n\to +\infty} +\infty \text{ car } \alpha < 1}}
			\end{align*}

			Donc, $\sum_{k=1}^n \frac{1}{k^\alpha} \tendsto{n\to +\infty} +\infty$.
	\end{itemize}
\end{prv}

\begin{thm}
	Soit $f : [a, +\infty[ \longrightarrow \R^+$ continue, décroissante de limite nulle, avec $a \in \N$. Alors, \[
		\sum_{n\ge a}f(n) \text{ converge} \iff \left( \int_{a}^{n} f(x)~\mathrm{d}x \right)_n \text{ converge}.
	\]
\end{thm}

\begin{prv}
	Soit $k \in \N$ tel que $k \ge a + 1$. \[
		\forall x \in [k, k+1], f(x) \le f(k)
	\] donc \[
		\int_{k}^{k+1} f(x)~\mathrm{d}x \le \int_{k}^{k+1} f(k)~\mathrm{d}x = f(k)
	\] et \[
		\forall x \in [k-1, k], f(x) \ge f(k)
	\] et donc \[
		\int_{k-1}^{k} f(x)~\mathrm{d}x \ge \int_{k-1}^{k} f(k)~\mathrm{d}x = f(k).
	\]

	Donc, $\forall n \in \N$ avec $n \ge a + 1$ \[
		\int_{a+1}^{n+1} f(x)~\mathrm{d}x \le \sum_{a+1\le k\le n} f(k) \le \int_{a}^{n} f(x)~\mathrm{d}x.
	\]

	\begin{itemize}
		\item[\underline{\sc Cas 1}] On suppose que $\left( \int_{a}^{n} f(x)~\mathrm{d}x \right)_n$ converge. Cette suite est croissante, donc majorée. Soit $M$ un majorant donc \[
				\forall n \ge a+ 1, \sum_{a \le k \le n} f(k) \le f(a) + M.
			\] donc la série converge.
		\item On suppose que $\left( \int_{a}^{n} f(x)~\mathrm{d}x \right)_n$ diverge donc, par croissance de cette suire, \[
			\lim_{n\to +\infty} \int_{a}^{n} f(x)~\mathrm{d}x = +\infty
		\] et donc
		\begin{align*}
			\forall n \ge a + 1, \sum_{k=a}^n f(k) &= f(a) + \sum_{k=a+1}^n f(k) \\
			&\ge \underbrace{f(a) + \int_{a}^{n} f(x)~\mathrm{d}x - \int_{a}^{a+1} f(x)~\mathrm{d}x}_{\tendsto{n\to +\infty} +\infty}
		\end{align*} donc la série diverge.
	\end{itemize}
\end{prv}

\begin{exo}\relax
	{\itshape Quelle est la nature de la série $\sum \frac{1}{n \ln(n)}$ ?}
\end{exo}

\begin{exo}\relax
	{\itshape Quelle est la nature de la série $\sum \frac{\ln^\alpha(n)}{n^\beta}$ en fonction de $\alpha$ et $\beta$ ?}
\end{exo}
