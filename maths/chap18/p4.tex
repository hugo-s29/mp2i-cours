\part{L'espace vectoriel $\mathbbm{K}[X]$}

\begin{rmk}
	[Rappel]
	$(\mathbbm{K}[X], +, \cdot)$ est un $\mathbbm{K}$-espace vectoriel engendré par $(1, X, X^2, \ldots)$
\end{rmk}

\begin{prop}
	La famille $(X^n)_{n\in\N}$ est libre.
\end{prop}

\begin{prv}
	Soit $(\lambda_n)_{n\in\N}$ une famille presque nulle de scalairees telle que $\sum_{n \in \N} \lambda_n X^n = 0$ \\
	$(\lambda_n)_{n\in\N}$ est un polynôme de $\mathbbm{K}[X]$: on le note $P$.\\
	Or, \[
		\sum_{n\in \N} \lambda_n X^n = (\lambda_0, \lambda_1, \ldots, \lambda_n, \ldots) = P
	\] 
	Donc $P = 0$ donc \[
		\forall n \in \N, \lambda_n = 0
	\] 
\end{prv}

\begin{crlr}
	\[
		\dim\big(\mathbbm{K}[X]\big) = +\infty
	\]\qed
\end{crlr}

\begin{defn}
	Pour $n \in \N$, on note \[
		\mathbbm{K}_n[X] = \big\{  P \in \mathbbm{K}[X] \mid \deg(P) \le n \big\} 
	\]
	\index{espace vectoriel des polynômes à coefficients inférieurs à $n$}
\end{defn}

\begin{thm}
	$\mathbbm{K}_n[X]$ est un sous-espace vectoriel de $\mathbbm{K}[X]$ de dimension $n+1$
\end{thm}

\begin{prv}
	$\mathbbm{K}_n[X] = \Vect\big(1, X, \ldots, X^n\big)$
\end{prv}

\begin{prop}
	Soit $(P_i)_{i\in I}$ une famille de polynômes non nuls telle que \[
		\forall  i \neq j, \deg(P_i) \neq \deg(P_j)
	\] Alors $(P_i)_{i\in I}$ est libre.
\end{prop}

\begin{prv}
	Soit $n \in \N$ et $i_1, \ldots, i_n$ des éléments distincts de $I$ \\
	Soient $\lambda_1, \ldots, \lambda_n \in \mathbbm{K}$. On suppose \[
		\lambda_1 P_{i_1} + \cdots + \lambda_n P_{i_n} = 0
	\] Quitte à renuméroter les polynômes, on peut supposer que \[
		\forall k \in \left\llbracket 1,n \right\rrbracket,
		\deg\left( P_{i_n} \right) > \deg\left( P_{i_k} \right)
	\] 

	Si $\lambda_n \neq 0$,
	\begin{align*}
		\deg\left( \lambda_1 P_{i_1} + \cdots + \lambda_n P_{i_n} \right)  = \deg\left( P_{i_n} \right)  \neq -\infty
	\end{align*}
	Donc $\lambda_n = 0$\\
	Donc $\lambda_1 P_{i_1} + \cdots + \lambda_{n-1} P_{i_{n-1}} = 0$ \\
	
	On conclut par récurrence sur $n$.
\end{prv}

\begin{thm}
	[Formule de Taylor]
	Soit $P \in \mathbbm{K}_n[X]$ et $a \in \mathbbm{K}$. \[
		P(X) = \sum_{k=0}^n \frac{P^{(k)}(a)}{k!} (X-a)^{k}
	\] 
\end{thm}

\begin{prv}
	$\big(1, X-a, \ldots, (X-a)^{n}\big)$ est libre.\\
	Comme $\dim\big(\mathbbm{K}_n[X]\big) = n+1$, c'est une base de $\mathbbm{K}_n[X]$.\\
	Donc, il existe $(\lambda_0, \ldots, \lambda_n) \in \mathbbm{K}^{n+1}$ tel que \[
		P = \sum_{k=0}^n \lambda_k (X-a)^k
	\] On remarque que \[
		P(a) = \lambda_0
	\]
	\begin{align*}
		\forall i \in \left\llbracket 1,n \right\rrbracket,
		P^{(i)}(a) &= \sum_{k=0}^n \lambda_k \underbrace{\left( (X-a)^k \right)^{(i)}}_{\mathrlap{\fontsize{7pt}{7pt}\selectfont = \begin{cases}
				0& \text{ si } k  < i\\
				i!& \text{ si } k = i\\
				\frac{k!}{(k-i)!} (X-a)^{k+1} &\text{ si } k > i
		\end{cases}}}(a)\\
		&= \lambda_i i! \\
	\end{align*}

	Donc $\lambda_i = \frac{P^{(i)}(a)}{i!}$
\end{prv}

\begin{prop}
	Soit $P \in \mathbbm{K}[X]$ et $a \in \mathbbm{K}$. \[
		\left.\begin{array}{r}
			a \text{ est une racine de } P\\
			\text{ de multiplicité } \mu
		\end{array}\right\} \iff \left\{\begin{array}{l}
			\forall k \le \mu-1, P^{(k)}(a) = 0\\[1mm]
			P^{(\mu)}(a) \neq 0
		\end{array}\right.
	\] 
\end{prop}

\begin{prv}
	On pose $n = \deg(P)$\\
	 \begin{itemize}
		\item[$``\impliedby"$ ]
			\begin{align*}
				P &= \sum_{k=0}^n \frac{P^{(k)}(a)}{k!}(X-a)^{k}\\
				&= \sum_{k=\mu}^n \frac{P^{(k)}(a)}{k!} (X-a)^k \\
				&= (X-a)^\mu \underbrace{\sum_{k=\mu}^n \frac{P^{(k)}(a)}{k!} (X-a)^{k-\mu}}_{Q \in \mathbbm{K}[X]} \\
			\end{align*}
			Donc $\begin{cases}
				(X-a)^\mu  \mid  P\\
				Q(a) = \frac{P^{(\mu)}(a)}{\mu!} \neq 0
			\end{cases}$ 
		\item[$``\implies"$]~\\
			$\begin{cases}
				 P = (X-a)^{\mu} Q\\
				 Q(a) \neq 0
			\end{cases}$ 

			\begin{align*}
				\forall k \le \mu - 1, 
				P^{(k)}(a) &= \sum_{j=0}^k {k \choose j} \left( (X-a)^\mu \right)^{(j)} (a) Q^{(k-j)}(a) \\
				&= \sum_{j=0}^k {k \choose j} \frac{\mu!}{(\mu-j)!} \underbrace{(a-a)^{\mu-j}}_{ = 0} Q^{(k-j)}(a) \\
				&= 0 \\
			\end{align*}

			\begin{align*}
				P^{(\mu)}(a) &= {\mu \choose \mu} \times \mu! \times 1 \times Q^{(0)}(a) \\
				&= Q(a) \\
				&\neq 0
			\end{align*}
	\end{itemize}
\end{prv}

\begin{crlr}
	Avec les notations précédentes, si $a$ est une racine de $P$ de multiplicité $\mu$, alors $a$ est une racine de $P'$ de multiplicité $\mu-1$ \\
	\qed
\end{crlr}

\begin{defn}
	On dit qu'un polynôme $P$ est \underline{scindé} sur $\mathbbm{K}$ si $P$ est un produit de polynômes de $\mathbbm{K}[X]$ de degré 1, i.e. toutes les racines de $P$ sont dans $\mathbbm{K}$
	\index{scindé (polynôme)}
\end{defn}

\begin{exo}
	\begin{enumerate}
		\item Soit $P \in \R[X]$ scindé sur $\R$ à racines simples avec $\deg(P) \ge 2$. Montrer que $P'$ est scindé sur $\R$ à racines simple.
		\item Soit $P \in \R[X]$ scindé avec $\deg(P) \ge 2$. Montrer que $P'$ est scindé.
	\end{enumerate}
	\vspace{5mm}
	\marginpar{\fbox{Solution}}
	\begin{enumerate}
		\item~\\
			\begin{minipage}{\linewidth}
				\begin{wrapfigure}
					{r}{5cm}
					\centering
					\begin{asy}
						import graph;
						size(5cm);
						axes(EndArrow);
						
						real[] coeffs = {-0.0052538717183996655,-0.0018668631201231925,0.7863750556279947,0.050000000000000704};

						real f(real x) {
							real y = 0;
							int k = coeffs.length;

							for(int i = 0; i < k; ++i) {
								y += x^i * coeffs[k-i-1];
							}

							return y;
						}

						draw(graph(f, -16, 16), deepmagenta);

						real[] roots = {-12.382, -0.064, 12.09};
						real[] droots = {-7.183, 6.946};
						real eps = 4;

						for(real root: roots) {
							dot((root, f(root)), deepcyan);
						}

						for(real droot: droots) {
							real y = f(droot);
							draw((droot - eps, y) -- (droot + eps, y), red, Arrows(TeXHead));
						}
					\end{asy}
				\end{wrapfigure}
				Soit $P \in \R[X]$ avec $\deg(P) = n$ scindé sur $\R$.\\
				On note $x_1< x_2< \cdots< x_n$ les $n$ racines de $P$\\
				Soit  $f_P: \R \to \R$ la fonction polynomiale. Aussi, $f_P$ est $\mathcal{C}^{\infty}$ sur $\R$.\\
				D'après le théorème de Rolle, \[
					\forall i \in \left\llbracket 1, n-1 \right\rrbracket, \exists y_i \in ]x_i, x_{i+1}[,
					f'_P(y_i) = 0
				\] Donc $y_1, \ldots, y_{n-1}$ sont racines de $P'$.\\
				De plus, \[
					y_1 < x_2 < y_2 < x_3 < y_3 < \cdots < y_{n-1}
				\] On a donc trouvé $n-1$ racines distinctes de $P'$. Or, $\deg(P') = n-1$.\\
				Donc, on a trouvé TOUTES les racines complexes de  $P'$. Donc $P'$ est sciendé à racines simples.\\
			\end{minipage}
		\item On note $x_1 < \cdots < x_p$ les racines de $P$ et $n = \deg(P)$. On note pour tout $i \in \left\llbracket 1,p \right\rrbracket$, $\mu_i$ la multiplicité de $x_i$. Donc, \[
			\sum_{i=1}^p \mu_i = n
		\] D'après le théorème de Rolle, \[
			\forall i \in \left\llbracket 1,p-1 \right\rrbracket, \exists y_i \in ]x_i, x_{i+1}[, P'(y_i) = 0
		\] On a trouvé $p-1$ racines réelles de $P'$.
		$\forall i \in \left\llbracket 1, p \right\rrbracket, x_i$ est une racine de $P'$ de multiplicité $\mu-1$.\\
		Ce qui fait, $\sum_{i=1}^p (\mu_i - 1) = n - p$ racines réelles de $P'$ comptées avec multiplicité.\\
		En tout, on a trouvé $n-1$ racines réelles de $P'$ comptées avec multiplicité.\\
		Comme $\deg(P') = n - 1$, $P'$ n'a pas d'autres racines donc $P'$ est scindé.
	\end{enumerate}
\end{exo}

\begin{exo}
	[Problème]
	
	\begin{wrapfigure}
		{r}{6cm}
		\centering
		\begin{asy}
			import graph;
			size(6cm);

			axes(EndArrow);

			real[] coeffs = {-0.01152728642222377,-0.058424162190707105,1.0180575851032627,2.3480560198039857};

			real f(real x) {
				real y = 0;
				int k = coeffs.length;

				for(int i = 0; i < k; ++i) {
					y += x^i * coeffs[k-i-1];
				}

				return y;
			}

			bool3 check(real x) {return abs(f(x)) < 9 && abs(f(-x)) < 9;}

			draw(graph(f, -16, 16, check), deepmagenta);
			real step = 32 / 6;
			real x0 = -16 + step * 1.4;
			int i = 0;
			for(real x = x0; x <= 16 - step; x += step) {
				pair z = (x, f(x));
				dot(z);
				draw(z -- (x,0), dotted);
				if(z.y > 0)
					label("$\small x_" + string(i + 1) + "$", (x,0), align = S);
				else
					label("$\small x_" + string(i + 1) + "$", (x,0), align = N);
				++i;
			}
		\end{asy}
	\end{wrapfigure}

	Soient $(x_1, \ldots, x_n) \in \mathbbm{K}^n$ tels que \[
		\forall i \neq j, x_i \neq x_j
	\] Soient $(y_1, \ldots, y_n) \in \mathbbm{K}^n$.
	On cherche $P \in \mathbbm{K}[X]$ de \underline{degré minimal} tel que \[
		(*) \quad \forall i \in \left\llbracket 1,n \right\rrbracket, P(x_i) = y_i
	\]

	Soit $\varphi : \begin{array}{rcl}
		\mathbbm{K}[X] &\longrightarrow& \mathbbm{K}^n \\
		P &\longmapsto& \big(P(x_1), \ldots, P(x_n)\big)
	\end{array}$\\
	\[
		(*) \iff \varphi(P) = (y_1, \ldots, y_n)
	\] On cherche, parmi tous les antécédants de $(y_1, \ldots, y_n)$ celui de plus bas degré.\\
	$\varphi$ est linéaire :
	\begin{align*}
		\forall P,Q \in \mathbbm{K}[X], \forall \alpha, \beta \in \mathbbm{K},\\
		\varphi(\alpha P + \beta Q) &= \big(\alpha P(x_1) + \beta Q(x_1), \ldots, \alpha P(x_n) + \beta Q(x_n) \big) \\
		&= \big(\alpha P(x_1), \ldots, \alpha P(x_n)\big) + \big(\beta Q(x_1), \ldots, \beta Q(x_n)\big) \\
		&= \alpha \varphi(P) + \beta \varphi(Q) \\
	\end{align*}
	\begin{itemize}
		\item Donc $\varphi$ est un morphisme de groues additifs.
		\item $(y_1, \ldots, y_n) = \sum_{i = 0}^n y_i e_i$ où $(e_1, \ldots, e_n)$ est la bade canonique de $\mathbbm{K}^n$
	\end{itemize}
	Si on trouve $L_1, \ldots, L_n \in \mathbbm{K}[X]$ tels que $\varphi(L_1) = e_1, ~\ldots~ \varphi(L_n) = e_n$, alors
	\begin{align*}
		\varphi\left(\sum_{i=1}^n y_i L_i\right) &= \sum_{i=0}^n y_i \varphi(L_i)\\
		&= \sum_{i=1}^n y_i e_i\\
		&= (y_1, \ldots, y_n) \\
	\end{align*}
	\begin{center}
		---
	\end{center}
	\begin{align*}
		P \in \Ker(\varphi) &\iff \varphi(P) = (0,\ldots, 0)\\
												&\iff \forall i \in \left\llbracket 1,n \right\rrbracket, P(x_i) = 0\\
												&\iff \exists Q \in \mathbbm{K}[X], P = (X - x_1)\cdots(X-x_n) Q
	\end{align*}
	Soit $i \in \left\llbracket 1,n \right\rrbracket$ et $L_i \in \mathbbm{K}[X]$.\\
	\begin{align*}
		\varphi(L_i) = e_i &\iff \big(L_i(x_1), L_i(x_2), \ldots, L_i(x_n)\big) = (0, \ldots, 0, \underbrace{1}_i, 0, \ldots, 0) \\
											&\iff \begin{cases}
												L_i(x_i) = 1\\
												\forall j \neq i, L_i(x_j) = 0
											\end{cases}\\
											&\iff \begin{cases}
												\exists Q \in \mathbbm{K}[X], L_i = \prod_{\substack{1 \le j \le n\\ j \neq i}} (X - x_j) Q\\
												1 = \prod_{\substack{1 \le j \le n\\ j \neq i}} (x_i - x_j) Q(x_i)\\
											\end{cases}\\
											&\impliedby L_i = \prod_{j \neq i} \frac{X - x_j}{x_i - x_j}
	\end{align*}
	D'où,
	\[
		\varphi(P) = (y_1, \ldots, y_n) \iff \exists Q \in \mathbbm{K}[X], P = \underbrace{\sum_{i=1}^n y_i L_i}_{\mathllap{\begin{array}{c}
			\text{solution particulière}\\
			\deg(\cdot ) \le n-1
		\end{array}}} + \underbrace{\prod_{k=1}^n (X-x_k)Q}_{\mathrlap{\begin{array}{c}
			\text{solutions de l'équation}\\
			\text{homogène associée}\\
			\deg(\cdot) \ge n
		\end{array}}}
	\]
	\underline{Le} polynôme de plus bas degré solution du problème d'interpolation est \[
		\sum_{i=1}^n y_i \prod_{j \neq i} \frac{X - x_j}{x_i - x_j}
	\]
\end{exo}

\begin{defn}
	Soit $(x_1, \ldots, x_n) \in \mathbbm{K}^n$ avec \[
		\forall i \neq j, x_i \neq x_j
	\] On pose \[
		\forall i \in \left\llbracket 1,n \right\rrbracket,
		L_i = \prod_{\substack{1 \le j \le n\\ j \neq i}} \frac{X - x_j}{x_i - x_j}
	\]
	$L_i$ est le \underline{$i$-ème polynôme interpolateur de Lagrange} associé à $(x_1, \ldots, x_n)$: \[
		\forall j \in \left\llbracket 1,n \right\rrbracket, L_i(x_j) = \delta_{i,j}
	\]
	\index{interpolateur de Lagrange (polynôme)}
\end{defn}

\begin{prop}
	Avec les notations précédentes, $(L_1, \ldots, L_n)$ est une base de $\mathbbm{K}_{n-1}[X]$.
\end{prop}

\begin{prv}
	\begin{itemize}
		\item $\forall i \in \left\llbracket 1,n \right\rrbracket, \deg(L_i) = n - 1$
		\item Soit $P \in \mathbbm{K}_{n-1}[X]$. On pose \[
				\forall i \in \left\llbracket 1,n \right\rrbracket, y_i = P(x_i)
			\]
			On pose $Q = \sum_{i=1}^{n-1}y_i L_i$. $Q$ est le seul polynôme de degré $\le n-1$ tel que $Q(x_i) = y_i$ pour tout $i$.\\
			Donc, $P = Q \in \Vect(l_1, \ldots, L_n)$. Donc $(L_1, \ldots, L_n)$ est une famille génératrice de $\mathbbm{K}_{n-1}[X]$. Or, $\dim\big(\mathbbm{K}_{n-1}[X]\big) = n$. Donc $(L_1, \ldots, L_n)$ est une base de $\mathbbm{K}_{n-1}[X]$
	\end{itemize}
\end{prv}

\begin{exm}
	$\mathbbm{K} = {}^\Z / {}_{5\Z}$ et $n = 3$
	
	\begin{multicols}
		{2}
		$x_1 = \overline{2}$ \\
		$x_2 = \overline{0}$ \\
		$x_3 = \overline{-1}$ \\
		$y_1 = \overline{1}$ \\
		$y_2 = \overline{1}$ \\
		$y_3 = \overline{2}$
	\end{multicols}
	Le seul polynôme de degré $\le 2$ tel que $P(x_i) = y_i$ pour tout $i \in \left\llbracket 1,2 \right\rrbracket$ est $\sum_{i=1}^3 y_i \prod_{\substack{1 \le j \le 3\\j \neq i}}\frac{X - x_j}{x_i - x_j}$\\

	\begin{align*}
		L_1 &= (x_1-x_2)^{-1} (x_1-x_3)^{-1} (X-x_2)(X-x_3) \\
		&= \overline{3}\times \overline{2}\times X\left(X+\overline{1}\right) = X\left( X + \overline{1} \right) = X^2 + X \\
		~\\
		L_2 &= (x_2 - x_1)^{-1} (x_2 - x_3)^{-1} (X - x_1)(X - x_3)\\
		&= \overline{2}\times \overline{1} \left( X - \overline{2} \right) \times \left( X - \overline{1} \right) \\
		&= \overline{2}X^2 + \overline{3}X + \overline{1} \\
		~\\
		L_3 &= (x_3 - x_1)^{-1} (x_3 - x_2)^{-1} (X - x_1) (X - x_2)\\
		&= \overline{3} \times  \overline{4} \times  \left(X - \overline{2}\right) \times X\\
		&= \overline{2}X\left( X - \overline{2} \right)  \\
		&= \overline{2}X^2 + X \\
	\end{align*}
	Donc,
	\begin{align*}
		P &= X^2 + X + \overline{2}X + \overline{3}X + \overline{1} + \overline{4}X^2 + \overline{2}\\
		&= \overline{2}X^2 + X + \overline{1} \\
	\end{align*}
	Vérification:
	\begin{align*}
		P\left( \overline{2} \right) = \overline{3} + \overline{2} + \overline{1} = \overline{1} = y_1\\
		P\left( \overline{0} \right) = \overline{1} = y_2\\
		P\left( \overline{-1} \right) = \overline{2} - \overline{1} + \overline{1} = \overline{2} = y_3
	\end{align*}
\end{exm}
