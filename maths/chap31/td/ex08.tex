\part{Exercice 8}

\begin{enumerate}
	\item $X(\Omega) = \left\llbracket 1,n+1 \right\rrbracket$.
		\[
			\forall k \in \left\llbracket 1,n \right\rrbracket,\, P(X=k) = q^{k-1}\,p
		.\] (cela correspond, à quelques détails près, à la loi géométrique)

		Également, $P(X = n+1) = q^{n}$.

		\underline{Vérification} : $\sum_{k=1}^n q^{k-1} p + q^{n} = p \frac{1-q^n}{1-q} q^n = 1$.
	\item
		\begin{align*}
			E(X) &= \sum_{k=1}^{n+1} k\:P(X = k)\\
			&= \sum_{k=1}^n k q^{k-1} p + (n+1)q^n \\
			&= p\,\frac{\mathrm{d}}{\mathrm{d}q}\left( \sum_{k=0}^n q^k \right) + (n+1)q^n \\
			&= p\, \frac{\mathrm{d}}{\mathrm{d}q}\left( \frac{1-q^{n}}{1-q} \right) + (n+1)q^n \\
			&= p \frac{-(n+1)q^n (1-q) + (1-q^{n+1})}{(1-q)^2} + (n+1)q^n \\
			&= -(n+1)q^n \frac{1-q^{n+1}}{p} + (n+1)q^n \\
			&= \frac{1-q^{n+1}}{p} \\
		\end{align*}
	\item On cherche $p$ tel que
		\begin{align*}
			P(X = n+1) \le \frac{1}{2} \iff& q^n \le \frac{1}{2}\\
			\iff& q \le \frac{1}{\sqrt[n]{2}}
		\end{align*}
\end{enumerate}

