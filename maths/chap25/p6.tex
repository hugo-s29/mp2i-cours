\part{Séries alternées}

\begin{thm}
	Soit $u \in \left(\R^+\right)^\N$ décroissante de limite nulle. Alors $\Sigma (-1)^n u_n$ converge.
\end{thm}

\begin{prv}
	On pose \[
		\forall n \in \N, S_n = \sum_{k=0}^n (-1)^k u_k
	\]

	Montrons que $(S_{2n})$ et $(S_{2n+1})$ sont adjacentes.

	\begin{itemize}
		\item Soit $n \in \N$.
			\begin{align*}
				S_{2n+2} - S_{2n} &= (-1)^{2n+1} u_{2n+1}+(-1)^{2n+2} u_{2n+2}\\
				&= u_{2n+2} - u_{2n+1} \le 0 \\
			\end{align*}

			Donc $(S_{2n})_n$ est décroissante.
		\item Soit $n \in \N$.
			\begin{align*}
				S_{2n+3} - S_{2n+1} &= u_{2n+2} - u_{2n+3} \ge 0 \\
			\end{align*}
			Donc la suite $\left( S_{2n+1} \right)_n$ est croissante.
		\item \(
				\forall n \in \N, S_{2n+1} - S_{2n} = -u_{2n+1}\tendsto{n\to +\infty} 0.
			\)
	\end{itemize}

	Ainsi, $(S_{2n})$ et $(S_{2n+1})$ convergent et ont la même limite, donc $(S_n)$ converge.
	On note $S = \lim_{n\to +\infty} S_n$ : \[
		\forall n \in \N, S_{2n+1} \le S \le S_{2n}
	\] donc \[
		\forall n \in \N, R_{2n+1} \ge 0 \ge R_{2n}.
	\]

	Soit $n \in \N$: \[
		\left| R_{2n+1} \right| = R_{2n+1} = S - S_{2n+1} \le S_{2n} - S_{2n+1} = u_{2n+1},
	\] \[
		\left| R_{2n} \right| = -R_{2n} = S_{2n} - S \le S_{2n} - S_{2n+1} \le u_{2n+1} \le u_{2n}.
	\]

	Ainsi, \[
		\forall n \in \N, |R_n| \le u_n.
	\]
\end{prv}

\begin{prop}
	Soit $u$ une suite de signe constant telle que $\big(|u_n|\big)_n$ est décroissante de limite nulle.
	Alors, $\Sigma(-1)^n u_n$ converge et \[
		\forall n \in \N, R_n \text{ est du signe de } (-1)^{n+1} u_{n+1} \et |R_n| \le |u_n|
	\]\qed
\end{prop}
