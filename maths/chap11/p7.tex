\part{Annexe}

\begin{prop}
	Soit $f: I \to I$ continue et $(u_n)_{n\in\N}$ définie par \[
	\begin{cases}
		u_0 \in I\\
		\forall n \in \N, u_{n+1} = f(u_n)
	\end{cases}
	\] 
	{\bf Si} $(u_n)$ converge vers $\ell\in \R$\\
	alors  $f(\ell) = \ell$ i.e. ($\ell$ est un point fixe de $f$)
\end{prop}

\begin{prv}
	On suppose que $(u_n)$ converge vers $\ell$.\\
	$\lim_{n\to +\infty} u_{n+1} = \ell$ car $(u_{n+1})$ est une sous suite de $(u_n)$. \[
	\forall n \in \N, u_{n+1} = f(u_n)
	\]
	Comme $f$ est continue alors $f(u_n) \tendsto{n\to +\infty} f(\ell)$. Par unicité de la limite, $\ell = f(\ell)$
\end{prv}

\begin{rmk}
	Soit $u \in \R^\N$ et $f: \R \to \R$ dérivable telle que $u_{n+1} = f(u_n)$.\\
	Soit $\ell \in \R$ un point fixe de $f$. Donc, $f(\ell) = \ell$.\\

	$\left| f'(\ell) \right| > 1$:\\
	\begin{center}
		\begin{asy}
			import graph;

			size(5cm);

			xaxis(EndArrow);
			draw((-1,0)--(1,0));

			dot("{}", (0,0), magenta);
			label("$\ell$", (0,0.15), magenta);
			

			draw((-.45,.1)--(-.4,.1)--(-.4,-.1)--(-.45,-.1), solid+1);
			draw((.45,.1)--(.4,.1)--(.4,-.1)--(.45,-.1), solid+1);

			draw((.15,.05)--(.2,0)--(.15,-.05), solid+1);
			draw((-.15,.05)--(-.2,0)--(-.15,-.05), solid+1);

			draw((-.4,-.25)--(.4,-.25),Arrows);
			label("répulsion", (0, -.4));
		\end{asy}
	\end{center}
	$\left| f'(\ell) \right| < 1$:\\
	\begin{center}
		\begin{asy}
			import graph;

			size(5cm);

			xaxis(EndArrow);
			draw((-1,0)--(1,0));

			dot("{}", (0,0), magenta);
			label("$\ell$", (0,0.15), magenta);
			

			draw((-.45,.1)--(-.4,.1)--(-.4,-.1)--(-.45,-.1), solid+1);
			draw((.45,.1)--(.4,.1)--(.4,-.1)--(.45,-.1), solid+1);

			draw((.2,.05)--(.15,0)--(.2,-.05), solid+1);
			draw((-.2,.05)--(-.15,0)--(-.2,-.05), solid+1);

			draw((-.4,-.25)--(.4,-.25),Arrows);
			label("bassin d'attraction", (0, -.4));
		\end{asy}
	\end{center}

	Par contre, si $\left| f'(\ell) \right| =1$, on ne sait pas.
\end{rmk}

\begin{rmk}
	[Suite arithético-géométrique]
	\[
		(*): \forall n \in \N, u_{n+1} = au_n + b = f(u_n)
	\]
	\begin{itemize}
		\item[\sc Méthode 1]~\\
			\begin{itemize}
				\item On cherche $v$ une suite constante solution de $(*)$ : \[
					\exists C \in \R, \forall n \in \N, v_n = C
				\] donc \[
					\forall n \in \N, C = aC + b = f(C)
				\] Si $a \neq 1$: $C = \frac{b}{1-a}$ 
			\item Soit $u$ qui vérifie $(*)$. On pose $w = u - v$.
				\begin{align*}
					\forall n \in \N, w_{n+1} &= u_{n+1} - v_{n+1} \\
					&= au_n + b - av_n - b \\
					&= a(u_n - v_n) \\
					&= aw_n\\
				\end{align*}
				Donc $\forall n \in \N, w_{n+1} = aw_n + 0$ : équation homogène associée à $(*)$
			\end{itemize}
			$(w_n)$ est géométrique donc \[
				\forall n \in \N, w_n = w_0 a^n
			\] et donc \[
				\forall n \in \N, u_n = w_0a^n + \frac{b}{1-a}
			\]
		\item[\sc Méthode 2]~\\
			\begin{align*}
				\varphi: \R^\N &\longrightarrow \R^\N \\
				(u_n) &\longmapsto (u_{n+1} - au_n)
			\end{align*}
			$\varphi$ morphisme de groupes additifs \[
				\varphi(u) = (b) \iff u = v + w \text{ avec } w \in \Ker(\varphi)
			\]
			\begin{align*}
				w \in \Ker(\varphi) &\iff \varphi(w) = 0\\
														&\iff \forall n \in \N, w_{n+1} - aw_n = 0\\
														&\iff \forall n \in \N, w_{n+1} = aw_n
			\end{align*}
	\end{itemize}
\end{rmk}
