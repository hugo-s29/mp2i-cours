\part{Exercice 7}

On suppose $n > 2$.
Soit $\sigma \in Z(S_n)$ : \[
	\forall \sigma'\in S_n, \sigma\;\sigma' = \sigma'\;\sigma.
\]

En particulier,
\begin{align*}
	\forall k \in \left\llbracket 2,n \right\rrbracket,\; &\sigma\mat{1&k}\sigma^{-1} = \mat{1&k}\\
	&\phantom{\sigma\mat{1}}\vrt=\\
	&\mat{\sigma(1)&\sigma(k)}
\end{align*}

Donc $\sigma(1) \in \{1,k\}\;\forall k$ et donc $\sigma(1) = 1$. On en déduit que \[
	\forall k \ge 2, \sigma(k) = k
\] et donc $\sigma = \id$.

$S_2$ est commutatif : $\mat{1&2}$ commute avec $\id$ et $\mat{1&2}$.

