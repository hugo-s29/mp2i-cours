\part{Algorithme de Metropolis-Hastings}

On a un texte à déchiffrer. Il a été codé par substitution : on a remplacé une lettre par une autre (par exemple, ``$A \to W$'' et ``$B \to U$''). L'algorithme de Metropolis-Hastings permet de déchiffrer ce type de message.

Cet algorithme fonctionne itérativement, et, pour chaque lettre, il tente de la permuter avec une autre et il conserve la meilleur.

Pour comparer deux permutations, il utilise des {\it bigrammes} : un couple de deux lettres consécutives (où l'on a supprimé les espaces). Avec un entrainement sur un autre texte de la même langue, l'algorithme determine la {\it plausibilité} de chaque bigramme. Au moment de comparer deux permutations, il compare les plausibilités des bigrammes obtenus à la suite des deux permutations.

On se place dans le groupe symétrique et la fonction qui donne la plausibilité associée à une permutation est une fonction du groupe symétrique dans les réels : $p : S_n \longrightarrow \R$ (en français, $n = 26$).

L'algorithme ressemble donc à :

\begin{mdframed}
	On choisit une permutation.

	Si, en l'appliquant au texte, la plausibilité augmente, on conserve cette permutation. Sinon, on ne la conserve pas.

	Et on répète.
\end{mdframed}

Cet algorithme utilise donc une marche aléatoire, il ne teste pas {\it toutes} les permutations (avec $n = 26$, on a $\#S_n \ge 10^{61}$).

\begin{defn}[Chaîne de \Markov]
	Le système est dans un état $E$ et peut passer dans d'autres états avec des probabilités associées. Ce changement d'état ne dépend {\bf que} de l'état actuel, et n'utilise pas les états passés.

	\begin{figure}[H]
		\centering
		\begin{asy}
			pair s1 = (0,0);
			pair s2 = (1,1);
			pair s3 = (1.3,0);
			pair s4 = (1, -1);
			pair s5 = (2, 2);

			size(5cm);

			draw(circle(s1, 0.3)); label("$S_1$", s1);
			draw(circle(s2, 0.3)); label("$S_2$", s2);
			draw(circle(s3, 0.3)); label("$S_3$", s3);
			draw(circle(s4, 0.3)); label("$S_4$", s4);
			draw(circle(s5, 0.3)); label("$S_5$", s5);
		\end{asy}
	\end{figure}

	On associe à une chaîne de \Markov~une matrice de transition : \[
		\begin{pNiceMatrix}[first-row,last-col]
			&\substack{i\\\vrt\to}&&\\
			&\\
			&P(S_j \mid S_i)&&\leftarrow j\\
			&\\
		\end{pNiceMatrix}
	.\]
\end{defn}

On souhaite trouver $\max_{x \in S} f(x)$. Pour cela, on considère le changement $x \longrightarrow x'$ et on pose $\textstyle \alpha = \frac{f(x')}{f(x)}$. On ne conserve $x'$ avec une probabilité de $\alpha$, sinon, on revient à $x$.

Mais, la probabilité de transition est proportionnelle à $f(x')$ et donc, pour que ce choix soit marqué, il faut que la fonction $f$ ait une grande variance.

Cette méthode de choix ne peut être appliquée que si la chaîne de \Markov~est symétrique et connexe. La symétrie de la chaîne de \Markov~est assurée par le fait qu'une permutation est bijective, on peut donc appliquer sont inverse pour revenir à un état précédent.


