\part{Déterminant d'un endomorphisme}

\begin{prop}
	Soit $f \in \mathcal{L}(E)$, $\mathcal{B} = (e_1, \ldots, e_n)$ une base de $E$. Alors, \[
		\exists ! \lambda \in \mathbbm{K},\, \forall (u_1, \ldots, u_n) \in E,\,
		\det_\mathcal{B}\big(f(u_1),\ldots,f(u_n)\big) = \lambda \det_\mathcal{B}(u_1, \ldots, u_n).
	\]
\end{prop}

\begin{prv}
	Soit $g : \begin{array}{rcl}
		E^n &\longrightarrow& \mathbbm{K} \\
		(u_1, \ldots,u_n) &\longmapsto& \det_\mathcal{B}\big(f(u_1),\ldots, f(u_n)\big)
	\end{array}$.
	$g$ est clairement alternée.

	Soit $i \in \left\llbracket 1,n \right\rrbracket$, $(u_1, \ldots, u_i, u_{i+1}, \ldots, u_n) \in E^{n-1}$.
	L'application $u\mapsto g(u_1, \ldots, u_{i-1}, u, u_{i+1}, \ldots, u_n)$ est la composée de $f$ et de $v \mapsto \det_\mathcal{B}\big(f(u_1), \ldots, f(u_{i-1}), v, f(u_{i+1}), \ldots, f(u_n)\big)$ donc elle est linéaire.

	Donc, $g$ est une forme multilinéaire alternée donc colinéaire à $\det_\mathcal{B}$.
\end{prv}

\begin{prop}
	Soit $f \in \mathcal{L}(E)$, $\mathcal{B}$ et $\mathcal{C}$ deux bases de $E$. Soient $(\lambda, \mu) \in \mathbbm{K}^2$ tel que \[
		\forall (u_1, \ldots, u_n) \in E^n,
		\begin{cases}
			\det_\mathcal{B}\big(f(u_1), \ldots, f(u_n)\big) = \lambda \det_\mathcal{B}(u_1, \ldots, u_n),\\
			\det_\mathcal{C}\big(f(u_1), \ldots, f(u_n)\big) = \mu \det_\mathcal{C}(u_1, \ldots, u_n).
		\end{cases}
	\] Alors, $\lambda = \mu$.
\end{prop}

\begin{prv}
	On pose $\mathcal{B} = (e_1, \ldots, e_n)$.
	\begin{align*}
		\det_\mathcal{C}\big(f(e_1),&\ldots, f(e_n)\big) = \mu \det_\mathcal{C}(e_1, \ldots, e_n) = \mu \det_\mathcal{C}(\mathcal{B})\\
		&\vrt=\\
		\det_\mathcal{C}(\mathcal{B})&\,\det_\mathcal{B}\big(f(e_1), \ldots, f(e_n)\big)\\
		&\vrt=\\
		\lambda\,\underbrace{\det_\mathcal{C}(\mathcal{B})}_{\neq 0}&\,\underbrace{\det_\mathcal{B}(\mathcal{B})}_{=1}
	\end{align*}
	donc $\lambda = \mu$.
\end{prv}

\begin{defn}
	Soit $f \in \mathcal{L}(E)$. Le \underline{déterminant} de $f$ est le seul scalaire vérifiant, \[
		\forall \mathcal{B} \text{ base de } E,\, \forall (u_1, \ldots, u_n), \det_\mathcal{B}\big(f(u_1),\, \ldots, f(u_n)\big) = \lambda \det_\mathcal{B}(u_1, \ldots, u_n)
	\] et on le note $\lambda = \det(f)$.
	\index{déterminant (d'une application)}
\end{defn}

\begin{rmk}
	Si $n = 2$, \[
		\mathrm{Aire}\big(f(u_1), f(u_2)\big) = \det(f)\, \mathrm{Aire}(u_1, u_2).
	\]
\end{rmk}

\begin{prop}
	Soient $f$ et $g$ deux endomorphismes de $E$. Alors, \[
		\det(f \circ g) = \det f \times \det g.
	\]
\end{prop}

\begin{prv}
	Soit $\mathcal{B} = (e_1, \ldots, e_n)$ une base de $E$.
	\begin{align*}
		\det_\mathcal{B}\big(f \circ g(e_1),& \ldots, f \circ g(e_n)\big) = \det(f \circ g) \det_\mathcal{B}(e_1, \ldots, e_n) = \det(f \circ g)\\
			&\vrt=\\
			\det_\mathcal{B}\Big(f\big(g(e_1)\big)&, \ldots, f\big(g(e_n)\big)\Big)\\
			&\vrt=\\
			(\det f)\; \det_\mathcal{B}\big(&g(e_1), \ldots, g(e_n)\big)\\
			&\vrt=\\
			(\det f)\,(\det g)&\;\underbrace{\det_\mathcal{B}(e_1, \ldots, e_n)}_{=1}.
	\end{align*}
\end{prv}

\begin{crlr}
	Si $f \in \mathrm{GL}(E)$, alors $\det(f) \neq 0$ et $\det(f^{-1}) = \det(f)^{-1}$.
\end{crlr}

\begin{prv}
	On suppose $f \in \mathrm{GL}(E)$ : \[
		f \circ f^{-1} = \id_E.
	\] Donc, \[
		\det(f)\;\det(f^{-1}) = \det(f \circ f^{-1}) = \det(\id_E).
	\] Soit $\mathcal{B} = (e_1, \ldots, e_{n})$ une base de $E$.
	\begin{align*}
		\det_\mathcal{B}\big(\id_E(e_1), \ldots&, \id_E(e_n)\big) = \det(id_E) \times 1\\
		&\vrt=\\
		\det_\mathcal{B}&(\mathcal{B}) = 1
	\end{align*}
\end{prv}

\begin{prop}
	Soit $f \in \mathcal{L}(E)$. On suppose $f\not\in \mathrm{GL}(E)$. Alors $\det(f) = 0$.
\end{prop}

\begin{prv}
	Soit $\mathcal{B} = (e_1, \ldots, e_n)$ une base de $E$. \[
		\det(f) = \det_\mathcal{B}\big(f(e_1), \ldots, f(e_n)\big).
	\] $f$ n'est pas surjective donc $\rg\big(f(e_1), \ldots, f(e_n)\big) < n$ donc $\big(f(e_1), \ldots, f(e_n)\big)$ est liée donc $\det_\mathcal{B}\big(f(e_1), \ldots, f(e_n)\big) = 0$.
\end{prv}

