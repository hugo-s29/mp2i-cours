\part{Exercice 4}

Soit $f : t \mapsto \begin{cases}
	\sin\left( \frac{1}{t} \right) &\text{ si } t \neq 0\\
	0 &\text{ sinon}.
\end{cases}$

On suppose $f$ continue par morceaux. Soit $I = [-1, 1]$ et $\sigma = (a_0, \ldots, a_n)$ une subdivision de $I$ telle que \[
	\begin{cases}
		\forall i \in \left\llbracket 1,n-1 \right\rrbracket,\, f_{\big|]a_i,a_{i+1}[\big.} \text{ continue};\\
		\forall i \in \left\llbracket 1,n-1 \right\rrbracket,\,f \text{ a une limite réelle à gauche et à droite}.
	\end{cases}
\]

Comme $f$ n'est pas continue en $0$, \[
	\exists i,\,0 = a_i
.\]

Or, $f$ n'a pas de limite en $0$ à droite : une contradiction.

En effet, les suites $(x_n)_{n\in\N}$ et $(y_n)_{n\in\N}$ ne convergent pas vers la même limite avec \[
	\forall n \in \N^*,\,\begin{cases}
		x_n = \frac{1}{n \pi} \tendsto{n\to +\infty} 0^+\\
		y_n = \frac{1}{2n\pi + \frac{\pi}{2}} \tendsto{n\to +\infty}0^+
	\end{cases}
\] car \[
	\forall n \in \N^*,\,
	\begin{cases}
		f(x_n) = \sin(n\pi) = 0\tendsto{n\to +\infty} 0\\
		f(y_n) = \sin\left( \frac{\pi}{2} + 2n\pi \right) = 1 \tendsto{n\to +\infty} 1 \neq 0.
	\end{cases}
\]

Donc $f$ n'a pas de limite en $0^+$.

