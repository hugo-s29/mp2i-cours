\documentclass[a4paper]{report}
\usepackage[utf8]{inputenc}
\usepackage[T1]{fontenc}
\usepackage{textcomp}
\usepackage[french]{babel}
\usepackage{amsmath, amssymb}
\usepackage{bbm}
\usepackage{amsthm}
\usepackage{tikz}
\usepackage{pgfplots}
\usepackage{mathtools}
\usepackage{tkz-tab}
\usepackage[inline]{asymptote}
\usepackage{frcursive}
\usepackage{verbatim}
\usepackage{moresize}
\usepackage{algorithm}
\usepackage{algpseudocode}
\usepackage{calligra}
\usepackage{diagbox}
\usepackage{centernot}
\usepackage{multicol}
\usepackage{nicematrix}
\usepackage{stmaryrd}
\usepackage{setspace}
\usepackage{chngpage}
\usepackage{cancel}
\usepackage{esvect}
\usepackage{wrapfig}
\usepackage{floatflt}
\usepackage{calligra}
\usepackage[cuteinductors,european,straightvoltages,europeanresistors]{circuitikz}

\usetikzlibrary{babel}
\usetikzlibrary{tikzmark,calc,fit,arrows}

\newif\ifsimple
\simplefalse
\let\underlin\underline

\usepackage{graphicx}
\newcommand\longvdots[1]{\raisebox{1em}{\rotatebox{-90}{\hbox to #1 {\dotfill}}}}

\pgfplotsset{compat=1.17}
\let\vec\vv

\definecolor{green}{HTML}{60A917}

\everymath{\displaystyle}
\let\textstyle\displaystyle
\let\scriptstyle\displaystyle
\let\scriptscriptstyle\displaystyle

\def\asydir{asy}

% figure support
\usepackage{import}
\usepackage{xifthen}
\pdfminorversion=7
\usepackage{pdfpages}
\usepackage{transparent}
\newcommand{\incfig}[1]{%
	\def\svgwidth{\columnwidth}
	\import{./figures/}{#1.pdf_tex}
}


\usepackage{calrsfs}
\usepackage{mathrsfs}
\usepackage{stmaryrd}
\usepackage{float}

\setlength{\parindent}{0em}
\setlength{\parskip}{0em}

\let\oldemptyset\emptyset
\let\emptyset\varnothing

\let\ge\geqslant
\let\le\leqslant

\newcommand{\C}{\mathbbm{C}}
\newcommand{\R}{\mathbbm{R}}
\newcommand{\Z}{\mathbbm{Z}}
\newcommand{\N}{\mathbbm{N}}
\newcommand{\Q}{\mathbbm{Q}}
\renewcommand{\O}{\emptyset}

\renewcommand\Re{\expandafter\mathfrak{Re}}
\renewcommand\Im{\expandafter\mathfrak{Im}}

\renewcommand{\thepart}{\Roman{part}} 

\DeclareMathOperator{\Arctan}{Arctan}
\DeclareMathOperator{\Card}{Card}
\DeclareMathOperator{\Ker}{Ker}
\DeclareMathOperator{\Aut}{Aut}
\DeclareMathOperator{\id}{id}
\DeclareMathOperator{\rg}{rg}
\DeclareMathOperator{\argmax}{argmax}
\DeclareMathOperator{\argmin}{argmin}
\DeclareMathOperator{\Vect}{Vect}
\DeclareMathOperator{\cotan}{cotan}
\DeclareMathOperator*{\po}{\text{\cursive o}}
\DeclareMathOperator*{\dom}{dom}

\pdfsuppresswarningpagegroup=1
\reversemarginpar

\newcommand{\defblock}[5]{
	\newenvironment{#1}[1][{}]
		{
			\if##1\empty\else
				{#3 ##1 #5}
			\fi
			\marginpar{#2}
			~\\
		}
		{
			#4
			\vspace{8mm}
		}
}

\newcommand{\defoptblock}[6]{
	\AtBeginDocument{
		\ifsimple
			\newenvironment{#1}[1][{}]
				{
					\expandafter\comment
				}
				{
					\expandafter\endcomment
					\vspace{-8mm}
					#6
					\vspace{8mm}
				}
		\else
			\defblock{#1}{#2}{#3}{#4}{#5}
		\fi
	}
}

\defblock{defn}{\bf Definition}{\bf\hfill}{}{\hfill}
\defblock{prop}{\bf Proposition}{\bf\hfill}{}{\hfill}
\defblock{crlr}{\bf Corollaire}{\bf\hfill}{}{\hfill}
\defblock{prop-defn}{\bf Proposition\\Définition}{\bf\hfill}{}{\hfill}
\defblock{thm}{\bf Théorème}{\bf\hfill}{}{\hfill}
\defblock{lem}{\bf Lemme}{\bf\hfill}{}{\hfill}

\defoptblock{exm}{Exemple}{}{}{}{}
\defoptblock{exo}{Exemple}{}{}{}{}

\defblock{rmk}{\it Remarque}{\it}{}{}
\defoptblock{prv}{\it Preuve}{\it}{\qed}{\hfill}{\hfill$\blacksquare$}

% \AtBeginDocument{
	% \newenvironment{cblk}[3][{}][{}]
		% {
			% \if#1\empty\else
				% {#1}
			% \fi
			% \marginpar{#3}
			% ~\\
		% }
		% {
			% #2
			% \vspace{8mm}
		% }
% }

\makeatother
\usepackage{fancyhdr}
\pagestyle{fancy}

\fancyhead[R]{}
\fancyhead[L]{\thepart}
\fancyhead[C]{\parttitle}

\fancyfoot[R]{\thepage}
\fancyfoot[L]{}
\fancyfoot[C]{}

\newcommand*\parttitle{}
\let\origpart\part
\renewcommand*{\part}[2][]{%
   \ifx\\#1\\% optional argument not present?
      \origpart{#2}%
      \renewcommand*\parttitle{#2}%
   \else
      \origpart[#1]{#2}%
      \renewcommand*\parttitle{#1}%
   \fi
}

\makeatletter

\newcommand{\tendsto}[1]{\xrightarrow[#1]{}}
\newcommand{\danger}{{\large\fontencoding{U}\fontfamily{futs}\selectfont\char 66\relax}\;}
\newcommand{\ex}{\fbox{ex}\;}
\renewcommand{\mod}[1]{~\left[ #1 \right]}

\newcommand{\vrt}[1]{\rotatebox{90}{$#1$}}

\DeclareMathOperator{\ou}{\text{ ou }}
\DeclareMathOperator{\et}{\text{ et }}
\DeclareMathOperator{\si}{\text{ si }}
\DeclareMathOperator{\non}{\text{ non }}

\renewcommand{\title}[2]{
	\AtBeginDocument{
		\begin{titlepage}
			\begin{center}
				\vspace{10cm}
				{\Large \sc Chapitre #1}\\
				\vspace{1cm}
				{\HUGE \cursive #2}\\
				\vfill
				Hugo {\sc Salou} MP2I\\
				{\ssmall Dernière mise à jour le \@date }
			\end{center}
		\end{titlepage}
	}
}


\newcommand{\chap}[2][0]{
	\setcounter{chapter}{#1 - 1}
	\chapter{#2}
	\renewcommand*\parttitle{#2}
}

\let\part\section
\let\section\subsection
\AtBeginDocument{\fulltrue}

\renewcommand{\thesection}{\arabic{section}}

\usepackage{pgfornament}
\usepackage{makeidx}
\usepackage{fancyhdr}
\usepackage[totoc]{idxlayout}
\usepackage{tocbibind}
\usepackage{titletoc}
\usepackage{xpatch}

\pagestyle{fancy}

%\fancyhead[R]{\itshape MP2I}
%\fancyhead[L]{\arabic{chapter}.\arabic{section}}
%\fancyhead[C]{\parttitle}

\fancyfoot{}
\fancyfoot[C]{\thepage}

\makeindex

\begin{document}
	\begin{titlepage}
		\begin{center}
			\vspace{10cm}
			{\Large \itshape 2021-2022}\\
			\vspace{3cm}
			\pgfornament[width=8cm]{88}\\
			\vspace{2mm}
			\vspace{0.5cm}
			{\HUGE Mathématiques}\\
			\vspace{0.5cm}
			{\fontsize{240pt}{260pt}\selectfont MP2I}\\
			\vspace{0.5cm}
			\pgfornament[width=8cm]{88}\\
			\vfill
			Hugo {\sc Salou}\\
		\end{center}
	\end{titlepage}

	\tableofcontents



	{
		\chap[00]{Logique (rudiments)}
		\renewcommand{\cwd}{../chap00}
		\begin{defn}
	Un \underline{proposition} est un énoncé qui est soit vrai, soit faux.
\end{defn}

\begin{exm}
	\begin{align*}
		\begin{rcases*}
			A: ``B \text{ est vraie }"\\
			B: ``A \text{ est fausse }"\\
		\end{rcases*} \text{ Le système $\{A,B\}$ est une \underline{auto-contradiction}}
	\end{align*}
\end{exm}

\begin{defn}
	\underline{Démontrer} une proposition revient à prouver qu'elle est vraie
\end{defn}

		\begin{defn}
	Soit $E$ un $\mathbbm{K}$-espace vectoriel. On dit que $E$ est de \underline{dimension finie} si $E$ a au moins une famille génératrice finie. On dit que $E$ est de \underline{dimension infinie} sinon.
	\index{dimension finie (espace vectoriel)}
	\index{dimension infinie (espace vectoriel)}
\end{defn}

\begin{thm}
	[Théorème de la base extraite]
	Soit $E$ un $\mathbbm{K}$-espace vectoriel non nul de dimension finie. Soit $\mathcal{G}$ une famille génératrice finie de $E$. Alors, il existe une base $\mathcal{B}$ de $\mathcal{E}$ telle que $\mathcal{B} \subset \mathcal{G}$.
\end{thm}

\begin{prv}
	[par récurrence sur $\#G = \Card(G)$]
	\begin{itemize}
		\item Soit $E$ un $\mathbbm{K}$-espace vectoriel non nul engendré par $\mathcal{G} = (u)$.\\
			Si $u = 0_E$, alors $E = \{0_E\}$: une contradiction $\lightning$ \\
			Donc $u \neq 0_E$ donc $(u)$ est libre. En effet, \[
				\forall \lambda \in \mathbbm{K}, \lambda u = 0_E \implies \lambda = 0_\mathbbm{K}
			\] Donc $\mathcal{G}$ est une base de $E$.\\
		\item Soit $n \in \N_*$. Soit $E$ un $\mathbbm{K}$-espace vectoriel. On suppose que si $E$ a une famille génératrice constituée de $n$ vecteurs, alors on peut extraire de cette famille une base de $E$.\\
			Soit $\mathcal{G}$ une famille génératrice de $E$ avec $n+1$ vecteurs.\\
			Si $\mathcal{G}$ est libre, alors $\mathcal{G}$ est une base de $E$. \\
			Si $\mathcal{G}$ n'est pas libre, alors il existe $u \in \mathcal{G}$ tel que $u \in \Vect(\mathcal{G}\setminus \{u\})$ \\
			Donc $\mathcal{G}\setminus \{u\}$ engendre $E$. Or, $\mathcal{G}\setminus \{u\}$ possède $n$ vecteurs. D'après l'hypothèse de récurrence, il existe une base $\mathcal{B}$ de $E$ telle que \[
				\mathcal{B} \subset \mathcal{G} \setminus \{u\} \subset \mathcal{G}
			\] 
	\end{itemize}
\end{prv}

\begin{crlr}
	Tout espace de dimension finie a une base.
	\qed
\end{crlr}

\begin{thm}
	[Théorème de la base incomplète]
	Soit $E$ un $\mathbbm{K}$-espace vectoriel de dimension finie, $\mathcal{G}$ une famille génératrice finie de $E$. $\mathcal{L}$ une famille libre de $E$. Alors, il existe une base $\mathcal{B}$ de $E$ telle que \[
		\mathcal{L} \subset \mathcal{B} \text{ et } \mathcal{B}\setminus \mathcal{L} \subset \mathcal{G}
	\] 
\end{thm}

\begin{prv}
	[par récurrence sur $\#(\mathcal{G}\setminus\mathcal{L})$]
	\begin{itemize}
		\item Avec les notations précédentes, on suppose que $\mathcal{G}\setminus\mathcal{L} \neq \O$ \[
				\forall u \in \mathcal{G}, u \in \mathcal{L}
			\] Donc $\mathcal{G} \subset \mathcal{L}$ donc $\mathcal{L}$ est génératrice donc $\mathcal{L}$ est une base de $E$. On pose $\mathcal{B} = \mathcal{L}$ et alors \[
				\mathcal{L} \subset  \mathcal{B} \text{ et } \mathcal{B}\setminus\mathcal{L} = \O \subset  \mathcal{G}
			\] 
		\item Soit $n \in \N$. On suppose que si $\mathcal{G}$ est génératrice et $\mathcal{L}$ libre avec $\#(\mathcal{G}\setminus\mathcal{L}) = n$ alors il existe une base $\mathcal{B}$ de $E$ telle que \[
			\mathcal{L}\subset \mathcal{B} \text{ et } \mathcal{B}\setminus\mathcal{L}\subset \mathcal{G}
		\] Soient à présent $\mathcal{G}$ une famille génératrice de $E$ et $\mathcal{L}$ une famille libre de $E$ telles que $\#(\mathcal{G}\setminus\mathcal{L}) = n+1 > 0$\\
		Si $\mathcal{L}$ engendre $E$, alors $\mathcal{L}$ est une base de $E$. On pose $\mathcal{B} = \mathcal{L}$ et on a bien \[
			\mathcal{L} \subset  \mathcal{B} \text{ et } \mathcal{B} \setminus \mathcal{L} = \O \subset  \mathcal{G}
		\] On suppose que $\mathcal{L}$ n'engendre pas $E$. Il existe $u \in \mathcal{G}$ tel que $u \not\in \Vec(\mathcal{L})$ (car sinon, $\mathcal{G} \subset \Vect(\mathcal{L})$ et donc $\underbrace{\Vect(\mathcal{G})}_{= E} \subset  \underbrace{\Vect(\mathcal{L})}_{ \subset E}$\\
		Donc $\mathcal{L} \cup \{u\} $ est libre. On pose $\mathcal{L}' = \mathcal{L} \cup \{u\} $ \[
			\mathcal{G}\setminus \mathcal{L}' = \mathcal{G}\setminus (\mathcal{L} \cup \{u\}) = (\mathcal{G}\setminus\mathcal{L})\setminus \{u\} 
		\] donc $\#(\mathcal{G}\setminus\mathcal{L}') = n+1 -1 = n$\\
		D'après l'hypothèse de récurrence, il existe $\mathcal{B}$ une base de $E$ telle que \[
			\mathcal{L} \subset  \mathcal{L}' \subset \mathcal{B} \text{ et } \mathcal{B}\setminus \mathcal{L}' \subset \mathcal{G}
		\] \[
			\mathcal{B} \setminus \mathcal{L} = \underbrace{\mathcal{B}\setminus\mathcal{L}'}_{\subset \mathcal{G}} \cup \underbrace{\{u\}}_{\subset \mathcal{G} \text{ car } u \in \mathcal{G}}
		\] On a $\mathcal{B}\setminus\mathcal{L}\subset \mathcal{G}$
	\end{itemize}
\end{prv}

\begin{thm}
	Soit $E$ un $\mathbbm{K}$-espace vectoriel de dimension finie. Toutes les bases de $E$ ont le même cardinal.
\end{thm}

\begin{prv}
	Soit $\mathcal{G}$ une famille génératrice finie de $E$ et $\mathcal{B} \subset  \mathcal{G}$ une base de $E$. On note $n = \#\mathcal{B}$ \\
	Soit $\mathcal{B}'$ une base de $E$. On pose $p = n - \#(\mathcal{B} \cap  \mathcal{B}')$. Montrons par récurrence sur  $p$ que $\#\mathcal{B} = \#\mathcal{B}'$ 
	\begin{itemize}
		\item On suppose que $p = 0$. Alors, $\#(\mathcal{B} \cap \mathcal{B}') = n$ \\
			Or, $\mathcal{B}' \cap \mathcal{B} \subset \mathcal{B}$ donc $\mathcal{B} \cap \mathcal{B}' = \mathcal{B}$ donc $\mathcal{B} \subset  \mathcal{B}'$ et donc $\mathcal{B} = \mathcal{B}'$ 
		\item Soit $p \in \N$. On suppose que si $\mathcal{B}'$ est une base de $E$ telle que $n - \#(\mathcal{B} \cap \mathcal{B}') = p$, alors $\#\mathcal{B}' = n$ \\
			Aoit $\mathcal{B}'$ une base de $E$ telle que $n - \#(\mathcal{B}\cap \mathcal{B}') = p+1 > 0$ \\
			Donc $\mathcal{B} \cap \mathcal{B}' \neq \mathcal{B}$. Soit $u \in \mathcal{B}' \setminus \mathcal{B}$. D'après le lemme d'échange, il existe $v \in \mathcal{B}\setminus \mathcal{B}'$ tel que $\mathcal{B}' \setminus \{u\} \cup \{v\}$ est une base de $E$. On pose $\mathcal{B}'' = \mathcal{B}' \setminus \{u\} \cup \{v\}$ 
			\begin{align*}
				\mathcal{B}'' \cap \mathcal{B} &= \left( (\mathcal{B}' \setminus \{u\})  \cap \mathcal{B} \right) \cup \{v\} \\
				&= (\mathcal{B}' \cap \mathcal{B}) \cup \{v\} \\
			\end{align*}
			donc,
			\begin{align*}
				n - \#(\mathcal{B}'' \cap \mathcal{B}) &= n - (\#(\mathcal{B}' \cap \mathcal{B}) + 1) \\
				&= p+1- 1 \\
				&= p \\
			\end{align*}
			D'après l'hypothèse de récurrence, \[
				\#\mathcal{B}'' = n
			\] Or, $\#\mathcal{B}'' = \#\mathcal{B}'$
	\end{itemize}
\end{prv}

\begin{lem}
	Soient $\mathcal{B}$ et $\mathcal{B}'$ deux bases de $E$ telles que $\mathcal{B}\subset \mathcal{B}'$. Alors, $\mathcal{B} = \mathcal{B}'$.
\end{lem}

\begin{prv}
	On suppose $\mathcal{B}' \neq \mathcal{B}$. Soit $u \in \mathcal{B}' \setminus \mathcal{B}$
	$u \in E = \Vect(\mathcal{B})$ donc $\mathcal{B} \cup \{u\}$ n'est pas libre.
	Donc $\mathcal{B}\cup \{u\} \subset \mathcal{B}'$ et $\mathcal{B}'$ est libre donc $\mathcal{B}\cup \{u\}$ est libre: une contradiction $\lightning$
\end{prv}

\begin{lem}
	[Lemme d'échange] Soient $\mathcal{B}_1$ et $\mathcal{B}_2$ deux bases de $E$ et $u \in \mathcal{B}_1 \setminus \mathcal{B}_2$. Alors, il existe $v \in \mathcal{B}_2$ tel que $(\mathcal{B}_1 \setminus \{u\}) \cup \{v\}$ soit une base de $E$.
\end{lem}

\begin{prv}
	[1${}^\text{nde}$ méthode]
	On suppose que pout tout $v \in \mathcal{B}_2$, $(\mathcal{B}_1\setminus \{u\}) \cup \{v\}$ n'est pas une base de $E$
	Soit $v \in \mathcal{B}_2$.
	\begin{itemize}
		\item Supposons $(\mathcal{B}_1\setminus \{u\})\cup \{v\}$ non libre. $\mathcal{B}_1 \setminus \{u\}$ est libre. Donc $v \in \Vect(\mathcal{B}_1 \setminus \{u\})$
		\item Supposons $(\mathcal{B}_1\setminus \{u\}) \cup \{v\}$ non génératrice.
			Comme $\mathcal{B}_1$ engendre $E$, $u \not\in \Vect(\mathcal{B}_1\setminus \{v\})$.
			On suppose que $\mathcal{B}_1 \neq \mathcal{B}_2$.
			$\forall v \in \mathcal{B}_2 \setminus \mathcal{B}_1, \Vect(\mathcal{B}_1 \setminus \{v\}) = \Vect(\mathcal{B}_1) = E \ni u$ 
			donc, $(\mathcal{B}_1\setminus \{u\}) \cup \{v\}$ engendre $E$ et donc \[
				v \in \Vect(\mathcal{B}_1 \setminus \{u\})
			\] On a aussi \[
				\forall v \in \mathcal{B}_1 \setminus \{u\}, v \in \Vect(\mathcal{B}_1\setminus \{u\})
			\] Comme $u \not\in \mathcal{B}_2$, on a \[
				\forall v \in \mathcal{B}_2, v \in \Vect(\mathcal{B}_1\setminus \{u\})
			\] docn \[
				E = \Vect(\mathcal{B}_2) \subset \Vect(\mathcal{B}_1\setminus \{u\})
			\] donc $\mathcal{B}_1\setminus \{u\}$ engendre $E$ donc $\mathcal{B}_1\setminus \{u\}$ est une base de $E$. Or, $\mathcal{B}_1 \setminus \{u\}  \subset  \mathcal{B}_1$, donc $\mathcal{B}_1\setminus \{u\} = \mathcal{B}_1$
	\end{itemize}
\end{prv}

\begin{prv}
	[2${}^\text{nde}$ méthode]
	On suppose que pout tout $v \in \mathcal{B}_2$, $(\mathcal{B}_1\setminus \{u\}) \cup \{v\}$ n'est pas une base de $E$
	\begin{itemize}
		\item Comme $u \in \mathcal{B}_1 \setminus \mathcal{B}_2$, nécéssairement $\mathcal{B}_1 \neq \mathcal{B}_2$ donc $\mathcal{B}_2 \not\subset \mathcal{B}_1$, donc $\mathcal{B}_2\setminus\mathcal{B}_1 \neq \O$ 
		\item Soit $v \in \mathcal{B}_2\setminus\mathcal{B}_1$. Il existe $(\lambda_w)_{w\in\mathcal{B}_1}$ une famille de scalaires presque nulle telle que \[
				v = \sum_{w \in \mathcal{B}_1} \lambda_w w - \lambda_u u + + \sum_{w \in \mathcal{B}_1\setminus \{u\}}\lambda_w w
			\]
			Si $\lambda_u \neq 0_E$, alors
			\begin{align*}
				u &= \lambda_u^{-1}\left( v - \sum_{w \in \mathcal{B}_1 \setminus \{u\}} \lambda_w w \right)\\
					&\in \Vect(\mathcal{B}_1\setminus \{u\} \cup v)
			\end{align*}
			 donc $\mathcal{B}_1 \subset \Vect(\mathcal{B}_1\setminus \{u\} \cup \{v\})$\\
			 et donc $E \subset  \Vect(\mathcal{B}_1 \setminus \{u\} \cup \{v\})$ \\
			 et donc $\mathcal{B}_1 \setminus \{u\} \cup \{v\}$ engendre $E$ \\
			 donc $\mathcal{B}_1 \setminus \{u\} \cup \{v\}$ n'est pas libre\\
			 donc $v \in \Vect(\mathcal{B}_1\setminus \{u\})$ (car $\mathcal{B}_1 \setminus \{u\}$ est libre\\
			 donc $\lambda_u = 0_\mathbbm{K}$ $\lightning$\\`

			 Donc, $\lambda_u = 0_\mathbbm{K}$, docn $v \in \Vect(\mathcal{B}_1\setminus \{u\})$ \\
			 On vient de prouver que
			 \begin{align*}
			 	\mathcal{B}_2 \setminus \mathcal{B}_1 \subset \Vect(\mathcal{B}_1 \setminus \{u\})\\
			 	\mathcal{B}_1 \setminus \{u\} \subset \Vect(\mathcal{B}_1 \setminus \{u\})\\
			 \end{align*}
			 Comme $u \not\in \mathcal{B}_2$, \[
			 	\mathcal{B}_2 \subset \Vect(\mathcal{B}_1 \setminus \{u\})
			 \] donc \[
			 	E = \Vect(\mathcal{B}_2) \subset  \Vect(\mathcal{B}_1 \setminus \{u\})
			 \] donc $\mathcal{B}_1 \setminus \{u\}$ engendre $E$. Donc,  $\mathcal{B}_1 \setminus \{u\}$ est une base de $E$.\\
			 Or, $\mathcal{B}_1 \setminus \{u\} \subset  \mathcal{B}_1$, donc $\mathcal{B}_1 \setminus \{u\} = \mathcal{B}_1$
	\end{itemize}
\end{prv}

\begin{defn}
	Soit $E$ un $\mathbbm{K}$-espace vectoriel de dimension finie. Le cardinal commun à toutes les bases de $E$ est appelé \underline{dimension} de $E$ est notée $\dim(E)$ ou $\dim_\mathbbm{K}(E)$\\
	C'est donc aussi le nombre de coordonnées de n'importe quel vecteur dans n'importe quelle base.
	\index{dimension (espace vectoriel)}
\end{defn}

\begin{exm}
	\begin{enumerate}
		\item $\dim_\R(\C) = 2$ et $\dim_\C(\C) = 1$ 
		\item $\dim_\mathbbm{K}(\mathbbm{K}^{n}) = n$ 
		\item $\dim_{\mathbbm{K}}(\mathcal{M}_{n,p}(\mathbbm{K})) = np$
	\end{enumerate}
\end{exm}

\begin{crlr}
	Soit $E$ un $\mathbbm{K}$-espace vectoriel de dimension finie, $\mathcal{L}$ une famille libre de $E$, $\mathcal{G}$ une famille génératrice de $E$. On note $n = \dim(E)$
	\begin{enumerate}
		\item $\#\mathcal{G} \ge n$ et $(\#\mathcal{G} = n \implies \mathcal{G} \text{ est une base de } E$)
		\item $\#\mathcal{L} \le n$ et $(\#\mathcal{L} = n \implies \mathcal{L} \text{ est une base de } E$)
	\end{enumerate}
\end{crlr}

\begin{crlr}
	$\R^{\R}$ est de dimension infinie.
	$\forall i \in \N, e_i: x \mapsto x^i$\\
	$(e_i)_{i\in\N}$ est libre dans $\R^\R$
\end{crlr}

\begin{prop}
	Soient $E$ et $F$ deux $\mathbbm{K}$-espaces vectoriels de dimension finie. Alors $E\times F$ est de dimension finie et $\dim(E\times F) = \dim(E) + \dim(F)$
\end{prop}

\begin{prv}
	Soit $(e_1,\ldots, e_n)$ une base de $E$, $(f_1, \ldots, f_p)$ une base de $F$.
	On pose \[
		\left\{\begin{array}
			{r c l}
			u_1 &=& (e_1,0_F)\\
			u_2 &=& (e_2,0_F)\\
					&\vdots&\\
			u_n &=& (e_n,0_F)\\
			u_{n+1} &=& (0_E, f_1)\\
			u_{n+2} &=& (0_E, f_2)\\
					&\vdots&\\
			u_{n+p} &=& (0_E,f_p)\\
		\end{array}\right.
	\]
	Soit $(x,y) \in E\times F$. \[
		\begin{cases}
			\exists (x_1,\ldots,x_n)\in \mathbbm{K}^n, x = \sum_{i=1}^{n} x_ie_i
			\exists (y_1,\ldots,y_n)\in \mathbbm{K}^n, x = \sum_{j=1}^{p} y_jf_j
		\end{cases}
	\] 
	\begin{align*}
		(x,y) &= \left( \sum_{i=1}^{n} x_ie_i, \sum_{i=1}^{p} y_jf_j \right)  \\
		&= \sum_{i=1}^{n} x_i (e_i + 0_F) + \sum_{j=1}^{p} y_j (0_E, f_j) \\
		&= \sum_{i=1}^{n} x_i u_i + \sum_{j=1}^{p} y_j u_{n+j} \\
	\end{align*}
	Donc, $E\times F = \Vect(u_1, \ldots, u_{n+p})$ donc $E\times F$ est de dimension finie.\\
	Soit $(\lambda_1, \ldots, \lambda_{n+p}) \in \mathbbm{K}^{n+p}$ tel que \[
		(*): \quad \sum_{k=1}^{n+p} \lambda_ku_k = 0_{E\times F} = (0_E, 0_F)
	\]
	\begin{align*}
		(*) &\iff \sum_{k=1}^{n} \lambda_k (e_k, 0_F) + \sum_{k=n+1}^{p} \lambda_k(0_E, f_{k-n}) = (0_E, 0_F)\\
				&\iff \begin{cases}
					\sum_{k=1}^{n} \lambda_k e_k = 0_E\\
					\sum_{k=n+1}^{p} \lambda_k f_{k-n} = 0_F
				\end{cases}\\
				&\iff \begin{cases}
					\forall k \in \left\llbracket 1,n \right\rrbracket, \lambda_k = 0_\mathbbm{K} \qquad&(\text{car $(e_1,\ldots,e_n)$ est libre})\\
					\forall k \in \left\llbracket n+1,n+p \right\rrbracket, \lambda_k = 0_\mathbbm{K} \qquad&(\text{car $(f_1,\ldots,f_n)$ est libre})\\
				\end{cases}
	\end{align*}
	Donc $(u_1, \ldots, u_{n+p})$ est une base de $E\times F$. Donc, $\dim(E\times F) = n + p = \dim(E) + \dim(F)$
\end{prv}

\begin{rmk}
	[Convention]
	\[\dim\big(\{0_E\}\big) = 0\]
\end{rmk}

\begin{thm}
	Soit $E$ un $\mathbbm{K}$-espace vectoriel de dimension finie, $F$ un sous-espace vectoriel de $E$. Alors, $F$ est de dimension finie et  $\dim(F) \le \dim(E)$\\
	Si $\dim(F) = \dim(E)$, alors $F = E$
\end{thm}

\begin{prv}
	On considère \[
		A = \{k \in \N \mid \text{il existe une famille libre de $F$ à $k$ éléments}\} 
	\]
	On suppose $F \neq \{0_E\}$.
	\begin{itemize}
		\item Soit $u \in F\setminus \{0_E\}$. $(u)$ est libre donc $1 \in A$ et donc $A \neq \O$
		\item Soit $\mathcal{L}$ une famille libre de $F$. Alors, $\mathcal{L}$ est une famille libre de $E$ \\
			donc $\#\mathcal{L} \le \dim(E)$\\
			Donc $A$ est majorée par $\dim(E)$ \\
			On en déduit que $A$ a un plus grand élément $p$.
		\item Soit $\mathcal{L}$ une famille libre de $F$ avec $p$ éléments.\\
			Si $\mathcal{L}$ n'engendre pas $F$, alors il existe $u\in F$ tel que $u\not\in \Vect(\mathcal{L})$ et donc $\mathcal{L} \cup \{u\}$ est une famille libre de $F$, donc $p+1 \in A$ en contradiction avec la maximalité de $p$.\\
			Donc $\mathcal{L}$ est une base de $F$ donc $F$ est de dimension finie et $\dim(F) = p \le \dim(E)$\\
	\end{itemize}

	Soit $\mathcal{B}$ une base de $F$. Alors, $\mathcal{B}$ est aussi une famille de libre de de $E$. Donc $\#\mathcal{B} \le \dim(E)$ donc $\dim(F) = \dim(E)$ \\
	Si $\dim(F) = \dim(E)$, alors $\mathcal{B}$ est une base de $E$, et donc $F = \Vect(\mathcal{B}) = E$
\end{prv}

\begin{prop}
	[Formule de Grassmann]
	Soit $E$ un $\mathbbm{K}$-espace vectoriel de dimension finie, $F$ et $G$ deux sous-espace vectoriels de $E$. Alors, \[
		\dim(F+G) = \dim(F) + \dim(G) - \dim(F\cap G)
	\] 
\end{prop}

\begin{prv}
	Soit $(e_1, \ldots, e_p)$ une base de $F\cap G$. $(e_1,\ldots,e_p)$ est une famille libre de $F$.\\
	On complète $(e_1, \ldots, e_p)$ en une base $(e_1, \ldots, e_p, u_1, \ldots, u_q)$ de $F$.\\
	De même, on complète $(e_1, \ldots, e_p)$ en une base $(e_1, \ldots, e_p, v_1, \ldots, v_r)$ de $G$.\\
	On pose  $\mathcal{B} = (e_1, \ldots, e_p, u_1, \ldots, u_q, v_1, \ldots, v_r)$. Montrons que $\mathcal{B}$ est une base de $F+G$
	\begin{itemize}
		\item Soit $u \in F+G$ \\
			On pose $u = v+w$ avec $\begin{cases}
				v\in F\\
				w \in G
			\end{cases}$.\\
			On pose $v = \sum_{i=1}^p \lambda_i e_i + \sum_{i=1}^q \mu_i u_i$ avec $(\lambda_1, \ldots, \lambda_p, \mu_1, \ldots, \lambda_q) \in \mathbbm{K}^{p+q}$\\
			On pose aussi $w = \sum_{i = 1}^p \lambda'_ie_i + \sum_{j=1}^r \nu_j v_j$ avec $(\lambda_1',\ldots,\lambda_p', \nu_1, \ldots, \nu_r) \in \mathbbm{K}^{p+r}$\\
			D'où, \[
				u = \sum_{i=1}^p (\lambda_i + \lambda'_i)e_i + \sum_{j=1}^q \mu_j u_j + \sum_{k=1}^r \nu_k v_k \in \Vect(\mathcal{B})
			\]
		\item Soient $(\lambda_1, \ldots, \lambda_p, \mu_1, \ldots, \mu_q, \nu_1, \ldots, \nu_r) \in \mathbbm{K}^{p+q+r}$.\\
			On suppose \[
				(*)\quad \sum_{i=1}^{p}\lambda_ie_i + \sum_{j=1}^q\mu_ju_j + \sum_{k=1}^r \nu_k v_k = 0_E
			\] 
			D'où, \[
				\underbrace{\sum_{i=1}^p\lambda_i e_i + \sum_{j=1}^q \mu_ju_j}_{\in F} = \underbrace{-\sum_{k=1}^r\nu_jv_k}_{\in G}
			\] 
			Donc, \[
				f = \sum_{i=1}^p \lambda_i e_i + \sum_{j=1}^q \mu_j u_j \in F\cap G
			\] Comme $(e_1, \ldots, e_p)$ est une base de $F\cap G$, $\exists ! (\lambda_1', \ldots, \lambda_p') \in \mathbbm{K}^p$ tel que \[
				f = \sum_{i=1}^p \lambda'_i e_i = \sum_{i=1}^p \lambda'_i e_i + \sum_{j=1}^q 0_\mathbbm{K}u_j
			\] Comme $(e_1, \ldots, e_p, u_1, \ldots, u_q)$ est une base de $F$, \[
				\forall k \in \left\llbracket 1, q \right\rrbracket, \mu_j = 0_\mathbbm{K}
			\] De même, \[
				\forall k \in \left\llbracket 1,r \right\rrbracket , \nu_k = 0_\mathbbm{K}
			\] On remplace dans $(*)$ pour trouver \[
				\sum_{i=1}^p \lambda_ie_i = 0_E
			\] Comme $(e_1, \ldots, e_p)$ est libre, \[
				\forall i \in \left\llbracket 1,p \right\rrbracket, \lambda_i = 0_\mathbbm{K}
			\] Donc $\mathcal{B}$ est libre.\\
			Donc, 
			\begin{align*}
				\dim(F+G) &=  p +q + r \\
				&= (p+q)+ (p+r) - p \\
				&= \dim(F) + \dim(G) - \dim(F\cap G) \\
			\end{align*}
	\end{itemize}
\end{prv}

\begin{crlr}
	Avec les hypothèse précédentes, \[
		E = F \oplus G \iff \begin{cases}
			F \cap  G = \{0_E\} \\
			\dim(E) = \dim(F) + \dim(G)
		\end{cases}
	\] 
\end{crlr}

\begin{prv}
	\begin{itemize}
		\item[``$\implies$''] On suppose $E = F \oplus G$ \\
			Comme la somme est directe, $F \cap G = \{0_E\}$ 
			\begin{align*}
				\dim(E) &= \dim(F)\\
				&= \dim(F) + \dim(G) - \dim(F\cap G)\\
				&= \dim(F) + \dim(G)\\
			\end{align*}
		\item[``$\impliedby$''] On suppose $F\cap G = \{0_E\}$ et $\dim(E) = \dim(F) + \dim(G)$.\\
			On sait déjà que $F+G = F \oplus G$\\
			 \begin{align*}
				\dim(F+G) = \dim(F) + \dim(G) - \dim(F \cap G) = \dim(E)
			\end{align*}
			Donc $F + G = E$
	\end{itemize}
\end{prv}

\begin{prop}
	Soit $F$ un $\mathbbm{K}$-espace vectoriel de dimension finie $n$. Soit $\mathcal{B} = (e_1, \ldots, e_n)$ une base de $F$. L'application
	\begin{align*}
		f: \mathbbm{K}^n &\longrightarrow F \\
		(\lambda_1, \ldots, \lambda_n) &\longmapsto \sum_{i=1}^n \lambda_i e_i
	\end{align*} est bijective.\\
	Si $\mathbbm{K}$ est infini, $\mathbbm{K}^n$ aussi et donc $F$ aussi.\\
	Si $\#\mathbbm{K} = p \in \N_*$,
	\begin{align*}
		\#&\mathbbm{K}^n = p^n\\
		&\vrt=\\
		\#&F
	\end{align*}
\end{prop}


		\part{Dérivation}

\underline{Motivation}:

{
\begin{wrapfigure}{l}{3cm}
	\centering
	\begin{asy}
		import three;

		size(3cm);
		settings.render=0;
		settings.prc=false;
		currentprojection = obliqueZ;

		draw(unitbox);
		draw(shift(1.1Z + 0.05X) * (O -- X), Arrows3(TeXHead2));
		draw(shift(1.1Z + 0.05Y) * (O -- Y), Arrows3(TeXHead2));
		draw(shift(1.1X + 0.05Z) * (O -- Z), Arrows3(TeXHead2));

		label("$x$", (X/2) + (1.1Z + 0.05X), align=S);
		label("$y$", (Y/2) + (1.1Z + 0.05Y), align=W);
		label("$z$", (Z/2) + X, align=SE);
	\end{asy}
\end{wrapfigure}

\begin{align*}
	&S(x,y,z) = 2(xy + xz + yz)\\
	&V(x,y,z) = xyz
\end{align*}

On cherche à minimiser $S$ avec la contrainte $V = 1$.

Soit $f : \begin{array}{rcl}
	\left( \R_*^+ \right)^2 &\longrightarrow& \R \\
	(x,y) &\longmapsto& S\left( x,y,\frac{1}{xy} \right) = 2\left( xy + \frac{1}{y} + \frac{1}{x} \right).
\end{array}$

On cherche $(a,b) \in \left( \R^+_* \right)^2$ tel que \[
	\forall (x,y) \in (\R^+_*), f(x,y) \ge f(a,b).
\]
}

\begin{defn}
	Soit $f: U \to \R$ où $U$ est un ouvert de $\R^2$. Soit $(a,b) \in U$.
	\vspace{2mm}

	Si $\lim_{x \to a} \frac{f(x,b) - f(a,b)}{x - a} \in \R$, alors on dit que $f$ a une dérivée partielle suivant $x$ en $(a,b)$ et cette limite est notée \[
		\partial f_1(a,b) = \frac{\partial f}{\partial x}(a,b).
	\]

	Si $\lim_{y \to b} \frac{f(a,y) - f(a,b)}{y - b} \in \R$, alors on dit que $f$ a une dérivée partielle suivant $y$ et la limite est notée \[
		\partial f_2(a,b) = \frac{\partial f}{\partial y}(a,b).
	\]
\end{defn}

\begin{exm}
	\begin{enumerate}
		\item $f: (x,y) \mapsto xy + x - y$.

			\begin{align*}
				&\frac{\partial f}{\partial x} : (x,y) \mapsto y + 1,\\
				&\frac{\partial f}{\partial y} : (x,y) \mapsto x - 1.
			\end{align*}

		\item $f: (x,y) \mapsto xy + \frac{1}{y}+ \frac{1}{x}$.

			\begin{align*}
				&\frac{\partial f}{\partial x}: (x,y) \mapsto y - \frac{1}{x^2},\\
				&\frac{\partial f}{\partial y}: (x,y) \mapsto x - \frac{1}{y^2}.
			\end{align*}

		\item Trouver $f$ telle que $\begin{cases}
				(1): \qquad \frac{\partial f}{\partial x}=y,\\[2mm]
				(2): \qquad \frac{\partial f}{\partial y} = x.
			\end{cases}$

			D'après $(1)$ : \[
				\forall (x,y), \exists C(y) \in \R, f(x,y) = xy + C(y)
			\] et donc \[
				\frac{\partial f}{\partial y}(x,y) = x + C'(y)
			\] donc $C'(y) = 0$ et donc $C$ est constante.

		\item Trouver $f$ telle que $\begin{cases}
			\frac{\partial f}{\partial x} = -y,\\[2mm]
			\frac{\partial f}{ƒ\partial y} = x.
		\end{cases}$

		Ce n'est pas possible !
	\end{enumerate}
\end{exm}

\begin{defn}~\\
	\begin{minipage}{\linewidth}
		\begin{wrapfigure}{r}{4cm}
			\centering
			\vspace{-5mm}
			\begin{asy}
				import three;
				import graph3;
				size(4cm);

				settings.render = 0;
				settings.prc = false;
				currentprojection = obliqueX;

				draw(O -- X, Arrow3(TeXHead2));
				draw(O -- Y, Arrow3(TeXHead2));
				draw(O -- Z, Arrow3(TeXHead2));

				triple f(real x, real y, real z = 0) { return (x,y,cos(x - 0.5) * cos(y - 0.5)/1.2 + 0.15); }

				real inc = 1 / 5;

				for(real x = 0; x <= 1; x += inc) {
					draw(graph(
						new real(real t) { return x; }, // x
						new real(real y) { return y; }, // y
						new real(real y) { return f(x,y).z; }, // z
						0, 1
					), gray);
				}

				for(real y = 0; y <= 1; y += inc) {
					draw(graph(
						new real(real x) { return x; }, // x
						new real(real t) { return y; }, // y
						new real(real x) { return f(x,y).z; }, // z
						0, 1
					), gray);
				}

				path3 path1 = (0.8, 0.2, 0) .. (0.5, 0.5, 0) .. (0.3, 0.7, 0);
				path3 path2 = f(0.8, 0.2, 0) .. f(0.5, 0.5, 0) .. f(0.3, 0.7, 0);
				path3 d = (0.2, 0.3, 0) .. (0.3, 0.4, 0) .. (0.2, 0.7, 0) .. (0.8, 0.9, 0) .. (0.6, 0.2, 0) .. cycle;

				draw(path1, red, Arrow3(TeXHead2));
				draw(path2, red, Arrow3(TeXHead2, position=0.8));

				dot((0.5, 0.5, 0));
				dot(f(0.5, 0.5, 0));
				draw((0.5, 0.5, 0) -- f(0.5, 0.5, 0), dashed);
				draw(d);

				label("$w$", (0.3, 0.7, 0), red, align=SE);
				label("$U$", (0.8, 0.9, 0), align=SE);
			\end{asy}
		\end{wrapfigure}

		Soit $f: U \to \R$ où $U$ est un ouvert. Soit $(a,b) \in U$. Soit $w = (w_1, w_2) \in \R^2$.

		Si 
		\[
			\lim_{t\to 0} \frac{f(a + tw_1, b + tw_2) - f(a,b)}{t}
		\] existe et est réelle, alors on dit que $f$ a une dérivée dans la direction de $w$ et la limite est notée \[
			\mathrm{d}f(w)\,(a,b) = D_w(f)\,(a,b).
		\]
	\end{minipage}
\end{defn}

\begin{exm}
	\begin{align*}
		f: \left( \R_*^+ \right)^2 &\longrightarrow \R \\
		(x,y) &\longmapsto xy+\frac{1}{x}+\frac{1}{y}.
	\end{align*}

	On pose $(a,b) = (1,2)$, $w = (w_1, w_2) = (1,1)$.
	\begin{align*}
		\frac{f(1+t, 2+t) - f(1,2)}{t} &= \frac{1}{t} \left( (1+t)(2+t) + \frac{1}{1+t} + \frac{1}{2+t} - 3 - \frac{1}{2} \right) \\
		&= \frac{1}{t}\left(\cancel 2 + 3t + \po(t) + \cancel 1 - t + \po(t) + \frac{1}{2}\left( \cancel 1 - \frac{t}{2} + \po(t) \right) - \cancel3 - \cancel{\frac{1}{2}} \right) \\
		&= \frac{1}{t} \left( \frac{7}{4} t + \po(t) \right)  \\
		&= \frac{7}{4} + \po(1) \tendsto{t \to 0} \frac{7}{4}. \\
	\end{align*}

	Donc, \[
		\mathrm{d}f(1,1)\,(1,2) = \frac{7}{4}.
	\]
\end{exm}

\begin{rmk}~\\
	\begin{figure}[H]
		\centering
		\begin{asy}
			import solids;
			import graph;
			size(5cm);

			settings.render = 0;
			settings.prc = false;

			path3 par = graph(
				new real(real x) { return x; },
				new real(real x) { return 0; },
				new real(real x) { return x^2; },
				0,3);
			revolution r = revolution(par, axis=Z);

			path3 par2 = graph(
				new real(real x) { return x; },
				new real(real x) { return 0; },
				new real(real x) { return x^2; },
				-3,3);

			draw(r,1,longitudinalpen=nullpen);
			draw(r.silhouette());

			draw((-4, 0, -1) -- (-4, 0, 10) -- (4, 0, 10) -- (4, 0, -1) -- cycle, red);
			draw(par2, deepred);

			draw((4,4.5) -- (7, 4.5), black+0.5mm, Arrow(TeXHead));

			path par2d = graph(new real(real x) { return x^2; }, -3, 3);
			draw(shift((11, 0)) * par2d, deepred);

			dot(O);
			dot((11, 0));
		\end{asy}
	\end{figure}
\end{rmk}


%todo ajouter théorème-définition
\begin{thm}
	Soit $f : U \to \R$, $(a,b) \in U$. On suppose que $\frac{\partial f}{\partial x}$ et $\frac{\partial f}{\partial y}$ existent en $(a,b)$ et sont {\bfseries continues} en $(a,b)$. Alors,
	\begin{align*}
		&\forall (h, k) \in \R^2 \text{ tel que } (a +h, b + k) \in U,\\
		&f(a+ h, b + k) = f(a,b) + h \frac{\partial f}{\partial x}(a,b) + k \frac{\partial f}{\partial y}(a,b) + \po_{(h,k)\to (0,0)}\big(\|(h,k)\|\big).
	\end{align*}

	On dit que $f$ est de classe $\mathcal{C}^1$ si $\frac{\partial f}{\partial x}$ et $\frac{\partial f}{\partial y}$ existent et sont continues.

	\qed
\end{thm}

\begin{rmk}
	En physique, cette formule correspond à : \[
		\mathrm{d}f = \frac{\partial f}{\partial x}\mathrm{d}x + \frac{\partial f}{\partial y} \mathrm{d}y.
	\] En effet :
	\begin{align*}
		\mathrm{d}f &= f(x+ \mathrm{d}x, y + \mathrm{d}y) - f(x,y) \\
		&= \frac{\partial f}{\partial x} \mathrm{d}x + \frac{\partial f}{\partial y} \mathrm{d}y.
	\end{align*}
\end{rmk}

\begin{prop}
	Soit $f: U \to \R$ de classe $\mathcal{C}^1$ en $(a,b) \in U$. Alors, \[
		\forall w = (w_1, w_2) \in \R^2, \mathrm{d}f(w)\,(a,b) = w_1 \frac{\partial f}{\partial x}(a,b) + w_2 \frac{\partial f}{\partial y}(a,b).
	\]
\end{prop}

\begin{prv}
	Soit $w = (w_1, w_2) \in \R^2$. Soit $t \in \R^*$.
	\begin{align*}
		\frac{1}{t}\big(f(a + tw_1, b + tw_2) - f(a,b)\big)
		&= \frac{1}{t} \left( tw_1 \frac{\partial f}{\partial x}(a,b) + tw_2 \frac{\partial f}{\partial y}(a,b) + \po_{t \to 0}\big(\|tw\|\big) \right) \\
		&= w_1 \frac{\partial f}{\partial x}(a,b) + w_2 \frac{\partial f}{\partial y}(a,b) + \po_{t\to 0}(1) \\
		&\tendsto{t\to 0} w_1 \frac{\partial f}{\partial x}(a,b) + w_2\frac{\partial f}{\partial y}(a,b).
	\end{align*}
\end{prv}


\begin{defn}
	Avec les hypothèses précédentes, en posant \[
		\nabla f(a,b) = \left( \frac{\partial f}{\partial x}(a,b), \frac{\partial f}{\partial y}(a,b) \right) 
	\]on obtient \[
		\mathrm{d}f(w)\,(a,b) = \left<w  \mid \nabla f(a,b) \right>
	\] où $\left<\cdot|\cdot \right>$ est le produit scalaire.

	Le vecteur $\nabla f(a,b)$ est appelé \underline{gradient de $f$ en $(a,b)$}.

	Le développement limité à l'ordre 1 de $f$ devient \[
		f\big((a,b)+w\big) = f(a,b) + \left<w \mid \nabla f(a,b) \right> + \po_{w\to 0}(\|w\|)
	\]
\end{defn}

\begin{prop}
	Soit $f : U \to \R$ de classe $\mathcal{C}^1$.

	\begin{figure}[H]
    \centering
    \incfig{gradient}
	\end{figure}

	$\nabla f$ est orthogonal au lignes de niveaux de $f$, son orientation va dans le sens d'une augmentation de $f$.
\end{prop}

\begin{prv}
	Soit $\gamma : I \to U$ une courbe de niveau : \[
		\forall t \in I, f\big(\gamma(t)\big) = \text{cste}.
	\] D'après le lemme suivant : \[
		\forall t \in I, 0 = (f \circ \gamma)'(t) = \mathrm{d}f\big(\gamma'(t)\big)\big(\gamma(t)\big) = \left<\gamma'(t)  \mid \nabla f\big(\gamma(t)\big) \right>
	\] Donc $\nabla f\big(\gamma(t)\big)$ est orthogonal à $\gamma'(t)$.

	Pour tout $t \in I$, on pose $w(t) = t\, \nabla f\big(\gamma(t)\big)$. Donc \[
		f\big(\gamma(t) + w(t)\big) = f\big(\gamma(t)\big) + t \|\nabla f(\gamma(t))\|^2 + \po_{t \to 0}(t)
	\] Pour $t$ assez petit, $f\big(\gamma(t) + w(t)\big) - f\big(\gamma(t)\big)$ est du même signe que $t$.
\end{prv}

\begin{rmk}
	\begin{align*}
		V: \R^3 &\longrightarrow \R \\
		(x,y,z) &\longmapsto -mgz
	\end{align*}
	l'énerge potentielle de pesenteur

	On a donc \[
		\nabla V(x,y,z) = \left( \frac{\partial V}{\partial x}, \frac{\partial V}{\partial y}, \frac{\partial V}{\partial z} \right) = (0, 0, -mg) = \vec{P}.
	\]
\end{rmk}

\begin{lem}
	Soit $f : U \to \R$ de classe $\mathcal{C}^1$, $\gamma : \begin{array}{rcl}
		I &\longrightarrow& U \\
		t &\longmapsto& \big(x(t), y(t)\big)
	\end{array}$ où $x$ et $y$ sont dérivables.

	On pose \[
		\forall t \in I, \gamma'(t) = \big(x'(t), y'(t)\big).
	\] Alors $f \circ \gamma : I \to \R$ est dérivable et
	\begin{align*}
		\forall t \in I, (f \circ \gamma)'(t) &= \mathrm{d}f\big(\gamma'(t)\big) \big(\gamma(t)\big)\\
		&= \left<\gamma'(t)  \mid \nabla f\big(\gamma(t)\big)  \right> \\
		&= x'(t) \frac{\partial f}{\partial x}\big(x(t), y(t)\big) + y'(t) \frac{\partial f}{\partial y}\big(x(t),y(t)\big). \\
	\end{align*}
\end{lem}

\begin{prv}
	On fixe $t \in I$.

	\begin{align*}
		\forall h \neq 0, \frac{f \circ \gamma(t + h) - f \circ \gamma(t)}{h}
		&= \frac{1}{h}\big(f(\gamma(t)) + h\gamma'(t) + \po_{h\to 0}(h) - f(\gamma(t))\big) \\
		&= \frac{1}{h}\bigg(\cancel{f(\gamma(t))} + \left<h\gamma'(t) \mid \nabla f(\gamma(t)) \right> + \po_{h\to 0}(\|h\gamma'(t)\|) - \cancel{f(\gamma(t))}\bigg)\\
		&= \left<\gamma'(t) \mid \nabla f(\gamma(t)) \right> + \po_{h\to 0}(1) \\
		&\tendsto{h\to 0} \left<\gamma'(t)  \mid \nabla f(\gamma(t)) \right>
	\end{align*}
\end{prv}

\begin{defn}
	Soit $f : U \to \R$ de classe $\mathcal{C}^1$ et $(a,b) \in U$. On dit que $(a,b)$ est un \underline{point critique} de $f$ si $\nabla f(a,b) = 0$ i.e. $\frac{\partial f}{\partial x}(a,b) = \frac{\partial f}{\partial y}(a,b) = 0$.

	Dans ce cas, $f(a,b)$ est appelé \underline{valeur critique} de $f$.
\end{defn}

\begin{prop}~\\
	\begin{minipage}{\linewidth}
		\begin{wrapfigure}{r}{3cm}
			\centering
			\vspace{-1cm}
			\begin{asy}
				import solids;
				import graph;
				size(3cm);

				settings.render = 0;
				settings.prc = false;

				path3 par = graph(
					new real(real x) { return x; },
					new real(real x) { return 0; },
					new real(real x) { return -x^2; },
					0,3);
				revolution r = revolution(par, axis=Z);

				draw(r,1,longitudinalpen=nullpen);
				draw(r.silhouette());

				dot("$(a,b)$", O, red, align=N);
				real s = sqrt(2.5);
				path3 g=(s,0,-2.5)..(0,s,-2.5)..(-s,0,-2.5)..(0,-s,-2.5)..cycle;
				draw(g, deepcyan);
			\end{asy}
		\end{wrapfigure}
		Soit $f: U \to \R$ de classe $\mathcal{C}^1$ et $(a,b) \in U$ tel que \[
			\exists r > 0, \forall (x,y) \in B_{(a,b)}(r), f(x,y) \le f(a,b)
		\] Alors $\nabla f(a,b) = (0,0)$.
	\end{minipage}
\end{prop}

\begin{prv}
	Soit $g: x \mapsto f(x,b)$. $g(a)$ est un maximum local de $g$ donc $g'(a) = 0$.

	Or, $g'(a) = \frac{\partial f}{\partial x}(a,b)$

	donc $\frac{\partial f}{\partial x}(a,b) = 0$.

	Soit $h : y \mapsto f(a,y)$. On a de même $h'(b) = 0$.

	Or, $h'(b) = \frac{\partial f}{\partial y}(a,b)$.

	Donc, $\nabla f(a,b) = (0,0)$.
\end{prv}

\begin{rmk}
	Un minimum local est aussi une valeur critique.
\end{rmk}

\begin{figure}[H]
	\centering
	\begin{subfigure}{3cm}
		\centering
		\begin{asy}
			import solids;
			import graph;
			size(3cm);

			settings.render = 0;
			settings.prc = false;

			path3 par = graph(
				new real(real x) { return x; },
				new real(real x) { return 0; },
				new real(real x) { return -x^2; },
				0,3);
			revolution r = revolution(par, axis=Z);

			draw(r,1,longitudinalpen=nullpen);
			draw(r.silhouette());

			dot(O, red);
		\end{asy}
		\caption{Maximum local}
	\end{subfigure}
	\begin{subfigure}{3cm}
		\centering
		\begin{asy}
			import solids;
			import graph;
			size(3cm);

			settings.render = 0;
			settings.prc = false;

			path3 par = graph(
				new real(real x) { return x; },
				new real(real x) { return 0; },
				new real(real x) { return x^2; },
				0,3);
			revolution r = revolution(par, axis=Z);

			draw(r,1,longitudinalpen=nullpen);
			draw(r.silhouette());

			dot(O, red);
		\end{asy}
		\caption{Minimum local}
	\end{subfigure}
	\begin{subfigure}{3cm}
		\centering
		\begin{asy}
			import solids;
			import graph;
			size(3cm);

			settings.render = 0;
			settings.prc = false;
			currentprojection = obliqueZ;

			draw(graph(
				new real(real x) { return x; },
				new real(real x) { return -x^2 / 3; },
				new real(real x) { return 3; },
				-3, 3
			));

			draw(graph(
				new real(real x) { return x; },
				new real(real x) { return -x^2 / 3; },
				new real(real x) { return -3; },
				-3, 3
			));

			draw(graph(
				new real(real x) { return x; },
				new real(real x) { return -x^2 / 3 - 1; },
				new real(real x) { return 0; },
				-3, 3
			));

			draw(graph(
				new real(real x) { return 0; },
				new real(real x) { return x^2 / 9 - 1; },
				new real(real x) { return x; },
				-3, 3
			));

			draw(graph(
				new real(real x) { return -3; },
				new real(real x) { return x^2 / 9 - 4; },
				new real(real x) { return x; },
				-3, 3
			));

			draw(graph(
				new real(real x) { return 3; },
				new real(real x) { return x^2 / 9 - 4; },
				new real(real x) { return x; },
				-3, 3
			));

			dot((0,-1,0), red);
		\end{asy}
		\caption{Point de selle / Point col}
	\end{subfigure}
\end{figure}

\begin{exm}
	On revient à l'exemple donné en introduction : 
	\begin{align*}
		f: \left( \R^*_+ \right)^2 &\longrightarrow \R \\
		(x,y) &\longmapsto 2\left( xy + \frac{1}{x} + \frac{1}{y} \right).
	\end{align*}

	$\left( \R^+_* \right)^2$ est un ouvert de $\R^2$. Soit $(x,y) \in \left( \R^+_* \right)^2$.
	
	On a \[
		\begin{cases}
			\frac{\partial f}{\partial x}(x,y) = 2\left( y - \frac{1}{x^2} \right),\\
			\frac{\partial f}{\partial y}(x,y) = 2\left( x - \frac{1}{y^2} \right).
		\end{cases}
	\]

	\begin{align*}
		&\frac{\partial f}{\partial x}(x,y) = \frac{\partial f}{\partial y}(x,y) = 0\\
		\iff& \begin{cases}
			y = \frac{1}{x^2}\\
			x = \frac{1}{y^2}
		\end{cases}\\
		\iff& \begin{cases}
			y = \frac{1}{x^2}\\
			x = x^4
		\end{cases}\\
		\iff& \begin{cases}
			x = 1\\
			y = 1
		\end{cases}
	\end{align*}

	On vérivie que $f$ présente en effet un minium local en $(1,1)$. \[
		f(1,1) = 6
	\] On fixe $y \in \R^+_*$ et \[
		g : x \mapsto 2\left( xy + \frac{1}{x} + \frac{1}{y} \right).
	\] Donc \[
		\forall x \in \R^+_*, g'(x) = 2\left( y - \frac{1}{x^2} \right).
	\]
	\begin{center}
		\begin{tikzpicture}
			\tkzTabInit{$x$/1,$g'(x)$/1,$g$/2.3}{$0$, $\frac{1}{\sqrt{y}}$, $+\infty$}
			\tkzTabLine{,-,z,+,}
			\tkzTabVar{+/{}, -/$2\left( 2\sqrt{y} +\frac{1}{y} \right)$, +/{}}
		\end{tikzpicture}
	\end{center}
	
	Ainsi, \[
		\forall x \in \R^+_*, \forall y \in \R^+_*, f(x,y) \ge 2\left( 2\sqrt{y} + \frac{1}{y} \right)
	\] Soit $h : y \mapsto 2\sqrt{y} + \frac{1}{y}$. On a \[
		\forall y > 0, h'(y) = \frac{1}{\sqrt{y}} - \frac{1}{y^2} = \frac{y\sqrt{y} - 1}{y^2} = \frac{y^{\frac{3}{2}} - 1}{y^2}
	\]

	\begin{center}
		\begin{tikzpicture}
			\tkzTabInit{$y$/0.7,$h'(y)$/0.7,$h$/1.4}{$0$, $1$, $+\infty$}
			\tkzTabLine{,-,z,+,}
			\tkzTabVar{+/{}, -/$3$, +/{}}
		\end{tikzpicture}
	\end{center}

	Donc, \[
		\forall x,y > 0, f(x,y) \ge 2\times 3 = 6 = f(1,1).
	\]
\end{exm}

\begin{prop}
	[règle de la chaîne]

	Soit $f : \begin{array}{rcl}
		U &\longrightarrow& \R^2 \\
		(x,y) &\longmapsto& f(x,y)
	\end{array}$ de classe $\mathcal{C}^1$ et $U, V$ deux ouverts de $\R^2$.

	Soit $\varphi : \begin{array}{rcl}
		V &\longrightarrow& U \\
		(u,v) &\longmapsto& \varphi(u,v) = \big(x(u,v), y(u,v)\big)
	\end{array}$.

	On suppose que $x$ et $y$ sont de classe $\mathcal{C}^1$ sur $V$.

	Alors,  $f \circ \varphi : \begin{array}{rcl}
		V &\longrightarrow& \R \\
		(u,v) &\longmapsto& f\big(\varphi(u,v)\big)
	\end{array}$ est de classe $\mathcal{C}^1$ et
	\begin{align*}
		\forall (u_0, v_0) \in V, \frac{\partial (f \circ \varphi)}{\partial u}(u_0, v_0)
		&= \frac{\partial f}{\partial x}\big(\varphi(u_0, v_0)\big) \times \frac{\partial x}{\partial u}(u_0, v_0)\\
		&+ \frac{\partial f}{\partial y}\big(\varphi(u_0,v_0)\big) \frac{\partial y}{\partial u}(u_0,v_0)
	\end{align*}
	\begin{align*}
		\forall (u_0, v_0) \in V, \frac{\partial (f \circ \varphi)}{\partial v}(u_0, v_0)
		&= \frac{\partial f}{\partial x}\big(\varphi(u_0, v_0)\big) \times \frac{\partial x}{\partial v}(u_0, v_0)\\
		&+ \frac{\partial f}{\partial y}\big(\varphi(u_0,v_0)\big) \frac{\partial y}{\partial v}(u_0,v_0)
	\end{align*}
\end{prop}

\begin{exm}
	[changement de coordonnées polaires]
	On pose \begin{align*}
		\varphi: \R^+_* \times ]0,2\pi[ &\longrightarrow \R^2\setminus \left( R^+_* \times \{0\} \right) \\
		(r, \theta) &\longmapsto (r \cos \theta, r \sin\theta),
	\end{align*}
	\begin{align*}
		f: \R^2\setminus \left( R^+_* \times \{0\} \right) &\longrightarrow \R \\
		(x,y) &\longmapsto f(x,y),
	\end{align*}
	\begin{align*}
		g: \overbrace{\R^+_* \times ]0, 2\pi[}^{=V} &\longrightarrow \R \\
		(r, \theta) &\longmapsto f(r\cos\theta, r\sin\theta).
	\end{align*}

	\begin{align*}
		\forall (r_0,\theta_0) \in V,&\\[5mm]
		\frac{\partial g}{\partial r}(r_0, \theta_0) &= \frac{\partial f}{\partial x}(r_0\cos\theta_0, r_0\sin\theta_0)\cos\theta_0\\
		&+ \frac{\partial f}{\partial y}(r_0 \cos\theta_0, r_0\sin\theta_0)\sin\theta_0\\
		&= 2r_0\cos^2\theta_0 + 2r_0\sin^2(\theta_0) \\
		&= 2r_0 \\[5mm]
		\frac{\partial g}{\partial \theta}(r_0, \theta_0) &= \frac{\partial f}{\partial x}(r_0\cos\theta_0, r_0\sin\theta_0)r_0\sin\theta_0\\
		&+ \frac{\partial f}{\partial y}(r_0 \cos\theta_0, r_0\sin\theta_0)r_0\cos\theta_0\\
		&= -2{r_0}^2\cos(\theta_0)\sin(\theta_0) + 2{r_0}^2 \sin(\theta_0)\cos(\theta_0)\\
		&= 0 \\
	\end{align*}

	Donc, \[
		g(r, \theta) = r^2.
	\]
\end{exm}

\begin{exm}
	Résoudre \[
		\begin{cases}
			\frac{\partial f}{\partial x} = \frac{x}{x^2+y^2},\\
			\frac{\partial f}{\partial y} = \frac{y}{x^2+y^2}.\\
		\end{cases}
	\]

	On pose $g: (r, \theta) \mapsto f(r \cos\theta, r \sin\theta)$.

	\begin{align*}
		&\frac{\partial g}{\partial r} = \frac{1}{r}\cos^2\theta + \frac{1}{r}\sin^2\theta = \frac{1}{r},\\
		&\frac{\partial g}{\partial \theta} = -\cos(\theta) \sin(\theta) + \sin(\theta)\cos(\theta) = 0.
	\end{align*}

	Donc, \[
		\exists C \in \R, g: (r, \theta) \mapsto \ln r + C
	\] d'où,
	\begin{align*}
		\forall (x,y) \in \R^2 \setminus \{(0,0)\}, f(x,y) &= \ln\left(\sqrt{x^2 + y^2} \right)  + C\\
		&= \frac{1}{2}\ln(x^2 + y^2) + C. \\
	\end{align*}
\end{exm}

\begin{rmk}
	Soit $\mathcal{B} = (e_1, e_2)$ la base canonique de $\R^2$, $f: U \to \R$ de classe $\mathcal{C}^1$ avec $U$ un ouvert de $\R^2$.

	Soit $(x,y) \in U$.

	\begin{align*}
		\Mat_{\mathcal{B}}\big(\nabla f(x,y)\big) = \begin{pmatrix}
			\frac{\partial f}{\partial x}(x,y)\\[2mm]
			\frac{\partial f}{\partial y}(x,y)
		\end{pmatrix}
	\end{align*}

	Soit  \begin{align*}
		\varphi: V &\longrightarrow U \\
		(u,v) &\longmapsto \big(x(u,v), y(u,v)\big) 
	\end{align*} avec $x,y$ de classe $\mathcal{C}^1$. Soit $g = f \circ \varphi$.
	\begin{align*}
		\Mat_{\mathcal{B}}\big(\nabla g(u,v)\big)
		&= \begin{pmatrix}
			\frac{\partial g}{\partial u}(u,v) \\[2mm]
			\frac{\partial g}{\partial v}(u,v)
		\end{pmatrix} \\
		&= \begin{pmatrix}
			\frac{\partial x}{\partial u}(u,v) \frac{\partial f}{\partial x}(x,y)
			+ \frac{\partial y}{\partial u}(u,v)\frac{\partial f}{\partial y}(x,y)\\[3mm]
			\frac{\partial x}{\partial v}(u,v) \frac{\partial f}{\partial x}(x,y)
			+ \frac{\partial y}{\partial v}(u,v) \frac{\partial f}{\partial y}(x,y)
		\end{pmatrix}  \\
		&= \underbrace{\begin{pmatrix}
				\frac{\partial x}{\partial u}(u,v)& \frac{\partial y}{\partial u}(u,v)\\[3mm]
				\frac{\partial x}{\partial v}(u,v)& \frac{\partial y}{\partial v}(u,v)
		\end{pmatrix}}_{J(u,v)} \begin{pmatrix}
			\frac{\partial f}{\partial x}(x,y)\\[3mm]
			\frac{\partial f}{\partial y}(x,y)
		\end{pmatrix} \\
		&= J(u,v) \Mat_{\mathcal{B}}\big(\nabla f(x,y)\big) \\
	\end{align*}
	où $J(u,v) = 
	\begin{pNiceArray}{c:c}
		\Mat_{\mathcal{B}}\big(\nabla x(u,v)\big) & \Mat_{\mathcal{B}}\big(\nabla y(u,v)\big)
	\end{pNiceArray}$.

	On dit que $J(u,v)$ est \underline{la jacobienne} de $\varphi$ en $(u,v)$.
	L'application linéaire canoniquement associée à $J(u,v)$ est la \underline{différentielle de $\varphi$} en $(u,v)$ noté $\mathrm{d}\varphi(u,v)$.

	On a $\mathrm{d}\varphi(u,v) \in \mathcal{L}(R^2)$ et $\Mat_{\mathcal{B}}\big(\mathrm{d}\varphi(u,v)\big) = J(u,v)$.

	Par exemple, la jacobienne du changement de coordonnées polaires est \[
		J = \begin{pmatrix}
			\frac{\partial x}{\partial r} & \frac{\partial y}{\partial r}\\[3mm]
			\frac{\partial x}{\partial \theta} & \frac{\partial y}{\partial \theta}
		\end{pmatrix}
		= \begin{pmatrix}
			\cos\theta&\sin\theta\\
			-r\sin\theta&r\cos\theta
		\end{pmatrix}.
	\]
	$\underbrace{\det(J)}_{\text{le jacobien}} = r\cos^2\theta + r\sin^2\theta = r$

	Dans une intégrale double, si $(x,y) = \varphi(u,v)$, alors $\mathrm{d}x\mathrm{d}y = \det(J)\mathrm{d}u\mathrm{d}v$.

	Ici, \[
		\mathrm{d}x\ \mathrm{d}y = r\ \mathrm{d}r\ \mathrm{d}\theta.
	\]
\end{rmk}

\begin{prv}
	On pose $(x_0, y_0) = \varphi(u_0, v_0)$. Pour tout $(h,k) \in \R^2$ tels que $(u_0 + h, v_0 + k) \in V$, en posant $g = f  \circ \varphi$.

	\begin{align*}
		g(u_0 + h, v_0 + h) &= f\big(x(u_0 + h, v_0 + k), y(u_0 + h, v_0 + k)\big) \\
		&= f\left(
			x(u_0,v_0) + h \frac{\partial x}{\partial u}(u_0,v_0) + k \frac{\partial x}{\partial v}(u_0, v_0) + \po\big(\|(h,k)\|\big), \right.\\
		&\phantom{ = f\bigg(\bigg.}\left. y(u_0, v_0) + h \frac{\partial y}{\partial u}(u_0, v_0) + k \frac{\partial y}{\partial v}(u_0, v_0) + \po\big(\|(h,k)\|\big)
		\right)  \\
		&= f(x_0,y_0) \\
		&~+ \left( h \frac{\partial x}{\partial u}(u_0,v_0) + k \frac{\partial x}{\partial v}(u_0, v_0) + \po(\|(h,k)\|) \right) \frac{\partial f}{\partial x}(x_0,y_0)\\
		&~+ \left( h \frac{\partial y}{\partial u}(u_0, v_0) + k\frac{\partial y}{\partial v}(u_0, v_0) + \po(\|(h,k)\|) \right) \frac{\partial f}{\partial y}(x_0, y_0)\\
		&~+ \po(\|(h,k)\|)\\
		&= f(x_0, y_0) \\
		&~+ h \left( \frac{\partial x}{\partial u}(u_0, v_0) \frac{\partial f}{\partial x}(x_0, y_0) + \frac{\partial y}{\partial u}(u_0, v_0) \frac{\partial f}{\partial y}(x_0, y_0) \right)  \\
		&~+ k\left( \frac{\partial x}{\partial v}(u_0, v_0) \frac{\partial f}{\partial x}(x_0, y_0) + \frac{\partial y}{\partial v}(u_0, v_0) \frac{\partial f}{\partial y}(x_0, y_0) \right) 
		&~+ \po(\|(h,k)\|)\\
		&= g(u_0, v_0) + h \frac{\partial g}{\partial u}(u_0, v_0) + k \frac{\partial g}{\partial v}(u_0, v_0) + \po(\|(h,k)\|) \\
	\end{align*}

	Par identification,
	\[
		\frac{\partial g}{\partial u}(u_0, v_0) = \frac{\partial x}{\partial u}(u_0, v_0) \frac{\partial f}{\partial x}(x_0, y_0) + \frac{\partial y}{\partial u}(u_0, v_0) \frac{\partial f}{\partial y}(x_0,y_0)
	\] et \[
		\frac{\partial g}{\partial v}(u_0, v_0) = \frac{\partial x}{\partial v}(u_0,v_0) \frac{\partial f}{\partial x}(x_0, y_0) + \frac{\partial y}{\partial v}(u_0, v_0) \frac{\partial f}{\partial y}(x_0, y_0).
	\] 
\end{prv}

\begin{exm}
	[Régression linéaire]~\\
	\begin{figure}[H]
		\centering
		\begin{asy}
			import graph;
			axes(EndArrow);
			size(5cm);

			real f(real x) { return x + 0.5; }

			real k = 35 / (7 - 0.5);

			for(int i = 0; i < 35; ++i) {
				real mag = exp(sin(100 * pi/exp(1) * i)) * 0.8 + exp(cos(i*40)/3);
				real eps = mag * cos(10 * exp(1)/pi * i) / 3;
				dot((i/k,f(i/k) + eps));
			}

			draw(graph(f, -1, 7), orange);
		\end{asy}
	\end{figure}
	\[
		y = a x + b
	\] 
	On fixe $(a,b) \in \R^2$. \[
		\varepsilon(a,b) = \sum_{i=1}^n\big( y_i - (ax_i + b) \big)^2
	\] l'erreur totale.

	On veut minimiser $\varepsilon(a,b)$. On a 
	\[
		\forall (a,b) \in \R^2,
		\begin{cases}
			\frac{\partial \varepsilon}{\partial a}(a,b) = -2\sum_{i=1}^{n}(y_i - ax_i - b)x_i,\\
			\frac{\partial \varepsilon}{\partial b}(a,b) = -2\sum_{i=1}^{n}(y_i - ax_i - b).
		\end{cases}
	\]

	Donc,
	\begin{align*}
		(a,b) \text{ point critique de } \varepsilon \iff& \begin{cases}
			a \sum_{i=1}^n {x_i}^2 + b\sum_{i=1}^{n}x_i = \sum_{i=1}^{n} y_ix_i\\
			a\sum_{i=1}^{n}x_i + nb = \sum_{i=1}^ny_i
		\end{cases}\\
		\iff& \begin{cases}
			a \left( \frac{1}{n}\sum_{i=1}^n {x_i}^2 - \overline{x}^2\right) = \overline{y} - \overline{x} \overline{y}\\
			b = \frac{1}{n}\sum_{i=1}^ny_i - \frac{a}{n}\sum_{i=1}^nx_i = \frac{1}{n}\sum_{i=1}^n x_i y_i - \overline{x} \overline{y}
		\end{cases}\\
		&\text{ où } \overline{x} = \frac{1}{n} \sum_{i=1}^n x_i,~\overline{y} = \frac{1}{n}\sum_{i=1}^n y_i\\
		\iff& \begin{cases}
			a = \frac{\Cov(x,y)}{V(x)}\\
			b = \overline{y} - a\overline{x}
		\end{cases}
	\end{align*}

	Coefficient de corrélation: $\frac{\Cov(x,y)}{\sigma_x \sigma_y} \in [-1, 1]$
\end{exm}












		\part{Corps}

\begin{exm}[Problème]
	\begin{itemize}
		\item 
			avec $A = \Z / 9 \Z$, résoudre $\overline{x}^2 = \overline{0}$ \\
			\begin{center}
				\begin{tabular}{|c|c|c|c|c|c|c|c|c|c|c|}
					\hline
					$\overline{x}$&$\overline{0}$& $\overline{1}$ &$\overline{2}$&$\overline{3}$ &$\overline{4}$ &$\overline{5}$ &$\overline{6}$ &$\overline{7}$ &$\overline{8}$& $\overline{9}$ \\
					\hline
					$\overline{x}^2$&$\overline{0}$ &$\overline{1}$ &$\overline{4}$ &$\overline{0}$ &$\overline{7}$ &$7$ &$\overline{0}$ &$\overline{4}$ &$\overline{1}$&$\overline{0}$\\
					\hline
				\end{tabular}
			\end{center}
			On a trouvé 3 solutions: $\overline{0}$, $\overline{3}$, $\overline{6}$.
		\item $\Z / 8\Z$
			\begin{center}
				\begin{tabular}{|c|c|c|c|c|c|c|c|c|}
					\hline
					$\overline{x}$& $\overline{0}$& $\overline{1}$& $\overline{2}$& $\overline{3}$& $\overline{4}$& $\overline{5}$& $\overline{6}$& $\overline{7}$\\
					\hline
					$\overline{x^2}$& $\overline{0}$& $\overline{1}$& $\overline{4}$& $\overline{1}$& $\overline{0}$& $\overline{1}$& $\overline{4}$& $\overline{1}$\\
					\hline
				\end{tabular}
			\end{center}
			$\overline{x}^2=7$ a 4 solutions: $\overline{1}, \overline{7}, \overline{3},\text{ et } \overline{5}$
		\item $A = \mathbbm{H} = \{a + bi + cj + dk  \mid  (a,b,c,d) \in \R^4\}$ \\
			$i^2 = j^2 = k^2 = -1$ 
			\begin{align*}
				\begin{array}{c c c}
					ij = k & jk = i & ji = j\\
					ji = -k & kj = -i & ik = -j
				\end{array}
			\end{align*}
			Dans cet anneau, $-1$ a 6 racines!
	\end{itemize}
\end{exm}

\begin{defn}
	Soit $(\mathbbm{K}, +, \times)$ un ensemble muni de deux lois de composition internes. On dit que c'est un \underline{corps} si
	 \begin{enumerate}
		\item $(\mathbbm{K}, \times)$ est un groupe abélien
		\item $(\mathbbm{K}, \times)$ est un monoïde commutatif
		\item $\forall x \in \mathbbm{K}\setminus \{0_\mathbbm{K}\}, \exists y \in \mathbbm{K}, xy = 1_\mathbbm{K}$
		\item $0_\mathbbm{K} \neq  1_\mathbbm{K}$
	\end{enumerate}
	\index{corps}
\end{defn}

\begin{exm}
	\begin{itemize}
		\item $(\C, +, \times)$ est un corps
		\item $(\R, +, \times)$ est un corps
		\item $(\Q, +, \times)$ est un corps
		\item $(\Z, +, \times)$ n'est pas un corps
	\end{itemize}
\end{exm}

\begin{prop}
	$(\Z / n\Z, +, \times)$ est un corps si et seulement si $n$ est premier.
\end{prop}

\begin{prv}
	\[
		\left( \Z / n\Z \right)^\times = \left\{ \overline{k}  \mid k \wedge n = 1 \right\}
	\] 
\end{prv}


\begin{prop}
	Tout corps est un anneau intègre.
\end{prop}

\begin{prv}
	Soit $(\mathbbm{K}, +, \times)$ un corps. Soient $(a,b) \in \mathbbm{K}^2$ tel que $a \times b = 0_\mathbbm{K}$.\\
	On suppose $a \neq  0_\mathbbm{K}$. Alors, $a$ est inversible et donc \[
		b = a^{-1} \times a \times b = a^{-1} \times 0_\mathbbm{K} = 0_\mathbbm{K}
	\] 
\end{prv}

\begin{exm}
	Soit $(\mathbbm{K},+,\times)$ un corps.\\
	Résoudre \[
		\begin{cases}
			x^2 = 1_\mathbbm{K}\\
			x \in \mathbbm{K}
		\end{cases}
	\]

	\begin{align*}
		x^2 = 1_\mathbbm{K} &\iff x^2 - 1_\mathbbm{K} = 0_\mathbbm{K}\\
		&\iff (x - 1_\mathbbm{K})(x+1_\mathbbm{K}) = 0_\mathbbm{K}\\
		&\iff x - 1_\mathbbm{K} = 0_\mathbbm{K} \text{ ou } x + 1_\mathbbm{K} = 0_\mathbbm{K}\\
		&\iff x = 1_\mathbbm{K} \text{ ou } x = -1_\mathbbm{K}
	\end{align*}

	Il y a au plus 2 solutions.
\end{exm}

\begin{prop}
	Soit $(\mathbbm{K},+,\times )$ un corps et $P$ un polynôme à coefficients dans $\mathbbm{K}$ de degré $n$. Alors, l'équation $P(x) = 0_{\mathbbm{K}}$ a au plus $n$ solutions dans $\mathbbm{K}$ 
	\qed
\end{prop}

\begin{crlr}[(Théorème de Wilson)]
	voir exercice 16 du TD 12
\end{crlr}


\begin{defn}
	Soit $(\mathbbm{K}, +, \times)$ un corps et $L\subset \mathbbm{K}$.\\
	On dit que $L$ est un \underline{sous corps} de $\mathbbm{K}$ si
	\begin{enumerate}
		\item $L$ est un anneau de $(\mathbbm{K}, +, \times)$ non nul
		\item $\forall x \in L\setminus \{0_\mathbbm{K}\}, x^{-1} \in L$ 
	\end{enumerate}
	\vspace{2mm}
	en d'autres termes si
	\begin{enumerate}
		\item $\forall (x,y) \in L^2, x - y \in L$
		\item $\forall (x,y) \in L^2, x \times y^{-1} \in L$
	\end{enumerate}
	\vspace{5mm}
	On dit aussi que $\mathbbm{K}$ est une \underline{extension} de $L$.
	\index{sous corps}
	\index{extension}
\end{defn}

\begin{prop}
	Tout sous corps est un corps. \qed
\end{prop}

\begin{defn}
	Soient $(\mathbbm{K}_1,+,\times )$ et $(\mathbbm{K}_2,+, \times)$ deux corps et $f: \mathbbm{K}_1 \to \mathbbm{K}_2$.\\
	On dit que $f$ est un \underline{morphisme de corps} si $f$ est un morphisme d'anneaux.\\
	i.e. si
	\[
		\begin{cases}
			\forall (x,y) \in {\mathbbm{K}_1}^2,& f(x+y) = f(x) + f(y)\\
			\forall (x,y) \in {\mathbbm{K}_1}^2,& f(x \times y) = f(x) \times f(y)\\
		\end{cases}
	\] 
	\index{homomorphisme (de corps)}
	\index{morphisme (de corps)}
\end{defn}

\begin{prop}
	Tout morphisme de corps est injectif.
\end{prop}

\begin{prv}
	Soit $f: \mathbbm{K}_1 \to \mathbbm{K}_2$ un morphisme de corps.\\
	\begin{itemize}
		\item $\Ker(f)$ est un sous groupe de $(\mathbbm{K}_1, +)$ 
		\item Soit $x \in \Ker(f)$ et $y \in \mathbbm{K}_1$ \[
				f(x \times y) = f(x) \times f(y) = 0_{\mathbbm{K}_2} \times f(y) = 0_{\mathbbm{K}_2}
			\]
		\item Soit $x \in \Ker(f) \setminus \{0_{\mathbbm{K}_1}\}$.\\
			Alors, $x$ est inversible.\\
			\begin{align*}
				\begin{rcases*}
					x \in \Ker(f)\\
					x^{-1} \in \mathbbm{K}_1
				\end{rcases*}& \text{ donc } x \times x ^{-1} \in \Ker(f)\\
				&\text{ donc } 1_{\mathbbm{K}_1} \in \Ker(f)\\
				&\text{ donc } f(1_{\mathbbm{K}_1}) = 0_{\mathbbm{K}_2}
			\end{align*}
			Or, $f(1_{\mathbbm{K}_1}) = 1_{\mathbbm{K}_2} \neq 0_{\mathbbm{K}_2}$
	\end{itemize}
	Donc, $\Ker(f) = \{0_{\mathbbm{K}_1}\}$ donc $f$ est injective.
\end{prv}

\begin{exm}
	$\begin{array}{cc}
		\C &\longrightarrow \C\\
		z &\longmapsto \overline{z}\\
	\end{array}$ est un morphisme de corps
\end{exm}



		\part{Opérations sur les séries}

\begin{prop}
	L'ensemble $E = \{u \in \C^\N  \mid \Sigma u_n \text{ converge}\}$ est un sous-espace vectoriel de $\C^\N$ et \begin{align*}
		S: E &\longrightarrow \C \\
		u &\longmapsto \sum_{n=0}^{+\infty} u_n
	\end{align*} est une forme linéaire.
	\qed
\end{prop}

\begin{rmk}
	La somme d'une série convergente et d'une série divergente diverge.
	Le produit d'une série divergente par un scalaire non nul diverge.
\end{rmk}

	}

	{
		\chap[01]{Calculs algébriques}
		\renewcommand{\cwd}{../chap01}
		\begin{defn}
	Soit $E$ un $\mathbbm{K}$-espace vectoriel. On dit que $E$ est de \underline{dimension finie} si $E$ a au moins une famille génératrice finie. On dit que $E$ est de \underline{dimension infinie} sinon.
	\index{dimension finie (espace vectoriel)}
	\index{dimension infinie (espace vectoriel)}
\end{defn}

\begin{thm}
	[Théorème de la base extraite]
	Soit $E$ un $\mathbbm{K}$-espace vectoriel non nul de dimension finie. Soit $\mathcal{G}$ une famille génératrice finie de $E$. Alors, il existe une base $\mathcal{B}$ de $\mathcal{E}$ telle que $\mathcal{B} \subset \mathcal{G}$.
\end{thm}

\begin{prv}
	[par récurrence sur $\#G = \Card(G)$]
	\begin{itemize}
		\item Soit $E$ un $\mathbbm{K}$-espace vectoriel non nul engendré par $\mathcal{G} = (u)$.\\
			Si $u = 0_E$, alors $E = \{0_E\}$: une contradiction $\lightning$ \\
			Donc $u \neq 0_E$ donc $(u)$ est libre. En effet, \[
				\forall \lambda \in \mathbbm{K}, \lambda u = 0_E \implies \lambda = 0_\mathbbm{K}
			\] Donc $\mathcal{G}$ est une base de $E$.\\
		\item Soit $n \in \N_*$. Soit $E$ un $\mathbbm{K}$-espace vectoriel. On suppose que si $E$ a une famille génératrice constituée de $n$ vecteurs, alors on peut extraire de cette famille une base de $E$.\\
			Soit $\mathcal{G}$ une famille génératrice de $E$ avec $n+1$ vecteurs.\\
			Si $\mathcal{G}$ est libre, alors $\mathcal{G}$ est une base de $E$. \\
			Si $\mathcal{G}$ n'est pas libre, alors il existe $u \in \mathcal{G}$ tel que $u \in \Vect(\mathcal{G}\setminus \{u\})$ \\
			Donc $\mathcal{G}\setminus \{u\}$ engendre $E$. Or, $\mathcal{G}\setminus \{u\}$ possède $n$ vecteurs. D'après l'hypothèse de récurrence, il existe une base $\mathcal{B}$ de $E$ telle que \[
				\mathcal{B} \subset \mathcal{G} \setminus \{u\} \subset \mathcal{G}
			\] 
	\end{itemize}
\end{prv}

\begin{crlr}
	Tout espace de dimension finie a une base.
	\qed
\end{crlr}

\begin{thm}
	[Théorème de la base incomplète]
	Soit $E$ un $\mathbbm{K}$-espace vectoriel de dimension finie, $\mathcal{G}$ une famille génératrice finie de $E$. $\mathcal{L}$ une famille libre de $E$. Alors, il existe une base $\mathcal{B}$ de $E$ telle que \[
		\mathcal{L} \subset \mathcal{B} \text{ et } \mathcal{B}\setminus \mathcal{L} \subset \mathcal{G}
	\] 
\end{thm}

\begin{prv}
	[par récurrence sur $\#(\mathcal{G}\setminus\mathcal{L})$]
	\begin{itemize}
		\item Avec les notations précédentes, on suppose que $\mathcal{G}\setminus\mathcal{L} \neq \O$ \[
				\forall u \in \mathcal{G}, u \in \mathcal{L}
			\] Donc $\mathcal{G} \subset \mathcal{L}$ donc $\mathcal{L}$ est génératrice donc $\mathcal{L}$ est une base de $E$. On pose $\mathcal{B} = \mathcal{L}$ et alors \[
				\mathcal{L} \subset  \mathcal{B} \text{ et } \mathcal{B}\setminus\mathcal{L} = \O \subset  \mathcal{G}
			\] 
		\item Soit $n \in \N$. On suppose que si $\mathcal{G}$ est génératrice et $\mathcal{L}$ libre avec $\#(\mathcal{G}\setminus\mathcal{L}) = n$ alors il existe une base $\mathcal{B}$ de $E$ telle que \[
			\mathcal{L}\subset \mathcal{B} \text{ et } \mathcal{B}\setminus\mathcal{L}\subset \mathcal{G}
		\] Soient à présent $\mathcal{G}$ une famille génératrice de $E$ et $\mathcal{L}$ une famille libre de $E$ telles que $\#(\mathcal{G}\setminus\mathcal{L}) = n+1 > 0$\\
		Si $\mathcal{L}$ engendre $E$, alors $\mathcal{L}$ est une base de $E$. On pose $\mathcal{B} = \mathcal{L}$ et on a bien \[
			\mathcal{L} \subset  \mathcal{B} \text{ et } \mathcal{B} \setminus \mathcal{L} = \O \subset  \mathcal{G}
		\] On suppose que $\mathcal{L}$ n'engendre pas $E$. Il existe $u \in \mathcal{G}$ tel que $u \not\in \Vec(\mathcal{L})$ (car sinon, $\mathcal{G} \subset \Vect(\mathcal{L})$ et donc $\underbrace{\Vect(\mathcal{G})}_{= E} \subset  \underbrace{\Vect(\mathcal{L})}_{ \subset E}$\\
		Donc $\mathcal{L} \cup \{u\} $ est libre. On pose $\mathcal{L}' = \mathcal{L} \cup \{u\} $ \[
			\mathcal{G}\setminus \mathcal{L}' = \mathcal{G}\setminus (\mathcal{L} \cup \{u\}) = (\mathcal{G}\setminus\mathcal{L})\setminus \{u\} 
		\] donc $\#(\mathcal{G}\setminus\mathcal{L}') = n+1 -1 = n$\\
		D'après l'hypothèse de récurrence, il existe $\mathcal{B}$ une base de $E$ telle que \[
			\mathcal{L} \subset  \mathcal{L}' \subset \mathcal{B} \text{ et } \mathcal{B}\setminus \mathcal{L}' \subset \mathcal{G}
		\] \[
			\mathcal{B} \setminus \mathcal{L} = \underbrace{\mathcal{B}\setminus\mathcal{L}'}_{\subset \mathcal{G}} \cup \underbrace{\{u\}}_{\subset \mathcal{G} \text{ car } u \in \mathcal{G}}
		\] On a $\mathcal{B}\setminus\mathcal{L}\subset \mathcal{G}$
	\end{itemize}
\end{prv}

\begin{thm}
	Soit $E$ un $\mathbbm{K}$-espace vectoriel de dimension finie. Toutes les bases de $E$ ont le même cardinal.
\end{thm}

\begin{prv}
	Soit $\mathcal{G}$ une famille génératrice finie de $E$ et $\mathcal{B} \subset  \mathcal{G}$ une base de $E$. On note $n = \#\mathcal{B}$ \\
	Soit $\mathcal{B}'$ une base de $E$. On pose $p = n - \#(\mathcal{B} \cap  \mathcal{B}')$. Montrons par récurrence sur  $p$ que $\#\mathcal{B} = \#\mathcal{B}'$ 
	\begin{itemize}
		\item On suppose que $p = 0$. Alors, $\#(\mathcal{B} \cap \mathcal{B}') = n$ \\
			Or, $\mathcal{B}' \cap \mathcal{B} \subset \mathcal{B}$ donc $\mathcal{B} \cap \mathcal{B}' = \mathcal{B}$ donc $\mathcal{B} \subset  \mathcal{B}'$ et donc $\mathcal{B} = \mathcal{B}'$ 
		\item Soit $p \in \N$. On suppose que si $\mathcal{B}'$ est une base de $E$ telle que $n - \#(\mathcal{B} \cap \mathcal{B}') = p$, alors $\#\mathcal{B}' = n$ \\
			Aoit $\mathcal{B}'$ une base de $E$ telle que $n - \#(\mathcal{B}\cap \mathcal{B}') = p+1 > 0$ \\
			Donc $\mathcal{B} \cap \mathcal{B}' \neq \mathcal{B}$. Soit $u \in \mathcal{B}' \setminus \mathcal{B}$. D'après le lemme d'échange, il existe $v \in \mathcal{B}\setminus \mathcal{B}'$ tel que $\mathcal{B}' \setminus \{u\} \cup \{v\}$ est une base de $E$. On pose $\mathcal{B}'' = \mathcal{B}' \setminus \{u\} \cup \{v\}$ 
			\begin{align*}
				\mathcal{B}'' \cap \mathcal{B} &= \left( (\mathcal{B}' \setminus \{u\})  \cap \mathcal{B} \right) \cup \{v\} \\
				&= (\mathcal{B}' \cap \mathcal{B}) \cup \{v\} \\
			\end{align*}
			donc,
			\begin{align*}
				n - \#(\mathcal{B}'' \cap \mathcal{B}) &= n - (\#(\mathcal{B}' \cap \mathcal{B}) + 1) \\
				&= p+1- 1 \\
				&= p \\
			\end{align*}
			D'après l'hypothèse de récurrence, \[
				\#\mathcal{B}'' = n
			\] Or, $\#\mathcal{B}'' = \#\mathcal{B}'$
	\end{itemize}
\end{prv}

\begin{lem}
	Soient $\mathcal{B}$ et $\mathcal{B}'$ deux bases de $E$ telles que $\mathcal{B}\subset \mathcal{B}'$. Alors, $\mathcal{B} = \mathcal{B}'$.
\end{lem}

\begin{prv}
	On suppose $\mathcal{B}' \neq \mathcal{B}$. Soit $u \in \mathcal{B}' \setminus \mathcal{B}$
	$u \in E = \Vect(\mathcal{B})$ donc $\mathcal{B} \cup \{u\}$ n'est pas libre.
	Donc $\mathcal{B}\cup \{u\} \subset \mathcal{B}'$ et $\mathcal{B}'$ est libre donc $\mathcal{B}\cup \{u\}$ est libre: une contradiction $\lightning$
\end{prv}

\begin{lem}
	[Lemme d'échange] Soient $\mathcal{B}_1$ et $\mathcal{B}_2$ deux bases de $E$ et $u \in \mathcal{B}_1 \setminus \mathcal{B}_2$. Alors, il existe $v \in \mathcal{B}_2$ tel que $(\mathcal{B}_1 \setminus \{u\}) \cup \{v\}$ soit une base de $E$.
\end{lem}

\begin{prv}
	[1${}^\text{nde}$ méthode]
	On suppose que pout tout $v \in \mathcal{B}_2$, $(\mathcal{B}_1\setminus \{u\}) \cup \{v\}$ n'est pas une base de $E$
	Soit $v \in \mathcal{B}_2$.
	\begin{itemize}
		\item Supposons $(\mathcal{B}_1\setminus \{u\})\cup \{v\}$ non libre. $\mathcal{B}_1 \setminus \{u\}$ est libre. Donc $v \in \Vect(\mathcal{B}_1 \setminus \{u\})$
		\item Supposons $(\mathcal{B}_1\setminus \{u\}) \cup \{v\}$ non génératrice.
			Comme $\mathcal{B}_1$ engendre $E$, $u \not\in \Vect(\mathcal{B}_1\setminus \{v\})$.
			On suppose que $\mathcal{B}_1 \neq \mathcal{B}_2$.
			$\forall v \in \mathcal{B}_2 \setminus \mathcal{B}_1, \Vect(\mathcal{B}_1 \setminus \{v\}) = \Vect(\mathcal{B}_1) = E \ni u$ 
			donc, $(\mathcal{B}_1\setminus \{u\}) \cup \{v\}$ engendre $E$ et donc \[
				v \in \Vect(\mathcal{B}_1 \setminus \{u\})
			\] On a aussi \[
				\forall v \in \mathcal{B}_1 \setminus \{u\}, v \in \Vect(\mathcal{B}_1\setminus \{u\})
			\] Comme $u \not\in \mathcal{B}_2$, on a \[
				\forall v \in \mathcal{B}_2, v \in \Vect(\mathcal{B}_1\setminus \{u\})
			\] docn \[
				E = \Vect(\mathcal{B}_2) \subset \Vect(\mathcal{B}_1\setminus \{u\})
			\] donc $\mathcal{B}_1\setminus \{u\}$ engendre $E$ donc $\mathcal{B}_1\setminus \{u\}$ est une base de $E$. Or, $\mathcal{B}_1 \setminus \{u\}  \subset  \mathcal{B}_1$, donc $\mathcal{B}_1\setminus \{u\} = \mathcal{B}_1$
	\end{itemize}
\end{prv}

\begin{prv}
	[2${}^\text{nde}$ méthode]
	On suppose que pout tout $v \in \mathcal{B}_2$, $(\mathcal{B}_1\setminus \{u\}) \cup \{v\}$ n'est pas une base de $E$
	\begin{itemize}
		\item Comme $u \in \mathcal{B}_1 \setminus \mathcal{B}_2$, nécéssairement $\mathcal{B}_1 \neq \mathcal{B}_2$ donc $\mathcal{B}_2 \not\subset \mathcal{B}_1$, donc $\mathcal{B}_2\setminus\mathcal{B}_1 \neq \O$ 
		\item Soit $v \in \mathcal{B}_2\setminus\mathcal{B}_1$. Il existe $(\lambda_w)_{w\in\mathcal{B}_1}$ une famille de scalaires presque nulle telle que \[
				v = \sum_{w \in \mathcal{B}_1} \lambda_w w - \lambda_u u + + \sum_{w \in \mathcal{B}_1\setminus \{u\}}\lambda_w w
			\]
			Si $\lambda_u \neq 0_E$, alors
			\begin{align*}
				u &= \lambda_u^{-1}\left( v - \sum_{w \in \mathcal{B}_1 \setminus \{u\}} \lambda_w w \right)\\
					&\in \Vect(\mathcal{B}_1\setminus \{u\} \cup v)
			\end{align*}
			 donc $\mathcal{B}_1 \subset \Vect(\mathcal{B}_1\setminus \{u\} \cup \{v\})$\\
			 et donc $E \subset  \Vect(\mathcal{B}_1 \setminus \{u\} \cup \{v\})$ \\
			 et donc $\mathcal{B}_1 \setminus \{u\} \cup \{v\}$ engendre $E$ \\
			 donc $\mathcal{B}_1 \setminus \{u\} \cup \{v\}$ n'est pas libre\\
			 donc $v \in \Vect(\mathcal{B}_1\setminus \{u\})$ (car $\mathcal{B}_1 \setminus \{u\}$ est libre\\
			 donc $\lambda_u = 0_\mathbbm{K}$ $\lightning$\\`

			 Donc, $\lambda_u = 0_\mathbbm{K}$, docn $v \in \Vect(\mathcal{B}_1\setminus \{u\})$ \\
			 On vient de prouver que
			 \begin{align*}
			 	\mathcal{B}_2 \setminus \mathcal{B}_1 \subset \Vect(\mathcal{B}_1 \setminus \{u\})\\
			 	\mathcal{B}_1 \setminus \{u\} \subset \Vect(\mathcal{B}_1 \setminus \{u\})\\
			 \end{align*}
			 Comme $u \not\in \mathcal{B}_2$, \[
			 	\mathcal{B}_2 \subset \Vect(\mathcal{B}_1 \setminus \{u\})
			 \] donc \[
			 	E = \Vect(\mathcal{B}_2) \subset  \Vect(\mathcal{B}_1 \setminus \{u\})
			 \] donc $\mathcal{B}_1 \setminus \{u\}$ engendre $E$. Donc,  $\mathcal{B}_1 \setminus \{u\}$ est une base de $E$.\\
			 Or, $\mathcal{B}_1 \setminus \{u\} \subset  \mathcal{B}_1$, donc $\mathcal{B}_1 \setminus \{u\} = \mathcal{B}_1$
	\end{itemize}
\end{prv}

\begin{defn}
	Soit $E$ un $\mathbbm{K}$-espace vectoriel de dimension finie. Le cardinal commun à toutes les bases de $E$ est appelé \underline{dimension} de $E$ est notée $\dim(E)$ ou $\dim_\mathbbm{K}(E)$\\
	C'est donc aussi le nombre de coordonnées de n'importe quel vecteur dans n'importe quelle base.
	\index{dimension (espace vectoriel)}
\end{defn}

\begin{exm}
	\begin{enumerate}
		\item $\dim_\R(\C) = 2$ et $\dim_\C(\C) = 1$ 
		\item $\dim_\mathbbm{K}(\mathbbm{K}^{n}) = n$ 
		\item $\dim_{\mathbbm{K}}(\mathcal{M}_{n,p}(\mathbbm{K})) = np$
	\end{enumerate}
\end{exm}

\begin{crlr}
	Soit $E$ un $\mathbbm{K}$-espace vectoriel de dimension finie, $\mathcal{L}$ une famille libre de $E$, $\mathcal{G}$ une famille génératrice de $E$. On note $n = \dim(E)$
	\begin{enumerate}
		\item $\#\mathcal{G} \ge n$ et $(\#\mathcal{G} = n \implies \mathcal{G} \text{ est une base de } E$)
		\item $\#\mathcal{L} \le n$ et $(\#\mathcal{L} = n \implies \mathcal{L} \text{ est une base de } E$)
	\end{enumerate}
\end{crlr}

\begin{crlr}
	$\R^{\R}$ est de dimension infinie.
	$\forall i \in \N, e_i: x \mapsto x^i$\\
	$(e_i)_{i\in\N}$ est libre dans $\R^\R$
\end{crlr}

\begin{prop}
	Soient $E$ et $F$ deux $\mathbbm{K}$-espaces vectoriels de dimension finie. Alors $E\times F$ est de dimension finie et $\dim(E\times F) = \dim(E) + \dim(F)$
\end{prop}

\begin{prv}
	Soit $(e_1,\ldots, e_n)$ une base de $E$, $(f_1, \ldots, f_p)$ une base de $F$.
	On pose \[
		\left\{\begin{array}
			{r c l}
			u_1 &=& (e_1,0_F)\\
			u_2 &=& (e_2,0_F)\\
					&\vdots&\\
			u_n &=& (e_n,0_F)\\
			u_{n+1} &=& (0_E, f_1)\\
			u_{n+2} &=& (0_E, f_2)\\
					&\vdots&\\
			u_{n+p} &=& (0_E,f_p)\\
		\end{array}\right.
	\]
	Soit $(x,y) \in E\times F$. \[
		\begin{cases}
			\exists (x_1,\ldots,x_n)\in \mathbbm{K}^n, x = \sum_{i=1}^{n} x_ie_i
			\exists (y_1,\ldots,y_n)\in \mathbbm{K}^n, x = \sum_{j=1}^{p} y_jf_j
		\end{cases}
	\] 
	\begin{align*}
		(x,y) &= \left( \sum_{i=1}^{n} x_ie_i, \sum_{i=1}^{p} y_jf_j \right)  \\
		&= \sum_{i=1}^{n} x_i (e_i + 0_F) + \sum_{j=1}^{p} y_j (0_E, f_j) \\
		&= \sum_{i=1}^{n} x_i u_i + \sum_{j=1}^{p} y_j u_{n+j} \\
	\end{align*}
	Donc, $E\times F = \Vect(u_1, \ldots, u_{n+p})$ donc $E\times F$ est de dimension finie.\\
	Soit $(\lambda_1, \ldots, \lambda_{n+p}) \in \mathbbm{K}^{n+p}$ tel que \[
		(*): \quad \sum_{k=1}^{n+p} \lambda_ku_k = 0_{E\times F} = (0_E, 0_F)
	\]
	\begin{align*}
		(*) &\iff \sum_{k=1}^{n} \lambda_k (e_k, 0_F) + \sum_{k=n+1}^{p} \lambda_k(0_E, f_{k-n}) = (0_E, 0_F)\\
				&\iff \begin{cases}
					\sum_{k=1}^{n} \lambda_k e_k = 0_E\\
					\sum_{k=n+1}^{p} \lambda_k f_{k-n} = 0_F
				\end{cases}\\
				&\iff \begin{cases}
					\forall k \in \left\llbracket 1,n \right\rrbracket, \lambda_k = 0_\mathbbm{K} \qquad&(\text{car $(e_1,\ldots,e_n)$ est libre})\\
					\forall k \in \left\llbracket n+1,n+p \right\rrbracket, \lambda_k = 0_\mathbbm{K} \qquad&(\text{car $(f_1,\ldots,f_n)$ est libre})\\
				\end{cases}
	\end{align*}
	Donc $(u_1, \ldots, u_{n+p})$ est une base de $E\times F$. Donc, $\dim(E\times F) = n + p = \dim(E) + \dim(F)$
\end{prv}

\begin{rmk}
	[Convention]
	\[\dim\big(\{0_E\}\big) = 0\]
\end{rmk}

\begin{thm}
	Soit $E$ un $\mathbbm{K}$-espace vectoriel de dimension finie, $F$ un sous-espace vectoriel de $E$. Alors, $F$ est de dimension finie et  $\dim(F) \le \dim(E)$\\
	Si $\dim(F) = \dim(E)$, alors $F = E$
\end{thm}

\begin{prv}
	On considère \[
		A = \{k \in \N \mid \text{il existe une famille libre de $F$ à $k$ éléments}\} 
	\]
	On suppose $F \neq \{0_E\}$.
	\begin{itemize}
		\item Soit $u \in F\setminus \{0_E\}$. $(u)$ est libre donc $1 \in A$ et donc $A \neq \O$
		\item Soit $\mathcal{L}$ une famille libre de $F$. Alors, $\mathcal{L}$ est une famille libre de $E$ \\
			donc $\#\mathcal{L} \le \dim(E)$\\
			Donc $A$ est majorée par $\dim(E)$ \\
			On en déduit que $A$ a un plus grand élément $p$.
		\item Soit $\mathcal{L}$ une famille libre de $F$ avec $p$ éléments.\\
			Si $\mathcal{L}$ n'engendre pas $F$, alors il existe $u\in F$ tel que $u\not\in \Vect(\mathcal{L})$ et donc $\mathcal{L} \cup \{u\}$ est une famille libre de $F$, donc $p+1 \in A$ en contradiction avec la maximalité de $p$.\\
			Donc $\mathcal{L}$ est une base de $F$ donc $F$ est de dimension finie et $\dim(F) = p \le \dim(E)$\\
	\end{itemize}

	Soit $\mathcal{B}$ une base de $F$. Alors, $\mathcal{B}$ est aussi une famille de libre de de $E$. Donc $\#\mathcal{B} \le \dim(E)$ donc $\dim(F) = \dim(E)$ \\
	Si $\dim(F) = \dim(E)$, alors $\mathcal{B}$ est une base de $E$, et donc $F = \Vect(\mathcal{B}) = E$
\end{prv}

\begin{prop}
	[Formule de Grassmann]
	Soit $E$ un $\mathbbm{K}$-espace vectoriel de dimension finie, $F$ et $G$ deux sous-espace vectoriels de $E$. Alors, \[
		\dim(F+G) = \dim(F) + \dim(G) - \dim(F\cap G)
	\] 
\end{prop}

\begin{prv}
	Soit $(e_1, \ldots, e_p)$ une base de $F\cap G$. $(e_1,\ldots,e_p)$ est une famille libre de $F$.\\
	On complète $(e_1, \ldots, e_p)$ en une base $(e_1, \ldots, e_p, u_1, \ldots, u_q)$ de $F$.\\
	De même, on complète $(e_1, \ldots, e_p)$ en une base $(e_1, \ldots, e_p, v_1, \ldots, v_r)$ de $G$.\\
	On pose  $\mathcal{B} = (e_1, \ldots, e_p, u_1, \ldots, u_q, v_1, \ldots, v_r)$. Montrons que $\mathcal{B}$ est une base de $F+G$
	\begin{itemize}
		\item Soit $u \in F+G$ \\
			On pose $u = v+w$ avec $\begin{cases}
				v\in F\\
				w \in G
			\end{cases}$.\\
			On pose $v = \sum_{i=1}^p \lambda_i e_i + \sum_{i=1}^q \mu_i u_i$ avec $(\lambda_1, \ldots, \lambda_p, \mu_1, \ldots, \lambda_q) \in \mathbbm{K}^{p+q}$\\
			On pose aussi $w = \sum_{i = 1}^p \lambda'_ie_i + \sum_{j=1}^r \nu_j v_j$ avec $(\lambda_1',\ldots,\lambda_p', \nu_1, \ldots, \nu_r) \in \mathbbm{K}^{p+r}$\\
			D'où, \[
				u = \sum_{i=1}^p (\lambda_i + \lambda'_i)e_i + \sum_{j=1}^q \mu_j u_j + \sum_{k=1}^r \nu_k v_k \in \Vect(\mathcal{B})
			\]
		\item Soient $(\lambda_1, \ldots, \lambda_p, \mu_1, \ldots, \mu_q, \nu_1, \ldots, \nu_r) \in \mathbbm{K}^{p+q+r}$.\\
			On suppose \[
				(*)\quad \sum_{i=1}^{p}\lambda_ie_i + \sum_{j=1}^q\mu_ju_j + \sum_{k=1}^r \nu_k v_k = 0_E
			\] 
			D'où, \[
				\underbrace{\sum_{i=1}^p\lambda_i e_i + \sum_{j=1}^q \mu_ju_j}_{\in F} = \underbrace{-\sum_{k=1}^r\nu_jv_k}_{\in G}
			\] 
			Donc, \[
				f = \sum_{i=1}^p \lambda_i e_i + \sum_{j=1}^q \mu_j u_j \in F\cap G
			\] Comme $(e_1, \ldots, e_p)$ est une base de $F\cap G$, $\exists ! (\lambda_1', \ldots, \lambda_p') \in \mathbbm{K}^p$ tel que \[
				f = \sum_{i=1}^p \lambda'_i e_i = \sum_{i=1}^p \lambda'_i e_i + \sum_{j=1}^q 0_\mathbbm{K}u_j
			\] Comme $(e_1, \ldots, e_p, u_1, \ldots, u_q)$ est une base de $F$, \[
				\forall k \in \left\llbracket 1, q \right\rrbracket, \mu_j = 0_\mathbbm{K}
			\] De même, \[
				\forall k \in \left\llbracket 1,r \right\rrbracket , \nu_k = 0_\mathbbm{K}
			\] On remplace dans $(*)$ pour trouver \[
				\sum_{i=1}^p \lambda_ie_i = 0_E
			\] Comme $(e_1, \ldots, e_p)$ est libre, \[
				\forall i \in \left\llbracket 1,p \right\rrbracket, \lambda_i = 0_\mathbbm{K}
			\] Donc $\mathcal{B}$ est libre.\\
			Donc, 
			\begin{align*}
				\dim(F+G) &=  p +q + r \\
				&= (p+q)+ (p+r) - p \\
				&= \dim(F) + \dim(G) - \dim(F\cap G) \\
			\end{align*}
	\end{itemize}
\end{prv}

\begin{crlr}
	Avec les hypothèse précédentes, \[
		E = F \oplus G \iff \begin{cases}
			F \cap  G = \{0_E\} \\
			\dim(E) = \dim(F) + \dim(G)
		\end{cases}
	\] 
\end{crlr}

\begin{prv}
	\begin{itemize}
		\item[``$\implies$''] On suppose $E = F \oplus G$ \\
			Comme la somme est directe, $F \cap G = \{0_E\}$ 
			\begin{align*}
				\dim(E) &= \dim(F)\\
				&= \dim(F) + \dim(G) - \dim(F\cap G)\\
				&= \dim(F) + \dim(G)\\
			\end{align*}
		\item[``$\impliedby$''] On suppose $F\cap G = \{0_E\}$ et $\dim(E) = \dim(F) + \dim(G)$.\\
			On sait déjà que $F+G = F \oplus G$\\
			 \begin{align*}
				\dim(F+G) = \dim(F) + \dim(G) - \dim(F \cap G) = \dim(E)
			\end{align*}
			Donc $F + G = E$
	\end{itemize}
\end{prv}

\begin{prop}
	Soit $F$ un $\mathbbm{K}$-espace vectoriel de dimension finie $n$. Soit $\mathcal{B} = (e_1, \ldots, e_n)$ une base de $F$. L'application
	\begin{align*}
		f: \mathbbm{K}^n &\longrightarrow F \\
		(\lambda_1, \ldots, \lambda_n) &\longmapsto \sum_{i=1}^n \lambda_i e_i
	\end{align*} est bijective.\\
	Si $\mathbbm{K}$ est infini, $\mathbbm{K}^n$ aussi et donc $F$ aussi.\\
	Si $\#\mathbbm{K} = p \in \N_*$,
	\begin{align*}
		\#&\mathbbm{K}^n = p^n\\
		&\vrt=\\
		\#&F
	\end{align*}
\end{prop}


		\part{Dérivation}

\underline{Motivation}:

{
\begin{wrapfigure}{l}{3cm}
	\centering
	\begin{asy}
		import three;

		size(3cm);
		settings.render=0;
		settings.prc=false;
		currentprojection = obliqueZ;

		draw(unitbox);
		draw(shift(1.1Z + 0.05X) * (O -- X), Arrows3(TeXHead2));
		draw(shift(1.1Z + 0.05Y) * (O -- Y), Arrows3(TeXHead2));
		draw(shift(1.1X + 0.05Z) * (O -- Z), Arrows3(TeXHead2));

		label("$x$", (X/2) + (1.1Z + 0.05X), align=S);
		label("$y$", (Y/2) + (1.1Z + 0.05Y), align=W);
		label("$z$", (Z/2) + X, align=SE);
	\end{asy}
\end{wrapfigure}

\begin{align*}
	&S(x,y,z) = 2(xy + xz + yz)\\
	&V(x,y,z) = xyz
\end{align*}

On cherche à minimiser $S$ avec la contrainte $V = 1$.

Soit $f : \begin{array}{rcl}
	\left( \R_*^+ \right)^2 &\longrightarrow& \R \\
	(x,y) &\longmapsto& S\left( x,y,\frac{1}{xy} \right) = 2\left( xy + \frac{1}{y} + \frac{1}{x} \right).
\end{array}$

On cherche $(a,b) \in \left( \R^+_* \right)^2$ tel que \[
	\forall (x,y) \in (\R^+_*), f(x,y) \ge f(a,b).
\]
}

\begin{defn}
	Soit $f: U \to \R$ où $U$ est un ouvert de $\R^2$. Soit $(a,b) \in U$.
	\vspace{2mm}

	Si $\lim_{x \to a} \frac{f(x,b) - f(a,b)}{x - a} \in \R$, alors on dit que $f$ a une dérivée partielle suivant $x$ en $(a,b)$ et cette limite est notée \[
		\partial f_1(a,b) = \frac{\partial f}{\partial x}(a,b).
	\]

	Si $\lim_{y \to b} \frac{f(a,y) - f(a,b)}{y - b} \in \R$, alors on dit que $f$ a une dérivée partielle suivant $y$ et la limite est notée \[
		\partial f_2(a,b) = \frac{\partial f}{\partial y}(a,b).
	\]
\end{defn}

\begin{exm}
	\begin{enumerate}
		\item $f: (x,y) \mapsto xy + x - y$.

			\begin{align*}
				&\frac{\partial f}{\partial x} : (x,y) \mapsto y + 1,\\
				&\frac{\partial f}{\partial y} : (x,y) \mapsto x - 1.
			\end{align*}

		\item $f: (x,y) \mapsto xy + \frac{1}{y}+ \frac{1}{x}$.

			\begin{align*}
				&\frac{\partial f}{\partial x}: (x,y) \mapsto y - \frac{1}{x^2},\\
				&\frac{\partial f}{\partial y}: (x,y) \mapsto x - \frac{1}{y^2}.
			\end{align*}

		\item Trouver $f$ telle que $\begin{cases}
				(1): \qquad \frac{\partial f}{\partial x}=y,\\[2mm]
				(2): \qquad \frac{\partial f}{\partial y} = x.
			\end{cases}$

			D'après $(1)$ : \[
				\forall (x,y), \exists C(y) \in \R, f(x,y) = xy + C(y)
			\] et donc \[
				\frac{\partial f}{\partial y}(x,y) = x + C'(y)
			\] donc $C'(y) = 0$ et donc $C$ est constante.

		\item Trouver $f$ telle que $\begin{cases}
			\frac{\partial f}{\partial x} = -y,\\[2mm]
			\frac{\partial f}{ƒ\partial y} = x.
		\end{cases}$

		Ce n'est pas possible !
	\end{enumerate}
\end{exm}

\begin{defn}~\\
	\begin{minipage}{\linewidth}
		\begin{wrapfigure}{r}{4cm}
			\centering
			\vspace{-5mm}
			\begin{asy}
				import three;
				import graph3;
				size(4cm);

				settings.render = 0;
				settings.prc = false;
				currentprojection = obliqueX;

				draw(O -- X, Arrow3(TeXHead2));
				draw(O -- Y, Arrow3(TeXHead2));
				draw(O -- Z, Arrow3(TeXHead2));

				triple f(real x, real y, real z = 0) { return (x,y,cos(x - 0.5) * cos(y - 0.5)/1.2 + 0.15); }

				real inc = 1 / 5;

				for(real x = 0; x <= 1; x += inc) {
					draw(graph(
						new real(real t) { return x; }, // x
						new real(real y) { return y; }, // y
						new real(real y) { return f(x,y).z; }, // z
						0, 1
					), gray);
				}

				for(real y = 0; y <= 1; y += inc) {
					draw(graph(
						new real(real x) { return x; }, // x
						new real(real t) { return y; }, // y
						new real(real x) { return f(x,y).z; }, // z
						0, 1
					), gray);
				}

				path3 path1 = (0.8, 0.2, 0) .. (0.5, 0.5, 0) .. (0.3, 0.7, 0);
				path3 path2 = f(0.8, 0.2, 0) .. f(0.5, 0.5, 0) .. f(0.3, 0.7, 0);
				path3 d = (0.2, 0.3, 0) .. (0.3, 0.4, 0) .. (0.2, 0.7, 0) .. (0.8, 0.9, 0) .. (0.6, 0.2, 0) .. cycle;

				draw(path1, red, Arrow3(TeXHead2));
				draw(path2, red, Arrow3(TeXHead2, position=0.8));

				dot((0.5, 0.5, 0));
				dot(f(0.5, 0.5, 0));
				draw((0.5, 0.5, 0) -- f(0.5, 0.5, 0), dashed);
				draw(d);

				label("$w$", (0.3, 0.7, 0), red, align=SE);
				label("$U$", (0.8, 0.9, 0), align=SE);
			\end{asy}
		\end{wrapfigure}

		Soit $f: U \to \R$ où $U$ est un ouvert. Soit $(a,b) \in U$. Soit $w = (w_1, w_2) \in \R^2$.

		Si 
		\[
			\lim_{t\to 0} \frac{f(a + tw_1, b + tw_2) - f(a,b)}{t}
		\] existe et est réelle, alors on dit que $f$ a une dérivée dans la direction de $w$ et la limite est notée \[
			\mathrm{d}f(w)\,(a,b) = D_w(f)\,(a,b).
		\]
	\end{minipage}
\end{defn}

\begin{exm}
	\begin{align*}
		f: \left( \R_*^+ \right)^2 &\longrightarrow \R \\
		(x,y) &\longmapsto xy+\frac{1}{x}+\frac{1}{y}.
	\end{align*}

	On pose $(a,b) = (1,2)$, $w = (w_1, w_2) = (1,1)$.
	\begin{align*}
		\frac{f(1+t, 2+t) - f(1,2)}{t} &= \frac{1}{t} \left( (1+t)(2+t) + \frac{1}{1+t} + \frac{1}{2+t} - 3 - \frac{1}{2} \right) \\
		&= \frac{1}{t}\left(\cancel 2 + 3t + \po(t) + \cancel 1 - t + \po(t) + \frac{1}{2}\left( \cancel 1 - \frac{t}{2} + \po(t) \right) - \cancel3 - \cancel{\frac{1}{2}} \right) \\
		&= \frac{1}{t} \left( \frac{7}{4} t + \po(t) \right)  \\
		&= \frac{7}{4} + \po(1) \tendsto{t \to 0} \frac{7}{4}. \\
	\end{align*}

	Donc, \[
		\mathrm{d}f(1,1)\,(1,2) = \frac{7}{4}.
	\]
\end{exm}

\begin{rmk}~\\
	\begin{figure}[H]
		\centering
		\begin{asy}
			import solids;
			import graph;
			size(5cm);

			settings.render = 0;
			settings.prc = false;

			path3 par = graph(
				new real(real x) { return x; },
				new real(real x) { return 0; },
				new real(real x) { return x^2; },
				0,3);
			revolution r = revolution(par, axis=Z);

			path3 par2 = graph(
				new real(real x) { return x; },
				new real(real x) { return 0; },
				new real(real x) { return x^2; },
				-3,3);

			draw(r,1,longitudinalpen=nullpen);
			draw(r.silhouette());

			draw((-4, 0, -1) -- (-4, 0, 10) -- (4, 0, 10) -- (4, 0, -1) -- cycle, red);
			draw(par2, deepred);

			draw((4,4.5) -- (7, 4.5), black+0.5mm, Arrow(TeXHead));

			path par2d = graph(new real(real x) { return x^2; }, -3, 3);
			draw(shift((11, 0)) * par2d, deepred);

			dot(O);
			dot((11, 0));
		\end{asy}
	\end{figure}
\end{rmk}


%todo ajouter théorème-définition
\begin{thm}
	Soit $f : U \to \R$, $(a,b) \in U$. On suppose que $\frac{\partial f}{\partial x}$ et $\frac{\partial f}{\partial y}$ existent en $(a,b)$ et sont {\bfseries continues} en $(a,b)$. Alors,
	\begin{align*}
		&\forall (h, k) \in \R^2 \text{ tel que } (a +h, b + k) \in U,\\
		&f(a+ h, b + k) = f(a,b) + h \frac{\partial f}{\partial x}(a,b) + k \frac{\partial f}{\partial y}(a,b) + \po_{(h,k)\to (0,0)}\big(\|(h,k)\|\big).
	\end{align*}

	On dit que $f$ est de classe $\mathcal{C}^1$ si $\frac{\partial f}{\partial x}$ et $\frac{\partial f}{\partial y}$ existent et sont continues.

	\qed
\end{thm}

\begin{rmk}
	En physique, cette formule correspond à : \[
		\mathrm{d}f = \frac{\partial f}{\partial x}\mathrm{d}x + \frac{\partial f}{\partial y} \mathrm{d}y.
	\] En effet :
	\begin{align*}
		\mathrm{d}f &= f(x+ \mathrm{d}x, y + \mathrm{d}y) - f(x,y) \\
		&= \frac{\partial f}{\partial x} \mathrm{d}x + \frac{\partial f}{\partial y} \mathrm{d}y.
	\end{align*}
\end{rmk}

\begin{prop}
	Soit $f: U \to \R$ de classe $\mathcal{C}^1$ en $(a,b) \in U$. Alors, \[
		\forall w = (w_1, w_2) \in \R^2, \mathrm{d}f(w)\,(a,b) = w_1 \frac{\partial f}{\partial x}(a,b) + w_2 \frac{\partial f}{\partial y}(a,b).
	\]
\end{prop}

\begin{prv}
	Soit $w = (w_1, w_2) \in \R^2$. Soit $t \in \R^*$.
	\begin{align*}
		\frac{1}{t}\big(f(a + tw_1, b + tw_2) - f(a,b)\big)
		&= \frac{1}{t} \left( tw_1 \frac{\partial f}{\partial x}(a,b) + tw_2 \frac{\partial f}{\partial y}(a,b) + \po_{t \to 0}\big(\|tw\|\big) \right) \\
		&= w_1 \frac{\partial f}{\partial x}(a,b) + w_2 \frac{\partial f}{\partial y}(a,b) + \po_{t\to 0}(1) \\
		&\tendsto{t\to 0} w_1 \frac{\partial f}{\partial x}(a,b) + w_2\frac{\partial f}{\partial y}(a,b).
	\end{align*}
\end{prv}


\begin{defn}
	Avec les hypothèses précédentes, en posant \[
		\nabla f(a,b) = \left( \frac{\partial f}{\partial x}(a,b), \frac{\partial f}{\partial y}(a,b) \right) 
	\]on obtient \[
		\mathrm{d}f(w)\,(a,b) = \left<w  \mid \nabla f(a,b) \right>
	\] où $\left<\cdot|\cdot \right>$ est le produit scalaire.

	Le vecteur $\nabla f(a,b)$ est appelé \underline{gradient de $f$ en $(a,b)$}.

	Le développement limité à l'ordre 1 de $f$ devient \[
		f\big((a,b)+w\big) = f(a,b) + \left<w \mid \nabla f(a,b) \right> + \po_{w\to 0}(\|w\|)
	\]
\end{defn}

\begin{prop}
	Soit $f : U \to \R$ de classe $\mathcal{C}^1$.

	\begin{figure}[H]
    \centering
    \incfig{gradient}
	\end{figure}

	$\nabla f$ est orthogonal au lignes de niveaux de $f$, son orientation va dans le sens d'une augmentation de $f$.
\end{prop}

\begin{prv}
	Soit $\gamma : I \to U$ une courbe de niveau : \[
		\forall t \in I, f\big(\gamma(t)\big) = \text{cste}.
	\] D'après le lemme suivant : \[
		\forall t \in I, 0 = (f \circ \gamma)'(t) = \mathrm{d}f\big(\gamma'(t)\big)\big(\gamma(t)\big) = \left<\gamma'(t)  \mid \nabla f\big(\gamma(t)\big) \right>
	\] Donc $\nabla f\big(\gamma(t)\big)$ est orthogonal à $\gamma'(t)$.

	Pour tout $t \in I$, on pose $w(t) = t\, \nabla f\big(\gamma(t)\big)$. Donc \[
		f\big(\gamma(t) + w(t)\big) = f\big(\gamma(t)\big) + t \|\nabla f(\gamma(t))\|^2 + \po_{t \to 0}(t)
	\] Pour $t$ assez petit, $f\big(\gamma(t) + w(t)\big) - f\big(\gamma(t)\big)$ est du même signe que $t$.
\end{prv}

\begin{rmk}
	\begin{align*}
		V: \R^3 &\longrightarrow \R \\
		(x,y,z) &\longmapsto -mgz
	\end{align*}
	l'énerge potentielle de pesenteur

	On a donc \[
		\nabla V(x,y,z) = \left( \frac{\partial V}{\partial x}, \frac{\partial V}{\partial y}, \frac{\partial V}{\partial z} \right) = (0, 0, -mg) = \vec{P}.
	\]
\end{rmk}

\begin{lem}
	Soit $f : U \to \R$ de classe $\mathcal{C}^1$, $\gamma : \begin{array}{rcl}
		I &\longrightarrow& U \\
		t &\longmapsto& \big(x(t), y(t)\big)
	\end{array}$ où $x$ et $y$ sont dérivables.

	On pose \[
		\forall t \in I, \gamma'(t) = \big(x'(t), y'(t)\big).
	\] Alors $f \circ \gamma : I \to \R$ est dérivable et
	\begin{align*}
		\forall t \in I, (f \circ \gamma)'(t) &= \mathrm{d}f\big(\gamma'(t)\big) \big(\gamma(t)\big)\\
		&= \left<\gamma'(t)  \mid \nabla f\big(\gamma(t)\big)  \right> \\
		&= x'(t) \frac{\partial f}{\partial x}\big(x(t), y(t)\big) + y'(t) \frac{\partial f}{\partial y}\big(x(t),y(t)\big). \\
	\end{align*}
\end{lem}

\begin{prv}
	On fixe $t \in I$.

	\begin{align*}
		\forall h \neq 0, \frac{f \circ \gamma(t + h) - f \circ \gamma(t)}{h}
		&= \frac{1}{h}\big(f(\gamma(t)) + h\gamma'(t) + \po_{h\to 0}(h) - f(\gamma(t))\big) \\
		&= \frac{1}{h}\bigg(\cancel{f(\gamma(t))} + \left<h\gamma'(t) \mid \nabla f(\gamma(t)) \right> + \po_{h\to 0}(\|h\gamma'(t)\|) - \cancel{f(\gamma(t))}\bigg)\\
		&= \left<\gamma'(t) \mid \nabla f(\gamma(t)) \right> + \po_{h\to 0}(1) \\
		&\tendsto{h\to 0} \left<\gamma'(t)  \mid \nabla f(\gamma(t)) \right>
	\end{align*}
\end{prv}

\begin{defn}
	Soit $f : U \to \R$ de classe $\mathcal{C}^1$ et $(a,b) \in U$. On dit que $(a,b)$ est un \underline{point critique} de $f$ si $\nabla f(a,b) = 0$ i.e. $\frac{\partial f}{\partial x}(a,b) = \frac{\partial f}{\partial y}(a,b) = 0$.

	Dans ce cas, $f(a,b)$ est appelé \underline{valeur critique} de $f$.
\end{defn}

\begin{prop}~\\
	\begin{minipage}{\linewidth}
		\begin{wrapfigure}{r}{3cm}
			\centering
			\vspace{-1cm}
			\begin{asy}
				import solids;
				import graph;
				size(3cm);

				settings.render = 0;
				settings.prc = false;

				path3 par = graph(
					new real(real x) { return x; },
					new real(real x) { return 0; },
					new real(real x) { return -x^2; },
					0,3);
				revolution r = revolution(par, axis=Z);

				draw(r,1,longitudinalpen=nullpen);
				draw(r.silhouette());

				dot("$(a,b)$", O, red, align=N);
				real s = sqrt(2.5);
				path3 g=(s,0,-2.5)..(0,s,-2.5)..(-s,0,-2.5)..(0,-s,-2.5)..cycle;
				draw(g, deepcyan);
			\end{asy}
		\end{wrapfigure}
		Soit $f: U \to \R$ de classe $\mathcal{C}^1$ et $(a,b) \in U$ tel que \[
			\exists r > 0, \forall (x,y) \in B_{(a,b)}(r), f(x,y) \le f(a,b)
		\] Alors $\nabla f(a,b) = (0,0)$.
	\end{minipage}
\end{prop}

\begin{prv}
	Soit $g: x \mapsto f(x,b)$. $g(a)$ est un maximum local de $g$ donc $g'(a) = 0$.

	Or, $g'(a) = \frac{\partial f}{\partial x}(a,b)$

	donc $\frac{\partial f}{\partial x}(a,b) = 0$.

	Soit $h : y \mapsto f(a,y)$. On a de même $h'(b) = 0$.

	Or, $h'(b) = \frac{\partial f}{\partial y}(a,b)$.

	Donc, $\nabla f(a,b) = (0,0)$.
\end{prv}

\begin{rmk}
	Un minimum local est aussi une valeur critique.
\end{rmk}

\begin{figure}[H]
	\centering
	\begin{subfigure}{3cm}
		\centering
		\begin{asy}
			import solids;
			import graph;
			size(3cm);

			settings.render = 0;
			settings.prc = false;

			path3 par = graph(
				new real(real x) { return x; },
				new real(real x) { return 0; },
				new real(real x) { return -x^2; },
				0,3);
			revolution r = revolution(par, axis=Z);

			draw(r,1,longitudinalpen=nullpen);
			draw(r.silhouette());

			dot(O, red);
		\end{asy}
		\caption{Maximum local}
	\end{subfigure}
	\begin{subfigure}{3cm}
		\centering
		\begin{asy}
			import solids;
			import graph;
			size(3cm);

			settings.render = 0;
			settings.prc = false;

			path3 par = graph(
				new real(real x) { return x; },
				new real(real x) { return 0; },
				new real(real x) { return x^2; },
				0,3);
			revolution r = revolution(par, axis=Z);

			draw(r,1,longitudinalpen=nullpen);
			draw(r.silhouette());

			dot(O, red);
		\end{asy}
		\caption{Minimum local}
	\end{subfigure}
	\begin{subfigure}{3cm}
		\centering
		\begin{asy}
			import solids;
			import graph;
			size(3cm);

			settings.render = 0;
			settings.prc = false;
			currentprojection = obliqueZ;

			draw(graph(
				new real(real x) { return x; },
				new real(real x) { return -x^2 / 3; },
				new real(real x) { return 3; },
				-3, 3
			));

			draw(graph(
				new real(real x) { return x; },
				new real(real x) { return -x^2 / 3; },
				new real(real x) { return -3; },
				-3, 3
			));

			draw(graph(
				new real(real x) { return x; },
				new real(real x) { return -x^2 / 3 - 1; },
				new real(real x) { return 0; },
				-3, 3
			));

			draw(graph(
				new real(real x) { return 0; },
				new real(real x) { return x^2 / 9 - 1; },
				new real(real x) { return x; },
				-3, 3
			));

			draw(graph(
				new real(real x) { return -3; },
				new real(real x) { return x^2 / 9 - 4; },
				new real(real x) { return x; },
				-3, 3
			));

			draw(graph(
				new real(real x) { return 3; },
				new real(real x) { return x^2 / 9 - 4; },
				new real(real x) { return x; },
				-3, 3
			));

			dot((0,-1,0), red);
		\end{asy}
		\caption{Point de selle / Point col}
	\end{subfigure}
\end{figure}

\begin{exm}
	On revient à l'exemple donné en introduction : 
	\begin{align*}
		f: \left( \R^*_+ \right)^2 &\longrightarrow \R \\
		(x,y) &\longmapsto 2\left( xy + \frac{1}{x} + \frac{1}{y} \right).
	\end{align*}

	$\left( \R^+_* \right)^2$ est un ouvert de $\R^2$. Soit $(x,y) \in \left( \R^+_* \right)^2$.
	
	On a \[
		\begin{cases}
			\frac{\partial f}{\partial x}(x,y) = 2\left( y - \frac{1}{x^2} \right),\\
			\frac{\partial f}{\partial y}(x,y) = 2\left( x - \frac{1}{y^2} \right).
		\end{cases}
	\]

	\begin{align*}
		&\frac{\partial f}{\partial x}(x,y) = \frac{\partial f}{\partial y}(x,y) = 0\\
		\iff& \begin{cases}
			y = \frac{1}{x^2}\\
			x = \frac{1}{y^2}
		\end{cases}\\
		\iff& \begin{cases}
			y = \frac{1}{x^2}\\
			x = x^4
		\end{cases}\\
		\iff& \begin{cases}
			x = 1\\
			y = 1
		\end{cases}
	\end{align*}

	On vérivie que $f$ présente en effet un minium local en $(1,1)$. \[
		f(1,1) = 6
	\] On fixe $y \in \R^+_*$ et \[
		g : x \mapsto 2\left( xy + \frac{1}{x} + \frac{1}{y} \right).
	\] Donc \[
		\forall x \in \R^+_*, g'(x) = 2\left( y - \frac{1}{x^2} \right).
	\]
	\begin{center}
		\begin{tikzpicture}
			\tkzTabInit{$x$/1,$g'(x)$/1,$g$/2.3}{$0$, $\frac{1}{\sqrt{y}}$, $+\infty$}
			\tkzTabLine{,-,z,+,}
			\tkzTabVar{+/{}, -/$2\left( 2\sqrt{y} +\frac{1}{y} \right)$, +/{}}
		\end{tikzpicture}
	\end{center}
	
	Ainsi, \[
		\forall x \in \R^+_*, \forall y \in \R^+_*, f(x,y) \ge 2\left( 2\sqrt{y} + \frac{1}{y} \right)
	\] Soit $h : y \mapsto 2\sqrt{y} + \frac{1}{y}$. On a \[
		\forall y > 0, h'(y) = \frac{1}{\sqrt{y}} - \frac{1}{y^2} = \frac{y\sqrt{y} - 1}{y^2} = \frac{y^{\frac{3}{2}} - 1}{y^2}
	\]

	\begin{center}
		\begin{tikzpicture}
			\tkzTabInit{$y$/0.7,$h'(y)$/0.7,$h$/1.4}{$0$, $1$, $+\infty$}
			\tkzTabLine{,-,z,+,}
			\tkzTabVar{+/{}, -/$3$, +/{}}
		\end{tikzpicture}
	\end{center}

	Donc, \[
		\forall x,y > 0, f(x,y) \ge 2\times 3 = 6 = f(1,1).
	\]
\end{exm}

\begin{prop}
	[règle de la chaîne]

	Soit $f : \begin{array}{rcl}
		U &\longrightarrow& \R^2 \\
		(x,y) &\longmapsto& f(x,y)
	\end{array}$ de classe $\mathcal{C}^1$ et $U, V$ deux ouverts de $\R^2$.

	Soit $\varphi : \begin{array}{rcl}
		V &\longrightarrow& U \\
		(u,v) &\longmapsto& \varphi(u,v) = \big(x(u,v), y(u,v)\big)
	\end{array}$.

	On suppose que $x$ et $y$ sont de classe $\mathcal{C}^1$ sur $V$.

	Alors,  $f \circ \varphi : \begin{array}{rcl}
		V &\longrightarrow& \R \\
		(u,v) &\longmapsto& f\big(\varphi(u,v)\big)
	\end{array}$ est de classe $\mathcal{C}^1$ et
	\begin{align*}
		\forall (u_0, v_0) \in V, \frac{\partial (f \circ \varphi)}{\partial u}(u_0, v_0)
		&= \frac{\partial f}{\partial x}\big(\varphi(u_0, v_0)\big) \times \frac{\partial x}{\partial u}(u_0, v_0)\\
		&+ \frac{\partial f}{\partial y}\big(\varphi(u_0,v_0)\big) \frac{\partial y}{\partial u}(u_0,v_0)
	\end{align*}
	\begin{align*}
		\forall (u_0, v_0) \in V, \frac{\partial (f \circ \varphi)}{\partial v}(u_0, v_0)
		&= \frac{\partial f}{\partial x}\big(\varphi(u_0, v_0)\big) \times \frac{\partial x}{\partial v}(u_0, v_0)\\
		&+ \frac{\partial f}{\partial y}\big(\varphi(u_0,v_0)\big) \frac{\partial y}{\partial v}(u_0,v_0)
	\end{align*}
\end{prop}

\begin{exm}
	[changement de coordonnées polaires]
	On pose \begin{align*}
		\varphi: \R^+_* \times ]0,2\pi[ &\longrightarrow \R^2\setminus \left( R^+_* \times \{0\} \right) \\
		(r, \theta) &\longmapsto (r \cos \theta, r \sin\theta),
	\end{align*}
	\begin{align*}
		f: \R^2\setminus \left( R^+_* \times \{0\} \right) &\longrightarrow \R \\
		(x,y) &\longmapsto f(x,y),
	\end{align*}
	\begin{align*}
		g: \overbrace{\R^+_* \times ]0, 2\pi[}^{=V} &\longrightarrow \R \\
		(r, \theta) &\longmapsto f(r\cos\theta, r\sin\theta).
	\end{align*}

	\begin{align*}
		\forall (r_0,\theta_0) \in V,&\\[5mm]
		\frac{\partial g}{\partial r}(r_0, \theta_0) &= \frac{\partial f}{\partial x}(r_0\cos\theta_0, r_0\sin\theta_0)\cos\theta_0\\
		&+ \frac{\partial f}{\partial y}(r_0 \cos\theta_0, r_0\sin\theta_0)\sin\theta_0\\
		&= 2r_0\cos^2\theta_0 + 2r_0\sin^2(\theta_0) \\
		&= 2r_0 \\[5mm]
		\frac{\partial g}{\partial \theta}(r_0, \theta_0) &= \frac{\partial f}{\partial x}(r_0\cos\theta_0, r_0\sin\theta_0)r_0\sin\theta_0\\
		&+ \frac{\partial f}{\partial y}(r_0 \cos\theta_0, r_0\sin\theta_0)r_0\cos\theta_0\\
		&= -2{r_0}^2\cos(\theta_0)\sin(\theta_0) + 2{r_0}^2 \sin(\theta_0)\cos(\theta_0)\\
		&= 0 \\
	\end{align*}

	Donc, \[
		g(r, \theta) = r^2.
	\]
\end{exm}

\begin{exm}
	Résoudre \[
		\begin{cases}
			\frac{\partial f}{\partial x} = \frac{x}{x^2+y^2},\\
			\frac{\partial f}{\partial y} = \frac{y}{x^2+y^2}.\\
		\end{cases}
	\]

	On pose $g: (r, \theta) \mapsto f(r \cos\theta, r \sin\theta)$.

	\begin{align*}
		&\frac{\partial g}{\partial r} = \frac{1}{r}\cos^2\theta + \frac{1}{r}\sin^2\theta = \frac{1}{r},\\
		&\frac{\partial g}{\partial \theta} = -\cos(\theta) \sin(\theta) + \sin(\theta)\cos(\theta) = 0.
	\end{align*}

	Donc, \[
		\exists C \in \R, g: (r, \theta) \mapsto \ln r + C
	\] d'où,
	\begin{align*}
		\forall (x,y) \in \R^2 \setminus \{(0,0)\}, f(x,y) &= \ln\left(\sqrt{x^2 + y^2} \right)  + C\\
		&= \frac{1}{2}\ln(x^2 + y^2) + C. \\
	\end{align*}
\end{exm}

\begin{rmk}
	Soit $\mathcal{B} = (e_1, e_2)$ la base canonique de $\R^2$, $f: U \to \R$ de classe $\mathcal{C}^1$ avec $U$ un ouvert de $\R^2$.

	Soit $(x,y) \in U$.

	\begin{align*}
		\Mat_{\mathcal{B}}\big(\nabla f(x,y)\big) = \begin{pmatrix}
			\frac{\partial f}{\partial x}(x,y)\\[2mm]
			\frac{\partial f}{\partial y}(x,y)
		\end{pmatrix}
	\end{align*}

	Soit  \begin{align*}
		\varphi: V &\longrightarrow U \\
		(u,v) &\longmapsto \big(x(u,v), y(u,v)\big) 
	\end{align*} avec $x,y$ de classe $\mathcal{C}^1$. Soit $g = f \circ \varphi$.
	\begin{align*}
		\Mat_{\mathcal{B}}\big(\nabla g(u,v)\big)
		&= \begin{pmatrix}
			\frac{\partial g}{\partial u}(u,v) \\[2mm]
			\frac{\partial g}{\partial v}(u,v)
		\end{pmatrix} \\
		&= \begin{pmatrix}
			\frac{\partial x}{\partial u}(u,v) \frac{\partial f}{\partial x}(x,y)
			+ \frac{\partial y}{\partial u}(u,v)\frac{\partial f}{\partial y}(x,y)\\[3mm]
			\frac{\partial x}{\partial v}(u,v) \frac{\partial f}{\partial x}(x,y)
			+ \frac{\partial y}{\partial v}(u,v) \frac{\partial f}{\partial y}(x,y)
		\end{pmatrix}  \\
		&= \underbrace{\begin{pmatrix}
				\frac{\partial x}{\partial u}(u,v)& \frac{\partial y}{\partial u}(u,v)\\[3mm]
				\frac{\partial x}{\partial v}(u,v)& \frac{\partial y}{\partial v}(u,v)
		\end{pmatrix}}_{J(u,v)} \begin{pmatrix}
			\frac{\partial f}{\partial x}(x,y)\\[3mm]
			\frac{\partial f}{\partial y}(x,y)
		\end{pmatrix} \\
		&= J(u,v) \Mat_{\mathcal{B}}\big(\nabla f(x,y)\big) \\
	\end{align*}
	où $J(u,v) = 
	\begin{pNiceArray}{c:c}
		\Mat_{\mathcal{B}}\big(\nabla x(u,v)\big) & \Mat_{\mathcal{B}}\big(\nabla y(u,v)\big)
	\end{pNiceArray}$.

	On dit que $J(u,v)$ est \underline{la jacobienne} de $\varphi$ en $(u,v)$.
	L'application linéaire canoniquement associée à $J(u,v)$ est la \underline{différentielle de $\varphi$} en $(u,v)$ noté $\mathrm{d}\varphi(u,v)$.

	On a $\mathrm{d}\varphi(u,v) \in \mathcal{L}(R^2)$ et $\Mat_{\mathcal{B}}\big(\mathrm{d}\varphi(u,v)\big) = J(u,v)$.

	Par exemple, la jacobienne du changement de coordonnées polaires est \[
		J = \begin{pmatrix}
			\frac{\partial x}{\partial r} & \frac{\partial y}{\partial r}\\[3mm]
			\frac{\partial x}{\partial \theta} & \frac{\partial y}{\partial \theta}
		\end{pmatrix}
		= \begin{pmatrix}
			\cos\theta&\sin\theta\\
			-r\sin\theta&r\cos\theta
		\end{pmatrix}.
	\]
	$\underbrace{\det(J)}_{\text{le jacobien}} = r\cos^2\theta + r\sin^2\theta = r$

	Dans une intégrale double, si $(x,y) = \varphi(u,v)$, alors $\mathrm{d}x\mathrm{d}y = \det(J)\mathrm{d}u\mathrm{d}v$.

	Ici, \[
		\mathrm{d}x\ \mathrm{d}y = r\ \mathrm{d}r\ \mathrm{d}\theta.
	\]
\end{rmk}

\begin{prv}
	On pose $(x_0, y_0) = \varphi(u_0, v_0)$. Pour tout $(h,k) \in \R^2$ tels que $(u_0 + h, v_0 + k) \in V$, en posant $g = f  \circ \varphi$.

	\begin{align*}
		g(u_0 + h, v_0 + h) &= f\big(x(u_0 + h, v_0 + k), y(u_0 + h, v_0 + k)\big) \\
		&= f\left(
			x(u_0,v_0) + h \frac{\partial x}{\partial u}(u_0,v_0) + k \frac{\partial x}{\partial v}(u_0, v_0) + \po\big(\|(h,k)\|\big), \right.\\
		&\phantom{ = f\bigg(\bigg.}\left. y(u_0, v_0) + h \frac{\partial y}{\partial u}(u_0, v_0) + k \frac{\partial y}{\partial v}(u_0, v_0) + \po\big(\|(h,k)\|\big)
		\right)  \\
		&= f(x_0,y_0) \\
		&~+ \left( h \frac{\partial x}{\partial u}(u_0,v_0) + k \frac{\partial x}{\partial v}(u_0, v_0) + \po(\|(h,k)\|) \right) \frac{\partial f}{\partial x}(x_0,y_0)\\
		&~+ \left( h \frac{\partial y}{\partial u}(u_0, v_0) + k\frac{\partial y}{\partial v}(u_0, v_0) + \po(\|(h,k)\|) \right) \frac{\partial f}{\partial y}(x_0, y_0)\\
		&~+ \po(\|(h,k)\|)\\
		&= f(x_0, y_0) \\
		&~+ h \left( \frac{\partial x}{\partial u}(u_0, v_0) \frac{\partial f}{\partial x}(x_0, y_0) + \frac{\partial y}{\partial u}(u_0, v_0) \frac{\partial f}{\partial y}(x_0, y_0) \right)  \\
		&~+ k\left( \frac{\partial x}{\partial v}(u_0, v_0) \frac{\partial f}{\partial x}(x_0, y_0) + \frac{\partial y}{\partial v}(u_0, v_0) \frac{\partial f}{\partial y}(x_0, y_0) \right) 
		&~+ \po(\|(h,k)\|)\\
		&= g(u_0, v_0) + h \frac{\partial g}{\partial u}(u_0, v_0) + k \frac{\partial g}{\partial v}(u_0, v_0) + \po(\|(h,k)\|) \\
	\end{align*}

	Par identification,
	\[
		\frac{\partial g}{\partial u}(u_0, v_0) = \frac{\partial x}{\partial u}(u_0, v_0) \frac{\partial f}{\partial x}(x_0, y_0) + \frac{\partial y}{\partial u}(u_0, v_0) \frac{\partial f}{\partial y}(x_0,y_0)
	\] et \[
		\frac{\partial g}{\partial v}(u_0, v_0) = \frac{\partial x}{\partial v}(u_0,v_0) \frac{\partial f}{\partial x}(x_0, y_0) + \frac{\partial y}{\partial v}(u_0, v_0) \frac{\partial f}{\partial y}(x_0, y_0).
	\] 
\end{prv}

\begin{exm}
	[Régression linéaire]~\\
	\begin{figure}[H]
		\centering
		\begin{asy}
			import graph;
			axes(EndArrow);
			size(5cm);

			real f(real x) { return x + 0.5; }

			real k = 35 / (7 - 0.5);

			for(int i = 0; i < 35; ++i) {
				real mag = exp(sin(100 * pi/exp(1) * i)) * 0.8 + exp(cos(i*40)/3);
				real eps = mag * cos(10 * exp(1)/pi * i) / 3;
				dot((i/k,f(i/k) + eps));
			}

			draw(graph(f, -1, 7), orange);
		\end{asy}
	\end{figure}
	\[
		y = a x + b
	\] 
	On fixe $(a,b) \in \R^2$. \[
		\varepsilon(a,b) = \sum_{i=1}^n\big( y_i - (ax_i + b) \big)^2
	\] l'erreur totale.

	On veut minimiser $\varepsilon(a,b)$. On a 
	\[
		\forall (a,b) \in \R^2,
		\begin{cases}
			\frac{\partial \varepsilon}{\partial a}(a,b) = -2\sum_{i=1}^{n}(y_i - ax_i - b)x_i,\\
			\frac{\partial \varepsilon}{\partial b}(a,b) = -2\sum_{i=1}^{n}(y_i - ax_i - b).
		\end{cases}
	\]

	Donc,
	\begin{align*}
		(a,b) \text{ point critique de } \varepsilon \iff& \begin{cases}
			a \sum_{i=1}^n {x_i}^2 + b\sum_{i=1}^{n}x_i = \sum_{i=1}^{n} y_ix_i\\
			a\sum_{i=1}^{n}x_i + nb = \sum_{i=1}^ny_i
		\end{cases}\\
		\iff& \begin{cases}
			a \left( \frac{1}{n}\sum_{i=1}^n {x_i}^2 - \overline{x}^2\right) = \overline{y} - \overline{x} \overline{y}\\
			b = \frac{1}{n}\sum_{i=1}^ny_i - \frac{a}{n}\sum_{i=1}^nx_i = \frac{1}{n}\sum_{i=1}^n x_i y_i - \overline{x} \overline{y}
		\end{cases}\\
		&\text{ où } \overline{x} = \frac{1}{n} \sum_{i=1}^n x_i,~\overline{y} = \frac{1}{n}\sum_{i=1}^n y_i\\
		\iff& \begin{cases}
			a = \frac{\Cov(x,y)}{V(x)}\\
			b = \overline{y} - a\overline{x}
		\end{cases}
	\end{align*}

	Coefficient de corrélation: $\frac{\Cov(x,y)}{\sigma_x \sigma_y} \in [-1, 1]$
\end{exm}












		\part{Corps}

\begin{exm}[Problème]
	\begin{itemize}
		\item 
			avec $A = \Z / 9 \Z$, résoudre $\overline{x}^2 = \overline{0}$ \\
			\begin{center}
				\begin{tabular}{|c|c|c|c|c|c|c|c|c|c|c|}
					\hline
					$\overline{x}$&$\overline{0}$& $\overline{1}$ &$\overline{2}$&$\overline{3}$ &$\overline{4}$ &$\overline{5}$ &$\overline{6}$ &$\overline{7}$ &$\overline{8}$& $\overline{9}$ \\
					\hline
					$\overline{x}^2$&$\overline{0}$ &$\overline{1}$ &$\overline{4}$ &$\overline{0}$ &$\overline{7}$ &$7$ &$\overline{0}$ &$\overline{4}$ &$\overline{1}$&$\overline{0}$\\
					\hline
				\end{tabular}
			\end{center}
			On a trouvé 3 solutions: $\overline{0}$, $\overline{3}$, $\overline{6}$.
		\item $\Z / 8\Z$
			\begin{center}
				\begin{tabular}{|c|c|c|c|c|c|c|c|c|}
					\hline
					$\overline{x}$& $\overline{0}$& $\overline{1}$& $\overline{2}$& $\overline{3}$& $\overline{4}$& $\overline{5}$& $\overline{6}$& $\overline{7}$\\
					\hline
					$\overline{x^2}$& $\overline{0}$& $\overline{1}$& $\overline{4}$& $\overline{1}$& $\overline{0}$& $\overline{1}$& $\overline{4}$& $\overline{1}$\\
					\hline
				\end{tabular}
			\end{center}
			$\overline{x}^2=7$ a 4 solutions: $\overline{1}, \overline{7}, \overline{3},\text{ et } \overline{5}$
		\item $A = \mathbbm{H} = \{a + bi + cj + dk  \mid  (a,b,c,d) \in \R^4\}$ \\
			$i^2 = j^2 = k^2 = -1$ 
			\begin{align*}
				\begin{array}{c c c}
					ij = k & jk = i & ji = j\\
					ji = -k & kj = -i & ik = -j
				\end{array}
			\end{align*}
			Dans cet anneau, $-1$ a 6 racines!
	\end{itemize}
\end{exm}

\begin{defn}
	Soit $(\mathbbm{K}, +, \times)$ un ensemble muni de deux lois de composition internes. On dit que c'est un \underline{corps} si
	 \begin{enumerate}
		\item $(\mathbbm{K}, \times)$ est un groupe abélien
		\item $(\mathbbm{K}, \times)$ est un monoïde commutatif
		\item $\forall x \in \mathbbm{K}\setminus \{0_\mathbbm{K}\}, \exists y \in \mathbbm{K}, xy = 1_\mathbbm{K}$
		\item $0_\mathbbm{K} \neq  1_\mathbbm{K}$
	\end{enumerate}
	\index{corps}
\end{defn}

\begin{exm}
	\begin{itemize}
		\item $(\C, +, \times)$ est un corps
		\item $(\R, +, \times)$ est un corps
		\item $(\Q, +, \times)$ est un corps
		\item $(\Z, +, \times)$ n'est pas un corps
	\end{itemize}
\end{exm}

\begin{prop}
	$(\Z / n\Z, +, \times)$ est un corps si et seulement si $n$ est premier.
\end{prop}

\begin{prv}
	\[
		\left( \Z / n\Z \right)^\times = \left\{ \overline{k}  \mid k \wedge n = 1 \right\}
	\] 
\end{prv}


\begin{prop}
	Tout corps est un anneau intègre.
\end{prop}

\begin{prv}
	Soit $(\mathbbm{K}, +, \times)$ un corps. Soient $(a,b) \in \mathbbm{K}^2$ tel que $a \times b = 0_\mathbbm{K}$.\\
	On suppose $a \neq  0_\mathbbm{K}$. Alors, $a$ est inversible et donc \[
		b = a^{-1} \times a \times b = a^{-1} \times 0_\mathbbm{K} = 0_\mathbbm{K}
	\] 
\end{prv}

\begin{exm}
	Soit $(\mathbbm{K},+,\times)$ un corps.\\
	Résoudre \[
		\begin{cases}
			x^2 = 1_\mathbbm{K}\\
			x \in \mathbbm{K}
		\end{cases}
	\]

	\begin{align*}
		x^2 = 1_\mathbbm{K} &\iff x^2 - 1_\mathbbm{K} = 0_\mathbbm{K}\\
		&\iff (x - 1_\mathbbm{K})(x+1_\mathbbm{K}) = 0_\mathbbm{K}\\
		&\iff x - 1_\mathbbm{K} = 0_\mathbbm{K} \text{ ou } x + 1_\mathbbm{K} = 0_\mathbbm{K}\\
		&\iff x = 1_\mathbbm{K} \text{ ou } x = -1_\mathbbm{K}
	\end{align*}

	Il y a au plus 2 solutions.
\end{exm}

\begin{prop}
	Soit $(\mathbbm{K},+,\times )$ un corps et $P$ un polynôme à coefficients dans $\mathbbm{K}$ de degré $n$. Alors, l'équation $P(x) = 0_{\mathbbm{K}}$ a au plus $n$ solutions dans $\mathbbm{K}$ 
	\qed
\end{prop}

\begin{crlr}[(Théorème de Wilson)]
	voir exercice 16 du TD 12
\end{crlr}


\begin{defn}
	Soit $(\mathbbm{K}, +, \times)$ un corps et $L\subset \mathbbm{K}$.\\
	On dit que $L$ est un \underline{sous corps} de $\mathbbm{K}$ si
	\begin{enumerate}
		\item $L$ est un anneau de $(\mathbbm{K}, +, \times)$ non nul
		\item $\forall x \in L\setminus \{0_\mathbbm{K}\}, x^{-1} \in L$ 
	\end{enumerate}
	\vspace{2mm}
	en d'autres termes si
	\begin{enumerate}
		\item $\forall (x,y) \in L^2, x - y \in L$
		\item $\forall (x,y) \in L^2, x \times y^{-1} \in L$
	\end{enumerate}
	\vspace{5mm}
	On dit aussi que $\mathbbm{K}$ est une \underline{extension} de $L$.
	\index{sous corps}
	\index{extension}
\end{defn}

\begin{prop}
	Tout sous corps est un corps. \qed
\end{prop}

\begin{defn}
	Soient $(\mathbbm{K}_1,+,\times )$ et $(\mathbbm{K}_2,+, \times)$ deux corps et $f: \mathbbm{K}_1 \to \mathbbm{K}_2$.\\
	On dit que $f$ est un \underline{morphisme de corps} si $f$ est un morphisme d'anneaux.\\
	i.e. si
	\[
		\begin{cases}
			\forall (x,y) \in {\mathbbm{K}_1}^2,& f(x+y) = f(x) + f(y)\\
			\forall (x,y) \in {\mathbbm{K}_1}^2,& f(x \times y) = f(x) \times f(y)\\
		\end{cases}
	\] 
	\index{homomorphisme (de corps)}
	\index{morphisme (de corps)}
\end{defn}

\begin{prop}
	Tout morphisme de corps est injectif.
\end{prop}

\begin{prv}
	Soit $f: \mathbbm{K}_1 \to \mathbbm{K}_2$ un morphisme de corps.\\
	\begin{itemize}
		\item $\Ker(f)$ est un sous groupe de $(\mathbbm{K}_1, +)$ 
		\item Soit $x \in \Ker(f)$ et $y \in \mathbbm{K}_1$ \[
				f(x \times y) = f(x) \times f(y) = 0_{\mathbbm{K}_2} \times f(y) = 0_{\mathbbm{K}_2}
			\]
		\item Soit $x \in \Ker(f) \setminus \{0_{\mathbbm{K}_1}\}$.\\
			Alors, $x$ est inversible.\\
			\begin{align*}
				\begin{rcases*}
					x \in \Ker(f)\\
					x^{-1} \in \mathbbm{K}_1
				\end{rcases*}& \text{ donc } x \times x ^{-1} \in \Ker(f)\\
				&\text{ donc } 1_{\mathbbm{K}_1} \in \Ker(f)\\
				&\text{ donc } f(1_{\mathbbm{K}_1}) = 0_{\mathbbm{K}_2}
			\end{align*}
			Or, $f(1_{\mathbbm{K}_1}) = 1_{\mathbbm{K}_2} \neq 0_{\mathbbm{K}_2}$
	\end{itemize}
	Donc, $\Ker(f) = \{0_{\mathbbm{K}_1}\}$ donc $f$ est injective.
\end{prv}

\begin{exm}
	$\begin{array}{cc}
		\C &\longrightarrow \C\\
		z &\longmapsto \overline{z}\\
	\end{array}$ est un morphisme de corps
\end{exm}



		\part{Opérations sur les séries}

\begin{prop}
	L'ensemble $E = \{u \in \C^\N  \mid \Sigma u_n \text{ converge}\}$ est un sous-espace vectoriel de $\C^\N$ et \begin{align*}
		S: E &\longrightarrow \C \\
		u &\longmapsto \sum_{n=0}^{+\infty} u_n
	\end{align*} est une forme linéaire.
	\qed
\end{prop}

\begin{rmk}
	La somme d'une série convergente et d'une série divergente diverge.
	Le produit d'une série divergente par un scalaire non nul diverge.
\end{rmk}

		\part{Comparaison de suites}

\begin{defn}
	Soient $u$ et $v$ deux suites réelles. On dit que $u$ est \underline{dominée} par  $v$ si \[
	\exists M\in \R, \exists N\in \N,\forall n\ge N,\left| u_n \right| \le M \left| v_n \right| 
	\] Dans ce cas, on note $u = O(v)$ ou $u_n = O(v_n)$ et on dit que "$u$ est un grand o de $v$"
\end{defn}

\begin{exm}
	En informatique, on dit qu'un alogirithme a une \underline{complexité linéaire} si son temps d'éxécution est un $O(n)$ 
	Par exemple, on calcule $a^n$ 

	\begin{itemize}
		\item Approche naïve
			\begin{algorithm}
				\begin{algorithmic}[1]
					\State $p \gets 1$
					\For{$i \in \left\llbracket 0,n-1 \right\rrbracket$}
						\State $p \gets p \times a$
					\EndFor
					\State \Return p
				\end{algorithmic}
			\end{algorithm}
			Complexité linéaire $O(n)$
		\item Exponentiation rapide\\
			On écrit $n$ en binaire: \begin{align*}
				n &= \overline{a_k a_{k-1}\ldots a_0}^{(2)}\\
					&= \sum_{i=0}^{k} a_i 2^i
			\end{align*} avec $(a_i) \in \left\{ 0,1 \right\} ^{k+1}$
			\begin{align*}
				a^n &= a^{\sum_{i=0}^{k} a_i 2^i} \\
				&= \prod_{i=0}^{k} a^{a_i 2^i}  \\
			\end{align*}
			
			\begin{algorithm}
				\begin{algorithmic}
					[1]

					\State $s \gets 0$
					\State $p \gets a$
					\For{ $i \in \left\llbracket 0, \log_2(n) \right\rrbracket$}
						\State $p \gets p \times p$
						\If{$a[i] = 1$}
							\State $s \gets s + p$
						\EndIf
					\EndFor
					\State \Return s
				\end{algorithmic}
			\end{algorithm}
			Compléxité logarithmique $O(\log_2(n))$
	\end{itemize}
\end{exm}


\begin{prop}
	$O$ est une relation réfléctive et transitive.
\end{prop}

\begin{prv}
	\begin{itemize}
		\item Soit $u$ une suite. On pose $M = 1$ et \[
			\forall n \in \N, \left| u_n \right| \le M \left| u_n \right|
			\] Donc $u = O(u)$.
		\item Soient $u, v, w$ trois suites telles que  \[
		\begin{cases}
			u = O(v)\\
			v = O(w)
		\end{cases}
		\] Soient $M_1,M_2 \in \R$ et $N_1,N_2\in \N$ tels que \[
		\begin{cases}
			\forall n \ge  N_1, \left| u_n \right| \le M_1 \left| v_n \right| \\
			\forall n \ge  N_2, \left| v_n \right| \le M_2 \left| w_n \right| \\
		\end{cases}
		\] 

		Nécéssairement, $M_1\ge 0$ et $M_2\ge 0$.\\
		Soit $N = \max(N_1,N_2)$. \[
		\forall n \ge  N, \left| u_n \right| \le M_1 \left| v_n \right| \le  M_1M_2 \left| w_n \right| 
		\] Donc $u = O(w)$
	\end{itemize}
\end{prv}

\begin{defn}
	Soient $u$ et $v$ deux suites. On dit que $u$ est \underline{négligeable} devant $v$ si \[
	\forall \varepsilon>0, \exists N\in \N, \forall n\ge N, \left| u_n \right| \le \varepsilon \left| v_n \right| 
	\] Dans ce cas, on note $u = o(v)$ ou $u_n = o(v_n)$ ou on le lit "$u$ est un petit o de $v$"
\end{defn}

\begin{prop}
	$o$ est une relation transitive, non-réfléctive
\end{prop}

\begin{prv}
	\begin{itemize}
		\item Soient $u$, $v$ et $w$ trois suites telles que \[
			\begin{cases}
				u = o(v)\\
				v = o(w)
			\end{cases}
			\] Soit $\varepsilon>0$. Soit $N_1\in \N$ tel que \[
			\forall n \ge N_1, \left| u_n \right| \le \sqrt{\varepsilon}  \left| v_n \right| 
			\] Soit $N_2\in \N$ tel que \[
			\forall n \ge N_2, \left| v_n \right| \le \sqrt{\varepsilon}  \left| w_n \right| 
			\] On pose $N = \max(N_1,N_2)$, alors \[
			\forall n \ge N, \left| u_n \right| \le \sqrt{\varepsilon}  \left| v_n \right| \le \underbrace{\sqrt{\varepsilon} \times \sqrt{\varepsilon}} _\varepsilon \left| w_n \right| 
			\] donc $u = o(w)$
		\item Soit $u$ une suite tel qu'il existe $N \in \N$ tel que \[
		\forall n \ge N, u_n > 0
		\] On suppose que $u = o(u)$, alors \[
		\forall \varepsilon>0,\exists N \in \N, \forall n \ge N, \left| u_n \right| \le \varepsilon \left| u_n \right| 
		\] On pose $\varepsilon = \frac{1}{2}$ alors \[
		\exists N \in \N, \forall n \ge N, \left| u_n \right| \le \frac{1}{2} \left| u_n \right| 
		\] une contradiction
	\end{itemize}
\end{prv}

\begin{prop}
	Soient $u$ et $v$ deux suites.
	\begin{itemize}
		\item $o(u) + o(u) = o(u)$
		\item $v \times o(u) = o(uv)$
		\item $o(u) \times o(v) = o(uv)$
		\item $o(o(u)) = o(u)$
	\end{itemize}
	\qed
\end{prop}

\begin{defn}
	Soient $u$ et $v$ deux suites. On dit que $u$ et $v$ sont \underline{équivalentes} si \[
	u = v + o(v)
	\] i.e. \[
	\forall \varepsilon >0, \exists N \in \N, \forall n \ge N, \left| u_n-v_n \right| \le \varepsilon\left| v_n \right| 
	\] Dans ce cas, on le note $u \sim v$
\end{defn}

\begin{prop}
	$\sim$ est une relation d'équivalence \qed
\end{prop}

\begin{prop}
	Soient $(u,v) \in \R^\N$. On suppose que $v$ ne s'annule pas à partir d'un certain rang
	\begin{enumerate}
		\item $u = o(v) \iff \left( \frac{u_n}{v_n} \right)$ bornée
		\item $u = o(v) \iff \frac{u_n}{v_n} \tendsto{n \to  +\infty} 0$
		\item $u \sim v \iff \frac{u_n}{v_n} \tendsto{n \to  +\infty} 1$
	\end{enumerate}
	\qed
\end{prop}

\begin{prop}
	[Suites de références]
	\begin{enumerate}
		\item $\ln^\alpha(n) = o(n^\beta)$ avec $(\alpha,\beta) \in \left( \R^+_* \right) ^2$ 
		\item $n^\beta = o(a^n)$ avec $\beta > 0$ et $a > 1$ 
		\item $a^n = o(n!)$ avec $a >1$ 
		\item $n! = o(n^n)$
	\end{enumerate}
\end{prop}


\begin{lem}
	[Exercice 10 du TD]
	Soit $u \in \left(\R^+_*\right)^\N$\\
	Si $\frac{u_{n+1}}{u_n} \tendsto{n \to +\infty} \ell < 1$ avec $\ell\in \R$,\\ alors $u_n \tendsto{n \to +\infty} 0$
\end{lem}

\begin{prv} [de la proposition]
	\begin{enumerate}
		\item par croissance comparée
		\item On pose $\forall n \in \N^*, u_n = \frac{n^\beta}{a^n}$. 
			\begin{align*}
				\forall  n \in \N^*, \frac{u_{n+1}}{u_n} &= \left( \frac{n+1}{n} \right) ^\beta \times \frac{1}{a} \\
				&= \frac{1}{a}\left( 1+\frac{1}{n} \right) ^\beta \\
				&\tendsto{n \to +\infty} \frac{1}{a} < 1
			\end{align*}
			Donc, $u_n \tendsto{n \to  +\infty} 0$
		\item On pose $\forall n \in \N, u_n = \frac{a^n}{n!}$ \[
			\forall n \in \N, \frac{u_{n+1}}{u_n} = \frac{a}{n+1} \tendsto{n \to +\infty} 0 < 1
			\] donc $u_n \tendsto{n \to +\infty} 0$
		\item On pose $\forall  n\in \N^*, u_n = \frac{n!}{n^n}$.
			\begin{align*}
				\forall n \in \N^*, \frac{u_{n+1}}{u_n}
				&= (n+1) {\frac{n^n}{(n+1)^{n+1}}} \\
				&= \left( \frac{n}{n+1} \right) ^n \\
				&= e^{n \ln\left( \frac{n}{n+1} \right) } \\
				&= e^{n \ln\left( 1+\frac{1}{n+1} \right)} \\
				&= e^{n(-\frac{1}{n} + o(\frac{1}{n})} \\
				&= e^{-1 + o(1)} \\
				&\tendsto{n \to  +\infty} e^{-1}<1
			\end{align*}
			donc $u_n \tendsto{n\to +\infty} 0$
	\end{enumerate}
\end{prv}

		\part{Matrices par blocs}

\begin{exm}
	Soit $p$ un projecteur de $E$ : \[
		E = \Ker p \oplus \mathrm{Im}\ p
	\] Soit $\mathcal{B} = (e_1, \ldots, e_k, e_{k+1}, \ldots, e_n)$ une base de $E$ avec $\begin{cases}
		\mathrm{Im}(p) = \Vect(e_1, \ldots, e_k)\\
		\Ker(p) = \Vect(e_{k+1}, \ldots, e_n)\\
	\end{cases}$

	Alors, 
	\begin{align*}
		\Mat_\mathcal{B}(p) =
		\left(\begin{NiceArray}{c c c | c c c}
				1&&&0&\Cdots&0\\
				 &\Ddots&&\Vdots&&\Vdots\\
				&&1&0&\Cdots&0\\\hline
				0&\Cdots&0&0&\Cdots&0\\
				\Vdots&&\Vdots&\Vdots&&\Vdots\\
				0&\Cdots&0&0&\Cdots&0\\
		\end{NiceArray}\right)
		= \left( \begin{array}{c|c}
				I_k & 0\\ \hline
				0&0
		\end{array}\right) \\
	\end{align*}

	De même, si $\s$ est une symétrie de $E$, \[
		E = \Ker(\s - \id_E) \oplus \Ker(\s + \id_E)
	.\] Soit $\mathcal{C} = (e_1', \ldots, e_\ell', e_{\ell+1}', \ldots, e'_n)$ avec $\begin{cases}
		\Vect(e'_1, \ldots, e'_\ell) = \Ker(\s - \id_E),\\
		\Vect(e'_{\ell+1}, \ldots, e'_n) = \Ker(\s + \id_E).\\
	\end{cases}$

	Alors
	\[
		\Mat_\mathcal{C}(\s) = \left(\begin{array}{c|c}
				I_\ell &0\\ \hline
				0&-I_{n-\ell}
		\end{array}\right) 
	\]
\end{exm}

\begin{prop}
	Soient $F$ et $G$ supplémentaires dans $E$ : \[
		E = F \oplus G.
	\] Soit $f \in \mathcal{L}(F)$ et $g \in \mathcal{L}(G)$. Alors \[
	\exists !h \in \mathcal{L}(E) h_{|F} = f,\ h_{|G} = g \et h = f \circ p + g \circ q
	\] où $\begin{cases}
		p \text{ est la projection sur $F$ parallèlement à $G$}\\
		q \text{ est la projection sur $G$ parallèlement à $F$}\\
	\end{cases}$.

	On a aussi $q = \id_E - p$.
\end{prop}

\begin{prv}
	\begin{itemize}
		\item[\sc \underline{Analyse}] Soit $h \in \mathcal{L}(E)$ tel que $\begin{cases}
				h_{|F}=f\\
				h_{|G}=g
			\end{cases}$.

			Soit $x \in E$. Alors \[
				x = \underbrace{p(x)}_{\in F} + \underbrace{q(x)}_{\in G}
			\]

			Donc,
			\begin{align*}
				h(x) &= h\big(p(x)\big) + h\big(q(x)\big)\\
				&= f\big(p(x)\big) + g\big(q(x)\big) \\
				&= (f \circ p + g \circ q)(x) \\
			\end{align*}
			Si $h$ existe, alors \[
				h = f \circ p + g \circ q
			\]
		\item[\underline{\sc Synthèse}] On pose $h = f \circ p + g  \circ q$.

			$p$, $q$, $f$ et $g$ sont linéaires donc $h$ aussi.

			Soit $x \in E$.
			\begin{align*}
				h(x) &= f\big(p(x)\big) + g\big(q(x)\big) \\
				&= f(x) + g(0_E) \\
				&= f(x) \\
			\end{align*}
			donc $h_{|F} = f$ et de même $h_{|G}=g$.
	\end{itemize}
\end{prv}

\begin{prop}
	On reprend les notations et hypothèses précédentes. Soit $(e_1, \ldots, e_p)$ une base de $F$, et $(f_1, \ldots, f_q)$ une base de $G$. Alors, $\mathcal{B} = (e_1, \ldots, e_p, f_1, \ldots, f_q)$ est une base de $E$ et \[
		\Mat_\mathcal{B}(h) = \left(
		\begin{array}{c|c}
			A&0\\ \hline
			0&B
		\end{array}\right)
	\] où $\begin{cases}
		A = \Mat_{(e_1, \ldots e_p)}(f)\\
		B = \Mat_{(f_1, \ldots, f_q)}(g)
	\end{cases}$
	\qed
\end{prop}

\begin{prop}
	Soient $(A,A') \in \mathcal{M}_n(\mathbbm{K})^2$ et $(B,B') \in \mathcal{M}_p(\mathbbm{K})^2$.
	\begin{enumerate}
		\item \[
				\left(\begin{array}{c|c}
					A&0\\ \hline
					0&B
				\end{array}\right)
				\left(\begin{array}{c|c}
					A'&0\\ \hline
					0&B'
				\end{array}\right) = 
				\left(\begin{array}{c|c}
					AA'&0\\ \hline
					0&BB'
				\end{array}\right)
			\]
		\item \[
				\left(\begin{array}{c|c}
					A&0\\ \hline
					0&B
				\end{array}\right) \in \mathrm{GL}_{n+p}(\mathbbm{K})	 \iff \begin{cases}
					 A \in \mathrm{GL}_n(\mathbbm{K})\\
					 B \in \mathrm{GL}_p(\mathbbm{K})
				\end{cases}
			\] et dans ce cas, \[
				\left(\begin{array}{c|c}
					A&0\\ \hline
					0&B
				\end{array}\right)^{-1} =
				\left(\begin{array}{c|c}
					A^{-1}&0\\ \hline
					0&B^{-1}
				\end{array}\right)
			\]
		\item \[
				\tr \left(\begin{array}{c|c}
					A&0\\ \hline
					0&B
				\end{array}\right) = \tr A + \tr B
			\]
	\end{enumerate}
\end{prop}

\begin{prv}
	\begin{enumerate}
		\item Soit $\begin{cases}
				f \in \mathcal{L}(F) \text{ tel que } \Mat_\mathcal{B}(f) = A,
				f' \in \mathcal{L}(F) \text{ tel que } \Mat_\mathcal{B}(f') = A',
				g \in \mathcal{L}(G) \text{ tel que } \Mat_\mathcal{C}(g) = B,
				g' \in \mathcal{L}(G) \text{ tel que } \Mat_\mathcal{C}(g') = B'
			\end{cases}$ où $\begin{cases}
				F \oplus G = \mathbbm{K}^{n+p},\\
				\dim(F) = n, \dim(G) = p,\\
				\mathcal{B} \text{ base de } F,\\
				\mathcal{C} \text{ base de } G.\\
			\end{cases}$
			Soit $\begin{cases}
				h \in \mathcal{L}(\mathbbm{K}^{n+p}) \text{ tel que } \begin{cases}
					h_{|F} = f\\
					h_{|G} = g
				\end{cases}\\
				h' \in \mathcal{L}(\mathbbm{K}^{n+p}) \text{ tel que } \begin{cases}
					h'_{|F} = f'\\
					h'_{|G} = g'\\
				\end{cases}
			\end{cases}$
			Soit $\mathcal{D} = \mathcal{B} \cup \mathcal{C}$ une base de $\mathbbm{K}^{n+p}$.
			\begin{align*}
				\left(\begin{array}{c|c}
					A&0\\ \hline
					0&B
				\end{array}\right)
				\left(\begin{array}{c|c}
					A'&0\\ \hline
					0&B'
				\end{array}\right) &= \Mat_{\mathcal{D}}(h) \Mat_{\mathcal{D}}(h')\\
				&= \Mat_{\mathcal{D}}(h \circ h') \\
			\end{align*}
			Or, $(h \circ h')_{|F} = f \circ f'$ et $(h \circ h')_{|G} = g \circ g'$.

			Donc,
			\begin{align*}
				\Mat_\mathcal{D}(h \circ h') &=
					\left(\begin{array}{c|c}
						\Mat_\mathcal{B}(f \circ f')&0\\ \hline
						0&\Mat_\mathcal{C}(g \circ g')
					\end{array}\right)\\
				&=\left(\begin{array}{c|c}
					AA'&0\\ \hline
					0&BB'
				\end{array}\right).
			\end{align*}
	\end{enumerate}
\end{prv}

\begin{prop}
	Soient $A,A' \in \mathcal{M}_n(\mathbbm{K})$, $B,B' \in \mathcal{M}_{n,p}(\mathbbm{K})$, $C,C' \in \mathcal{M}_{p,n}(\mathbbm{K})$ et $D, D' \in \mathcal{M}_p(\mathbbm{K})$.

	\[
		\left(\begin{array}{c|c}
			A&B\\ \hline
			C&D
		\end{array}\right)
		\left(\begin{array}{c|c}
			A'&B'\\ \hline
			C'&D'
		\end{array}\right) = 
		\left(\begin{array}{c|c}
			AA' + BC'& AB' + BD'\\ \hline
			CA' + DC'&CB' + DD'
		\end{array}\right)
	\] Cette formule se généralise à un nombre quelconque de blocs : \[
		\left(\begin{array}{c|c|c|c}
				A_{11}&A_{12}&\cdots&A_{1,n}\\ \hline
				A_{21}&A_{22}&\cdots&A_{2,n}\\ \hline
				\vdots&\vdots&\ddots&\vdots\\ \hline
				A_{p,1}&A_{p,2}&\cdots&A_{p,n}
		\end{array}\right)
		\left(\begin{array}{c|c|c|c}
				A'_{11}&A'_{12}&\cdots&A'_{1,n}\\ \hline
				A'_{21}&A'_{22}&\cdots&A'_{2,n}\\ \hline
				\vdots&\vdots&\ddots&\vdots\\ \hline
				A'_{p,1}&A'_{p,2}&\cdots&A'_{p,n}
		\end{array}\right)
	\] Cette matrice se calcyle comme on s'y attend si les dimensions des blocs autorisent les produits.
\end{prop}

\begin{prop}
	Le rang d'une matrice $A$, c'est la taille de la plus grande matrice carrée inversible que l'on peut extraire de $A$.
	\qed
\end{prop}




	}

	{
		\chap[02]{Nombres complexes}
		\renewcommand{\cwd}{../chap02}
		\begin{defn}
	Soit $E$ un $\mathbbm{K}$-espace vectoriel. On dit que $E$ est de \underline{dimension finie} si $E$ a au moins une famille génératrice finie. On dit que $E$ est de \underline{dimension infinie} sinon.
	\index{dimension finie (espace vectoriel)}
	\index{dimension infinie (espace vectoriel)}
\end{defn}

\begin{thm}
	[Théorème de la base extraite]
	Soit $E$ un $\mathbbm{K}$-espace vectoriel non nul de dimension finie. Soit $\mathcal{G}$ une famille génératrice finie de $E$. Alors, il existe une base $\mathcal{B}$ de $\mathcal{E}$ telle que $\mathcal{B} \subset \mathcal{G}$.
\end{thm}

\begin{prv}
	[par récurrence sur $\#G = \Card(G)$]
	\begin{itemize}
		\item Soit $E$ un $\mathbbm{K}$-espace vectoriel non nul engendré par $\mathcal{G} = (u)$.\\
			Si $u = 0_E$, alors $E = \{0_E\}$: une contradiction $\lightning$ \\
			Donc $u \neq 0_E$ donc $(u)$ est libre. En effet, \[
				\forall \lambda \in \mathbbm{K}, \lambda u = 0_E \implies \lambda = 0_\mathbbm{K}
			\] Donc $\mathcal{G}$ est une base de $E$.\\
		\item Soit $n \in \N_*$. Soit $E$ un $\mathbbm{K}$-espace vectoriel. On suppose que si $E$ a une famille génératrice constituée de $n$ vecteurs, alors on peut extraire de cette famille une base de $E$.\\
			Soit $\mathcal{G}$ une famille génératrice de $E$ avec $n+1$ vecteurs.\\
			Si $\mathcal{G}$ est libre, alors $\mathcal{G}$ est une base de $E$. \\
			Si $\mathcal{G}$ n'est pas libre, alors il existe $u \in \mathcal{G}$ tel que $u \in \Vect(\mathcal{G}\setminus \{u\})$ \\
			Donc $\mathcal{G}\setminus \{u\}$ engendre $E$. Or, $\mathcal{G}\setminus \{u\}$ possède $n$ vecteurs. D'après l'hypothèse de récurrence, il existe une base $\mathcal{B}$ de $E$ telle que \[
				\mathcal{B} \subset \mathcal{G} \setminus \{u\} \subset \mathcal{G}
			\] 
	\end{itemize}
\end{prv}

\begin{crlr}
	Tout espace de dimension finie a une base.
	\qed
\end{crlr}

\begin{thm}
	[Théorème de la base incomplète]
	Soit $E$ un $\mathbbm{K}$-espace vectoriel de dimension finie, $\mathcal{G}$ une famille génératrice finie de $E$. $\mathcal{L}$ une famille libre de $E$. Alors, il existe une base $\mathcal{B}$ de $E$ telle que \[
		\mathcal{L} \subset \mathcal{B} \text{ et } \mathcal{B}\setminus \mathcal{L} \subset \mathcal{G}
	\] 
\end{thm}

\begin{prv}
	[par récurrence sur $\#(\mathcal{G}\setminus\mathcal{L})$]
	\begin{itemize}
		\item Avec les notations précédentes, on suppose que $\mathcal{G}\setminus\mathcal{L} \neq \O$ \[
				\forall u \in \mathcal{G}, u \in \mathcal{L}
			\] Donc $\mathcal{G} \subset \mathcal{L}$ donc $\mathcal{L}$ est génératrice donc $\mathcal{L}$ est une base de $E$. On pose $\mathcal{B} = \mathcal{L}$ et alors \[
				\mathcal{L} \subset  \mathcal{B} \text{ et } \mathcal{B}\setminus\mathcal{L} = \O \subset  \mathcal{G}
			\] 
		\item Soit $n \in \N$. On suppose que si $\mathcal{G}$ est génératrice et $\mathcal{L}$ libre avec $\#(\mathcal{G}\setminus\mathcal{L}) = n$ alors il existe une base $\mathcal{B}$ de $E$ telle que \[
			\mathcal{L}\subset \mathcal{B} \text{ et } \mathcal{B}\setminus\mathcal{L}\subset \mathcal{G}
		\] Soient à présent $\mathcal{G}$ une famille génératrice de $E$ et $\mathcal{L}$ une famille libre de $E$ telles que $\#(\mathcal{G}\setminus\mathcal{L}) = n+1 > 0$\\
		Si $\mathcal{L}$ engendre $E$, alors $\mathcal{L}$ est une base de $E$. On pose $\mathcal{B} = \mathcal{L}$ et on a bien \[
			\mathcal{L} \subset  \mathcal{B} \text{ et } \mathcal{B} \setminus \mathcal{L} = \O \subset  \mathcal{G}
		\] On suppose que $\mathcal{L}$ n'engendre pas $E$. Il existe $u \in \mathcal{G}$ tel que $u \not\in \Vec(\mathcal{L})$ (car sinon, $\mathcal{G} \subset \Vect(\mathcal{L})$ et donc $\underbrace{\Vect(\mathcal{G})}_{= E} \subset  \underbrace{\Vect(\mathcal{L})}_{ \subset E}$\\
		Donc $\mathcal{L} \cup \{u\} $ est libre. On pose $\mathcal{L}' = \mathcal{L} \cup \{u\} $ \[
			\mathcal{G}\setminus \mathcal{L}' = \mathcal{G}\setminus (\mathcal{L} \cup \{u\}) = (\mathcal{G}\setminus\mathcal{L})\setminus \{u\} 
		\] donc $\#(\mathcal{G}\setminus\mathcal{L}') = n+1 -1 = n$\\
		D'après l'hypothèse de récurrence, il existe $\mathcal{B}$ une base de $E$ telle que \[
			\mathcal{L} \subset  \mathcal{L}' \subset \mathcal{B} \text{ et } \mathcal{B}\setminus \mathcal{L}' \subset \mathcal{G}
		\] \[
			\mathcal{B} \setminus \mathcal{L} = \underbrace{\mathcal{B}\setminus\mathcal{L}'}_{\subset \mathcal{G}} \cup \underbrace{\{u\}}_{\subset \mathcal{G} \text{ car } u \in \mathcal{G}}
		\] On a $\mathcal{B}\setminus\mathcal{L}\subset \mathcal{G}$
	\end{itemize}
\end{prv}

\begin{thm}
	Soit $E$ un $\mathbbm{K}$-espace vectoriel de dimension finie. Toutes les bases de $E$ ont le même cardinal.
\end{thm}

\begin{prv}
	Soit $\mathcal{G}$ une famille génératrice finie de $E$ et $\mathcal{B} \subset  \mathcal{G}$ une base de $E$. On note $n = \#\mathcal{B}$ \\
	Soit $\mathcal{B}'$ une base de $E$. On pose $p = n - \#(\mathcal{B} \cap  \mathcal{B}')$. Montrons par récurrence sur  $p$ que $\#\mathcal{B} = \#\mathcal{B}'$ 
	\begin{itemize}
		\item On suppose que $p = 0$. Alors, $\#(\mathcal{B} \cap \mathcal{B}') = n$ \\
			Or, $\mathcal{B}' \cap \mathcal{B} \subset \mathcal{B}$ donc $\mathcal{B} \cap \mathcal{B}' = \mathcal{B}$ donc $\mathcal{B} \subset  \mathcal{B}'$ et donc $\mathcal{B} = \mathcal{B}'$ 
		\item Soit $p \in \N$. On suppose que si $\mathcal{B}'$ est une base de $E$ telle que $n - \#(\mathcal{B} \cap \mathcal{B}') = p$, alors $\#\mathcal{B}' = n$ \\
			Aoit $\mathcal{B}'$ une base de $E$ telle que $n - \#(\mathcal{B}\cap \mathcal{B}') = p+1 > 0$ \\
			Donc $\mathcal{B} \cap \mathcal{B}' \neq \mathcal{B}$. Soit $u \in \mathcal{B}' \setminus \mathcal{B}$. D'après le lemme d'échange, il existe $v \in \mathcal{B}\setminus \mathcal{B}'$ tel que $\mathcal{B}' \setminus \{u\} \cup \{v\}$ est une base de $E$. On pose $\mathcal{B}'' = \mathcal{B}' \setminus \{u\} \cup \{v\}$ 
			\begin{align*}
				\mathcal{B}'' \cap \mathcal{B} &= \left( (\mathcal{B}' \setminus \{u\})  \cap \mathcal{B} \right) \cup \{v\} \\
				&= (\mathcal{B}' \cap \mathcal{B}) \cup \{v\} \\
			\end{align*}
			donc,
			\begin{align*}
				n - \#(\mathcal{B}'' \cap \mathcal{B}) &= n - (\#(\mathcal{B}' \cap \mathcal{B}) + 1) \\
				&= p+1- 1 \\
				&= p \\
			\end{align*}
			D'après l'hypothèse de récurrence, \[
				\#\mathcal{B}'' = n
			\] Or, $\#\mathcal{B}'' = \#\mathcal{B}'$
	\end{itemize}
\end{prv}

\begin{lem}
	Soient $\mathcal{B}$ et $\mathcal{B}'$ deux bases de $E$ telles que $\mathcal{B}\subset \mathcal{B}'$. Alors, $\mathcal{B} = \mathcal{B}'$.
\end{lem}

\begin{prv}
	On suppose $\mathcal{B}' \neq \mathcal{B}$. Soit $u \in \mathcal{B}' \setminus \mathcal{B}$
	$u \in E = \Vect(\mathcal{B})$ donc $\mathcal{B} \cup \{u\}$ n'est pas libre.
	Donc $\mathcal{B}\cup \{u\} \subset \mathcal{B}'$ et $\mathcal{B}'$ est libre donc $\mathcal{B}\cup \{u\}$ est libre: une contradiction $\lightning$
\end{prv}

\begin{lem}
	[Lemme d'échange] Soient $\mathcal{B}_1$ et $\mathcal{B}_2$ deux bases de $E$ et $u \in \mathcal{B}_1 \setminus \mathcal{B}_2$. Alors, il existe $v \in \mathcal{B}_2$ tel que $(\mathcal{B}_1 \setminus \{u\}) \cup \{v\}$ soit une base de $E$.
\end{lem}

\begin{prv}
	[1${}^\text{nde}$ méthode]
	On suppose que pout tout $v \in \mathcal{B}_2$, $(\mathcal{B}_1\setminus \{u\}) \cup \{v\}$ n'est pas une base de $E$
	Soit $v \in \mathcal{B}_2$.
	\begin{itemize}
		\item Supposons $(\mathcal{B}_1\setminus \{u\})\cup \{v\}$ non libre. $\mathcal{B}_1 \setminus \{u\}$ est libre. Donc $v \in \Vect(\mathcal{B}_1 \setminus \{u\})$
		\item Supposons $(\mathcal{B}_1\setminus \{u\}) \cup \{v\}$ non génératrice.
			Comme $\mathcal{B}_1$ engendre $E$, $u \not\in \Vect(\mathcal{B}_1\setminus \{v\})$.
			On suppose que $\mathcal{B}_1 \neq \mathcal{B}_2$.
			$\forall v \in \mathcal{B}_2 \setminus \mathcal{B}_1, \Vect(\mathcal{B}_1 \setminus \{v\}) = \Vect(\mathcal{B}_1) = E \ni u$ 
			donc, $(\mathcal{B}_1\setminus \{u\}) \cup \{v\}$ engendre $E$ et donc \[
				v \in \Vect(\mathcal{B}_1 \setminus \{u\})
			\] On a aussi \[
				\forall v \in \mathcal{B}_1 \setminus \{u\}, v \in \Vect(\mathcal{B}_1\setminus \{u\})
			\] Comme $u \not\in \mathcal{B}_2$, on a \[
				\forall v \in \mathcal{B}_2, v \in \Vect(\mathcal{B}_1\setminus \{u\})
			\] docn \[
				E = \Vect(\mathcal{B}_2) \subset \Vect(\mathcal{B}_1\setminus \{u\})
			\] donc $\mathcal{B}_1\setminus \{u\}$ engendre $E$ donc $\mathcal{B}_1\setminus \{u\}$ est une base de $E$. Or, $\mathcal{B}_1 \setminus \{u\}  \subset  \mathcal{B}_1$, donc $\mathcal{B}_1\setminus \{u\} = \mathcal{B}_1$
	\end{itemize}
\end{prv}

\begin{prv}
	[2${}^\text{nde}$ méthode]
	On suppose que pout tout $v \in \mathcal{B}_2$, $(\mathcal{B}_1\setminus \{u\}) \cup \{v\}$ n'est pas une base de $E$
	\begin{itemize}
		\item Comme $u \in \mathcal{B}_1 \setminus \mathcal{B}_2$, nécéssairement $\mathcal{B}_1 \neq \mathcal{B}_2$ donc $\mathcal{B}_2 \not\subset \mathcal{B}_1$, donc $\mathcal{B}_2\setminus\mathcal{B}_1 \neq \O$ 
		\item Soit $v \in \mathcal{B}_2\setminus\mathcal{B}_1$. Il existe $(\lambda_w)_{w\in\mathcal{B}_1}$ une famille de scalaires presque nulle telle que \[
				v = \sum_{w \in \mathcal{B}_1} \lambda_w w - \lambda_u u + + \sum_{w \in \mathcal{B}_1\setminus \{u\}}\lambda_w w
			\]
			Si $\lambda_u \neq 0_E$, alors
			\begin{align*}
				u &= \lambda_u^{-1}\left( v - \sum_{w \in \mathcal{B}_1 \setminus \{u\}} \lambda_w w \right)\\
					&\in \Vect(\mathcal{B}_1\setminus \{u\} \cup v)
			\end{align*}
			 donc $\mathcal{B}_1 \subset \Vect(\mathcal{B}_1\setminus \{u\} \cup \{v\})$\\
			 et donc $E \subset  \Vect(\mathcal{B}_1 \setminus \{u\} \cup \{v\})$ \\
			 et donc $\mathcal{B}_1 \setminus \{u\} \cup \{v\}$ engendre $E$ \\
			 donc $\mathcal{B}_1 \setminus \{u\} \cup \{v\}$ n'est pas libre\\
			 donc $v \in \Vect(\mathcal{B}_1\setminus \{u\})$ (car $\mathcal{B}_1 \setminus \{u\}$ est libre\\
			 donc $\lambda_u = 0_\mathbbm{K}$ $\lightning$\\`

			 Donc, $\lambda_u = 0_\mathbbm{K}$, docn $v \in \Vect(\mathcal{B}_1\setminus \{u\})$ \\
			 On vient de prouver que
			 \begin{align*}
			 	\mathcal{B}_2 \setminus \mathcal{B}_1 \subset \Vect(\mathcal{B}_1 \setminus \{u\})\\
			 	\mathcal{B}_1 \setminus \{u\} \subset \Vect(\mathcal{B}_1 \setminus \{u\})\\
			 \end{align*}
			 Comme $u \not\in \mathcal{B}_2$, \[
			 	\mathcal{B}_2 \subset \Vect(\mathcal{B}_1 \setminus \{u\})
			 \] donc \[
			 	E = \Vect(\mathcal{B}_2) \subset  \Vect(\mathcal{B}_1 \setminus \{u\})
			 \] donc $\mathcal{B}_1 \setminus \{u\}$ engendre $E$. Donc,  $\mathcal{B}_1 \setminus \{u\}$ est une base de $E$.\\
			 Or, $\mathcal{B}_1 \setminus \{u\} \subset  \mathcal{B}_1$, donc $\mathcal{B}_1 \setminus \{u\} = \mathcal{B}_1$
	\end{itemize}
\end{prv}

\begin{defn}
	Soit $E$ un $\mathbbm{K}$-espace vectoriel de dimension finie. Le cardinal commun à toutes les bases de $E$ est appelé \underline{dimension} de $E$ est notée $\dim(E)$ ou $\dim_\mathbbm{K}(E)$\\
	C'est donc aussi le nombre de coordonnées de n'importe quel vecteur dans n'importe quelle base.
	\index{dimension (espace vectoriel)}
\end{defn}

\begin{exm}
	\begin{enumerate}
		\item $\dim_\R(\C) = 2$ et $\dim_\C(\C) = 1$ 
		\item $\dim_\mathbbm{K}(\mathbbm{K}^{n}) = n$ 
		\item $\dim_{\mathbbm{K}}(\mathcal{M}_{n,p}(\mathbbm{K})) = np$
	\end{enumerate}
\end{exm}

\begin{crlr}
	Soit $E$ un $\mathbbm{K}$-espace vectoriel de dimension finie, $\mathcal{L}$ une famille libre de $E$, $\mathcal{G}$ une famille génératrice de $E$. On note $n = \dim(E)$
	\begin{enumerate}
		\item $\#\mathcal{G} \ge n$ et $(\#\mathcal{G} = n \implies \mathcal{G} \text{ est une base de } E$)
		\item $\#\mathcal{L} \le n$ et $(\#\mathcal{L} = n \implies \mathcal{L} \text{ est une base de } E$)
	\end{enumerate}
\end{crlr}

\begin{crlr}
	$\R^{\R}$ est de dimension infinie.
	$\forall i \in \N, e_i: x \mapsto x^i$\\
	$(e_i)_{i\in\N}$ est libre dans $\R^\R$
\end{crlr}

\begin{prop}
	Soient $E$ et $F$ deux $\mathbbm{K}$-espaces vectoriels de dimension finie. Alors $E\times F$ est de dimension finie et $\dim(E\times F) = \dim(E) + \dim(F)$
\end{prop}

\begin{prv}
	Soit $(e_1,\ldots, e_n)$ une base de $E$, $(f_1, \ldots, f_p)$ une base de $F$.
	On pose \[
		\left\{\begin{array}
			{r c l}
			u_1 &=& (e_1,0_F)\\
			u_2 &=& (e_2,0_F)\\
					&\vdots&\\
			u_n &=& (e_n,0_F)\\
			u_{n+1} &=& (0_E, f_1)\\
			u_{n+2} &=& (0_E, f_2)\\
					&\vdots&\\
			u_{n+p} &=& (0_E,f_p)\\
		\end{array}\right.
	\]
	Soit $(x,y) \in E\times F$. \[
		\begin{cases}
			\exists (x_1,\ldots,x_n)\in \mathbbm{K}^n, x = \sum_{i=1}^{n} x_ie_i
			\exists (y_1,\ldots,y_n)\in \mathbbm{K}^n, x = \sum_{j=1}^{p} y_jf_j
		\end{cases}
	\] 
	\begin{align*}
		(x,y) &= \left( \sum_{i=1}^{n} x_ie_i, \sum_{i=1}^{p} y_jf_j \right)  \\
		&= \sum_{i=1}^{n} x_i (e_i + 0_F) + \sum_{j=1}^{p} y_j (0_E, f_j) \\
		&= \sum_{i=1}^{n} x_i u_i + \sum_{j=1}^{p} y_j u_{n+j} \\
	\end{align*}
	Donc, $E\times F = \Vect(u_1, \ldots, u_{n+p})$ donc $E\times F$ est de dimension finie.\\
	Soit $(\lambda_1, \ldots, \lambda_{n+p}) \in \mathbbm{K}^{n+p}$ tel que \[
		(*): \quad \sum_{k=1}^{n+p} \lambda_ku_k = 0_{E\times F} = (0_E, 0_F)
	\]
	\begin{align*}
		(*) &\iff \sum_{k=1}^{n} \lambda_k (e_k, 0_F) + \sum_{k=n+1}^{p} \lambda_k(0_E, f_{k-n}) = (0_E, 0_F)\\
				&\iff \begin{cases}
					\sum_{k=1}^{n} \lambda_k e_k = 0_E\\
					\sum_{k=n+1}^{p} \lambda_k f_{k-n} = 0_F
				\end{cases}\\
				&\iff \begin{cases}
					\forall k \in \left\llbracket 1,n \right\rrbracket, \lambda_k = 0_\mathbbm{K} \qquad&(\text{car $(e_1,\ldots,e_n)$ est libre})\\
					\forall k \in \left\llbracket n+1,n+p \right\rrbracket, \lambda_k = 0_\mathbbm{K} \qquad&(\text{car $(f_1,\ldots,f_n)$ est libre})\\
				\end{cases}
	\end{align*}
	Donc $(u_1, \ldots, u_{n+p})$ est une base de $E\times F$. Donc, $\dim(E\times F) = n + p = \dim(E) + \dim(F)$
\end{prv}

\begin{rmk}
	[Convention]
	\[\dim\big(\{0_E\}\big) = 0\]
\end{rmk}

\begin{thm}
	Soit $E$ un $\mathbbm{K}$-espace vectoriel de dimension finie, $F$ un sous-espace vectoriel de $E$. Alors, $F$ est de dimension finie et  $\dim(F) \le \dim(E)$\\
	Si $\dim(F) = \dim(E)$, alors $F = E$
\end{thm}

\begin{prv}
	On considère \[
		A = \{k \in \N \mid \text{il existe une famille libre de $F$ à $k$ éléments}\} 
	\]
	On suppose $F \neq \{0_E\}$.
	\begin{itemize}
		\item Soit $u \in F\setminus \{0_E\}$. $(u)$ est libre donc $1 \in A$ et donc $A \neq \O$
		\item Soit $\mathcal{L}$ une famille libre de $F$. Alors, $\mathcal{L}$ est une famille libre de $E$ \\
			donc $\#\mathcal{L} \le \dim(E)$\\
			Donc $A$ est majorée par $\dim(E)$ \\
			On en déduit que $A$ a un plus grand élément $p$.
		\item Soit $\mathcal{L}$ une famille libre de $F$ avec $p$ éléments.\\
			Si $\mathcal{L}$ n'engendre pas $F$, alors il existe $u\in F$ tel que $u\not\in \Vect(\mathcal{L})$ et donc $\mathcal{L} \cup \{u\}$ est une famille libre de $F$, donc $p+1 \in A$ en contradiction avec la maximalité de $p$.\\
			Donc $\mathcal{L}$ est une base de $F$ donc $F$ est de dimension finie et $\dim(F) = p \le \dim(E)$\\
	\end{itemize}

	Soit $\mathcal{B}$ une base de $F$. Alors, $\mathcal{B}$ est aussi une famille de libre de de $E$. Donc $\#\mathcal{B} \le \dim(E)$ donc $\dim(F) = \dim(E)$ \\
	Si $\dim(F) = \dim(E)$, alors $\mathcal{B}$ est une base de $E$, et donc $F = \Vect(\mathcal{B}) = E$
\end{prv}

\begin{prop}
	[Formule de Grassmann]
	Soit $E$ un $\mathbbm{K}$-espace vectoriel de dimension finie, $F$ et $G$ deux sous-espace vectoriels de $E$. Alors, \[
		\dim(F+G) = \dim(F) + \dim(G) - \dim(F\cap G)
	\] 
\end{prop}

\begin{prv}
	Soit $(e_1, \ldots, e_p)$ une base de $F\cap G$. $(e_1,\ldots,e_p)$ est une famille libre de $F$.\\
	On complète $(e_1, \ldots, e_p)$ en une base $(e_1, \ldots, e_p, u_1, \ldots, u_q)$ de $F$.\\
	De même, on complète $(e_1, \ldots, e_p)$ en une base $(e_1, \ldots, e_p, v_1, \ldots, v_r)$ de $G$.\\
	On pose  $\mathcal{B} = (e_1, \ldots, e_p, u_1, \ldots, u_q, v_1, \ldots, v_r)$. Montrons que $\mathcal{B}$ est une base de $F+G$
	\begin{itemize}
		\item Soit $u \in F+G$ \\
			On pose $u = v+w$ avec $\begin{cases}
				v\in F\\
				w \in G
			\end{cases}$.\\
			On pose $v = \sum_{i=1}^p \lambda_i e_i + \sum_{i=1}^q \mu_i u_i$ avec $(\lambda_1, \ldots, \lambda_p, \mu_1, \ldots, \lambda_q) \in \mathbbm{K}^{p+q}$\\
			On pose aussi $w = \sum_{i = 1}^p \lambda'_ie_i + \sum_{j=1}^r \nu_j v_j$ avec $(\lambda_1',\ldots,\lambda_p', \nu_1, \ldots, \nu_r) \in \mathbbm{K}^{p+r}$\\
			D'où, \[
				u = \sum_{i=1}^p (\lambda_i + \lambda'_i)e_i + \sum_{j=1}^q \mu_j u_j + \sum_{k=1}^r \nu_k v_k \in \Vect(\mathcal{B})
			\]
		\item Soient $(\lambda_1, \ldots, \lambda_p, \mu_1, \ldots, \mu_q, \nu_1, \ldots, \nu_r) \in \mathbbm{K}^{p+q+r}$.\\
			On suppose \[
				(*)\quad \sum_{i=1}^{p}\lambda_ie_i + \sum_{j=1}^q\mu_ju_j + \sum_{k=1}^r \nu_k v_k = 0_E
			\] 
			D'où, \[
				\underbrace{\sum_{i=1}^p\lambda_i e_i + \sum_{j=1}^q \mu_ju_j}_{\in F} = \underbrace{-\sum_{k=1}^r\nu_jv_k}_{\in G}
			\] 
			Donc, \[
				f = \sum_{i=1}^p \lambda_i e_i + \sum_{j=1}^q \mu_j u_j \in F\cap G
			\] Comme $(e_1, \ldots, e_p)$ est une base de $F\cap G$, $\exists ! (\lambda_1', \ldots, \lambda_p') \in \mathbbm{K}^p$ tel que \[
				f = \sum_{i=1}^p \lambda'_i e_i = \sum_{i=1}^p \lambda'_i e_i + \sum_{j=1}^q 0_\mathbbm{K}u_j
			\] Comme $(e_1, \ldots, e_p, u_1, \ldots, u_q)$ est une base de $F$, \[
				\forall k \in \left\llbracket 1, q \right\rrbracket, \mu_j = 0_\mathbbm{K}
			\] De même, \[
				\forall k \in \left\llbracket 1,r \right\rrbracket , \nu_k = 0_\mathbbm{K}
			\] On remplace dans $(*)$ pour trouver \[
				\sum_{i=1}^p \lambda_ie_i = 0_E
			\] Comme $(e_1, \ldots, e_p)$ est libre, \[
				\forall i \in \left\llbracket 1,p \right\rrbracket, \lambda_i = 0_\mathbbm{K}
			\] Donc $\mathcal{B}$ est libre.\\
			Donc, 
			\begin{align*}
				\dim(F+G) &=  p +q + r \\
				&= (p+q)+ (p+r) - p \\
				&= \dim(F) + \dim(G) - \dim(F\cap G) \\
			\end{align*}
	\end{itemize}
\end{prv}

\begin{crlr}
	Avec les hypothèse précédentes, \[
		E = F \oplus G \iff \begin{cases}
			F \cap  G = \{0_E\} \\
			\dim(E) = \dim(F) + \dim(G)
		\end{cases}
	\] 
\end{crlr}

\begin{prv}
	\begin{itemize}
		\item[``$\implies$''] On suppose $E = F \oplus G$ \\
			Comme la somme est directe, $F \cap G = \{0_E\}$ 
			\begin{align*}
				\dim(E) &= \dim(F)\\
				&= \dim(F) + \dim(G) - \dim(F\cap G)\\
				&= \dim(F) + \dim(G)\\
			\end{align*}
		\item[``$\impliedby$''] On suppose $F\cap G = \{0_E\}$ et $\dim(E) = \dim(F) + \dim(G)$.\\
			On sait déjà que $F+G = F \oplus G$\\
			 \begin{align*}
				\dim(F+G) = \dim(F) + \dim(G) - \dim(F \cap G) = \dim(E)
			\end{align*}
			Donc $F + G = E$
	\end{itemize}
\end{prv}

\begin{prop}
	Soit $F$ un $\mathbbm{K}$-espace vectoriel de dimension finie $n$. Soit $\mathcal{B} = (e_1, \ldots, e_n)$ une base de $F$. L'application
	\begin{align*}
		f: \mathbbm{K}^n &\longrightarrow F \\
		(\lambda_1, \ldots, \lambda_n) &\longmapsto \sum_{i=1}^n \lambda_i e_i
	\end{align*} est bijective.\\
	Si $\mathbbm{K}$ est infini, $\mathbbm{K}^n$ aussi et donc $F$ aussi.\\
	Si $\#\mathbbm{K} = p \in \N_*$,
	\begin{align*}
		\#&\mathbbm{K}^n = p^n\\
		&\vrt=\\
		\#&F
	\end{align*}
\end{prop}


		\part{Dérivation}

\underline{Motivation}:

{
\begin{wrapfigure}{l}{3cm}
	\centering
	\begin{asy}
		import three;

		size(3cm);
		settings.render=0;
		settings.prc=false;
		currentprojection = obliqueZ;

		draw(unitbox);
		draw(shift(1.1Z + 0.05X) * (O -- X), Arrows3(TeXHead2));
		draw(shift(1.1Z + 0.05Y) * (O -- Y), Arrows3(TeXHead2));
		draw(shift(1.1X + 0.05Z) * (O -- Z), Arrows3(TeXHead2));

		label("$x$", (X/2) + (1.1Z + 0.05X), align=S);
		label("$y$", (Y/2) + (1.1Z + 0.05Y), align=W);
		label("$z$", (Z/2) + X, align=SE);
	\end{asy}
\end{wrapfigure}

\begin{align*}
	&S(x,y,z) = 2(xy + xz + yz)\\
	&V(x,y,z) = xyz
\end{align*}

On cherche à minimiser $S$ avec la contrainte $V = 1$.

Soit $f : \begin{array}{rcl}
	\left( \R_*^+ \right)^2 &\longrightarrow& \R \\
	(x,y) &\longmapsto& S\left( x,y,\frac{1}{xy} \right) = 2\left( xy + \frac{1}{y} + \frac{1}{x} \right).
\end{array}$

On cherche $(a,b) \in \left( \R^+_* \right)^2$ tel que \[
	\forall (x,y) \in (\R^+_*), f(x,y) \ge f(a,b).
\]
}

\begin{defn}
	Soit $f: U \to \R$ où $U$ est un ouvert de $\R^2$. Soit $(a,b) \in U$.
	\vspace{2mm}

	Si $\lim_{x \to a} \frac{f(x,b) - f(a,b)}{x - a} \in \R$, alors on dit que $f$ a une dérivée partielle suivant $x$ en $(a,b)$ et cette limite est notée \[
		\partial f_1(a,b) = \frac{\partial f}{\partial x}(a,b).
	\]

	Si $\lim_{y \to b} \frac{f(a,y) - f(a,b)}{y - b} \in \R$, alors on dit que $f$ a une dérivée partielle suivant $y$ et la limite est notée \[
		\partial f_2(a,b) = \frac{\partial f}{\partial y}(a,b).
	\]
\end{defn}

\begin{exm}
	\begin{enumerate}
		\item $f: (x,y) \mapsto xy + x - y$.

			\begin{align*}
				&\frac{\partial f}{\partial x} : (x,y) \mapsto y + 1,\\
				&\frac{\partial f}{\partial y} : (x,y) \mapsto x - 1.
			\end{align*}

		\item $f: (x,y) \mapsto xy + \frac{1}{y}+ \frac{1}{x}$.

			\begin{align*}
				&\frac{\partial f}{\partial x}: (x,y) \mapsto y - \frac{1}{x^2},\\
				&\frac{\partial f}{\partial y}: (x,y) \mapsto x - \frac{1}{y^2}.
			\end{align*}

		\item Trouver $f$ telle que $\begin{cases}
				(1): \qquad \frac{\partial f}{\partial x}=y,\\[2mm]
				(2): \qquad \frac{\partial f}{\partial y} = x.
			\end{cases}$

			D'après $(1)$ : \[
				\forall (x,y), \exists C(y) \in \R, f(x,y) = xy + C(y)
			\] et donc \[
				\frac{\partial f}{\partial y}(x,y) = x + C'(y)
			\] donc $C'(y) = 0$ et donc $C$ est constante.

		\item Trouver $f$ telle que $\begin{cases}
			\frac{\partial f}{\partial x} = -y,\\[2mm]
			\frac{\partial f}{ƒ\partial y} = x.
		\end{cases}$

		Ce n'est pas possible !
	\end{enumerate}
\end{exm}

\begin{defn}~\\
	\begin{minipage}{\linewidth}
		\begin{wrapfigure}{r}{4cm}
			\centering
			\vspace{-5mm}
			\begin{asy}
				import three;
				import graph3;
				size(4cm);

				settings.render = 0;
				settings.prc = false;
				currentprojection = obliqueX;

				draw(O -- X, Arrow3(TeXHead2));
				draw(O -- Y, Arrow3(TeXHead2));
				draw(O -- Z, Arrow3(TeXHead2));

				triple f(real x, real y, real z = 0) { return (x,y,cos(x - 0.5) * cos(y - 0.5)/1.2 + 0.15); }

				real inc = 1 / 5;

				for(real x = 0; x <= 1; x += inc) {
					draw(graph(
						new real(real t) { return x; }, // x
						new real(real y) { return y; }, // y
						new real(real y) { return f(x,y).z; }, // z
						0, 1
					), gray);
				}

				for(real y = 0; y <= 1; y += inc) {
					draw(graph(
						new real(real x) { return x; }, // x
						new real(real t) { return y; }, // y
						new real(real x) { return f(x,y).z; }, // z
						0, 1
					), gray);
				}

				path3 path1 = (0.8, 0.2, 0) .. (0.5, 0.5, 0) .. (0.3, 0.7, 0);
				path3 path2 = f(0.8, 0.2, 0) .. f(0.5, 0.5, 0) .. f(0.3, 0.7, 0);
				path3 d = (0.2, 0.3, 0) .. (0.3, 0.4, 0) .. (0.2, 0.7, 0) .. (0.8, 0.9, 0) .. (0.6, 0.2, 0) .. cycle;

				draw(path1, red, Arrow3(TeXHead2));
				draw(path2, red, Arrow3(TeXHead2, position=0.8));

				dot((0.5, 0.5, 0));
				dot(f(0.5, 0.5, 0));
				draw((0.5, 0.5, 0) -- f(0.5, 0.5, 0), dashed);
				draw(d);

				label("$w$", (0.3, 0.7, 0), red, align=SE);
				label("$U$", (0.8, 0.9, 0), align=SE);
			\end{asy}
		\end{wrapfigure}

		Soit $f: U \to \R$ où $U$ est un ouvert. Soit $(a,b) \in U$. Soit $w = (w_1, w_2) \in \R^2$.

		Si 
		\[
			\lim_{t\to 0} \frac{f(a + tw_1, b + tw_2) - f(a,b)}{t}
		\] existe et est réelle, alors on dit que $f$ a une dérivée dans la direction de $w$ et la limite est notée \[
			\mathrm{d}f(w)\,(a,b) = D_w(f)\,(a,b).
		\]
	\end{minipage}
\end{defn}

\begin{exm}
	\begin{align*}
		f: \left( \R_*^+ \right)^2 &\longrightarrow \R \\
		(x,y) &\longmapsto xy+\frac{1}{x}+\frac{1}{y}.
	\end{align*}

	On pose $(a,b) = (1,2)$, $w = (w_1, w_2) = (1,1)$.
	\begin{align*}
		\frac{f(1+t, 2+t) - f(1,2)}{t} &= \frac{1}{t} \left( (1+t)(2+t) + \frac{1}{1+t} + \frac{1}{2+t} - 3 - \frac{1}{2} \right) \\
		&= \frac{1}{t}\left(\cancel 2 + 3t + \po(t) + \cancel 1 - t + \po(t) + \frac{1}{2}\left( \cancel 1 - \frac{t}{2} + \po(t) \right) - \cancel3 - \cancel{\frac{1}{2}} \right) \\
		&= \frac{1}{t} \left( \frac{7}{4} t + \po(t) \right)  \\
		&= \frac{7}{4} + \po(1) \tendsto{t \to 0} \frac{7}{4}. \\
	\end{align*}

	Donc, \[
		\mathrm{d}f(1,1)\,(1,2) = \frac{7}{4}.
	\]
\end{exm}

\begin{rmk}~\\
	\begin{figure}[H]
		\centering
		\begin{asy}
			import solids;
			import graph;
			size(5cm);

			settings.render = 0;
			settings.prc = false;

			path3 par = graph(
				new real(real x) { return x; },
				new real(real x) { return 0; },
				new real(real x) { return x^2; },
				0,3);
			revolution r = revolution(par, axis=Z);

			path3 par2 = graph(
				new real(real x) { return x; },
				new real(real x) { return 0; },
				new real(real x) { return x^2; },
				-3,3);

			draw(r,1,longitudinalpen=nullpen);
			draw(r.silhouette());

			draw((-4, 0, -1) -- (-4, 0, 10) -- (4, 0, 10) -- (4, 0, -1) -- cycle, red);
			draw(par2, deepred);

			draw((4,4.5) -- (7, 4.5), black+0.5mm, Arrow(TeXHead));

			path par2d = graph(new real(real x) { return x^2; }, -3, 3);
			draw(shift((11, 0)) * par2d, deepred);

			dot(O);
			dot((11, 0));
		\end{asy}
	\end{figure}
\end{rmk}


%todo ajouter théorème-définition
\begin{thm}
	Soit $f : U \to \R$, $(a,b) \in U$. On suppose que $\frac{\partial f}{\partial x}$ et $\frac{\partial f}{\partial y}$ existent en $(a,b)$ et sont {\bfseries continues} en $(a,b)$. Alors,
	\begin{align*}
		&\forall (h, k) \in \R^2 \text{ tel que } (a +h, b + k) \in U,\\
		&f(a+ h, b + k) = f(a,b) + h \frac{\partial f}{\partial x}(a,b) + k \frac{\partial f}{\partial y}(a,b) + \po_{(h,k)\to (0,0)}\big(\|(h,k)\|\big).
	\end{align*}

	On dit que $f$ est de classe $\mathcal{C}^1$ si $\frac{\partial f}{\partial x}$ et $\frac{\partial f}{\partial y}$ existent et sont continues.

	\qed
\end{thm}

\begin{rmk}
	En physique, cette formule correspond à : \[
		\mathrm{d}f = \frac{\partial f}{\partial x}\mathrm{d}x + \frac{\partial f}{\partial y} \mathrm{d}y.
	\] En effet :
	\begin{align*}
		\mathrm{d}f &= f(x+ \mathrm{d}x, y + \mathrm{d}y) - f(x,y) \\
		&= \frac{\partial f}{\partial x} \mathrm{d}x + \frac{\partial f}{\partial y} \mathrm{d}y.
	\end{align*}
\end{rmk}

\begin{prop}
	Soit $f: U \to \R$ de classe $\mathcal{C}^1$ en $(a,b) \in U$. Alors, \[
		\forall w = (w_1, w_2) \in \R^2, \mathrm{d}f(w)\,(a,b) = w_1 \frac{\partial f}{\partial x}(a,b) + w_2 \frac{\partial f}{\partial y}(a,b).
	\]
\end{prop}

\begin{prv}
	Soit $w = (w_1, w_2) \in \R^2$. Soit $t \in \R^*$.
	\begin{align*}
		\frac{1}{t}\big(f(a + tw_1, b + tw_2) - f(a,b)\big)
		&= \frac{1}{t} \left( tw_1 \frac{\partial f}{\partial x}(a,b) + tw_2 \frac{\partial f}{\partial y}(a,b) + \po_{t \to 0}\big(\|tw\|\big) \right) \\
		&= w_1 \frac{\partial f}{\partial x}(a,b) + w_2 \frac{\partial f}{\partial y}(a,b) + \po_{t\to 0}(1) \\
		&\tendsto{t\to 0} w_1 \frac{\partial f}{\partial x}(a,b) + w_2\frac{\partial f}{\partial y}(a,b).
	\end{align*}
\end{prv}


\begin{defn}
	Avec les hypothèses précédentes, en posant \[
		\nabla f(a,b) = \left( \frac{\partial f}{\partial x}(a,b), \frac{\partial f}{\partial y}(a,b) \right) 
	\]on obtient \[
		\mathrm{d}f(w)\,(a,b) = \left<w  \mid \nabla f(a,b) \right>
	\] où $\left<\cdot|\cdot \right>$ est le produit scalaire.

	Le vecteur $\nabla f(a,b)$ est appelé \underline{gradient de $f$ en $(a,b)$}.

	Le développement limité à l'ordre 1 de $f$ devient \[
		f\big((a,b)+w\big) = f(a,b) + \left<w \mid \nabla f(a,b) \right> + \po_{w\to 0}(\|w\|)
	\]
\end{defn}

\begin{prop}
	Soit $f : U \to \R$ de classe $\mathcal{C}^1$.

	\begin{figure}[H]
    \centering
    \incfig{gradient}
	\end{figure}

	$\nabla f$ est orthogonal au lignes de niveaux de $f$, son orientation va dans le sens d'une augmentation de $f$.
\end{prop}

\begin{prv}
	Soit $\gamma : I \to U$ une courbe de niveau : \[
		\forall t \in I, f\big(\gamma(t)\big) = \text{cste}.
	\] D'après le lemme suivant : \[
		\forall t \in I, 0 = (f \circ \gamma)'(t) = \mathrm{d}f\big(\gamma'(t)\big)\big(\gamma(t)\big) = \left<\gamma'(t)  \mid \nabla f\big(\gamma(t)\big) \right>
	\] Donc $\nabla f\big(\gamma(t)\big)$ est orthogonal à $\gamma'(t)$.

	Pour tout $t \in I$, on pose $w(t) = t\, \nabla f\big(\gamma(t)\big)$. Donc \[
		f\big(\gamma(t) + w(t)\big) = f\big(\gamma(t)\big) + t \|\nabla f(\gamma(t))\|^2 + \po_{t \to 0}(t)
	\] Pour $t$ assez petit, $f\big(\gamma(t) + w(t)\big) - f\big(\gamma(t)\big)$ est du même signe que $t$.
\end{prv}

\begin{rmk}
	\begin{align*}
		V: \R^3 &\longrightarrow \R \\
		(x,y,z) &\longmapsto -mgz
	\end{align*}
	l'énerge potentielle de pesenteur

	On a donc \[
		\nabla V(x,y,z) = \left( \frac{\partial V}{\partial x}, \frac{\partial V}{\partial y}, \frac{\partial V}{\partial z} \right) = (0, 0, -mg) = \vec{P}.
	\]
\end{rmk}

\begin{lem}
	Soit $f : U \to \R$ de classe $\mathcal{C}^1$, $\gamma : \begin{array}{rcl}
		I &\longrightarrow& U \\
		t &\longmapsto& \big(x(t), y(t)\big)
	\end{array}$ où $x$ et $y$ sont dérivables.

	On pose \[
		\forall t \in I, \gamma'(t) = \big(x'(t), y'(t)\big).
	\] Alors $f \circ \gamma : I \to \R$ est dérivable et
	\begin{align*}
		\forall t \in I, (f \circ \gamma)'(t) &= \mathrm{d}f\big(\gamma'(t)\big) \big(\gamma(t)\big)\\
		&= \left<\gamma'(t)  \mid \nabla f\big(\gamma(t)\big)  \right> \\
		&= x'(t) \frac{\partial f}{\partial x}\big(x(t), y(t)\big) + y'(t) \frac{\partial f}{\partial y}\big(x(t),y(t)\big). \\
	\end{align*}
\end{lem}

\begin{prv}
	On fixe $t \in I$.

	\begin{align*}
		\forall h \neq 0, \frac{f \circ \gamma(t + h) - f \circ \gamma(t)}{h}
		&= \frac{1}{h}\big(f(\gamma(t)) + h\gamma'(t) + \po_{h\to 0}(h) - f(\gamma(t))\big) \\
		&= \frac{1}{h}\bigg(\cancel{f(\gamma(t))} + \left<h\gamma'(t) \mid \nabla f(\gamma(t)) \right> + \po_{h\to 0}(\|h\gamma'(t)\|) - \cancel{f(\gamma(t))}\bigg)\\
		&= \left<\gamma'(t) \mid \nabla f(\gamma(t)) \right> + \po_{h\to 0}(1) \\
		&\tendsto{h\to 0} \left<\gamma'(t)  \mid \nabla f(\gamma(t)) \right>
	\end{align*}
\end{prv}

\begin{defn}
	Soit $f : U \to \R$ de classe $\mathcal{C}^1$ et $(a,b) \in U$. On dit que $(a,b)$ est un \underline{point critique} de $f$ si $\nabla f(a,b) = 0$ i.e. $\frac{\partial f}{\partial x}(a,b) = \frac{\partial f}{\partial y}(a,b) = 0$.

	Dans ce cas, $f(a,b)$ est appelé \underline{valeur critique} de $f$.
\end{defn}

\begin{prop}~\\
	\begin{minipage}{\linewidth}
		\begin{wrapfigure}{r}{3cm}
			\centering
			\vspace{-1cm}
			\begin{asy}
				import solids;
				import graph;
				size(3cm);

				settings.render = 0;
				settings.prc = false;

				path3 par = graph(
					new real(real x) { return x; },
					new real(real x) { return 0; },
					new real(real x) { return -x^2; },
					0,3);
				revolution r = revolution(par, axis=Z);

				draw(r,1,longitudinalpen=nullpen);
				draw(r.silhouette());

				dot("$(a,b)$", O, red, align=N);
				real s = sqrt(2.5);
				path3 g=(s,0,-2.5)..(0,s,-2.5)..(-s,0,-2.5)..(0,-s,-2.5)..cycle;
				draw(g, deepcyan);
			\end{asy}
		\end{wrapfigure}
		Soit $f: U \to \R$ de classe $\mathcal{C}^1$ et $(a,b) \in U$ tel que \[
			\exists r > 0, \forall (x,y) \in B_{(a,b)}(r), f(x,y) \le f(a,b)
		\] Alors $\nabla f(a,b) = (0,0)$.
	\end{minipage}
\end{prop}

\begin{prv}
	Soit $g: x \mapsto f(x,b)$. $g(a)$ est un maximum local de $g$ donc $g'(a) = 0$.

	Or, $g'(a) = \frac{\partial f}{\partial x}(a,b)$

	donc $\frac{\partial f}{\partial x}(a,b) = 0$.

	Soit $h : y \mapsto f(a,y)$. On a de même $h'(b) = 0$.

	Or, $h'(b) = \frac{\partial f}{\partial y}(a,b)$.

	Donc, $\nabla f(a,b) = (0,0)$.
\end{prv}

\begin{rmk}
	Un minimum local est aussi une valeur critique.
\end{rmk}

\begin{figure}[H]
	\centering
	\begin{subfigure}{3cm}
		\centering
		\begin{asy}
			import solids;
			import graph;
			size(3cm);

			settings.render = 0;
			settings.prc = false;

			path3 par = graph(
				new real(real x) { return x; },
				new real(real x) { return 0; },
				new real(real x) { return -x^2; },
				0,3);
			revolution r = revolution(par, axis=Z);

			draw(r,1,longitudinalpen=nullpen);
			draw(r.silhouette());

			dot(O, red);
		\end{asy}
		\caption{Maximum local}
	\end{subfigure}
	\begin{subfigure}{3cm}
		\centering
		\begin{asy}
			import solids;
			import graph;
			size(3cm);

			settings.render = 0;
			settings.prc = false;

			path3 par = graph(
				new real(real x) { return x; },
				new real(real x) { return 0; },
				new real(real x) { return x^2; },
				0,3);
			revolution r = revolution(par, axis=Z);

			draw(r,1,longitudinalpen=nullpen);
			draw(r.silhouette());

			dot(O, red);
		\end{asy}
		\caption{Minimum local}
	\end{subfigure}
	\begin{subfigure}{3cm}
		\centering
		\begin{asy}
			import solids;
			import graph;
			size(3cm);

			settings.render = 0;
			settings.prc = false;
			currentprojection = obliqueZ;

			draw(graph(
				new real(real x) { return x; },
				new real(real x) { return -x^2 / 3; },
				new real(real x) { return 3; },
				-3, 3
			));

			draw(graph(
				new real(real x) { return x; },
				new real(real x) { return -x^2 / 3; },
				new real(real x) { return -3; },
				-3, 3
			));

			draw(graph(
				new real(real x) { return x; },
				new real(real x) { return -x^2 / 3 - 1; },
				new real(real x) { return 0; },
				-3, 3
			));

			draw(graph(
				new real(real x) { return 0; },
				new real(real x) { return x^2 / 9 - 1; },
				new real(real x) { return x; },
				-3, 3
			));

			draw(graph(
				new real(real x) { return -3; },
				new real(real x) { return x^2 / 9 - 4; },
				new real(real x) { return x; },
				-3, 3
			));

			draw(graph(
				new real(real x) { return 3; },
				new real(real x) { return x^2 / 9 - 4; },
				new real(real x) { return x; },
				-3, 3
			));

			dot((0,-1,0), red);
		\end{asy}
		\caption{Point de selle / Point col}
	\end{subfigure}
\end{figure}

\begin{exm}
	On revient à l'exemple donné en introduction : 
	\begin{align*}
		f: \left( \R^*_+ \right)^2 &\longrightarrow \R \\
		(x,y) &\longmapsto 2\left( xy + \frac{1}{x} + \frac{1}{y} \right).
	\end{align*}

	$\left( \R^+_* \right)^2$ est un ouvert de $\R^2$. Soit $(x,y) \in \left( \R^+_* \right)^2$.
	
	On a \[
		\begin{cases}
			\frac{\partial f}{\partial x}(x,y) = 2\left( y - \frac{1}{x^2} \right),\\
			\frac{\partial f}{\partial y}(x,y) = 2\left( x - \frac{1}{y^2} \right).
		\end{cases}
	\]

	\begin{align*}
		&\frac{\partial f}{\partial x}(x,y) = \frac{\partial f}{\partial y}(x,y) = 0\\
		\iff& \begin{cases}
			y = \frac{1}{x^2}\\
			x = \frac{1}{y^2}
		\end{cases}\\
		\iff& \begin{cases}
			y = \frac{1}{x^2}\\
			x = x^4
		\end{cases}\\
		\iff& \begin{cases}
			x = 1\\
			y = 1
		\end{cases}
	\end{align*}

	On vérivie que $f$ présente en effet un minium local en $(1,1)$. \[
		f(1,1) = 6
	\] On fixe $y \in \R^+_*$ et \[
		g : x \mapsto 2\left( xy + \frac{1}{x} + \frac{1}{y} \right).
	\] Donc \[
		\forall x \in \R^+_*, g'(x) = 2\left( y - \frac{1}{x^2} \right).
	\]
	\begin{center}
		\begin{tikzpicture}
			\tkzTabInit{$x$/1,$g'(x)$/1,$g$/2.3}{$0$, $\frac{1}{\sqrt{y}}$, $+\infty$}
			\tkzTabLine{,-,z,+,}
			\tkzTabVar{+/{}, -/$2\left( 2\sqrt{y} +\frac{1}{y} \right)$, +/{}}
		\end{tikzpicture}
	\end{center}
	
	Ainsi, \[
		\forall x \in \R^+_*, \forall y \in \R^+_*, f(x,y) \ge 2\left( 2\sqrt{y} + \frac{1}{y} \right)
	\] Soit $h : y \mapsto 2\sqrt{y} + \frac{1}{y}$. On a \[
		\forall y > 0, h'(y) = \frac{1}{\sqrt{y}} - \frac{1}{y^2} = \frac{y\sqrt{y} - 1}{y^2} = \frac{y^{\frac{3}{2}} - 1}{y^2}
	\]

	\begin{center}
		\begin{tikzpicture}
			\tkzTabInit{$y$/0.7,$h'(y)$/0.7,$h$/1.4}{$0$, $1$, $+\infty$}
			\tkzTabLine{,-,z,+,}
			\tkzTabVar{+/{}, -/$3$, +/{}}
		\end{tikzpicture}
	\end{center}

	Donc, \[
		\forall x,y > 0, f(x,y) \ge 2\times 3 = 6 = f(1,1).
	\]
\end{exm}

\begin{prop}
	[règle de la chaîne]

	Soit $f : \begin{array}{rcl}
		U &\longrightarrow& \R^2 \\
		(x,y) &\longmapsto& f(x,y)
	\end{array}$ de classe $\mathcal{C}^1$ et $U, V$ deux ouverts de $\R^2$.

	Soit $\varphi : \begin{array}{rcl}
		V &\longrightarrow& U \\
		(u,v) &\longmapsto& \varphi(u,v) = \big(x(u,v), y(u,v)\big)
	\end{array}$.

	On suppose que $x$ et $y$ sont de classe $\mathcal{C}^1$ sur $V$.

	Alors,  $f \circ \varphi : \begin{array}{rcl}
		V &\longrightarrow& \R \\
		(u,v) &\longmapsto& f\big(\varphi(u,v)\big)
	\end{array}$ est de classe $\mathcal{C}^1$ et
	\begin{align*}
		\forall (u_0, v_0) \in V, \frac{\partial (f \circ \varphi)}{\partial u}(u_0, v_0)
		&= \frac{\partial f}{\partial x}\big(\varphi(u_0, v_0)\big) \times \frac{\partial x}{\partial u}(u_0, v_0)\\
		&+ \frac{\partial f}{\partial y}\big(\varphi(u_0,v_0)\big) \frac{\partial y}{\partial u}(u_0,v_0)
	\end{align*}
	\begin{align*}
		\forall (u_0, v_0) \in V, \frac{\partial (f \circ \varphi)}{\partial v}(u_0, v_0)
		&= \frac{\partial f}{\partial x}\big(\varphi(u_0, v_0)\big) \times \frac{\partial x}{\partial v}(u_0, v_0)\\
		&+ \frac{\partial f}{\partial y}\big(\varphi(u_0,v_0)\big) \frac{\partial y}{\partial v}(u_0,v_0)
	\end{align*}
\end{prop}

\begin{exm}
	[changement de coordonnées polaires]
	On pose \begin{align*}
		\varphi: \R^+_* \times ]0,2\pi[ &\longrightarrow \R^2\setminus \left( R^+_* \times \{0\} \right) \\
		(r, \theta) &\longmapsto (r \cos \theta, r \sin\theta),
	\end{align*}
	\begin{align*}
		f: \R^2\setminus \left( R^+_* \times \{0\} \right) &\longrightarrow \R \\
		(x,y) &\longmapsto f(x,y),
	\end{align*}
	\begin{align*}
		g: \overbrace{\R^+_* \times ]0, 2\pi[}^{=V} &\longrightarrow \R \\
		(r, \theta) &\longmapsto f(r\cos\theta, r\sin\theta).
	\end{align*}

	\begin{align*}
		\forall (r_0,\theta_0) \in V,&\\[5mm]
		\frac{\partial g}{\partial r}(r_0, \theta_0) &= \frac{\partial f}{\partial x}(r_0\cos\theta_0, r_0\sin\theta_0)\cos\theta_0\\
		&+ \frac{\partial f}{\partial y}(r_0 \cos\theta_0, r_0\sin\theta_0)\sin\theta_0\\
		&= 2r_0\cos^2\theta_0 + 2r_0\sin^2(\theta_0) \\
		&= 2r_0 \\[5mm]
		\frac{\partial g}{\partial \theta}(r_0, \theta_0) &= \frac{\partial f}{\partial x}(r_0\cos\theta_0, r_0\sin\theta_0)r_0\sin\theta_0\\
		&+ \frac{\partial f}{\partial y}(r_0 \cos\theta_0, r_0\sin\theta_0)r_0\cos\theta_0\\
		&= -2{r_0}^2\cos(\theta_0)\sin(\theta_0) + 2{r_0}^2 \sin(\theta_0)\cos(\theta_0)\\
		&= 0 \\
	\end{align*}

	Donc, \[
		g(r, \theta) = r^2.
	\]
\end{exm}

\begin{exm}
	Résoudre \[
		\begin{cases}
			\frac{\partial f}{\partial x} = \frac{x}{x^2+y^2},\\
			\frac{\partial f}{\partial y} = \frac{y}{x^2+y^2}.\\
		\end{cases}
	\]

	On pose $g: (r, \theta) \mapsto f(r \cos\theta, r \sin\theta)$.

	\begin{align*}
		&\frac{\partial g}{\partial r} = \frac{1}{r}\cos^2\theta + \frac{1}{r}\sin^2\theta = \frac{1}{r},\\
		&\frac{\partial g}{\partial \theta} = -\cos(\theta) \sin(\theta) + \sin(\theta)\cos(\theta) = 0.
	\end{align*}

	Donc, \[
		\exists C \in \R, g: (r, \theta) \mapsto \ln r + C
	\] d'où,
	\begin{align*}
		\forall (x,y) \in \R^2 \setminus \{(0,0)\}, f(x,y) &= \ln\left(\sqrt{x^2 + y^2} \right)  + C\\
		&= \frac{1}{2}\ln(x^2 + y^2) + C. \\
	\end{align*}
\end{exm}

\begin{rmk}
	Soit $\mathcal{B} = (e_1, e_2)$ la base canonique de $\R^2$, $f: U \to \R$ de classe $\mathcal{C}^1$ avec $U$ un ouvert de $\R^2$.

	Soit $(x,y) \in U$.

	\begin{align*}
		\Mat_{\mathcal{B}}\big(\nabla f(x,y)\big) = \begin{pmatrix}
			\frac{\partial f}{\partial x}(x,y)\\[2mm]
			\frac{\partial f}{\partial y}(x,y)
		\end{pmatrix}
	\end{align*}

	Soit  \begin{align*}
		\varphi: V &\longrightarrow U \\
		(u,v) &\longmapsto \big(x(u,v), y(u,v)\big) 
	\end{align*} avec $x,y$ de classe $\mathcal{C}^1$. Soit $g = f \circ \varphi$.
	\begin{align*}
		\Mat_{\mathcal{B}}\big(\nabla g(u,v)\big)
		&= \begin{pmatrix}
			\frac{\partial g}{\partial u}(u,v) \\[2mm]
			\frac{\partial g}{\partial v}(u,v)
		\end{pmatrix} \\
		&= \begin{pmatrix}
			\frac{\partial x}{\partial u}(u,v) \frac{\partial f}{\partial x}(x,y)
			+ \frac{\partial y}{\partial u}(u,v)\frac{\partial f}{\partial y}(x,y)\\[3mm]
			\frac{\partial x}{\partial v}(u,v) \frac{\partial f}{\partial x}(x,y)
			+ \frac{\partial y}{\partial v}(u,v) \frac{\partial f}{\partial y}(x,y)
		\end{pmatrix}  \\
		&= \underbrace{\begin{pmatrix}
				\frac{\partial x}{\partial u}(u,v)& \frac{\partial y}{\partial u}(u,v)\\[3mm]
				\frac{\partial x}{\partial v}(u,v)& \frac{\partial y}{\partial v}(u,v)
		\end{pmatrix}}_{J(u,v)} \begin{pmatrix}
			\frac{\partial f}{\partial x}(x,y)\\[3mm]
			\frac{\partial f}{\partial y}(x,y)
		\end{pmatrix} \\
		&= J(u,v) \Mat_{\mathcal{B}}\big(\nabla f(x,y)\big) \\
	\end{align*}
	où $J(u,v) = 
	\begin{pNiceArray}{c:c}
		\Mat_{\mathcal{B}}\big(\nabla x(u,v)\big) & \Mat_{\mathcal{B}}\big(\nabla y(u,v)\big)
	\end{pNiceArray}$.

	On dit que $J(u,v)$ est \underline{la jacobienne} de $\varphi$ en $(u,v)$.
	L'application linéaire canoniquement associée à $J(u,v)$ est la \underline{différentielle de $\varphi$} en $(u,v)$ noté $\mathrm{d}\varphi(u,v)$.

	On a $\mathrm{d}\varphi(u,v) \in \mathcal{L}(R^2)$ et $\Mat_{\mathcal{B}}\big(\mathrm{d}\varphi(u,v)\big) = J(u,v)$.

	Par exemple, la jacobienne du changement de coordonnées polaires est \[
		J = \begin{pmatrix}
			\frac{\partial x}{\partial r} & \frac{\partial y}{\partial r}\\[3mm]
			\frac{\partial x}{\partial \theta} & \frac{\partial y}{\partial \theta}
		\end{pmatrix}
		= \begin{pmatrix}
			\cos\theta&\sin\theta\\
			-r\sin\theta&r\cos\theta
		\end{pmatrix}.
	\]
	$\underbrace{\det(J)}_{\text{le jacobien}} = r\cos^2\theta + r\sin^2\theta = r$

	Dans une intégrale double, si $(x,y) = \varphi(u,v)$, alors $\mathrm{d}x\mathrm{d}y = \det(J)\mathrm{d}u\mathrm{d}v$.

	Ici, \[
		\mathrm{d}x\ \mathrm{d}y = r\ \mathrm{d}r\ \mathrm{d}\theta.
	\]
\end{rmk}

\begin{prv}
	On pose $(x_0, y_0) = \varphi(u_0, v_0)$. Pour tout $(h,k) \in \R^2$ tels que $(u_0 + h, v_0 + k) \in V$, en posant $g = f  \circ \varphi$.

	\begin{align*}
		g(u_0 + h, v_0 + h) &= f\big(x(u_0 + h, v_0 + k), y(u_0 + h, v_0 + k)\big) \\
		&= f\left(
			x(u_0,v_0) + h \frac{\partial x}{\partial u}(u_0,v_0) + k \frac{\partial x}{\partial v}(u_0, v_0) + \po\big(\|(h,k)\|\big), \right.\\
		&\phantom{ = f\bigg(\bigg.}\left. y(u_0, v_0) + h \frac{\partial y}{\partial u}(u_0, v_0) + k \frac{\partial y}{\partial v}(u_0, v_0) + \po\big(\|(h,k)\|\big)
		\right)  \\
		&= f(x_0,y_0) \\
		&~+ \left( h \frac{\partial x}{\partial u}(u_0,v_0) + k \frac{\partial x}{\partial v}(u_0, v_0) + \po(\|(h,k)\|) \right) \frac{\partial f}{\partial x}(x_0,y_0)\\
		&~+ \left( h \frac{\partial y}{\partial u}(u_0, v_0) + k\frac{\partial y}{\partial v}(u_0, v_0) + \po(\|(h,k)\|) \right) \frac{\partial f}{\partial y}(x_0, y_0)\\
		&~+ \po(\|(h,k)\|)\\
		&= f(x_0, y_0) \\
		&~+ h \left( \frac{\partial x}{\partial u}(u_0, v_0) \frac{\partial f}{\partial x}(x_0, y_0) + \frac{\partial y}{\partial u}(u_0, v_0) \frac{\partial f}{\partial y}(x_0, y_0) \right)  \\
		&~+ k\left( \frac{\partial x}{\partial v}(u_0, v_0) \frac{\partial f}{\partial x}(x_0, y_0) + \frac{\partial y}{\partial v}(u_0, v_0) \frac{\partial f}{\partial y}(x_0, y_0) \right) 
		&~+ \po(\|(h,k)\|)\\
		&= g(u_0, v_0) + h \frac{\partial g}{\partial u}(u_0, v_0) + k \frac{\partial g}{\partial v}(u_0, v_0) + \po(\|(h,k)\|) \\
	\end{align*}

	Par identification,
	\[
		\frac{\partial g}{\partial u}(u_0, v_0) = \frac{\partial x}{\partial u}(u_0, v_0) \frac{\partial f}{\partial x}(x_0, y_0) + \frac{\partial y}{\partial u}(u_0, v_0) \frac{\partial f}{\partial y}(x_0,y_0)
	\] et \[
		\frac{\partial g}{\partial v}(u_0, v_0) = \frac{\partial x}{\partial v}(u_0,v_0) \frac{\partial f}{\partial x}(x_0, y_0) + \frac{\partial y}{\partial v}(u_0, v_0) \frac{\partial f}{\partial y}(x_0, y_0).
	\] 
\end{prv}

\begin{exm}
	[Régression linéaire]~\\
	\begin{figure}[H]
		\centering
		\begin{asy}
			import graph;
			axes(EndArrow);
			size(5cm);

			real f(real x) { return x + 0.5; }

			real k = 35 / (7 - 0.5);

			for(int i = 0; i < 35; ++i) {
				real mag = exp(sin(100 * pi/exp(1) * i)) * 0.8 + exp(cos(i*40)/3);
				real eps = mag * cos(10 * exp(1)/pi * i) / 3;
				dot((i/k,f(i/k) + eps));
			}

			draw(graph(f, -1, 7), orange);
		\end{asy}
	\end{figure}
	\[
		y = a x + b
	\] 
	On fixe $(a,b) \in \R^2$. \[
		\varepsilon(a,b) = \sum_{i=1}^n\big( y_i - (ax_i + b) \big)^2
	\] l'erreur totale.

	On veut minimiser $\varepsilon(a,b)$. On a 
	\[
		\forall (a,b) \in \R^2,
		\begin{cases}
			\frac{\partial \varepsilon}{\partial a}(a,b) = -2\sum_{i=1}^{n}(y_i - ax_i - b)x_i,\\
			\frac{\partial \varepsilon}{\partial b}(a,b) = -2\sum_{i=1}^{n}(y_i - ax_i - b).
		\end{cases}
	\]

	Donc,
	\begin{align*}
		(a,b) \text{ point critique de } \varepsilon \iff& \begin{cases}
			a \sum_{i=1}^n {x_i}^2 + b\sum_{i=1}^{n}x_i = \sum_{i=1}^{n} y_ix_i\\
			a\sum_{i=1}^{n}x_i + nb = \sum_{i=1}^ny_i
		\end{cases}\\
		\iff& \begin{cases}
			a \left( \frac{1}{n}\sum_{i=1}^n {x_i}^2 - \overline{x}^2\right) = \overline{y} - \overline{x} \overline{y}\\
			b = \frac{1}{n}\sum_{i=1}^ny_i - \frac{a}{n}\sum_{i=1}^nx_i = \frac{1}{n}\sum_{i=1}^n x_i y_i - \overline{x} \overline{y}
		\end{cases}\\
		&\text{ où } \overline{x} = \frac{1}{n} \sum_{i=1}^n x_i,~\overline{y} = \frac{1}{n}\sum_{i=1}^n y_i\\
		\iff& \begin{cases}
			a = \frac{\Cov(x,y)}{V(x)}\\
			b = \overline{y} - a\overline{x}
		\end{cases}
	\end{align*}

	Coefficient de corrélation: $\frac{\Cov(x,y)}{\sigma_x \sigma_y} \in [-1, 1]$
\end{exm}












		\part{Corps}

\begin{exm}[Problème]
	\begin{itemize}
		\item 
			avec $A = \Z / 9 \Z$, résoudre $\overline{x}^2 = \overline{0}$ \\
			\begin{center}
				\begin{tabular}{|c|c|c|c|c|c|c|c|c|c|c|}
					\hline
					$\overline{x}$&$\overline{0}$& $\overline{1}$ &$\overline{2}$&$\overline{3}$ &$\overline{4}$ &$\overline{5}$ &$\overline{6}$ &$\overline{7}$ &$\overline{8}$& $\overline{9}$ \\
					\hline
					$\overline{x}^2$&$\overline{0}$ &$\overline{1}$ &$\overline{4}$ &$\overline{0}$ &$\overline{7}$ &$7$ &$\overline{0}$ &$\overline{4}$ &$\overline{1}$&$\overline{0}$\\
					\hline
				\end{tabular}
			\end{center}
			On a trouvé 3 solutions: $\overline{0}$, $\overline{3}$, $\overline{6}$.
		\item $\Z / 8\Z$
			\begin{center}
				\begin{tabular}{|c|c|c|c|c|c|c|c|c|}
					\hline
					$\overline{x}$& $\overline{0}$& $\overline{1}$& $\overline{2}$& $\overline{3}$& $\overline{4}$& $\overline{5}$& $\overline{6}$& $\overline{7}$\\
					\hline
					$\overline{x^2}$& $\overline{0}$& $\overline{1}$& $\overline{4}$& $\overline{1}$& $\overline{0}$& $\overline{1}$& $\overline{4}$& $\overline{1}$\\
					\hline
				\end{tabular}
			\end{center}
			$\overline{x}^2=7$ a 4 solutions: $\overline{1}, \overline{7}, \overline{3},\text{ et } \overline{5}$
		\item $A = \mathbbm{H} = \{a + bi + cj + dk  \mid  (a,b,c,d) \in \R^4\}$ \\
			$i^2 = j^2 = k^2 = -1$ 
			\begin{align*}
				\begin{array}{c c c}
					ij = k & jk = i & ji = j\\
					ji = -k & kj = -i & ik = -j
				\end{array}
			\end{align*}
			Dans cet anneau, $-1$ a 6 racines!
	\end{itemize}
\end{exm}

\begin{defn}
	Soit $(\mathbbm{K}, +, \times)$ un ensemble muni de deux lois de composition internes. On dit que c'est un \underline{corps} si
	 \begin{enumerate}
		\item $(\mathbbm{K}, \times)$ est un groupe abélien
		\item $(\mathbbm{K}, \times)$ est un monoïde commutatif
		\item $\forall x \in \mathbbm{K}\setminus \{0_\mathbbm{K}\}, \exists y \in \mathbbm{K}, xy = 1_\mathbbm{K}$
		\item $0_\mathbbm{K} \neq  1_\mathbbm{K}$
	\end{enumerate}
	\index{corps}
\end{defn}

\begin{exm}
	\begin{itemize}
		\item $(\C, +, \times)$ est un corps
		\item $(\R, +, \times)$ est un corps
		\item $(\Q, +, \times)$ est un corps
		\item $(\Z, +, \times)$ n'est pas un corps
	\end{itemize}
\end{exm}

\begin{prop}
	$(\Z / n\Z, +, \times)$ est un corps si et seulement si $n$ est premier.
\end{prop}

\begin{prv}
	\[
		\left( \Z / n\Z \right)^\times = \left\{ \overline{k}  \mid k \wedge n = 1 \right\}
	\] 
\end{prv}


\begin{prop}
	Tout corps est un anneau intègre.
\end{prop}

\begin{prv}
	Soit $(\mathbbm{K}, +, \times)$ un corps. Soient $(a,b) \in \mathbbm{K}^2$ tel que $a \times b = 0_\mathbbm{K}$.\\
	On suppose $a \neq  0_\mathbbm{K}$. Alors, $a$ est inversible et donc \[
		b = a^{-1} \times a \times b = a^{-1} \times 0_\mathbbm{K} = 0_\mathbbm{K}
	\] 
\end{prv}

\begin{exm}
	Soit $(\mathbbm{K},+,\times)$ un corps.\\
	Résoudre \[
		\begin{cases}
			x^2 = 1_\mathbbm{K}\\
			x \in \mathbbm{K}
		\end{cases}
	\]

	\begin{align*}
		x^2 = 1_\mathbbm{K} &\iff x^2 - 1_\mathbbm{K} = 0_\mathbbm{K}\\
		&\iff (x - 1_\mathbbm{K})(x+1_\mathbbm{K}) = 0_\mathbbm{K}\\
		&\iff x - 1_\mathbbm{K} = 0_\mathbbm{K} \text{ ou } x + 1_\mathbbm{K} = 0_\mathbbm{K}\\
		&\iff x = 1_\mathbbm{K} \text{ ou } x = -1_\mathbbm{K}
	\end{align*}

	Il y a au plus 2 solutions.
\end{exm}

\begin{prop}
	Soit $(\mathbbm{K},+,\times )$ un corps et $P$ un polynôme à coefficients dans $\mathbbm{K}$ de degré $n$. Alors, l'équation $P(x) = 0_{\mathbbm{K}}$ a au plus $n$ solutions dans $\mathbbm{K}$ 
	\qed
\end{prop}

\begin{crlr}[(Théorème de Wilson)]
	voir exercice 16 du TD 12
\end{crlr}


\begin{defn}
	Soit $(\mathbbm{K}, +, \times)$ un corps et $L\subset \mathbbm{K}$.\\
	On dit que $L$ est un \underline{sous corps} de $\mathbbm{K}$ si
	\begin{enumerate}
		\item $L$ est un anneau de $(\mathbbm{K}, +, \times)$ non nul
		\item $\forall x \in L\setminus \{0_\mathbbm{K}\}, x^{-1} \in L$ 
	\end{enumerate}
	\vspace{2mm}
	en d'autres termes si
	\begin{enumerate}
		\item $\forall (x,y) \in L^2, x - y \in L$
		\item $\forall (x,y) \in L^2, x \times y^{-1} \in L$
	\end{enumerate}
	\vspace{5mm}
	On dit aussi que $\mathbbm{K}$ est une \underline{extension} de $L$.
	\index{sous corps}
	\index{extension}
\end{defn}

\begin{prop}
	Tout sous corps est un corps. \qed
\end{prop}

\begin{defn}
	Soient $(\mathbbm{K}_1,+,\times )$ et $(\mathbbm{K}_2,+, \times)$ deux corps et $f: \mathbbm{K}_1 \to \mathbbm{K}_2$.\\
	On dit que $f$ est un \underline{morphisme de corps} si $f$ est un morphisme d'anneaux.\\
	i.e. si
	\[
		\begin{cases}
			\forall (x,y) \in {\mathbbm{K}_1}^2,& f(x+y) = f(x) + f(y)\\
			\forall (x,y) \in {\mathbbm{K}_1}^2,& f(x \times y) = f(x) \times f(y)\\
		\end{cases}
	\] 
	\index{homomorphisme (de corps)}
	\index{morphisme (de corps)}
\end{defn}

\begin{prop}
	Tout morphisme de corps est injectif.
\end{prop}

\begin{prv}
	Soit $f: \mathbbm{K}_1 \to \mathbbm{K}_2$ un morphisme de corps.\\
	\begin{itemize}
		\item $\Ker(f)$ est un sous groupe de $(\mathbbm{K}_1, +)$ 
		\item Soit $x \in \Ker(f)$ et $y \in \mathbbm{K}_1$ \[
				f(x \times y) = f(x) \times f(y) = 0_{\mathbbm{K}_2} \times f(y) = 0_{\mathbbm{K}_2}
			\]
		\item Soit $x \in \Ker(f) \setminus \{0_{\mathbbm{K}_1}\}$.\\
			Alors, $x$ est inversible.\\
			\begin{align*}
				\begin{rcases*}
					x \in \Ker(f)\\
					x^{-1} \in \mathbbm{K}_1
				\end{rcases*}& \text{ donc } x \times x ^{-1} \in \Ker(f)\\
				&\text{ donc } 1_{\mathbbm{K}_1} \in \Ker(f)\\
				&\text{ donc } f(1_{\mathbbm{K}_1}) = 0_{\mathbbm{K}_2}
			\end{align*}
			Or, $f(1_{\mathbbm{K}_1}) = 1_{\mathbbm{K}_2} \neq 0_{\mathbbm{K}_2}$
	\end{itemize}
	Donc, $\Ker(f) = \{0_{\mathbbm{K}_1}\}$ donc $f$ est injective.
\end{prv}

\begin{exm}
	$\begin{array}{cc}
		\C &\longrightarrow \C\\
		z &\longmapsto \overline{z}\\
	\end{array}$ est un morphisme de corps
\end{exm}



		\part{Opérations sur les séries}

\begin{prop}
	L'ensemble $E = \{u \in \C^\N  \mid \Sigma u_n \text{ converge}\}$ est un sous-espace vectoriel de $\C^\N$ et \begin{align*}
		S: E &\longrightarrow \C \\
		u &\longmapsto \sum_{n=0}^{+\infty} u_n
	\end{align*} est une forme linéaire.
	\qed
\end{prop}

\begin{rmk}
	La somme d'une série convergente et d'une série divergente diverge.
	Le produit d'une série divergente par un scalaire non nul diverge.
\end{rmk}

		\part{Comparaison de suites}

\begin{defn}
	Soient $u$ et $v$ deux suites réelles. On dit que $u$ est \underline{dominée} par  $v$ si \[
	\exists M\in \R, \exists N\in \N,\forall n\ge N,\left| u_n \right| \le M \left| v_n \right| 
	\] Dans ce cas, on note $u = O(v)$ ou $u_n = O(v_n)$ et on dit que "$u$ est un grand o de $v$"
\end{defn}

\begin{exm}
	En informatique, on dit qu'un alogirithme a une \underline{complexité linéaire} si son temps d'éxécution est un $O(n)$ 
	Par exemple, on calcule $a^n$ 

	\begin{itemize}
		\item Approche naïve
			\begin{algorithm}
				\begin{algorithmic}[1]
					\State $p \gets 1$
					\For{$i \in \left\llbracket 0,n-1 \right\rrbracket$}
						\State $p \gets p \times a$
					\EndFor
					\State \Return p
				\end{algorithmic}
			\end{algorithm}
			Complexité linéaire $O(n)$
		\item Exponentiation rapide\\
			On écrit $n$ en binaire: \begin{align*}
				n &= \overline{a_k a_{k-1}\ldots a_0}^{(2)}\\
					&= \sum_{i=0}^{k} a_i 2^i
			\end{align*} avec $(a_i) \in \left\{ 0,1 \right\} ^{k+1}$
			\begin{align*}
				a^n &= a^{\sum_{i=0}^{k} a_i 2^i} \\
				&= \prod_{i=0}^{k} a^{a_i 2^i}  \\
			\end{align*}
			
			\begin{algorithm}
				\begin{algorithmic}
					[1]

					\State $s \gets 0$
					\State $p \gets a$
					\For{ $i \in \left\llbracket 0, \log_2(n) \right\rrbracket$}
						\State $p \gets p \times p$
						\If{$a[i] = 1$}
							\State $s \gets s + p$
						\EndIf
					\EndFor
					\State \Return s
				\end{algorithmic}
			\end{algorithm}
			Compléxité logarithmique $O(\log_2(n))$
	\end{itemize}
\end{exm}


\begin{prop}
	$O$ est une relation réfléctive et transitive.
\end{prop}

\begin{prv}
	\begin{itemize}
		\item Soit $u$ une suite. On pose $M = 1$ et \[
			\forall n \in \N, \left| u_n \right| \le M \left| u_n \right|
			\] Donc $u = O(u)$.
		\item Soient $u, v, w$ trois suites telles que  \[
		\begin{cases}
			u = O(v)\\
			v = O(w)
		\end{cases}
		\] Soient $M_1,M_2 \in \R$ et $N_1,N_2\in \N$ tels que \[
		\begin{cases}
			\forall n \ge  N_1, \left| u_n \right| \le M_1 \left| v_n \right| \\
			\forall n \ge  N_2, \left| v_n \right| \le M_2 \left| w_n \right| \\
		\end{cases}
		\] 

		Nécéssairement, $M_1\ge 0$ et $M_2\ge 0$.\\
		Soit $N = \max(N_1,N_2)$. \[
		\forall n \ge  N, \left| u_n \right| \le M_1 \left| v_n \right| \le  M_1M_2 \left| w_n \right| 
		\] Donc $u = O(w)$
	\end{itemize}
\end{prv}

\begin{defn}
	Soient $u$ et $v$ deux suites. On dit que $u$ est \underline{négligeable} devant $v$ si \[
	\forall \varepsilon>0, \exists N\in \N, \forall n\ge N, \left| u_n \right| \le \varepsilon \left| v_n \right| 
	\] Dans ce cas, on note $u = o(v)$ ou $u_n = o(v_n)$ ou on le lit "$u$ est un petit o de $v$"
\end{defn}

\begin{prop}
	$o$ est une relation transitive, non-réfléctive
\end{prop}

\begin{prv}
	\begin{itemize}
		\item Soient $u$, $v$ et $w$ trois suites telles que \[
			\begin{cases}
				u = o(v)\\
				v = o(w)
			\end{cases}
			\] Soit $\varepsilon>0$. Soit $N_1\in \N$ tel que \[
			\forall n \ge N_1, \left| u_n \right| \le \sqrt{\varepsilon}  \left| v_n \right| 
			\] Soit $N_2\in \N$ tel que \[
			\forall n \ge N_2, \left| v_n \right| \le \sqrt{\varepsilon}  \left| w_n \right| 
			\] On pose $N = \max(N_1,N_2)$, alors \[
			\forall n \ge N, \left| u_n \right| \le \sqrt{\varepsilon}  \left| v_n \right| \le \underbrace{\sqrt{\varepsilon} \times \sqrt{\varepsilon}} _\varepsilon \left| w_n \right| 
			\] donc $u = o(w)$
		\item Soit $u$ une suite tel qu'il existe $N \in \N$ tel que \[
		\forall n \ge N, u_n > 0
		\] On suppose que $u = o(u)$, alors \[
		\forall \varepsilon>0,\exists N \in \N, \forall n \ge N, \left| u_n \right| \le \varepsilon \left| u_n \right| 
		\] On pose $\varepsilon = \frac{1}{2}$ alors \[
		\exists N \in \N, \forall n \ge N, \left| u_n \right| \le \frac{1}{2} \left| u_n \right| 
		\] une contradiction
	\end{itemize}
\end{prv}

\begin{prop}
	Soient $u$ et $v$ deux suites.
	\begin{itemize}
		\item $o(u) + o(u) = o(u)$
		\item $v \times o(u) = o(uv)$
		\item $o(u) \times o(v) = o(uv)$
		\item $o(o(u)) = o(u)$
	\end{itemize}
	\qed
\end{prop}

\begin{defn}
	Soient $u$ et $v$ deux suites. On dit que $u$ et $v$ sont \underline{équivalentes} si \[
	u = v + o(v)
	\] i.e. \[
	\forall \varepsilon >0, \exists N \in \N, \forall n \ge N, \left| u_n-v_n \right| \le \varepsilon\left| v_n \right| 
	\] Dans ce cas, on le note $u \sim v$
\end{defn}

\begin{prop}
	$\sim$ est une relation d'équivalence \qed
\end{prop}

\begin{prop}
	Soient $(u,v) \in \R^\N$. On suppose que $v$ ne s'annule pas à partir d'un certain rang
	\begin{enumerate}
		\item $u = o(v) \iff \left( \frac{u_n}{v_n} \right)$ bornée
		\item $u = o(v) \iff \frac{u_n}{v_n} \tendsto{n \to  +\infty} 0$
		\item $u \sim v \iff \frac{u_n}{v_n} \tendsto{n \to  +\infty} 1$
	\end{enumerate}
	\qed
\end{prop}

\begin{prop}
	[Suites de références]
	\begin{enumerate}
		\item $\ln^\alpha(n) = o(n^\beta)$ avec $(\alpha,\beta) \in \left( \R^+_* \right) ^2$ 
		\item $n^\beta = o(a^n)$ avec $\beta > 0$ et $a > 1$ 
		\item $a^n = o(n!)$ avec $a >1$ 
		\item $n! = o(n^n)$
	\end{enumerate}
\end{prop}


\begin{lem}
	[Exercice 10 du TD]
	Soit $u \in \left(\R^+_*\right)^\N$\\
	Si $\frac{u_{n+1}}{u_n} \tendsto{n \to +\infty} \ell < 1$ avec $\ell\in \R$,\\ alors $u_n \tendsto{n \to +\infty} 0$
\end{lem}

\begin{prv} [de la proposition]
	\begin{enumerate}
		\item par croissance comparée
		\item On pose $\forall n \in \N^*, u_n = \frac{n^\beta}{a^n}$. 
			\begin{align*}
				\forall  n \in \N^*, \frac{u_{n+1}}{u_n} &= \left( \frac{n+1}{n} \right) ^\beta \times \frac{1}{a} \\
				&= \frac{1}{a}\left( 1+\frac{1}{n} \right) ^\beta \\
				&\tendsto{n \to +\infty} \frac{1}{a} < 1
			\end{align*}
			Donc, $u_n \tendsto{n \to  +\infty} 0$
		\item On pose $\forall n \in \N, u_n = \frac{a^n}{n!}$ \[
			\forall n \in \N, \frac{u_{n+1}}{u_n} = \frac{a}{n+1} \tendsto{n \to +\infty} 0 < 1
			\] donc $u_n \tendsto{n \to +\infty} 0$
		\item On pose $\forall  n\in \N^*, u_n = \frac{n!}{n^n}$.
			\begin{align*}
				\forall n \in \N^*, \frac{u_{n+1}}{u_n}
				&= (n+1) {\frac{n^n}{(n+1)^{n+1}}} \\
				&= \left( \frac{n}{n+1} \right) ^n \\
				&= e^{n \ln\left( \frac{n}{n+1} \right) } \\
				&= e^{n \ln\left( 1+\frac{1}{n+1} \right)} \\
				&= e^{n(-\frac{1}{n} + o(\frac{1}{n})} \\
				&= e^{-1 + o(1)} \\
				&\tendsto{n \to  +\infty} e^{-1}<1
			\end{align*}
			donc $u_n \tendsto{n\to +\infty} 0$
	\end{enumerate}
\end{prv}

	}

	{
		\chap[03]{Étude de fonctions}
		\renewcommand{\cwd}{../chap03}
		\begin{defn}
	Un \underline{proposition} est un énoncé qui est soit vrai, soit faux.
\end{defn}

\begin{exm}
	\begin{align*}
		\begin{rcases*}
			A: ``B \text{ est vraie }"\\
			B: ``A \text{ est fausse }"\\
		\end{rcases*} \text{ Le système $\{A,B\}$ est une \underline{auto-contradiction}}
	\end{align*}
\end{exm}

\begin{defn}
	\underline{Démontrer} une proposition revient à prouver qu'elle est vraie
\end{defn}

		\begin{defn}
	Soit $E$ un $\mathbbm{K}$-espace vectoriel. On dit que $E$ est de \underline{dimension finie} si $E$ a au moins une famille génératrice finie. On dit que $E$ est de \underline{dimension infinie} sinon.
	\index{dimension finie (espace vectoriel)}
	\index{dimension infinie (espace vectoriel)}
\end{defn}

\begin{thm}
	[Théorème de la base extraite]
	Soit $E$ un $\mathbbm{K}$-espace vectoriel non nul de dimension finie. Soit $\mathcal{G}$ une famille génératrice finie de $E$. Alors, il existe une base $\mathcal{B}$ de $\mathcal{E}$ telle que $\mathcal{B} \subset \mathcal{G}$.
\end{thm}

\begin{prv}
	[par récurrence sur $\#G = \Card(G)$]
	\begin{itemize}
		\item Soit $E$ un $\mathbbm{K}$-espace vectoriel non nul engendré par $\mathcal{G} = (u)$.\\
			Si $u = 0_E$, alors $E = \{0_E\}$: une contradiction $\lightning$ \\
			Donc $u \neq 0_E$ donc $(u)$ est libre. En effet, \[
				\forall \lambda \in \mathbbm{K}, \lambda u = 0_E \implies \lambda = 0_\mathbbm{K}
			\] Donc $\mathcal{G}$ est une base de $E$.\\
		\item Soit $n \in \N_*$. Soit $E$ un $\mathbbm{K}$-espace vectoriel. On suppose que si $E$ a une famille génératrice constituée de $n$ vecteurs, alors on peut extraire de cette famille une base de $E$.\\
			Soit $\mathcal{G}$ une famille génératrice de $E$ avec $n+1$ vecteurs.\\
			Si $\mathcal{G}$ est libre, alors $\mathcal{G}$ est une base de $E$. \\
			Si $\mathcal{G}$ n'est pas libre, alors il existe $u \in \mathcal{G}$ tel que $u \in \Vect(\mathcal{G}\setminus \{u\})$ \\
			Donc $\mathcal{G}\setminus \{u\}$ engendre $E$. Or, $\mathcal{G}\setminus \{u\}$ possède $n$ vecteurs. D'après l'hypothèse de récurrence, il existe une base $\mathcal{B}$ de $E$ telle que \[
				\mathcal{B} \subset \mathcal{G} \setminus \{u\} \subset \mathcal{G}
			\] 
	\end{itemize}
\end{prv}

\begin{crlr}
	Tout espace de dimension finie a une base.
	\qed
\end{crlr}

\begin{thm}
	[Théorème de la base incomplète]
	Soit $E$ un $\mathbbm{K}$-espace vectoriel de dimension finie, $\mathcal{G}$ une famille génératrice finie de $E$. $\mathcal{L}$ une famille libre de $E$. Alors, il existe une base $\mathcal{B}$ de $E$ telle que \[
		\mathcal{L} \subset \mathcal{B} \text{ et } \mathcal{B}\setminus \mathcal{L} \subset \mathcal{G}
	\] 
\end{thm}

\begin{prv}
	[par récurrence sur $\#(\mathcal{G}\setminus\mathcal{L})$]
	\begin{itemize}
		\item Avec les notations précédentes, on suppose que $\mathcal{G}\setminus\mathcal{L} \neq \O$ \[
				\forall u \in \mathcal{G}, u \in \mathcal{L}
			\] Donc $\mathcal{G} \subset \mathcal{L}$ donc $\mathcal{L}$ est génératrice donc $\mathcal{L}$ est une base de $E$. On pose $\mathcal{B} = \mathcal{L}$ et alors \[
				\mathcal{L} \subset  \mathcal{B} \text{ et } \mathcal{B}\setminus\mathcal{L} = \O \subset  \mathcal{G}
			\] 
		\item Soit $n \in \N$. On suppose que si $\mathcal{G}$ est génératrice et $\mathcal{L}$ libre avec $\#(\mathcal{G}\setminus\mathcal{L}) = n$ alors il existe une base $\mathcal{B}$ de $E$ telle que \[
			\mathcal{L}\subset \mathcal{B} \text{ et } \mathcal{B}\setminus\mathcal{L}\subset \mathcal{G}
		\] Soient à présent $\mathcal{G}$ une famille génératrice de $E$ et $\mathcal{L}$ une famille libre de $E$ telles que $\#(\mathcal{G}\setminus\mathcal{L}) = n+1 > 0$\\
		Si $\mathcal{L}$ engendre $E$, alors $\mathcal{L}$ est une base de $E$. On pose $\mathcal{B} = \mathcal{L}$ et on a bien \[
			\mathcal{L} \subset  \mathcal{B} \text{ et } \mathcal{B} \setminus \mathcal{L} = \O \subset  \mathcal{G}
		\] On suppose que $\mathcal{L}$ n'engendre pas $E$. Il existe $u \in \mathcal{G}$ tel que $u \not\in \Vec(\mathcal{L})$ (car sinon, $\mathcal{G} \subset \Vect(\mathcal{L})$ et donc $\underbrace{\Vect(\mathcal{G})}_{= E} \subset  \underbrace{\Vect(\mathcal{L})}_{ \subset E}$\\
		Donc $\mathcal{L} \cup \{u\} $ est libre. On pose $\mathcal{L}' = \mathcal{L} \cup \{u\} $ \[
			\mathcal{G}\setminus \mathcal{L}' = \mathcal{G}\setminus (\mathcal{L} \cup \{u\}) = (\mathcal{G}\setminus\mathcal{L})\setminus \{u\} 
		\] donc $\#(\mathcal{G}\setminus\mathcal{L}') = n+1 -1 = n$\\
		D'après l'hypothèse de récurrence, il existe $\mathcal{B}$ une base de $E$ telle que \[
			\mathcal{L} \subset  \mathcal{L}' \subset \mathcal{B} \text{ et } \mathcal{B}\setminus \mathcal{L}' \subset \mathcal{G}
		\] \[
			\mathcal{B} \setminus \mathcal{L} = \underbrace{\mathcal{B}\setminus\mathcal{L}'}_{\subset \mathcal{G}} \cup \underbrace{\{u\}}_{\subset \mathcal{G} \text{ car } u \in \mathcal{G}}
		\] On a $\mathcal{B}\setminus\mathcal{L}\subset \mathcal{G}$
	\end{itemize}
\end{prv}

\begin{thm}
	Soit $E$ un $\mathbbm{K}$-espace vectoriel de dimension finie. Toutes les bases de $E$ ont le même cardinal.
\end{thm}

\begin{prv}
	Soit $\mathcal{G}$ une famille génératrice finie de $E$ et $\mathcal{B} \subset  \mathcal{G}$ une base de $E$. On note $n = \#\mathcal{B}$ \\
	Soit $\mathcal{B}'$ une base de $E$. On pose $p = n - \#(\mathcal{B} \cap  \mathcal{B}')$. Montrons par récurrence sur  $p$ que $\#\mathcal{B} = \#\mathcal{B}'$ 
	\begin{itemize}
		\item On suppose que $p = 0$. Alors, $\#(\mathcal{B} \cap \mathcal{B}') = n$ \\
			Or, $\mathcal{B}' \cap \mathcal{B} \subset \mathcal{B}$ donc $\mathcal{B} \cap \mathcal{B}' = \mathcal{B}$ donc $\mathcal{B} \subset  \mathcal{B}'$ et donc $\mathcal{B} = \mathcal{B}'$ 
		\item Soit $p \in \N$. On suppose que si $\mathcal{B}'$ est une base de $E$ telle que $n - \#(\mathcal{B} \cap \mathcal{B}') = p$, alors $\#\mathcal{B}' = n$ \\
			Aoit $\mathcal{B}'$ une base de $E$ telle que $n - \#(\mathcal{B}\cap \mathcal{B}') = p+1 > 0$ \\
			Donc $\mathcal{B} \cap \mathcal{B}' \neq \mathcal{B}$. Soit $u \in \mathcal{B}' \setminus \mathcal{B}$. D'après le lemme d'échange, il existe $v \in \mathcal{B}\setminus \mathcal{B}'$ tel que $\mathcal{B}' \setminus \{u\} \cup \{v\}$ est une base de $E$. On pose $\mathcal{B}'' = \mathcal{B}' \setminus \{u\} \cup \{v\}$ 
			\begin{align*}
				\mathcal{B}'' \cap \mathcal{B} &= \left( (\mathcal{B}' \setminus \{u\})  \cap \mathcal{B} \right) \cup \{v\} \\
				&= (\mathcal{B}' \cap \mathcal{B}) \cup \{v\} \\
			\end{align*}
			donc,
			\begin{align*}
				n - \#(\mathcal{B}'' \cap \mathcal{B}) &= n - (\#(\mathcal{B}' \cap \mathcal{B}) + 1) \\
				&= p+1- 1 \\
				&= p \\
			\end{align*}
			D'après l'hypothèse de récurrence, \[
				\#\mathcal{B}'' = n
			\] Or, $\#\mathcal{B}'' = \#\mathcal{B}'$
	\end{itemize}
\end{prv}

\begin{lem}
	Soient $\mathcal{B}$ et $\mathcal{B}'$ deux bases de $E$ telles que $\mathcal{B}\subset \mathcal{B}'$. Alors, $\mathcal{B} = \mathcal{B}'$.
\end{lem}

\begin{prv}
	On suppose $\mathcal{B}' \neq \mathcal{B}$. Soit $u \in \mathcal{B}' \setminus \mathcal{B}$
	$u \in E = \Vect(\mathcal{B})$ donc $\mathcal{B} \cup \{u\}$ n'est pas libre.
	Donc $\mathcal{B}\cup \{u\} \subset \mathcal{B}'$ et $\mathcal{B}'$ est libre donc $\mathcal{B}\cup \{u\}$ est libre: une contradiction $\lightning$
\end{prv}

\begin{lem}
	[Lemme d'échange] Soient $\mathcal{B}_1$ et $\mathcal{B}_2$ deux bases de $E$ et $u \in \mathcal{B}_1 \setminus \mathcal{B}_2$. Alors, il existe $v \in \mathcal{B}_2$ tel que $(\mathcal{B}_1 \setminus \{u\}) \cup \{v\}$ soit une base de $E$.
\end{lem}

\begin{prv}
	[1${}^\text{nde}$ méthode]
	On suppose que pout tout $v \in \mathcal{B}_2$, $(\mathcal{B}_1\setminus \{u\}) \cup \{v\}$ n'est pas une base de $E$
	Soit $v \in \mathcal{B}_2$.
	\begin{itemize}
		\item Supposons $(\mathcal{B}_1\setminus \{u\})\cup \{v\}$ non libre. $\mathcal{B}_1 \setminus \{u\}$ est libre. Donc $v \in \Vect(\mathcal{B}_1 \setminus \{u\})$
		\item Supposons $(\mathcal{B}_1\setminus \{u\}) \cup \{v\}$ non génératrice.
			Comme $\mathcal{B}_1$ engendre $E$, $u \not\in \Vect(\mathcal{B}_1\setminus \{v\})$.
			On suppose que $\mathcal{B}_1 \neq \mathcal{B}_2$.
			$\forall v \in \mathcal{B}_2 \setminus \mathcal{B}_1, \Vect(\mathcal{B}_1 \setminus \{v\}) = \Vect(\mathcal{B}_1) = E \ni u$ 
			donc, $(\mathcal{B}_1\setminus \{u\}) \cup \{v\}$ engendre $E$ et donc \[
				v \in \Vect(\mathcal{B}_1 \setminus \{u\})
			\] On a aussi \[
				\forall v \in \mathcal{B}_1 \setminus \{u\}, v \in \Vect(\mathcal{B}_1\setminus \{u\})
			\] Comme $u \not\in \mathcal{B}_2$, on a \[
				\forall v \in \mathcal{B}_2, v \in \Vect(\mathcal{B}_1\setminus \{u\})
			\] docn \[
				E = \Vect(\mathcal{B}_2) \subset \Vect(\mathcal{B}_1\setminus \{u\})
			\] donc $\mathcal{B}_1\setminus \{u\}$ engendre $E$ donc $\mathcal{B}_1\setminus \{u\}$ est une base de $E$. Or, $\mathcal{B}_1 \setminus \{u\}  \subset  \mathcal{B}_1$, donc $\mathcal{B}_1\setminus \{u\} = \mathcal{B}_1$
	\end{itemize}
\end{prv}

\begin{prv}
	[2${}^\text{nde}$ méthode]
	On suppose que pout tout $v \in \mathcal{B}_2$, $(\mathcal{B}_1\setminus \{u\}) \cup \{v\}$ n'est pas une base de $E$
	\begin{itemize}
		\item Comme $u \in \mathcal{B}_1 \setminus \mathcal{B}_2$, nécéssairement $\mathcal{B}_1 \neq \mathcal{B}_2$ donc $\mathcal{B}_2 \not\subset \mathcal{B}_1$, donc $\mathcal{B}_2\setminus\mathcal{B}_1 \neq \O$ 
		\item Soit $v \in \mathcal{B}_2\setminus\mathcal{B}_1$. Il existe $(\lambda_w)_{w\in\mathcal{B}_1}$ une famille de scalaires presque nulle telle que \[
				v = \sum_{w \in \mathcal{B}_1} \lambda_w w - \lambda_u u + + \sum_{w \in \mathcal{B}_1\setminus \{u\}}\lambda_w w
			\]
			Si $\lambda_u \neq 0_E$, alors
			\begin{align*}
				u &= \lambda_u^{-1}\left( v - \sum_{w \in \mathcal{B}_1 \setminus \{u\}} \lambda_w w \right)\\
					&\in \Vect(\mathcal{B}_1\setminus \{u\} \cup v)
			\end{align*}
			 donc $\mathcal{B}_1 \subset \Vect(\mathcal{B}_1\setminus \{u\} \cup \{v\})$\\
			 et donc $E \subset  \Vect(\mathcal{B}_1 \setminus \{u\} \cup \{v\})$ \\
			 et donc $\mathcal{B}_1 \setminus \{u\} \cup \{v\}$ engendre $E$ \\
			 donc $\mathcal{B}_1 \setminus \{u\} \cup \{v\}$ n'est pas libre\\
			 donc $v \in \Vect(\mathcal{B}_1\setminus \{u\})$ (car $\mathcal{B}_1 \setminus \{u\}$ est libre\\
			 donc $\lambda_u = 0_\mathbbm{K}$ $\lightning$\\`

			 Donc, $\lambda_u = 0_\mathbbm{K}$, docn $v \in \Vect(\mathcal{B}_1\setminus \{u\})$ \\
			 On vient de prouver que
			 \begin{align*}
			 	\mathcal{B}_2 \setminus \mathcal{B}_1 \subset \Vect(\mathcal{B}_1 \setminus \{u\})\\
			 	\mathcal{B}_1 \setminus \{u\} \subset \Vect(\mathcal{B}_1 \setminus \{u\})\\
			 \end{align*}
			 Comme $u \not\in \mathcal{B}_2$, \[
			 	\mathcal{B}_2 \subset \Vect(\mathcal{B}_1 \setminus \{u\})
			 \] donc \[
			 	E = \Vect(\mathcal{B}_2) \subset  \Vect(\mathcal{B}_1 \setminus \{u\})
			 \] donc $\mathcal{B}_1 \setminus \{u\}$ engendre $E$. Donc,  $\mathcal{B}_1 \setminus \{u\}$ est une base de $E$.\\
			 Or, $\mathcal{B}_1 \setminus \{u\} \subset  \mathcal{B}_1$, donc $\mathcal{B}_1 \setminus \{u\} = \mathcal{B}_1$
	\end{itemize}
\end{prv}

\begin{defn}
	Soit $E$ un $\mathbbm{K}$-espace vectoriel de dimension finie. Le cardinal commun à toutes les bases de $E$ est appelé \underline{dimension} de $E$ est notée $\dim(E)$ ou $\dim_\mathbbm{K}(E)$\\
	C'est donc aussi le nombre de coordonnées de n'importe quel vecteur dans n'importe quelle base.
	\index{dimension (espace vectoriel)}
\end{defn}

\begin{exm}
	\begin{enumerate}
		\item $\dim_\R(\C) = 2$ et $\dim_\C(\C) = 1$ 
		\item $\dim_\mathbbm{K}(\mathbbm{K}^{n}) = n$ 
		\item $\dim_{\mathbbm{K}}(\mathcal{M}_{n,p}(\mathbbm{K})) = np$
	\end{enumerate}
\end{exm}

\begin{crlr}
	Soit $E$ un $\mathbbm{K}$-espace vectoriel de dimension finie, $\mathcal{L}$ une famille libre de $E$, $\mathcal{G}$ une famille génératrice de $E$. On note $n = \dim(E)$
	\begin{enumerate}
		\item $\#\mathcal{G} \ge n$ et $(\#\mathcal{G} = n \implies \mathcal{G} \text{ est une base de } E$)
		\item $\#\mathcal{L} \le n$ et $(\#\mathcal{L} = n \implies \mathcal{L} \text{ est une base de } E$)
	\end{enumerate}
\end{crlr}

\begin{crlr}
	$\R^{\R}$ est de dimension infinie.
	$\forall i \in \N, e_i: x \mapsto x^i$\\
	$(e_i)_{i\in\N}$ est libre dans $\R^\R$
\end{crlr}

\begin{prop}
	Soient $E$ et $F$ deux $\mathbbm{K}$-espaces vectoriels de dimension finie. Alors $E\times F$ est de dimension finie et $\dim(E\times F) = \dim(E) + \dim(F)$
\end{prop}

\begin{prv}
	Soit $(e_1,\ldots, e_n)$ une base de $E$, $(f_1, \ldots, f_p)$ une base de $F$.
	On pose \[
		\left\{\begin{array}
			{r c l}
			u_1 &=& (e_1,0_F)\\
			u_2 &=& (e_2,0_F)\\
					&\vdots&\\
			u_n &=& (e_n,0_F)\\
			u_{n+1} &=& (0_E, f_1)\\
			u_{n+2} &=& (0_E, f_2)\\
					&\vdots&\\
			u_{n+p} &=& (0_E,f_p)\\
		\end{array}\right.
	\]
	Soit $(x,y) \in E\times F$. \[
		\begin{cases}
			\exists (x_1,\ldots,x_n)\in \mathbbm{K}^n, x = \sum_{i=1}^{n} x_ie_i
			\exists (y_1,\ldots,y_n)\in \mathbbm{K}^n, x = \sum_{j=1}^{p} y_jf_j
		\end{cases}
	\] 
	\begin{align*}
		(x,y) &= \left( \sum_{i=1}^{n} x_ie_i, \sum_{i=1}^{p} y_jf_j \right)  \\
		&= \sum_{i=1}^{n} x_i (e_i + 0_F) + \sum_{j=1}^{p} y_j (0_E, f_j) \\
		&= \sum_{i=1}^{n} x_i u_i + \sum_{j=1}^{p} y_j u_{n+j} \\
	\end{align*}
	Donc, $E\times F = \Vect(u_1, \ldots, u_{n+p})$ donc $E\times F$ est de dimension finie.\\
	Soit $(\lambda_1, \ldots, \lambda_{n+p}) \in \mathbbm{K}^{n+p}$ tel que \[
		(*): \quad \sum_{k=1}^{n+p} \lambda_ku_k = 0_{E\times F} = (0_E, 0_F)
	\]
	\begin{align*}
		(*) &\iff \sum_{k=1}^{n} \lambda_k (e_k, 0_F) + \sum_{k=n+1}^{p} \lambda_k(0_E, f_{k-n}) = (0_E, 0_F)\\
				&\iff \begin{cases}
					\sum_{k=1}^{n} \lambda_k e_k = 0_E\\
					\sum_{k=n+1}^{p} \lambda_k f_{k-n} = 0_F
				\end{cases}\\
				&\iff \begin{cases}
					\forall k \in \left\llbracket 1,n \right\rrbracket, \lambda_k = 0_\mathbbm{K} \qquad&(\text{car $(e_1,\ldots,e_n)$ est libre})\\
					\forall k \in \left\llbracket n+1,n+p \right\rrbracket, \lambda_k = 0_\mathbbm{K} \qquad&(\text{car $(f_1,\ldots,f_n)$ est libre})\\
				\end{cases}
	\end{align*}
	Donc $(u_1, \ldots, u_{n+p})$ est une base de $E\times F$. Donc, $\dim(E\times F) = n + p = \dim(E) + \dim(F)$
\end{prv}

\begin{rmk}
	[Convention]
	\[\dim\big(\{0_E\}\big) = 0\]
\end{rmk}

\begin{thm}
	Soit $E$ un $\mathbbm{K}$-espace vectoriel de dimension finie, $F$ un sous-espace vectoriel de $E$. Alors, $F$ est de dimension finie et  $\dim(F) \le \dim(E)$\\
	Si $\dim(F) = \dim(E)$, alors $F = E$
\end{thm}

\begin{prv}
	On considère \[
		A = \{k \in \N \mid \text{il existe une famille libre de $F$ à $k$ éléments}\} 
	\]
	On suppose $F \neq \{0_E\}$.
	\begin{itemize}
		\item Soit $u \in F\setminus \{0_E\}$. $(u)$ est libre donc $1 \in A$ et donc $A \neq \O$
		\item Soit $\mathcal{L}$ une famille libre de $F$. Alors, $\mathcal{L}$ est une famille libre de $E$ \\
			donc $\#\mathcal{L} \le \dim(E)$\\
			Donc $A$ est majorée par $\dim(E)$ \\
			On en déduit que $A$ a un plus grand élément $p$.
		\item Soit $\mathcal{L}$ une famille libre de $F$ avec $p$ éléments.\\
			Si $\mathcal{L}$ n'engendre pas $F$, alors il existe $u\in F$ tel que $u\not\in \Vect(\mathcal{L})$ et donc $\mathcal{L} \cup \{u\}$ est une famille libre de $F$, donc $p+1 \in A$ en contradiction avec la maximalité de $p$.\\
			Donc $\mathcal{L}$ est une base de $F$ donc $F$ est de dimension finie et $\dim(F) = p \le \dim(E)$\\
	\end{itemize}

	Soit $\mathcal{B}$ une base de $F$. Alors, $\mathcal{B}$ est aussi une famille de libre de de $E$. Donc $\#\mathcal{B} \le \dim(E)$ donc $\dim(F) = \dim(E)$ \\
	Si $\dim(F) = \dim(E)$, alors $\mathcal{B}$ est une base de $E$, et donc $F = \Vect(\mathcal{B}) = E$
\end{prv}

\begin{prop}
	[Formule de Grassmann]
	Soit $E$ un $\mathbbm{K}$-espace vectoriel de dimension finie, $F$ et $G$ deux sous-espace vectoriels de $E$. Alors, \[
		\dim(F+G) = \dim(F) + \dim(G) - \dim(F\cap G)
	\] 
\end{prop}

\begin{prv}
	Soit $(e_1, \ldots, e_p)$ une base de $F\cap G$. $(e_1,\ldots,e_p)$ est une famille libre de $F$.\\
	On complète $(e_1, \ldots, e_p)$ en une base $(e_1, \ldots, e_p, u_1, \ldots, u_q)$ de $F$.\\
	De même, on complète $(e_1, \ldots, e_p)$ en une base $(e_1, \ldots, e_p, v_1, \ldots, v_r)$ de $G$.\\
	On pose  $\mathcal{B} = (e_1, \ldots, e_p, u_1, \ldots, u_q, v_1, \ldots, v_r)$. Montrons que $\mathcal{B}$ est une base de $F+G$
	\begin{itemize}
		\item Soit $u \in F+G$ \\
			On pose $u = v+w$ avec $\begin{cases}
				v\in F\\
				w \in G
			\end{cases}$.\\
			On pose $v = \sum_{i=1}^p \lambda_i e_i + \sum_{i=1}^q \mu_i u_i$ avec $(\lambda_1, \ldots, \lambda_p, \mu_1, \ldots, \lambda_q) \in \mathbbm{K}^{p+q}$\\
			On pose aussi $w = \sum_{i = 1}^p \lambda'_ie_i + \sum_{j=1}^r \nu_j v_j$ avec $(\lambda_1',\ldots,\lambda_p', \nu_1, \ldots, \nu_r) \in \mathbbm{K}^{p+r}$\\
			D'où, \[
				u = \sum_{i=1}^p (\lambda_i + \lambda'_i)e_i + \sum_{j=1}^q \mu_j u_j + \sum_{k=1}^r \nu_k v_k \in \Vect(\mathcal{B})
			\]
		\item Soient $(\lambda_1, \ldots, \lambda_p, \mu_1, \ldots, \mu_q, \nu_1, \ldots, \nu_r) \in \mathbbm{K}^{p+q+r}$.\\
			On suppose \[
				(*)\quad \sum_{i=1}^{p}\lambda_ie_i + \sum_{j=1}^q\mu_ju_j + \sum_{k=1}^r \nu_k v_k = 0_E
			\] 
			D'où, \[
				\underbrace{\sum_{i=1}^p\lambda_i e_i + \sum_{j=1}^q \mu_ju_j}_{\in F} = \underbrace{-\sum_{k=1}^r\nu_jv_k}_{\in G}
			\] 
			Donc, \[
				f = \sum_{i=1}^p \lambda_i e_i + \sum_{j=1}^q \mu_j u_j \in F\cap G
			\] Comme $(e_1, \ldots, e_p)$ est une base de $F\cap G$, $\exists ! (\lambda_1', \ldots, \lambda_p') \in \mathbbm{K}^p$ tel que \[
				f = \sum_{i=1}^p \lambda'_i e_i = \sum_{i=1}^p \lambda'_i e_i + \sum_{j=1}^q 0_\mathbbm{K}u_j
			\] Comme $(e_1, \ldots, e_p, u_1, \ldots, u_q)$ est une base de $F$, \[
				\forall k \in \left\llbracket 1, q \right\rrbracket, \mu_j = 0_\mathbbm{K}
			\] De même, \[
				\forall k \in \left\llbracket 1,r \right\rrbracket , \nu_k = 0_\mathbbm{K}
			\] On remplace dans $(*)$ pour trouver \[
				\sum_{i=1}^p \lambda_ie_i = 0_E
			\] Comme $(e_1, \ldots, e_p)$ est libre, \[
				\forall i \in \left\llbracket 1,p \right\rrbracket, \lambda_i = 0_\mathbbm{K}
			\] Donc $\mathcal{B}$ est libre.\\
			Donc, 
			\begin{align*}
				\dim(F+G) &=  p +q + r \\
				&= (p+q)+ (p+r) - p \\
				&= \dim(F) + \dim(G) - \dim(F\cap G) \\
			\end{align*}
	\end{itemize}
\end{prv}

\begin{crlr}
	Avec les hypothèse précédentes, \[
		E = F \oplus G \iff \begin{cases}
			F \cap  G = \{0_E\} \\
			\dim(E) = \dim(F) + \dim(G)
		\end{cases}
	\] 
\end{crlr}

\begin{prv}
	\begin{itemize}
		\item[``$\implies$''] On suppose $E = F \oplus G$ \\
			Comme la somme est directe, $F \cap G = \{0_E\}$ 
			\begin{align*}
				\dim(E) &= \dim(F)\\
				&= \dim(F) + \dim(G) - \dim(F\cap G)\\
				&= \dim(F) + \dim(G)\\
			\end{align*}
		\item[``$\impliedby$''] On suppose $F\cap G = \{0_E\}$ et $\dim(E) = \dim(F) + \dim(G)$.\\
			On sait déjà que $F+G = F \oplus G$\\
			 \begin{align*}
				\dim(F+G) = \dim(F) + \dim(G) - \dim(F \cap G) = \dim(E)
			\end{align*}
			Donc $F + G = E$
	\end{itemize}
\end{prv}

\begin{prop}
	Soit $F$ un $\mathbbm{K}$-espace vectoriel de dimension finie $n$. Soit $\mathcal{B} = (e_1, \ldots, e_n)$ une base de $F$. L'application
	\begin{align*}
		f: \mathbbm{K}^n &\longrightarrow F \\
		(\lambda_1, \ldots, \lambda_n) &\longmapsto \sum_{i=1}^n \lambda_i e_i
	\end{align*} est bijective.\\
	Si $\mathbbm{K}$ est infini, $\mathbbm{K}^n$ aussi et donc $F$ aussi.\\
	Si $\#\mathbbm{K} = p \in \N_*$,
	\begin{align*}
		\#&\mathbbm{K}^n = p^n\\
		&\vrt=\\
		\#&F
	\end{align*}
\end{prop}


		\part{Dérivation}

\underline{Motivation}:

{
\begin{wrapfigure}{l}{3cm}
	\centering
	\begin{asy}
		import three;

		size(3cm);
		settings.render=0;
		settings.prc=false;
		currentprojection = obliqueZ;

		draw(unitbox);
		draw(shift(1.1Z + 0.05X) * (O -- X), Arrows3(TeXHead2));
		draw(shift(1.1Z + 0.05Y) * (O -- Y), Arrows3(TeXHead2));
		draw(shift(1.1X + 0.05Z) * (O -- Z), Arrows3(TeXHead2));

		label("$x$", (X/2) + (1.1Z + 0.05X), align=S);
		label("$y$", (Y/2) + (1.1Z + 0.05Y), align=W);
		label("$z$", (Z/2) + X, align=SE);
	\end{asy}
\end{wrapfigure}

\begin{align*}
	&S(x,y,z) = 2(xy + xz + yz)\\
	&V(x,y,z) = xyz
\end{align*}

On cherche à minimiser $S$ avec la contrainte $V = 1$.

Soit $f : \begin{array}{rcl}
	\left( \R_*^+ \right)^2 &\longrightarrow& \R \\
	(x,y) &\longmapsto& S\left( x,y,\frac{1}{xy} \right) = 2\left( xy + \frac{1}{y} + \frac{1}{x} \right).
\end{array}$

On cherche $(a,b) \in \left( \R^+_* \right)^2$ tel que \[
	\forall (x,y) \in (\R^+_*), f(x,y) \ge f(a,b).
\]
}

\begin{defn}
	Soit $f: U \to \R$ où $U$ est un ouvert de $\R^2$. Soit $(a,b) \in U$.
	\vspace{2mm}

	Si $\lim_{x \to a} \frac{f(x,b) - f(a,b)}{x - a} \in \R$, alors on dit que $f$ a une dérivée partielle suivant $x$ en $(a,b)$ et cette limite est notée \[
		\partial f_1(a,b) = \frac{\partial f}{\partial x}(a,b).
	\]

	Si $\lim_{y \to b} \frac{f(a,y) - f(a,b)}{y - b} \in \R$, alors on dit que $f$ a une dérivée partielle suivant $y$ et la limite est notée \[
		\partial f_2(a,b) = \frac{\partial f}{\partial y}(a,b).
	\]
\end{defn}

\begin{exm}
	\begin{enumerate}
		\item $f: (x,y) \mapsto xy + x - y$.

			\begin{align*}
				&\frac{\partial f}{\partial x} : (x,y) \mapsto y + 1,\\
				&\frac{\partial f}{\partial y} : (x,y) \mapsto x - 1.
			\end{align*}

		\item $f: (x,y) \mapsto xy + \frac{1}{y}+ \frac{1}{x}$.

			\begin{align*}
				&\frac{\partial f}{\partial x}: (x,y) \mapsto y - \frac{1}{x^2},\\
				&\frac{\partial f}{\partial y}: (x,y) \mapsto x - \frac{1}{y^2}.
			\end{align*}

		\item Trouver $f$ telle que $\begin{cases}
				(1): \qquad \frac{\partial f}{\partial x}=y,\\[2mm]
				(2): \qquad \frac{\partial f}{\partial y} = x.
			\end{cases}$

			D'après $(1)$ : \[
				\forall (x,y), \exists C(y) \in \R, f(x,y) = xy + C(y)
			\] et donc \[
				\frac{\partial f}{\partial y}(x,y) = x + C'(y)
			\] donc $C'(y) = 0$ et donc $C$ est constante.

		\item Trouver $f$ telle que $\begin{cases}
			\frac{\partial f}{\partial x} = -y,\\[2mm]
			\frac{\partial f}{ƒ\partial y} = x.
		\end{cases}$

		Ce n'est pas possible !
	\end{enumerate}
\end{exm}

\begin{defn}~\\
	\begin{minipage}{\linewidth}
		\begin{wrapfigure}{r}{4cm}
			\centering
			\vspace{-5mm}
			\begin{asy}
				import three;
				import graph3;
				size(4cm);

				settings.render = 0;
				settings.prc = false;
				currentprojection = obliqueX;

				draw(O -- X, Arrow3(TeXHead2));
				draw(O -- Y, Arrow3(TeXHead2));
				draw(O -- Z, Arrow3(TeXHead2));

				triple f(real x, real y, real z = 0) { return (x,y,cos(x - 0.5) * cos(y - 0.5)/1.2 + 0.15); }

				real inc = 1 / 5;

				for(real x = 0; x <= 1; x += inc) {
					draw(graph(
						new real(real t) { return x; }, // x
						new real(real y) { return y; }, // y
						new real(real y) { return f(x,y).z; }, // z
						0, 1
					), gray);
				}

				for(real y = 0; y <= 1; y += inc) {
					draw(graph(
						new real(real x) { return x; }, // x
						new real(real t) { return y; }, // y
						new real(real x) { return f(x,y).z; }, // z
						0, 1
					), gray);
				}

				path3 path1 = (0.8, 0.2, 0) .. (0.5, 0.5, 0) .. (0.3, 0.7, 0);
				path3 path2 = f(0.8, 0.2, 0) .. f(0.5, 0.5, 0) .. f(0.3, 0.7, 0);
				path3 d = (0.2, 0.3, 0) .. (0.3, 0.4, 0) .. (0.2, 0.7, 0) .. (0.8, 0.9, 0) .. (0.6, 0.2, 0) .. cycle;

				draw(path1, red, Arrow3(TeXHead2));
				draw(path2, red, Arrow3(TeXHead2, position=0.8));

				dot((0.5, 0.5, 0));
				dot(f(0.5, 0.5, 0));
				draw((0.5, 0.5, 0) -- f(0.5, 0.5, 0), dashed);
				draw(d);

				label("$w$", (0.3, 0.7, 0), red, align=SE);
				label("$U$", (0.8, 0.9, 0), align=SE);
			\end{asy}
		\end{wrapfigure}

		Soit $f: U \to \R$ où $U$ est un ouvert. Soit $(a,b) \in U$. Soit $w = (w_1, w_2) \in \R^2$.

		Si 
		\[
			\lim_{t\to 0} \frac{f(a + tw_1, b + tw_2) - f(a,b)}{t}
		\] existe et est réelle, alors on dit que $f$ a une dérivée dans la direction de $w$ et la limite est notée \[
			\mathrm{d}f(w)\,(a,b) = D_w(f)\,(a,b).
		\]
	\end{minipage}
\end{defn}

\begin{exm}
	\begin{align*}
		f: \left( \R_*^+ \right)^2 &\longrightarrow \R \\
		(x,y) &\longmapsto xy+\frac{1}{x}+\frac{1}{y}.
	\end{align*}

	On pose $(a,b) = (1,2)$, $w = (w_1, w_2) = (1,1)$.
	\begin{align*}
		\frac{f(1+t, 2+t) - f(1,2)}{t} &= \frac{1}{t} \left( (1+t)(2+t) + \frac{1}{1+t} + \frac{1}{2+t} - 3 - \frac{1}{2} \right) \\
		&= \frac{1}{t}\left(\cancel 2 + 3t + \po(t) + \cancel 1 - t + \po(t) + \frac{1}{2}\left( \cancel 1 - \frac{t}{2} + \po(t) \right) - \cancel3 - \cancel{\frac{1}{2}} \right) \\
		&= \frac{1}{t} \left( \frac{7}{4} t + \po(t) \right)  \\
		&= \frac{7}{4} + \po(1) \tendsto{t \to 0} \frac{7}{4}. \\
	\end{align*}

	Donc, \[
		\mathrm{d}f(1,1)\,(1,2) = \frac{7}{4}.
	\]
\end{exm}

\begin{rmk}~\\
	\begin{figure}[H]
		\centering
		\begin{asy}
			import solids;
			import graph;
			size(5cm);

			settings.render = 0;
			settings.prc = false;

			path3 par = graph(
				new real(real x) { return x; },
				new real(real x) { return 0; },
				new real(real x) { return x^2; },
				0,3);
			revolution r = revolution(par, axis=Z);

			path3 par2 = graph(
				new real(real x) { return x; },
				new real(real x) { return 0; },
				new real(real x) { return x^2; },
				-3,3);

			draw(r,1,longitudinalpen=nullpen);
			draw(r.silhouette());

			draw((-4, 0, -1) -- (-4, 0, 10) -- (4, 0, 10) -- (4, 0, -1) -- cycle, red);
			draw(par2, deepred);

			draw((4,4.5) -- (7, 4.5), black+0.5mm, Arrow(TeXHead));

			path par2d = graph(new real(real x) { return x^2; }, -3, 3);
			draw(shift((11, 0)) * par2d, deepred);

			dot(O);
			dot((11, 0));
		\end{asy}
	\end{figure}
\end{rmk}


%todo ajouter théorème-définition
\begin{thm}
	Soit $f : U \to \R$, $(a,b) \in U$. On suppose que $\frac{\partial f}{\partial x}$ et $\frac{\partial f}{\partial y}$ existent en $(a,b)$ et sont {\bfseries continues} en $(a,b)$. Alors,
	\begin{align*}
		&\forall (h, k) \in \R^2 \text{ tel que } (a +h, b + k) \in U,\\
		&f(a+ h, b + k) = f(a,b) + h \frac{\partial f}{\partial x}(a,b) + k \frac{\partial f}{\partial y}(a,b) + \po_{(h,k)\to (0,0)}\big(\|(h,k)\|\big).
	\end{align*}

	On dit que $f$ est de classe $\mathcal{C}^1$ si $\frac{\partial f}{\partial x}$ et $\frac{\partial f}{\partial y}$ existent et sont continues.

	\qed
\end{thm}

\begin{rmk}
	En physique, cette formule correspond à : \[
		\mathrm{d}f = \frac{\partial f}{\partial x}\mathrm{d}x + \frac{\partial f}{\partial y} \mathrm{d}y.
	\] En effet :
	\begin{align*}
		\mathrm{d}f &= f(x+ \mathrm{d}x, y + \mathrm{d}y) - f(x,y) \\
		&= \frac{\partial f}{\partial x} \mathrm{d}x + \frac{\partial f}{\partial y} \mathrm{d}y.
	\end{align*}
\end{rmk}

\begin{prop}
	Soit $f: U \to \R$ de classe $\mathcal{C}^1$ en $(a,b) \in U$. Alors, \[
		\forall w = (w_1, w_2) \in \R^2, \mathrm{d}f(w)\,(a,b) = w_1 \frac{\partial f}{\partial x}(a,b) + w_2 \frac{\partial f}{\partial y}(a,b).
	\]
\end{prop}

\begin{prv}
	Soit $w = (w_1, w_2) \in \R^2$. Soit $t \in \R^*$.
	\begin{align*}
		\frac{1}{t}\big(f(a + tw_1, b + tw_2) - f(a,b)\big)
		&= \frac{1}{t} \left( tw_1 \frac{\partial f}{\partial x}(a,b) + tw_2 \frac{\partial f}{\partial y}(a,b) + \po_{t \to 0}\big(\|tw\|\big) \right) \\
		&= w_1 \frac{\partial f}{\partial x}(a,b) + w_2 \frac{\partial f}{\partial y}(a,b) + \po_{t\to 0}(1) \\
		&\tendsto{t\to 0} w_1 \frac{\partial f}{\partial x}(a,b) + w_2\frac{\partial f}{\partial y}(a,b).
	\end{align*}
\end{prv}


\begin{defn}
	Avec les hypothèses précédentes, en posant \[
		\nabla f(a,b) = \left( \frac{\partial f}{\partial x}(a,b), \frac{\partial f}{\partial y}(a,b) \right) 
	\]on obtient \[
		\mathrm{d}f(w)\,(a,b) = \left<w  \mid \nabla f(a,b) \right>
	\] où $\left<\cdot|\cdot \right>$ est le produit scalaire.

	Le vecteur $\nabla f(a,b)$ est appelé \underline{gradient de $f$ en $(a,b)$}.

	Le développement limité à l'ordre 1 de $f$ devient \[
		f\big((a,b)+w\big) = f(a,b) + \left<w \mid \nabla f(a,b) \right> + \po_{w\to 0}(\|w\|)
	\]
\end{defn}

\begin{prop}
	Soit $f : U \to \R$ de classe $\mathcal{C}^1$.

	\begin{figure}[H]
    \centering
    \incfig{gradient}
	\end{figure}

	$\nabla f$ est orthogonal au lignes de niveaux de $f$, son orientation va dans le sens d'une augmentation de $f$.
\end{prop}

\begin{prv}
	Soit $\gamma : I \to U$ une courbe de niveau : \[
		\forall t \in I, f\big(\gamma(t)\big) = \text{cste}.
	\] D'après le lemme suivant : \[
		\forall t \in I, 0 = (f \circ \gamma)'(t) = \mathrm{d}f\big(\gamma'(t)\big)\big(\gamma(t)\big) = \left<\gamma'(t)  \mid \nabla f\big(\gamma(t)\big) \right>
	\] Donc $\nabla f\big(\gamma(t)\big)$ est orthogonal à $\gamma'(t)$.

	Pour tout $t \in I$, on pose $w(t) = t\, \nabla f\big(\gamma(t)\big)$. Donc \[
		f\big(\gamma(t) + w(t)\big) = f\big(\gamma(t)\big) + t \|\nabla f(\gamma(t))\|^2 + \po_{t \to 0}(t)
	\] Pour $t$ assez petit, $f\big(\gamma(t) + w(t)\big) - f\big(\gamma(t)\big)$ est du même signe que $t$.
\end{prv}

\begin{rmk}
	\begin{align*}
		V: \R^3 &\longrightarrow \R \\
		(x,y,z) &\longmapsto -mgz
	\end{align*}
	l'énerge potentielle de pesenteur

	On a donc \[
		\nabla V(x,y,z) = \left( \frac{\partial V}{\partial x}, \frac{\partial V}{\partial y}, \frac{\partial V}{\partial z} \right) = (0, 0, -mg) = \vec{P}.
	\]
\end{rmk}

\begin{lem}
	Soit $f : U \to \R$ de classe $\mathcal{C}^1$, $\gamma : \begin{array}{rcl}
		I &\longrightarrow& U \\
		t &\longmapsto& \big(x(t), y(t)\big)
	\end{array}$ où $x$ et $y$ sont dérivables.

	On pose \[
		\forall t \in I, \gamma'(t) = \big(x'(t), y'(t)\big).
	\] Alors $f \circ \gamma : I \to \R$ est dérivable et
	\begin{align*}
		\forall t \in I, (f \circ \gamma)'(t) &= \mathrm{d}f\big(\gamma'(t)\big) \big(\gamma(t)\big)\\
		&= \left<\gamma'(t)  \mid \nabla f\big(\gamma(t)\big)  \right> \\
		&= x'(t) \frac{\partial f}{\partial x}\big(x(t), y(t)\big) + y'(t) \frac{\partial f}{\partial y}\big(x(t),y(t)\big). \\
	\end{align*}
\end{lem}

\begin{prv}
	On fixe $t \in I$.

	\begin{align*}
		\forall h \neq 0, \frac{f \circ \gamma(t + h) - f \circ \gamma(t)}{h}
		&= \frac{1}{h}\big(f(\gamma(t)) + h\gamma'(t) + \po_{h\to 0}(h) - f(\gamma(t))\big) \\
		&= \frac{1}{h}\bigg(\cancel{f(\gamma(t))} + \left<h\gamma'(t) \mid \nabla f(\gamma(t)) \right> + \po_{h\to 0}(\|h\gamma'(t)\|) - \cancel{f(\gamma(t))}\bigg)\\
		&= \left<\gamma'(t) \mid \nabla f(\gamma(t)) \right> + \po_{h\to 0}(1) \\
		&\tendsto{h\to 0} \left<\gamma'(t)  \mid \nabla f(\gamma(t)) \right>
	\end{align*}
\end{prv}

\begin{defn}
	Soit $f : U \to \R$ de classe $\mathcal{C}^1$ et $(a,b) \in U$. On dit que $(a,b)$ est un \underline{point critique} de $f$ si $\nabla f(a,b) = 0$ i.e. $\frac{\partial f}{\partial x}(a,b) = \frac{\partial f}{\partial y}(a,b) = 0$.

	Dans ce cas, $f(a,b)$ est appelé \underline{valeur critique} de $f$.
\end{defn}

\begin{prop}~\\
	\begin{minipage}{\linewidth}
		\begin{wrapfigure}{r}{3cm}
			\centering
			\vspace{-1cm}
			\begin{asy}
				import solids;
				import graph;
				size(3cm);

				settings.render = 0;
				settings.prc = false;

				path3 par = graph(
					new real(real x) { return x; },
					new real(real x) { return 0; },
					new real(real x) { return -x^2; },
					0,3);
				revolution r = revolution(par, axis=Z);

				draw(r,1,longitudinalpen=nullpen);
				draw(r.silhouette());

				dot("$(a,b)$", O, red, align=N);
				real s = sqrt(2.5);
				path3 g=(s,0,-2.5)..(0,s,-2.5)..(-s,0,-2.5)..(0,-s,-2.5)..cycle;
				draw(g, deepcyan);
			\end{asy}
		\end{wrapfigure}
		Soit $f: U \to \R$ de classe $\mathcal{C}^1$ et $(a,b) \in U$ tel que \[
			\exists r > 0, \forall (x,y) \in B_{(a,b)}(r), f(x,y) \le f(a,b)
		\] Alors $\nabla f(a,b) = (0,0)$.
	\end{minipage}
\end{prop}

\begin{prv}
	Soit $g: x \mapsto f(x,b)$. $g(a)$ est un maximum local de $g$ donc $g'(a) = 0$.

	Or, $g'(a) = \frac{\partial f}{\partial x}(a,b)$

	donc $\frac{\partial f}{\partial x}(a,b) = 0$.

	Soit $h : y \mapsto f(a,y)$. On a de même $h'(b) = 0$.

	Or, $h'(b) = \frac{\partial f}{\partial y}(a,b)$.

	Donc, $\nabla f(a,b) = (0,0)$.
\end{prv}

\begin{rmk}
	Un minimum local est aussi une valeur critique.
\end{rmk}

\begin{figure}[H]
	\centering
	\begin{subfigure}{3cm}
		\centering
		\begin{asy}
			import solids;
			import graph;
			size(3cm);

			settings.render = 0;
			settings.prc = false;

			path3 par = graph(
				new real(real x) { return x; },
				new real(real x) { return 0; },
				new real(real x) { return -x^2; },
				0,3);
			revolution r = revolution(par, axis=Z);

			draw(r,1,longitudinalpen=nullpen);
			draw(r.silhouette());

			dot(O, red);
		\end{asy}
		\caption{Maximum local}
	\end{subfigure}
	\begin{subfigure}{3cm}
		\centering
		\begin{asy}
			import solids;
			import graph;
			size(3cm);

			settings.render = 0;
			settings.prc = false;

			path3 par = graph(
				new real(real x) { return x; },
				new real(real x) { return 0; },
				new real(real x) { return x^2; },
				0,3);
			revolution r = revolution(par, axis=Z);

			draw(r,1,longitudinalpen=nullpen);
			draw(r.silhouette());

			dot(O, red);
		\end{asy}
		\caption{Minimum local}
	\end{subfigure}
	\begin{subfigure}{3cm}
		\centering
		\begin{asy}
			import solids;
			import graph;
			size(3cm);

			settings.render = 0;
			settings.prc = false;
			currentprojection = obliqueZ;

			draw(graph(
				new real(real x) { return x; },
				new real(real x) { return -x^2 / 3; },
				new real(real x) { return 3; },
				-3, 3
			));

			draw(graph(
				new real(real x) { return x; },
				new real(real x) { return -x^2 / 3; },
				new real(real x) { return -3; },
				-3, 3
			));

			draw(graph(
				new real(real x) { return x; },
				new real(real x) { return -x^2 / 3 - 1; },
				new real(real x) { return 0; },
				-3, 3
			));

			draw(graph(
				new real(real x) { return 0; },
				new real(real x) { return x^2 / 9 - 1; },
				new real(real x) { return x; },
				-3, 3
			));

			draw(graph(
				new real(real x) { return -3; },
				new real(real x) { return x^2 / 9 - 4; },
				new real(real x) { return x; },
				-3, 3
			));

			draw(graph(
				new real(real x) { return 3; },
				new real(real x) { return x^2 / 9 - 4; },
				new real(real x) { return x; },
				-3, 3
			));

			dot((0,-1,0), red);
		\end{asy}
		\caption{Point de selle / Point col}
	\end{subfigure}
\end{figure}

\begin{exm}
	On revient à l'exemple donné en introduction : 
	\begin{align*}
		f: \left( \R^*_+ \right)^2 &\longrightarrow \R \\
		(x,y) &\longmapsto 2\left( xy + \frac{1}{x} + \frac{1}{y} \right).
	\end{align*}

	$\left( \R^+_* \right)^2$ est un ouvert de $\R^2$. Soit $(x,y) \in \left( \R^+_* \right)^2$.
	
	On a \[
		\begin{cases}
			\frac{\partial f}{\partial x}(x,y) = 2\left( y - \frac{1}{x^2} \right),\\
			\frac{\partial f}{\partial y}(x,y) = 2\left( x - \frac{1}{y^2} \right).
		\end{cases}
	\]

	\begin{align*}
		&\frac{\partial f}{\partial x}(x,y) = \frac{\partial f}{\partial y}(x,y) = 0\\
		\iff& \begin{cases}
			y = \frac{1}{x^2}\\
			x = \frac{1}{y^2}
		\end{cases}\\
		\iff& \begin{cases}
			y = \frac{1}{x^2}\\
			x = x^4
		\end{cases}\\
		\iff& \begin{cases}
			x = 1\\
			y = 1
		\end{cases}
	\end{align*}

	On vérivie que $f$ présente en effet un minium local en $(1,1)$. \[
		f(1,1) = 6
	\] On fixe $y \in \R^+_*$ et \[
		g : x \mapsto 2\left( xy + \frac{1}{x} + \frac{1}{y} \right).
	\] Donc \[
		\forall x \in \R^+_*, g'(x) = 2\left( y - \frac{1}{x^2} \right).
	\]
	\begin{center}
		\begin{tikzpicture}
			\tkzTabInit{$x$/1,$g'(x)$/1,$g$/2.3}{$0$, $\frac{1}{\sqrt{y}}$, $+\infty$}
			\tkzTabLine{,-,z,+,}
			\tkzTabVar{+/{}, -/$2\left( 2\sqrt{y} +\frac{1}{y} \right)$, +/{}}
		\end{tikzpicture}
	\end{center}
	
	Ainsi, \[
		\forall x \in \R^+_*, \forall y \in \R^+_*, f(x,y) \ge 2\left( 2\sqrt{y} + \frac{1}{y} \right)
	\] Soit $h : y \mapsto 2\sqrt{y} + \frac{1}{y}$. On a \[
		\forall y > 0, h'(y) = \frac{1}{\sqrt{y}} - \frac{1}{y^2} = \frac{y\sqrt{y} - 1}{y^2} = \frac{y^{\frac{3}{2}} - 1}{y^2}
	\]

	\begin{center}
		\begin{tikzpicture}
			\tkzTabInit{$y$/0.7,$h'(y)$/0.7,$h$/1.4}{$0$, $1$, $+\infty$}
			\tkzTabLine{,-,z,+,}
			\tkzTabVar{+/{}, -/$3$, +/{}}
		\end{tikzpicture}
	\end{center}

	Donc, \[
		\forall x,y > 0, f(x,y) \ge 2\times 3 = 6 = f(1,1).
	\]
\end{exm}

\begin{prop}
	[règle de la chaîne]

	Soit $f : \begin{array}{rcl}
		U &\longrightarrow& \R^2 \\
		(x,y) &\longmapsto& f(x,y)
	\end{array}$ de classe $\mathcal{C}^1$ et $U, V$ deux ouverts de $\R^2$.

	Soit $\varphi : \begin{array}{rcl}
		V &\longrightarrow& U \\
		(u,v) &\longmapsto& \varphi(u,v) = \big(x(u,v), y(u,v)\big)
	\end{array}$.

	On suppose que $x$ et $y$ sont de classe $\mathcal{C}^1$ sur $V$.

	Alors,  $f \circ \varphi : \begin{array}{rcl}
		V &\longrightarrow& \R \\
		(u,v) &\longmapsto& f\big(\varphi(u,v)\big)
	\end{array}$ est de classe $\mathcal{C}^1$ et
	\begin{align*}
		\forall (u_0, v_0) \in V, \frac{\partial (f \circ \varphi)}{\partial u}(u_0, v_0)
		&= \frac{\partial f}{\partial x}\big(\varphi(u_0, v_0)\big) \times \frac{\partial x}{\partial u}(u_0, v_0)\\
		&+ \frac{\partial f}{\partial y}\big(\varphi(u_0,v_0)\big) \frac{\partial y}{\partial u}(u_0,v_0)
	\end{align*}
	\begin{align*}
		\forall (u_0, v_0) \in V, \frac{\partial (f \circ \varphi)}{\partial v}(u_0, v_0)
		&= \frac{\partial f}{\partial x}\big(\varphi(u_0, v_0)\big) \times \frac{\partial x}{\partial v}(u_0, v_0)\\
		&+ \frac{\partial f}{\partial y}\big(\varphi(u_0,v_0)\big) \frac{\partial y}{\partial v}(u_0,v_0)
	\end{align*}
\end{prop}

\begin{exm}
	[changement de coordonnées polaires]
	On pose \begin{align*}
		\varphi: \R^+_* \times ]0,2\pi[ &\longrightarrow \R^2\setminus \left( R^+_* \times \{0\} \right) \\
		(r, \theta) &\longmapsto (r \cos \theta, r \sin\theta),
	\end{align*}
	\begin{align*}
		f: \R^2\setminus \left( R^+_* \times \{0\} \right) &\longrightarrow \R \\
		(x,y) &\longmapsto f(x,y),
	\end{align*}
	\begin{align*}
		g: \overbrace{\R^+_* \times ]0, 2\pi[}^{=V} &\longrightarrow \R \\
		(r, \theta) &\longmapsto f(r\cos\theta, r\sin\theta).
	\end{align*}

	\begin{align*}
		\forall (r_0,\theta_0) \in V,&\\[5mm]
		\frac{\partial g}{\partial r}(r_0, \theta_0) &= \frac{\partial f}{\partial x}(r_0\cos\theta_0, r_0\sin\theta_0)\cos\theta_0\\
		&+ \frac{\partial f}{\partial y}(r_0 \cos\theta_0, r_0\sin\theta_0)\sin\theta_0\\
		&= 2r_0\cos^2\theta_0 + 2r_0\sin^2(\theta_0) \\
		&= 2r_0 \\[5mm]
		\frac{\partial g}{\partial \theta}(r_0, \theta_0) &= \frac{\partial f}{\partial x}(r_0\cos\theta_0, r_0\sin\theta_0)r_0\sin\theta_0\\
		&+ \frac{\partial f}{\partial y}(r_0 \cos\theta_0, r_0\sin\theta_0)r_0\cos\theta_0\\
		&= -2{r_0}^2\cos(\theta_0)\sin(\theta_0) + 2{r_0}^2 \sin(\theta_0)\cos(\theta_0)\\
		&= 0 \\
	\end{align*}

	Donc, \[
		g(r, \theta) = r^2.
	\]
\end{exm}

\begin{exm}
	Résoudre \[
		\begin{cases}
			\frac{\partial f}{\partial x} = \frac{x}{x^2+y^2},\\
			\frac{\partial f}{\partial y} = \frac{y}{x^2+y^2}.\\
		\end{cases}
	\]

	On pose $g: (r, \theta) \mapsto f(r \cos\theta, r \sin\theta)$.

	\begin{align*}
		&\frac{\partial g}{\partial r} = \frac{1}{r}\cos^2\theta + \frac{1}{r}\sin^2\theta = \frac{1}{r},\\
		&\frac{\partial g}{\partial \theta} = -\cos(\theta) \sin(\theta) + \sin(\theta)\cos(\theta) = 0.
	\end{align*}

	Donc, \[
		\exists C \in \R, g: (r, \theta) \mapsto \ln r + C
	\] d'où,
	\begin{align*}
		\forall (x,y) \in \R^2 \setminus \{(0,0)\}, f(x,y) &= \ln\left(\sqrt{x^2 + y^2} \right)  + C\\
		&= \frac{1}{2}\ln(x^2 + y^2) + C. \\
	\end{align*}
\end{exm}

\begin{rmk}
	Soit $\mathcal{B} = (e_1, e_2)$ la base canonique de $\R^2$, $f: U \to \R$ de classe $\mathcal{C}^1$ avec $U$ un ouvert de $\R^2$.

	Soit $(x,y) \in U$.

	\begin{align*}
		\Mat_{\mathcal{B}}\big(\nabla f(x,y)\big) = \begin{pmatrix}
			\frac{\partial f}{\partial x}(x,y)\\[2mm]
			\frac{\partial f}{\partial y}(x,y)
		\end{pmatrix}
	\end{align*}

	Soit  \begin{align*}
		\varphi: V &\longrightarrow U \\
		(u,v) &\longmapsto \big(x(u,v), y(u,v)\big) 
	\end{align*} avec $x,y$ de classe $\mathcal{C}^1$. Soit $g = f \circ \varphi$.
	\begin{align*}
		\Mat_{\mathcal{B}}\big(\nabla g(u,v)\big)
		&= \begin{pmatrix}
			\frac{\partial g}{\partial u}(u,v) \\[2mm]
			\frac{\partial g}{\partial v}(u,v)
		\end{pmatrix} \\
		&= \begin{pmatrix}
			\frac{\partial x}{\partial u}(u,v) \frac{\partial f}{\partial x}(x,y)
			+ \frac{\partial y}{\partial u}(u,v)\frac{\partial f}{\partial y}(x,y)\\[3mm]
			\frac{\partial x}{\partial v}(u,v) \frac{\partial f}{\partial x}(x,y)
			+ \frac{\partial y}{\partial v}(u,v) \frac{\partial f}{\partial y}(x,y)
		\end{pmatrix}  \\
		&= \underbrace{\begin{pmatrix}
				\frac{\partial x}{\partial u}(u,v)& \frac{\partial y}{\partial u}(u,v)\\[3mm]
				\frac{\partial x}{\partial v}(u,v)& \frac{\partial y}{\partial v}(u,v)
		\end{pmatrix}}_{J(u,v)} \begin{pmatrix}
			\frac{\partial f}{\partial x}(x,y)\\[3mm]
			\frac{\partial f}{\partial y}(x,y)
		\end{pmatrix} \\
		&= J(u,v) \Mat_{\mathcal{B}}\big(\nabla f(x,y)\big) \\
	\end{align*}
	où $J(u,v) = 
	\begin{pNiceArray}{c:c}
		\Mat_{\mathcal{B}}\big(\nabla x(u,v)\big) & \Mat_{\mathcal{B}}\big(\nabla y(u,v)\big)
	\end{pNiceArray}$.

	On dit que $J(u,v)$ est \underline{la jacobienne} de $\varphi$ en $(u,v)$.
	L'application linéaire canoniquement associée à $J(u,v)$ est la \underline{différentielle de $\varphi$} en $(u,v)$ noté $\mathrm{d}\varphi(u,v)$.

	On a $\mathrm{d}\varphi(u,v) \in \mathcal{L}(R^2)$ et $\Mat_{\mathcal{B}}\big(\mathrm{d}\varphi(u,v)\big) = J(u,v)$.

	Par exemple, la jacobienne du changement de coordonnées polaires est \[
		J = \begin{pmatrix}
			\frac{\partial x}{\partial r} & \frac{\partial y}{\partial r}\\[3mm]
			\frac{\partial x}{\partial \theta} & \frac{\partial y}{\partial \theta}
		\end{pmatrix}
		= \begin{pmatrix}
			\cos\theta&\sin\theta\\
			-r\sin\theta&r\cos\theta
		\end{pmatrix}.
	\]
	$\underbrace{\det(J)}_{\text{le jacobien}} = r\cos^2\theta + r\sin^2\theta = r$

	Dans une intégrale double, si $(x,y) = \varphi(u,v)$, alors $\mathrm{d}x\mathrm{d}y = \det(J)\mathrm{d}u\mathrm{d}v$.

	Ici, \[
		\mathrm{d}x\ \mathrm{d}y = r\ \mathrm{d}r\ \mathrm{d}\theta.
	\]
\end{rmk}

\begin{prv}
	On pose $(x_0, y_0) = \varphi(u_0, v_0)$. Pour tout $(h,k) \in \R^2$ tels que $(u_0 + h, v_0 + k) \in V$, en posant $g = f  \circ \varphi$.

	\begin{align*}
		g(u_0 + h, v_0 + h) &= f\big(x(u_0 + h, v_0 + k), y(u_0 + h, v_0 + k)\big) \\
		&= f\left(
			x(u_0,v_0) + h \frac{\partial x}{\partial u}(u_0,v_0) + k \frac{\partial x}{\partial v}(u_0, v_0) + \po\big(\|(h,k)\|\big), \right.\\
		&\phantom{ = f\bigg(\bigg.}\left. y(u_0, v_0) + h \frac{\partial y}{\partial u}(u_0, v_0) + k \frac{\partial y}{\partial v}(u_0, v_0) + \po\big(\|(h,k)\|\big)
		\right)  \\
		&= f(x_0,y_0) \\
		&~+ \left( h \frac{\partial x}{\partial u}(u_0,v_0) + k \frac{\partial x}{\partial v}(u_0, v_0) + \po(\|(h,k)\|) \right) \frac{\partial f}{\partial x}(x_0,y_0)\\
		&~+ \left( h \frac{\partial y}{\partial u}(u_0, v_0) + k\frac{\partial y}{\partial v}(u_0, v_0) + \po(\|(h,k)\|) \right) \frac{\partial f}{\partial y}(x_0, y_0)\\
		&~+ \po(\|(h,k)\|)\\
		&= f(x_0, y_0) \\
		&~+ h \left( \frac{\partial x}{\partial u}(u_0, v_0) \frac{\partial f}{\partial x}(x_0, y_0) + \frac{\partial y}{\partial u}(u_0, v_0) \frac{\partial f}{\partial y}(x_0, y_0) \right)  \\
		&~+ k\left( \frac{\partial x}{\partial v}(u_0, v_0) \frac{\partial f}{\partial x}(x_0, y_0) + \frac{\partial y}{\partial v}(u_0, v_0) \frac{\partial f}{\partial y}(x_0, y_0) \right) 
		&~+ \po(\|(h,k)\|)\\
		&= g(u_0, v_0) + h \frac{\partial g}{\partial u}(u_0, v_0) + k \frac{\partial g}{\partial v}(u_0, v_0) + \po(\|(h,k)\|) \\
	\end{align*}

	Par identification,
	\[
		\frac{\partial g}{\partial u}(u_0, v_0) = \frac{\partial x}{\partial u}(u_0, v_0) \frac{\partial f}{\partial x}(x_0, y_0) + \frac{\partial y}{\partial u}(u_0, v_0) \frac{\partial f}{\partial y}(x_0,y_0)
	\] et \[
		\frac{\partial g}{\partial v}(u_0, v_0) = \frac{\partial x}{\partial v}(u_0,v_0) \frac{\partial f}{\partial x}(x_0, y_0) + \frac{\partial y}{\partial v}(u_0, v_0) \frac{\partial f}{\partial y}(x_0, y_0).
	\] 
\end{prv}

\begin{exm}
	[Régression linéaire]~\\
	\begin{figure}[H]
		\centering
		\begin{asy}
			import graph;
			axes(EndArrow);
			size(5cm);

			real f(real x) { return x + 0.5; }

			real k = 35 / (7 - 0.5);

			for(int i = 0; i < 35; ++i) {
				real mag = exp(sin(100 * pi/exp(1) * i)) * 0.8 + exp(cos(i*40)/3);
				real eps = mag * cos(10 * exp(1)/pi * i) / 3;
				dot((i/k,f(i/k) + eps));
			}

			draw(graph(f, -1, 7), orange);
		\end{asy}
	\end{figure}
	\[
		y = a x + b
	\] 
	On fixe $(a,b) \in \R^2$. \[
		\varepsilon(a,b) = \sum_{i=1}^n\big( y_i - (ax_i + b) \big)^2
	\] l'erreur totale.

	On veut minimiser $\varepsilon(a,b)$. On a 
	\[
		\forall (a,b) \in \R^2,
		\begin{cases}
			\frac{\partial \varepsilon}{\partial a}(a,b) = -2\sum_{i=1}^{n}(y_i - ax_i - b)x_i,\\
			\frac{\partial \varepsilon}{\partial b}(a,b) = -2\sum_{i=1}^{n}(y_i - ax_i - b).
		\end{cases}
	\]

	Donc,
	\begin{align*}
		(a,b) \text{ point critique de } \varepsilon \iff& \begin{cases}
			a \sum_{i=1}^n {x_i}^2 + b\sum_{i=1}^{n}x_i = \sum_{i=1}^{n} y_ix_i\\
			a\sum_{i=1}^{n}x_i + nb = \sum_{i=1}^ny_i
		\end{cases}\\
		\iff& \begin{cases}
			a \left( \frac{1}{n}\sum_{i=1}^n {x_i}^2 - \overline{x}^2\right) = \overline{y} - \overline{x} \overline{y}\\
			b = \frac{1}{n}\sum_{i=1}^ny_i - \frac{a}{n}\sum_{i=1}^nx_i = \frac{1}{n}\sum_{i=1}^n x_i y_i - \overline{x} \overline{y}
		\end{cases}\\
		&\text{ où } \overline{x} = \frac{1}{n} \sum_{i=1}^n x_i,~\overline{y} = \frac{1}{n}\sum_{i=1}^n y_i\\
		\iff& \begin{cases}
			a = \frac{\Cov(x,y)}{V(x)}\\
			b = \overline{y} - a\overline{x}
		\end{cases}
	\end{align*}

	Coefficient de corrélation: $\frac{\Cov(x,y)}{\sigma_x \sigma_y} \in [-1, 1]$
\end{exm}












	}

	{
		\chap[04]{Fonctions usuelles}
		\renewcommand{\cwd}{../chap04}
		\begin{defn}
	Soit $E$ un $\mathbbm{K}$-espace vectoriel. On dit que $E$ est de \underline{dimension finie} si $E$ a au moins une famille génératrice finie. On dit que $E$ est de \underline{dimension infinie} sinon.
	\index{dimension finie (espace vectoriel)}
	\index{dimension infinie (espace vectoriel)}
\end{defn}

\begin{thm}
	[Théorème de la base extraite]
	Soit $E$ un $\mathbbm{K}$-espace vectoriel non nul de dimension finie. Soit $\mathcal{G}$ une famille génératrice finie de $E$. Alors, il existe une base $\mathcal{B}$ de $\mathcal{E}$ telle que $\mathcal{B} \subset \mathcal{G}$.
\end{thm}

\begin{prv}
	[par récurrence sur $\#G = \Card(G)$]
	\begin{itemize}
		\item Soit $E$ un $\mathbbm{K}$-espace vectoriel non nul engendré par $\mathcal{G} = (u)$.\\
			Si $u = 0_E$, alors $E = \{0_E\}$: une contradiction $\lightning$ \\
			Donc $u \neq 0_E$ donc $(u)$ est libre. En effet, \[
				\forall \lambda \in \mathbbm{K}, \lambda u = 0_E \implies \lambda = 0_\mathbbm{K}
			\] Donc $\mathcal{G}$ est une base de $E$.\\
		\item Soit $n \in \N_*$. Soit $E$ un $\mathbbm{K}$-espace vectoriel. On suppose que si $E$ a une famille génératrice constituée de $n$ vecteurs, alors on peut extraire de cette famille une base de $E$.\\
			Soit $\mathcal{G}$ une famille génératrice de $E$ avec $n+1$ vecteurs.\\
			Si $\mathcal{G}$ est libre, alors $\mathcal{G}$ est une base de $E$. \\
			Si $\mathcal{G}$ n'est pas libre, alors il existe $u \in \mathcal{G}$ tel que $u \in \Vect(\mathcal{G}\setminus \{u\})$ \\
			Donc $\mathcal{G}\setminus \{u\}$ engendre $E$. Or, $\mathcal{G}\setminus \{u\}$ possède $n$ vecteurs. D'après l'hypothèse de récurrence, il existe une base $\mathcal{B}$ de $E$ telle que \[
				\mathcal{B} \subset \mathcal{G} \setminus \{u\} \subset \mathcal{G}
			\] 
	\end{itemize}
\end{prv}

\begin{crlr}
	Tout espace de dimension finie a une base.
	\qed
\end{crlr}

\begin{thm}
	[Théorème de la base incomplète]
	Soit $E$ un $\mathbbm{K}$-espace vectoriel de dimension finie, $\mathcal{G}$ une famille génératrice finie de $E$. $\mathcal{L}$ une famille libre de $E$. Alors, il existe une base $\mathcal{B}$ de $E$ telle que \[
		\mathcal{L} \subset \mathcal{B} \text{ et } \mathcal{B}\setminus \mathcal{L} \subset \mathcal{G}
	\] 
\end{thm}

\begin{prv}
	[par récurrence sur $\#(\mathcal{G}\setminus\mathcal{L})$]
	\begin{itemize}
		\item Avec les notations précédentes, on suppose que $\mathcal{G}\setminus\mathcal{L} \neq \O$ \[
				\forall u \in \mathcal{G}, u \in \mathcal{L}
			\] Donc $\mathcal{G} \subset \mathcal{L}$ donc $\mathcal{L}$ est génératrice donc $\mathcal{L}$ est une base de $E$. On pose $\mathcal{B} = \mathcal{L}$ et alors \[
				\mathcal{L} \subset  \mathcal{B} \text{ et } \mathcal{B}\setminus\mathcal{L} = \O \subset  \mathcal{G}
			\] 
		\item Soit $n \in \N$. On suppose que si $\mathcal{G}$ est génératrice et $\mathcal{L}$ libre avec $\#(\mathcal{G}\setminus\mathcal{L}) = n$ alors il existe une base $\mathcal{B}$ de $E$ telle que \[
			\mathcal{L}\subset \mathcal{B} \text{ et } \mathcal{B}\setminus\mathcal{L}\subset \mathcal{G}
		\] Soient à présent $\mathcal{G}$ une famille génératrice de $E$ et $\mathcal{L}$ une famille libre de $E$ telles que $\#(\mathcal{G}\setminus\mathcal{L}) = n+1 > 0$\\
		Si $\mathcal{L}$ engendre $E$, alors $\mathcal{L}$ est une base de $E$. On pose $\mathcal{B} = \mathcal{L}$ et on a bien \[
			\mathcal{L} \subset  \mathcal{B} \text{ et } \mathcal{B} \setminus \mathcal{L} = \O \subset  \mathcal{G}
		\] On suppose que $\mathcal{L}$ n'engendre pas $E$. Il existe $u \in \mathcal{G}$ tel que $u \not\in \Vec(\mathcal{L})$ (car sinon, $\mathcal{G} \subset \Vect(\mathcal{L})$ et donc $\underbrace{\Vect(\mathcal{G})}_{= E} \subset  \underbrace{\Vect(\mathcal{L})}_{ \subset E}$\\
		Donc $\mathcal{L} \cup \{u\} $ est libre. On pose $\mathcal{L}' = \mathcal{L} \cup \{u\} $ \[
			\mathcal{G}\setminus \mathcal{L}' = \mathcal{G}\setminus (\mathcal{L} \cup \{u\}) = (\mathcal{G}\setminus\mathcal{L})\setminus \{u\} 
		\] donc $\#(\mathcal{G}\setminus\mathcal{L}') = n+1 -1 = n$\\
		D'après l'hypothèse de récurrence, il existe $\mathcal{B}$ une base de $E$ telle que \[
			\mathcal{L} \subset  \mathcal{L}' \subset \mathcal{B} \text{ et } \mathcal{B}\setminus \mathcal{L}' \subset \mathcal{G}
		\] \[
			\mathcal{B} \setminus \mathcal{L} = \underbrace{\mathcal{B}\setminus\mathcal{L}'}_{\subset \mathcal{G}} \cup \underbrace{\{u\}}_{\subset \mathcal{G} \text{ car } u \in \mathcal{G}}
		\] On a $\mathcal{B}\setminus\mathcal{L}\subset \mathcal{G}$
	\end{itemize}
\end{prv}

\begin{thm}
	Soit $E$ un $\mathbbm{K}$-espace vectoriel de dimension finie. Toutes les bases de $E$ ont le même cardinal.
\end{thm}

\begin{prv}
	Soit $\mathcal{G}$ une famille génératrice finie de $E$ et $\mathcal{B} \subset  \mathcal{G}$ une base de $E$. On note $n = \#\mathcal{B}$ \\
	Soit $\mathcal{B}'$ une base de $E$. On pose $p = n - \#(\mathcal{B} \cap  \mathcal{B}')$. Montrons par récurrence sur  $p$ que $\#\mathcal{B} = \#\mathcal{B}'$ 
	\begin{itemize}
		\item On suppose que $p = 0$. Alors, $\#(\mathcal{B} \cap \mathcal{B}') = n$ \\
			Or, $\mathcal{B}' \cap \mathcal{B} \subset \mathcal{B}$ donc $\mathcal{B} \cap \mathcal{B}' = \mathcal{B}$ donc $\mathcal{B} \subset  \mathcal{B}'$ et donc $\mathcal{B} = \mathcal{B}'$ 
		\item Soit $p \in \N$. On suppose que si $\mathcal{B}'$ est une base de $E$ telle que $n - \#(\mathcal{B} \cap \mathcal{B}') = p$, alors $\#\mathcal{B}' = n$ \\
			Aoit $\mathcal{B}'$ une base de $E$ telle que $n - \#(\mathcal{B}\cap \mathcal{B}') = p+1 > 0$ \\
			Donc $\mathcal{B} \cap \mathcal{B}' \neq \mathcal{B}$. Soit $u \in \mathcal{B}' \setminus \mathcal{B}$. D'après le lemme d'échange, il existe $v \in \mathcal{B}\setminus \mathcal{B}'$ tel que $\mathcal{B}' \setminus \{u\} \cup \{v\}$ est une base de $E$. On pose $\mathcal{B}'' = \mathcal{B}' \setminus \{u\} \cup \{v\}$ 
			\begin{align*}
				\mathcal{B}'' \cap \mathcal{B} &= \left( (\mathcal{B}' \setminus \{u\})  \cap \mathcal{B} \right) \cup \{v\} \\
				&= (\mathcal{B}' \cap \mathcal{B}) \cup \{v\} \\
			\end{align*}
			donc,
			\begin{align*}
				n - \#(\mathcal{B}'' \cap \mathcal{B}) &= n - (\#(\mathcal{B}' \cap \mathcal{B}) + 1) \\
				&= p+1- 1 \\
				&= p \\
			\end{align*}
			D'après l'hypothèse de récurrence, \[
				\#\mathcal{B}'' = n
			\] Or, $\#\mathcal{B}'' = \#\mathcal{B}'$
	\end{itemize}
\end{prv}

\begin{lem}
	Soient $\mathcal{B}$ et $\mathcal{B}'$ deux bases de $E$ telles que $\mathcal{B}\subset \mathcal{B}'$. Alors, $\mathcal{B} = \mathcal{B}'$.
\end{lem}

\begin{prv}
	On suppose $\mathcal{B}' \neq \mathcal{B}$. Soit $u \in \mathcal{B}' \setminus \mathcal{B}$
	$u \in E = \Vect(\mathcal{B})$ donc $\mathcal{B} \cup \{u\}$ n'est pas libre.
	Donc $\mathcal{B}\cup \{u\} \subset \mathcal{B}'$ et $\mathcal{B}'$ est libre donc $\mathcal{B}\cup \{u\}$ est libre: une contradiction $\lightning$
\end{prv}

\begin{lem}
	[Lemme d'échange] Soient $\mathcal{B}_1$ et $\mathcal{B}_2$ deux bases de $E$ et $u \in \mathcal{B}_1 \setminus \mathcal{B}_2$. Alors, il existe $v \in \mathcal{B}_2$ tel que $(\mathcal{B}_1 \setminus \{u\}) \cup \{v\}$ soit une base de $E$.
\end{lem}

\begin{prv}
	[1${}^\text{nde}$ méthode]
	On suppose que pout tout $v \in \mathcal{B}_2$, $(\mathcal{B}_1\setminus \{u\}) \cup \{v\}$ n'est pas une base de $E$
	Soit $v \in \mathcal{B}_2$.
	\begin{itemize}
		\item Supposons $(\mathcal{B}_1\setminus \{u\})\cup \{v\}$ non libre. $\mathcal{B}_1 \setminus \{u\}$ est libre. Donc $v \in \Vect(\mathcal{B}_1 \setminus \{u\})$
		\item Supposons $(\mathcal{B}_1\setminus \{u\}) \cup \{v\}$ non génératrice.
			Comme $\mathcal{B}_1$ engendre $E$, $u \not\in \Vect(\mathcal{B}_1\setminus \{v\})$.
			On suppose que $\mathcal{B}_1 \neq \mathcal{B}_2$.
			$\forall v \in \mathcal{B}_2 \setminus \mathcal{B}_1, \Vect(\mathcal{B}_1 \setminus \{v\}) = \Vect(\mathcal{B}_1) = E \ni u$ 
			donc, $(\mathcal{B}_1\setminus \{u\}) \cup \{v\}$ engendre $E$ et donc \[
				v \in \Vect(\mathcal{B}_1 \setminus \{u\})
			\] On a aussi \[
				\forall v \in \mathcal{B}_1 \setminus \{u\}, v \in \Vect(\mathcal{B}_1\setminus \{u\})
			\] Comme $u \not\in \mathcal{B}_2$, on a \[
				\forall v \in \mathcal{B}_2, v \in \Vect(\mathcal{B}_1\setminus \{u\})
			\] docn \[
				E = \Vect(\mathcal{B}_2) \subset \Vect(\mathcal{B}_1\setminus \{u\})
			\] donc $\mathcal{B}_1\setminus \{u\}$ engendre $E$ donc $\mathcal{B}_1\setminus \{u\}$ est une base de $E$. Or, $\mathcal{B}_1 \setminus \{u\}  \subset  \mathcal{B}_1$, donc $\mathcal{B}_1\setminus \{u\} = \mathcal{B}_1$
	\end{itemize}
\end{prv}

\begin{prv}
	[2${}^\text{nde}$ méthode]
	On suppose que pout tout $v \in \mathcal{B}_2$, $(\mathcal{B}_1\setminus \{u\}) \cup \{v\}$ n'est pas une base de $E$
	\begin{itemize}
		\item Comme $u \in \mathcal{B}_1 \setminus \mathcal{B}_2$, nécéssairement $\mathcal{B}_1 \neq \mathcal{B}_2$ donc $\mathcal{B}_2 \not\subset \mathcal{B}_1$, donc $\mathcal{B}_2\setminus\mathcal{B}_1 \neq \O$ 
		\item Soit $v \in \mathcal{B}_2\setminus\mathcal{B}_1$. Il existe $(\lambda_w)_{w\in\mathcal{B}_1}$ une famille de scalaires presque nulle telle que \[
				v = \sum_{w \in \mathcal{B}_1} \lambda_w w - \lambda_u u + + \sum_{w \in \mathcal{B}_1\setminus \{u\}}\lambda_w w
			\]
			Si $\lambda_u \neq 0_E$, alors
			\begin{align*}
				u &= \lambda_u^{-1}\left( v - \sum_{w \in \mathcal{B}_1 \setminus \{u\}} \lambda_w w \right)\\
					&\in \Vect(\mathcal{B}_1\setminus \{u\} \cup v)
			\end{align*}
			 donc $\mathcal{B}_1 \subset \Vect(\mathcal{B}_1\setminus \{u\} \cup \{v\})$\\
			 et donc $E \subset  \Vect(\mathcal{B}_1 \setminus \{u\} \cup \{v\})$ \\
			 et donc $\mathcal{B}_1 \setminus \{u\} \cup \{v\}$ engendre $E$ \\
			 donc $\mathcal{B}_1 \setminus \{u\} \cup \{v\}$ n'est pas libre\\
			 donc $v \in \Vect(\mathcal{B}_1\setminus \{u\})$ (car $\mathcal{B}_1 \setminus \{u\}$ est libre\\
			 donc $\lambda_u = 0_\mathbbm{K}$ $\lightning$\\`

			 Donc, $\lambda_u = 0_\mathbbm{K}$, docn $v \in \Vect(\mathcal{B}_1\setminus \{u\})$ \\
			 On vient de prouver que
			 \begin{align*}
			 	\mathcal{B}_2 \setminus \mathcal{B}_1 \subset \Vect(\mathcal{B}_1 \setminus \{u\})\\
			 	\mathcal{B}_1 \setminus \{u\} \subset \Vect(\mathcal{B}_1 \setminus \{u\})\\
			 \end{align*}
			 Comme $u \not\in \mathcal{B}_2$, \[
			 	\mathcal{B}_2 \subset \Vect(\mathcal{B}_1 \setminus \{u\})
			 \] donc \[
			 	E = \Vect(\mathcal{B}_2) \subset  \Vect(\mathcal{B}_1 \setminus \{u\})
			 \] donc $\mathcal{B}_1 \setminus \{u\}$ engendre $E$. Donc,  $\mathcal{B}_1 \setminus \{u\}$ est une base de $E$.\\
			 Or, $\mathcal{B}_1 \setminus \{u\} \subset  \mathcal{B}_1$, donc $\mathcal{B}_1 \setminus \{u\} = \mathcal{B}_1$
	\end{itemize}
\end{prv}

\begin{defn}
	Soit $E$ un $\mathbbm{K}$-espace vectoriel de dimension finie. Le cardinal commun à toutes les bases de $E$ est appelé \underline{dimension} de $E$ est notée $\dim(E)$ ou $\dim_\mathbbm{K}(E)$\\
	C'est donc aussi le nombre de coordonnées de n'importe quel vecteur dans n'importe quelle base.
	\index{dimension (espace vectoriel)}
\end{defn}

\begin{exm}
	\begin{enumerate}
		\item $\dim_\R(\C) = 2$ et $\dim_\C(\C) = 1$ 
		\item $\dim_\mathbbm{K}(\mathbbm{K}^{n}) = n$ 
		\item $\dim_{\mathbbm{K}}(\mathcal{M}_{n,p}(\mathbbm{K})) = np$
	\end{enumerate}
\end{exm}

\begin{crlr}
	Soit $E$ un $\mathbbm{K}$-espace vectoriel de dimension finie, $\mathcal{L}$ une famille libre de $E$, $\mathcal{G}$ une famille génératrice de $E$. On note $n = \dim(E)$
	\begin{enumerate}
		\item $\#\mathcal{G} \ge n$ et $(\#\mathcal{G} = n \implies \mathcal{G} \text{ est une base de } E$)
		\item $\#\mathcal{L} \le n$ et $(\#\mathcal{L} = n \implies \mathcal{L} \text{ est une base de } E$)
	\end{enumerate}
\end{crlr}

\begin{crlr}
	$\R^{\R}$ est de dimension infinie.
	$\forall i \in \N, e_i: x \mapsto x^i$\\
	$(e_i)_{i\in\N}$ est libre dans $\R^\R$
\end{crlr}

\begin{prop}
	Soient $E$ et $F$ deux $\mathbbm{K}$-espaces vectoriels de dimension finie. Alors $E\times F$ est de dimension finie et $\dim(E\times F) = \dim(E) + \dim(F)$
\end{prop}

\begin{prv}
	Soit $(e_1,\ldots, e_n)$ une base de $E$, $(f_1, \ldots, f_p)$ une base de $F$.
	On pose \[
		\left\{\begin{array}
			{r c l}
			u_1 &=& (e_1,0_F)\\
			u_2 &=& (e_2,0_F)\\
					&\vdots&\\
			u_n &=& (e_n,0_F)\\
			u_{n+1} &=& (0_E, f_1)\\
			u_{n+2} &=& (0_E, f_2)\\
					&\vdots&\\
			u_{n+p} &=& (0_E,f_p)\\
		\end{array}\right.
	\]
	Soit $(x,y) \in E\times F$. \[
		\begin{cases}
			\exists (x_1,\ldots,x_n)\in \mathbbm{K}^n, x = \sum_{i=1}^{n} x_ie_i
			\exists (y_1,\ldots,y_n)\in \mathbbm{K}^n, x = \sum_{j=1}^{p} y_jf_j
		\end{cases}
	\] 
	\begin{align*}
		(x,y) &= \left( \sum_{i=1}^{n} x_ie_i, \sum_{i=1}^{p} y_jf_j \right)  \\
		&= \sum_{i=1}^{n} x_i (e_i + 0_F) + \sum_{j=1}^{p} y_j (0_E, f_j) \\
		&= \sum_{i=1}^{n} x_i u_i + \sum_{j=1}^{p} y_j u_{n+j} \\
	\end{align*}
	Donc, $E\times F = \Vect(u_1, \ldots, u_{n+p})$ donc $E\times F$ est de dimension finie.\\
	Soit $(\lambda_1, \ldots, \lambda_{n+p}) \in \mathbbm{K}^{n+p}$ tel que \[
		(*): \quad \sum_{k=1}^{n+p} \lambda_ku_k = 0_{E\times F} = (0_E, 0_F)
	\]
	\begin{align*}
		(*) &\iff \sum_{k=1}^{n} \lambda_k (e_k, 0_F) + \sum_{k=n+1}^{p} \lambda_k(0_E, f_{k-n}) = (0_E, 0_F)\\
				&\iff \begin{cases}
					\sum_{k=1}^{n} \lambda_k e_k = 0_E\\
					\sum_{k=n+1}^{p} \lambda_k f_{k-n} = 0_F
				\end{cases}\\
				&\iff \begin{cases}
					\forall k \in \left\llbracket 1,n \right\rrbracket, \lambda_k = 0_\mathbbm{K} \qquad&(\text{car $(e_1,\ldots,e_n)$ est libre})\\
					\forall k \in \left\llbracket n+1,n+p \right\rrbracket, \lambda_k = 0_\mathbbm{K} \qquad&(\text{car $(f_1,\ldots,f_n)$ est libre})\\
				\end{cases}
	\end{align*}
	Donc $(u_1, \ldots, u_{n+p})$ est une base de $E\times F$. Donc, $\dim(E\times F) = n + p = \dim(E) + \dim(F)$
\end{prv}

\begin{rmk}
	[Convention]
	\[\dim\big(\{0_E\}\big) = 0\]
\end{rmk}

\begin{thm}
	Soit $E$ un $\mathbbm{K}$-espace vectoriel de dimension finie, $F$ un sous-espace vectoriel de $E$. Alors, $F$ est de dimension finie et  $\dim(F) \le \dim(E)$\\
	Si $\dim(F) = \dim(E)$, alors $F = E$
\end{thm}

\begin{prv}
	On considère \[
		A = \{k \in \N \mid \text{il existe une famille libre de $F$ à $k$ éléments}\} 
	\]
	On suppose $F \neq \{0_E\}$.
	\begin{itemize}
		\item Soit $u \in F\setminus \{0_E\}$. $(u)$ est libre donc $1 \in A$ et donc $A \neq \O$
		\item Soit $\mathcal{L}$ une famille libre de $F$. Alors, $\mathcal{L}$ est une famille libre de $E$ \\
			donc $\#\mathcal{L} \le \dim(E)$\\
			Donc $A$ est majorée par $\dim(E)$ \\
			On en déduit que $A$ a un plus grand élément $p$.
		\item Soit $\mathcal{L}$ une famille libre de $F$ avec $p$ éléments.\\
			Si $\mathcal{L}$ n'engendre pas $F$, alors il existe $u\in F$ tel que $u\not\in \Vect(\mathcal{L})$ et donc $\mathcal{L} \cup \{u\}$ est une famille libre de $F$, donc $p+1 \in A$ en contradiction avec la maximalité de $p$.\\
			Donc $\mathcal{L}$ est une base de $F$ donc $F$ est de dimension finie et $\dim(F) = p \le \dim(E)$\\
	\end{itemize}

	Soit $\mathcal{B}$ une base de $F$. Alors, $\mathcal{B}$ est aussi une famille de libre de de $E$. Donc $\#\mathcal{B} \le \dim(E)$ donc $\dim(F) = \dim(E)$ \\
	Si $\dim(F) = \dim(E)$, alors $\mathcal{B}$ est une base de $E$, et donc $F = \Vect(\mathcal{B}) = E$
\end{prv}

\begin{prop}
	[Formule de Grassmann]
	Soit $E$ un $\mathbbm{K}$-espace vectoriel de dimension finie, $F$ et $G$ deux sous-espace vectoriels de $E$. Alors, \[
		\dim(F+G) = \dim(F) + \dim(G) - \dim(F\cap G)
	\] 
\end{prop}

\begin{prv}
	Soit $(e_1, \ldots, e_p)$ une base de $F\cap G$. $(e_1,\ldots,e_p)$ est une famille libre de $F$.\\
	On complète $(e_1, \ldots, e_p)$ en une base $(e_1, \ldots, e_p, u_1, \ldots, u_q)$ de $F$.\\
	De même, on complète $(e_1, \ldots, e_p)$ en une base $(e_1, \ldots, e_p, v_1, \ldots, v_r)$ de $G$.\\
	On pose  $\mathcal{B} = (e_1, \ldots, e_p, u_1, \ldots, u_q, v_1, \ldots, v_r)$. Montrons que $\mathcal{B}$ est une base de $F+G$
	\begin{itemize}
		\item Soit $u \in F+G$ \\
			On pose $u = v+w$ avec $\begin{cases}
				v\in F\\
				w \in G
			\end{cases}$.\\
			On pose $v = \sum_{i=1}^p \lambda_i e_i + \sum_{i=1}^q \mu_i u_i$ avec $(\lambda_1, \ldots, \lambda_p, \mu_1, \ldots, \lambda_q) \in \mathbbm{K}^{p+q}$\\
			On pose aussi $w = \sum_{i = 1}^p \lambda'_ie_i + \sum_{j=1}^r \nu_j v_j$ avec $(\lambda_1',\ldots,\lambda_p', \nu_1, \ldots, \nu_r) \in \mathbbm{K}^{p+r}$\\
			D'où, \[
				u = \sum_{i=1}^p (\lambda_i + \lambda'_i)e_i + \sum_{j=1}^q \mu_j u_j + \sum_{k=1}^r \nu_k v_k \in \Vect(\mathcal{B})
			\]
		\item Soient $(\lambda_1, \ldots, \lambda_p, \mu_1, \ldots, \mu_q, \nu_1, \ldots, \nu_r) \in \mathbbm{K}^{p+q+r}$.\\
			On suppose \[
				(*)\quad \sum_{i=1}^{p}\lambda_ie_i + \sum_{j=1}^q\mu_ju_j + \sum_{k=1}^r \nu_k v_k = 0_E
			\] 
			D'où, \[
				\underbrace{\sum_{i=1}^p\lambda_i e_i + \sum_{j=1}^q \mu_ju_j}_{\in F} = \underbrace{-\sum_{k=1}^r\nu_jv_k}_{\in G}
			\] 
			Donc, \[
				f = \sum_{i=1}^p \lambda_i e_i + \sum_{j=1}^q \mu_j u_j \in F\cap G
			\] Comme $(e_1, \ldots, e_p)$ est une base de $F\cap G$, $\exists ! (\lambda_1', \ldots, \lambda_p') \in \mathbbm{K}^p$ tel que \[
				f = \sum_{i=1}^p \lambda'_i e_i = \sum_{i=1}^p \lambda'_i e_i + \sum_{j=1}^q 0_\mathbbm{K}u_j
			\] Comme $(e_1, \ldots, e_p, u_1, \ldots, u_q)$ est une base de $F$, \[
				\forall k \in \left\llbracket 1, q \right\rrbracket, \mu_j = 0_\mathbbm{K}
			\] De même, \[
				\forall k \in \left\llbracket 1,r \right\rrbracket , \nu_k = 0_\mathbbm{K}
			\] On remplace dans $(*)$ pour trouver \[
				\sum_{i=1}^p \lambda_ie_i = 0_E
			\] Comme $(e_1, \ldots, e_p)$ est libre, \[
				\forall i \in \left\llbracket 1,p \right\rrbracket, \lambda_i = 0_\mathbbm{K}
			\] Donc $\mathcal{B}$ est libre.\\
			Donc, 
			\begin{align*}
				\dim(F+G) &=  p +q + r \\
				&= (p+q)+ (p+r) - p \\
				&= \dim(F) + \dim(G) - \dim(F\cap G) \\
			\end{align*}
	\end{itemize}
\end{prv}

\begin{crlr}
	Avec les hypothèse précédentes, \[
		E = F \oplus G \iff \begin{cases}
			F \cap  G = \{0_E\} \\
			\dim(E) = \dim(F) + \dim(G)
		\end{cases}
	\] 
\end{crlr}

\begin{prv}
	\begin{itemize}
		\item[``$\implies$''] On suppose $E = F \oplus G$ \\
			Comme la somme est directe, $F \cap G = \{0_E\}$ 
			\begin{align*}
				\dim(E) &= \dim(F)\\
				&= \dim(F) + \dim(G) - \dim(F\cap G)\\
				&= \dim(F) + \dim(G)\\
			\end{align*}
		\item[``$\impliedby$''] On suppose $F\cap G = \{0_E\}$ et $\dim(E) = \dim(F) + \dim(G)$.\\
			On sait déjà que $F+G = F \oplus G$\\
			 \begin{align*}
				\dim(F+G) = \dim(F) + \dim(G) - \dim(F \cap G) = \dim(E)
			\end{align*}
			Donc $F + G = E$
	\end{itemize}
\end{prv}

\begin{prop}
	Soit $F$ un $\mathbbm{K}$-espace vectoriel de dimension finie $n$. Soit $\mathcal{B} = (e_1, \ldots, e_n)$ une base de $F$. L'application
	\begin{align*}
		f: \mathbbm{K}^n &\longrightarrow F \\
		(\lambda_1, \ldots, \lambda_n) &\longmapsto \sum_{i=1}^n \lambda_i e_i
	\end{align*} est bijective.\\
	Si $\mathbbm{K}$ est infini, $\mathbbm{K}^n$ aussi et donc $F$ aussi.\\
	Si $\#\mathbbm{K} = p \in \N_*$,
	\begin{align*}
		\#&\mathbbm{K}^n = p^n\\
		&\vrt=\\
		\#&F
	\end{align*}
\end{prop}


		\part{Dérivation}

\underline{Motivation}:

{
\begin{wrapfigure}{l}{3cm}
	\centering
	\begin{asy}
		import three;

		size(3cm);
		settings.render=0;
		settings.prc=false;
		currentprojection = obliqueZ;

		draw(unitbox);
		draw(shift(1.1Z + 0.05X) * (O -- X), Arrows3(TeXHead2));
		draw(shift(1.1Z + 0.05Y) * (O -- Y), Arrows3(TeXHead2));
		draw(shift(1.1X + 0.05Z) * (O -- Z), Arrows3(TeXHead2));

		label("$x$", (X/2) + (1.1Z + 0.05X), align=S);
		label("$y$", (Y/2) + (1.1Z + 0.05Y), align=W);
		label("$z$", (Z/2) + X, align=SE);
	\end{asy}
\end{wrapfigure}

\begin{align*}
	&S(x,y,z) = 2(xy + xz + yz)\\
	&V(x,y,z) = xyz
\end{align*}

On cherche à minimiser $S$ avec la contrainte $V = 1$.

Soit $f : \begin{array}{rcl}
	\left( \R_*^+ \right)^2 &\longrightarrow& \R \\
	(x,y) &\longmapsto& S\left( x,y,\frac{1}{xy} \right) = 2\left( xy + \frac{1}{y} + \frac{1}{x} \right).
\end{array}$

On cherche $(a,b) \in \left( \R^+_* \right)^2$ tel que \[
	\forall (x,y) \in (\R^+_*), f(x,y) \ge f(a,b).
\]
}

\begin{defn}
	Soit $f: U \to \R$ où $U$ est un ouvert de $\R^2$. Soit $(a,b) \in U$.
	\vspace{2mm}

	Si $\lim_{x \to a} \frac{f(x,b) - f(a,b)}{x - a} \in \R$, alors on dit que $f$ a une dérivée partielle suivant $x$ en $(a,b)$ et cette limite est notée \[
		\partial f_1(a,b) = \frac{\partial f}{\partial x}(a,b).
	\]

	Si $\lim_{y \to b} \frac{f(a,y) - f(a,b)}{y - b} \in \R$, alors on dit que $f$ a une dérivée partielle suivant $y$ et la limite est notée \[
		\partial f_2(a,b) = \frac{\partial f}{\partial y}(a,b).
	\]
\end{defn}

\begin{exm}
	\begin{enumerate}
		\item $f: (x,y) \mapsto xy + x - y$.

			\begin{align*}
				&\frac{\partial f}{\partial x} : (x,y) \mapsto y + 1,\\
				&\frac{\partial f}{\partial y} : (x,y) \mapsto x - 1.
			\end{align*}

		\item $f: (x,y) \mapsto xy + \frac{1}{y}+ \frac{1}{x}$.

			\begin{align*}
				&\frac{\partial f}{\partial x}: (x,y) \mapsto y - \frac{1}{x^2},\\
				&\frac{\partial f}{\partial y}: (x,y) \mapsto x - \frac{1}{y^2}.
			\end{align*}

		\item Trouver $f$ telle que $\begin{cases}
				(1): \qquad \frac{\partial f}{\partial x}=y,\\[2mm]
				(2): \qquad \frac{\partial f}{\partial y} = x.
			\end{cases}$

			D'après $(1)$ : \[
				\forall (x,y), \exists C(y) \in \R, f(x,y) = xy + C(y)
			\] et donc \[
				\frac{\partial f}{\partial y}(x,y) = x + C'(y)
			\] donc $C'(y) = 0$ et donc $C$ est constante.

		\item Trouver $f$ telle que $\begin{cases}
			\frac{\partial f}{\partial x} = -y,\\[2mm]
			\frac{\partial f}{ƒ\partial y} = x.
		\end{cases}$

		Ce n'est pas possible !
	\end{enumerate}
\end{exm}

\begin{defn}~\\
	\begin{minipage}{\linewidth}
		\begin{wrapfigure}{r}{4cm}
			\centering
			\vspace{-5mm}
			\begin{asy}
				import three;
				import graph3;
				size(4cm);

				settings.render = 0;
				settings.prc = false;
				currentprojection = obliqueX;

				draw(O -- X, Arrow3(TeXHead2));
				draw(O -- Y, Arrow3(TeXHead2));
				draw(O -- Z, Arrow3(TeXHead2));

				triple f(real x, real y, real z = 0) { return (x,y,cos(x - 0.5) * cos(y - 0.5)/1.2 + 0.15); }

				real inc = 1 / 5;

				for(real x = 0; x <= 1; x += inc) {
					draw(graph(
						new real(real t) { return x; }, // x
						new real(real y) { return y; }, // y
						new real(real y) { return f(x,y).z; }, // z
						0, 1
					), gray);
				}

				for(real y = 0; y <= 1; y += inc) {
					draw(graph(
						new real(real x) { return x; }, // x
						new real(real t) { return y; }, // y
						new real(real x) { return f(x,y).z; }, // z
						0, 1
					), gray);
				}

				path3 path1 = (0.8, 0.2, 0) .. (0.5, 0.5, 0) .. (0.3, 0.7, 0);
				path3 path2 = f(0.8, 0.2, 0) .. f(0.5, 0.5, 0) .. f(0.3, 0.7, 0);
				path3 d = (0.2, 0.3, 0) .. (0.3, 0.4, 0) .. (0.2, 0.7, 0) .. (0.8, 0.9, 0) .. (0.6, 0.2, 0) .. cycle;

				draw(path1, red, Arrow3(TeXHead2));
				draw(path2, red, Arrow3(TeXHead2, position=0.8));

				dot((0.5, 0.5, 0));
				dot(f(0.5, 0.5, 0));
				draw((0.5, 0.5, 0) -- f(0.5, 0.5, 0), dashed);
				draw(d);

				label("$w$", (0.3, 0.7, 0), red, align=SE);
				label("$U$", (0.8, 0.9, 0), align=SE);
			\end{asy}
		\end{wrapfigure}

		Soit $f: U \to \R$ où $U$ est un ouvert. Soit $(a,b) \in U$. Soit $w = (w_1, w_2) \in \R^2$.

		Si 
		\[
			\lim_{t\to 0} \frac{f(a + tw_1, b + tw_2) - f(a,b)}{t}
		\] existe et est réelle, alors on dit que $f$ a une dérivée dans la direction de $w$ et la limite est notée \[
			\mathrm{d}f(w)\,(a,b) = D_w(f)\,(a,b).
		\]
	\end{minipage}
\end{defn}

\begin{exm}
	\begin{align*}
		f: \left( \R_*^+ \right)^2 &\longrightarrow \R \\
		(x,y) &\longmapsto xy+\frac{1}{x}+\frac{1}{y}.
	\end{align*}

	On pose $(a,b) = (1,2)$, $w = (w_1, w_2) = (1,1)$.
	\begin{align*}
		\frac{f(1+t, 2+t) - f(1,2)}{t} &= \frac{1}{t} \left( (1+t)(2+t) + \frac{1}{1+t} + \frac{1}{2+t} - 3 - \frac{1}{2} \right) \\
		&= \frac{1}{t}\left(\cancel 2 + 3t + \po(t) + \cancel 1 - t + \po(t) + \frac{1}{2}\left( \cancel 1 - \frac{t}{2} + \po(t) \right) - \cancel3 - \cancel{\frac{1}{2}} \right) \\
		&= \frac{1}{t} \left( \frac{7}{4} t + \po(t) \right)  \\
		&= \frac{7}{4} + \po(1) \tendsto{t \to 0} \frac{7}{4}. \\
	\end{align*}

	Donc, \[
		\mathrm{d}f(1,1)\,(1,2) = \frac{7}{4}.
	\]
\end{exm}

\begin{rmk}~\\
	\begin{figure}[H]
		\centering
		\begin{asy}
			import solids;
			import graph;
			size(5cm);

			settings.render = 0;
			settings.prc = false;

			path3 par = graph(
				new real(real x) { return x; },
				new real(real x) { return 0; },
				new real(real x) { return x^2; },
				0,3);
			revolution r = revolution(par, axis=Z);

			path3 par2 = graph(
				new real(real x) { return x; },
				new real(real x) { return 0; },
				new real(real x) { return x^2; },
				-3,3);

			draw(r,1,longitudinalpen=nullpen);
			draw(r.silhouette());

			draw((-4, 0, -1) -- (-4, 0, 10) -- (4, 0, 10) -- (4, 0, -1) -- cycle, red);
			draw(par2, deepred);

			draw((4,4.5) -- (7, 4.5), black+0.5mm, Arrow(TeXHead));

			path par2d = graph(new real(real x) { return x^2; }, -3, 3);
			draw(shift((11, 0)) * par2d, deepred);

			dot(O);
			dot((11, 0));
		\end{asy}
	\end{figure}
\end{rmk}


%todo ajouter théorème-définition
\begin{thm}
	Soit $f : U \to \R$, $(a,b) \in U$. On suppose que $\frac{\partial f}{\partial x}$ et $\frac{\partial f}{\partial y}$ existent en $(a,b)$ et sont {\bfseries continues} en $(a,b)$. Alors,
	\begin{align*}
		&\forall (h, k) \in \R^2 \text{ tel que } (a +h, b + k) \in U,\\
		&f(a+ h, b + k) = f(a,b) + h \frac{\partial f}{\partial x}(a,b) + k \frac{\partial f}{\partial y}(a,b) + \po_{(h,k)\to (0,0)}\big(\|(h,k)\|\big).
	\end{align*}

	On dit que $f$ est de classe $\mathcal{C}^1$ si $\frac{\partial f}{\partial x}$ et $\frac{\partial f}{\partial y}$ existent et sont continues.

	\qed
\end{thm}

\begin{rmk}
	En physique, cette formule correspond à : \[
		\mathrm{d}f = \frac{\partial f}{\partial x}\mathrm{d}x + \frac{\partial f}{\partial y} \mathrm{d}y.
	\] En effet :
	\begin{align*}
		\mathrm{d}f &= f(x+ \mathrm{d}x, y + \mathrm{d}y) - f(x,y) \\
		&= \frac{\partial f}{\partial x} \mathrm{d}x + \frac{\partial f}{\partial y} \mathrm{d}y.
	\end{align*}
\end{rmk}

\begin{prop}
	Soit $f: U \to \R$ de classe $\mathcal{C}^1$ en $(a,b) \in U$. Alors, \[
		\forall w = (w_1, w_2) \in \R^2, \mathrm{d}f(w)\,(a,b) = w_1 \frac{\partial f}{\partial x}(a,b) + w_2 \frac{\partial f}{\partial y}(a,b).
	\]
\end{prop}

\begin{prv}
	Soit $w = (w_1, w_2) \in \R^2$. Soit $t \in \R^*$.
	\begin{align*}
		\frac{1}{t}\big(f(a + tw_1, b + tw_2) - f(a,b)\big)
		&= \frac{1}{t} \left( tw_1 \frac{\partial f}{\partial x}(a,b) + tw_2 \frac{\partial f}{\partial y}(a,b) + \po_{t \to 0}\big(\|tw\|\big) \right) \\
		&= w_1 \frac{\partial f}{\partial x}(a,b) + w_2 \frac{\partial f}{\partial y}(a,b) + \po_{t\to 0}(1) \\
		&\tendsto{t\to 0} w_1 \frac{\partial f}{\partial x}(a,b) + w_2\frac{\partial f}{\partial y}(a,b).
	\end{align*}
\end{prv}


\begin{defn}
	Avec les hypothèses précédentes, en posant \[
		\nabla f(a,b) = \left( \frac{\partial f}{\partial x}(a,b), \frac{\partial f}{\partial y}(a,b) \right) 
	\]on obtient \[
		\mathrm{d}f(w)\,(a,b) = \left<w  \mid \nabla f(a,b) \right>
	\] où $\left<\cdot|\cdot \right>$ est le produit scalaire.

	Le vecteur $\nabla f(a,b)$ est appelé \underline{gradient de $f$ en $(a,b)$}.

	Le développement limité à l'ordre 1 de $f$ devient \[
		f\big((a,b)+w\big) = f(a,b) + \left<w \mid \nabla f(a,b) \right> + \po_{w\to 0}(\|w\|)
	\]
\end{defn}

\begin{prop}
	Soit $f : U \to \R$ de classe $\mathcal{C}^1$.

	\begin{figure}[H]
    \centering
    \incfig{gradient}
	\end{figure}

	$\nabla f$ est orthogonal au lignes de niveaux de $f$, son orientation va dans le sens d'une augmentation de $f$.
\end{prop}

\begin{prv}
	Soit $\gamma : I \to U$ une courbe de niveau : \[
		\forall t \in I, f\big(\gamma(t)\big) = \text{cste}.
	\] D'après le lemme suivant : \[
		\forall t \in I, 0 = (f \circ \gamma)'(t) = \mathrm{d}f\big(\gamma'(t)\big)\big(\gamma(t)\big) = \left<\gamma'(t)  \mid \nabla f\big(\gamma(t)\big) \right>
	\] Donc $\nabla f\big(\gamma(t)\big)$ est orthogonal à $\gamma'(t)$.

	Pour tout $t \in I$, on pose $w(t) = t\, \nabla f\big(\gamma(t)\big)$. Donc \[
		f\big(\gamma(t) + w(t)\big) = f\big(\gamma(t)\big) + t \|\nabla f(\gamma(t))\|^2 + \po_{t \to 0}(t)
	\] Pour $t$ assez petit, $f\big(\gamma(t) + w(t)\big) - f\big(\gamma(t)\big)$ est du même signe que $t$.
\end{prv}

\begin{rmk}
	\begin{align*}
		V: \R^3 &\longrightarrow \R \\
		(x,y,z) &\longmapsto -mgz
	\end{align*}
	l'énerge potentielle de pesenteur

	On a donc \[
		\nabla V(x,y,z) = \left( \frac{\partial V}{\partial x}, \frac{\partial V}{\partial y}, \frac{\partial V}{\partial z} \right) = (0, 0, -mg) = \vec{P}.
	\]
\end{rmk}

\begin{lem}
	Soit $f : U \to \R$ de classe $\mathcal{C}^1$, $\gamma : \begin{array}{rcl}
		I &\longrightarrow& U \\
		t &\longmapsto& \big(x(t), y(t)\big)
	\end{array}$ où $x$ et $y$ sont dérivables.

	On pose \[
		\forall t \in I, \gamma'(t) = \big(x'(t), y'(t)\big).
	\] Alors $f \circ \gamma : I \to \R$ est dérivable et
	\begin{align*}
		\forall t \in I, (f \circ \gamma)'(t) &= \mathrm{d}f\big(\gamma'(t)\big) \big(\gamma(t)\big)\\
		&= \left<\gamma'(t)  \mid \nabla f\big(\gamma(t)\big)  \right> \\
		&= x'(t) \frac{\partial f}{\partial x}\big(x(t), y(t)\big) + y'(t) \frac{\partial f}{\partial y}\big(x(t),y(t)\big). \\
	\end{align*}
\end{lem}

\begin{prv}
	On fixe $t \in I$.

	\begin{align*}
		\forall h \neq 0, \frac{f \circ \gamma(t + h) - f \circ \gamma(t)}{h}
		&= \frac{1}{h}\big(f(\gamma(t)) + h\gamma'(t) + \po_{h\to 0}(h) - f(\gamma(t))\big) \\
		&= \frac{1}{h}\bigg(\cancel{f(\gamma(t))} + \left<h\gamma'(t) \mid \nabla f(\gamma(t)) \right> + \po_{h\to 0}(\|h\gamma'(t)\|) - \cancel{f(\gamma(t))}\bigg)\\
		&= \left<\gamma'(t) \mid \nabla f(\gamma(t)) \right> + \po_{h\to 0}(1) \\
		&\tendsto{h\to 0} \left<\gamma'(t)  \mid \nabla f(\gamma(t)) \right>
	\end{align*}
\end{prv}

\begin{defn}
	Soit $f : U \to \R$ de classe $\mathcal{C}^1$ et $(a,b) \in U$. On dit que $(a,b)$ est un \underline{point critique} de $f$ si $\nabla f(a,b) = 0$ i.e. $\frac{\partial f}{\partial x}(a,b) = \frac{\partial f}{\partial y}(a,b) = 0$.

	Dans ce cas, $f(a,b)$ est appelé \underline{valeur critique} de $f$.
\end{defn}

\begin{prop}~\\
	\begin{minipage}{\linewidth}
		\begin{wrapfigure}{r}{3cm}
			\centering
			\vspace{-1cm}
			\begin{asy}
				import solids;
				import graph;
				size(3cm);

				settings.render = 0;
				settings.prc = false;

				path3 par = graph(
					new real(real x) { return x; },
					new real(real x) { return 0; },
					new real(real x) { return -x^2; },
					0,3);
				revolution r = revolution(par, axis=Z);

				draw(r,1,longitudinalpen=nullpen);
				draw(r.silhouette());

				dot("$(a,b)$", O, red, align=N);
				real s = sqrt(2.5);
				path3 g=(s,0,-2.5)..(0,s,-2.5)..(-s,0,-2.5)..(0,-s,-2.5)..cycle;
				draw(g, deepcyan);
			\end{asy}
		\end{wrapfigure}
		Soit $f: U \to \R$ de classe $\mathcal{C}^1$ et $(a,b) \in U$ tel que \[
			\exists r > 0, \forall (x,y) \in B_{(a,b)}(r), f(x,y) \le f(a,b)
		\] Alors $\nabla f(a,b) = (0,0)$.
	\end{minipage}
\end{prop}

\begin{prv}
	Soit $g: x \mapsto f(x,b)$. $g(a)$ est un maximum local de $g$ donc $g'(a) = 0$.

	Or, $g'(a) = \frac{\partial f}{\partial x}(a,b)$

	donc $\frac{\partial f}{\partial x}(a,b) = 0$.

	Soit $h : y \mapsto f(a,y)$. On a de même $h'(b) = 0$.

	Or, $h'(b) = \frac{\partial f}{\partial y}(a,b)$.

	Donc, $\nabla f(a,b) = (0,0)$.
\end{prv}

\begin{rmk}
	Un minimum local est aussi une valeur critique.
\end{rmk}

\begin{figure}[H]
	\centering
	\begin{subfigure}{3cm}
		\centering
		\begin{asy}
			import solids;
			import graph;
			size(3cm);

			settings.render = 0;
			settings.prc = false;

			path3 par = graph(
				new real(real x) { return x; },
				new real(real x) { return 0; },
				new real(real x) { return -x^2; },
				0,3);
			revolution r = revolution(par, axis=Z);

			draw(r,1,longitudinalpen=nullpen);
			draw(r.silhouette());

			dot(O, red);
		\end{asy}
		\caption{Maximum local}
	\end{subfigure}
	\begin{subfigure}{3cm}
		\centering
		\begin{asy}
			import solids;
			import graph;
			size(3cm);

			settings.render = 0;
			settings.prc = false;

			path3 par = graph(
				new real(real x) { return x; },
				new real(real x) { return 0; },
				new real(real x) { return x^2; },
				0,3);
			revolution r = revolution(par, axis=Z);

			draw(r,1,longitudinalpen=nullpen);
			draw(r.silhouette());

			dot(O, red);
		\end{asy}
		\caption{Minimum local}
	\end{subfigure}
	\begin{subfigure}{3cm}
		\centering
		\begin{asy}
			import solids;
			import graph;
			size(3cm);

			settings.render = 0;
			settings.prc = false;
			currentprojection = obliqueZ;

			draw(graph(
				new real(real x) { return x; },
				new real(real x) { return -x^2 / 3; },
				new real(real x) { return 3; },
				-3, 3
			));

			draw(graph(
				new real(real x) { return x; },
				new real(real x) { return -x^2 / 3; },
				new real(real x) { return -3; },
				-3, 3
			));

			draw(graph(
				new real(real x) { return x; },
				new real(real x) { return -x^2 / 3 - 1; },
				new real(real x) { return 0; },
				-3, 3
			));

			draw(graph(
				new real(real x) { return 0; },
				new real(real x) { return x^2 / 9 - 1; },
				new real(real x) { return x; },
				-3, 3
			));

			draw(graph(
				new real(real x) { return -3; },
				new real(real x) { return x^2 / 9 - 4; },
				new real(real x) { return x; },
				-3, 3
			));

			draw(graph(
				new real(real x) { return 3; },
				new real(real x) { return x^2 / 9 - 4; },
				new real(real x) { return x; },
				-3, 3
			));

			dot((0,-1,0), red);
		\end{asy}
		\caption{Point de selle / Point col}
	\end{subfigure}
\end{figure}

\begin{exm}
	On revient à l'exemple donné en introduction : 
	\begin{align*}
		f: \left( \R^*_+ \right)^2 &\longrightarrow \R \\
		(x,y) &\longmapsto 2\left( xy + \frac{1}{x} + \frac{1}{y} \right).
	\end{align*}

	$\left( \R^+_* \right)^2$ est un ouvert de $\R^2$. Soit $(x,y) \in \left( \R^+_* \right)^2$.
	
	On a \[
		\begin{cases}
			\frac{\partial f}{\partial x}(x,y) = 2\left( y - \frac{1}{x^2} \right),\\
			\frac{\partial f}{\partial y}(x,y) = 2\left( x - \frac{1}{y^2} \right).
		\end{cases}
	\]

	\begin{align*}
		&\frac{\partial f}{\partial x}(x,y) = \frac{\partial f}{\partial y}(x,y) = 0\\
		\iff& \begin{cases}
			y = \frac{1}{x^2}\\
			x = \frac{1}{y^2}
		\end{cases}\\
		\iff& \begin{cases}
			y = \frac{1}{x^2}\\
			x = x^4
		\end{cases}\\
		\iff& \begin{cases}
			x = 1\\
			y = 1
		\end{cases}
	\end{align*}

	On vérivie que $f$ présente en effet un minium local en $(1,1)$. \[
		f(1,1) = 6
	\] On fixe $y \in \R^+_*$ et \[
		g : x \mapsto 2\left( xy + \frac{1}{x} + \frac{1}{y} \right).
	\] Donc \[
		\forall x \in \R^+_*, g'(x) = 2\left( y - \frac{1}{x^2} \right).
	\]
	\begin{center}
		\begin{tikzpicture}
			\tkzTabInit{$x$/1,$g'(x)$/1,$g$/2.3}{$0$, $\frac{1}{\sqrt{y}}$, $+\infty$}
			\tkzTabLine{,-,z,+,}
			\tkzTabVar{+/{}, -/$2\left( 2\sqrt{y} +\frac{1}{y} \right)$, +/{}}
		\end{tikzpicture}
	\end{center}
	
	Ainsi, \[
		\forall x \in \R^+_*, \forall y \in \R^+_*, f(x,y) \ge 2\left( 2\sqrt{y} + \frac{1}{y} \right)
	\] Soit $h : y \mapsto 2\sqrt{y} + \frac{1}{y}$. On a \[
		\forall y > 0, h'(y) = \frac{1}{\sqrt{y}} - \frac{1}{y^2} = \frac{y\sqrt{y} - 1}{y^2} = \frac{y^{\frac{3}{2}} - 1}{y^2}
	\]

	\begin{center}
		\begin{tikzpicture}
			\tkzTabInit{$y$/0.7,$h'(y)$/0.7,$h$/1.4}{$0$, $1$, $+\infty$}
			\tkzTabLine{,-,z,+,}
			\tkzTabVar{+/{}, -/$3$, +/{}}
		\end{tikzpicture}
	\end{center}

	Donc, \[
		\forall x,y > 0, f(x,y) \ge 2\times 3 = 6 = f(1,1).
	\]
\end{exm}

\begin{prop}
	[règle de la chaîne]

	Soit $f : \begin{array}{rcl}
		U &\longrightarrow& \R^2 \\
		(x,y) &\longmapsto& f(x,y)
	\end{array}$ de classe $\mathcal{C}^1$ et $U, V$ deux ouverts de $\R^2$.

	Soit $\varphi : \begin{array}{rcl}
		V &\longrightarrow& U \\
		(u,v) &\longmapsto& \varphi(u,v) = \big(x(u,v), y(u,v)\big)
	\end{array}$.

	On suppose que $x$ et $y$ sont de classe $\mathcal{C}^1$ sur $V$.

	Alors,  $f \circ \varphi : \begin{array}{rcl}
		V &\longrightarrow& \R \\
		(u,v) &\longmapsto& f\big(\varphi(u,v)\big)
	\end{array}$ est de classe $\mathcal{C}^1$ et
	\begin{align*}
		\forall (u_0, v_0) \in V, \frac{\partial (f \circ \varphi)}{\partial u}(u_0, v_0)
		&= \frac{\partial f}{\partial x}\big(\varphi(u_0, v_0)\big) \times \frac{\partial x}{\partial u}(u_0, v_0)\\
		&+ \frac{\partial f}{\partial y}\big(\varphi(u_0,v_0)\big) \frac{\partial y}{\partial u}(u_0,v_0)
	\end{align*}
	\begin{align*}
		\forall (u_0, v_0) \in V, \frac{\partial (f \circ \varphi)}{\partial v}(u_0, v_0)
		&= \frac{\partial f}{\partial x}\big(\varphi(u_0, v_0)\big) \times \frac{\partial x}{\partial v}(u_0, v_0)\\
		&+ \frac{\partial f}{\partial y}\big(\varphi(u_0,v_0)\big) \frac{\partial y}{\partial v}(u_0,v_0)
	\end{align*}
\end{prop}

\begin{exm}
	[changement de coordonnées polaires]
	On pose \begin{align*}
		\varphi: \R^+_* \times ]0,2\pi[ &\longrightarrow \R^2\setminus \left( R^+_* \times \{0\} \right) \\
		(r, \theta) &\longmapsto (r \cos \theta, r \sin\theta),
	\end{align*}
	\begin{align*}
		f: \R^2\setminus \left( R^+_* \times \{0\} \right) &\longrightarrow \R \\
		(x,y) &\longmapsto f(x,y),
	\end{align*}
	\begin{align*}
		g: \overbrace{\R^+_* \times ]0, 2\pi[}^{=V} &\longrightarrow \R \\
		(r, \theta) &\longmapsto f(r\cos\theta, r\sin\theta).
	\end{align*}

	\begin{align*}
		\forall (r_0,\theta_0) \in V,&\\[5mm]
		\frac{\partial g}{\partial r}(r_0, \theta_0) &= \frac{\partial f}{\partial x}(r_0\cos\theta_0, r_0\sin\theta_0)\cos\theta_0\\
		&+ \frac{\partial f}{\partial y}(r_0 \cos\theta_0, r_0\sin\theta_0)\sin\theta_0\\
		&= 2r_0\cos^2\theta_0 + 2r_0\sin^2(\theta_0) \\
		&= 2r_0 \\[5mm]
		\frac{\partial g}{\partial \theta}(r_0, \theta_0) &= \frac{\partial f}{\partial x}(r_0\cos\theta_0, r_0\sin\theta_0)r_0\sin\theta_0\\
		&+ \frac{\partial f}{\partial y}(r_0 \cos\theta_0, r_0\sin\theta_0)r_0\cos\theta_0\\
		&= -2{r_0}^2\cos(\theta_0)\sin(\theta_0) + 2{r_0}^2 \sin(\theta_0)\cos(\theta_0)\\
		&= 0 \\
	\end{align*}

	Donc, \[
		g(r, \theta) = r^2.
	\]
\end{exm}

\begin{exm}
	Résoudre \[
		\begin{cases}
			\frac{\partial f}{\partial x} = \frac{x}{x^2+y^2},\\
			\frac{\partial f}{\partial y} = \frac{y}{x^2+y^2}.\\
		\end{cases}
	\]

	On pose $g: (r, \theta) \mapsto f(r \cos\theta, r \sin\theta)$.

	\begin{align*}
		&\frac{\partial g}{\partial r} = \frac{1}{r}\cos^2\theta + \frac{1}{r}\sin^2\theta = \frac{1}{r},\\
		&\frac{\partial g}{\partial \theta} = -\cos(\theta) \sin(\theta) + \sin(\theta)\cos(\theta) = 0.
	\end{align*}

	Donc, \[
		\exists C \in \R, g: (r, \theta) \mapsto \ln r + C
	\] d'où,
	\begin{align*}
		\forall (x,y) \in \R^2 \setminus \{(0,0)\}, f(x,y) &= \ln\left(\sqrt{x^2 + y^2} \right)  + C\\
		&= \frac{1}{2}\ln(x^2 + y^2) + C. \\
	\end{align*}
\end{exm}

\begin{rmk}
	Soit $\mathcal{B} = (e_1, e_2)$ la base canonique de $\R^2$, $f: U \to \R$ de classe $\mathcal{C}^1$ avec $U$ un ouvert de $\R^2$.

	Soit $(x,y) \in U$.

	\begin{align*}
		\Mat_{\mathcal{B}}\big(\nabla f(x,y)\big) = \begin{pmatrix}
			\frac{\partial f}{\partial x}(x,y)\\[2mm]
			\frac{\partial f}{\partial y}(x,y)
		\end{pmatrix}
	\end{align*}

	Soit  \begin{align*}
		\varphi: V &\longrightarrow U \\
		(u,v) &\longmapsto \big(x(u,v), y(u,v)\big) 
	\end{align*} avec $x,y$ de classe $\mathcal{C}^1$. Soit $g = f \circ \varphi$.
	\begin{align*}
		\Mat_{\mathcal{B}}\big(\nabla g(u,v)\big)
		&= \begin{pmatrix}
			\frac{\partial g}{\partial u}(u,v) \\[2mm]
			\frac{\partial g}{\partial v}(u,v)
		\end{pmatrix} \\
		&= \begin{pmatrix}
			\frac{\partial x}{\partial u}(u,v) \frac{\partial f}{\partial x}(x,y)
			+ \frac{\partial y}{\partial u}(u,v)\frac{\partial f}{\partial y}(x,y)\\[3mm]
			\frac{\partial x}{\partial v}(u,v) \frac{\partial f}{\partial x}(x,y)
			+ \frac{\partial y}{\partial v}(u,v) \frac{\partial f}{\partial y}(x,y)
		\end{pmatrix}  \\
		&= \underbrace{\begin{pmatrix}
				\frac{\partial x}{\partial u}(u,v)& \frac{\partial y}{\partial u}(u,v)\\[3mm]
				\frac{\partial x}{\partial v}(u,v)& \frac{\partial y}{\partial v}(u,v)
		\end{pmatrix}}_{J(u,v)} \begin{pmatrix}
			\frac{\partial f}{\partial x}(x,y)\\[3mm]
			\frac{\partial f}{\partial y}(x,y)
		\end{pmatrix} \\
		&= J(u,v) \Mat_{\mathcal{B}}\big(\nabla f(x,y)\big) \\
	\end{align*}
	où $J(u,v) = 
	\begin{pNiceArray}{c:c}
		\Mat_{\mathcal{B}}\big(\nabla x(u,v)\big) & \Mat_{\mathcal{B}}\big(\nabla y(u,v)\big)
	\end{pNiceArray}$.

	On dit que $J(u,v)$ est \underline{la jacobienne} de $\varphi$ en $(u,v)$.
	L'application linéaire canoniquement associée à $J(u,v)$ est la \underline{différentielle de $\varphi$} en $(u,v)$ noté $\mathrm{d}\varphi(u,v)$.

	On a $\mathrm{d}\varphi(u,v) \in \mathcal{L}(R^2)$ et $\Mat_{\mathcal{B}}\big(\mathrm{d}\varphi(u,v)\big) = J(u,v)$.

	Par exemple, la jacobienne du changement de coordonnées polaires est \[
		J = \begin{pmatrix}
			\frac{\partial x}{\partial r} & \frac{\partial y}{\partial r}\\[3mm]
			\frac{\partial x}{\partial \theta} & \frac{\partial y}{\partial \theta}
		\end{pmatrix}
		= \begin{pmatrix}
			\cos\theta&\sin\theta\\
			-r\sin\theta&r\cos\theta
		\end{pmatrix}.
	\]
	$\underbrace{\det(J)}_{\text{le jacobien}} = r\cos^2\theta + r\sin^2\theta = r$

	Dans une intégrale double, si $(x,y) = \varphi(u,v)$, alors $\mathrm{d}x\mathrm{d}y = \det(J)\mathrm{d}u\mathrm{d}v$.

	Ici, \[
		\mathrm{d}x\ \mathrm{d}y = r\ \mathrm{d}r\ \mathrm{d}\theta.
	\]
\end{rmk}

\begin{prv}
	On pose $(x_0, y_0) = \varphi(u_0, v_0)$. Pour tout $(h,k) \in \R^2$ tels que $(u_0 + h, v_0 + k) \in V$, en posant $g = f  \circ \varphi$.

	\begin{align*}
		g(u_0 + h, v_0 + h) &= f\big(x(u_0 + h, v_0 + k), y(u_0 + h, v_0 + k)\big) \\
		&= f\left(
			x(u_0,v_0) + h \frac{\partial x}{\partial u}(u_0,v_0) + k \frac{\partial x}{\partial v}(u_0, v_0) + \po\big(\|(h,k)\|\big), \right.\\
		&\phantom{ = f\bigg(\bigg.}\left. y(u_0, v_0) + h \frac{\partial y}{\partial u}(u_0, v_0) + k \frac{\partial y}{\partial v}(u_0, v_0) + \po\big(\|(h,k)\|\big)
		\right)  \\
		&= f(x_0,y_0) \\
		&~+ \left( h \frac{\partial x}{\partial u}(u_0,v_0) + k \frac{\partial x}{\partial v}(u_0, v_0) + \po(\|(h,k)\|) \right) \frac{\partial f}{\partial x}(x_0,y_0)\\
		&~+ \left( h \frac{\partial y}{\partial u}(u_0, v_0) + k\frac{\partial y}{\partial v}(u_0, v_0) + \po(\|(h,k)\|) \right) \frac{\partial f}{\partial y}(x_0, y_0)\\
		&~+ \po(\|(h,k)\|)\\
		&= f(x_0, y_0) \\
		&~+ h \left( \frac{\partial x}{\partial u}(u_0, v_0) \frac{\partial f}{\partial x}(x_0, y_0) + \frac{\partial y}{\partial u}(u_0, v_0) \frac{\partial f}{\partial y}(x_0, y_0) \right)  \\
		&~+ k\left( \frac{\partial x}{\partial v}(u_0, v_0) \frac{\partial f}{\partial x}(x_0, y_0) + \frac{\partial y}{\partial v}(u_0, v_0) \frac{\partial f}{\partial y}(x_0, y_0) \right) 
		&~+ \po(\|(h,k)\|)\\
		&= g(u_0, v_0) + h \frac{\partial g}{\partial u}(u_0, v_0) + k \frac{\partial g}{\partial v}(u_0, v_0) + \po(\|(h,k)\|) \\
	\end{align*}

	Par identification,
	\[
		\frac{\partial g}{\partial u}(u_0, v_0) = \frac{\partial x}{\partial u}(u_0, v_0) \frac{\partial f}{\partial x}(x_0, y_0) + \frac{\partial y}{\partial u}(u_0, v_0) \frac{\partial f}{\partial y}(x_0,y_0)
	\] et \[
		\frac{\partial g}{\partial v}(u_0, v_0) = \frac{\partial x}{\partial v}(u_0,v_0) \frac{\partial f}{\partial x}(x_0, y_0) + \frac{\partial y}{\partial v}(u_0, v_0) \frac{\partial f}{\partial y}(x_0, y_0).
	\] 
\end{prv}

\begin{exm}
	[Régression linéaire]~\\
	\begin{figure}[H]
		\centering
		\begin{asy}
			import graph;
			axes(EndArrow);
			size(5cm);

			real f(real x) { return x + 0.5; }

			real k = 35 / (7 - 0.5);

			for(int i = 0; i < 35; ++i) {
				real mag = exp(sin(100 * pi/exp(1) * i)) * 0.8 + exp(cos(i*40)/3);
				real eps = mag * cos(10 * exp(1)/pi * i) / 3;
				dot((i/k,f(i/k) + eps));
			}

			draw(graph(f, -1, 7), orange);
		\end{asy}
	\end{figure}
	\[
		y = a x + b
	\] 
	On fixe $(a,b) \in \R^2$. \[
		\varepsilon(a,b) = \sum_{i=1}^n\big( y_i - (ax_i + b) \big)^2
	\] l'erreur totale.

	On veut minimiser $\varepsilon(a,b)$. On a 
	\[
		\forall (a,b) \in \R^2,
		\begin{cases}
			\frac{\partial \varepsilon}{\partial a}(a,b) = -2\sum_{i=1}^{n}(y_i - ax_i - b)x_i,\\
			\frac{\partial \varepsilon}{\partial b}(a,b) = -2\sum_{i=1}^{n}(y_i - ax_i - b).
		\end{cases}
	\]

	Donc,
	\begin{align*}
		(a,b) \text{ point critique de } \varepsilon \iff& \begin{cases}
			a \sum_{i=1}^n {x_i}^2 + b\sum_{i=1}^{n}x_i = \sum_{i=1}^{n} y_ix_i\\
			a\sum_{i=1}^{n}x_i + nb = \sum_{i=1}^ny_i
		\end{cases}\\
		\iff& \begin{cases}
			a \left( \frac{1}{n}\sum_{i=1}^n {x_i}^2 - \overline{x}^2\right) = \overline{y} - \overline{x} \overline{y}\\
			b = \frac{1}{n}\sum_{i=1}^ny_i - \frac{a}{n}\sum_{i=1}^nx_i = \frac{1}{n}\sum_{i=1}^n x_i y_i - \overline{x} \overline{y}
		\end{cases}\\
		&\text{ où } \overline{x} = \frac{1}{n} \sum_{i=1}^n x_i,~\overline{y} = \frac{1}{n}\sum_{i=1}^n y_i\\
		\iff& \begin{cases}
			a = \frac{\Cov(x,y)}{V(x)}\\
			b = \overline{y} - a\overline{x}
		\end{cases}
	\end{align*}

	Coefficient de corrélation: $\frac{\Cov(x,y)}{\sigma_x \sigma_y} \in [-1, 1]$
\end{exm}












		\part{Corps}

\begin{exm}[Problème]
	\begin{itemize}
		\item 
			avec $A = \Z / 9 \Z$, résoudre $\overline{x}^2 = \overline{0}$ \\
			\begin{center}
				\begin{tabular}{|c|c|c|c|c|c|c|c|c|c|c|}
					\hline
					$\overline{x}$&$\overline{0}$& $\overline{1}$ &$\overline{2}$&$\overline{3}$ &$\overline{4}$ &$\overline{5}$ &$\overline{6}$ &$\overline{7}$ &$\overline{8}$& $\overline{9}$ \\
					\hline
					$\overline{x}^2$&$\overline{0}$ &$\overline{1}$ &$\overline{4}$ &$\overline{0}$ &$\overline{7}$ &$7$ &$\overline{0}$ &$\overline{4}$ &$\overline{1}$&$\overline{0}$\\
					\hline
				\end{tabular}
			\end{center}
			On a trouvé 3 solutions: $\overline{0}$, $\overline{3}$, $\overline{6}$.
		\item $\Z / 8\Z$
			\begin{center}
				\begin{tabular}{|c|c|c|c|c|c|c|c|c|}
					\hline
					$\overline{x}$& $\overline{0}$& $\overline{1}$& $\overline{2}$& $\overline{3}$& $\overline{4}$& $\overline{5}$& $\overline{6}$& $\overline{7}$\\
					\hline
					$\overline{x^2}$& $\overline{0}$& $\overline{1}$& $\overline{4}$& $\overline{1}$& $\overline{0}$& $\overline{1}$& $\overline{4}$& $\overline{1}$\\
					\hline
				\end{tabular}
			\end{center}
			$\overline{x}^2=7$ a 4 solutions: $\overline{1}, \overline{7}, \overline{3},\text{ et } \overline{5}$
		\item $A = \mathbbm{H} = \{a + bi + cj + dk  \mid  (a,b,c,d) \in \R^4\}$ \\
			$i^2 = j^2 = k^2 = -1$ 
			\begin{align*}
				\begin{array}{c c c}
					ij = k & jk = i & ji = j\\
					ji = -k & kj = -i & ik = -j
				\end{array}
			\end{align*}
			Dans cet anneau, $-1$ a 6 racines!
	\end{itemize}
\end{exm}

\begin{defn}
	Soit $(\mathbbm{K}, +, \times)$ un ensemble muni de deux lois de composition internes. On dit que c'est un \underline{corps} si
	 \begin{enumerate}
		\item $(\mathbbm{K}, \times)$ est un groupe abélien
		\item $(\mathbbm{K}, \times)$ est un monoïde commutatif
		\item $\forall x \in \mathbbm{K}\setminus \{0_\mathbbm{K}\}, \exists y \in \mathbbm{K}, xy = 1_\mathbbm{K}$
		\item $0_\mathbbm{K} \neq  1_\mathbbm{K}$
	\end{enumerate}
	\index{corps}
\end{defn}

\begin{exm}
	\begin{itemize}
		\item $(\C, +, \times)$ est un corps
		\item $(\R, +, \times)$ est un corps
		\item $(\Q, +, \times)$ est un corps
		\item $(\Z, +, \times)$ n'est pas un corps
	\end{itemize}
\end{exm}

\begin{prop}
	$(\Z / n\Z, +, \times)$ est un corps si et seulement si $n$ est premier.
\end{prop}

\begin{prv}
	\[
		\left( \Z / n\Z \right)^\times = \left\{ \overline{k}  \mid k \wedge n = 1 \right\}
	\] 
\end{prv}


\begin{prop}
	Tout corps est un anneau intègre.
\end{prop}

\begin{prv}
	Soit $(\mathbbm{K}, +, \times)$ un corps. Soient $(a,b) \in \mathbbm{K}^2$ tel que $a \times b = 0_\mathbbm{K}$.\\
	On suppose $a \neq  0_\mathbbm{K}$. Alors, $a$ est inversible et donc \[
		b = a^{-1} \times a \times b = a^{-1} \times 0_\mathbbm{K} = 0_\mathbbm{K}
	\] 
\end{prv}

\begin{exm}
	Soit $(\mathbbm{K},+,\times)$ un corps.\\
	Résoudre \[
		\begin{cases}
			x^2 = 1_\mathbbm{K}\\
			x \in \mathbbm{K}
		\end{cases}
	\]

	\begin{align*}
		x^2 = 1_\mathbbm{K} &\iff x^2 - 1_\mathbbm{K} = 0_\mathbbm{K}\\
		&\iff (x - 1_\mathbbm{K})(x+1_\mathbbm{K}) = 0_\mathbbm{K}\\
		&\iff x - 1_\mathbbm{K} = 0_\mathbbm{K} \text{ ou } x + 1_\mathbbm{K} = 0_\mathbbm{K}\\
		&\iff x = 1_\mathbbm{K} \text{ ou } x = -1_\mathbbm{K}
	\end{align*}

	Il y a au plus 2 solutions.
\end{exm}

\begin{prop}
	Soit $(\mathbbm{K},+,\times )$ un corps et $P$ un polynôme à coefficients dans $\mathbbm{K}$ de degré $n$. Alors, l'équation $P(x) = 0_{\mathbbm{K}}$ a au plus $n$ solutions dans $\mathbbm{K}$ 
	\qed
\end{prop}

\begin{crlr}[(Théorème de Wilson)]
	voir exercice 16 du TD 12
\end{crlr}


\begin{defn}
	Soit $(\mathbbm{K}, +, \times)$ un corps et $L\subset \mathbbm{K}$.\\
	On dit que $L$ est un \underline{sous corps} de $\mathbbm{K}$ si
	\begin{enumerate}
		\item $L$ est un anneau de $(\mathbbm{K}, +, \times)$ non nul
		\item $\forall x \in L\setminus \{0_\mathbbm{K}\}, x^{-1} \in L$ 
	\end{enumerate}
	\vspace{2mm}
	en d'autres termes si
	\begin{enumerate}
		\item $\forall (x,y) \in L^2, x - y \in L$
		\item $\forall (x,y) \in L^2, x \times y^{-1} \in L$
	\end{enumerate}
	\vspace{5mm}
	On dit aussi que $\mathbbm{K}$ est une \underline{extension} de $L$.
	\index{sous corps}
	\index{extension}
\end{defn}

\begin{prop}
	Tout sous corps est un corps. \qed
\end{prop}

\begin{defn}
	Soient $(\mathbbm{K}_1,+,\times )$ et $(\mathbbm{K}_2,+, \times)$ deux corps et $f: \mathbbm{K}_1 \to \mathbbm{K}_2$.\\
	On dit que $f$ est un \underline{morphisme de corps} si $f$ est un morphisme d'anneaux.\\
	i.e. si
	\[
		\begin{cases}
			\forall (x,y) \in {\mathbbm{K}_1}^2,& f(x+y) = f(x) + f(y)\\
			\forall (x,y) \in {\mathbbm{K}_1}^2,& f(x \times y) = f(x) \times f(y)\\
		\end{cases}
	\] 
	\index{homomorphisme (de corps)}
	\index{morphisme (de corps)}
\end{defn}

\begin{prop}
	Tout morphisme de corps est injectif.
\end{prop}

\begin{prv}
	Soit $f: \mathbbm{K}_1 \to \mathbbm{K}_2$ un morphisme de corps.\\
	\begin{itemize}
		\item $\Ker(f)$ est un sous groupe de $(\mathbbm{K}_1, +)$ 
		\item Soit $x \in \Ker(f)$ et $y \in \mathbbm{K}_1$ \[
				f(x \times y) = f(x) \times f(y) = 0_{\mathbbm{K}_2} \times f(y) = 0_{\mathbbm{K}_2}
			\]
		\item Soit $x \in \Ker(f) \setminus \{0_{\mathbbm{K}_1}\}$.\\
			Alors, $x$ est inversible.\\
			\begin{align*}
				\begin{rcases*}
					x \in \Ker(f)\\
					x^{-1} \in \mathbbm{K}_1
				\end{rcases*}& \text{ donc } x \times x ^{-1} \in \Ker(f)\\
				&\text{ donc } 1_{\mathbbm{K}_1} \in \Ker(f)\\
				&\text{ donc } f(1_{\mathbbm{K}_1}) = 0_{\mathbbm{K}_2}
			\end{align*}
			Or, $f(1_{\mathbbm{K}_1}) = 1_{\mathbbm{K}_2} \neq 0_{\mathbbm{K}_2}$
	\end{itemize}
	Donc, $\Ker(f) = \{0_{\mathbbm{K}_1}\}$ donc $f$ est injective.
\end{prv}

\begin{exm}
	$\begin{array}{cc}
		\C &\longrightarrow \C\\
		z &\longmapsto \overline{z}\\
	\end{array}$ est un morphisme de corps
\end{exm}



		\part{Opérations sur les séries}

\begin{prop}
	L'ensemble $E = \{u \in \C^\N  \mid \Sigma u_n \text{ converge}\}$ est un sous-espace vectoriel de $\C^\N$ et \begin{align*}
		S: E &\longrightarrow \C \\
		u &\longmapsto \sum_{n=0}^{+\infty} u_n
	\end{align*} est une forme linéaire.
	\qed
\end{prop}

\begin{rmk}
	La somme d'une série convergente et d'une série divergente diverge.
	Le produit d'une série divergente par un scalaire non nul diverge.
\end{rmk}

		\part{Comparaison de suites}

\begin{defn}
	Soient $u$ et $v$ deux suites réelles. On dit que $u$ est \underline{dominée} par  $v$ si \[
	\exists M\in \R, \exists N\in \N,\forall n\ge N,\left| u_n \right| \le M \left| v_n \right| 
	\] Dans ce cas, on note $u = O(v)$ ou $u_n = O(v_n)$ et on dit que "$u$ est un grand o de $v$"
\end{defn}

\begin{exm}
	En informatique, on dit qu'un alogirithme a une \underline{complexité linéaire} si son temps d'éxécution est un $O(n)$ 
	Par exemple, on calcule $a^n$ 

	\begin{itemize}
		\item Approche naïve
			\begin{algorithm}
				\begin{algorithmic}[1]
					\State $p \gets 1$
					\For{$i \in \left\llbracket 0,n-1 \right\rrbracket$}
						\State $p \gets p \times a$
					\EndFor
					\State \Return p
				\end{algorithmic}
			\end{algorithm}
			Complexité linéaire $O(n)$
		\item Exponentiation rapide\\
			On écrit $n$ en binaire: \begin{align*}
				n &= \overline{a_k a_{k-1}\ldots a_0}^{(2)}\\
					&= \sum_{i=0}^{k} a_i 2^i
			\end{align*} avec $(a_i) \in \left\{ 0,1 \right\} ^{k+1}$
			\begin{align*}
				a^n &= a^{\sum_{i=0}^{k} a_i 2^i} \\
				&= \prod_{i=0}^{k} a^{a_i 2^i}  \\
			\end{align*}
			
			\begin{algorithm}
				\begin{algorithmic}
					[1]

					\State $s \gets 0$
					\State $p \gets a$
					\For{ $i \in \left\llbracket 0, \log_2(n) \right\rrbracket$}
						\State $p \gets p \times p$
						\If{$a[i] = 1$}
							\State $s \gets s + p$
						\EndIf
					\EndFor
					\State \Return s
				\end{algorithmic}
			\end{algorithm}
			Compléxité logarithmique $O(\log_2(n))$
	\end{itemize}
\end{exm}


\begin{prop}
	$O$ est une relation réfléctive et transitive.
\end{prop}

\begin{prv}
	\begin{itemize}
		\item Soit $u$ une suite. On pose $M = 1$ et \[
			\forall n \in \N, \left| u_n \right| \le M \left| u_n \right|
			\] Donc $u = O(u)$.
		\item Soient $u, v, w$ trois suites telles que  \[
		\begin{cases}
			u = O(v)\\
			v = O(w)
		\end{cases}
		\] Soient $M_1,M_2 \in \R$ et $N_1,N_2\in \N$ tels que \[
		\begin{cases}
			\forall n \ge  N_1, \left| u_n \right| \le M_1 \left| v_n \right| \\
			\forall n \ge  N_2, \left| v_n \right| \le M_2 \left| w_n \right| \\
		\end{cases}
		\] 

		Nécéssairement, $M_1\ge 0$ et $M_2\ge 0$.\\
		Soit $N = \max(N_1,N_2)$. \[
		\forall n \ge  N, \left| u_n \right| \le M_1 \left| v_n \right| \le  M_1M_2 \left| w_n \right| 
		\] Donc $u = O(w)$
	\end{itemize}
\end{prv}

\begin{defn}
	Soient $u$ et $v$ deux suites. On dit que $u$ est \underline{négligeable} devant $v$ si \[
	\forall \varepsilon>0, \exists N\in \N, \forall n\ge N, \left| u_n \right| \le \varepsilon \left| v_n \right| 
	\] Dans ce cas, on note $u = o(v)$ ou $u_n = o(v_n)$ ou on le lit "$u$ est un petit o de $v$"
\end{defn}

\begin{prop}
	$o$ est une relation transitive, non-réfléctive
\end{prop}

\begin{prv}
	\begin{itemize}
		\item Soient $u$, $v$ et $w$ trois suites telles que \[
			\begin{cases}
				u = o(v)\\
				v = o(w)
			\end{cases}
			\] Soit $\varepsilon>0$. Soit $N_1\in \N$ tel que \[
			\forall n \ge N_1, \left| u_n \right| \le \sqrt{\varepsilon}  \left| v_n \right| 
			\] Soit $N_2\in \N$ tel que \[
			\forall n \ge N_2, \left| v_n \right| \le \sqrt{\varepsilon}  \left| w_n \right| 
			\] On pose $N = \max(N_1,N_2)$, alors \[
			\forall n \ge N, \left| u_n \right| \le \sqrt{\varepsilon}  \left| v_n \right| \le \underbrace{\sqrt{\varepsilon} \times \sqrt{\varepsilon}} _\varepsilon \left| w_n \right| 
			\] donc $u = o(w)$
		\item Soit $u$ une suite tel qu'il existe $N \in \N$ tel que \[
		\forall n \ge N, u_n > 0
		\] On suppose que $u = o(u)$, alors \[
		\forall \varepsilon>0,\exists N \in \N, \forall n \ge N, \left| u_n \right| \le \varepsilon \left| u_n \right| 
		\] On pose $\varepsilon = \frac{1}{2}$ alors \[
		\exists N \in \N, \forall n \ge N, \left| u_n \right| \le \frac{1}{2} \left| u_n \right| 
		\] une contradiction
	\end{itemize}
\end{prv}

\begin{prop}
	Soient $u$ et $v$ deux suites.
	\begin{itemize}
		\item $o(u) + o(u) = o(u)$
		\item $v \times o(u) = o(uv)$
		\item $o(u) \times o(v) = o(uv)$
		\item $o(o(u)) = o(u)$
	\end{itemize}
	\qed
\end{prop}

\begin{defn}
	Soient $u$ et $v$ deux suites. On dit que $u$ et $v$ sont \underline{équivalentes} si \[
	u = v + o(v)
	\] i.e. \[
	\forall \varepsilon >0, \exists N \in \N, \forall n \ge N, \left| u_n-v_n \right| \le \varepsilon\left| v_n \right| 
	\] Dans ce cas, on le note $u \sim v$
\end{defn}

\begin{prop}
	$\sim$ est une relation d'équivalence \qed
\end{prop}

\begin{prop}
	Soient $(u,v) \in \R^\N$. On suppose que $v$ ne s'annule pas à partir d'un certain rang
	\begin{enumerate}
		\item $u = o(v) \iff \left( \frac{u_n}{v_n} \right)$ bornée
		\item $u = o(v) \iff \frac{u_n}{v_n} \tendsto{n \to  +\infty} 0$
		\item $u \sim v \iff \frac{u_n}{v_n} \tendsto{n \to  +\infty} 1$
	\end{enumerate}
	\qed
\end{prop}

\begin{prop}
	[Suites de références]
	\begin{enumerate}
		\item $\ln^\alpha(n) = o(n^\beta)$ avec $(\alpha,\beta) \in \left( \R^+_* \right) ^2$ 
		\item $n^\beta = o(a^n)$ avec $\beta > 0$ et $a > 1$ 
		\item $a^n = o(n!)$ avec $a >1$ 
		\item $n! = o(n^n)$
	\end{enumerate}
\end{prop}


\begin{lem}
	[Exercice 10 du TD]
	Soit $u \in \left(\R^+_*\right)^\N$\\
	Si $\frac{u_{n+1}}{u_n} \tendsto{n \to +\infty} \ell < 1$ avec $\ell\in \R$,\\ alors $u_n \tendsto{n \to +\infty} 0$
\end{lem}

\begin{prv} [de la proposition]
	\begin{enumerate}
		\item par croissance comparée
		\item On pose $\forall n \in \N^*, u_n = \frac{n^\beta}{a^n}$. 
			\begin{align*}
				\forall  n \in \N^*, \frac{u_{n+1}}{u_n} &= \left( \frac{n+1}{n} \right) ^\beta \times \frac{1}{a} \\
				&= \frac{1}{a}\left( 1+\frac{1}{n} \right) ^\beta \\
				&\tendsto{n \to +\infty} \frac{1}{a} < 1
			\end{align*}
			Donc, $u_n \tendsto{n \to  +\infty} 0$
		\item On pose $\forall n \in \N, u_n = \frac{a^n}{n!}$ \[
			\forall n \in \N, \frac{u_{n+1}}{u_n} = \frac{a}{n+1} \tendsto{n \to +\infty} 0 < 1
			\] donc $u_n \tendsto{n \to +\infty} 0$
		\item On pose $\forall  n\in \N^*, u_n = \frac{n!}{n^n}$.
			\begin{align*}
				\forall n \in \N^*, \frac{u_{n+1}}{u_n}
				&= (n+1) {\frac{n^n}{(n+1)^{n+1}}} \\
				&= \left( \frac{n}{n+1} \right) ^n \\
				&= e^{n \ln\left( \frac{n}{n+1} \right) } \\
				&= e^{n \ln\left( 1+\frac{1}{n+1} \right)} \\
				&= e^{n(-\frac{1}{n} + o(\frac{1}{n})} \\
				&= e^{-1 + o(1)} \\
				&\tendsto{n \to  +\infty} e^{-1}<1
			\end{align*}
			donc $u_n \tendsto{n\to +\infty} 0$
	\end{enumerate}
\end{prv}

		\part{Matrices par blocs}

\begin{exm}
	Soit $p$ un projecteur de $E$ : \[
		E = \Ker p \oplus \mathrm{Im}\ p
	\] Soit $\mathcal{B} = (e_1, \ldots, e_k, e_{k+1}, \ldots, e_n)$ une base de $E$ avec $\begin{cases}
		\mathrm{Im}(p) = \Vect(e_1, \ldots, e_k)\\
		\Ker(p) = \Vect(e_{k+1}, \ldots, e_n)\\
	\end{cases}$

	Alors, 
	\begin{align*}
		\Mat_\mathcal{B}(p) =
		\left(\begin{NiceArray}{c c c | c c c}
				1&&&0&\Cdots&0\\
				 &\Ddots&&\Vdots&&\Vdots\\
				&&1&0&\Cdots&0\\\hline
				0&\Cdots&0&0&\Cdots&0\\
				\Vdots&&\Vdots&\Vdots&&\Vdots\\
				0&\Cdots&0&0&\Cdots&0\\
		\end{NiceArray}\right)
		= \left( \begin{array}{c|c}
				I_k & 0\\ \hline
				0&0
		\end{array}\right) \\
	\end{align*}

	De même, si $\s$ est une symétrie de $E$, \[
		E = \Ker(\s - \id_E) \oplus \Ker(\s + \id_E)
	.\] Soit $\mathcal{C} = (e_1', \ldots, e_\ell', e_{\ell+1}', \ldots, e'_n)$ avec $\begin{cases}
		\Vect(e'_1, \ldots, e'_\ell) = \Ker(\s - \id_E),\\
		\Vect(e'_{\ell+1}, \ldots, e'_n) = \Ker(\s + \id_E).\\
	\end{cases}$

	Alors
	\[
		\Mat_\mathcal{C}(\s) = \left(\begin{array}{c|c}
				I_\ell &0\\ \hline
				0&-I_{n-\ell}
		\end{array}\right) 
	\]
\end{exm}

\begin{prop}
	Soient $F$ et $G$ supplémentaires dans $E$ : \[
		E = F \oplus G.
	\] Soit $f \in \mathcal{L}(F)$ et $g \in \mathcal{L}(G)$. Alors \[
	\exists !h \in \mathcal{L}(E) h_{|F} = f,\ h_{|G} = g \et h = f \circ p + g \circ q
	\] où $\begin{cases}
		p \text{ est la projection sur $F$ parallèlement à $G$}\\
		q \text{ est la projection sur $G$ parallèlement à $F$}\\
	\end{cases}$.

	On a aussi $q = \id_E - p$.
\end{prop}

\begin{prv}
	\begin{itemize}
		\item[\sc \underline{Analyse}] Soit $h \in \mathcal{L}(E)$ tel que $\begin{cases}
				h_{|F}=f\\
				h_{|G}=g
			\end{cases}$.

			Soit $x \in E$. Alors \[
				x = \underbrace{p(x)}_{\in F} + \underbrace{q(x)}_{\in G}
			\]

			Donc,
			\begin{align*}
				h(x) &= h\big(p(x)\big) + h\big(q(x)\big)\\
				&= f\big(p(x)\big) + g\big(q(x)\big) \\
				&= (f \circ p + g \circ q)(x) \\
			\end{align*}
			Si $h$ existe, alors \[
				h = f \circ p + g \circ q
			\]
		\item[\underline{\sc Synthèse}] On pose $h = f \circ p + g  \circ q$.

			$p$, $q$, $f$ et $g$ sont linéaires donc $h$ aussi.

			Soit $x \in E$.
			\begin{align*}
				h(x) &= f\big(p(x)\big) + g\big(q(x)\big) \\
				&= f(x) + g(0_E) \\
				&= f(x) \\
			\end{align*}
			donc $h_{|F} = f$ et de même $h_{|G}=g$.
	\end{itemize}
\end{prv}

\begin{prop}
	On reprend les notations et hypothèses précédentes. Soit $(e_1, \ldots, e_p)$ une base de $F$, et $(f_1, \ldots, f_q)$ une base de $G$. Alors, $\mathcal{B} = (e_1, \ldots, e_p, f_1, \ldots, f_q)$ est une base de $E$ et \[
		\Mat_\mathcal{B}(h) = \left(
		\begin{array}{c|c}
			A&0\\ \hline
			0&B
		\end{array}\right)
	\] où $\begin{cases}
		A = \Mat_{(e_1, \ldots e_p)}(f)\\
		B = \Mat_{(f_1, \ldots, f_q)}(g)
	\end{cases}$
	\qed
\end{prop}

\begin{prop}
	Soient $(A,A') \in \mathcal{M}_n(\mathbbm{K})^2$ et $(B,B') \in \mathcal{M}_p(\mathbbm{K})^2$.
	\begin{enumerate}
		\item \[
				\left(\begin{array}{c|c}
					A&0\\ \hline
					0&B
				\end{array}\right)
				\left(\begin{array}{c|c}
					A'&0\\ \hline
					0&B'
				\end{array}\right) = 
				\left(\begin{array}{c|c}
					AA'&0\\ \hline
					0&BB'
				\end{array}\right)
			\]
		\item \[
				\left(\begin{array}{c|c}
					A&0\\ \hline
					0&B
				\end{array}\right) \in \mathrm{GL}_{n+p}(\mathbbm{K})	 \iff \begin{cases}
					 A \in \mathrm{GL}_n(\mathbbm{K})\\
					 B \in \mathrm{GL}_p(\mathbbm{K})
				\end{cases}
			\] et dans ce cas, \[
				\left(\begin{array}{c|c}
					A&0\\ \hline
					0&B
				\end{array}\right)^{-1} =
				\left(\begin{array}{c|c}
					A^{-1}&0\\ \hline
					0&B^{-1}
				\end{array}\right)
			\]
		\item \[
				\tr \left(\begin{array}{c|c}
					A&0\\ \hline
					0&B
				\end{array}\right) = \tr A + \tr B
			\]
	\end{enumerate}
\end{prop}

\begin{prv}
	\begin{enumerate}
		\item Soit $\begin{cases}
				f \in \mathcal{L}(F) \text{ tel que } \Mat_\mathcal{B}(f) = A,
				f' \in \mathcal{L}(F) \text{ tel que } \Mat_\mathcal{B}(f') = A',
				g \in \mathcal{L}(G) \text{ tel que } \Mat_\mathcal{C}(g) = B,
				g' \in \mathcal{L}(G) \text{ tel que } \Mat_\mathcal{C}(g') = B'
			\end{cases}$ où $\begin{cases}
				F \oplus G = \mathbbm{K}^{n+p},\\
				\dim(F) = n, \dim(G) = p,\\
				\mathcal{B} \text{ base de } F,\\
				\mathcal{C} \text{ base de } G.\\
			\end{cases}$
			Soit $\begin{cases}
				h \in \mathcal{L}(\mathbbm{K}^{n+p}) \text{ tel que } \begin{cases}
					h_{|F} = f\\
					h_{|G} = g
				\end{cases}\\
				h' \in \mathcal{L}(\mathbbm{K}^{n+p}) \text{ tel que } \begin{cases}
					h'_{|F} = f'\\
					h'_{|G} = g'\\
				\end{cases}
			\end{cases}$
			Soit $\mathcal{D} = \mathcal{B} \cup \mathcal{C}$ une base de $\mathbbm{K}^{n+p}$.
			\begin{align*}
				\left(\begin{array}{c|c}
					A&0\\ \hline
					0&B
				\end{array}\right)
				\left(\begin{array}{c|c}
					A'&0\\ \hline
					0&B'
				\end{array}\right) &= \Mat_{\mathcal{D}}(h) \Mat_{\mathcal{D}}(h')\\
				&= \Mat_{\mathcal{D}}(h \circ h') \\
			\end{align*}
			Or, $(h \circ h')_{|F} = f \circ f'$ et $(h \circ h')_{|G} = g \circ g'$.

			Donc,
			\begin{align*}
				\Mat_\mathcal{D}(h \circ h') &=
					\left(\begin{array}{c|c}
						\Mat_\mathcal{B}(f \circ f')&0\\ \hline
						0&\Mat_\mathcal{C}(g \circ g')
					\end{array}\right)\\
				&=\left(\begin{array}{c|c}
					AA'&0\\ \hline
					0&BB'
				\end{array}\right).
			\end{align*}
	\end{enumerate}
\end{prv}

\begin{prop}
	Soient $A,A' \in \mathcal{M}_n(\mathbbm{K})$, $B,B' \in \mathcal{M}_{n,p}(\mathbbm{K})$, $C,C' \in \mathcal{M}_{p,n}(\mathbbm{K})$ et $D, D' \in \mathcal{M}_p(\mathbbm{K})$.

	\[
		\left(\begin{array}{c|c}
			A&B\\ \hline
			C&D
		\end{array}\right)
		\left(\begin{array}{c|c}
			A'&B'\\ \hline
			C'&D'
		\end{array}\right) = 
		\left(\begin{array}{c|c}
			AA' + BC'& AB' + BD'\\ \hline
			CA' + DC'&CB' + DD'
		\end{array}\right)
	\] Cette formule se généralise à un nombre quelconque de blocs : \[
		\left(\begin{array}{c|c|c|c}
				A_{11}&A_{12}&\cdots&A_{1,n}\\ \hline
				A_{21}&A_{22}&\cdots&A_{2,n}\\ \hline
				\vdots&\vdots&\ddots&\vdots\\ \hline
				A_{p,1}&A_{p,2}&\cdots&A_{p,n}
		\end{array}\right)
		\left(\begin{array}{c|c|c|c}
				A'_{11}&A'_{12}&\cdots&A'_{1,n}\\ \hline
				A'_{21}&A'_{22}&\cdots&A'_{2,n}\\ \hline
				\vdots&\vdots&\ddots&\vdots\\ \hline
				A'_{p,1}&A'_{p,2}&\cdots&A'_{p,n}
		\end{array}\right)
	\] Cette matrice se calcyle comme on s'y attend si les dimensions des blocs autorisent les produits.
\end{prop}

\begin{prop}
	Le rang d'une matrice $A$, c'est la taille de la plus grande matrice carrée inversible que l'on peut extraire de $A$.
	\qed
\end{prop}




		\part{Trigonométrie hyperbolique}

\begin{defn}
	Pour tout $x \in \R$, on pose \[
		\begin{cases}
			\ch x = \frac{e^x + e^{-x}}{2},\\
			\sh x = \frac{e^x - e^{-x}}{2},\\
			\th x = \frac{\sh x}{\ch x}.
		\end{cases}
	\]

	$\ch$ est appelé \underline{cosinus hyperbolique}, $\sh$ est appelé \underline{sinus hyperbolique} et $\th$ est appelé \underline{tangeante hyperbolique}.
	\index{cosinus hyperbolique}
	\index{sinus hyperbolique}
	\index{tangente hyperbolique}
\end{defn}

\begin{rmk}
	Ces formules rappèlent les formules d'Euler : pour tout $x \in \R$,
	\begin{align*}
		\cos x = \frac{e^{ix} + e^{-ix}}{2}\quad\longleftrightarrow\quad\ch x = \frac{e^x + e^{-x}}{2}\\
		\sin x = \frac{e^{ix} - e^{-ix}}{2i}\quad\longleftrightarrow\quad\sh x = \frac{e^x - e^{-x}}{2}\\
	\end{align*}
\end{rmk}

\begin{figure}[H]
	\centering
	\begin{asy}
		import graph;

		size(12cm);

		pair A = (-2, 0);
		pair B = (2, 0);

		real eps = 0.05;

		draw(shift(A) * ((0, -1.3) -- (0, 1.3)), Arrow(TeXHead));
		draw(shift(A) * ((-1.3, 0) -- (1.3, 0)), Arrow(TeXHead));

		draw(circle(A, 1), magenta);
		
		real theta = 38;
		pair M = dir(theta) + A;
		draw(A -- M, red);
		draw(arc(A, 0.35, 0, theta), red, Arrow(TeXHead));
		draw(M -- (A.x-eps, M.y), dashed);
		draw(M -- (M.x, A.y-eps), dashed);
		label("\small$\theta$", 0.5dir(theta/2) + A, red);
		label("\small$\cos\theta$", (M.x, A.y), align=S);
		label("\small$\sin\theta$", (A.x, M.y), align=1.2W);
		dot("\small$M$", M);

		label("\small$x^2 + y^2 = 1$", A + 1.5dir(45+180));

		draw(shift(B) * ((0, -1.3) -- (0, 1.3)), Arrow(TeXHead));
		draw(shift(B) * ((-1.3, 0) -- (1.3, 0)), Arrow(TeXHead));

		real ch(real x) { return (exp(x) + exp(-x)) / 2.; }
		real sh(real x) { return (exp(x) - exp(-x)) / 2.; }
		real nch(real x) { return -ch(x); }

		real k = 1.9; real r = 1.2;
		real t = 1.4;

		draw(shift(B) * scale(0.35) * graph(ch, sh, -k, k), magenta);
		draw(shift(B) * scale(0.35) * graph(nch, sh, -k, k), magenta);

		label("\small$x^2 - y^2 = 1$", B + 1.5dir(45+180) + (0, -0.2));

		M = B + 0.35(ch(t), sh(t));

		draw(M -- (B.x-eps, M.y), dashed);
		draw(M -- (M.x, B.y-eps), dashed);
		dot("\small$M$", M);
		label("\small$\ch x$", (M.x, B.y), align=S);
		label("\small$\sh x$", (B.x, M.y), align=1.2W);

		draw(shift(B) * ((-r, -r)--(r,r)), gray + dashed);
		draw(shift(B) * ((r, -r)--(-r,r)), gray + dashed);
	\end{asy}
\end{figure}


	}

	{
		\chap[05]{Calcul intégral}
		\renewcommand{\cwd}{../chap05}
		\begin{defn}
	Soit $E$ un $\mathbbm{K}$-espace vectoriel. On dit que $E$ est de \underline{dimension finie} si $E$ a au moins une famille génératrice finie. On dit que $E$ est de \underline{dimension infinie} sinon.
	\index{dimension finie (espace vectoriel)}
	\index{dimension infinie (espace vectoriel)}
\end{defn}

\begin{thm}
	[Théorème de la base extraite]
	Soit $E$ un $\mathbbm{K}$-espace vectoriel non nul de dimension finie. Soit $\mathcal{G}$ une famille génératrice finie de $E$. Alors, il existe une base $\mathcal{B}$ de $\mathcal{E}$ telle que $\mathcal{B} \subset \mathcal{G}$.
\end{thm}

\begin{prv}
	[par récurrence sur $\#G = \Card(G)$]
	\begin{itemize}
		\item Soit $E$ un $\mathbbm{K}$-espace vectoriel non nul engendré par $\mathcal{G} = (u)$.\\
			Si $u = 0_E$, alors $E = \{0_E\}$: une contradiction $\lightning$ \\
			Donc $u \neq 0_E$ donc $(u)$ est libre. En effet, \[
				\forall \lambda \in \mathbbm{K}, \lambda u = 0_E \implies \lambda = 0_\mathbbm{K}
			\] Donc $\mathcal{G}$ est une base de $E$.\\
		\item Soit $n \in \N_*$. Soit $E$ un $\mathbbm{K}$-espace vectoriel. On suppose que si $E$ a une famille génératrice constituée de $n$ vecteurs, alors on peut extraire de cette famille une base de $E$.\\
			Soit $\mathcal{G}$ une famille génératrice de $E$ avec $n+1$ vecteurs.\\
			Si $\mathcal{G}$ est libre, alors $\mathcal{G}$ est une base de $E$. \\
			Si $\mathcal{G}$ n'est pas libre, alors il existe $u \in \mathcal{G}$ tel que $u \in \Vect(\mathcal{G}\setminus \{u\})$ \\
			Donc $\mathcal{G}\setminus \{u\}$ engendre $E$. Or, $\mathcal{G}\setminus \{u\}$ possède $n$ vecteurs. D'après l'hypothèse de récurrence, il existe une base $\mathcal{B}$ de $E$ telle que \[
				\mathcal{B} \subset \mathcal{G} \setminus \{u\} \subset \mathcal{G}
			\] 
	\end{itemize}
\end{prv}

\begin{crlr}
	Tout espace de dimension finie a une base.
	\qed
\end{crlr}

\begin{thm}
	[Théorème de la base incomplète]
	Soit $E$ un $\mathbbm{K}$-espace vectoriel de dimension finie, $\mathcal{G}$ une famille génératrice finie de $E$. $\mathcal{L}$ une famille libre de $E$. Alors, il existe une base $\mathcal{B}$ de $E$ telle que \[
		\mathcal{L} \subset \mathcal{B} \text{ et } \mathcal{B}\setminus \mathcal{L} \subset \mathcal{G}
	\] 
\end{thm}

\begin{prv}
	[par récurrence sur $\#(\mathcal{G}\setminus\mathcal{L})$]
	\begin{itemize}
		\item Avec les notations précédentes, on suppose que $\mathcal{G}\setminus\mathcal{L} \neq \O$ \[
				\forall u \in \mathcal{G}, u \in \mathcal{L}
			\] Donc $\mathcal{G} \subset \mathcal{L}$ donc $\mathcal{L}$ est génératrice donc $\mathcal{L}$ est une base de $E$. On pose $\mathcal{B} = \mathcal{L}$ et alors \[
				\mathcal{L} \subset  \mathcal{B} \text{ et } \mathcal{B}\setminus\mathcal{L} = \O \subset  \mathcal{G}
			\] 
		\item Soit $n \in \N$. On suppose que si $\mathcal{G}$ est génératrice et $\mathcal{L}$ libre avec $\#(\mathcal{G}\setminus\mathcal{L}) = n$ alors il existe une base $\mathcal{B}$ de $E$ telle que \[
			\mathcal{L}\subset \mathcal{B} \text{ et } \mathcal{B}\setminus\mathcal{L}\subset \mathcal{G}
		\] Soient à présent $\mathcal{G}$ une famille génératrice de $E$ et $\mathcal{L}$ une famille libre de $E$ telles que $\#(\mathcal{G}\setminus\mathcal{L}) = n+1 > 0$\\
		Si $\mathcal{L}$ engendre $E$, alors $\mathcal{L}$ est une base de $E$. On pose $\mathcal{B} = \mathcal{L}$ et on a bien \[
			\mathcal{L} \subset  \mathcal{B} \text{ et } \mathcal{B} \setminus \mathcal{L} = \O \subset  \mathcal{G}
		\] On suppose que $\mathcal{L}$ n'engendre pas $E$. Il existe $u \in \mathcal{G}$ tel que $u \not\in \Vec(\mathcal{L})$ (car sinon, $\mathcal{G} \subset \Vect(\mathcal{L})$ et donc $\underbrace{\Vect(\mathcal{G})}_{= E} \subset  \underbrace{\Vect(\mathcal{L})}_{ \subset E}$\\
		Donc $\mathcal{L} \cup \{u\} $ est libre. On pose $\mathcal{L}' = \mathcal{L} \cup \{u\} $ \[
			\mathcal{G}\setminus \mathcal{L}' = \mathcal{G}\setminus (\mathcal{L} \cup \{u\}) = (\mathcal{G}\setminus\mathcal{L})\setminus \{u\} 
		\] donc $\#(\mathcal{G}\setminus\mathcal{L}') = n+1 -1 = n$\\
		D'après l'hypothèse de récurrence, il existe $\mathcal{B}$ une base de $E$ telle que \[
			\mathcal{L} \subset  \mathcal{L}' \subset \mathcal{B} \text{ et } \mathcal{B}\setminus \mathcal{L}' \subset \mathcal{G}
		\] \[
			\mathcal{B} \setminus \mathcal{L} = \underbrace{\mathcal{B}\setminus\mathcal{L}'}_{\subset \mathcal{G}} \cup \underbrace{\{u\}}_{\subset \mathcal{G} \text{ car } u \in \mathcal{G}}
		\] On a $\mathcal{B}\setminus\mathcal{L}\subset \mathcal{G}$
	\end{itemize}
\end{prv}

\begin{thm}
	Soit $E$ un $\mathbbm{K}$-espace vectoriel de dimension finie. Toutes les bases de $E$ ont le même cardinal.
\end{thm}

\begin{prv}
	Soit $\mathcal{G}$ une famille génératrice finie de $E$ et $\mathcal{B} \subset  \mathcal{G}$ une base de $E$. On note $n = \#\mathcal{B}$ \\
	Soit $\mathcal{B}'$ une base de $E$. On pose $p = n - \#(\mathcal{B} \cap  \mathcal{B}')$. Montrons par récurrence sur  $p$ que $\#\mathcal{B} = \#\mathcal{B}'$ 
	\begin{itemize}
		\item On suppose que $p = 0$. Alors, $\#(\mathcal{B} \cap \mathcal{B}') = n$ \\
			Or, $\mathcal{B}' \cap \mathcal{B} \subset \mathcal{B}$ donc $\mathcal{B} \cap \mathcal{B}' = \mathcal{B}$ donc $\mathcal{B} \subset  \mathcal{B}'$ et donc $\mathcal{B} = \mathcal{B}'$ 
		\item Soit $p \in \N$. On suppose que si $\mathcal{B}'$ est une base de $E$ telle que $n - \#(\mathcal{B} \cap \mathcal{B}') = p$, alors $\#\mathcal{B}' = n$ \\
			Aoit $\mathcal{B}'$ une base de $E$ telle que $n - \#(\mathcal{B}\cap \mathcal{B}') = p+1 > 0$ \\
			Donc $\mathcal{B} \cap \mathcal{B}' \neq \mathcal{B}$. Soit $u \in \mathcal{B}' \setminus \mathcal{B}$. D'après le lemme d'échange, il existe $v \in \mathcal{B}\setminus \mathcal{B}'$ tel que $\mathcal{B}' \setminus \{u\} \cup \{v\}$ est une base de $E$. On pose $\mathcal{B}'' = \mathcal{B}' \setminus \{u\} \cup \{v\}$ 
			\begin{align*}
				\mathcal{B}'' \cap \mathcal{B} &= \left( (\mathcal{B}' \setminus \{u\})  \cap \mathcal{B} \right) \cup \{v\} \\
				&= (\mathcal{B}' \cap \mathcal{B}) \cup \{v\} \\
			\end{align*}
			donc,
			\begin{align*}
				n - \#(\mathcal{B}'' \cap \mathcal{B}) &= n - (\#(\mathcal{B}' \cap \mathcal{B}) + 1) \\
				&= p+1- 1 \\
				&= p \\
			\end{align*}
			D'après l'hypothèse de récurrence, \[
				\#\mathcal{B}'' = n
			\] Or, $\#\mathcal{B}'' = \#\mathcal{B}'$
	\end{itemize}
\end{prv}

\begin{lem}
	Soient $\mathcal{B}$ et $\mathcal{B}'$ deux bases de $E$ telles que $\mathcal{B}\subset \mathcal{B}'$. Alors, $\mathcal{B} = \mathcal{B}'$.
\end{lem}

\begin{prv}
	On suppose $\mathcal{B}' \neq \mathcal{B}$. Soit $u \in \mathcal{B}' \setminus \mathcal{B}$
	$u \in E = \Vect(\mathcal{B})$ donc $\mathcal{B} \cup \{u\}$ n'est pas libre.
	Donc $\mathcal{B}\cup \{u\} \subset \mathcal{B}'$ et $\mathcal{B}'$ est libre donc $\mathcal{B}\cup \{u\}$ est libre: une contradiction $\lightning$
\end{prv}

\begin{lem}
	[Lemme d'échange] Soient $\mathcal{B}_1$ et $\mathcal{B}_2$ deux bases de $E$ et $u \in \mathcal{B}_1 \setminus \mathcal{B}_2$. Alors, il existe $v \in \mathcal{B}_2$ tel que $(\mathcal{B}_1 \setminus \{u\}) \cup \{v\}$ soit une base de $E$.
\end{lem}

\begin{prv}
	[1${}^\text{nde}$ méthode]
	On suppose que pout tout $v \in \mathcal{B}_2$, $(\mathcal{B}_1\setminus \{u\}) \cup \{v\}$ n'est pas une base de $E$
	Soit $v \in \mathcal{B}_2$.
	\begin{itemize}
		\item Supposons $(\mathcal{B}_1\setminus \{u\})\cup \{v\}$ non libre. $\mathcal{B}_1 \setminus \{u\}$ est libre. Donc $v \in \Vect(\mathcal{B}_1 \setminus \{u\})$
		\item Supposons $(\mathcal{B}_1\setminus \{u\}) \cup \{v\}$ non génératrice.
			Comme $\mathcal{B}_1$ engendre $E$, $u \not\in \Vect(\mathcal{B}_1\setminus \{v\})$.
			On suppose que $\mathcal{B}_1 \neq \mathcal{B}_2$.
			$\forall v \in \mathcal{B}_2 \setminus \mathcal{B}_1, \Vect(\mathcal{B}_1 \setminus \{v\}) = \Vect(\mathcal{B}_1) = E \ni u$ 
			donc, $(\mathcal{B}_1\setminus \{u\}) \cup \{v\}$ engendre $E$ et donc \[
				v \in \Vect(\mathcal{B}_1 \setminus \{u\})
			\] On a aussi \[
				\forall v \in \mathcal{B}_1 \setminus \{u\}, v \in \Vect(\mathcal{B}_1\setminus \{u\})
			\] Comme $u \not\in \mathcal{B}_2$, on a \[
				\forall v \in \mathcal{B}_2, v \in \Vect(\mathcal{B}_1\setminus \{u\})
			\] docn \[
				E = \Vect(\mathcal{B}_2) \subset \Vect(\mathcal{B}_1\setminus \{u\})
			\] donc $\mathcal{B}_1\setminus \{u\}$ engendre $E$ donc $\mathcal{B}_1\setminus \{u\}$ est une base de $E$. Or, $\mathcal{B}_1 \setminus \{u\}  \subset  \mathcal{B}_1$, donc $\mathcal{B}_1\setminus \{u\} = \mathcal{B}_1$
	\end{itemize}
\end{prv}

\begin{prv}
	[2${}^\text{nde}$ méthode]
	On suppose que pout tout $v \in \mathcal{B}_2$, $(\mathcal{B}_1\setminus \{u\}) \cup \{v\}$ n'est pas une base de $E$
	\begin{itemize}
		\item Comme $u \in \mathcal{B}_1 \setminus \mathcal{B}_2$, nécéssairement $\mathcal{B}_1 \neq \mathcal{B}_2$ donc $\mathcal{B}_2 \not\subset \mathcal{B}_1$, donc $\mathcal{B}_2\setminus\mathcal{B}_1 \neq \O$ 
		\item Soit $v \in \mathcal{B}_2\setminus\mathcal{B}_1$. Il existe $(\lambda_w)_{w\in\mathcal{B}_1}$ une famille de scalaires presque nulle telle que \[
				v = \sum_{w \in \mathcal{B}_1} \lambda_w w - \lambda_u u + + \sum_{w \in \mathcal{B}_1\setminus \{u\}}\lambda_w w
			\]
			Si $\lambda_u \neq 0_E$, alors
			\begin{align*}
				u &= \lambda_u^{-1}\left( v - \sum_{w \in \mathcal{B}_1 \setminus \{u\}} \lambda_w w \right)\\
					&\in \Vect(\mathcal{B}_1\setminus \{u\} \cup v)
			\end{align*}
			 donc $\mathcal{B}_1 \subset \Vect(\mathcal{B}_1\setminus \{u\} \cup \{v\})$\\
			 et donc $E \subset  \Vect(\mathcal{B}_1 \setminus \{u\} \cup \{v\})$ \\
			 et donc $\mathcal{B}_1 \setminus \{u\} \cup \{v\}$ engendre $E$ \\
			 donc $\mathcal{B}_1 \setminus \{u\} \cup \{v\}$ n'est pas libre\\
			 donc $v \in \Vect(\mathcal{B}_1\setminus \{u\})$ (car $\mathcal{B}_1 \setminus \{u\}$ est libre\\
			 donc $\lambda_u = 0_\mathbbm{K}$ $\lightning$\\`

			 Donc, $\lambda_u = 0_\mathbbm{K}$, docn $v \in \Vect(\mathcal{B}_1\setminus \{u\})$ \\
			 On vient de prouver que
			 \begin{align*}
			 	\mathcal{B}_2 \setminus \mathcal{B}_1 \subset \Vect(\mathcal{B}_1 \setminus \{u\})\\
			 	\mathcal{B}_1 \setminus \{u\} \subset \Vect(\mathcal{B}_1 \setminus \{u\})\\
			 \end{align*}
			 Comme $u \not\in \mathcal{B}_2$, \[
			 	\mathcal{B}_2 \subset \Vect(\mathcal{B}_1 \setminus \{u\})
			 \] donc \[
			 	E = \Vect(\mathcal{B}_2) \subset  \Vect(\mathcal{B}_1 \setminus \{u\})
			 \] donc $\mathcal{B}_1 \setminus \{u\}$ engendre $E$. Donc,  $\mathcal{B}_1 \setminus \{u\}$ est une base de $E$.\\
			 Or, $\mathcal{B}_1 \setminus \{u\} \subset  \mathcal{B}_1$, donc $\mathcal{B}_1 \setminus \{u\} = \mathcal{B}_1$
	\end{itemize}
\end{prv}

\begin{defn}
	Soit $E$ un $\mathbbm{K}$-espace vectoriel de dimension finie. Le cardinal commun à toutes les bases de $E$ est appelé \underline{dimension} de $E$ est notée $\dim(E)$ ou $\dim_\mathbbm{K}(E)$\\
	C'est donc aussi le nombre de coordonnées de n'importe quel vecteur dans n'importe quelle base.
	\index{dimension (espace vectoriel)}
\end{defn}

\begin{exm}
	\begin{enumerate}
		\item $\dim_\R(\C) = 2$ et $\dim_\C(\C) = 1$ 
		\item $\dim_\mathbbm{K}(\mathbbm{K}^{n}) = n$ 
		\item $\dim_{\mathbbm{K}}(\mathcal{M}_{n,p}(\mathbbm{K})) = np$
	\end{enumerate}
\end{exm}

\begin{crlr}
	Soit $E$ un $\mathbbm{K}$-espace vectoriel de dimension finie, $\mathcal{L}$ une famille libre de $E$, $\mathcal{G}$ une famille génératrice de $E$. On note $n = \dim(E)$
	\begin{enumerate}
		\item $\#\mathcal{G} \ge n$ et $(\#\mathcal{G} = n \implies \mathcal{G} \text{ est une base de } E$)
		\item $\#\mathcal{L} \le n$ et $(\#\mathcal{L} = n \implies \mathcal{L} \text{ est une base de } E$)
	\end{enumerate}
\end{crlr}

\begin{crlr}
	$\R^{\R}$ est de dimension infinie.
	$\forall i \in \N, e_i: x \mapsto x^i$\\
	$(e_i)_{i\in\N}$ est libre dans $\R^\R$
\end{crlr}

\begin{prop}
	Soient $E$ et $F$ deux $\mathbbm{K}$-espaces vectoriels de dimension finie. Alors $E\times F$ est de dimension finie et $\dim(E\times F) = \dim(E) + \dim(F)$
\end{prop}

\begin{prv}
	Soit $(e_1,\ldots, e_n)$ une base de $E$, $(f_1, \ldots, f_p)$ une base de $F$.
	On pose \[
		\left\{\begin{array}
			{r c l}
			u_1 &=& (e_1,0_F)\\
			u_2 &=& (e_2,0_F)\\
					&\vdots&\\
			u_n &=& (e_n,0_F)\\
			u_{n+1} &=& (0_E, f_1)\\
			u_{n+2} &=& (0_E, f_2)\\
					&\vdots&\\
			u_{n+p} &=& (0_E,f_p)\\
		\end{array}\right.
	\]
	Soit $(x,y) \in E\times F$. \[
		\begin{cases}
			\exists (x_1,\ldots,x_n)\in \mathbbm{K}^n, x = \sum_{i=1}^{n} x_ie_i
			\exists (y_1,\ldots,y_n)\in \mathbbm{K}^n, x = \sum_{j=1}^{p} y_jf_j
		\end{cases}
	\] 
	\begin{align*}
		(x,y) &= \left( \sum_{i=1}^{n} x_ie_i, \sum_{i=1}^{p} y_jf_j \right)  \\
		&= \sum_{i=1}^{n} x_i (e_i + 0_F) + \sum_{j=1}^{p} y_j (0_E, f_j) \\
		&= \sum_{i=1}^{n} x_i u_i + \sum_{j=1}^{p} y_j u_{n+j} \\
	\end{align*}
	Donc, $E\times F = \Vect(u_1, \ldots, u_{n+p})$ donc $E\times F$ est de dimension finie.\\
	Soit $(\lambda_1, \ldots, \lambda_{n+p}) \in \mathbbm{K}^{n+p}$ tel que \[
		(*): \quad \sum_{k=1}^{n+p} \lambda_ku_k = 0_{E\times F} = (0_E, 0_F)
	\]
	\begin{align*}
		(*) &\iff \sum_{k=1}^{n} \lambda_k (e_k, 0_F) + \sum_{k=n+1}^{p} \lambda_k(0_E, f_{k-n}) = (0_E, 0_F)\\
				&\iff \begin{cases}
					\sum_{k=1}^{n} \lambda_k e_k = 0_E\\
					\sum_{k=n+1}^{p} \lambda_k f_{k-n} = 0_F
				\end{cases}\\
				&\iff \begin{cases}
					\forall k \in \left\llbracket 1,n \right\rrbracket, \lambda_k = 0_\mathbbm{K} \qquad&(\text{car $(e_1,\ldots,e_n)$ est libre})\\
					\forall k \in \left\llbracket n+1,n+p \right\rrbracket, \lambda_k = 0_\mathbbm{K} \qquad&(\text{car $(f_1,\ldots,f_n)$ est libre})\\
				\end{cases}
	\end{align*}
	Donc $(u_1, \ldots, u_{n+p})$ est une base de $E\times F$. Donc, $\dim(E\times F) = n + p = \dim(E) + \dim(F)$
\end{prv}

\begin{rmk}
	[Convention]
	\[\dim\big(\{0_E\}\big) = 0\]
\end{rmk}

\begin{thm}
	Soit $E$ un $\mathbbm{K}$-espace vectoriel de dimension finie, $F$ un sous-espace vectoriel de $E$. Alors, $F$ est de dimension finie et  $\dim(F) \le \dim(E)$\\
	Si $\dim(F) = \dim(E)$, alors $F = E$
\end{thm}

\begin{prv}
	On considère \[
		A = \{k \in \N \mid \text{il existe une famille libre de $F$ à $k$ éléments}\} 
	\]
	On suppose $F \neq \{0_E\}$.
	\begin{itemize}
		\item Soit $u \in F\setminus \{0_E\}$. $(u)$ est libre donc $1 \in A$ et donc $A \neq \O$
		\item Soit $\mathcal{L}$ une famille libre de $F$. Alors, $\mathcal{L}$ est une famille libre de $E$ \\
			donc $\#\mathcal{L} \le \dim(E)$\\
			Donc $A$ est majorée par $\dim(E)$ \\
			On en déduit que $A$ a un plus grand élément $p$.
		\item Soit $\mathcal{L}$ une famille libre de $F$ avec $p$ éléments.\\
			Si $\mathcal{L}$ n'engendre pas $F$, alors il existe $u\in F$ tel que $u\not\in \Vect(\mathcal{L})$ et donc $\mathcal{L} \cup \{u\}$ est une famille libre de $F$, donc $p+1 \in A$ en contradiction avec la maximalité de $p$.\\
			Donc $\mathcal{L}$ est une base de $F$ donc $F$ est de dimension finie et $\dim(F) = p \le \dim(E)$\\
	\end{itemize}

	Soit $\mathcal{B}$ une base de $F$. Alors, $\mathcal{B}$ est aussi une famille de libre de de $E$. Donc $\#\mathcal{B} \le \dim(E)$ donc $\dim(F) = \dim(E)$ \\
	Si $\dim(F) = \dim(E)$, alors $\mathcal{B}$ est une base de $E$, et donc $F = \Vect(\mathcal{B}) = E$
\end{prv}

\begin{prop}
	[Formule de Grassmann]
	Soit $E$ un $\mathbbm{K}$-espace vectoriel de dimension finie, $F$ et $G$ deux sous-espace vectoriels de $E$. Alors, \[
		\dim(F+G) = \dim(F) + \dim(G) - \dim(F\cap G)
	\] 
\end{prop}

\begin{prv}
	Soit $(e_1, \ldots, e_p)$ une base de $F\cap G$. $(e_1,\ldots,e_p)$ est une famille libre de $F$.\\
	On complète $(e_1, \ldots, e_p)$ en une base $(e_1, \ldots, e_p, u_1, \ldots, u_q)$ de $F$.\\
	De même, on complète $(e_1, \ldots, e_p)$ en une base $(e_1, \ldots, e_p, v_1, \ldots, v_r)$ de $G$.\\
	On pose  $\mathcal{B} = (e_1, \ldots, e_p, u_1, \ldots, u_q, v_1, \ldots, v_r)$. Montrons que $\mathcal{B}$ est une base de $F+G$
	\begin{itemize}
		\item Soit $u \in F+G$ \\
			On pose $u = v+w$ avec $\begin{cases}
				v\in F\\
				w \in G
			\end{cases}$.\\
			On pose $v = \sum_{i=1}^p \lambda_i e_i + \sum_{i=1}^q \mu_i u_i$ avec $(\lambda_1, \ldots, \lambda_p, \mu_1, \ldots, \lambda_q) \in \mathbbm{K}^{p+q}$\\
			On pose aussi $w = \sum_{i = 1}^p \lambda'_ie_i + \sum_{j=1}^r \nu_j v_j$ avec $(\lambda_1',\ldots,\lambda_p', \nu_1, \ldots, \nu_r) \in \mathbbm{K}^{p+r}$\\
			D'où, \[
				u = \sum_{i=1}^p (\lambda_i + \lambda'_i)e_i + \sum_{j=1}^q \mu_j u_j + \sum_{k=1}^r \nu_k v_k \in \Vect(\mathcal{B})
			\]
		\item Soient $(\lambda_1, \ldots, \lambda_p, \mu_1, \ldots, \mu_q, \nu_1, \ldots, \nu_r) \in \mathbbm{K}^{p+q+r}$.\\
			On suppose \[
				(*)\quad \sum_{i=1}^{p}\lambda_ie_i + \sum_{j=1}^q\mu_ju_j + \sum_{k=1}^r \nu_k v_k = 0_E
			\] 
			D'où, \[
				\underbrace{\sum_{i=1}^p\lambda_i e_i + \sum_{j=1}^q \mu_ju_j}_{\in F} = \underbrace{-\sum_{k=1}^r\nu_jv_k}_{\in G}
			\] 
			Donc, \[
				f = \sum_{i=1}^p \lambda_i e_i + \sum_{j=1}^q \mu_j u_j \in F\cap G
			\] Comme $(e_1, \ldots, e_p)$ est une base de $F\cap G$, $\exists ! (\lambda_1', \ldots, \lambda_p') \in \mathbbm{K}^p$ tel que \[
				f = \sum_{i=1}^p \lambda'_i e_i = \sum_{i=1}^p \lambda'_i e_i + \sum_{j=1}^q 0_\mathbbm{K}u_j
			\] Comme $(e_1, \ldots, e_p, u_1, \ldots, u_q)$ est une base de $F$, \[
				\forall k \in \left\llbracket 1, q \right\rrbracket, \mu_j = 0_\mathbbm{K}
			\] De même, \[
				\forall k \in \left\llbracket 1,r \right\rrbracket , \nu_k = 0_\mathbbm{K}
			\] On remplace dans $(*)$ pour trouver \[
				\sum_{i=1}^p \lambda_ie_i = 0_E
			\] Comme $(e_1, \ldots, e_p)$ est libre, \[
				\forall i \in \left\llbracket 1,p \right\rrbracket, \lambda_i = 0_\mathbbm{K}
			\] Donc $\mathcal{B}$ est libre.\\
			Donc, 
			\begin{align*}
				\dim(F+G) &=  p +q + r \\
				&= (p+q)+ (p+r) - p \\
				&= \dim(F) + \dim(G) - \dim(F\cap G) \\
			\end{align*}
	\end{itemize}
\end{prv}

\begin{crlr}
	Avec les hypothèse précédentes, \[
		E = F \oplus G \iff \begin{cases}
			F \cap  G = \{0_E\} \\
			\dim(E) = \dim(F) + \dim(G)
		\end{cases}
	\] 
\end{crlr}

\begin{prv}
	\begin{itemize}
		\item[``$\implies$''] On suppose $E = F \oplus G$ \\
			Comme la somme est directe, $F \cap G = \{0_E\}$ 
			\begin{align*}
				\dim(E) &= \dim(F)\\
				&= \dim(F) + \dim(G) - \dim(F\cap G)\\
				&= \dim(F) + \dim(G)\\
			\end{align*}
		\item[``$\impliedby$''] On suppose $F\cap G = \{0_E\}$ et $\dim(E) = \dim(F) + \dim(G)$.\\
			On sait déjà que $F+G = F \oplus G$\\
			 \begin{align*}
				\dim(F+G) = \dim(F) + \dim(G) - \dim(F \cap G) = \dim(E)
			\end{align*}
			Donc $F + G = E$
	\end{itemize}
\end{prv}

\begin{prop}
	Soit $F$ un $\mathbbm{K}$-espace vectoriel de dimension finie $n$. Soit $\mathcal{B} = (e_1, \ldots, e_n)$ une base de $F$. L'application
	\begin{align*}
		f: \mathbbm{K}^n &\longrightarrow F \\
		(\lambda_1, \ldots, \lambda_n) &\longmapsto \sum_{i=1}^n \lambda_i e_i
	\end{align*} est bijective.\\
	Si $\mathbbm{K}$ est infini, $\mathbbm{K}^n$ aussi et donc $F$ aussi.\\
	Si $\#\mathbbm{K} = p \in \N_*$,
	\begin{align*}
		\#&\mathbbm{K}^n = p^n\\
		&\vrt=\\
		\#&F
	\end{align*}
\end{prop}


	}

	{
		\chap[06]{Équations différentielles linéaires}
		\renewcommand{\cwd}{../chap06}
		\begin{defn}
	Soit $E$ un $\mathbbm{K}$-espace vectoriel. On dit que $E$ est de \underline{dimension finie} si $E$ a au moins une famille génératrice finie. On dit que $E$ est de \underline{dimension infinie} sinon.
	\index{dimension finie (espace vectoriel)}
	\index{dimension infinie (espace vectoriel)}
\end{defn}

\begin{thm}
	[Théorème de la base extraite]
	Soit $E$ un $\mathbbm{K}$-espace vectoriel non nul de dimension finie. Soit $\mathcal{G}$ une famille génératrice finie de $E$. Alors, il existe une base $\mathcal{B}$ de $\mathcal{E}$ telle que $\mathcal{B} \subset \mathcal{G}$.
\end{thm}

\begin{prv}
	[par récurrence sur $\#G = \Card(G)$]
	\begin{itemize}
		\item Soit $E$ un $\mathbbm{K}$-espace vectoriel non nul engendré par $\mathcal{G} = (u)$.\\
			Si $u = 0_E$, alors $E = \{0_E\}$: une contradiction $\lightning$ \\
			Donc $u \neq 0_E$ donc $(u)$ est libre. En effet, \[
				\forall \lambda \in \mathbbm{K}, \lambda u = 0_E \implies \lambda = 0_\mathbbm{K}
			\] Donc $\mathcal{G}$ est une base de $E$.\\
		\item Soit $n \in \N_*$. Soit $E$ un $\mathbbm{K}$-espace vectoriel. On suppose que si $E$ a une famille génératrice constituée de $n$ vecteurs, alors on peut extraire de cette famille une base de $E$.\\
			Soit $\mathcal{G}$ une famille génératrice de $E$ avec $n+1$ vecteurs.\\
			Si $\mathcal{G}$ est libre, alors $\mathcal{G}$ est une base de $E$. \\
			Si $\mathcal{G}$ n'est pas libre, alors il existe $u \in \mathcal{G}$ tel que $u \in \Vect(\mathcal{G}\setminus \{u\})$ \\
			Donc $\mathcal{G}\setminus \{u\}$ engendre $E$. Or, $\mathcal{G}\setminus \{u\}$ possède $n$ vecteurs. D'après l'hypothèse de récurrence, il existe une base $\mathcal{B}$ de $E$ telle que \[
				\mathcal{B} \subset \mathcal{G} \setminus \{u\} \subset \mathcal{G}
			\] 
	\end{itemize}
\end{prv}

\begin{crlr}
	Tout espace de dimension finie a une base.
	\qed
\end{crlr}

\begin{thm}
	[Théorème de la base incomplète]
	Soit $E$ un $\mathbbm{K}$-espace vectoriel de dimension finie, $\mathcal{G}$ une famille génératrice finie de $E$. $\mathcal{L}$ une famille libre de $E$. Alors, il existe une base $\mathcal{B}$ de $E$ telle que \[
		\mathcal{L} \subset \mathcal{B} \text{ et } \mathcal{B}\setminus \mathcal{L} \subset \mathcal{G}
	\] 
\end{thm}

\begin{prv}
	[par récurrence sur $\#(\mathcal{G}\setminus\mathcal{L})$]
	\begin{itemize}
		\item Avec les notations précédentes, on suppose que $\mathcal{G}\setminus\mathcal{L} \neq \O$ \[
				\forall u \in \mathcal{G}, u \in \mathcal{L}
			\] Donc $\mathcal{G} \subset \mathcal{L}$ donc $\mathcal{L}$ est génératrice donc $\mathcal{L}$ est une base de $E$. On pose $\mathcal{B} = \mathcal{L}$ et alors \[
				\mathcal{L} \subset  \mathcal{B} \text{ et } \mathcal{B}\setminus\mathcal{L} = \O \subset  \mathcal{G}
			\] 
		\item Soit $n \in \N$. On suppose que si $\mathcal{G}$ est génératrice et $\mathcal{L}$ libre avec $\#(\mathcal{G}\setminus\mathcal{L}) = n$ alors il existe une base $\mathcal{B}$ de $E$ telle que \[
			\mathcal{L}\subset \mathcal{B} \text{ et } \mathcal{B}\setminus\mathcal{L}\subset \mathcal{G}
		\] Soient à présent $\mathcal{G}$ une famille génératrice de $E$ et $\mathcal{L}$ une famille libre de $E$ telles que $\#(\mathcal{G}\setminus\mathcal{L}) = n+1 > 0$\\
		Si $\mathcal{L}$ engendre $E$, alors $\mathcal{L}$ est une base de $E$. On pose $\mathcal{B} = \mathcal{L}$ et on a bien \[
			\mathcal{L} \subset  \mathcal{B} \text{ et } \mathcal{B} \setminus \mathcal{L} = \O \subset  \mathcal{G}
		\] On suppose que $\mathcal{L}$ n'engendre pas $E$. Il existe $u \in \mathcal{G}$ tel que $u \not\in \Vec(\mathcal{L})$ (car sinon, $\mathcal{G} \subset \Vect(\mathcal{L})$ et donc $\underbrace{\Vect(\mathcal{G})}_{= E} \subset  \underbrace{\Vect(\mathcal{L})}_{ \subset E}$\\
		Donc $\mathcal{L} \cup \{u\} $ est libre. On pose $\mathcal{L}' = \mathcal{L} \cup \{u\} $ \[
			\mathcal{G}\setminus \mathcal{L}' = \mathcal{G}\setminus (\mathcal{L} \cup \{u\}) = (\mathcal{G}\setminus\mathcal{L})\setminus \{u\} 
		\] donc $\#(\mathcal{G}\setminus\mathcal{L}') = n+1 -1 = n$\\
		D'après l'hypothèse de récurrence, il existe $\mathcal{B}$ une base de $E$ telle que \[
			\mathcal{L} \subset  \mathcal{L}' \subset \mathcal{B} \text{ et } \mathcal{B}\setminus \mathcal{L}' \subset \mathcal{G}
		\] \[
			\mathcal{B} \setminus \mathcal{L} = \underbrace{\mathcal{B}\setminus\mathcal{L}'}_{\subset \mathcal{G}} \cup \underbrace{\{u\}}_{\subset \mathcal{G} \text{ car } u \in \mathcal{G}}
		\] On a $\mathcal{B}\setminus\mathcal{L}\subset \mathcal{G}$
	\end{itemize}
\end{prv}

\begin{thm}
	Soit $E$ un $\mathbbm{K}$-espace vectoriel de dimension finie. Toutes les bases de $E$ ont le même cardinal.
\end{thm}

\begin{prv}
	Soit $\mathcal{G}$ une famille génératrice finie de $E$ et $\mathcal{B} \subset  \mathcal{G}$ une base de $E$. On note $n = \#\mathcal{B}$ \\
	Soit $\mathcal{B}'$ une base de $E$. On pose $p = n - \#(\mathcal{B} \cap  \mathcal{B}')$. Montrons par récurrence sur  $p$ que $\#\mathcal{B} = \#\mathcal{B}'$ 
	\begin{itemize}
		\item On suppose que $p = 0$. Alors, $\#(\mathcal{B} \cap \mathcal{B}') = n$ \\
			Or, $\mathcal{B}' \cap \mathcal{B} \subset \mathcal{B}$ donc $\mathcal{B} \cap \mathcal{B}' = \mathcal{B}$ donc $\mathcal{B} \subset  \mathcal{B}'$ et donc $\mathcal{B} = \mathcal{B}'$ 
		\item Soit $p \in \N$. On suppose que si $\mathcal{B}'$ est une base de $E$ telle que $n - \#(\mathcal{B} \cap \mathcal{B}') = p$, alors $\#\mathcal{B}' = n$ \\
			Aoit $\mathcal{B}'$ une base de $E$ telle que $n - \#(\mathcal{B}\cap \mathcal{B}') = p+1 > 0$ \\
			Donc $\mathcal{B} \cap \mathcal{B}' \neq \mathcal{B}$. Soit $u \in \mathcal{B}' \setminus \mathcal{B}$. D'après le lemme d'échange, il existe $v \in \mathcal{B}\setminus \mathcal{B}'$ tel que $\mathcal{B}' \setminus \{u\} \cup \{v\}$ est une base de $E$. On pose $\mathcal{B}'' = \mathcal{B}' \setminus \{u\} \cup \{v\}$ 
			\begin{align*}
				\mathcal{B}'' \cap \mathcal{B} &= \left( (\mathcal{B}' \setminus \{u\})  \cap \mathcal{B} \right) \cup \{v\} \\
				&= (\mathcal{B}' \cap \mathcal{B}) \cup \{v\} \\
			\end{align*}
			donc,
			\begin{align*}
				n - \#(\mathcal{B}'' \cap \mathcal{B}) &= n - (\#(\mathcal{B}' \cap \mathcal{B}) + 1) \\
				&= p+1- 1 \\
				&= p \\
			\end{align*}
			D'après l'hypothèse de récurrence, \[
				\#\mathcal{B}'' = n
			\] Or, $\#\mathcal{B}'' = \#\mathcal{B}'$
	\end{itemize}
\end{prv}

\begin{lem}
	Soient $\mathcal{B}$ et $\mathcal{B}'$ deux bases de $E$ telles que $\mathcal{B}\subset \mathcal{B}'$. Alors, $\mathcal{B} = \mathcal{B}'$.
\end{lem}

\begin{prv}
	On suppose $\mathcal{B}' \neq \mathcal{B}$. Soit $u \in \mathcal{B}' \setminus \mathcal{B}$
	$u \in E = \Vect(\mathcal{B})$ donc $\mathcal{B} \cup \{u\}$ n'est pas libre.
	Donc $\mathcal{B}\cup \{u\} \subset \mathcal{B}'$ et $\mathcal{B}'$ est libre donc $\mathcal{B}\cup \{u\}$ est libre: une contradiction $\lightning$
\end{prv}

\begin{lem}
	[Lemme d'échange] Soient $\mathcal{B}_1$ et $\mathcal{B}_2$ deux bases de $E$ et $u \in \mathcal{B}_1 \setminus \mathcal{B}_2$. Alors, il existe $v \in \mathcal{B}_2$ tel que $(\mathcal{B}_1 \setminus \{u\}) \cup \{v\}$ soit une base de $E$.
\end{lem}

\begin{prv}
	[1${}^\text{nde}$ méthode]
	On suppose que pout tout $v \in \mathcal{B}_2$, $(\mathcal{B}_1\setminus \{u\}) \cup \{v\}$ n'est pas une base de $E$
	Soit $v \in \mathcal{B}_2$.
	\begin{itemize}
		\item Supposons $(\mathcal{B}_1\setminus \{u\})\cup \{v\}$ non libre. $\mathcal{B}_1 \setminus \{u\}$ est libre. Donc $v \in \Vect(\mathcal{B}_1 \setminus \{u\})$
		\item Supposons $(\mathcal{B}_1\setminus \{u\}) \cup \{v\}$ non génératrice.
			Comme $\mathcal{B}_1$ engendre $E$, $u \not\in \Vect(\mathcal{B}_1\setminus \{v\})$.
			On suppose que $\mathcal{B}_1 \neq \mathcal{B}_2$.
			$\forall v \in \mathcal{B}_2 \setminus \mathcal{B}_1, \Vect(\mathcal{B}_1 \setminus \{v\}) = \Vect(\mathcal{B}_1) = E \ni u$ 
			donc, $(\mathcal{B}_1\setminus \{u\}) \cup \{v\}$ engendre $E$ et donc \[
				v \in \Vect(\mathcal{B}_1 \setminus \{u\})
			\] On a aussi \[
				\forall v \in \mathcal{B}_1 \setminus \{u\}, v \in \Vect(\mathcal{B}_1\setminus \{u\})
			\] Comme $u \not\in \mathcal{B}_2$, on a \[
				\forall v \in \mathcal{B}_2, v \in \Vect(\mathcal{B}_1\setminus \{u\})
			\] docn \[
				E = \Vect(\mathcal{B}_2) \subset \Vect(\mathcal{B}_1\setminus \{u\})
			\] donc $\mathcal{B}_1\setminus \{u\}$ engendre $E$ donc $\mathcal{B}_1\setminus \{u\}$ est une base de $E$. Or, $\mathcal{B}_1 \setminus \{u\}  \subset  \mathcal{B}_1$, donc $\mathcal{B}_1\setminus \{u\} = \mathcal{B}_1$
	\end{itemize}
\end{prv}

\begin{prv}
	[2${}^\text{nde}$ méthode]
	On suppose que pout tout $v \in \mathcal{B}_2$, $(\mathcal{B}_1\setminus \{u\}) \cup \{v\}$ n'est pas une base de $E$
	\begin{itemize}
		\item Comme $u \in \mathcal{B}_1 \setminus \mathcal{B}_2$, nécéssairement $\mathcal{B}_1 \neq \mathcal{B}_2$ donc $\mathcal{B}_2 \not\subset \mathcal{B}_1$, donc $\mathcal{B}_2\setminus\mathcal{B}_1 \neq \O$ 
		\item Soit $v \in \mathcal{B}_2\setminus\mathcal{B}_1$. Il existe $(\lambda_w)_{w\in\mathcal{B}_1}$ une famille de scalaires presque nulle telle que \[
				v = \sum_{w \in \mathcal{B}_1} \lambda_w w - \lambda_u u + + \sum_{w \in \mathcal{B}_1\setminus \{u\}}\lambda_w w
			\]
			Si $\lambda_u \neq 0_E$, alors
			\begin{align*}
				u &= \lambda_u^{-1}\left( v - \sum_{w \in \mathcal{B}_1 \setminus \{u\}} \lambda_w w \right)\\
					&\in \Vect(\mathcal{B}_1\setminus \{u\} \cup v)
			\end{align*}
			 donc $\mathcal{B}_1 \subset \Vect(\mathcal{B}_1\setminus \{u\} \cup \{v\})$\\
			 et donc $E \subset  \Vect(\mathcal{B}_1 \setminus \{u\} \cup \{v\})$ \\
			 et donc $\mathcal{B}_1 \setminus \{u\} \cup \{v\}$ engendre $E$ \\
			 donc $\mathcal{B}_1 \setminus \{u\} \cup \{v\}$ n'est pas libre\\
			 donc $v \in \Vect(\mathcal{B}_1\setminus \{u\})$ (car $\mathcal{B}_1 \setminus \{u\}$ est libre\\
			 donc $\lambda_u = 0_\mathbbm{K}$ $\lightning$\\`

			 Donc, $\lambda_u = 0_\mathbbm{K}$, docn $v \in \Vect(\mathcal{B}_1\setminus \{u\})$ \\
			 On vient de prouver que
			 \begin{align*}
			 	\mathcal{B}_2 \setminus \mathcal{B}_1 \subset \Vect(\mathcal{B}_1 \setminus \{u\})\\
			 	\mathcal{B}_1 \setminus \{u\} \subset \Vect(\mathcal{B}_1 \setminus \{u\})\\
			 \end{align*}
			 Comme $u \not\in \mathcal{B}_2$, \[
			 	\mathcal{B}_2 \subset \Vect(\mathcal{B}_1 \setminus \{u\})
			 \] donc \[
			 	E = \Vect(\mathcal{B}_2) \subset  \Vect(\mathcal{B}_1 \setminus \{u\})
			 \] donc $\mathcal{B}_1 \setminus \{u\}$ engendre $E$. Donc,  $\mathcal{B}_1 \setminus \{u\}$ est une base de $E$.\\
			 Or, $\mathcal{B}_1 \setminus \{u\} \subset  \mathcal{B}_1$, donc $\mathcal{B}_1 \setminus \{u\} = \mathcal{B}_1$
	\end{itemize}
\end{prv}

\begin{defn}
	Soit $E$ un $\mathbbm{K}$-espace vectoriel de dimension finie. Le cardinal commun à toutes les bases de $E$ est appelé \underline{dimension} de $E$ est notée $\dim(E)$ ou $\dim_\mathbbm{K}(E)$\\
	C'est donc aussi le nombre de coordonnées de n'importe quel vecteur dans n'importe quelle base.
	\index{dimension (espace vectoriel)}
\end{defn}

\begin{exm}
	\begin{enumerate}
		\item $\dim_\R(\C) = 2$ et $\dim_\C(\C) = 1$ 
		\item $\dim_\mathbbm{K}(\mathbbm{K}^{n}) = n$ 
		\item $\dim_{\mathbbm{K}}(\mathcal{M}_{n,p}(\mathbbm{K})) = np$
	\end{enumerate}
\end{exm}

\begin{crlr}
	Soit $E$ un $\mathbbm{K}$-espace vectoriel de dimension finie, $\mathcal{L}$ une famille libre de $E$, $\mathcal{G}$ une famille génératrice de $E$. On note $n = \dim(E)$
	\begin{enumerate}
		\item $\#\mathcal{G} \ge n$ et $(\#\mathcal{G} = n \implies \mathcal{G} \text{ est une base de } E$)
		\item $\#\mathcal{L} \le n$ et $(\#\mathcal{L} = n \implies \mathcal{L} \text{ est une base de } E$)
	\end{enumerate}
\end{crlr}

\begin{crlr}
	$\R^{\R}$ est de dimension infinie.
	$\forall i \in \N, e_i: x \mapsto x^i$\\
	$(e_i)_{i\in\N}$ est libre dans $\R^\R$
\end{crlr}

\begin{prop}
	Soient $E$ et $F$ deux $\mathbbm{K}$-espaces vectoriels de dimension finie. Alors $E\times F$ est de dimension finie et $\dim(E\times F) = \dim(E) + \dim(F)$
\end{prop}

\begin{prv}
	Soit $(e_1,\ldots, e_n)$ une base de $E$, $(f_1, \ldots, f_p)$ une base de $F$.
	On pose \[
		\left\{\begin{array}
			{r c l}
			u_1 &=& (e_1,0_F)\\
			u_2 &=& (e_2,0_F)\\
					&\vdots&\\
			u_n &=& (e_n,0_F)\\
			u_{n+1} &=& (0_E, f_1)\\
			u_{n+2} &=& (0_E, f_2)\\
					&\vdots&\\
			u_{n+p} &=& (0_E,f_p)\\
		\end{array}\right.
	\]
	Soit $(x,y) \in E\times F$. \[
		\begin{cases}
			\exists (x_1,\ldots,x_n)\in \mathbbm{K}^n, x = \sum_{i=1}^{n} x_ie_i
			\exists (y_1,\ldots,y_n)\in \mathbbm{K}^n, x = \sum_{j=1}^{p} y_jf_j
		\end{cases}
	\] 
	\begin{align*}
		(x,y) &= \left( \sum_{i=1}^{n} x_ie_i, \sum_{i=1}^{p} y_jf_j \right)  \\
		&= \sum_{i=1}^{n} x_i (e_i + 0_F) + \sum_{j=1}^{p} y_j (0_E, f_j) \\
		&= \sum_{i=1}^{n} x_i u_i + \sum_{j=1}^{p} y_j u_{n+j} \\
	\end{align*}
	Donc, $E\times F = \Vect(u_1, \ldots, u_{n+p})$ donc $E\times F$ est de dimension finie.\\
	Soit $(\lambda_1, \ldots, \lambda_{n+p}) \in \mathbbm{K}^{n+p}$ tel que \[
		(*): \quad \sum_{k=1}^{n+p} \lambda_ku_k = 0_{E\times F} = (0_E, 0_F)
	\]
	\begin{align*}
		(*) &\iff \sum_{k=1}^{n} \lambda_k (e_k, 0_F) + \sum_{k=n+1}^{p} \lambda_k(0_E, f_{k-n}) = (0_E, 0_F)\\
				&\iff \begin{cases}
					\sum_{k=1}^{n} \lambda_k e_k = 0_E\\
					\sum_{k=n+1}^{p} \lambda_k f_{k-n} = 0_F
				\end{cases}\\
				&\iff \begin{cases}
					\forall k \in \left\llbracket 1,n \right\rrbracket, \lambda_k = 0_\mathbbm{K} \qquad&(\text{car $(e_1,\ldots,e_n)$ est libre})\\
					\forall k \in \left\llbracket n+1,n+p \right\rrbracket, \lambda_k = 0_\mathbbm{K} \qquad&(\text{car $(f_1,\ldots,f_n)$ est libre})\\
				\end{cases}
	\end{align*}
	Donc $(u_1, \ldots, u_{n+p})$ est une base de $E\times F$. Donc, $\dim(E\times F) = n + p = \dim(E) + \dim(F)$
\end{prv}

\begin{rmk}
	[Convention]
	\[\dim\big(\{0_E\}\big) = 0\]
\end{rmk}

\begin{thm}
	Soit $E$ un $\mathbbm{K}$-espace vectoriel de dimension finie, $F$ un sous-espace vectoriel de $E$. Alors, $F$ est de dimension finie et  $\dim(F) \le \dim(E)$\\
	Si $\dim(F) = \dim(E)$, alors $F = E$
\end{thm}

\begin{prv}
	On considère \[
		A = \{k \in \N \mid \text{il existe une famille libre de $F$ à $k$ éléments}\} 
	\]
	On suppose $F \neq \{0_E\}$.
	\begin{itemize}
		\item Soit $u \in F\setminus \{0_E\}$. $(u)$ est libre donc $1 \in A$ et donc $A \neq \O$
		\item Soit $\mathcal{L}$ une famille libre de $F$. Alors, $\mathcal{L}$ est une famille libre de $E$ \\
			donc $\#\mathcal{L} \le \dim(E)$\\
			Donc $A$ est majorée par $\dim(E)$ \\
			On en déduit que $A$ a un plus grand élément $p$.
		\item Soit $\mathcal{L}$ une famille libre de $F$ avec $p$ éléments.\\
			Si $\mathcal{L}$ n'engendre pas $F$, alors il existe $u\in F$ tel que $u\not\in \Vect(\mathcal{L})$ et donc $\mathcal{L} \cup \{u\}$ est une famille libre de $F$, donc $p+1 \in A$ en contradiction avec la maximalité de $p$.\\
			Donc $\mathcal{L}$ est une base de $F$ donc $F$ est de dimension finie et $\dim(F) = p \le \dim(E)$\\
	\end{itemize}

	Soit $\mathcal{B}$ une base de $F$. Alors, $\mathcal{B}$ est aussi une famille de libre de de $E$. Donc $\#\mathcal{B} \le \dim(E)$ donc $\dim(F) = \dim(E)$ \\
	Si $\dim(F) = \dim(E)$, alors $\mathcal{B}$ est une base de $E$, et donc $F = \Vect(\mathcal{B}) = E$
\end{prv}

\begin{prop}
	[Formule de Grassmann]
	Soit $E$ un $\mathbbm{K}$-espace vectoriel de dimension finie, $F$ et $G$ deux sous-espace vectoriels de $E$. Alors, \[
		\dim(F+G) = \dim(F) + \dim(G) - \dim(F\cap G)
	\] 
\end{prop}

\begin{prv}
	Soit $(e_1, \ldots, e_p)$ une base de $F\cap G$. $(e_1,\ldots,e_p)$ est une famille libre de $F$.\\
	On complète $(e_1, \ldots, e_p)$ en une base $(e_1, \ldots, e_p, u_1, \ldots, u_q)$ de $F$.\\
	De même, on complète $(e_1, \ldots, e_p)$ en une base $(e_1, \ldots, e_p, v_1, \ldots, v_r)$ de $G$.\\
	On pose  $\mathcal{B} = (e_1, \ldots, e_p, u_1, \ldots, u_q, v_1, \ldots, v_r)$. Montrons que $\mathcal{B}$ est une base de $F+G$
	\begin{itemize}
		\item Soit $u \in F+G$ \\
			On pose $u = v+w$ avec $\begin{cases}
				v\in F\\
				w \in G
			\end{cases}$.\\
			On pose $v = \sum_{i=1}^p \lambda_i e_i + \sum_{i=1}^q \mu_i u_i$ avec $(\lambda_1, \ldots, \lambda_p, \mu_1, \ldots, \lambda_q) \in \mathbbm{K}^{p+q}$\\
			On pose aussi $w = \sum_{i = 1}^p \lambda'_ie_i + \sum_{j=1}^r \nu_j v_j$ avec $(\lambda_1',\ldots,\lambda_p', \nu_1, \ldots, \nu_r) \in \mathbbm{K}^{p+r}$\\
			D'où, \[
				u = \sum_{i=1}^p (\lambda_i + \lambda'_i)e_i + \sum_{j=1}^q \mu_j u_j + \sum_{k=1}^r \nu_k v_k \in \Vect(\mathcal{B})
			\]
		\item Soient $(\lambda_1, \ldots, \lambda_p, \mu_1, \ldots, \mu_q, \nu_1, \ldots, \nu_r) \in \mathbbm{K}^{p+q+r}$.\\
			On suppose \[
				(*)\quad \sum_{i=1}^{p}\lambda_ie_i + \sum_{j=1}^q\mu_ju_j + \sum_{k=1}^r \nu_k v_k = 0_E
			\] 
			D'où, \[
				\underbrace{\sum_{i=1}^p\lambda_i e_i + \sum_{j=1}^q \mu_ju_j}_{\in F} = \underbrace{-\sum_{k=1}^r\nu_jv_k}_{\in G}
			\] 
			Donc, \[
				f = \sum_{i=1}^p \lambda_i e_i + \sum_{j=1}^q \mu_j u_j \in F\cap G
			\] Comme $(e_1, \ldots, e_p)$ est une base de $F\cap G$, $\exists ! (\lambda_1', \ldots, \lambda_p') \in \mathbbm{K}^p$ tel que \[
				f = \sum_{i=1}^p \lambda'_i e_i = \sum_{i=1}^p \lambda'_i e_i + \sum_{j=1}^q 0_\mathbbm{K}u_j
			\] Comme $(e_1, \ldots, e_p, u_1, \ldots, u_q)$ est une base de $F$, \[
				\forall k \in \left\llbracket 1, q \right\rrbracket, \mu_j = 0_\mathbbm{K}
			\] De même, \[
				\forall k \in \left\llbracket 1,r \right\rrbracket , \nu_k = 0_\mathbbm{K}
			\] On remplace dans $(*)$ pour trouver \[
				\sum_{i=1}^p \lambda_ie_i = 0_E
			\] Comme $(e_1, \ldots, e_p)$ est libre, \[
				\forall i \in \left\llbracket 1,p \right\rrbracket, \lambda_i = 0_\mathbbm{K}
			\] Donc $\mathcal{B}$ est libre.\\
			Donc, 
			\begin{align*}
				\dim(F+G) &=  p +q + r \\
				&= (p+q)+ (p+r) - p \\
				&= \dim(F) + \dim(G) - \dim(F\cap G) \\
			\end{align*}
	\end{itemize}
\end{prv}

\begin{crlr}
	Avec les hypothèse précédentes, \[
		E = F \oplus G \iff \begin{cases}
			F \cap  G = \{0_E\} \\
			\dim(E) = \dim(F) + \dim(G)
		\end{cases}
	\] 
\end{crlr}

\begin{prv}
	\begin{itemize}
		\item[``$\implies$''] On suppose $E = F \oplus G$ \\
			Comme la somme est directe, $F \cap G = \{0_E\}$ 
			\begin{align*}
				\dim(E) &= \dim(F)\\
				&= \dim(F) + \dim(G) - \dim(F\cap G)\\
				&= \dim(F) + \dim(G)\\
			\end{align*}
		\item[``$\impliedby$''] On suppose $F\cap G = \{0_E\}$ et $\dim(E) = \dim(F) + \dim(G)$.\\
			On sait déjà que $F+G = F \oplus G$\\
			 \begin{align*}
				\dim(F+G) = \dim(F) + \dim(G) - \dim(F \cap G) = \dim(E)
			\end{align*}
			Donc $F + G = E$
	\end{itemize}
\end{prv}

\begin{prop}
	Soit $F$ un $\mathbbm{K}$-espace vectoriel de dimension finie $n$. Soit $\mathcal{B} = (e_1, \ldots, e_n)$ une base de $F$. L'application
	\begin{align*}
		f: \mathbbm{K}^n &\longrightarrow F \\
		(\lambda_1, \ldots, \lambda_n) &\longmapsto \sum_{i=1}^n \lambda_i e_i
	\end{align*} est bijective.\\
	Si $\mathbbm{K}$ est infini, $\mathbbm{K}^n$ aussi et donc $F$ aussi.\\
	Si $\#\mathbbm{K} = p \in \N_*$,
	\begin{align*}
		\#&\mathbbm{K}^n = p^n\\
		&\vrt=\\
		\#&F
	\end{align*}
\end{prop}


		\part{Dérivation}

\underline{Motivation}:

{
\begin{wrapfigure}{l}{3cm}
	\centering
	\begin{asy}
		import three;

		size(3cm);
		settings.render=0;
		settings.prc=false;
		currentprojection = obliqueZ;

		draw(unitbox);
		draw(shift(1.1Z + 0.05X) * (O -- X), Arrows3(TeXHead2));
		draw(shift(1.1Z + 0.05Y) * (O -- Y), Arrows3(TeXHead2));
		draw(shift(1.1X + 0.05Z) * (O -- Z), Arrows3(TeXHead2));

		label("$x$", (X/2) + (1.1Z + 0.05X), align=S);
		label("$y$", (Y/2) + (1.1Z + 0.05Y), align=W);
		label("$z$", (Z/2) + X, align=SE);
	\end{asy}
\end{wrapfigure}

\begin{align*}
	&S(x,y,z) = 2(xy + xz + yz)\\
	&V(x,y,z) = xyz
\end{align*}

On cherche à minimiser $S$ avec la contrainte $V = 1$.

Soit $f : \begin{array}{rcl}
	\left( \R_*^+ \right)^2 &\longrightarrow& \R \\
	(x,y) &\longmapsto& S\left( x,y,\frac{1}{xy} \right) = 2\left( xy + \frac{1}{y} + \frac{1}{x} \right).
\end{array}$

On cherche $(a,b) \in \left( \R^+_* \right)^2$ tel que \[
	\forall (x,y) \in (\R^+_*), f(x,y) \ge f(a,b).
\]
}

\begin{defn}
	Soit $f: U \to \R$ où $U$ est un ouvert de $\R^2$. Soit $(a,b) \in U$.
	\vspace{2mm}

	Si $\lim_{x \to a} \frac{f(x,b) - f(a,b)}{x - a} \in \R$, alors on dit que $f$ a une dérivée partielle suivant $x$ en $(a,b)$ et cette limite est notée \[
		\partial f_1(a,b) = \frac{\partial f}{\partial x}(a,b).
	\]

	Si $\lim_{y \to b} \frac{f(a,y) - f(a,b)}{y - b} \in \R$, alors on dit que $f$ a une dérivée partielle suivant $y$ et la limite est notée \[
		\partial f_2(a,b) = \frac{\partial f}{\partial y}(a,b).
	\]
\end{defn}

\begin{exm}
	\begin{enumerate}
		\item $f: (x,y) \mapsto xy + x - y$.

			\begin{align*}
				&\frac{\partial f}{\partial x} : (x,y) \mapsto y + 1,\\
				&\frac{\partial f}{\partial y} : (x,y) \mapsto x - 1.
			\end{align*}

		\item $f: (x,y) \mapsto xy + \frac{1}{y}+ \frac{1}{x}$.

			\begin{align*}
				&\frac{\partial f}{\partial x}: (x,y) \mapsto y - \frac{1}{x^2},\\
				&\frac{\partial f}{\partial y}: (x,y) \mapsto x - \frac{1}{y^2}.
			\end{align*}

		\item Trouver $f$ telle que $\begin{cases}
				(1): \qquad \frac{\partial f}{\partial x}=y,\\[2mm]
				(2): \qquad \frac{\partial f}{\partial y} = x.
			\end{cases}$

			D'après $(1)$ : \[
				\forall (x,y), \exists C(y) \in \R, f(x,y) = xy + C(y)
			\] et donc \[
				\frac{\partial f}{\partial y}(x,y) = x + C'(y)
			\] donc $C'(y) = 0$ et donc $C$ est constante.

		\item Trouver $f$ telle que $\begin{cases}
			\frac{\partial f}{\partial x} = -y,\\[2mm]
			\frac{\partial f}{ƒ\partial y} = x.
		\end{cases}$

		Ce n'est pas possible !
	\end{enumerate}
\end{exm}

\begin{defn}~\\
	\begin{minipage}{\linewidth}
		\begin{wrapfigure}{r}{4cm}
			\centering
			\vspace{-5mm}
			\begin{asy}
				import three;
				import graph3;
				size(4cm);

				settings.render = 0;
				settings.prc = false;
				currentprojection = obliqueX;

				draw(O -- X, Arrow3(TeXHead2));
				draw(O -- Y, Arrow3(TeXHead2));
				draw(O -- Z, Arrow3(TeXHead2));

				triple f(real x, real y, real z = 0) { return (x,y,cos(x - 0.5) * cos(y - 0.5)/1.2 + 0.15); }

				real inc = 1 / 5;

				for(real x = 0; x <= 1; x += inc) {
					draw(graph(
						new real(real t) { return x; }, // x
						new real(real y) { return y; }, // y
						new real(real y) { return f(x,y).z; }, // z
						0, 1
					), gray);
				}

				for(real y = 0; y <= 1; y += inc) {
					draw(graph(
						new real(real x) { return x; }, // x
						new real(real t) { return y; }, // y
						new real(real x) { return f(x,y).z; }, // z
						0, 1
					), gray);
				}

				path3 path1 = (0.8, 0.2, 0) .. (0.5, 0.5, 0) .. (0.3, 0.7, 0);
				path3 path2 = f(0.8, 0.2, 0) .. f(0.5, 0.5, 0) .. f(0.3, 0.7, 0);
				path3 d = (0.2, 0.3, 0) .. (0.3, 0.4, 0) .. (0.2, 0.7, 0) .. (0.8, 0.9, 0) .. (0.6, 0.2, 0) .. cycle;

				draw(path1, red, Arrow3(TeXHead2));
				draw(path2, red, Arrow3(TeXHead2, position=0.8));

				dot((0.5, 0.5, 0));
				dot(f(0.5, 0.5, 0));
				draw((0.5, 0.5, 0) -- f(0.5, 0.5, 0), dashed);
				draw(d);

				label("$w$", (0.3, 0.7, 0), red, align=SE);
				label("$U$", (0.8, 0.9, 0), align=SE);
			\end{asy}
		\end{wrapfigure}

		Soit $f: U \to \R$ où $U$ est un ouvert. Soit $(a,b) \in U$. Soit $w = (w_1, w_2) \in \R^2$.

		Si 
		\[
			\lim_{t\to 0} \frac{f(a + tw_1, b + tw_2) - f(a,b)}{t}
		\] existe et est réelle, alors on dit que $f$ a une dérivée dans la direction de $w$ et la limite est notée \[
			\mathrm{d}f(w)\,(a,b) = D_w(f)\,(a,b).
		\]
	\end{minipage}
\end{defn}

\begin{exm}
	\begin{align*}
		f: \left( \R_*^+ \right)^2 &\longrightarrow \R \\
		(x,y) &\longmapsto xy+\frac{1}{x}+\frac{1}{y}.
	\end{align*}

	On pose $(a,b) = (1,2)$, $w = (w_1, w_2) = (1,1)$.
	\begin{align*}
		\frac{f(1+t, 2+t) - f(1,2)}{t} &= \frac{1}{t} \left( (1+t)(2+t) + \frac{1}{1+t} + \frac{1}{2+t} - 3 - \frac{1}{2} \right) \\
		&= \frac{1}{t}\left(\cancel 2 + 3t + \po(t) + \cancel 1 - t + \po(t) + \frac{1}{2}\left( \cancel 1 - \frac{t}{2} + \po(t) \right) - \cancel3 - \cancel{\frac{1}{2}} \right) \\
		&= \frac{1}{t} \left( \frac{7}{4} t + \po(t) \right)  \\
		&= \frac{7}{4} + \po(1) \tendsto{t \to 0} \frac{7}{4}. \\
	\end{align*}

	Donc, \[
		\mathrm{d}f(1,1)\,(1,2) = \frac{7}{4}.
	\]
\end{exm}

\begin{rmk}~\\
	\begin{figure}[H]
		\centering
		\begin{asy}
			import solids;
			import graph;
			size(5cm);

			settings.render = 0;
			settings.prc = false;

			path3 par = graph(
				new real(real x) { return x; },
				new real(real x) { return 0; },
				new real(real x) { return x^2; },
				0,3);
			revolution r = revolution(par, axis=Z);

			path3 par2 = graph(
				new real(real x) { return x; },
				new real(real x) { return 0; },
				new real(real x) { return x^2; },
				-3,3);

			draw(r,1,longitudinalpen=nullpen);
			draw(r.silhouette());

			draw((-4, 0, -1) -- (-4, 0, 10) -- (4, 0, 10) -- (4, 0, -1) -- cycle, red);
			draw(par2, deepred);

			draw((4,4.5) -- (7, 4.5), black+0.5mm, Arrow(TeXHead));

			path par2d = graph(new real(real x) { return x^2; }, -3, 3);
			draw(shift((11, 0)) * par2d, deepred);

			dot(O);
			dot((11, 0));
		\end{asy}
	\end{figure}
\end{rmk}


%todo ajouter théorème-définition
\begin{thm}
	Soit $f : U \to \R$, $(a,b) \in U$. On suppose que $\frac{\partial f}{\partial x}$ et $\frac{\partial f}{\partial y}$ existent en $(a,b)$ et sont {\bfseries continues} en $(a,b)$. Alors,
	\begin{align*}
		&\forall (h, k) \in \R^2 \text{ tel que } (a +h, b + k) \in U,\\
		&f(a+ h, b + k) = f(a,b) + h \frac{\partial f}{\partial x}(a,b) + k \frac{\partial f}{\partial y}(a,b) + \po_{(h,k)\to (0,0)}\big(\|(h,k)\|\big).
	\end{align*}

	On dit que $f$ est de classe $\mathcal{C}^1$ si $\frac{\partial f}{\partial x}$ et $\frac{\partial f}{\partial y}$ existent et sont continues.

	\qed
\end{thm}

\begin{rmk}
	En physique, cette formule correspond à : \[
		\mathrm{d}f = \frac{\partial f}{\partial x}\mathrm{d}x + \frac{\partial f}{\partial y} \mathrm{d}y.
	\] En effet :
	\begin{align*}
		\mathrm{d}f &= f(x+ \mathrm{d}x, y + \mathrm{d}y) - f(x,y) \\
		&= \frac{\partial f}{\partial x} \mathrm{d}x + \frac{\partial f}{\partial y} \mathrm{d}y.
	\end{align*}
\end{rmk}

\begin{prop}
	Soit $f: U \to \R$ de classe $\mathcal{C}^1$ en $(a,b) \in U$. Alors, \[
		\forall w = (w_1, w_2) \in \R^2, \mathrm{d}f(w)\,(a,b) = w_1 \frac{\partial f}{\partial x}(a,b) + w_2 \frac{\partial f}{\partial y}(a,b).
	\]
\end{prop}

\begin{prv}
	Soit $w = (w_1, w_2) \in \R^2$. Soit $t \in \R^*$.
	\begin{align*}
		\frac{1}{t}\big(f(a + tw_1, b + tw_2) - f(a,b)\big)
		&= \frac{1}{t} \left( tw_1 \frac{\partial f}{\partial x}(a,b) + tw_2 \frac{\partial f}{\partial y}(a,b) + \po_{t \to 0}\big(\|tw\|\big) \right) \\
		&= w_1 \frac{\partial f}{\partial x}(a,b) + w_2 \frac{\partial f}{\partial y}(a,b) + \po_{t\to 0}(1) \\
		&\tendsto{t\to 0} w_1 \frac{\partial f}{\partial x}(a,b) + w_2\frac{\partial f}{\partial y}(a,b).
	\end{align*}
\end{prv}


\begin{defn}
	Avec les hypothèses précédentes, en posant \[
		\nabla f(a,b) = \left( \frac{\partial f}{\partial x}(a,b), \frac{\partial f}{\partial y}(a,b) \right) 
	\]on obtient \[
		\mathrm{d}f(w)\,(a,b) = \left<w  \mid \nabla f(a,b) \right>
	\] où $\left<\cdot|\cdot \right>$ est le produit scalaire.

	Le vecteur $\nabla f(a,b)$ est appelé \underline{gradient de $f$ en $(a,b)$}.

	Le développement limité à l'ordre 1 de $f$ devient \[
		f\big((a,b)+w\big) = f(a,b) + \left<w \mid \nabla f(a,b) \right> + \po_{w\to 0}(\|w\|)
	\]
\end{defn}

\begin{prop}
	Soit $f : U \to \R$ de classe $\mathcal{C}^1$.

	\begin{figure}[H]
    \centering
    \incfig{gradient}
	\end{figure}

	$\nabla f$ est orthogonal au lignes de niveaux de $f$, son orientation va dans le sens d'une augmentation de $f$.
\end{prop}

\begin{prv}
	Soit $\gamma : I \to U$ une courbe de niveau : \[
		\forall t \in I, f\big(\gamma(t)\big) = \text{cste}.
	\] D'après le lemme suivant : \[
		\forall t \in I, 0 = (f \circ \gamma)'(t) = \mathrm{d}f\big(\gamma'(t)\big)\big(\gamma(t)\big) = \left<\gamma'(t)  \mid \nabla f\big(\gamma(t)\big) \right>
	\] Donc $\nabla f\big(\gamma(t)\big)$ est orthogonal à $\gamma'(t)$.

	Pour tout $t \in I$, on pose $w(t) = t\, \nabla f\big(\gamma(t)\big)$. Donc \[
		f\big(\gamma(t) + w(t)\big) = f\big(\gamma(t)\big) + t \|\nabla f(\gamma(t))\|^2 + \po_{t \to 0}(t)
	\] Pour $t$ assez petit, $f\big(\gamma(t) + w(t)\big) - f\big(\gamma(t)\big)$ est du même signe que $t$.
\end{prv}

\begin{rmk}
	\begin{align*}
		V: \R^3 &\longrightarrow \R \\
		(x,y,z) &\longmapsto -mgz
	\end{align*}
	l'énerge potentielle de pesenteur

	On a donc \[
		\nabla V(x,y,z) = \left( \frac{\partial V}{\partial x}, \frac{\partial V}{\partial y}, \frac{\partial V}{\partial z} \right) = (0, 0, -mg) = \vec{P}.
	\]
\end{rmk}

\begin{lem}
	Soit $f : U \to \R$ de classe $\mathcal{C}^1$, $\gamma : \begin{array}{rcl}
		I &\longrightarrow& U \\
		t &\longmapsto& \big(x(t), y(t)\big)
	\end{array}$ où $x$ et $y$ sont dérivables.

	On pose \[
		\forall t \in I, \gamma'(t) = \big(x'(t), y'(t)\big).
	\] Alors $f \circ \gamma : I \to \R$ est dérivable et
	\begin{align*}
		\forall t \in I, (f \circ \gamma)'(t) &= \mathrm{d}f\big(\gamma'(t)\big) \big(\gamma(t)\big)\\
		&= \left<\gamma'(t)  \mid \nabla f\big(\gamma(t)\big)  \right> \\
		&= x'(t) \frac{\partial f}{\partial x}\big(x(t), y(t)\big) + y'(t) \frac{\partial f}{\partial y}\big(x(t),y(t)\big). \\
	\end{align*}
\end{lem}

\begin{prv}
	On fixe $t \in I$.

	\begin{align*}
		\forall h \neq 0, \frac{f \circ \gamma(t + h) - f \circ \gamma(t)}{h}
		&= \frac{1}{h}\big(f(\gamma(t)) + h\gamma'(t) + \po_{h\to 0}(h) - f(\gamma(t))\big) \\
		&= \frac{1}{h}\bigg(\cancel{f(\gamma(t))} + \left<h\gamma'(t) \mid \nabla f(\gamma(t)) \right> + \po_{h\to 0}(\|h\gamma'(t)\|) - \cancel{f(\gamma(t))}\bigg)\\
		&= \left<\gamma'(t) \mid \nabla f(\gamma(t)) \right> + \po_{h\to 0}(1) \\
		&\tendsto{h\to 0} \left<\gamma'(t)  \mid \nabla f(\gamma(t)) \right>
	\end{align*}
\end{prv}

\begin{defn}
	Soit $f : U \to \R$ de classe $\mathcal{C}^1$ et $(a,b) \in U$. On dit que $(a,b)$ est un \underline{point critique} de $f$ si $\nabla f(a,b) = 0$ i.e. $\frac{\partial f}{\partial x}(a,b) = \frac{\partial f}{\partial y}(a,b) = 0$.

	Dans ce cas, $f(a,b)$ est appelé \underline{valeur critique} de $f$.
\end{defn}

\begin{prop}~\\
	\begin{minipage}{\linewidth}
		\begin{wrapfigure}{r}{3cm}
			\centering
			\vspace{-1cm}
			\begin{asy}
				import solids;
				import graph;
				size(3cm);

				settings.render = 0;
				settings.prc = false;

				path3 par = graph(
					new real(real x) { return x; },
					new real(real x) { return 0; },
					new real(real x) { return -x^2; },
					0,3);
				revolution r = revolution(par, axis=Z);

				draw(r,1,longitudinalpen=nullpen);
				draw(r.silhouette());

				dot("$(a,b)$", O, red, align=N);
				real s = sqrt(2.5);
				path3 g=(s,0,-2.5)..(0,s,-2.5)..(-s,0,-2.5)..(0,-s,-2.5)..cycle;
				draw(g, deepcyan);
			\end{asy}
		\end{wrapfigure}
		Soit $f: U \to \R$ de classe $\mathcal{C}^1$ et $(a,b) \in U$ tel que \[
			\exists r > 0, \forall (x,y) \in B_{(a,b)}(r), f(x,y) \le f(a,b)
		\] Alors $\nabla f(a,b) = (0,0)$.
	\end{minipage}
\end{prop}

\begin{prv}
	Soit $g: x \mapsto f(x,b)$. $g(a)$ est un maximum local de $g$ donc $g'(a) = 0$.

	Or, $g'(a) = \frac{\partial f}{\partial x}(a,b)$

	donc $\frac{\partial f}{\partial x}(a,b) = 0$.

	Soit $h : y \mapsto f(a,y)$. On a de même $h'(b) = 0$.

	Or, $h'(b) = \frac{\partial f}{\partial y}(a,b)$.

	Donc, $\nabla f(a,b) = (0,0)$.
\end{prv}

\begin{rmk}
	Un minimum local est aussi une valeur critique.
\end{rmk}

\begin{figure}[H]
	\centering
	\begin{subfigure}{3cm}
		\centering
		\begin{asy}
			import solids;
			import graph;
			size(3cm);

			settings.render = 0;
			settings.prc = false;

			path3 par = graph(
				new real(real x) { return x; },
				new real(real x) { return 0; },
				new real(real x) { return -x^2; },
				0,3);
			revolution r = revolution(par, axis=Z);

			draw(r,1,longitudinalpen=nullpen);
			draw(r.silhouette());

			dot(O, red);
		\end{asy}
		\caption{Maximum local}
	\end{subfigure}
	\begin{subfigure}{3cm}
		\centering
		\begin{asy}
			import solids;
			import graph;
			size(3cm);

			settings.render = 0;
			settings.prc = false;

			path3 par = graph(
				new real(real x) { return x; },
				new real(real x) { return 0; },
				new real(real x) { return x^2; },
				0,3);
			revolution r = revolution(par, axis=Z);

			draw(r,1,longitudinalpen=nullpen);
			draw(r.silhouette());

			dot(O, red);
		\end{asy}
		\caption{Minimum local}
	\end{subfigure}
	\begin{subfigure}{3cm}
		\centering
		\begin{asy}
			import solids;
			import graph;
			size(3cm);

			settings.render = 0;
			settings.prc = false;
			currentprojection = obliqueZ;

			draw(graph(
				new real(real x) { return x; },
				new real(real x) { return -x^2 / 3; },
				new real(real x) { return 3; },
				-3, 3
			));

			draw(graph(
				new real(real x) { return x; },
				new real(real x) { return -x^2 / 3; },
				new real(real x) { return -3; },
				-3, 3
			));

			draw(graph(
				new real(real x) { return x; },
				new real(real x) { return -x^2 / 3 - 1; },
				new real(real x) { return 0; },
				-3, 3
			));

			draw(graph(
				new real(real x) { return 0; },
				new real(real x) { return x^2 / 9 - 1; },
				new real(real x) { return x; },
				-3, 3
			));

			draw(graph(
				new real(real x) { return -3; },
				new real(real x) { return x^2 / 9 - 4; },
				new real(real x) { return x; },
				-3, 3
			));

			draw(graph(
				new real(real x) { return 3; },
				new real(real x) { return x^2 / 9 - 4; },
				new real(real x) { return x; },
				-3, 3
			));

			dot((0,-1,0), red);
		\end{asy}
		\caption{Point de selle / Point col}
	\end{subfigure}
\end{figure}

\begin{exm}
	On revient à l'exemple donné en introduction : 
	\begin{align*}
		f: \left( \R^*_+ \right)^2 &\longrightarrow \R \\
		(x,y) &\longmapsto 2\left( xy + \frac{1}{x} + \frac{1}{y} \right).
	\end{align*}

	$\left( \R^+_* \right)^2$ est un ouvert de $\R^2$. Soit $(x,y) \in \left( \R^+_* \right)^2$.
	
	On a \[
		\begin{cases}
			\frac{\partial f}{\partial x}(x,y) = 2\left( y - \frac{1}{x^2} \right),\\
			\frac{\partial f}{\partial y}(x,y) = 2\left( x - \frac{1}{y^2} \right).
		\end{cases}
	\]

	\begin{align*}
		&\frac{\partial f}{\partial x}(x,y) = \frac{\partial f}{\partial y}(x,y) = 0\\
		\iff& \begin{cases}
			y = \frac{1}{x^2}\\
			x = \frac{1}{y^2}
		\end{cases}\\
		\iff& \begin{cases}
			y = \frac{1}{x^2}\\
			x = x^4
		\end{cases}\\
		\iff& \begin{cases}
			x = 1\\
			y = 1
		\end{cases}
	\end{align*}

	On vérivie que $f$ présente en effet un minium local en $(1,1)$. \[
		f(1,1) = 6
	\] On fixe $y \in \R^+_*$ et \[
		g : x \mapsto 2\left( xy + \frac{1}{x} + \frac{1}{y} \right).
	\] Donc \[
		\forall x \in \R^+_*, g'(x) = 2\left( y - \frac{1}{x^2} \right).
	\]
	\begin{center}
		\begin{tikzpicture}
			\tkzTabInit{$x$/1,$g'(x)$/1,$g$/2.3}{$0$, $\frac{1}{\sqrt{y}}$, $+\infty$}
			\tkzTabLine{,-,z,+,}
			\tkzTabVar{+/{}, -/$2\left( 2\sqrt{y} +\frac{1}{y} \right)$, +/{}}
		\end{tikzpicture}
	\end{center}
	
	Ainsi, \[
		\forall x \in \R^+_*, \forall y \in \R^+_*, f(x,y) \ge 2\left( 2\sqrt{y} + \frac{1}{y} \right)
	\] Soit $h : y \mapsto 2\sqrt{y} + \frac{1}{y}$. On a \[
		\forall y > 0, h'(y) = \frac{1}{\sqrt{y}} - \frac{1}{y^2} = \frac{y\sqrt{y} - 1}{y^2} = \frac{y^{\frac{3}{2}} - 1}{y^2}
	\]

	\begin{center}
		\begin{tikzpicture}
			\tkzTabInit{$y$/0.7,$h'(y)$/0.7,$h$/1.4}{$0$, $1$, $+\infty$}
			\tkzTabLine{,-,z,+,}
			\tkzTabVar{+/{}, -/$3$, +/{}}
		\end{tikzpicture}
	\end{center}

	Donc, \[
		\forall x,y > 0, f(x,y) \ge 2\times 3 = 6 = f(1,1).
	\]
\end{exm}

\begin{prop}
	[règle de la chaîne]

	Soit $f : \begin{array}{rcl}
		U &\longrightarrow& \R^2 \\
		(x,y) &\longmapsto& f(x,y)
	\end{array}$ de classe $\mathcal{C}^1$ et $U, V$ deux ouverts de $\R^2$.

	Soit $\varphi : \begin{array}{rcl}
		V &\longrightarrow& U \\
		(u,v) &\longmapsto& \varphi(u,v) = \big(x(u,v), y(u,v)\big)
	\end{array}$.

	On suppose que $x$ et $y$ sont de classe $\mathcal{C}^1$ sur $V$.

	Alors,  $f \circ \varphi : \begin{array}{rcl}
		V &\longrightarrow& \R \\
		(u,v) &\longmapsto& f\big(\varphi(u,v)\big)
	\end{array}$ est de classe $\mathcal{C}^1$ et
	\begin{align*}
		\forall (u_0, v_0) \in V, \frac{\partial (f \circ \varphi)}{\partial u}(u_0, v_0)
		&= \frac{\partial f}{\partial x}\big(\varphi(u_0, v_0)\big) \times \frac{\partial x}{\partial u}(u_0, v_0)\\
		&+ \frac{\partial f}{\partial y}\big(\varphi(u_0,v_0)\big) \frac{\partial y}{\partial u}(u_0,v_0)
	\end{align*}
	\begin{align*}
		\forall (u_0, v_0) \in V, \frac{\partial (f \circ \varphi)}{\partial v}(u_0, v_0)
		&= \frac{\partial f}{\partial x}\big(\varphi(u_0, v_0)\big) \times \frac{\partial x}{\partial v}(u_0, v_0)\\
		&+ \frac{\partial f}{\partial y}\big(\varphi(u_0,v_0)\big) \frac{\partial y}{\partial v}(u_0,v_0)
	\end{align*}
\end{prop}

\begin{exm}
	[changement de coordonnées polaires]
	On pose \begin{align*}
		\varphi: \R^+_* \times ]0,2\pi[ &\longrightarrow \R^2\setminus \left( R^+_* \times \{0\} \right) \\
		(r, \theta) &\longmapsto (r \cos \theta, r \sin\theta),
	\end{align*}
	\begin{align*}
		f: \R^2\setminus \left( R^+_* \times \{0\} \right) &\longrightarrow \R \\
		(x,y) &\longmapsto f(x,y),
	\end{align*}
	\begin{align*}
		g: \overbrace{\R^+_* \times ]0, 2\pi[}^{=V} &\longrightarrow \R \\
		(r, \theta) &\longmapsto f(r\cos\theta, r\sin\theta).
	\end{align*}

	\begin{align*}
		\forall (r_0,\theta_0) \in V,&\\[5mm]
		\frac{\partial g}{\partial r}(r_0, \theta_0) &= \frac{\partial f}{\partial x}(r_0\cos\theta_0, r_0\sin\theta_0)\cos\theta_0\\
		&+ \frac{\partial f}{\partial y}(r_0 \cos\theta_0, r_0\sin\theta_0)\sin\theta_0\\
		&= 2r_0\cos^2\theta_0 + 2r_0\sin^2(\theta_0) \\
		&= 2r_0 \\[5mm]
		\frac{\partial g}{\partial \theta}(r_0, \theta_0) &= \frac{\partial f}{\partial x}(r_0\cos\theta_0, r_0\sin\theta_0)r_0\sin\theta_0\\
		&+ \frac{\partial f}{\partial y}(r_0 \cos\theta_0, r_0\sin\theta_0)r_0\cos\theta_0\\
		&= -2{r_0}^2\cos(\theta_0)\sin(\theta_0) + 2{r_0}^2 \sin(\theta_0)\cos(\theta_0)\\
		&= 0 \\
	\end{align*}

	Donc, \[
		g(r, \theta) = r^2.
	\]
\end{exm}

\begin{exm}
	Résoudre \[
		\begin{cases}
			\frac{\partial f}{\partial x} = \frac{x}{x^2+y^2},\\
			\frac{\partial f}{\partial y} = \frac{y}{x^2+y^2}.\\
		\end{cases}
	\]

	On pose $g: (r, \theta) \mapsto f(r \cos\theta, r \sin\theta)$.

	\begin{align*}
		&\frac{\partial g}{\partial r} = \frac{1}{r}\cos^2\theta + \frac{1}{r}\sin^2\theta = \frac{1}{r},\\
		&\frac{\partial g}{\partial \theta} = -\cos(\theta) \sin(\theta) + \sin(\theta)\cos(\theta) = 0.
	\end{align*}

	Donc, \[
		\exists C \in \R, g: (r, \theta) \mapsto \ln r + C
	\] d'où,
	\begin{align*}
		\forall (x,y) \in \R^2 \setminus \{(0,0)\}, f(x,y) &= \ln\left(\sqrt{x^2 + y^2} \right)  + C\\
		&= \frac{1}{2}\ln(x^2 + y^2) + C. \\
	\end{align*}
\end{exm}

\begin{rmk}
	Soit $\mathcal{B} = (e_1, e_2)$ la base canonique de $\R^2$, $f: U \to \R$ de classe $\mathcal{C}^1$ avec $U$ un ouvert de $\R^2$.

	Soit $(x,y) \in U$.

	\begin{align*}
		\Mat_{\mathcal{B}}\big(\nabla f(x,y)\big) = \begin{pmatrix}
			\frac{\partial f}{\partial x}(x,y)\\[2mm]
			\frac{\partial f}{\partial y}(x,y)
		\end{pmatrix}
	\end{align*}

	Soit  \begin{align*}
		\varphi: V &\longrightarrow U \\
		(u,v) &\longmapsto \big(x(u,v), y(u,v)\big) 
	\end{align*} avec $x,y$ de classe $\mathcal{C}^1$. Soit $g = f \circ \varphi$.
	\begin{align*}
		\Mat_{\mathcal{B}}\big(\nabla g(u,v)\big)
		&= \begin{pmatrix}
			\frac{\partial g}{\partial u}(u,v) \\[2mm]
			\frac{\partial g}{\partial v}(u,v)
		\end{pmatrix} \\
		&= \begin{pmatrix}
			\frac{\partial x}{\partial u}(u,v) \frac{\partial f}{\partial x}(x,y)
			+ \frac{\partial y}{\partial u}(u,v)\frac{\partial f}{\partial y}(x,y)\\[3mm]
			\frac{\partial x}{\partial v}(u,v) \frac{\partial f}{\partial x}(x,y)
			+ \frac{\partial y}{\partial v}(u,v) \frac{\partial f}{\partial y}(x,y)
		\end{pmatrix}  \\
		&= \underbrace{\begin{pmatrix}
				\frac{\partial x}{\partial u}(u,v)& \frac{\partial y}{\partial u}(u,v)\\[3mm]
				\frac{\partial x}{\partial v}(u,v)& \frac{\partial y}{\partial v}(u,v)
		\end{pmatrix}}_{J(u,v)} \begin{pmatrix}
			\frac{\partial f}{\partial x}(x,y)\\[3mm]
			\frac{\partial f}{\partial y}(x,y)
		\end{pmatrix} \\
		&= J(u,v) \Mat_{\mathcal{B}}\big(\nabla f(x,y)\big) \\
	\end{align*}
	où $J(u,v) = 
	\begin{pNiceArray}{c:c}
		\Mat_{\mathcal{B}}\big(\nabla x(u,v)\big) & \Mat_{\mathcal{B}}\big(\nabla y(u,v)\big)
	\end{pNiceArray}$.

	On dit que $J(u,v)$ est \underline{la jacobienne} de $\varphi$ en $(u,v)$.
	L'application linéaire canoniquement associée à $J(u,v)$ est la \underline{différentielle de $\varphi$} en $(u,v)$ noté $\mathrm{d}\varphi(u,v)$.

	On a $\mathrm{d}\varphi(u,v) \in \mathcal{L}(R^2)$ et $\Mat_{\mathcal{B}}\big(\mathrm{d}\varphi(u,v)\big) = J(u,v)$.

	Par exemple, la jacobienne du changement de coordonnées polaires est \[
		J = \begin{pmatrix}
			\frac{\partial x}{\partial r} & \frac{\partial y}{\partial r}\\[3mm]
			\frac{\partial x}{\partial \theta} & \frac{\partial y}{\partial \theta}
		\end{pmatrix}
		= \begin{pmatrix}
			\cos\theta&\sin\theta\\
			-r\sin\theta&r\cos\theta
		\end{pmatrix}.
	\]
	$\underbrace{\det(J)}_{\text{le jacobien}} = r\cos^2\theta + r\sin^2\theta = r$

	Dans une intégrale double, si $(x,y) = \varphi(u,v)$, alors $\mathrm{d}x\mathrm{d}y = \det(J)\mathrm{d}u\mathrm{d}v$.

	Ici, \[
		\mathrm{d}x\ \mathrm{d}y = r\ \mathrm{d}r\ \mathrm{d}\theta.
	\]
\end{rmk}

\begin{prv}
	On pose $(x_0, y_0) = \varphi(u_0, v_0)$. Pour tout $(h,k) \in \R^2$ tels que $(u_0 + h, v_0 + k) \in V$, en posant $g = f  \circ \varphi$.

	\begin{align*}
		g(u_0 + h, v_0 + h) &= f\big(x(u_0 + h, v_0 + k), y(u_0 + h, v_0 + k)\big) \\
		&= f\left(
			x(u_0,v_0) + h \frac{\partial x}{\partial u}(u_0,v_0) + k \frac{\partial x}{\partial v}(u_0, v_0) + \po\big(\|(h,k)\|\big), \right.\\
		&\phantom{ = f\bigg(\bigg.}\left. y(u_0, v_0) + h \frac{\partial y}{\partial u}(u_0, v_0) + k \frac{\partial y}{\partial v}(u_0, v_0) + \po\big(\|(h,k)\|\big)
		\right)  \\
		&= f(x_0,y_0) \\
		&~+ \left( h \frac{\partial x}{\partial u}(u_0,v_0) + k \frac{\partial x}{\partial v}(u_0, v_0) + \po(\|(h,k)\|) \right) \frac{\partial f}{\partial x}(x_0,y_0)\\
		&~+ \left( h \frac{\partial y}{\partial u}(u_0, v_0) + k\frac{\partial y}{\partial v}(u_0, v_0) + \po(\|(h,k)\|) \right) \frac{\partial f}{\partial y}(x_0, y_0)\\
		&~+ \po(\|(h,k)\|)\\
		&= f(x_0, y_0) \\
		&~+ h \left( \frac{\partial x}{\partial u}(u_0, v_0) \frac{\partial f}{\partial x}(x_0, y_0) + \frac{\partial y}{\partial u}(u_0, v_0) \frac{\partial f}{\partial y}(x_0, y_0) \right)  \\
		&~+ k\left( \frac{\partial x}{\partial v}(u_0, v_0) \frac{\partial f}{\partial x}(x_0, y_0) + \frac{\partial y}{\partial v}(u_0, v_0) \frac{\partial f}{\partial y}(x_0, y_0) \right) 
		&~+ \po(\|(h,k)\|)\\
		&= g(u_0, v_0) + h \frac{\partial g}{\partial u}(u_0, v_0) + k \frac{\partial g}{\partial v}(u_0, v_0) + \po(\|(h,k)\|) \\
	\end{align*}

	Par identification,
	\[
		\frac{\partial g}{\partial u}(u_0, v_0) = \frac{\partial x}{\partial u}(u_0, v_0) \frac{\partial f}{\partial x}(x_0, y_0) + \frac{\partial y}{\partial u}(u_0, v_0) \frac{\partial f}{\partial y}(x_0,y_0)
	\] et \[
		\frac{\partial g}{\partial v}(u_0, v_0) = \frac{\partial x}{\partial v}(u_0,v_0) \frac{\partial f}{\partial x}(x_0, y_0) + \frac{\partial y}{\partial v}(u_0, v_0) \frac{\partial f}{\partial y}(x_0, y_0).
	\] 
\end{prv}

\begin{exm}
	[Régression linéaire]~\\
	\begin{figure}[H]
		\centering
		\begin{asy}
			import graph;
			axes(EndArrow);
			size(5cm);

			real f(real x) { return x + 0.5; }

			real k = 35 / (7 - 0.5);

			for(int i = 0; i < 35; ++i) {
				real mag = exp(sin(100 * pi/exp(1) * i)) * 0.8 + exp(cos(i*40)/3);
				real eps = mag * cos(10 * exp(1)/pi * i) / 3;
				dot((i/k,f(i/k) + eps));
			}

			draw(graph(f, -1, 7), orange);
		\end{asy}
	\end{figure}
	\[
		y = a x + b
	\] 
	On fixe $(a,b) \in \R^2$. \[
		\varepsilon(a,b) = \sum_{i=1}^n\big( y_i - (ax_i + b) \big)^2
	\] l'erreur totale.

	On veut minimiser $\varepsilon(a,b)$. On a 
	\[
		\forall (a,b) \in \R^2,
		\begin{cases}
			\frac{\partial \varepsilon}{\partial a}(a,b) = -2\sum_{i=1}^{n}(y_i - ax_i - b)x_i,\\
			\frac{\partial \varepsilon}{\partial b}(a,b) = -2\sum_{i=1}^{n}(y_i - ax_i - b).
		\end{cases}
	\]

	Donc,
	\begin{align*}
		(a,b) \text{ point critique de } \varepsilon \iff& \begin{cases}
			a \sum_{i=1}^n {x_i}^2 + b\sum_{i=1}^{n}x_i = \sum_{i=1}^{n} y_ix_i\\
			a\sum_{i=1}^{n}x_i + nb = \sum_{i=1}^ny_i
		\end{cases}\\
		\iff& \begin{cases}
			a \left( \frac{1}{n}\sum_{i=1}^n {x_i}^2 - \overline{x}^2\right) = \overline{y} - \overline{x} \overline{y}\\
			b = \frac{1}{n}\sum_{i=1}^ny_i - \frac{a}{n}\sum_{i=1}^nx_i = \frac{1}{n}\sum_{i=1}^n x_i y_i - \overline{x} \overline{y}
		\end{cases}\\
		&\text{ où } \overline{x} = \frac{1}{n} \sum_{i=1}^n x_i,~\overline{y} = \frac{1}{n}\sum_{i=1}^n y_i\\
		\iff& \begin{cases}
			a = \frac{\Cov(x,y)}{V(x)}\\
			b = \overline{y} - a\overline{x}
		\end{cases}
	\end{align*}

	Coefficient de corrélation: $\frac{\Cov(x,y)}{\sigma_x \sigma_y} \in [-1, 1]$
\end{exm}












		\part{Corps}

\begin{exm}[Problème]
	\begin{itemize}
		\item 
			avec $A = \Z / 9 \Z$, résoudre $\overline{x}^2 = \overline{0}$ \\
			\begin{center}
				\begin{tabular}{|c|c|c|c|c|c|c|c|c|c|c|}
					\hline
					$\overline{x}$&$\overline{0}$& $\overline{1}$ &$\overline{2}$&$\overline{3}$ &$\overline{4}$ &$\overline{5}$ &$\overline{6}$ &$\overline{7}$ &$\overline{8}$& $\overline{9}$ \\
					\hline
					$\overline{x}^2$&$\overline{0}$ &$\overline{1}$ &$\overline{4}$ &$\overline{0}$ &$\overline{7}$ &$7$ &$\overline{0}$ &$\overline{4}$ &$\overline{1}$&$\overline{0}$\\
					\hline
				\end{tabular}
			\end{center}
			On a trouvé 3 solutions: $\overline{0}$, $\overline{3}$, $\overline{6}$.
		\item $\Z / 8\Z$
			\begin{center}
				\begin{tabular}{|c|c|c|c|c|c|c|c|c|}
					\hline
					$\overline{x}$& $\overline{0}$& $\overline{1}$& $\overline{2}$& $\overline{3}$& $\overline{4}$& $\overline{5}$& $\overline{6}$& $\overline{7}$\\
					\hline
					$\overline{x^2}$& $\overline{0}$& $\overline{1}$& $\overline{4}$& $\overline{1}$& $\overline{0}$& $\overline{1}$& $\overline{4}$& $\overline{1}$\\
					\hline
				\end{tabular}
			\end{center}
			$\overline{x}^2=7$ a 4 solutions: $\overline{1}, \overline{7}, \overline{3},\text{ et } \overline{5}$
		\item $A = \mathbbm{H} = \{a + bi + cj + dk  \mid  (a,b,c,d) \in \R^4\}$ \\
			$i^2 = j^2 = k^2 = -1$ 
			\begin{align*}
				\begin{array}{c c c}
					ij = k & jk = i & ji = j\\
					ji = -k & kj = -i & ik = -j
				\end{array}
			\end{align*}
			Dans cet anneau, $-1$ a 6 racines!
	\end{itemize}
\end{exm}

\begin{defn}
	Soit $(\mathbbm{K}, +, \times)$ un ensemble muni de deux lois de composition internes. On dit que c'est un \underline{corps} si
	 \begin{enumerate}
		\item $(\mathbbm{K}, \times)$ est un groupe abélien
		\item $(\mathbbm{K}, \times)$ est un monoïde commutatif
		\item $\forall x \in \mathbbm{K}\setminus \{0_\mathbbm{K}\}, \exists y \in \mathbbm{K}, xy = 1_\mathbbm{K}$
		\item $0_\mathbbm{K} \neq  1_\mathbbm{K}$
	\end{enumerate}
	\index{corps}
\end{defn}

\begin{exm}
	\begin{itemize}
		\item $(\C, +, \times)$ est un corps
		\item $(\R, +, \times)$ est un corps
		\item $(\Q, +, \times)$ est un corps
		\item $(\Z, +, \times)$ n'est pas un corps
	\end{itemize}
\end{exm}

\begin{prop}
	$(\Z / n\Z, +, \times)$ est un corps si et seulement si $n$ est premier.
\end{prop}

\begin{prv}
	\[
		\left( \Z / n\Z \right)^\times = \left\{ \overline{k}  \mid k \wedge n = 1 \right\}
	\] 
\end{prv}


\begin{prop}
	Tout corps est un anneau intègre.
\end{prop}

\begin{prv}
	Soit $(\mathbbm{K}, +, \times)$ un corps. Soient $(a,b) \in \mathbbm{K}^2$ tel que $a \times b = 0_\mathbbm{K}$.\\
	On suppose $a \neq  0_\mathbbm{K}$. Alors, $a$ est inversible et donc \[
		b = a^{-1} \times a \times b = a^{-1} \times 0_\mathbbm{K} = 0_\mathbbm{K}
	\] 
\end{prv}

\begin{exm}
	Soit $(\mathbbm{K},+,\times)$ un corps.\\
	Résoudre \[
		\begin{cases}
			x^2 = 1_\mathbbm{K}\\
			x \in \mathbbm{K}
		\end{cases}
	\]

	\begin{align*}
		x^2 = 1_\mathbbm{K} &\iff x^2 - 1_\mathbbm{K} = 0_\mathbbm{K}\\
		&\iff (x - 1_\mathbbm{K})(x+1_\mathbbm{K}) = 0_\mathbbm{K}\\
		&\iff x - 1_\mathbbm{K} = 0_\mathbbm{K} \text{ ou } x + 1_\mathbbm{K} = 0_\mathbbm{K}\\
		&\iff x = 1_\mathbbm{K} \text{ ou } x = -1_\mathbbm{K}
	\end{align*}

	Il y a au plus 2 solutions.
\end{exm}

\begin{prop}
	Soit $(\mathbbm{K},+,\times )$ un corps et $P$ un polynôme à coefficients dans $\mathbbm{K}$ de degré $n$. Alors, l'équation $P(x) = 0_{\mathbbm{K}}$ a au plus $n$ solutions dans $\mathbbm{K}$ 
	\qed
\end{prop}

\begin{crlr}[(Théorème de Wilson)]
	voir exercice 16 du TD 12
\end{crlr}


\begin{defn}
	Soit $(\mathbbm{K}, +, \times)$ un corps et $L\subset \mathbbm{K}$.\\
	On dit que $L$ est un \underline{sous corps} de $\mathbbm{K}$ si
	\begin{enumerate}
		\item $L$ est un anneau de $(\mathbbm{K}, +, \times)$ non nul
		\item $\forall x \in L\setminus \{0_\mathbbm{K}\}, x^{-1} \in L$ 
	\end{enumerate}
	\vspace{2mm}
	en d'autres termes si
	\begin{enumerate}
		\item $\forall (x,y) \in L^2, x - y \in L$
		\item $\forall (x,y) \in L^2, x \times y^{-1} \in L$
	\end{enumerate}
	\vspace{5mm}
	On dit aussi que $\mathbbm{K}$ est une \underline{extension} de $L$.
	\index{sous corps}
	\index{extension}
\end{defn}

\begin{prop}
	Tout sous corps est un corps. \qed
\end{prop}

\begin{defn}
	Soient $(\mathbbm{K}_1,+,\times )$ et $(\mathbbm{K}_2,+, \times)$ deux corps et $f: \mathbbm{K}_1 \to \mathbbm{K}_2$.\\
	On dit que $f$ est un \underline{morphisme de corps} si $f$ est un morphisme d'anneaux.\\
	i.e. si
	\[
		\begin{cases}
			\forall (x,y) \in {\mathbbm{K}_1}^2,& f(x+y) = f(x) + f(y)\\
			\forall (x,y) \in {\mathbbm{K}_1}^2,& f(x \times y) = f(x) \times f(y)\\
		\end{cases}
	\] 
	\index{homomorphisme (de corps)}
	\index{morphisme (de corps)}
\end{defn}

\begin{prop}
	Tout morphisme de corps est injectif.
\end{prop}

\begin{prv}
	Soit $f: \mathbbm{K}_1 \to \mathbbm{K}_2$ un morphisme de corps.\\
	\begin{itemize}
		\item $\Ker(f)$ est un sous groupe de $(\mathbbm{K}_1, +)$ 
		\item Soit $x \in \Ker(f)$ et $y \in \mathbbm{K}_1$ \[
				f(x \times y) = f(x) \times f(y) = 0_{\mathbbm{K}_2} \times f(y) = 0_{\mathbbm{K}_2}
			\]
		\item Soit $x \in \Ker(f) \setminus \{0_{\mathbbm{K}_1}\}$.\\
			Alors, $x$ est inversible.\\
			\begin{align*}
				\begin{rcases*}
					x \in \Ker(f)\\
					x^{-1} \in \mathbbm{K}_1
				\end{rcases*}& \text{ donc } x \times x ^{-1} \in \Ker(f)\\
				&\text{ donc } 1_{\mathbbm{K}_1} \in \Ker(f)\\
				&\text{ donc } f(1_{\mathbbm{K}_1}) = 0_{\mathbbm{K}_2}
			\end{align*}
			Or, $f(1_{\mathbbm{K}_1}) = 1_{\mathbbm{K}_2} \neq 0_{\mathbbm{K}_2}$
	\end{itemize}
	Donc, $\Ker(f) = \{0_{\mathbbm{K}_1}\}$ donc $f$ est injective.
\end{prv}

\begin{exm}
	$\begin{array}{cc}
		\C &\longrightarrow \C\\
		z &\longmapsto \overline{z}\\
	\end{array}$ est un morphisme de corps
\end{exm}



	}

	{
		\chap[07]{Développements limités}
		\renewcommand{\cwd}{../chap07}
		\begin{defn}
	Soit $E$ un $\mathbbm{K}$-espace vectoriel. On dit que $E$ est de \underline{dimension finie} si $E$ a au moins une famille génératrice finie. On dit que $E$ est de \underline{dimension infinie} sinon.
	\index{dimension finie (espace vectoriel)}
	\index{dimension infinie (espace vectoriel)}
\end{defn}

\begin{thm}
	[Théorème de la base extraite]
	Soit $E$ un $\mathbbm{K}$-espace vectoriel non nul de dimension finie. Soit $\mathcal{G}$ une famille génératrice finie de $E$. Alors, il existe une base $\mathcal{B}$ de $\mathcal{E}$ telle que $\mathcal{B} \subset \mathcal{G}$.
\end{thm}

\begin{prv}
	[par récurrence sur $\#G = \Card(G)$]
	\begin{itemize}
		\item Soit $E$ un $\mathbbm{K}$-espace vectoriel non nul engendré par $\mathcal{G} = (u)$.\\
			Si $u = 0_E$, alors $E = \{0_E\}$: une contradiction $\lightning$ \\
			Donc $u \neq 0_E$ donc $(u)$ est libre. En effet, \[
				\forall \lambda \in \mathbbm{K}, \lambda u = 0_E \implies \lambda = 0_\mathbbm{K}
			\] Donc $\mathcal{G}$ est une base de $E$.\\
		\item Soit $n \in \N_*$. Soit $E$ un $\mathbbm{K}$-espace vectoriel. On suppose que si $E$ a une famille génératrice constituée de $n$ vecteurs, alors on peut extraire de cette famille une base de $E$.\\
			Soit $\mathcal{G}$ une famille génératrice de $E$ avec $n+1$ vecteurs.\\
			Si $\mathcal{G}$ est libre, alors $\mathcal{G}$ est une base de $E$. \\
			Si $\mathcal{G}$ n'est pas libre, alors il existe $u \in \mathcal{G}$ tel que $u \in \Vect(\mathcal{G}\setminus \{u\})$ \\
			Donc $\mathcal{G}\setminus \{u\}$ engendre $E$. Or, $\mathcal{G}\setminus \{u\}$ possède $n$ vecteurs. D'après l'hypothèse de récurrence, il existe une base $\mathcal{B}$ de $E$ telle que \[
				\mathcal{B} \subset \mathcal{G} \setminus \{u\} \subset \mathcal{G}
			\] 
	\end{itemize}
\end{prv}

\begin{crlr}
	Tout espace de dimension finie a une base.
	\qed
\end{crlr}

\begin{thm}
	[Théorème de la base incomplète]
	Soit $E$ un $\mathbbm{K}$-espace vectoriel de dimension finie, $\mathcal{G}$ une famille génératrice finie de $E$. $\mathcal{L}$ une famille libre de $E$. Alors, il existe une base $\mathcal{B}$ de $E$ telle que \[
		\mathcal{L} \subset \mathcal{B} \text{ et } \mathcal{B}\setminus \mathcal{L} \subset \mathcal{G}
	\] 
\end{thm}

\begin{prv}
	[par récurrence sur $\#(\mathcal{G}\setminus\mathcal{L})$]
	\begin{itemize}
		\item Avec les notations précédentes, on suppose que $\mathcal{G}\setminus\mathcal{L} \neq \O$ \[
				\forall u \in \mathcal{G}, u \in \mathcal{L}
			\] Donc $\mathcal{G} \subset \mathcal{L}$ donc $\mathcal{L}$ est génératrice donc $\mathcal{L}$ est une base de $E$. On pose $\mathcal{B} = \mathcal{L}$ et alors \[
				\mathcal{L} \subset  \mathcal{B} \text{ et } \mathcal{B}\setminus\mathcal{L} = \O \subset  \mathcal{G}
			\] 
		\item Soit $n \in \N$. On suppose que si $\mathcal{G}$ est génératrice et $\mathcal{L}$ libre avec $\#(\mathcal{G}\setminus\mathcal{L}) = n$ alors il existe une base $\mathcal{B}$ de $E$ telle que \[
			\mathcal{L}\subset \mathcal{B} \text{ et } \mathcal{B}\setminus\mathcal{L}\subset \mathcal{G}
		\] Soient à présent $\mathcal{G}$ une famille génératrice de $E$ et $\mathcal{L}$ une famille libre de $E$ telles que $\#(\mathcal{G}\setminus\mathcal{L}) = n+1 > 0$\\
		Si $\mathcal{L}$ engendre $E$, alors $\mathcal{L}$ est une base de $E$. On pose $\mathcal{B} = \mathcal{L}$ et on a bien \[
			\mathcal{L} \subset  \mathcal{B} \text{ et } \mathcal{B} \setminus \mathcal{L} = \O \subset  \mathcal{G}
		\] On suppose que $\mathcal{L}$ n'engendre pas $E$. Il existe $u \in \mathcal{G}$ tel que $u \not\in \Vec(\mathcal{L})$ (car sinon, $\mathcal{G} \subset \Vect(\mathcal{L})$ et donc $\underbrace{\Vect(\mathcal{G})}_{= E} \subset  \underbrace{\Vect(\mathcal{L})}_{ \subset E}$\\
		Donc $\mathcal{L} \cup \{u\} $ est libre. On pose $\mathcal{L}' = \mathcal{L} \cup \{u\} $ \[
			\mathcal{G}\setminus \mathcal{L}' = \mathcal{G}\setminus (\mathcal{L} \cup \{u\}) = (\mathcal{G}\setminus\mathcal{L})\setminus \{u\} 
		\] donc $\#(\mathcal{G}\setminus\mathcal{L}') = n+1 -1 = n$\\
		D'après l'hypothèse de récurrence, il existe $\mathcal{B}$ une base de $E$ telle que \[
			\mathcal{L} \subset  \mathcal{L}' \subset \mathcal{B} \text{ et } \mathcal{B}\setminus \mathcal{L}' \subset \mathcal{G}
		\] \[
			\mathcal{B} \setminus \mathcal{L} = \underbrace{\mathcal{B}\setminus\mathcal{L}'}_{\subset \mathcal{G}} \cup \underbrace{\{u\}}_{\subset \mathcal{G} \text{ car } u \in \mathcal{G}}
		\] On a $\mathcal{B}\setminus\mathcal{L}\subset \mathcal{G}$
	\end{itemize}
\end{prv}

\begin{thm}
	Soit $E$ un $\mathbbm{K}$-espace vectoriel de dimension finie. Toutes les bases de $E$ ont le même cardinal.
\end{thm}

\begin{prv}
	Soit $\mathcal{G}$ une famille génératrice finie de $E$ et $\mathcal{B} \subset  \mathcal{G}$ une base de $E$. On note $n = \#\mathcal{B}$ \\
	Soit $\mathcal{B}'$ une base de $E$. On pose $p = n - \#(\mathcal{B} \cap  \mathcal{B}')$. Montrons par récurrence sur  $p$ que $\#\mathcal{B} = \#\mathcal{B}'$ 
	\begin{itemize}
		\item On suppose que $p = 0$. Alors, $\#(\mathcal{B} \cap \mathcal{B}') = n$ \\
			Or, $\mathcal{B}' \cap \mathcal{B} \subset \mathcal{B}$ donc $\mathcal{B} \cap \mathcal{B}' = \mathcal{B}$ donc $\mathcal{B} \subset  \mathcal{B}'$ et donc $\mathcal{B} = \mathcal{B}'$ 
		\item Soit $p \in \N$. On suppose que si $\mathcal{B}'$ est une base de $E$ telle que $n - \#(\mathcal{B} \cap \mathcal{B}') = p$, alors $\#\mathcal{B}' = n$ \\
			Aoit $\mathcal{B}'$ une base de $E$ telle que $n - \#(\mathcal{B}\cap \mathcal{B}') = p+1 > 0$ \\
			Donc $\mathcal{B} \cap \mathcal{B}' \neq \mathcal{B}$. Soit $u \in \mathcal{B}' \setminus \mathcal{B}$. D'après le lemme d'échange, il existe $v \in \mathcal{B}\setminus \mathcal{B}'$ tel que $\mathcal{B}' \setminus \{u\} \cup \{v\}$ est une base de $E$. On pose $\mathcal{B}'' = \mathcal{B}' \setminus \{u\} \cup \{v\}$ 
			\begin{align*}
				\mathcal{B}'' \cap \mathcal{B} &= \left( (\mathcal{B}' \setminus \{u\})  \cap \mathcal{B} \right) \cup \{v\} \\
				&= (\mathcal{B}' \cap \mathcal{B}) \cup \{v\} \\
			\end{align*}
			donc,
			\begin{align*}
				n - \#(\mathcal{B}'' \cap \mathcal{B}) &= n - (\#(\mathcal{B}' \cap \mathcal{B}) + 1) \\
				&= p+1- 1 \\
				&= p \\
			\end{align*}
			D'après l'hypothèse de récurrence, \[
				\#\mathcal{B}'' = n
			\] Or, $\#\mathcal{B}'' = \#\mathcal{B}'$
	\end{itemize}
\end{prv}

\begin{lem}
	Soient $\mathcal{B}$ et $\mathcal{B}'$ deux bases de $E$ telles que $\mathcal{B}\subset \mathcal{B}'$. Alors, $\mathcal{B} = \mathcal{B}'$.
\end{lem}

\begin{prv}
	On suppose $\mathcal{B}' \neq \mathcal{B}$. Soit $u \in \mathcal{B}' \setminus \mathcal{B}$
	$u \in E = \Vect(\mathcal{B})$ donc $\mathcal{B} \cup \{u\}$ n'est pas libre.
	Donc $\mathcal{B}\cup \{u\} \subset \mathcal{B}'$ et $\mathcal{B}'$ est libre donc $\mathcal{B}\cup \{u\}$ est libre: une contradiction $\lightning$
\end{prv}

\begin{lem}
	[Lemme d'échange] Soient $\mathcal{B}_1$ et $\mathcal{B}_2$ deux bases de $E$ et $u \in \mathcal{B}_1 \setminus \mathcal{B}_2$. Alors, il existe $v \in \mathcal{B}_2$ tel que $(\mathcal{B}_1 \setminus \{u\}) \cup \{v\}$ soit une base de $E$.
\end{lem}

\begin{prv}
	[1${}^\text{nde}$ méthode]
	On suppose que pout tout $v \in \mathcal{B}_2$, $(\mathcal{B}_1\setminus \{u\}) \cup \{v\}$ n'est pas une base de $E$
	Soit $v \in \mathcal{B}_2$.
	\begin{itemize}
		\item Supposons $(\mathcal{B}_1\setminus \{u\})\cup \{v\}$ non libre. $\mathcal{B}_1 \setminus \{u\}$ est libre. Donc $v \in \Vect(\mathcal{B}_1 \setminus \{u\})$
		\item Supposons $(\mathcal{B}_1\setminus \{u\}) \cup \{v\}$ non génératrice.
			Comme $\mathcal{B}_1$ engendre $E$, $u \not\in \Vect(\mathcal{B}_1\setminus \{v\})$.
			On suppose que $\mathcal{B}_1 \neq \mathcal{B}_2$.
			$\forall v \in \mathcal{B}_2 \setminus \mathcal{B}_1, \Vect(\mathcal{B}_1 \setminus \{v\}) = \Vect(\mathcal{B}_1) = E \ni u$ 
			donc, $(\mathcal{B}_1\setminus \{u\}) \cup \{v\}$ engendre $E$ et donc \[
				v \in \Vect(\mathcal{B}_1 \setminus \{u\})
			\] On a aussi \[
				\forall v \in \mathcal{B}_1 \setminus \{u\}, v \in \Vect(\mathcal{B}_1\setminus \{u\})
			\] Comme $u \not\in \mathcal{B}_2$, on a \[
				\forall v \in \mathcal{B}_2, v \in \Vect(\mathcal{B}_1\setminus \{u\})
			\] docn \[
				E = \Vect(\mathcal{B}_2) \subset \Vect(\mathcal{B}_1\setminus \{u\})
			\] donc $\mathcal{B}_1\setminus \{u\}$ engendre $E$ donc $\mathcal{B}_1\setminus \{u\}$ est une base de $E$. Or, $\mathcal{B}_1 \setminus \{u\}  \subset  \mathcal{B}_1$, donc $\mathcal{B}_1\setminus \{u\} = \mathcal{B}_1$
	\end{itemize}
\end{prv}

\begin{prv}
	[2${}^\text{nde}$ méthode]
	On suppose que pout tout $v \in \mathcal{B}_2$, $(\mathcal{B}_1\setminus \{u\}) \cup \{v\}$ n'est pas une base de $E$
	\begin{itemize}
		\item Comme $u \in \mathcal{B}_1 \setminus \mathcal{B}_2$, nécéssairement $\mathcal{B}_1 \neq \mathcal{B}_2$ donc $\mathcal{B}_2 \not\subset \mathcal{B}_1$, donc $\mathcal{B}_2\setminus\mathcal{B}_1 \neq \O$ 
		\item Soit $v \in \mathcal{B}_2\setminus\mathcal{B}_1$. Il existe $(\lambda_w)_{w\in\mathcal{B}_1}$ une famille de scalaires presque nulle telle que \[
				v = \sum_{w \in \mathcal{B}_1} \lambda_w w - \lambda_u u + + \sum_{w \in \mathcal{B}_1\setminus \{u\}}\lambda_w w
			\]
			Si $\lambda_u \neq 0_E$, alors
			\begin{align*}
				u &= \lambda_u^{-1}\left( v - \sum_{w \in \mathcal{B}_1 \setminus \{u\}} \lambda_w w \right)\\
					&\in \Vect(\mathcal{B}_1\setminus \{u\} \cup v)
			\end{align*}
			 donc $\mathcal{B}_1 \subset \Vect(\mathcal{B}_1\setminus \{u\} \cup \{v\})$\\
			 et donc $E \subset  \Vect(\mathcal{B}_1 \setminus \{u\} \cup \{v\})$ \\
			 et donc $\mathcal{B}_1 \setminus \{u\} \cup \{v\}$ engendre $E$ \\
			 donc $\mathcal{B}_1 \setminus \{u\} \cup \{v\}$ n'est pas libre\\
			 donc $v \in \Vect(\mathcal{B}_1\setminus \{u\})$ (car $\mathcal{B}_1 \setminus \{u\}$ est libre\\
			 donc $\lambda_u = 0_\mathbbm{K}$ $\lightning$\\`

			 Donc, $\lambda_u = 0_\mathbbm{K}$, docn $v \in \Vect(\mathcal{B}_1\setminus \{u\})$ \\
			 On vient de prouver que
			 \begin{align*}
			 	\mathcal{B}_2 \setminus \mathcal{B}_1 \subset \Vect(\mathcal{B}_1 \setminus \{u\})\\
			 	\mathcal{B}_1 \setminus \{u\} \subset \Vect(\mathcal{B}_1 \setminus \{u\})\\
			 \end{align*}
			 Comme $u \not\in \mathcal{B}_2$, \[
			 	\mathcal{B}_2 \subset \Vect(\mathcal{B}_1 \setminus \{u\})
			 \] donc \[
			 	E = \Vect(\mathcal{B}_2) \subset  \Vect(\mathcal{B}_1 \setminus \{u\})
			 \] donc $\mathcal{B}_1 \setminus \{u\}$ engendre $E$. Donc,  $\mathcal{B}_1 \setminus \{u\}$ est une base de $E$.\\
			 Or, $\mathcal{B}_1 \setminus \{u\} \subset  \mathcal{B}_1$, donc $\mathcal{B}_1 \setminus \{u\} = \mathcal{B}_1$
	\end{itemize}
\end{prv}

\begin{defn}
	Soit $E$ un $\mathbbm{K}$-espace vectoriel de dimension finie. Le cardinal commun à toutes les bases de $E$ est appelé \underline{dimension} de $E$ est notée $\dim(E)$ ou $\dim_\mathbbm{K}(E)$\\
	C'est donc aussi le nombre de coordonnées de n'importe quel vecteur dans n'importe quelle base.
	\index{dimension (espace vectoriel)}
\end{defn}

\begin{exm}
	\begin{enumerate}
		\item $\dim_\R(\C) = 2$ et $\dim_\C(\C) = 1$ 
		\item $\dim_\mathbbm{K}(\mathbbm{K}^{n}) = n$ 
		\item $\dim_{\mathbbm{K}}(\mathcal{M}_{n,p}(\mathbbm{K})) = np$
	\end{enumerate}
\end{exm}

\begin{crlr}
	Soit $E$ un $\mathbbm{K}$-espace vectoriel de dimension finie, $\mathcal{L}$ une famille libre de $E$, $\mathcal{G}$ une famille génératrice de $E$. On note $n = \dim(E)$
	\begin{enumerate}
		\item $\#\mathcal{G} \ge n$ et $(\#\mathcal{G} = n \implies \mathcal{G} \text{ est une base de } E$)
		\item $\#\mathcal{L} \le n$ et $(\#\mathcal{L} = n \implies \mathcal{L} \text{ est une base de } E$)
	\end{enumerate}
\end{crlr}

\begin{crlr}
	$\R^{\R}$ est de dimension infinie.
	$\forall i \in \N, e_i: x \mapsto x^i$\\
	$(e_i)_{i\in\N}$ est libre dans $\R^\R$
\end{crlr}

\begin{prop}
	Soient $E$ et $F$ deux $\mathbbm{K}$-espaces vectoriels de dimension finie. Alors $E\times F$ est de dimension finie et $\dim(E\times F) = \dim(E) + \dim(F)$
\end{prop}

\begin{prv}
	Soit $(e_1,\ldots, e_n)$ une base de $E$, $(f_1, \ldots, f_p)$ une base de $F$.
	On pose \[
		\left\{\begin{array}
			{r c l}
			u_1 &=& (e_1,0_F)\\
			u_2 &=& (e_2,0_F)\\
					&\vdots&\\
			u_n &=& (e_n,0_F)\\
			u_{n+1} &=& (0_E, f_1)\\
			u_{n+2} &=& (0_E, f_2)\\
					&\vdots&\\
			u_{n+p} &=& (0_E,f_p)\\
		\end{array}\right.
	\]
	Soit $(x,y) \in E\times F$. \[
		\begin{cases}
			\exists (x_1,\ldots,x_n)\in \mathbbm{K}^n, x = \sum_{i=1}^{n} x_ie_i
			\exists (y_1,\ldots,y_n)\in \mathbbm{K}^n, x = \sum_{j=1}^{p} y_jf_j
		\end{cases}
	\] 
	\begin{align*}
		(x,y) &= \left( \sum_{i=1}^{n} x_ie_i, \sum_{i=1}^{p} y_jf_j \right)  \\
		&= \sum_{i=1}^{n} x_i (e_i + 0_F) + \sum_{j=1}^{p} y_j (0_E, f_j) \\
		&= \sum_{i=1}^{n} x_i u_i + \sum_{j=1}^{p} y_j u_{n+j} \\
	\end{align*}
	Donc, $E\times F = \Vect(u_1, \ldots, u_{n+p})$ donc $E\times F$ est de dimension finie.\\
	Soit $(\lambda_1, \ldots, \lambda_{n+p}) \in \mathbbm{K}^{n+p}$ tel que \[
		(*): \quad \sum_{k=1}^{n+p} \lambda_ku_k = 0_{E\times F} = (0_E, 0_F)
	\]
	\begin{align*}
		(*) &\iff \sum_{k=1}^{n} \lambda_k (e_k, 0_F) + \sum_{k=n+1}^{p} \lambda_k(0_E, f_{k-n}) = (0_E, 0_F)\\
				&\iff \begin{cases}
					\sum_{k=1}^{n} \lambda_k e_k = 0_E\\
					\sum_{k=n+1}^{p} \lambda_k f_{k-n} = 0_F
				\end{cases}\\
				&\iff \begin{cases}
					\forall k \in \left\llbracket 1,n \right\rrbracket, \lambda_k = 0_\mathbbm{K} \qquad&(\text{car $(e_1,\ldots,e_n)$ est libre})\\
					\forall k \in \left\llbracket n+1,n+p \right\rrbracket, \lambda_k = 0_\mathbbm{K} \qquad&(\text{car $(f_1,\ldots,f_n)$ est libre})\\
				\end{cases}
	\end{align*}
	Donc $(u_1, \ldots, u_{n+p})$ est une base de $E\times F$. Donc, $\dim(E\times F) = n + p = \dim(E) + \dim(F)$
\end{prv}

\begin{rmk}
	[Convention]
	\[\dim\big(\{0_E\}\big) = 0\]
\end{rmk}

\begin{thm}
	Soit $E$ un $\mathbbm{K}$-espace vectoriel de dimension finie, $F$ un sous-espace vectoriel de $E$. Alors, $F$ est de dimension finie et  $\dim(F) \le \dim(E)$\\
	Si $\dim(F) = \dim(E)$, alors $F = E$
\end{thm}

\begin{prv}
	On considère \[
		A = \{k \in \N \mid \text{il existe une famille libre de $F$ à $k$ éléments}\} 
	\]
	On suppose $F \neq \{0_E\}$.
	\begin{itemize}
		\item Soit $u \in F\setminus \{0_E\}$. $(u)$ est libre donc $1 \in A$ et donc $A \neq \O$
		\item Soit $\mathcal{L}$ une famille libre de $F$. Alors, $\mathcal{L}$ est une famille libre de $E$ \\
			donc $\#\mathcal{L} \le \dim(E)$\\
			Donc $A$ est majorée par $\dim(E)$ \\
			On en déduit que $A$ a un plus grand élément $p$.
		\item Soit $\mathcal{L}$ une famille libre de $F$ avec $p$ éléments.\\
			Si $\mathcal{L}$ n'engendre pas $F$, alors il existe $u\in F$ tel que $u\not\in \Vect(\mathcal{L})$ et donc $\mathcal{L} \cup \{u\}$ est une famille libre de $F$, donc $p+1 \in A$ en contradiction avec la maximalité de $p$.\\
			Donc $\mathcal{L}$ est une base de $F$ donc $F$ est de dimension finie et $\dim(F) = p \le \dim(E)$\\
	\end{itemize}

	Soit $\mathcal{B}$ une base de $F$. Alors, $\mathcal{B}$ est aussi une famille de libre de de $E$. Donc $\#\mathcal{B} \le \dim(E)$ donc $\dim(F) = \dim(E)$ \\
	Si $\dim(F) = \dim(E)$, alors $\mathcal{B}$ est une base de $E$, et donc $F = \Vect(\mathcal{B}) = E$
\end{prv}

\begin{prop}
	[Formule de Grassmann]
	Soit $E$ un $\mathbbm{K}$-espace vectoriel de dimension finie, $F$ et $G$ deux sous-espace vectoriels de $E$. Alors, \[
		\dim(F+G) = \dim(F) + \dim(G) - \dim(F\cap G)
	\] 
\end{prop}

\begin{prv}
	Soit $(e_1, \ldots, e_p)$ une base de $F\cap G$. $(e_1,\ldots,e_p)$ est une famille libre de $F$.\\
	On complète $(e_1, \ldots, e_p)$ en une base $(e_1, \ldots, e_p, u_1, \ldots, u_q)$ de $F$.\\
	De même, on complète $(e_1, \ldots, e_p)$ en une base $(e_1, \ldots, e_p, v_1, \ldots, v_r)$ de $G$.\\
	On pose  $\mathcal{B} = (e_1, \ldots, e_p, u_1, \ldots, u_q, v_1, \ldots, v_r)$. Montrons que $\mathcal{B}$ est une base de $F+G$
	\begin{itemize}
		\item Soit $u \in F+G$ \\
			On pose $u = v+w$ avec $\begin{cases}
				v\in F\\
				w \in G
			\end{cases}$.\\
			On pose $v = \sum_{i=1}^p \lambda_i e_i + \sum_{i=1}^q \mu_i u_i$ avec $(\lambda_1, \ldots, \lambda_p, \mu_1, \ldots, \lambda_q) \in \mathbbm{K}^{p+q}$\\
			On pose aussi $w = \sum_{i = 1}^p \lambda'_ie_i + \sum_{j=1}^r \nu_j v_j$ avec $(\lambda_1',\ldots,\lambda_p', \nu_1, \ldots, \nu_r) \in \mathbbm{K}^{p+r}$\\
			D'où, \[
				u = \sum_{i=1}^p (\lambda_i + \lambda'_i)e_i + \sum_{j=1}^q \mu_j u_j + \sum_{k=1}^r \nu_k v_k \in \Vect(\mathcal{B})
			\]
		\item Soient $(\lambda_1, \ldots, \lambda_p, \mu_1, \ldots, \mu_q, \nu_1, \ldots, \nu_r) \in \mathbbm{K}^{p+q+r}$.\\
			On suppose \[
				(*)\quad \sum_{i=1}^{p}\lambda_ie_i + \sum_{j=1}^q\mu_ju_j + \sum_{k=1}^r \nu_k v_k = 0_E
			\] 
			D'où, \[
				\underbrace{\sum_{i=1}^p\lambda_i e_i + \sum_{j=1}^q \mu_ju_j}_{\in F} = \underbrace{-\sum_{k=1}^r\nu_jv_k}_{\in G}
			\] 
			Donc, \[
				f = \sum_{i=1}^p \lambda_i e_i + \sum_{j=1}^q \mu_j u_j \in F\cap G
			\] Comme $(e_1, \ldots, e_p)$ est une base de $F\cap G$, $\exists ! (\lambda_1', \ldots, \lambda_p') \in \mathbbm{K}^p$ tel que \[
				f = \sum_{i=1}^p \lambda'_i e_i = \sum_{i=1}^p \lambda'_i e_i + \sum_{j=1}^q 0_\mathbbm{K}u_j
			\] Comme $(e_1, \ldots, e_p, u_1, \ldots, u_q)$ est une base de $F$, \[
				\forall k \in \left\llbracket 1, q \right\rrbracket, \mu_j = 0_\mathbbm{K}
			\] De même, \[
				\forall k \in \left\llbracket 1,r \right\rrbracket , \nu_k = 0_\mathbbm{K}
			\] On remplace dans $(*)$ pour trouver \[
				\sum_{i=1}^p \lambda_ie_i = 0_E
			\] Comme $(e_1, \ldots, e_p)$ est libre, \[
				\forall i \in \left\llbracket 1,p \right\rrbracket, \lambda_i = 0_\mathbbm{K}
			\] Donc $\mathcal{B}$ est libre.\\
			Donc, 
			\begin{align*}
				\dim(F+G) &=  p +q + r \\
				&= (p+q)+ (p+r) - p \\
				&= \dim(F) + \dim(G) - \dim(F\cap G) \\
			\end{align*}
	\end{itemize}
\end{prv}

\begin{crlr}
	Avec les hypothèse précédentes, \[
		E = F \oplus G \iff \begin{cases}
			F \cap  G = \{0_E\} \\
			\dim(E) = \dim(F) + \dim(G)
		\end{cases}
	\] 
\end{crlr}

\begin{prv}
	\begin{itemize}
		\item[``$\implies$''] On suppose $E = F \oplus G$ \\
			Comme la somme est directe, $F \cap G = \{0_E\}$ 
			\begin{align*}
				\dim(E) &= \dim(F)\\
				&= \dim(F) + \dim(G) - \dim(F\cap G)\\
				&= \dim(F) + \dim(G)\\
			\end{align*}
		\item[``$\impliedby$''] On suppose $F\cap G = \{0_E\}$ et $\dim(E) = \dim(F) + \dim(G)$.\\
			On sait déjà que $F+G = F \oplus G$\\
			 \begin{align*}
				\dim(F+G) = \dim(F) + \dim(G) - \dim(F \cap G) = \dim(E)
			\end{align*}
			Donc $F + G = E$
	\end{itemize}
\end{prv}

\begin{prop}
	Soit $F$ un $\mathbbm{K}$-espace vectoriel de dimension finie $n$. Soit $\mathcal{B} = (e_1, \ldots, e_n)$ une base de $F$. L'application
	\begin{align*}
		f: \mathbbm{K}^n &\longrightarrow F \\
		(\lambda_1, \ldots, \lambda_n) &\longmapsto \sum_{i=1}^n \lambda_i e_i
	\end{align*} est bijective.\\
	Si $\mathbbm{K}$ est infini, $\mathbbm{K}^n$ aussi et donc $F$ aussi.\\
	Si $\#\mathbbm{K} = p \in \N_*$,
	\begin{align*}
		\#&\mathbbm{K}^n = p^n\\
		&\vrt=\\
		\#&F
	\end{align*}
\end{prop}


	}

	{
		\chap[08]{Ensembles, applications, relations et lois de composition}
		\renewcommand{\cwd}{../chap08}
		\begin{defn}
	Soit $E$ un $\mathbbm{K}$-espace vectoriel. On dit que $E$ est de \underline{dimension finie} si $E$ a au moins une famille génératrice finie. On dit que $E$ est de \underline{dimension infinie} sinon.
	\index{dimension finie (espace vectoriel)}
	\index{dimension infinie (espace vectoriel)}
\end{defn}

\begin{thm}
	[Théorème de la base extraite]
	Soit $E$ un $\mathbbm{K}$-espace vectoriel non nul de dimension finie. Soit $\mathcal{G}$ une famille génératrice finie de $E$. Alors, il existe une base $\mathcal{B}$ de $\mathcal{E}$ telle que $\mathcal{B} \subset \mathcal{G}$.
\end{thm}

\begin{prv}
	[par récurrence sur $\#G = \Card(G)$]
	\begin{itemize}
		\item Soit $E$ un $\mathbbm{K}$-espace vectoriel non nul engendré par $\mathcal{G} = (u)$.\\
			Si $u = 0_E$, alors $E = \{0_E\}$: une contradiction $\lightning$ \\
			Donc $u \neq 0_E$ donc $(u)$ est libre. En effet, \[
				\forall \lambda \in \mathbbm{K}, \lambda u = 0_E \implies \lambda = 0_\mathbbm{K}
			\] Donc $\mathcal{G}$ est une base de $E$.\\
		\item Soit $n \in \N_*$. Soit $E$ un $\mathbbm{K}$-espace vectoriel. On suppose que si $E$ a une famille génératrice constituée de $n$ vecteurs, alors on peut extraire de cette famille une base de $E$.\\
			Soit $\mathcal{G}$ une famille génératrice de $E$ avec $n+1$ vecteurs.\\
			Si $\mathcal{G}$ est libre, alors $\mathcal{G}$ est une base de $E$. \\
			Si $\mathcal{G}$ n'est pas libre, alors il existe $u \in \mathcal{G}$ tel que $u \in \Vect(\mathcal{G}\setminus \{u\})$ \\
			Donc $\mathcal{G}\setminus \{u\}$ engendre $E$. Or, $\mathcal{G}\setminus \{u\}$ possède $n$ vecteurs. D'après l'hypothèse de récurrence, il existe une base $\mathcal{B}$ de $E$ telle que \[
				\mathcal{B} \subset \mathcal{G} \setminus \{u\} \subset \mathcal{G}
			\] 
	\end{itemize}
\end{prv}

\begin{crlr}
	Tout espace de dimension finie a une base.
	\qed
\end{crlr}

\begin{thm}
	[Théorème de la base incomplète]
	Soit $E$ un $\mathbbm{K}$-espace vectoriel de dimension finie, $\mathcal{G}$ une famille génératrice finie de $E$. $\mathcal{L}$ une famille libre de $E$. Alors, il existe une base $\mathcal{B}$ de $E$ telle que \[
		\mathcal{L} \subset \mathcal{B} \text{ et } \mathcal{B}\setminus \mathcal{L} \subset \mathcal{G}
	\] 
\end{thm}

\begin{prv}
	[par récurrence sur $\#(\mathcal{G}\setminus\mathcal{L})$]
	\begin{itemize}
		\item Avec les notations précédentes, on suppose que $\mathcal{G}\setminus\mathcal{L} \neq \O$ \[
				\forall u \in \mathcal{G}, u \in \mathcal{L}
			\] Donc $\mathcal{G} \subset \mathcal{L}$ donc $\mathcal{L}$ est génératrice donc $\mathcal{L}$ est une base de $E$. On pose $\mathcal{B} = \mathcal{L}$ et alors \[
				\mathcal{L} \subset  \mathcal{B} \text{ et } \mathcal{B}\setminus\mathcal{L} = \O \subset  \mathcal{G}
			\] 
		\item Soit $n \in \N$. On suppose que si $\mathcal{G}$ est génératrice et $\mathcal{L}$ libre avec $\#(\mathcal{G}\setminus\mathcal{L}) = n$ alors il existe une base $\mathcal{B}$ de $E$ telle que \[
			\mathcal{L}\subset \mathcal{B} \text{ et } \mathcal{B}\setminus\mathcal{L}\subset \mathcal{G}
		\] Soient à présent $\mathcal{G}$ une famille génératrice de $E$ et $\mathcal{L}$ une famille libre de $E$ telles que $\#(\mathcal{G}\setminus\mathcal{L}) = n+1 > 0$\\
		Si $\mathcal{L}$ engendre $E$, alors $\mathcal{L}$ est une base de $E$. On pose $\mathcal{B} = \mathcal{L}$ et on a bien \[
			\mathcal{L} \subset  \mathcal{B} \text{ et } \mathcal{B} \setminus \mathcal{L} = \O \subset  \mathcal{G}
		\] On suppose que $\mathcal{L}$ n'engendre pas $E$. Il existe $u \in \mathcal{G}$ tel que $u \not\in \Vec(\mathcal{L})$ (car sinon, $\mathcal{G} \subset \Vect(\mathcal{L})$ et donc $\underbrace{\Vect(\mathcal{G})}_{= E} \subset  \underbrace{\Vect(\mathcal{L})}_{ \subset E}$\\
		Donc $\mathcal{L} \cup \{u\} $ est libre. On pose $\mathcal{L}' = \mathcal{L} \cup \{u\} $ \[
			\mathcal{G}\setminus \mathcal{L}' = \mathcal{G}\setminus (\mathcal{L} \cup \{u\}) = (\mathcal{G}\setminus\mathcal{L})\setminus \{u\} 
		\] donc $\#(\mathcal{G}\setminus\mathcal{L}') = n+1 -1 = n$\\
		D'après l'hypothèse de récurrence, il existe $\mathcal{B}$ une base de $E$ telle que \[
			\mathcal{L} \subset  \mathcal{L}' \subset \mathcal{B} \text{ et } \mathcal{B}\setminus \mathcal{L}' \subset \mathcal{G}
		\] \[
			\mathcal{B} \setminus \mathcal{L} = \underbrace{\mathcal{B}\setminus\mathcal{L}'}_{\subset \mathcal{G}} \cup \underbrace{\{u\}}_{\subset \mathcal{G} \text{ car } u \in \mathcal{G}}
		\] On a $\mathcal{B}\setminus\mathcal{L}\subset \mathcal{G}$
	\end{itemize}
\end{prv}

\begin{thm}
	Soit $E$ un $\mathbbm{K}$-espace vectoriel de dimension finie. Toutes les bases de $E$ ont le même cardinal.
\end{thm}

\begin{prv}
	Soit $\mathcal{G}$ une famille génératrice finie de $E$ et $\mathcal{B} \subset  \mathcal{G}$ une base de $E$. On note $n = \#\mathcal{B}$ \\
	Soit $\mathcal{B}'$ une base de $E$. On pose $p = n - \#(\mathcal{B} \cap  \mathcal{B}')$. Montrons par récurrence sur  $p$ que $\#\mathcal{B} = \#\mathcal{B}'$ 
	\begin{itemize}
		\item On suppose que $p = 0$. Alors, $\#(\mathcal{B} \cap \mathcal{B}') = n$ \\
			Or, $\mathcal{B}' \cap \mathcal{B} \subset \mathcal{B}$ donc $\mathcal{B} \cap \mathcal{B}' = \mathcal{B}$ donc $\mathcal{B} \subset  \mathcal{B}'$ et donc $\mathcal{B} = \mathcal{B}'$ 
		\item Soit $p \in \N$. On suppose que si $\mathcal{B}'$ est une base de $E$ telle que $n - \#(\mathcal{B} \cap \mathcal{B}') = p$, alors $\#\mathcal{B}' = n$ \\
			Aoit $\mathcal{B}'$ une base de $E$ telle que $n - \#(\mathcal{B}\cap \mathcal{B}') = p+1 > 0$ \\
			Donc $\mathcal{B} \cap \mathcal{B}' \neq \mathcal{B}$. Soit $u \in \mathcal{B}' \setminus \mathcal{B}$. D'après le lemme d'échange, il existe $v \in \mathcal{B}\setminus \mathcal{B}'$ tel que $\mathcal{B}' \setminus \{u\} \cup \{v\}$ est une base de $E$. On pose $\mathcal{B}'' = \mathcal{B}' \setminus \{u\} \cup \{v\}$ 
			\begin{align*}
				\mathcal{B}'' \cap \mathcal{B} &= \left( (\mathcal{B}' \setminus \{u\})  \cap \mathcal{B} \right) \cup \{v\} \\
				&= (\mathcal{B}' \cap \mathcal{B}) \cup \{v\} \\
			\end{align*}
			donc,
			\begin{align*}
				n - \#(\mathcal{B}'' \cap \mathcal{B}) &= n - (\#(\mathcal{B}' \cap \mathcal{B}) + 1) \\
				&= p+1- 1 \\
				&= p \\
			\end{align*}
			D'après l'hypothèse de récurrence, \[
				\#\mathcal{B}'' = n
			\] Or, $\#\mathcal{B}'' = \#\mathcal{B}'$
	\end{itemize}
\end{prv}

\begin{lem}
	Soient $\mathcal{B}$ et $\mathcal{B}'$ deux bases de $E$ telles que $\mathcal{B}\subset \mathcal{B}'$. Alors, $\mathcal{B} = \mathcal{B}'$.
\end{lem}

\begin{prv}
	On suppose $\mathcal{B}' \neq \mathcal{B}$. Soit $u \in \mathcal{B}' \setminus \mathcal{B}$
	$u \in E = \Vect(\mathcal{B})$ donc $\mathcal{B} \cup \{u\}$ n'est pas libre.
	Donc $\mathcal{B}\cup \{u\} \subset \mathcal{B}'$ et $\mathcal{B}'$ est libre donc $\mathcal{B}\cup \{u\}$ est libre: une contradiction $\lightning$
\end{prv}

\begin{lem}
	[Lemme d'échange] Soient $\mathcal{B}_1$ et $\mathcal{B}_2$ deux bases de $E$ et $u \in \mathcal{B}_1 \setminus \mathcal{B}_2$. Alors, il existe $v \in \mathcal{B}_2$ tel que $(\mathcal{B}_1 \setminus \{u\}) \cup \{v\}$ soit une base de $E$.
\end{lem}

\begin{prv}
	[1${}^\text{nde}$ méthode]
	On suppose que pout tout $v \in \mathcal{B}_2$, $(\mathcal{B}_1\setminus \{u\}) \cup \{v\}$ n'est pas une base de $E$
	Soit $v \in \mathcal{B}_2$.
	\begin{itemize}
		\item Supposons $(\mathcal{B}_1\setminus \{u\})\cup \{v\}$ non libre. $\mathcal{B}_1 \setminus \{u\}$ est libre. Donc $v \in \Vect(\mathcal{B}_1 \setminus \{u\})$
		\item Supposons $(\mathcal{B}_1\setminus \{u\}) \cup \{v\}$ non génératrice.
			Comme $\mathcal{B}_1$ engendre $E$, $u \not\in \Vect(\mathcal{B}_1\setminus \{v\})$.
			On suppose que $\mathcal{B}_1 \neq \mathcal{B}_2$.
			$\forall v \in \mathcal{B}_2 \setminus \mathcal{B}_1, \Vect(\mathcal{B}_1 \setminus \{v\}) = \Vect(\mathcal{B}_1) = E \ni u$ 
			donc, $(\mathcal{B}_1\setminus \{u\}) \cup \{v\}$ engendre $E$ et donc \[
				v \in \Vect(\mathcal{B}_1 \setminus \{u\})
			\] On a aussi \[
				\forall v \in \mathcal{B}_1 \setminus \{u\}, v \in \Vect(\mathcal{B}_1\setminus \{u\})
			\] Comme $u \not\in \mathcal{B}_2$, on a \[
				\forall v \in \mathcal{B}_2, v \in \Vect(\mathcal{B}_1\setminus \{u\})
			\] docn \[
				E = \Vect(\mathcal{B}_2) \subset \Vect(\mathcal{B}_1\setminus \{u\})
			\] donc $\mathcal{B}_1\setminus \{u\}$ engendre $E$ donc $\mathcal{B}_1\setminus \{u\}$ est une base de $E$. Or, $\mathcal{B}_1 \setminus \{u\}  \subset  \mathcal{B}_1$, donc $\mathcal{B}_1\setminus \{u\} = \mathcal{B}_1$
	\end{itemize}
\end{prv}

\begin{prv}
	[2${}^\text{nde}$ méthode]
	On suppose que pout tout $v \in \mathcal{B}_2$, $(\mathcal{B}_1\setminus \{u\}) \cup \{v\}$ n'est pas une base de $E$
	\begin{itemize}
		\item Comme $u \in \mathcal{B}_1 \setminus \mathcal{B}_2$, nécéssairement $\mathcal{B}_1 \neq \mathcal{B}_2$ donc $\mathcal{B}_2 \not\subset \mathcal{B}_1$, donc $\mathcal{B}_2\setminus\mathcal{B}_1 \neq \O$ 
		\item Soit $v \in \mathcal{B}_2\setminus\mathcal{B}_1$. Il existe $(\lambda_w)_{w\in\mathcal{B}_1}$ une famille de scalaires presque nulle telle que \[
				v = \sum_{w \in \mathcal{B}_1} \lambda_w w - \lambda_u u + + \sum_{w \in \mathcal{B}_1\setminus \{u\}}\lambda_w w
			\]
			Si $\lambda_u \neq 0_E$, alors
			\begin{align*}
				u &= \lambda_u^{-1}\left( v - \sum_{w \in \mathcal{B}_1 \setminus \{u\}} \lambda_w w \right)\\
					&\in \Vect(\mathcal{B}_1\setminus \{u\} \cup v)
			\end{align*}
			 donc $\mathcal{B}_1 \subset \Vect(\mathcal{B}_1\setminus \{u\} \cup \{v\})$\\
			 et donc $E \subset  \Vect(\mathcal{B}_1 \setminus \{u\} \cup \{v\})$ \\
			 et donc $\mathcal{B}_1 \setminus \{u\} \cup \{v\}$ engendre $E$ \\
			 donc $\mathcal{B}_1 \setminus \{u\} \cup \{v\}$ n'est pas libre\\
			 donc $v \in \Vect(\mathcal{B}_1\setminus \{u\})$ (car $\mathcal{B}_1 \setminus \{u\}$ est libre\\
			 donc $\lambda_u = 0_\mathbbm{K}$ $\lightning$\\`

			 Donc, $\lambda_u = 0_\mathbbm{K}$, docn $v \in \Vect(\mathcal{B}_1\setminus \{u\})$ \\
			 On vient de prouver que
			 \begin{align*}
			 	\mathcal{B}_2 \setminus \mathcal{B}_1 \subset \Vect(\mathcal{B}_1 \setminus \{u\})\\
			 	\mathcal{B}_1 \setminus \{u\} \subset \Vect(\mathcal{B}_1 \setminus \{u\})\\
			 \end{align*}
			 Comme $u \not\in \mathcal{B}_2$, \[
			 	\mathcal{B}_2 \subset \Vect(\mathcal{B}_1 \setminus \{u\})
			 \] donc \[
			 	E = \Vect(\mathcal{B}_2) \subset  \Vect(\mathcal{B}_1 \setminus \{u\})
			 \] donc $\mathcal{B}_1 \setminus \{u\}$ engendre $E$. Donc,  $\mathcal{B}_1 \setminus \{u\}$ est une base de $E$.\\
			 Or, $\mathcal{B}_1 \setminus \{u\} \subset  \mathcal{B}_1$, donc $\mathcal{B}_1 \setminus \{u\} = \mathcal{B}_1$
	\end{itemize}
\end{prv}

\begin{defn}
	Soit $E$ un $\mathbbm{K}$-espace vectoriel de dimension finie. Le cardinal commun à toutes les bases de $E$ est appelé \underline{dimension} de $E$ est notée $\dim(E)$ ou $\dim_\mathbbm{K}(E)$\\
	C'est donc aussi le nombre de coordonnées de n'importe quel vecteur dans n'importe quelle base.
	\index{dimension (espace vectoriel)}
\end{defn}

\begin{exm}
	\begin{enumerate}
		\item $\dim_\R(\C) = 2$ et $\dim_\C(\C) = 1$ 
		\item $\dim_\mathbbm{K}(\mathbbm{K}^{n}) = n$ 
		\item $\dim_{\mathbbm{K}}(\mathcal{M}_{n,p}(\mathbbm{K})) = np$
	\end{enumerate}
\end{exm}

\begin{crlr}
	Soit $E$ un $\mathbbm{K}$-espace vectoriel de dimension finie, $\mathcal{L}$ une famille libre de $E$, $\mathcal{G}$ une famille génératrice de $E$. On note $n = \dim(E)$
	\begin{enumerate}
		\item $\#\mathcal{G} \ge n$ et $(\#\mathcal{G} = n \implies \mathcal{G} \text{ est une base de } E$)
		\item $\#\mathcal{L} \le n$ et $(\#\mathcal{L} = n \implies \mathcal{L} \text{ est une base de } E$)
	\end{enumerate}
\end{crlr}

\begin{crlr}
	$\R^{\R}$ est de dimension infinie.
	$\forall i \in \N, e_i: x \mapsto x^i$\\
	$(e_i)_{i\in\N}$ est libre dans $\R^\R$
\end{crlr}

\begin{prop}
	Soient $E$ et $F$ deux $\mathbbm{K}$-espaces vectoriels de dimension finie. Alors $E\times F$ est de dimension finie et $\dim(E\times F) = \dim(E) + \dim(F)$
\end{prop}

\begin{prv}
	Soit $(e_1,\ldots, e_n)$ une base de $E$, $(f_1, \ldots, f_p)$ une base de $F$.
	On pose \[
		\left\{\begin{array}
			{r c l}
			u_1 &=& (e_1,0_F)\\
			u_2 &=& (e_2,0_F)\\
					&\vdots&\\
			u_n &=& (e_n,0_F)\\
			u_{n+1} &=& (0_E, f_1)\\
			u_{n+2} &=& (0_E, f_2)\\
					&\vdots&\\
			u_{n+p} &=& (0_E,f_p)\\
		\end{array}\right.
	\]
	Soit $(x,y) \in E\times F$. \[
		\begin{cases}
			\exists (x_1,\ldots,x_n)\in \mathbbm{K}^n, x = \sum_{i=1}^{n} x_ie_i
			\exists (y_1,\ldots,y_n)\in \mathbbm{K}^n, x = \sum_{j=1}^{p} y_jf_j
		\end{cases}
	\] 
	\begin{align*}
		(x,y) &= \left( \sum_{i=1}^{n} x_ie_i, \sum_{i=1}^{p} y_jf_j \right)  \\
		&= \sum_{i=1}^{n} x_i (e_i + 0_F) + \sum_{j=1}^{p} y_j (0_E, f_j) \\
		&= \sum_{i=1}^{n} x_i u_i + \sum_{j=1}^{p} y_j u_{n+j} \\
	\end{align*}
	Donc, $E\times F = \Vect(u_1, \ldots, u_{n+p})$ donc $E\times F$ est de dimension finie.\\
	Soit $(\lambda_1, \ldots, \lambda_{n+p}) \in \mathbbm{K}^{n+p}$ tel que \[
		(*): \quad \sum_{k=1}^{n+p} \lambda_ku_k = 0_{E\times F} = (0_E, 0_F)
	\]
	\begin{align*}
		(*) &\iff \sum_{k=1}^{n} \lambda_k (e_k, 0_F) + \sum_{k=n+1}^{p} \lambda_k(0_E, f_{k-n}) = (0_E, 0_F)\\
				&\iff \begin{cases}
					\sum_{k=1}^{n} \lambda_k e_k = 0_E\\
					\sum_{k=n+1}^{p} \lambda_k f_{k-n} = 0_F
				\end{cases}\\
				&\iff \begin{cases}
					\forall k \in \left\llbracket 1,n \right\rrbracket, \lambda_k = 0_\mathbbm{K} \qquad&(\text{car $(e_1,\ldots,e_n)$ est libre})\\
					\forall k \in \left\llbracket n+1,n+p \right\rrbracket, \lambda_k = 0_\mathbbm{K} \qquad&(\text{car $(f_1,\ldots,f_n)$ est libre})\\
				\end{cases}
	\end{align*}
	Donc $(u_1, \ldots, u_{n+p})$ est une base de $E\times F$. Donc, $\dim(E\times F) = n + p = \dim(E) + \dim(F)$
\end{prv}

\begin{rmk}
	[Convention]
	\[\dim\big(\{0_E\}\big) = 0\]
\end{rmk}

\begin{thm}
	Soit $E$ un $\mathbbm{K}$-espace vectoriel de dimension finie, $F$ un sous-espace vectoriel de $E$. Alors, $F$ est de dimension finie et  $\dim(F) \le \dim(E)$\\
	Si $\dim(F) = \dim(E)$, alors $F = E$
\end{thm}

\begin{prv}
	On considère \[
		A = \{k \in \N \mid \text{il existe une famille libre de $F$ à $k$ éléments}\} 
	\]
	On suppose $F \neq \{0_E\}$.
	\begin{itemize}
		\item Soit $u \in F\setminus \{0_E\}$. $(u)$ est libre donc $1 \in A$ et donc $A \neq \O$
		\item Soit $\mathcal{L}$ une famille libre de $F$. Alors, $\mathcal{L}$ est une famille libre de $E$ \\
			donc $\#\mathcal{L} \le \dim(E)$\\
			Donc $A$ est majorée par $\dim(E)$ \\
			On en déduit que $A$ a un plus grand élément $p$.
		\item Soit $\mathcal{L}$ une famille libre de $F$ avec $p$ éléments.\\
			Si $\mathcal{L}$ n'engendre pas $F$, alors il existe $u\in F$ tel que $u\not\in \Vect(\mathcal{L})$ et donc $\mathcal{L} \cup \{u\}$ est une famille libre de $F$, donc $p+1 \in A$ en contradiction avec la maximalité de $p$.\\
			Donc $\mathcal{L}$ est une base de $F$ donc $F$ est de dimension finie et $\dim(F) = p \le \dim(E)$\\
	\end{itemize}

	Soit $\mathcal{B}$ une base de $F$. Alors, $\mathcal{B}$ est aussi une famille de libre de de $E$. Donc $\#\mathcal{B} \le \dim(E)$ donc $\dim(F) = \dim(E)$ \\
	Si $\dim(F) = \dim(E)$, alors $\mathcal{B}$ est une base de $E$, et donc $F = \Vect(\mathcal{B}) = E$
\end{prv}

\begin{prop}
	[Formule de Grassmann]
	Soit $E$ un $\mathbbm{K}$-espace vectoriel de dimension finie, $F$ et $G$ deux sous-espace vectoriels de $E$. Alors, \[
		\dim(F+G) = \dim(F) + \dim(G) - \dim(F\cap G)
	\] 
\end{prop}

\begin{prv}
	Soit $(e_1, \ldots, e_p)$ une base de $F\cap G$. $(e_1,\ldots,e_p)$ est une famille libre de $F$.\\
	On complète $(e_1, \ldots, e_p)$ en une base $(e_1, \ldots, e_p, u_1, \ldots, u_q)$ de $F$.\\
	De même, on complète $(e_1, \ldots, e_p)$ en une base $(e_1, \ldots, e_p, v_1, \ldots, v_r)$ de $G$.\\
	On pose  $\mathcal{B} = (e_1, \ldots, e_p, u_1, \ldots, u_q, v_1, \ldots, v_r)$. Montrons que $\mathcal{B}$ est une base de $F+G$
	\begin{itemize}
		\item Soit $u \in F+G$ \\
			On pose $u = v+w$ avec $\begin{cases}
				v\in F\\
				w \in G
			\end{cases}$.\\
			On pose $v = \sum_{i=1}^p \lambda_i e_i + \sum_{i=1}^q \mu_i u_i$ avec $(\lambda_1, \ldots, \lambda_p, \mu_1, \ldots, \lambda_q) \in \mathbbm{K}^{p+q}$\\
			On pose aussi $w = \sum_{i = 1}^p \lambda'_ie_i + \sum_{j=1}^r \nu_j v_j$ avec $(\lambda_1',\ldots,\lambda_p', \nu_1, \ldots, \nu_r) \in \mathbbm{K}^{p+r}$\\
			D'où, \[
				u = \sum_{i=1}^p (\lambda_i + \lambda'_i)e_i + \sum_{j=1}^q \mu_j u_j + \sum_{k=1}^r \nu_k v_k \in \Vect(\mathcal{B})
			\]
		\item Soient $(\lambda_1, \ldots, \lambda_p, \mu_1, \ldots, \mu_q, \nu_1, \ldots, \nu_r) \in \mathbbm{K}^{p+q+r}$.\\
			On suppose \[
				(*)\quad \sum_{i=1}^{p}\lambda_ie_i + \sum_{j=1}^q\mu_ju_j + \sum_{k=1}^r \nu_k v_k = 0_E
			\] 
			D'où, \[
				\underbrace{\sum_{i=1}^p\lambda_i e_i + \sum_{j=1}^q \mu_ju_j}_{\in F} = \underbrace{-\sum_{k=1}^r\nu_jv_k}_{\in G}
			\] 
			Donc, \[
				f = \sum_{i=1}^p \lambda_i e_i + \sum_{j=1}^q \mu_j u_j \in F\cap G
			\] Comme $(e_1, \ldots, e_p)$ est une base de $F\cap G$, $\exists ! (\lambda_1', \ldots, \lambda_p') \in \mathbbm{K}^p$ tel que \[
				f = \sum_{i=1}^p \lambda'_i e_i = \sum_{i=1}^p \lambda'_i e_i + \sum_{j=1}^q 0_\mathbbm{K}u_j
			\] Comme $(e_1, \ldots, e_p, u_1, \ldots, u_q)$ est une base de $F$, \[
				\forall k \in \left\llbracket 1, q \right\rrbracket, \mu_j = 0_\mathbbm{K}
			\] De même, \[
				\forall k \in \left\llbracket 1,r \right\rrbracket , \nu_k = 0_\mathbbm{K}
			\] On remplace dans $(*)$ pour trouver \[
				\sum_{i=1}^p \lambda_ie_i = 0_E
			\] Comme $(e_1, \ldots, e_p)$ est libre, \[
				\forall i \in \left\llbracket 1,p \right\rrbracket, \lambda_i = 0_\mathbbm{K}
			\] Donc $\mathcal{B}$ est libre.\\
			Donc, 
			\begin{align*}
				\dim(F+G) &=  p +q + r \\
				&= (p+q)+ (p+r) - p \\
				&= \dim(F) + \dim(G) - \dim(F\cap G) \\
			\end{align*}
	\end{itemize}
\end{prv}

\begin{crlr}
	Avec les hypothèse précédentes, \[
		E = F \oplus G \iff \begin{cases}
			F \cap  G = \{0_E\} \\
			\dim(E) = \dim(F) + \dim(G)
		\end{cases}
	\] 
\end{crlr}

\begin{prv}
	\begin{itemize}
		\item[``$\implies$''] On suppose $E = F \oplus G$ \\
			Comme la somme est directe, $F \cap G = \{0_E\}$ 
			\begin{align*}
				\dim(E) &= \dim(F)\\
				&= \dim(F) + \dim(G) - \dim(F\cap G)\\
				&= \dim(F) + \dim(G)\\
			\end{align*}
		\item[``$\impliedby$''] On suppose $F\cap G = \{0_E\}$ et $\dim(E) = \dim(F) + \dim(G)$.\\
			On sait déjà que $F+G = F \oplus G$\\
			 \begin{align*}
				\dim(F+G) = \dim(F) + \dim(G) - \dim(F \cap G) = \dim(E)
			\end{align*}
			Donc $F + G = E$
	\end{itemize}
\end{prv}

\begin{prop}
	Soit $F$ un $\mathbbm{K}$-espace vectoriel de dimension finie $n$. Soit $\mathcal{B} = (e_1, \ldots, e_n)$ une base de $F$. L'application
	\begin{align*}
		f: \mathbbm{K}^n &\longrightarrow F \\
		(\lambda_1, \ldots, \lambda_n) &\longmapsto \sum_{i=1}^n \lambda_i e_i
	\end{align*} est bijective.\\
	Si $\mathbbm{K}$ est infini, $\mathbbm{K}^n$ aussi et donc $F$ aussi.\\
	Si $\#\mathbbm{K} = p \in \N_*$,
	\begin{align*}
		\#&\mathbbm{K}^n = p^n\\
		&\vrt=\\
		\#&F
	\end{align*}
\end{prop}


		\part{Dérivation}

\underline{Motivation}:

{
\begin{wrapfigure}{l}{3cm}
	\centering
	\begin{asy}
		import three;

		size(3cm);
		settings.render=0;
		settings.prc=false;
		currentprojection = obliqueZ;

		draw(unitbox);
		draw(shift(1.1Z + 0.05X) * (O -- X), Arrows3(TeXHead2));
		draw(shift(1.1Z + 0.05Y) * (O -- Y), Arrows3(TeXHead2));
		draw(shift(1.1X + 0.05Z) * (O -- Z), Arrows3(TeXHead2));

		label("$x$", (X/2) + (1.1Z + 0.05X), align=S);
		label("$y$", (Y/2) + (1.1Z + 0.05Y), align=W);
		label("$z$", (Z/2) + X, align=SE);
	\end{asy}
\end{wrapfigure}

\begin{align*}
	&S(x,y,z) = 2(xy + xz + yz)\\
	&V(x,y,z) = xyz
\end{align*}

On cherche à minimiser $S$ avec la contrainte $V = 1$.

Soit $f : \begin{array}{rcl}
	\left( \R_*^+ \right)^2 &\longrightarrow& \R \\
	(x,y) &\longmapsto& S\left( x,y,\frac{1}{xy} \right) = 2\left( xy + \frac{1}{y} + \frac{1}{x} \right).
\end{array}$

On cherche $(a,b) \in \left( \R^+_* \right)^2$ tel que \[
	\forall (x,y) \in (\R^+_*), f(x,y) \ge f(a,b).
\]
}

\begin{defn}
	Soit $f: U \to \R$ où $U$ est un ouvert de $\R^2$. Soit $(a,b) \in U$.
	\vspace{2mm}

	Si $\lim_{x \to a} \frac{f(x,b) - f(a,b)}{x - a} \in \R$, alors on dit que $f$ a une dérivée partielle suivant $x$ en $(a,b)$ et cette limite est notée \[
		\partial f_1(a,b) = \frac{\partial f}{\partial x}(a,b).
	\]

	Si $\lim_{y \to b} \frac{f(a,y) - f(a,b)}{y - b} \in \R$, alors on dit que $f$ a une dérivée partielle suivant $y$ et la limite est notée \[
		\partial f_2(a,b) = \frac{\partial f}{\partial y}(a,b).
	\]
\end{defn}

\begin{exm}
	\begin{enumerate}
		\item $f: (x,y) \mapsto xy + x - y$.

			\begin{align*}
				&\frac{\partial f}{\partial x} : (x,y) \mapsto y + 1,\\
				&\frac{\partial f}{\partial y} : (x,y) \mapsto x - 1.
			\end{align*}

		\item $f: (x,y) \mapsto xy + \frac{1}{y}+ \frac{1}{x}$.

			\begin{align*}
				&\frac{\partial f}{\partial x}: (x,y) \mapsto y - \frac{1}{x^2},\\
				&\frac{\partial f}{\partial y}: (x,y) \mapsto x - \frac{1}{y^2}.
			\end{align*}

		\item Trouver $f$ telle que $\begin{cases}
				(1): \qquad \frac{\partial f}{\partial x}=y,\\[2mm]
				(2): \qquad \frac{\partial f}{\partial y} = x.
			\end{cases}$

			D'après $(1)$ : \[
				\forall (x,y), \exists C(y) \in \R, f(x,y) = xy + C(y)
			\] et donc \[
				\frac{\partial f}{\partial y}(x,y) = x + C'(y)
			\] donc $C'(y) = 0$ et donc $C$ est constante.

		\item Trouver $f$ telle que $\begin{cases}
			\frac{\partial f}{\partial x} = -y,\\[2mm]
			\frac{\partial f}{ƒ\partial y} = x.
		\end{cases}$

		Ce n'est pas possible !
	\end{enumerate}
\end{exm}

\begin{defn}~\\
	\begin{minipage}{\linewidth}
		\begin{wrapfigure}{r}{4cm}
			\centering
			\vspace{-5mm}
			\begin{asy}
				import three;
				import graph3;
				size(4cm);

				settings.render = 0;
				settings.prc = false;
				currentprojection = obliqueX;

				draw(O -- X, Arrow3(TeXHead2));
				draw(O -- Y, Arrow3(TeXHead2));
				draw(O -- Z, Arrow3(TeXHead2));

				triple f(real x, real y, real z = 0) { return (x,y,cos(x - 0.5) * cos(y - 0.5)/1.2 + 0.15); }

				real inc = 1 / 5;

				for(real x = 0; x <= 1; x += inc) {
					draw(graph(
						new real(real t) { return x; }, // x
						new real(real y) { return y; }, // y
						new real(real y) { return f(x,y).z; }, // z
						0, 1
					), gray);
				}

				for(real y = 0; y <= 1; y += inc) {
					draw(graph(
						new real(real x) { return x; }, // x
						new real(real t) { return y; }, // y
						new real(real x) { return f(x,y).z; }, // z
						0, 1
					), gray);
				}

				path3 path1 = (0.8, 0.2, 0) .. (0.5, 0.5, 0) .. (0.3, 0.7, 0);
				path3 path2 = f(0.8, 0.2, 0) .. f(0.5, 0.5, 0) .. f(0.3, 0.7, 0);
				path3 d = (0.2, 0.3, 0) .. (0.3, 0.4, 0) .. (0.2, 0.7, 0) .. (0.8, 0.9, 0) .. (0.6, 0.2, 0) .. cycle;

				draw(path1, red, Arrow3(TeXHead2));
				draw(path2, red, Arrow3(TeXHead2, position=0.8));

				dot((0.5, 0.5, 0));
				dot(f(0.5, 0.5, 0));
				draw((0.5, 0.5, 0) -- f(0.5, 0.5, 0), dashed);
				draw(d);

				label("$w$", (0.3, 0.7, 0), red, align=SE);
				label("$U$", (0.8, 0.9, 0), align=SE);
			\end{asy}
		\end{wrapfigure}

		Soit $f: U \to \R$ où $U$ est un ouvert. Soit $(a,b) \in U$. Soit $w = (w_1, w_2) \in \R^2$.

		Si 
		\[
			\lim_{t\to 0} \frac{f(a + tw_1, b + tw_2) - f(a,b)}{t}
		\] existe et est réelle, alors on dit que $f$ a une dérivée dans la direction de $w$ et la limite est notée \[
			\mathrm{d}f(w)\,(a,b) = D_w(f)\,(a,b).
		\]
	\end{minipage}
\end{defn}

\begin{exm}
	\begin{align*}
		f: \left( \R_*^+ \right)^2 &\longrightarrow \R \\
		(x,y) &\longmapsto xy+\frac{1}{x}+\frac{1}{y}.
	\end{align*}

	On pose $(a,b) = (1,2)$, $w = (w_1, w_2) = (1,1)$.
	\begin{align*}
		\frac{f(1+t, 2+t) - f(1,2)}{t} &= \frac{1}{t} \left( (1+t)(2+t) + \frac{1}{1+t} + \frac{1}{2+t} - 3 - \frac{1}{2} \right) \\
		&= \frac{1}{t}\left(\cancel 2 + 3t + \po(t) + \cancel 1 - t + \po(t) + \frac{1}{2}\left( \cancel 1 - \frac{t}{2} + \po(t) \right) - \cancel3 - \cancel{\frac{1}{2}} \right) \\
		&= \frac{1}{t} \left( \frac{7}{4} t + \po(t) \right)  \\
		&= \frac{7}{4} + \po(1) \tendsto{t \to 0} \frac{7}{4}. \\
	\end{align*}

	Donc, \[
		\mathrm{d}f(1,1)\,(1,2) = \frac{7}{4}.
	\]
\end{exm}

\begin{rmk}~\\
	\begin{figure}[H]
		\centering
		\begin{asy}
			import solids;
			import graph;
			size(5cm);

			settings.render = 0;
			settings.prc = false;

			path3 par = graph(
				new real(real x) { return x; },
				new real(real x) { return 0; },
				new real(real x) { return x^2; },
				0,3);
			revolution r = revolution(par, axis=Z);

			path3 par2 = graph(
				new real(real x) { return x; },
				new real(real x) { return 0; },
				new real(real x) { return x^2; },
				-3,3);

			draw(r,1,longitudinalpen=nullpen);
			draw(r.silhouette());

			draw((-4, 0, -1) -- (-4, 0, 10) -- (4, 0, 10) -- (4, 0, -1) -- cycle, red);
			draw(par2, deepred);

			draw((4,4.5) -- (7, 4.5), black+0.5mm, Arrow(TeXHead));

			path par2d = graph(new real(real x) { return x^2; }, -3, 3);
			draw(shift((11, 0)) * par2d, deepred);

			dot(O);
			dot((11, 0));
		\end{asy}
	\end{figure}
\end{rmk}


%todo ajouter théorème-définition
\begin{thm}
	Soit $f : U \to \R$, $(a,b) \in U$. On suppose que $\frac{\partial f}{\partial x}$ et $\frac{\partial f}{\partial y}$ existent en $(a,b)$ et sont {\bfseries continues} en $(a,b)$. Alors,
	\begin{align*}
		&\forall (h, k) \in \R^2 \text{ tel que } (a +h, b + k) \in U,\\
		&f(a+ h, b + k) = f(a,b) + h \frac{\partial f}{\partial x}(a,b) + k \frac{\partial f}{\partial y}(a,b) + \po_{(h,k)\to (0,0)}\big(\|(h,k)\|\big).
	\end{align*}

	On dit que $f$ est de classe $\mathcal{C}^1$ si $\frac{\partial f}{\partial x}$ et $\frac{\partial f}{\partial y}$ existent et sont continues.

	\qed
\end{thm}

\begin{rmk}
	En physique, cette formule correspond à : \[
		\mathrm{d}f = \frac{\partial f}{\partial x}\mathrm{d}x + \frac{\partial f}{\partial y} \mathrm{d}y.
	\] En effet :
	\begin{align*}
		\mathrm{d}f &= f(x+ \mathrm{d}x, y + \mathrm{d}y) - f(x,y) \\
		&= \frac{\partial f}{\partial x} \mathrm{d}x + \frac{\partial f}{\partial y} \mathrm{d}y.
	\end{align*}
\end{rmk}

\begin{prop}
	Soit $f: U \to \R$ de classe $\mathcal{C}^1$ en $(a,b) \in U$. Alors, \[
		\forall w = (w_1, w_2) \in \R^2, \mathrm{d}f(w)\,(a,b) = w_1 \frac{\partial f}{\partial x}(a,b) + w_2 \frac{\partial f}{\partial y}(a,b).
	\]
\end{prop}

\begin{prv}
	Soit $w = (w_1, w_2) \in \R^2$. Soit $t \in \R^*$.
	\begin{align*}
		\frac{1}{t}\big(f(a + tw_1, b + tw_2) - f(a,b)\big)
		&= \frac{1}{t} \left( tw_1 \frac{\partial f}{\partial x}(a,b) + tw_2 \frac{\partial f}{\partial y}(a,b) + \po_{t \to 0}\big(\|tw\|\big) \right) \\
		&= w_1 \frac{\partial f}{\partial x}(a,b) + w_2 \frac{\partial f}{\partial y}(a,b) + \po_{t\to 0}(1) \\
		&\tendsto{t\to 0} w_1 \frac{\partial f}{\partial x}(a,b) + w_2\frac{\partial f}{\partial y}(a,b).
	\end{align*}
\end{prv}


\begin{defn}
	Avec les hypothèses précédentes, en posant \[
		\nabla f(a,b) = \left( \frac{\partial f}{\partial x}(a,b), \frac{\partial f}{\partial y}(a,b) \right) 
	\]on obtient \[
		\mathrm{d}f(w)\,(a,b) = \left<w  \mid \nabla f(a,b) \right>
	\] où $\left<\cdot|\cdot \right>$ est le produit scalaire.

	Le vecteur $\nabla f(a,b)$ est appelé \underline{gradient de $f$ en $(a,b)$}.

	Le développement limité à l'ordre 1 de $f$ devient \[
		f\big((a,b)+w\big) = f(a,b) + \left<w \mid \nabla f(a,b) \right> + \po_{w\to 0}(\|w\|)
	\]
\end{defn}

\begin{prop}
	Soit $f : U \to \R$ de classe $\mathcal{C}^1$.

	\begin{figure}[H]
    \centering
    \incfig{gradient}
	\end{figure}

	$\nabla f$ est orthogonal au lignes de niveaux de $f$, son orientation va dans le sens d'une augmentation de $f$.
\end{prop}

\begin{prv}
	Soit $\gamma : I \to U$ une courbe de niveau : \[
		\forall t \in I, f\big(\gamma(t)\big) = \text{cste}.
	\] D'après le lemme suivant : \[
		\forall t \in I, 0 = (f \circ \gamma)'(t) = \mathrm{d}f\big(\gamma'(t)\big)\big(\gamma(t)\big) = \left<\gamma'(t)  \mid \nabla f\big(\gamma(t)\big) \right>
	\] Donc $\nabla f\big(\gamma(t)\big)$ est orthogonal à $\gamma'(t)$.

	Pour tout $t \in I$, on pose $w(t) = t\, \nabla f\big(\gamma(t)\big)$. Donc \[
		f\big(\gamma(t) + w(t)\big) = f\big(\gamma(t)\big) + t \|\nabla f(\gamma(t))\|^2 + \po_{t \to 0}(t)
	\] Pour $t$ assez petit, $f\big(\gamma(t) + w(t)\big) - f\big(\gamma(t)\big)$ est du même signe que $t$.
\end{prv}

\begin{rmk}
	\begin{align*}
		V: \R^3 &\longrightarrow \R \\
		(x,y,z) &\longmapsto -mgz
	\end{align*}
	l'énerge potentielle de pesenteur

	On a donc \[
		\nabla V(x,y,z) = \left( \frac{\partial V}{\partial x}, \frac{\partial V}{\partial y}, \frac{\partial V}{\partial z} \right) = (0, 0, -mg) = \vec{P}.
	\]
\end{rmk}

\begin{lem}
	Soit $f : U \to \R$ de classe $\mathcal{C}^1$, $\gamma : \begin{array}{rcl}
		I &\longrightarrow& U \\
		t &\longmapsto& \big(x(t), y(t)\big)
	\end{array}$ où $x$ et $y$ sont dérivables.

	On pose \[
		\forall t \in I, \gamma'(t) = \big(x'(t), y'(t)\big).
	\] Alors $f \circ \gamma : I \to \R$ est dérivable et
	\begin{align*}
		\forall t \in I, (f \circ \gamma)'(t) &= \mathrm{d}f\big(\gamma'(t)\big) \big(\gamma(t)\big)\\
		&= \left<\gamma'(t)  \mid \nabla f\big(\gamma(t)\big)  \right> \\
		&= x'(t) \frac{\partial f}{\partial x}\big(x(t), y(t)\big) + y'(t) \frac{\partial f}{\partial y}\big(x(t),y(t)\big). \\
	\end{align*}
\end{lem}

\begin{prv}
	On fixe $t \in I$.

	\begin{align*}
		\forall h \neq 0, \frac{f \circ \gamma(t + h) - f \circ \gamma(t)}{h}
		&= \frac{1}{h}\big(f(\gamma(t)) + h\gamma'(t) + \po_{h\to 0}(h) - f(\gamma(t))\big) \\
		&= \frac{1}{h}\bigg(\cancel{f(\gamma(t))} + \left<h\gamma'(t) \mid \nabla f(\gamma(t)) \right> + \po_{h\to 0}(\|h\gamma'(t)\|) - \cancel{f(\gamma(t))}\bigg)\\
		&= \left<\gamma'(t) \mid \nabla f(\gamma(t)) \right> + \po_{h\to 0}(1) \\
		&\tendsto{h\to 0} \left<\gamma'(t)  \mid \nabla f(\gamma(t)) \right>
	\end{align*}
\end{prv}

\begin{defn}
	Soit $f : U \to \R$ de classe $\mathcal{C}^1$ et $(a,b) \in U$. On dit que $(a,b)$ est un \underline{point critique} de $f$ si $\nabla f(a,b) = 0$ i.e. $\frac{\partial f}{\partial x}(a,b) = \frac{\partial f}{\partial y}(a,b) = 0$.

	Dans ce cas, $f(a,b)$ est appelé \underline{valeur critique} de $f$.
\end{defn}

\begin{prop}~\\
	\begin{minipage}{\linewidth}
		\begin{wrapfigure}{r}{3cm}
			\centering
			\vspace{-1cm}
			\begin{asy}
				import solids;
				import graph;
				size(3cm);

				settings.render = 0;
				settings.prc = false;

				path3 par = graph(
					new real(real x) { return x; },
					new real(real x) { return 0; },
					new real(real x) { return -x^2; },
					0,3);
				revolution r = revolution(par, axis=Z);

				draw(r,1,longitudinalpen=nullpen);
				draw(r.silhouette());

				dot("$(a,b)$", O, red, align=N);
				real s = sqrt(2.5);
				path3 g=(s,0,-2.5)..(0,s,-2.5)..(-s,0,-2.5)..(0,-s,-2.5)..cycle;
				draw(g, deepcyan);
			\end{asy}
		\end{wrapfigure}
		Soit $f: U \to \R$ de classe $\mathcal{C}^1$ et $(a,b) \in U$ tel que \[
			\exists r > 0, \forall (x,y) \in B_{(a,b)}(r), f(x,y) \le f(a,b)
		\] Alors $\nabla f(a,b) = (0,0)$.
	\end{minipage}
\end{prop}

\begin{prv}
	Soit $g: x \mapsto f(x,b)$. $g(a)$ est un maximum local de $g$ donc $g'(a) = 0$.

	Or, $g'(a) = \frac{\partial f}{\partial x}(a,b)$

	donc $\frac{\partial f}{\partial x}(a,b) = 0$.

	Soit $h : y \mapsto f(a,y)$. On a de même $h'(b) = 0$.

	Or, $h'(b) = \frac{\partial f}{\partial y}(a,b)$.

	Donc, $\nabla f(a,b) = (0,0)$.
\end{prv}

\begin{rmk}
	Un minimum local est aussi une valeur critique.
\end{rmk}

\begin{figure}[H]
	\centering
	\begin{subfigure}{3cm}
		\centering
		\begin{asy}
			import solids;
			import graph;
			size(3cm);

			settings.render = 0;
			settings.prc = false;

			path3 par = graph(
				new real(real x) { return x; },
				new real(real x) { return 0; },
				new real(real x) { return -x^2; },
				0,3);
			revolution r = revolution(par, axis=Z);

			draw(r,1,longitudinalpen=nullpen);
			draw(r.silhouette());

			dot(O, red);
		\end{asy}
		\caption{Maximum local}
	\end{subfigure}
	\begin{subfigure}{3cm}
		\centering
		\begin{asy}
			import solids;
			import graph;
			size(3cm);

			settings.render = 0;
			settings.prc = false;

			path3 par = graph(
				new real(real x) { return x; },
				new real(real x) { return 0; },
				new real(real x) { return x^2; },
				0,3);
			revolution r = revolution(par, axis=Z);

			draw(r,1,longitudinalpen=nullpen);
			draw(r.silhouette());

			dot(O, red);
		\end{asy}
		\caption{Minimum local}
	\end{subfigure}
	\begin{subfigure}{3cm}
		\centering
		\begin{asy}
			import solids;
			import graph;
			size(3cm);

			settings.render = 0;
			settings.prc = false;
			currentprojection = obliqueZ;

			draw(graph(
				new real(real x) { return x; },
				new real(real x) { return -x^2 / 3; },
				new real(real x) { return 3; },
				-3, 3
			));

			draw(graph(
				new real(real x) { return x; },
				new real(real x) { return -x^2 / 3; },
				new real(real x) { return -3; },
				-3, 3
			));

			draw(graph(
				new real(real x) { return x; },
				new real(real x) { return -x^2 / 3 - 1; },
				new real(real x) { return 0; },
				-3, 3
			));

			draw(graph(
				new real(real x) { return 0; },
				new real(real x) { return x^2 / 9 - 1; },
				new real(real x) { return x; },
				-3, 3
			));

			draw(graph(
				new real(real x) { return -3; },
				new real(real x) { return x^2 / 9 - 4; },
				new real(real x) { return x; },
				-3, 3
			));

			draw(graph(
				new real(real x) { return 3; },
				new real(real x) { return x^2 / 9 - 4; },
				new real(real x) { return x; },
				-3, 3
			));

			dot((0,-1,0), red);
		\end{asy}
		\caption{Point de selle / Point col}
	\end{subfigure}
\end{figure}

\begin{exm}
	On revient à l'exemple donné en introduction : 
	\begin{align*}
		f: \left( \R^*_+ \right)^2 &\longrightarrow \R \\
		(x,y) &\longmapsto 2\left( xy + \frac{1}{x} + \frac{1}{y} \right).
	\end{align*}

	$\left( \R^+_* \right)^2$ est un ouvert de $\R^2$. Soit $(x,y) \in \left( \R^+_* \right)^2$.
	
	On a \[
		\begin{cases}
			\frac{\partial f}{\partial x}(x,y) = 2\left( y - \frac{1}{x^2} \right),\\
			\frac{\partial f}{\partial y}(x,y) = 2\left( x - \frac{1}{y^2} \right).
		\end{cases}
	\]

	\begin{align*}
		&\frac{\partial f}{\partial x}(x,y) = \frac{\partial f}{\partial y}(x,y) = 0\\
		\iff& \begin{cases}
			y = \frac{1}{x^2}\\
			x = \frac{1}{y^2}
		\end{cases}\\
		\iff& \begin{cases}
			y = \frac{1}{x^2}\\
			x = x^4
		\end{cases}\\
		\iff& \begin{cases}
			x = 1\\
			y = 1
		\end{cases}
	\end{align*}

	On vérivie que $f$ présente en effet un minium local en $(1,1)$. \[
		f(1,1) = 6
	\] On fixe $y \in \R^+_*$ et \[
		g : x \mapsto 2\left( xy + \frac{1}{x} + \frac{1}{y} \right).
	\] Donc \[
		\forall x \in \R^+_*, g'(x) = 2\left( y - \frac{1}{x^2} \right).
	\]
	\begin{center}
		\begin{tikzpicture}
			\tkzTabInit{$x$/1,$g'(x)$/1,$g$/2.3}{$0$, $\frac{1}{\sqrt{y}}$, $+\infty$}
			\tkzTabLine{,-,z,+,}
			\tkzTabVar{+/{}, -/$2\left( 2\sqrt{y} +\frac{1}{y} \right)$, +/{}}
		\end{tikzpicture}
	\end{center}
	
	Ainsi, \[
		\forall x \in \R^+_*, \forall y \in \R^+_*, f(x,y) \ge 2\left( 2\sqrt{y} + \frac{1}{y} \right)
	\] Soit $h : y \mapsto 2\sqrt{y} + \frac{1}{y}$. On a \[
		\forall y > 0, h'(y) = \frac{1}{\sqrt{y}} - \frac{1}{y^2} = \frac{y\sqrt{y} - 1}{y^2} = \frac{y^{\frac{3}{2}} - 1}{y^2}
	\]

	\begin{center}
		\begin{tikzpicture}
			\tkzTabInit{$y$/0.7,$h'(y)$/0.7,$h$/1.4}{$0$, $1$, $+\infty$}
			\tkzTabLine{,-,z,+,}
			\tkzTabVar{+/{}, -/$3$, +/{}}
		\end{tikzpicture}
	\end{center}

	Donc, \[
		\forall x,y > 0, f(x,y) \ge 2\times 3 = 6 = f(1,1).
	\]
\end{exm}

\begin{prop}
	[règle de la chaîne]

	Soit $f : \begin{array}{rcl}
		U &\longrightarrow& \R^2 \\
		(x,y) &\longmapsto& f(x,y)
	\end{array}$ de classe $\mathcal{C}^1$ et $U, V$ deux ouverts de $\R^2$.

	Soit $\varphi : \begin{array}{rcl}
		V &\longrightarrow& U \\
		(u,v) &\longmapsto& \varphi(u,v) = \big(x(u,v), y(u,v)\big)
	\end{array}$.

	On suppose que $x$ et $y$ sont de classe $\mathcal{C}^1$ sur $V$.

	Alors,  $f \circ \varphi : \begin{array}{rcl}
		V &\longrightarrow& \R \\
		(u,v) &\longmapsto& f\big(\varphi(u,v)\big)
	\end{array}$ est de classe $\mathcal{C}^1$ et
	\begin{align*}
		\forall (u_0, v_0) \in V, \frac{\partial (f \circ \varphi)}{\partial u}(u_0, v_0)
		&= \frac{\partial f}{\partial x}\big(\varphi(u_0, v_0)\big) \times \frac{\partial x}{\partial u}(u_0, v_0)\\
		&+ \frac{\partial f}{\partial y}\big(\varphi(u_0,v_0)\big) \frac{\partial y}{\partial u}(u_0,v_0)
	\end{align*}
	\begin{align*}
		\forall (u_0, v_0) \in V, \frac{\partial (f \circ \varphi)}{\partial v}(u_0, v_0)
		&= \frac{\partial f}{\partial x}\big(\varphi(u_0, v_0)\big) \times \frac{\partial x}{\partial v}(u_0, v_0)\\
		&+ \frac{\partial f}{\partial y}\big(\varphi(u_0,v_0)\big) \frac{\partial y}{\partial v}(u_0,v_0)
	\end{align*}
\end{prop}

\begin{exm}
	[changement de coordonnées polaires]
	On pose \begin{align*}
		\varphi: \R^+_* \times ]0,2\pi[ &\longrightarrow \R^2\setminus \left( R^+_* \times \{0\} \right) \\
		(r, \theta) &\longmapsto (r \cos \theta, r \sin\theta),
	\end{align*}
	\begin{align*}
		f: \R^2\setminus \left( R^+_* \times \{0\} \right) &\longrightarrow \R \\
		(x,y) &\longmapsto f(x,y),
	\end{align*}
	\begin{align*}
		g: \overbrace{\R^+_* \times ]0, 2\pi[}^{=V} &\longrightarrow \R \\
		(r, \theta) &\longmapsto f(r\cos\theta, r\sin\theta).
	\end{align*}

	\begin{align*}
		\forall (r_0,\theta_0) \in V,&\\[5mm]
		\frac{\partial g}{\partial r}(r_0, \theta_0) &= \frac{\partial f}{\partial x}(r_0\cos\theta_0, r_0\sin\theta_0)\cos\theta_0\\
		&+ \frac{\partial f}{\partial y}(r_0 \cos\theta_0, r_0\sin\theta_0)\sin\theta_0\\
		&= 2r_0\cos^2\theta_0 + 2r_0\sin^2(\theta_0) \\
		&= 2r_0 \\[5mm]
		\frac{\partial g}{\partial \theta}(r_0, \theta_0) &= \frac{\partial f}{\partial x}(r_0\cos\theta_0, r_0\sin\theta_0)r_0\sin\theta_0\\
		&+ \frac{\partial f}{\partial y}(r_0 \cos\theta_0, r_0\sin\theta_0)r_0\cos\theta_0\\
		&= -2{r_0}^2\cos(\theta_0)\sin(\theta_0) + 2{r_0}^2 \sin(\theta_0)\cos(\theta_0)\\
		&= 0 \\
	\end{align*}

	Donc, \[
		g(r, \theta) = r^2.
	\]
\end{exm}

\begin{exm}
	Résoudre \[
		\begin{cases}
			\frac{\partial f}{\partial x} = \frac{x}{x^2+y^2},\\
			\frac{\partial f}{\partial y} = \frac{y}{x^2+y^2}.\\
		\end{cases}
	\]

	On pose $g: (r, \theta) \mapsto f(r \cos\theta, r \sin\theta)$.

	\begin{align*}
		&\frac{\partial g}{\partial r} = \frac{1}{r}\cos^2\theta + \frac{1}{r}\sin^2\theta = \frac{1}{r},\\
		&\frac{\partial g}{\partial \theta} = -\cos(\theta) \sin(\theta) + \sin(\theta)\cos(\theta) = 0.
	\end{align*}

	Donc, \[
		\exists C \in \R, g: (r, \theta) \mapsto \ln r + C
	\] d'où,
	\begin{align*}
		\forall (x,y) \in \R^2 \setminus \{(0,0)\}, f(x,y) &= \ln\left(\sqrt{x^2 + y^2} \right)  + C\\
		&= \frac{1}{2}\ln(x^2 + y^2) + C. \\
	\end{align*}
\end{exm}

\begin{rmk}
	Soit $\mathcal{B} = (e_1, e_2)$ la base canonique de $\R^2$, $f: U \to \R$ de classe $\mathcal{C}^1$ avec $U$ un ouvert de $\R^2$.

	Soit $(x,y) \in U$.

	\begin{align*}
		\Mat_{\mathcal{B}}\big(\nabla f(x,y)\big) = \begin{pmatrix}
			\frac{\partial f}{\partial x}(x,y)\\[2mm]
			\frac{\partial f}{\partial y}(x,y)
		\end{pmatrix}
	\end{align*}

	Soit  \begin{align*}
		\varphi: V &\longrightarrow U \\
		(u,v) &\longmapsto \big(x(u,v), y(u,v)\big) 
	\end{align*} avec $x,y$ de classe $\mathcal{C}^1$. Soit $g = f \circ \varphi$.
	\begin{align*}
		\Mat_{\mathcal{B}}\big(\nabla g(u,v)\big)
		&= \begin{pmatrix}
			\frac{\partial g}{\partial u}(u,v) \\[2mm]
			\frac{\partial g}{\partial v}(u,v)
		\end{pmatrix} \\
		&= \begin{pmatrix}
			\frac{\partial x}{\partial u}(u,v) \frac{\partial f}{\partial x}(x,y)
			+ \frac{\partial y}{\partial u}(u,v)\frac{\partial f}{\partial y}(x,y)\\[3mm]
			\frac{\partial x}{\partial v}(u,v) \frac{\partial f}{\partial x}(x,y)
			+ \frac{\partial y}{\partial v}(u,v) \frac{\partial f}{\partial y}(x,y)
		\end{pmatrix}  \\
		&= \underbrace{\begin{pmatrix}
				\frac{\partial x}{\partial u}(u,v)& \frac{\partial y}{\partial u}(u,v)\\[3mm]
				\frac{\partial x}{\partial v}(u,v)& \frac{\partial y}{\partial v}(u,v)
		\end{pmatrix}}_{J(u,v)} \begin{pmatrix}
			\frac{\partial f}{\partial x}(x,y)\\[3mm]
			\frac{\partial f}{\partial y}(x,y)
		\end{pmatrix} \\
		&= J(u,v) \Mat_{\mathcal{B}}\big(\nabla f(x,y)\big) \\
	\end{align*}
	où $J(u,v) = 
	\begin{pNiceArray}{c:c}
		\Mat_{\mathcal{B}}\big(\nabla x(u,v)\big) & \Mat_{\mathcal{B}}\big(\nabla y(u,v)\big)
	\end{pNiceArray}$.

	On dit que $J(u,v)$ est \underline{la jacobienne} de $\varphi$ en $(u,v)$.
	L'application linéaire canoniquement associée à $J(u,v)$ est la \underline{différentielle de $\varphi$} en $(u,v)$ noté $\mathrm{d}\varphi(u,v)$.

	On a $\mathrm{d}\varphi(u,v) \in \mathcal{L}(R^2)$ et $\Mat_{\mathcal{B}}\big(\mathrm{d}\varphi(u,v)\big) = J(u,v)$.

	Par exemple, la jacobienne du changement de coordonnées polaires est \[
		J = \begin{pmatrix}
			\frac{\partial x}{\partial r} & \frac{\partial y}{\partial r}\\[3mm]
			\frac{\partial x}{\partial \theta} & \frac{\partial y}{\partial \theta}
		\end{pmatrix}
		= \begin{pmatrix}
			\cos\theta&\sin\theta\\
			-r\sin\theta&r\cos\theta
		\end{pmatrix}.
	\]
	$\underbrace{\det(J)}_{\text{le jacobien}} = r\cos^2\theta + r\sin^2\theta = r$

	Dans une intégrale double, si $(x,y) = \varphi(u,v)$, alors $\mathrm{d}x\mathrm{d}y = \det(J)\mathrm{d}u\mathrm{d}v$.

	Ici, \[
		\mathrm{d}x\ \mathrm{d}y = r\ \mathrm{d}r\ \mathrm{d}\theta.
	\]
\end{rmk}

\begin{prv}
	On pose $(x_0, y_0) = \varphi(u_0, v_0)$. Pour tout $(h,k) \in \R^2$ tels que $(u_0 + h, v_0 + k) \in V$, en posant $g = f  \circ \varphi$.

	\begin{align*}
		g(u_0 + h, v_0 + h) &= f\big(x(u_0 + h, v_0 + k), y(u_0 + h, v_0 + k)\big) \\
		&= f\left(
			x(u_0,v_0) + h \frac{\partial x}{\partial u}(u_0,v_0) + k \frac{\partial x}{\partial v}(u_0, v_0) + \po\big(\|(h,k)\|\big), \right.\\
		&\phantom{ = f\bigg(\bigg.}\left. y(u_0, v_0) + h \frac{\partial y}{\partial u}(u_0, v_0) + k \frac{\partial y}{\partial v}(u_0, v_0) + \po\big(\|(h,k)\|\big)
		\right)  \\
		&= f(x_0,y_0) \\
		&~+ \left( h \frac{\partial x}{\partial u}(u_0,v_0) + k \frac{\partial x}{\partial v}(u_0, v_0) + \po(\|(h,k)\|) \right) \frac{\partial f}{\partial x}(x_0,y_0)\\
		&~+ \left( h \frac{\partial y}{\partial u}(u_0, v_0) + k\frac{\partial y}{\partial v}(u_0, v_0) + \po(\|(h,k)\|) \right) \frac{\partial f}{\partial y}(x_0, y_0)\\
		&~+ \po(\|(h,k)\|)\\
		&= f(x_0, y_0) \\
		&~+ h \left( \frac{\partial x}{\partial u}(u_0, v_0) \frac{\partial f}{\partial x}(x_0, y_0) + \frac{\partial y}{\partial u}(u_0, v_0) \frac{\partial f}{\partial y}(x_0, y_0) \right)  \\
		&~+ k\left( \frac{\partial x}{\partial v}(u_0, v_0) \frac{\partial f}{\partial x}(x_0, y_0) + \frac{\partial y}{\partial v}(u_0, v_0) \frac{\partial f}{\partial y}(x_0, y_0) \right) 
		&~+ \po(\|(h,k)\|)\\
		&= g(u_0, v_0) + h \frac{\partial g}{\partial u}(u_0, v_0) + k \frac{\partial g}{\partial v}(u_0, v_0) + \po(\|(h,k)\|) \\
	\end{align*}

	Par identification,
	\[
		\frac{\partial g}{\partial u}(u_0, v_0) = \frac{\partial x}{\partial u}(u_0, v_0) \frac{\partial f}{\partial x}(x_0, y_0) + \frac{\partial y}{\partial u}(u_0, v_0) \frac{\partial f}{\partial y}(x_0,y_0)
	\] et \[
		\frac{\partial g}{\partial v}(u_0, v_0) = \frac{\partial x}{\partial v}(u_0,v_0) \frac{\partial f}{\partial x}(x_0, y_0) + \frac{\partial y}{\partial v}(u_0, v_0) \frac{\partial f}{\partial y}(x_0, y_0).
	\] 
\end{prv}

\begin{exm}
	[Régression linéaire]~\\
	\begin{figure}[H]
		\centering
		\begin{asy}
			import graph;
			axes(EndArrow);
			size(5cm);

			real f(real x) { return x + 0.5; }

			real k = 35 / (7 - 0.5);

			for(int i = 0; i < 35; ++i) {
				real mag = exp(sin(100 * pi/exp(1) * i)) * 0.8 + exp(cos(i*40)/3);
				real eps = mag * cos(10 * exp(1)/pi * i) / 3;
				dot((i/k,f(i/k) + eps));
			}

			draw(graph(f, -1, 7), orange);
		\end{asy}
	\end{figure}
	\[
		y = a x + b
	\] 
	On fixe $(a,b) \in \R^2$. \[
		\varepsilon(a,b) = \sum_{i=1}^n\big( y_i - (ax_i + b) \big)^2
	\] l'erreur totale.

	On veut minimiser $\varepsilon(a,b)$. On a 
	\[
		\forall (a,b) \in \R^2,
		\begin{cases}
			\frac{\partial \varepsilon}{\partial a}(a,b) = -2\sum_{i=1}^{n}(y_i - ax_i - b)x_i,\\
			\frac{\partial \varepsilon}{\partial b}(a,b) = -2\sum_{i=1}^{n}(y_i - ax_i - b).
		\end{cases}
	\]

	Donc,
	\begin{align*}
		(a,b) \text{ point critique de } \varepsilon \iff& \begin{cases}
			a \sum_{i=1}^n {x_i}^2 + b\sum_{i=1}^{n}x_i = \sum_{i=1}^{n} y_ix_i\\
			a\sum_{i=1}^{n}x_i + nb = \sum_{i=1}^ny_i
		\end{cases}\\
		\iff& \begin{cases}
			a \left( \frac{1}{n}\sum_{i=1}^n {x_i}^2 - \overline{x}^2\right) = \overline{y} - \overline{x} \overline{y}\\
			b = \frac{1}{n}\sum_{i=1}^ny_i - \frac{a}{n}\sum_{i=1}^nx_i = \frac{1}{n}\sum_{i=1}^n x_i y_i - \overline{x} \overline{y}
		\end{cases}\\
		&\text{ où } \overline{x} = \frac{1}{n} \sum_{i=1}^n x_i,~\overline{y} = \frac{1}{n}\sum_{i=1}^n y_i\\
		\iff& \begin{cases}
			a = \frac{\Cov(x,y)}{V(x)}\\
			b = \overline{y} - a\overline{x}
		\end{cases}
	\end{align*}

	Coefficient de corrélation: $\frac{\Cov(x,y)}{\sigma_x \sigma_y} \in [-1, 1]$
\end{exm}












		\part{Corps}

\begin{exm}[Problème]
	\begin{itemize}
		\item 
			avec $A = \Z / 9 \Z$, résoudre $\overline{x}^2 = \overline{0}$ \\
			\begin{center}
				\begin{tabular}{|c|c|c|c|c|c|c|c|c|c|c|}
					\hline
					$\overline{x}$&$\overline{0}$& $\overline{1}$ &$\overline{2}$&$\overline{3}$ &$\overline{4}$ &$\overline{5}$ &$\overline{6}$ &$\overline{7}$ &$\overline{8}$& $\overline{9}$ \\
					\hline
					$\overline{x}^2$&$\overline{0}$ &$\overline{1}$ &$\overline{4}$ &$\overline{0}$ &$\overline{7}$ &$7$ &$\overline{0}$ &$\overline{4}$ &$\overline{1}$&$\overline{0}$\\
					\hline
				\end{tabular}
			\end{center}
			On a trouvé 3 solutions: $\overline{0}$, $\overline{3}$, $\overline{6}$.
		\item $\Z / 8\Z$
			\begin{center}
				\begin{tabular}{|c|c|c|c|c|c|c|c|c|}
					\hline
					$\overline{x}$& $\overline{0}$& $\overline{1}$& $\overline{2}$& $\overline{3}$& $\overline{4}$& $\overline{5}$& $\overline{6}$& $\overline{7}$\\
					\hline
					$\overline{x^2}$& $\overline{0}$& $\overline{1}$& $\overline{4}$& $\overline{1}$& $\overline{0}$& $\overline{1}$& $\overline{4}$& $\overline{1}$\\
					\hline
				\end{tabular}
			\end{center}
			$\overline{x}^2=7$ a 4 solutions: $\overline{1}, \overline{7}, \overline{3},\text{ et } \overline{5}$
		\item $A = \mathbbm{H} = \{a + bi + cj + dk  \mid  (a,b,c,d) \in \R^4\}$ \\
			$i^2 = j^2 = k^2 = -1$ 
			\begin{align*}
				\begin{array}{c c c}
					ij = k & jk = i & ji = j\\
					ji = -k & kj = -i & ik = -j
				\end{array}
			\end{align*}
			Dans cet anneau, $-1$ a 6 racines!
	\end{itemize}
\end{exm}

\begin{defn}
	Soit $(\mathbbm{K}, +, \times)$ un ensemble muni de deux lois de composition internes. On dit que c'est un \underline{corps} si
	 \begin{enumerate}
		\item $(\mathbbm{K}, \times)$ est un groupe abélien
		\item $(\mathbbm{K}, \times)$ est un monoïde commutatif
		\item $\forall x \in \mathbbm{K}\setminus \{0_\mathbbm{K}\}, \exists y \in \mathbbm{K}, xy = 1_\mathbbm{K}$
		\item $0_\mathbbm{K} \neq  1_\mathbbm{K}$
	\end{enumerate}
	\index{corps}
\end{defn}

\begin{exm}
	\begin{itemize}
		\item $(\C, +, \times)$ est un corps
		\item $(\R, +, \times)$ est un corps
		\item $(\Q, +, \times)$ est un corps
		\item $(\Z, +, \times)$ n'est pas un corps
	\end{itemize}
\end{exm}

\begin{prop}
	$(\Z / n\Z, +, \times)$ est un corps si et seulement si $n$ est premier.
\end{prop}

\begin{prv}
	\[
		\left( \Z / n\Z \right)^\times = \left\{ \overline{k}  \mid k \wedge n = 1 \right\}
	\] 
\end{prv}


\begin{prop}
	Tout corps est un anneau intègre.
\end{prop}

\begin{prv}
	Soit $(\mathbbm{K}, +, \times)$ un corps. Soient $(a,b) \in \mathbbm{K}^2$ tel que $a \times b = 0_\mathbbm{K}$.\\
	On suppose $a \neq  0_\mathbbm{K}$. Alors, $a$ est inversible et donc \[
		b = a^{-1} \times a \times b = a^{-1} \times 0_\mathbbm{K} = 0_\mathbbm{K}
	\] 
\end{prv}

\begin{exm}
	Soit $(\mathbbm{K},+,\times)$ un corps.\\
	Résoudre \[
		\begin{cases}
			x^2 = 1_\mathbbm{K}\\
			x \in \mathbbm{K}
		\end{cases}
	\]

	\begin{align*}
		x^2 = 1_\mathbbm{K} &\iff x^2 - 1_\mathbbm{K} = 0_\mathbbm{K}\\
		&\iff (x - 1_\mathbbm{K})(x+1_\mathbbm{K}) = 0_\mathbbm{K}\\
		&\iff x - 1_\mathbbm{K} = 0_\mathbbm{K} \text{ ou } x + 1_\mathbbm{K} = 0_\mathbbm{K}\\
		&\iff x = 1_\mathbbm{K} \text{ ou } x = -1_\mathbbm{K}
	\end{align*}

	Il y a au plus 2 solutions.
\end{exm}

\begin{prop}
	Soit $(\mathbbm{K},+,\times )$ un corps et $P$ un polynôme à coefficients dans $\mathbbm{K}$ de degré $n$. Alors, l'équation $P(x) = 0_{\mathbbm{K}}$ a au plus $n$ solutions dans $\mathbbm{K}$ 
	\qed
\end{prop}

\begin{crlr}[(Théorème de Wilson)]
	voir exercice 16 du TD 12
\end{crlr}


\begin{defn}
	Soit $(\mathbbm{K}, +, \times)$ un corps et $L\subset \mathbbm{K}$.\\
	On dit que $L$ est un \underline{sous corps} de $\mathbbm{K}$ si
	\begin{enumerate}
		\item $L$ est un anneau de $(\mathbbm{K}, +, \times)$ non nul
		\item $\forall x \in L\setminus \{0_\mathbbm{K}\}, x^{-1} \in L$ 
	\end{enumerate}
	\vspace{2mm}
	en d'autres termes si
	\begin{enumerate}
		\item $\forall (x,y) \in L^2, x - y \in L$
		\item $\forall (x,y) \in L^2, x \times y^{-1} \in L$
	\end{enumerate}
	\vspace{5mm}
	On dit aussi que $\mathbbm{K}$ est une \underline{extension} de $L$.
	\index{sous corps}
	\index{extension}
\end{defn}

\begin{prop}
	Tout sous corps est un corps. \qed
\end{prop}

\begin{defn}
	Soient $(\mathbbm{K}_1,+,\times )$ et $(\mathbbm{K}_2,+, \times)$ deux corps et $f: \mathbbm{K}_1 \to \mathbbm{K}_2$.\\
	On dit que $f$ est un \underline{morphisme de corps} si $f$ est un morphisme d'anneaux.\\
	i.e. si
	\[
		\begin{cases}
			\forall (x,y) \in {\mathbbm{K}_1}^2,& f(x+y) = f(x) + f(y)\\
			\forall (x,y) \in {\mathbbm{K}_1}^2,& f(x \times y) = f(x) \times f(y)\\
		\end{cases}
	\] 
	\index{homomorphisme (de corps)}
	\index{morphisme (de corps)}
\end{defn}

\begin{prop}
	Tout morphisme de corps est injectif.
\end{prop}

\begin{prv}
	Soit $f: \mathbbm{K}_1 \to \mathbbm{K}_2$ un morphisme de corps.\\
	\begin{itemize}
		\item $\Ker(f)$ est un sous groupe de $(\mathbbm{K}_1, +)$ 
		\item Soit $x \in \Ker(f)$ et $y \in \mathbbm{K}_1$ \[
				f(x \times y) = f(x) \times f(y) = 0_{\mathbbm{K}_2} \times f(y) = 0_{\mathbbm{K}_2}
			\]
		\item Soit $x \in \Ker(f) \setminus \{0_{\mathbbm{K}_1}\}$.\\
			Alors, $x$ est inversible.\\
			\begin{align*}
				\begin{rcases*}
					x \in \Ker(f)\\
					x^{-1} \in \mathbbm{K}_1
				\end{rcases*}& \text{ donc } x \times x ^{-1} \in \Ker(f)\\
				&\text{ donc } 1_{\mathbbm{K}_1} \in \Ker(f)\\
				&\text{ donc } f(1_{\mathbbm{K}_1}) = 0_{\mathbbm{K}_2}
			\end{align*}
			Or, $f(1_{\mathbbm{K}_1}) = 1_{\mathbbm{K}_2} \neq 0_{\mathbbm{K}_2}$
	\end{itemize}
	Donc, $\Ker(f) = \{0_{\mathbbm{K}_1}\}$ donc $f$ est injective.
\end{prv}

\begin{exm}
	$\begin{array}{cc}
		\C &\longrightarrow \C\\
		z &\longmapsto \overline{z}\\
	\end{array}$ est un morphisme de corps
\end{exm}



		\part{Opérations sur les séries}

\begin{prop}
	L'ensemble $E = \{u \in \C^\N  \mid \Sigma u_n \text{ converge}\}$ est un sous-espace vectoriel de $\C^\N$ et \begin{align*}
		S: E &\longrightarrow \C \\
		u &\longmapsto \sum_{n=0}^{+\infty} u_n
	\end{align*} est une forme linéaire.
	\qed
\end{prop}

\begin{rmk}
	La somme d'une série convergente et d'une série divergente diverge.
	Le produit d'une série divergente par un scalaire non nul diverge.
\end{rmk}

		\part{Comparaison de suites}

\begin{defn}
	Soient $u$ et $v$ deux suites réelles. On dit que $u$ est \underline{dominée} par  $v$ si \[
	\exists M\in \R, \exists N\in \N,\forall n\ge N,\left| u_n \right| \le M \left| v_n \right| 
	\] Dans ce cas, on note $u = O(v)$ ou $u_n = O(v_n)$ et on dit que "$u$ est un grand o de $v$"
\end{defn}

\begin{exm}
	En informatique, on dit qu'un alogirithme a une \underline{complexité linéaire} si son temps d'éxécution est un $O(n)$ 
	Par exemple, on calcule $a^n$ 

	\begin{itemize}
		\item Approche naïve
			\begin{algorithm}
				\begin{algorithmic}[1]
					\State $p \gets 1$
					\For{$i \in \left\llbracket 0,n-1 \right\rrbracket$}
						\State $p \gets p \times a$
					\EndFor
					\State \Return p
				\end{algorithmic}
			\end{algorithm}
			Complexité linéaire $O(n)$
		\item Exponentiation rapide\\
			On écrit $n$ en binaire: \begin{align*}
				n &= \overline{a_k a_{k-1}\ldots a_0}^{(2)}\\
					&= \sum_{i=0}^{k} a_i 2^i
			\end{align*} avec $(a_i) \in \left\{ 0,1 \right\} ^{k+1}$
			\begin{align*}
				a^n &= a^{\sum_{i=0}^{k} a_i 2^i} \\
				&= \prod_{i=0}^{k} a^{a_i 2^i}  \\
			\end{align*}
			
			\begin{algorithm}
				\begin{algorithmic}
					[1]

					\State $s \gets 0$
					\State $p \gets a$
					\For{ $i \in \left\llbracket 0, \log_2(n) \right\rrbracket$}
						\State $p \gets p \times p$
						\If{$a[i] = 1$}
							\State $s \gets s + p$
						\EndIf
					\EndFor
					\State \Return s
				\end{algorithmic}
			\end{algorithm}
			Compléxité logarithmique $O(\log_2(n))$
	\end{itemize}
\end{exm}


\begin{prop}
	$O$ est une relation réfléctive et transitive.
\end{prop}

\begin{prv}
	\begin{itemize}
		\item Soit $u$ une suite. On pose $M = 1$ et \[
			\forall n \in \N, \left| u_n \right| \le M \left| u_n \right|
			\] Donc $u = O(u)$.
		\item Soient $u, v, w$ trois suites telles que  \[
		\begin{cases}
			u = O(v)\\
			v = O(w)
		\end{cases}
		\] Soient $M_1,M_2 \in \R$ et $N_1,N_2\in \N$ tels que \[
		\begin{cases}
			\forall n \ge  N_1, \left| u_n \right| \le M_1 \left| v_n \right| \\
			\forall n \ge  N_2, \left| v_n \right| \le M_2 \left| w_n \right| \\
		\end{cases}
		\] 

		Nécéssairement, $M_1\ge 0$ et $M_2\ge 0$.\\
		Soit $N = \max(N_1,N_2)$. \[
		\forall n \ge  N, \left| u_n \right| \le M_1 \left| v_n \right| \le  M_1M_2 \left| w_n \right| 
		\] Donc $u = O(w)$
	\end{itemize}
\end{prv}

\begin{defn}
	Soient $u$ et $v$ deux suites. On dit que $u$ est \underline{négligeable} devant $v$ si \[
	\forall \varepsilon>0, \exists N\in \N, \forall n\ge N, \left| u_n \right| \le \varepsilon \left| v_n \right| 
	\] Dans ce cas, on note $u = o(v)$ ou $u_n = o(v_n)$ ou on le lit "$u$ est un petit o de $v$"
\end{defn}

\begin{prop}
	$o$ est une relation transitive, non-réfléctive
\end{prop}

\begin{prv}
	\begin{itemize}
		\item Soient $u$, $v$ et $w$ trois suites telles que \[
			\begin{cases}
				u = o(v)\\
				v = o(w)
			\end{cases}
			\] Soit $\varepsilon>0$. Soit $N_1\in \N$ tel que \[
			\forall n \ge N_1, \left| u_n \right| \le \sqrt{\varepsilon}  \left| v_n \right| 
			\] Soit $N_2\in \N$ tel que \[
			\forall n \ge N_2, \left| v_n \right| \le \sqrt{\varepsilon}  \left| w_n \right| 
			\] On pose $N = \max(N_1,N_2)$, alors \[
			\forall n \ge N, \left| u_n \right| \le \sqrt{\varepsilon}  \left| v_n \right| \le \underbrace{\sqrt{\varepsilon} \times \sqrt{\varepsilon}} _\varepsilon \left| w_n \right| 
			\] donc $u = o(w)$
		\item Soit $u$ une suite tel qu'il existe $N \in \N$ tel que \[
		\forall n \ge N, u_n > 0
		\] On suppose que $u = o(u)$, alors \[
		\forall \varepsilon>0,\exists N \in \N, \forall n \ge N, \left| u_n \right| \le \varepsilon \left| u_n \right| 
		\] On pose $\varepsilon = \frac{1}{2}$ alors \[
		\exists N \in \N, \forall n \ge N, \left| u_n \right| \le \frac{1}{2} \left| u_n \right| 
		\] une contradiction
	\end{itemize}
\end{prv}

\begin{prop}
	Soient $u$ et $v$ deux suites.
	\begin{itemize}
		\item $o(u) + o(u) = o(u)$
		\item $v \times o(u) = o(uv)$
		\item $o(u) \times o(v) = o(uv)$
		\item $o(o(u)) = o(u)$
	\end{itemize}
	\qed
\end{prop}

\begin{defn}
	Soient $u$ et $v$ deux suites. On dit que $u$ et $v$ sont \underline{équivalentes} si \[
	u = v + o(v)
	\] i.e. \[
	\forall \varepsilon >0, \exists N \in \N, \forall n \ge N, \left| u_n-v_n \right| \le \varepsilon\left| v_n \right| 
	\] Dans ce cas, on le note $u \sim v$
\end{defn}

\begin{prop}
	$\sim$ est une relation d'équivalence \qed
\end{prop}

\begin{prop}
	Soient $(u,v) \in \R^\N$. On suppose que $v$ ne s'annule pas à partir d'un certain rang
	\begin{enumerate}
		\item $u = o(v) \iff \left( \frac{u_n}{v_n} \right)$ bornée
		\item $u = o(v) \iff \frac{u_n}{v_n} \tendsto{n \to  +\infty} 0$
		\item $u \sim v \iff \frac{u_n}{v_n} \tendsto{n \to  +\infty} 1$
	\end{enumerate}
	\qed
\end{prop}

\begin{prop}
	[Suites de références]
	\begin{enumerate}
		\item $\ln^\alpha(n) = o(n^\beta)$ avec $(\alpha,\beta) \in \left( \R^+_* \right) ^2$ 
		\item $n^\beta = o(a^n)$ avec $\beta > 0$ et $a > 1$ 
		\item $a^n = o(n!)$ avec $a >1$ 
		\item $n! = o(n^n)$
	\end{enumerate}
\end{prop}


\begin{lem}
	[Exercice 10 du TD]
	Soit $u \in \left(\R^+_*\right)^\N$\\
	Si $\frac{u_{n+1}}{u_n} \tendsto{n \to +\infty} \ell < 1$ avec $\ell\in \R$,\\ alors $u_n \tendsto{n \to +\infty} 0$
\end{lem}

\begin{prv} [de la proposition]
	\begin{enumerate}
		\item par croissance comparée
		\item On pose $\forall n \in \N^*, u_n = \frac{n^\beta}{a^n}$. 
			\begin{align*}
				\forall  n \in \N^*, \frac{u_{n+1}}{u_n} &= \left( \frac{n+1}{n} \right) ^\beta \times \frac{1}{a} \\
				&= \frac{1}{a}\left( 1+\frac{1}{n} \right) ^\beta \\
				&\tendsto{n \to +\infty} \frac{1}{a} < 1
			\end{align*}
			Donc, $u_n \tendsto{n \to  +\infty} 0$
		\item On pose $\forall n \in \N, u_n = \frac{a^n}{n!}$ \[
			\forall n \in \N, \frac{u_{n+1}}{u_n} = \frac{a}{n+1} \tendsto{n \to +\infty} 0 < 1
			\] donc $u_n \tendsto{n \to +\infty} 0$
		\item On pose $\forall  n\in \N^*, u_n = \frac{n!}{n^n}$.
			\begin{align*}
				\forall n \in \N^*, \frac{u_{n+1}}{u_n}
				&= (n+1) {\frac{n^n}{(n+1)^{n+1}}} \\
				&= \left( \frac{n}{n+1} \right) ^n \\
				&= e^{n \ln\left( \frac{n}{n+1} \right) } \\
				&= e^{n \ln\left( 1+\frac{1}{n+1} \right)} \\
				&= e^{n(-\frac{1}{n} + o(\frac{1}{n})} \\
				&= e^{-1 + o(1)} \\
				&\tendsto{n \to  +\infty} e^{-1}<1
			\end{align*}
			donc $u_n \tendsto{n\to +\infty} 0$
	\end{enumerate}
\end{prv}

	}

	{
		\chap[09]{Inégalités dans $\R$}
		\renewcommand{\cwd}{../chap09}
		\begin{defn}
	Soit $E$ un $\mathbbm{K}$-espace vectoriel. On dit que $E$ est de \underline{dimension finie} si $E$ a au moins une famille génératrice finie. On dit que $E$ est de \underline{dimension infinie} sinon.
	\index{dimension finie (espace vectoriel)}
	\index{dimension infinie (espace vectoriel)}
\end{defn}

\begin{thm}
	[Théorème de la base extraite]
	Soit $E$ un $\mathbbm{K}$-espace vectoriel non nul de dimension finie. Soit $\mathcal{G}$ une famille génératrice finie de $E$. Alors, il existe une base $\mathcal{B}$ de $\mathcal{E}$ telle que $\mathcal{B} \subset \mathcal{G}$.
\end{thm}

\begin{prv}
	[par récurrence sur $\#G = \Card(G)$]
	\begin{itemize}
		\item Soit $E$ un $\mathbbm{K}$-espace vectoriel non nul engendré par $\mathcal{G} = (u)$.\\
			Si $u = 0_E$, alors $E = \{0_E\}$: une contradiction $\lightning$ \\
			Donc $u \neq 0_E$ donc $(u)$ est libre. En effet, \[
				\forall \lambda \in \mathbbm{K}, \lambda u = 0_E \implies \lambda = 0_\mathbbm{K}
			\] Donc $\mathcal{G}$ est une base de $E$.\\
		\item Soit $n \in \N_*$. Soit $E$ un $\mathbbm{K}$-espace vectoriel. On suppose que si $E$ a une famille génératrice constituée de $n$ vecteurs, alors on peut extraire de cette famille une base de $E$.\\
			Soit $\mathcal{G}$ une famille génératrice de $E$ avec $n+1$ vecteurs.\\
			Si $\mathcal{G}$ est libre, alors $\mathcal{G}$ est une base de $E$. \\
			Si $\mathcal{G}$ n'est pas libre, alors il existe $u \in \mathcal{G}$ tel que $u \in \Vect(\mathcal{G}\setminus \{u\})$ \\
			Donc $\mathcal{G}\setminus \{u\}$ engendre $E$. Or, $\mathcal{G}\setminus \{u\}$ possède $n$ vecteurs. D'après l'hypothèse de récurrence, il existe une base $\mathcal{B}$ de $E$ telle que \[
				\mathcal{B} \subset \mathcal{G} \setminus \{u\} \subset \mathcal{G}
			\] 
	\end{itemize}
\end{prv}

\begin{crlr}
	Tout espace de dimension finie a une base.
	\qed
\end{crlr}

\begin{thm}
	[Théorème de la base incomplète]
	Soit $E$ un $\mathbbm{K}$-espace vectoriel de dimension finie, $\mathcal{G}$ une famille génératrice finie de $E$. $\mathcal{L}$ une famille libre de $E$. Alors, il existe une base $\mathcal{B}$ de $E$ telle que \[
		\mathcal{L} \subset \mathcal{B} \text{ et } \mathcal{B}\setminus \mathcal{L} \subset \mathcal{G}
	\] 
\end{thm}

\begin{prv}
	[par récurrence sur $\#(\mathcal{G}\setminus\mathcal{L})$]
	\begin{itemize}
		\item Avec les notations précédentes, on suppose que $\mathcal{G}\setminus\mathcal{L} \neq \O$ \[
				\forall u \in \mathcal{G}, u \in \mathcal{L}
			\] Donc $\mathcal{G} \subset \mathcal{L}$ donc $\mathcal{L}$ est génératrice donc $\mathcal{L}$ est une base de $E$. On pose $\mathcal{B} = \mathcal{L}$ et alors \[
				\mathcal{L} \subset  \mathcal{B} \text{ et } \mathcal{B}\setminus\mathcal{L} = \O \subset  \mathcal{G}
			\] 
		\item Soit $n \in \N$. On suppose que si $\mathcal{G}$ est génératrice et $\mathcal{L}$ libre avec $\#(\mathcal{G}\setminus\mathcal{L}) = n$ alors il existe une base $\mathcal{B}$ de $E$ telle que \[
			\mathcal{L}\subset \mathcal{B} \text{ et } \mathcal{B}\setminus\mathcal{L}\subset \mathcal{G}
		\] Soient à présent $\mathcal{G}$ une famille génératrice de $E$ et $\mathcal{L}$ une famille libre de $E$ telles que $\#(\mathcal{G}\setminus\mathcal{L}) = n+1 > 0$\\
		Si $\mathcal{L}$ engendre $E$, alors $\mathcal{L}$ est une base de $E$. On pose $\mathcal{B} = \mathcal{L}$ et on a bien \[
			\mathcal{L} \subset  \mathcal{B} \text{ et } \mathcal{B} \setminus \mathcal{L} = \O \subset  \mathcal{G}
		\] On suppose que $\mathcal{L}$ n'engendre pas $E$. Il existe $u \in \mathcal{G}$ tel que $u \not\in \Vec(\mathcal{L})$ (car sinon, $\mathcal{G} \subset \Vect(\mathcal{L})$ et donc $\underbrace{\Vect(\mathcal{G})}_{= E} \subset  \underbrace{\Vect(\mathcal{L})}_{ \subset E}$\\
		Donc $\mathcal{L} \cup \{u\} $ est libre. On pose $\mathcal{L}' = \mathcal{L} \cup \{u\} $ \[
			\mathcal{G}\setminus \mathcal{L}' = \mathcal{G}\setminus (\mathcal{L} \cup \{u\}) = (\mathcal{G}\setminus\mathcal{L})\setminus \{u\} 
		\] donc $\#(\mathcal{G}\setminus\mathcal{L}') = n+1 -1 = n$\\
		D'après l'hypothèse de récurrence, il existe $\mathcal{B}$ une base de $E$ telle que \[
			\mathcal{L} \subset  \mathcal{L}' \subset \mathcal{B} \text{ et } \mathcal{B}\setminus \mathcal{L}' \subset \mathcal{G}
		\] \[
			\mathcal{B} \setminus \mathcal{L} = \underbrace{\mathcal{B}\setminus\mathcal{L}'}_{\subset \mathcal{G}} \cup \underbrace{\{u\}}_{\subset \mathcal{G} \text{ car } u \in \mathcal{G}}
		\] On a $\mathcal{B}\setminus\mathcal{L}\subset \mathcal{G}$
	\end{itemize}
\end{prv}

\begin{thm}
	Soit $E$ un $\mathbbm{K}$-espace vectoriel de dimension finie. Toutes les bases de $E$ ont le même cardinal.
\end{thm}

\begin{prv}
	Soit $\mathcal{G}$ une famille génératrice finie de $E$ et $\mathcal{B} \subset  \mathcal{G}$ une base de $E$. On note $n = \#\mathcal{B}$ \\
	Soit $\mathcal{B}'$ une base de $E$. On pose $p = n - \#(\mathcal{B} \cap  \mathcal{B}')$. Montrons par récurrence sur  $p$ que $\#\mathcal{B} = \#\mathcal{B}'$ 
	\begin{itemize}
		\item On suppose que $p = 0$. Alors, $\#(\mathcal{B} \cap \mathcal{B}') = n$ \\
			Or, $\mathcal{B}' \cap \mathcal{B} \subset \mathcal{B}$ donc $\mathcal{B} \cap \mathcal{B}' = \mathcal{B}$ donc $\mathcal{B} \subset  \mathcal{B}'$ et donc $\mathcal{B} = \mathcal{B}'$ 
		\item Soit $p \in \N$. On suppose que si $\mathcal{B}'$ est une base de $E$ telle que $n - \#(\mathcal{B} \cap \mathcal{B}') = p$, alors $\#\mathcal{B}' = n$ \\
			Aoit $\mathcal{B}'$ une base de $E$ telle que $n - \#(\mathcal{B}\cap \mathcal{B}') = p+1 > 0$ \\
			Donc $\mathcal{B} \cap \mathcal{B}' \neq \mathcal{B}$. Soit $u \in \mathcal{B}' \setminus \mathcal{B}$. D'après le lemme d'échange, il existe $v \in \mathcal{B}\setminus \mathcal{B}'$ tel que $\mathcal{B}' \setminus \{u\} \cup \{v\}$ est une base de $E$. On pose $\mathcal{B}'' = \mathcal{B}' \setminus \{u\} \cup \{v\}$ 
			\begin{align*}
				\mathcal{B}'' \cap \mathcal{B} &= \left( (\mathcal{B}' \setminus \{u\})  \cap \mathcal{B} \right) \cup \{v\} \\
				&= (\mathcal{B}' \cap \mathcal{B}) \cup \{v\} \\
			\end{align*}
			donc,
			\begin{align*}
				n - \#(\mathcal{B}'' \cap \mathcal{B}) &= n - (\#(\mathcal{B}' \cap \mathcal{B}) + 1) \\
				&= p+1- 1 \\
				&= p \\
			\end{align*}
			D'après l'hypothèse de récurrence, \[
				\#\mathcal{B}'' = n
			\] Or, $\#\mathcal{B}'' = \#\mathcal{B}'$
	\end{itemize}
\end{prv}

\begin{lem}
	Soient $\mathcal{B}$ et $\mathcal{B}'$ deux bases de $E$ telles que $\mathcal{B}\subset \mathcal{B}'$. Alors, $\mathcal{B} = \mathcal{B}'$.
\end{lem}

\begin{prv}
	On suppose $\mathcal{B}' \neq \mathcal{B}$. Soit $u \in \mathcal{B}' \setminus \mathcal{B}$
	$u \in E = \Vect(\mathcal{B})$ donc $\mathcal{B} \cup \{u\}$ n'est pas libre.
	Donc $\mathcal{B}\cup \{u\} \subset \mathcal{B}'$ et $\mathcal{B}'$ est libre donc $\mathcal{B}\cup \{u\}$ est libre: une contradiction $\lightning$
\end{prv}

\begin{lem}
	[Lemme d'échange] Soient $\mathcal{B}_1$ et $\mathcal{B}_2$ deux bases de $E$ et $u \in \mathcal{B}_1 \setminus \mathcal{B}_2$. Alors, il existe $v \in \mathcal{B}_2$ tel que $(\mathcal{B}_1 \setminus \{u\}) \cup \{v\}$ soit une base de $E$.
\end{lem}

\begin{prv}
	[1${}^\text{nde}$ méthode]
	On suppose que pout tout $v \in \mathcal{B}_2$, $(\mathcal{B}_1\setminus \{u\}) \cup \{v\}$ n'est pas une base de $E$
	Soit $v \in \mathcal{B}_2$.
	\begin{itemize}
		\item Supposons $(\mathcal{B}_1\setminus \{u\})\cup \{v\}$ non libre. $\mathcal{B}_1 \setminus \{u\}$ est libre. Donc $v \in \Vect(\mathcal{B}_1 \setminus \{u\})$
		\item Supposons $(\mathcal{B}_1\setminus \{u\}) \cup \{v\}$ non génératrice.
			Comme $\mathcal{B}_1$ engendre $E$, $u \not\in \Vect(\mathcal{B}_1\setminus \{v\})$.
			On suppose que $\mathcal{B}_1 \neq \mathcal{B}_2$.
			$\forall v \in \mathcal{B}_2 \setminus \mathcal{B}_1, \Vect(\mathcal{B}_1 \setminus \{v\}) = \Vect(\mathcal{B}_1) = E \ni u$ 
			donc, $(\mathcal{B}_1\setminus \{u\}) \cup \{v\}$ engendre $E$ et donc \[
				v \in \Vect(\mathcal{B}_1 \setminus \{u\})
			\] On a aussi \[
				\forall v \in \mathcal{B}_1 \setminus \{u\}, v \in \Vect(\mathcal{B}_1\setminus \{u\})
			\] Comme $u \not\in \mathcal{B}_2$, on a \[
				\forall v \in \mathcal{B}_2, v \in \Vect(\mathcal{B}_1\setminus \{u\})
			\] docn \[
				E = \Vect(\mathcal{B}_2) \subset \Vect(\mathcal{B}_1\setminus \{u\})
			\] donc $\mathcal{B}_1\setminus \{u\}$ engendre $E$ donc $\mathcal{B}_1\setminus \{u\}$ est une base de $E$. Or, $\mathcal{B}_1 \setminus \{u\}  \subset  \mathcal{B}_1$, donc $\mathcal{B}_1\setminus \{u\} = \mathcal{B}_1$
	\end{itemize}
\end{prv}

\begin{prv}
	[2${}^\text{nde}$ méthode]
	On suppose que pout tout $v \in \mathcal{B}_2$, $(\mathcal{B}_1\setminus \{u\}) \cup \{v\}$ n'est pas une base de $E$
	\begin{itemize}
		\item Comme $u \in \mathcal{B}_1 \setminus \mathcal{B}_2$, nécéssairement $\mathcal{B}_1 \neq \mathcal{B}_2$ donc $\mathcal{B}_2 \not\subset \mathcal{B}_1$, donc $\mathcal{B}_2\setminus\mathcal{B}_1 \neq \O$ 
		\item Soit $v \in \mathcal{B}_2\setminus\mathcal{B}_1$. Il existe $(\lambda_w)_{w\in\mathcal{B}_1}$ une famille de scalaires presque nulle telle que \[
				v = \sum_{w \in \mathcal{B}_1} \lambda_w w - \lambda_u u + + \sum_{w \in \mathcal{B}_1\setminus \{u\}}\lambda_w w
			\]
			Si $\lambda_u \neq 0_E$, alors
			\begin{align*}
				u &= \lambda_u^{-1}\left( v - \sum_{w \in \mathcal{B}_1 \setminus \{u\}} \lambda_w w \right)\\
					&\in \Vect(\mathcal{B}_1\setminus \{u\} \cup v)
			\end{align*}
			 donc $\mathcal{B}_1 \subset \Vect(\mathcal{B}_1\setminus \{u\} \cup \{v\})$\\
			 et donc $E \subset  \Vect(\mathcal{B}_1 \setminus \{u\} \cup \{v\})$ \\
			 et donc $\mathcal{B}_1 \setminus \{u\} \cup \{v\}$ engendre $E$ \\
			 donc $\mathcal{B}_1 \setminus \{u\} \cup \{v\}$ n'est pas libre\\
			 donc $v \in \Vect(\mathcal{B}_1\setminus \{u\})$ (car $\mathcal{B}_1 \setminus \{u\}$ est libre\\
			 donc $\lambda_u = 0_\mathbbm{K}$ $\lightning$\\`

			 Donc, $\lambda_u = 0_\mathbbm{K}$, docn $v \in \Vect(\mathcal{B}_1\setminus \{u\})$ \\
			 On vient de prouver que
			 \begin{align*}
			 	\mathcal{B}_2 \setminus \mathcal{B}_1 \subset \Vect(\mathcal{B}_1 \setminus \{u\})\\
			 	\mathcal{B}_1 \setminus \{u\} \subset \Vect(\mathcal{B}_1 \setminus \{u\})\\
			 \end{align*}
			 Comme $u \not\in \mathcal{B}_2$, \[
			 	\mathcal{B}_2 \subset \Vect(\mathcal{B}_1 \setminus \{u\})
			 \] donc \[
			 	E = \Vect(\mathcal{B}_2) \subset  \Vect(\mathcal{B}_1 \setminus \{u\})
			 \] donc $\mathcal{B}_1 \setminus \{u\}$ engendre $E$. Donc,  $\mathcal{B}_1 \setminus \{u\}$ est une base de $E$.\\
			 Or, $\mathcal{B}_1 \setminus \{u\} \subset  \mathcal{B}_1$, donc $\mathcal{B}_1 \setminus \{u\} = \mathcal{B}_1$
	\end{itemize}
\end{prv}

\begin{defn}
	Soit $E$ un $\mathbbm{K}$-espace vectoriel de dimension finie. Le cardinal commun à toutes les bases de $E$ est appelé \underline{dimension} de $E$ est notée $\dim(E)$ ou $\dim_\mathbbm{K}(E)$\\
	C'est donc aussi le nombre de coordonnées de n'importe quel vecteur dans n'importe quelle base.
	\index{dimension (espace vectoriel)}
\end{defn}

\begin{exm}
	\begin{enumerate}
		\item $\dim_\R(\C) = 2$ et $\dim_\C(\C) = 1$ 
		\item $\dim_\mathbbm{K}(\mathbbm{K}^{n}) = n$ 
		\item $\dim_{\mathbbm{K}}(\mathcal{M}_{n,p}(\mathbbm{K})) = np$
	\end{enumerate}
\end{exm}

\begin{crlr}
	Soit $E$ un $\mathbbm{K}$-espace vectoriel de dimension finie, $\mathcal{L}$ une famille libre de $E$, $\mathcal{G}$ une famille génératrice de $E$. On note $n = \dim(E)$
	\begin{enumerate}
		\item $\#\mathcal{G} \ge n$ et $(\#\mathcal{G} = n \implies \mathcal{G} \text{ est une base de } E$)
		\item $\#\mathcal{L} \le n$ et $(\#\mathcal{L} = n \implies \mathcal{L} \text{ est une base de } E$)
	\end{enumerate}
\end{crlr}

\begin{crlr}
	$\R^{\R}$ est de dimension infinie.
	$\forall i \in \N, e_i: x \mapsto x^i$\\
	$(e_i)_{i\in\N}$ est libre dans $\R^\R$
\end{crlr}

\begin{prop}
	Soient $E$ et $F$ deux $\mathbbm{K}$-espaces vectoriels de dimension finie. Alors $E\times F$ est de dimension finie et $\dim(E\times F) = \dim(E) + \dim(F)$
\end{prop}

\begin{prv}
	Soit $(e_1,\ldots, e_n)$ une base de $E$, $(f_1, \ldots, f_p)$ une base de $F$.
	On pose \[
		\left\{\begin{array}
			{r c l}
			u_1 &=& (e_1,0_F)\\
			u_2 &=& (e_2,0_F)\\
					&\vdots&\\
			u_n &=& (e_n,0_F)\\
			u_{n+1} &=& (0_E, f_1)\\
			u_{n+2} &=& (0_E, f_2)\\
					&\vdots&\\
			u_{n+p} &=& (0_E,f_p)\\
		\end{array}\right.
	\]
	Soit $(x,y) \in E\times F$. \[
		\begin{cases}
			\exists (x_1,\ldots,x_n)\in \mathbbm{K}^n, x = \sum_{i=1}^{n} x_ie_i
			\exists (y_1,\ldots,y_n)\in \mathbbm{K}^n, x = \sum_{j=1}^{p} y_jf_j
		\end{cases}
	\] 
	\begin{align*}
		(x,y) &= \left( \sum_{i=1}^{n} x_ie_i, \sum_{i=1}^{p} y_jf_j \right)  \\
		&= \sum_{i=1}^{n} x_i (e_i + 0_F) + \sum_{j=1}^{p} y_j (0_E, f_j) \\
		&= \sum_{i=1}^{n} x_i u_i + \sum_{j=1}^{p} y_j u_{n+j} \\
	\end{align*}
	Donc, $E\times F = \Vect(u_1, \ldots, u_{n+p})$ donc $E\times F$ est de dimension finie.\\
	Soit $(\lambda_1, \ldots, \lambda_{n+p}) \in \mathbbm{K}^{n+p}$ tel que \[
		(*): \quad \sum_{k=1}^{n+p} \lambda_ku_k = 0_{E\times F} = (0_E, 0_F)
	\]
	\begin{align*}
		(*) &\iff \sum_{k=1}^{n} \lambda_k (e_k, 0_F) + \sum_{k=n+1}^{p} \lambda_k(0_E, f_{k-n}) = (0_E, 0_F)\\
				&\iff \begin{cases}
					\sum_{k=1}^{n} \lambda_k e_k = 0_E\\
					\sum_{k=n+1}^{p} \lambda_k f_{k-n} = 0_F
				\end{cases}\\
				&\iff \begin{cases}
					\forall k \in \left\llbracket 1,n \right\rrbracket, \lambda_k = 0_\mathbbm{K} \qquad&(\text{car $(e_1,\ldots,e_n)$ est libre})\\
					\forall k \in \left\llbracket n+1,n+p \right\rrbracket, \lambda_k = 0_\mathbbm{K} \qquad&(\text{car $(f_1,\ldots,f_n)$ est libre})\\
				\end{cases}
	\end{align*}
	Donc $(u_1, \ldots, u_{n+p})$ est une base de $E\times F$. Donc, $\dim(E\times F) = n + p = \dim(E) + \dim(F)$
\end{prv}

\begin{rmk}
	[Convention]
	\[\dim\big(\{0_E\}\big) = 0\]
\end{rmk}

\begin{thm}
	Soit $E$ un $\mathbbm{K}$-espace vectoriel de dimension finie, $F$ un sous-espace vectoriel de $E$. Alors, $F$ est de dimension finie et  $\dim(F) \le \dim(E)$\\
	Si $\dim(F) = \dim(E)$, alors $F = E$
\end{thm}

\begin{prv}
	On considère \[
		A = \{k \in \N \mid \text{il existe une famille libre de $F$ à $k$ éléments}\} 
	\]
	On suppose $F \neq \{0_E\}$.
	\begin{itemize}
		\item Soit $u \in F\setminus \{0_E\}$. $(u)$ est libre donc $1 \in A$ et donc $A \neq \O$
		\item Soit $\mathcal{L}$ une famille libre de $F$. Alors, $\mathcal{L}$ est une famille libre de $E$ \\
			donc $\#\mathcal{L} \le \dim(E)$\\
			Donc $A$ est majorée par $\dim(E)$ \\
			On en déduit que $A$ a un plus grand élément $p$.
		\item Soit $\mathcal{L}$ une famille libre de $F$ avec $p$ éléments.\\
			Si $\mathcal{L}$ n'engendre pas $F$, alors il existe $u\in F$ tel que $u\not\in \Vect(\mathcal{L})$ et donc $\mathcal{L} \cup \{u\}$ est une famille libre de $F$, donc $p+1 \in A$ en contradiction avec la maximalité de $p$.\\
			Donc $\mathcal{L}$ est une base de $F$ donc $F$ est de dimension finie et $\dim(F) = p \le \dim(E)$\\
	\end{itemize}

	Soit $\mathcal{B}$ une base de $F$. Alors, $\mathcal{B}$ est aussi une famille de libre de de $E$. Donc $\#\mathcal{B} \le \dim(E)$ donc $\dim(F) = \dim(E)$ \\
	Si $\dim(F) = \dim(E)$, alors $\mathcal{B}$ est une base de $E$, et donc $F = \Vect(\mathcal{B}) = E$
\end{prv}

\begin{prop}
	[Formule de Grassmann]
	Soit $E$ un $\mathbbm{K}$-espace vectoriel de dimension finie, $F$ et $G$ deux sous-espace vectoriels de $E$. Alors, \[
		\dim(F+G) = \dim(F) + \dim(G) - \dim(F\cap G)
	\] 
\end{prop}

\begin{prv}
	Soit $(e_1, \ldots, e_p)$ une base de $F\cap G$. $(e_1,\ldots,e_p)$ est une famille libre de $F$.\\
	On complète $(e_1, \ldots, e_p)$ en une base $(e_1, \ldots, e_p, u_1, \ldots, u_q)$ de $F$.\\
	De même, on complète $(e_1, \ldots, e_p)$ en une base $(e_1, \ldots, e_p, v_1, \ldots, v_r)$ de $G$.\\
	On pose  $\mathcal{B} = (e_1, \ldots, e_p, u_1, \ldots, u_q, v_1, \ldots, v_r)$. Montrons que $\mathcal{B}$ est une base de $F+G$
	\begin{itemize}
		\item Soit $u \in F+G$ \\
			On pose $u = v+w$ avec $\begin{cases}
				v\in F\\
				w \in G
			\end{cases}$.\\
			On pose $v = \sum_{i=1}^p \lambda_i e_i + \sum_{i=1}^q \mu_i u_i$ avec $(\lambda_1, \ldots, \lambda_p, \mu_1, \ldots, \lambda_q) \in \mathbbm{K}^{p+q}$\\
			On pose aussi $w = \sum_{i = 1}^p \lambda'_ie_i + \sum_{j=1}^r \nu_j v_j$ avec $(\lambda_1',\ldots,\lambda_p', \nu_1, \ldots, \nu_r) \in \mathbbm{K}^{p+r}$\\
			D'où, \[
				u = \sum_{i=1}^p (\lambda_i + \lambda'_i)e_i + \sum_{j=1}^q \mu_j u_j + \sum_{k=1}^r \nu_k v_k \in \Vect(\mathcal{B})
			\]
		\item Soient $(\lambda_1, \ldots, \lambda_p, \mu_1, \ldots, \mu_q, \nu_1, \ldots, \nu_r) \in \mathbbm{K}^{p+q+r}$.\\
			On suppose \[
				(*)\quad \sum_{i=1}^{p}\lambda_ie_i + \sum_{j=1}^q\mu_ju_j + \sum_{k=1}^r \nu_k v_k = 0_E
			\] 
			D'où, \[
				\underbrace{\sum_{i=1}^p\lambda_i e_i + \sum_{j=1}^q \mu_ju_j}_{\in F} = \underbrace{-\sum_{k=1}^r\nu_jv_k}_{\in G}
			\] 
			Donc, \[
				f = \sum_{i=1}^p \lambda_i e_i + \sum_{j=1}^q \mu_j u_j \in F\cap G
			\] Comme $(e_1, \ldots, e_p)$ est une base de $F\cap G$, $\exists ! (\lambda_1', \ldots, \lambda_p') \in \mathbbm{K}^p$ tel que \[
				f = \sum_{i=1}^p \lambda'_i e_i = \sum_{i=1}^p \lambda'_i e_i + \sum_{j=1}^q 0_\mathbbm{K}u_j
			\] Comme $(e_1, \ldots, e_p, u_1, \ldots, u_q)$ est une base de $F$, \[
				\forall k \in \left\llbracket 1, q \right\rrbracket, \mu_j = 0_\mathbbm{K}
			\] De même, \[
				\forall k \in \left\llbracket 1,r \right\rrbracket , \nu_k = 0_\mathbbm{K}
			\] On remplace dans $(*)$ pour trouver \[
				\sum_{i=1}^p \lambda_ie_i = 0_E
			\] Comme $(e_1, \ldots, e_p)$ est libre, \[
				\forall i \in \left\llbracket 1,p \right\rrbracket, \lambda_i = 0_\mathbbm{K}
			\] Donc $\mathcal{B}$ est libre.\\
			Donc, 
			\begin{align*}
				\dim(F+G) &=  p +q + r \\
				&= (p+q)+ (p+r) - p \\
				&= \dim(F) + \dim(G) - \dim(F\cap G) \\
			\end{align*}
	\end{itemize}
\end{prv}

\begin{crlr}
	Avec les hypothèse précédentes, \[
		E = F \oplus G \iff \begin{cases}
			F \cap  G = \{0_E\} \\
			\dim(E) = \dim(F) + \dim(G)
		\end{cases}
	\] 
\end{crlr}

\begin{prv}
	\begin{itemize}
		\item[``$\implies$''] On suppose $E = F \oplus G$ \\
			Comme la somme est directe, $F \cap G = \{0_E\}$ 
			\begin{align*}
				\dim(E) &= \dim(F)\\
				&= \dim(F) + \dim(G) - \dim(F\cap G)\\
				&= \dim(F) + \dim(G)\\
			\end{align*}
		\item[``$\impliedby$''] On suppose $F\cap G = \{0_E\}$ et $\dim(E) = \dim(F) + \dim(G)$.\\
			On sait déjà que $F+G = F \oplus G$\\
			 \begin{align*}
				\dim(F+G) = \dim(F) + \dim(G) - \dim(F \cap G) = \dim(E)
			\end{align*}
			Donc $F + G = E$
	\end{itemize}
\end{prv}

\begin{prop}
	Soit $F$ un $\mathbbm{K}$-espace vectoriel de dimension finie $n$. Soit $\mathcal{B} = (e_1, \ldots, e_n)$ une base de $F$. L'application
	\begin{align*}
		f: \mathbbm{K}^n &\longrightarrow F \\
		(\lambda_1, \ldots, \lambda_n) &\longmapsto \sum_{i=1}^n \lambda_i e_i
	\end{align*} est bijective.\\
	Si $\mathbbm{K}$ est infini, $\mathbbm{K}^n$ aussi et donc $F$ aussi.\\
	Si $\#\mathbbm{K} = p \in \N_*$,
	\begin{align*}
		\#&\mathbbm{K}^n = p^n\\
		&\vrt=\\
		\#&F
	\end{align*}
\end{prop}


		\part{Dérivation}

\underline{Motivation}:

{
\begin{wrapfigure}{l}{3cm}
	\centering
	\begin{asy}
		import three;

		size(3cm);
		settings.render=0;
		settings.prc=false;
		currentprojection = obliqueZ;

		draw(unitbox);
		draw(shift(1.1Z + 0.05X) * (O -- X), Arrows3(TeXHead2));
		draw(shift(1.1Z + 0.05Y) * (O -- Y), Arrows3(TeXHead2));
		draw(shift(1.1X + 0.05Z) * (O -- Z), Arrows3(TeXHead2));

		label("$x$", (X/2) + (1.1Z + 0.05X), align=S);
		label("$y$", (Y/2) + (1.1Z + 0.05Y), align=W);
		label("$z$", (Z/2) + X, align=SE);
	\end{asy}
\end{wrapfigure}

\begin{align*}
	&S(x,y,z) = 2(xy + xz + yz)\\
	&V(x,y,z) = xyz
\end{align*}

On cherche à minimiser $S$ avec la contrainte $V = 1$.

Soit $f : \begin{array}{rcl}
	\left( \R_*^+ \right)^2 &\longrightarrow& \R \\
	(x,y) &\longmapsto& S\left( x,y,\frac{1}{xy} \right) = 2\left( xy + \frac{1}{y} + \frac{1}{x} \right).
\end{array}$

On cherche $(a,b) \in \left( \R^+_* \right)^2$ tel que \[
	\forall (x,y) \in (\R^+_*), f(x,y) \ge f(a,b).
\]
}

\begin{defn}
	Soit $f: U \to \R$ où $U$ est un ouvert de $\R^2$. Soit $(a,b) \in U$.
	\vspace{2mm}

	Si $\lim_{x \to a} \frac{f(x,b) - f(a,b)}{x - a} \in \R$, alors on dit que $f$ a une dérivée partielle suivant $x$ en $(a,b)$ et cette limite est notée \[
		\partial f_1(a,b) = \frac{\partial f}{\partial x}(a,b).
	\]

	Si $\lim_{y \to b} \frac{f(a,y) - f(a,b)}{y - b} \in \R$, alors on dit que $f$ a une dérivée partielle suivant $y$ et la limite est notée \[
		\partial f_2(a,b) = \frac{\partial f}{\partial y}(a,b).
	\]
\end{defn}

\begin{exm}
	\begin{enumerate}
		\item $f: (x,y) \mapsto xy + x - y$.

			\begin{align*}
				&\frac{\partial f}{\partial x} : (x,y) \mapsto y + 1,\\
				&\frac{\partial f}{\partial y} : (x,y) \mapsto x - 1.
			\end{align*}

		\item $f: (x,y) \mapsto xy + \frac{1}{y}+ \frac{1}{x}$.

			\begin{align*}
				&\frac{\partial f}{\partial x}: (x,y) \mapsto y - \frac{1}{x^2},\\
				&\frac{\partial f}{\partial y}: (x,y) \mapsto x - \frac{1}{y^2}.
			\end{align*}

		\item Trouver $f$ telle que $\begin{cases}
				(1): \qquad \frac{\partial f}{\partial x}=y,\\[2mm]
				(2): \qquad \frac{\partial f}{\partial y} = x.
			\end{cases}$

			D'après $(1)$ : \[
				\forall (x,y), \exists C(y) \in \R, f(x,y) = xy + C(y)
			\] et donc \[
				\frac{\partial f}{\partial y}(x,y) = x + C'(y)
			\] donc $C'(y) = 0$ et donc $C$ est constante.

		\item Trouver $f$ telle que $\begin{cases}
			\frac{\partial f}{\partial x} = -y,\\[2mm]
			\frac{\partial f}{ƒ\partial y} = x.
		\end{cases}$

		Ce n'est pas possible !
	\end{enumerate}
\end{exm}

\begin{defn}~\\
	\begin{minipage}{\linewidth}
		\begin{wrapfigure}{r}{4cm}
			\centering
			\vspace{-5mm}
			\begin{asy}
				import three;
				import graph3;
				size(4cm);

				settings.render = 0;
				settings.prc = false;
				currentprojection = obliqueX;

				draw(O -- X, Arrow3(TeXHead2));
				draw(O -- Y, Arrow3(TeXHead2));
				draw(O -- Z, Arrow3(TeXHead2));

				triple f(real x, real y, real z = 0) { return (x,y,cos(x - 0.5) * cos(y - 0.5)/1.2 + 0.15); }

				real inc = 1 / 5;

				for(real x = 0; x <= 1; x += inc) {
					draw(graph(
						new real(real t) { return x; }, // x
						new real(real y) { return y; }, // y
						new real(real y) { return f(x,y).z; }, // z
						0, 1
					), gray);
				}

				for(real y = 0; y <= 1; y += inc) {
					draw(graph(
						new real(real x) { return x; }, // x
						new real(real t) { return y; }, // y
						new real(real x) { return f(x,y).z; }, // z
						0, 1
					), gray);
				}

				path3 path1 = (0.8, 0.2, 0) .. (0.5, 0.5, 0) .. (0.3, 0.7, 0);
				path3 path2 = f(0.8, 0.2, 0) .. f(0.5, 0.5, 0) .. f(0.3, 0.7, 0);
				path3 d = (0.2, 0.3, 0) .. (0.3, 0.4, 0) .. (0.2, 0.7, 0) .. (0.8, 0.9, 0) .. (0.6, 0.2, 0) .. cycle;

				draw(path1, red, Arrow3(TeXHead2));
				draw(path2, red, Arrow3(TeXHead2, position=0.8));

				dot((0.5, 0.5, 0));
				dot(f(0.5, 0.5, 0));
				draw((0.5, 0.5, 0) -- f(0.5, 0.5, 0), dashed);
				draw(d);

				label("$w$", (0.3, 0.7, 0), red, align=SE);
				label("$U$", (0.8, 0.9, 0), align=SE);
			\end{asy}
		\end{wrapfigure}

		Soit $f: U \to \R$ où $U$ est un ouvert. Soit $(a,b) \in U$. Soit $w = (w_1, w_2) \in \R^2$.

		Si 
		\[
			\lim_{t\to 0} \frac{f(a + tw_1, b + tw_2) - f(a,b)}{t}
		\] existe et est réelle, alors on dit que $f$ a une dérivée dans la direction de $w$ et la limite est notée \[
			\mathrm{d}f(w)\,(a,b) = D_w(f)\,(a,b).
		\]
	\end{minipage}
\end{defn}

\begin{exm}
	\begin{align*}
		f: \left( \R_*^+ \right)^2 &\longrightarrow \R \\
		(x,y) &\longmapsto xy+\frac{1}{x}+\frac{1}{y}.
	\end{align*}

	On pose $(a,b) = (1,2)$, $w = (w_1, w_2) = (1,1)$.
	\begin{align*}
		\frac{f(1+t, 2+t) - f(1,2)}{t} &= \frac{1}{t} \left( (1+t)(2+t) + \frac{1}{1+t} + \frac{1}{2+t} - 3 - \frac{1}{2} \right) \\
		&= \frac{1}{t}\left(\cancel 2 + 3t + \po(t) + \cancel 1 - t + \po(t) + \frac{1}{2}\left( \cancel 1 - \frac{t}{2} + \po(t) \right) - \cancel3 - \cancel{\frac{1}{2}} \right) \\
		&= \frac{1}{t} \left( \frac{7}{4} t + \po(t) \right)  \\
		&= \frac{7}{4} + \po(1) \tendsto{t \to 0} \frac{7}{4}. \\
	\end{align*}

	Donc, \[
		\mathrm{d}f(1,1)\,(1,2) = \frac{7}{4}.
	\]
\end{exm}

\begin{rmk}~\\
	\begin{figure}[H]
		\centering
		\begin{asy}
			import solids;
			import graph;
			size(5cm);

			settings.render = 0;
			settings.prc = false;

			path3 par = graph(
				new real(real x) { return x; },
				new real(real x) { return 0; },
				new real(real x) { return x^2; },
				0,3);
			revolution r = revolution(par, axis=Z);

			path3 par2 = graph(
				new real(real x) { return x; },
				new real(real x) { return 0; },
				new real(real x) { return x^2; },
				-3,3);

			draw(r,1,longitudinalpen=nullpen);
			draw(r.silhouette());

			draw((-4, 0, -1) -- (-4, 0, 10) -- (4, 0, 10) -- (4, 0, -1) -- cycle, red);
			draw(par2, deepred);

			draw((4,4.5) -- (7, 4.5), black+0.5mm, Arrow(TeXHead));

			path par2d = graph(new real(real x) { return x^2; }, -3, 3);
			draw(shift((11, 0)) * par2d, deepred);

			dot(O);
			dot((11, 0));
		\end{asy}
	\end{figure}
\end{rmk}


%todo ajouter théorème-définition
\begin{thm}
	Soit $f : U \to \R$, $(a,b) \in U$. On suppose que $\frac{\partial f}{\partial x}$ et $\frac{\partial f}{\partial y}$ existent en $(a,b)$ et sont {\bfseries continues} en $(a,b)$. Alors,
	\begin{align*}
		&\forall (h, k) \in \R^2 \text{ tel que } (a +h, b + k) \in U,\\
		&f(a+ h, b + k) = f(a,b) + h \frac{\partial f}{\partial x}(a,b) + k \frac{\partial f}{\partial y}(a,b) + \po_{(h,k)\to (0,0)}\big(\|(h,k)\|\big).
	\end{align*}

	On dit que $f$ est de classe $\mathcal{C}^1$ si $\frac{\partial f}{\partial x}$ et $\frac{\partial f}{\partial y}$ existent et sont continues.

	\qed
\end{thm}

\begin{rmk}
	En physique, cette formule correspond à : \[
		\mathrm{d}f = \frac{\partial f}{\partial x}\mathrm{d}x + \frac{\partial f}{\partial y} \mathrm{d}y.
	\] En effet :
	\begin{align*}
		\mathrm{d}f &= f(x+ \mathrm{d}x, y + \mathrm{d}y) - f(x,y) \\
		&= \frac{\partial f}{\partial x} \mathrm{d}x + \frac{\partial f}{\partial y} \mathrm{d}y.
	\end{align*}
\end{rmk}

\begin{prop}
	Soit $f: U \to \R$ de classe $\mathcal{C}^1$ en $(a,b) \in U$. Alors, \[
		\forall w = (w_1, w_2) \in \R^2, \mathrm{d}f(w)\,(a,b) = w_1 \frac{\partial f}{\partial x}(a,b) + w_2 \frac{\partial f}{\partial y}(a,b).
	\]
\end{prop}

\begin{prv}
	Soit $w = (w_1, w_2) \in \R^2$. Soit $t \in \R^*$.
	\begin{align*}
		\frac{1}{t}\big(f(a + tw_1, b + tw_2) - f(a,b)\big)
		&= \frac{1}{t} \left( tw_1 \frac{\partial f}{\partial x}(a,b) + tw_2 \frac{\partial f}{\partial y}(a,b) + \po_{t \to 0}\big(\|tw\|\big) \right) \\
		&= w_1 \frac{\partial f}{\partial x}(a,b) + w_2 \frac{\partial f}{\partial y}(a,b) + \po_{t\to 0}(1) \\
		&\tendsto{t\to 0} w_1 \frac{\partial f}{\partial x}(a,b) + w_2\frac{\partial f}{\partial y}(a,b).
	\end{align*}
\end{prv}


\begin{defn}
	Avec les hypothèses précédentes, en posant \[
		\nabla f(a,b) = \left( \frac{\partial f}{\partial x}(a,b), \frac{\partial f}{\partial y}(a,b) \right) 
	\]on obtient \[
		\mathrm{d}f(w)\,(a,b) = \left<w  \mid \nabla f(a,b) \right>
	\] où $\left<\cdot|\cdot \right>$ est le produit scalaire.

	Le vecteur $\nabla f(a,b)$ est appelé \underline{gradient de $f$ en $(a,b)$}.

	Le développement limité à l'ordre 1 de $f$ devient \[
		f\big((a,b)+w\big) = f(a,b) + \left<w \mid \nabla f(a,b) \right> + \po_{w\to 0}(\|w\|)
	\]
\end{defn}

\begin{prop}
	Soit $f : U \to \R$ de classe $\mathcal{C}^1$.

	\begin{figure}[H]
    \centering
    \incfig{gradient}
	\end{figure}

	$\nabla f$ est orthogonal au lignes de niveaux de $f$, son orientation va dans le sens d'une augmentation de $f$.
\end{prop}

\begin{prv}
	Soit $\gamma : I \to U$ une courbe de niveau : \[
		\forall t \in I, f\big(\gamma(t)\big) = \text{cste}.
	\] D'après le lemme suivant : \[
		\forall t \in I, 0 = (f \circ \gamma)'(t) = \mathrm{d}f\big(\gamma'(t)\big)\big(\gamma(t)\big) = \left<\gamma'(t)  \mid \nabla f\big(\gamma(t)\big) \right>
	\] Donc $\nabla f\big(\gamma(t)\big)$ est orthogonal à $\gamma'(t)$.

	Pour tout $t \in I$, on pose $w(t) = t\, \nabla f\big(\gamma(t)\big)$. Donc \[
		f\big(\gamma(t) + w(t)\big) = f\big(\gamma(t)\big) + t \|\nabla f(\gamma(t))\|^2 + \po_{t \to 0}(t)
	\] Pour $t$ assez petit, $f\big(\gamma(t) + w(t)\big) - f\big(\gamma(t)\big)$ est du même signe que $t$.
\end{prv}

\begin{rmk}
	\begin{align*}
		V: \R^3 &\longrightarrow \R \\
		(x,y,z) &\longmapsto -mgz
	\end{align*}
	l'énerge potentielle de pesenteur

	On a donc \[
		\nabla V(x,y,z) = \left( \frac{\partial V}{\partial x}, \frac{\partial V}{\partial y}, \frac{\partial V}{\partial z} \right) = (0, 0, -mg) = \vec{P}.
	\]
\end{rmk}

\begin{lem}
	Soit $f : U \to \R$ de classe $\mathcal{C}^1$, $\gamma : \begin{array}{rcl}
		I &\longrightarrow& U \\
		t &\longmapsto& \big(x(t), y(t)\big)
	\end{array}$ où $x$ et $y$ sont dérivables.

	On pose \[
		\forall t \in I, \gamma'(t) = \big(x'(t), y'(t)\big).
	\] Alors $f \circ \gamma : I \to \R$ est dérivable et
	\begin{align*}
		\forall t \in I, (f \circ \gamma)'(t) &= \mathrm{d}f\big(\gamma'(t)\big) \big(\gamma(t)\big)\\
		&= \left<\gamma'(t)  \mid \nabla f\big(\gamma(t)\big)  \right> \\
		&= x'(t) \frac{\partial f}{\partial x}\big(x(t), y(t)\big) + y'(t) \frac{\partial f}{\partial y}\big(x(t),y(t)\big). \\
	\end{align*}
\end{lem}

\begin{prv}
	On fixe $t \in I$.

	\begin{align*}
		\forall h \neq 0, \frac{f \circ \gamma(t + h) - f \circ \gamma(t)}{h}
		&= \frac{1}{h}\big(f(\gamma(t)) + h\gamma'(t) + \po_{h\to 0}(h) - f(\gamma(t))\big) \\
		&= \frac{1}{h}\bigg(\cancel{f(\gamma(t))} + \left<h\gamma'(t) \mid \nabla f(\gamma(t)) \right> + \po_{h\to 0}(\|h\gamma'(t)\|) - \cancel{f(\gamma(t))}\bigg)\\
		&= \left<\gamma'(t) \mid \nabla f(\gamma(t)) \right> + \po_{h\to 0}(1) \\
		&\tendsto{h\to 0} \left<\gamma'(t)  \mid \nabla f(\gamma(t)) \right>
	\end{align*}
\end{prv}

\begin{defn}
	Soit $f : U \to \R$ de classe $\mathcal{C}^1$ et $(a,b) \in U$. On dit que $(a,b)$ est un \underline{point critique} de $f$ si $\nabla f(a,b) = 0$ i.e. $\frac{\partial f}{\partial x}(a,b) = \frac{\partial f}{\partial y}(a,b) = 0$.

	Dans ce cas, $f(a,b)$ est appelé \underline{valeur critique} de $f$.
\end{defn}

\begin{prop}~\\
	\begin{minipage}{\linewidth}
		\begin{wrapfigure}{r}{3cm}
			\centering
			\vspace{-1cm}
			\begin{asy}
				import solids;
				import graph;
				size(3cm);

				settings.render = 0;
				settings.prc = false;

				path3 par = graph(
					new real(real x) { return x; },
					new real(real x) { return 0; },
					new real(real x) { return -x^2; },
					0,3);
				revolution r = revolution(par, axis=Z);

				draw(r,1,longitudinalpen=nullpen);
				draw(r.silhouette());

				dot("$(a,b)$", O, red, align=N);
				real s = sqrt(2.5);
				path3 g=(s,0,-2.5)..(0,s,-2.5)..(-s,0,-2.5)..(0,-s,-2.5)..cycle;
				draw(g, deepcyan);
			\end{asy}
		\end{wrapfigure}
		Soit $f: U \to \R$ de classe $\mathcal{C}^1$ et $(a,b) \in U$ tel que \[
			\exists r > 0, \forall (x,y) \in B_{(a,b)}(r), f(x,y) \le f(a,b)
		\] Alors $\nabla f(a,b) = (0,0)$.
	\end{minipage}
\end{prop}

\begin{prv}
	Soit $g: x \mapsto f(x,b)$. $g(a)$ est un maximum local de $g$ donc $g'(a) = 0$.

	Or, $g'(a) = \frac{\partial f}{\partial x}(a,b)$

	donc $\frac{\partial f}{\partial x}(a,b) = 0$.

	Soit $h : y \mapsto f(a,y)$. On a de même $h'(b) = 0$.

	Or, $h'(b) = \frac{\partial f}{\partial y}(a,b)$.

	Donc, $\nabla f(a,b) = (0,0)$.
\end{prv}

\begin{rmk}
	Un minimum local est aussi une valeur critique.
\end{rmk}

\begin{figure}[H]
	\centering
	\begin{subfigure}{3cm}
		\centering
		\begin{asy}
			import solids;
			import graph;
			size(3cm);

			settings.render = 0;
			settings.prc = false;

			path3 par = graph(
				new real(real x) { return x; },
				new real(real x) { return 0; },
				new real(real x) { return -x^2; },
				0,3);
			revolution r = revolution(par, axis=Z);

			draw(r,1,longitudinalpen=nullpen);
			draw(r.silhouette());

			dot(O, red);
		\end{asy}
		\caption{Maximum local}
	\end{subfigure}
	\begin{subfigure}{3cm}
		\centering
		\begin{asy}
			import solids;
			import graph;
			size(3cm);

			settings.render = 0;
			settings.prc = false;

			path3 par = graph(
				new real(real x) { return x; },
				new real(real x) { return 0; },
				new real(real x) { return x^2; },
				0,3);
			revolution r = revolution(par, axis=Z);

			draw(r,1,longitudinalpen=nullpen);
			draw(r.silhouette());

			dot(O, red);
		\end{asy}
		\caption{Minimum local}
	\end{subfigure}
	\begin{subfigure}{3cm}
		\centering
		\begin{asy}
			import solids;
			import graph;
			size(3cm);

			settings.render = 0;
			settings.prc = false;
			currentprojection = obliqueZ;

			draw(graph(
				new real(real x) { return x; },
				new real(real x) { return -x^2 / 3; },
				new real(real x) { return 3; },
				-3, 3
			));

			draw(graph(
				new real(real x) { return x; },
				new real(real x) { return -x^2 / 3; },
				new real(real x) { return -3; },
				-3, 3
			));

			draw(graph(
				new real(real x) { return x; },
				new real(real x) { return -x^2 / 3 - 1; },
				new real(real x) { return 0; },
				-3, 3
			));

			draw(graph(
				new real(real x) { return 0; },
				new real(real x) { return x^2 / 9 - 1; },
				new real(real x) { return x; },
				-3, 3
			));

			draw(graph(
				new real(real x) { return -3; },
				new real(real x) { return x^2 / 9 - 4; },
				new real(real x) { return x; },
				-3, 3
			));

			draw(graph(
				new real(real x) { return 3; },
				new real(real x) { return x^2 / 9 - 4; },
				new real(real x) { return x; },
				-3, 3
			));

			dot((0,-1,0), red);
		\end{asy}
		\caption{Point de selle / Point col}
	\end{subfigure}
\end{figure}

\begin{exm}
	On revient à l'exemple donné en introduction : 
	\begin{align*}
		f: \left( \R^*_+ \right)^2 &\longrightarrow \R \\
		(x,y) &\longmapsto 2\left( xy + \frac{1}{x} + \frac{1}{y} \right).
	\end{align*}

	$\left( \R^+_* \right)^2$ est un ouvert de $\R^2$. Soit $(x,y) \in \left( \R^+_* \right)^2$.
	
	On a \[
		\begin{cases}
			\frac{\partial f}{\partial x}(x,y) = 2\left( y - \frac{1}{x^2} \right),\\
			\frac{\partial f}{\partial y}(x,y) = 2\left( x - \frac{1}{y^2} \right).
		\end{cases}
	\]

	\begin{align*}
		&\frac{\partial f}{\partial x}(x,y) = \frac{\partial f}{\partial y}(x,y) = 0\\
		\iff& \begin{cases}
			y = \frac{1}{x^2}\\
			x = \frac{1}{y^2}
		\end{cases}\\
		\iff& \begin{cases}
			y = \frac{1}{x^2}\\
			x = x^4
		\end{cases}\\
		\iff& \begin{cases}
			x = 1\\
			y = 1
		\end{cases}
	\end{align*}

	On vérivie que $f$ présente en effet un minium local en $(1,1)$. \[
		f(1,1) = 6
	\] On fixe $y \in \R^+_*$ et \[
		g : x \mapsto 2\left( xy + \frac{1}{x} + \frac{1}{y} \right).
	\] Donc \[
		\forall x \in \R^+_*, g'(x) = 2\left( y - \frac{1}{x^2} \right).
	\]
	\begin{center}
		\begin{tikzpicture}
			\tkzTabInit{$x$/1,$g'(x)$/1,$g$/2.3}{$0$, $\frac{1}{\sqrt{y}}$, $+\infty$}
			\tkzTabLine{,-,z,+,}
			\tkzTabVar{+/{}, -/$2\left( 2\sqrt{y} +\frac{1}{y} \right)$, +/{}}
		\end{tikzpicture}
	\end{center}
	
	Ainsi, \[
		\forall x \in \R^+_*, \forall y \in \R^+_*, f(x,y) \ge 2\left( 2\sqrt{y} + \frac{1}{y} \right)
	\] Soit $h : y \mapsto 2\sqrt{y} + \frac{1}{y}$. On a \[
		\forall y > 0, h'(y) = \frac{1}{\sqrt{y}} - \frac{1}{y^2} = \frac{y\sqrt{y} - 1}{y^2} = \frac{y^{\frac{3}{2}} - 1}{y^2}
	\]

	\begin{center}
		\begin{tikzpicture}
			\tkzTabInit{$y$/0.7,$h'(y)$/0.7,$h$/1.4}{$0$, $1$, $+\infty$}
			\tkzTabLine{,-,z,+,}
			\tkzTabVar{+/{}, -/$3$, +/{}}
		\end{tikzpicture}
	\end{center}

	Donc, \[
		\forall x,y > 0, f(x,y) \ge 2\times 3 = 6 = f(1,1).
	\]
\end{exm}

\begin{prop}
	[règle de la chaîne]

	Soit $f : \begin{array}{rcl}
		U &\longrightarrow& \R^2 \\
		(x,y) &\longmapsto& f(x,y)
	\end{array}$ de classe $\mathcal{C}^1$ et $U, V$ deux ouverts de $\R^2$.

	Soit $\varphi : \begin{array}{rcl}
		V &\longrightarrow& U \\
		(u,v) &\longmapsto& \varphi(u,v) = \big(x(u,v), y(u,v)\big)
	\end{array}$.

	On suppose que $x$ et $y$ sont de classe $\mathcal{C}^1$ sur $V$.

	Alors,  $f \circ \varphi : \begin{array}{rcl}
		V &\longrightarrow& \R \\
		(u,v) &\longmapsto& f\big(\varphi(u,v)\big)
	\end{array}$ est de classe $\mathcal{C}^1$ et
	\begin{align*}
		\forall (u_0, v_0) \in V, \frac{\partial (f \circ \varphi)}{\partial u}(u_0, v_0)
		&= \frac{\partial f}{\partial x}\big(\varphi(u_0, v_0)\big) \times \frac{\partial x}{\partial u}(u_0, v_0)\\
		&+ \frac{\partial f}{\partial y}\big(\varphi(u_0,v_0)\big) \frac{\partial y}{\partial u}(u_0,v_0)
	\end{align*}
	\begin{align*}
		\forall (u_0, v_0) \in V, \frac{\partial (f \circ \varphi)}{\partial v}(u_0, v_0)
		&= \frac{\partial f}{\partial x}\big(\varphi(u_0, v_0)\big) \times \frac{\partial x}{\partial v}(u_0, v_0)\\
		&+ \frac{\partial f}{\partial y}\big(\varphi(u_0,v_0)\big) \frac{\partial y}{\partial v}(u_0,v_0)
	\end{align*}
\end{prop}

\begin{exm}
	[changement de coordonnées polaires]
	On pose \begin{align*}
		\varphi: \R^+_* \times ]0,2\pi[ &\longrightarrow \R^2\setminus \left( R^+_* \times \{0\} \right) \\
		(r, \theta) &\longmapsto (r \cos \theta, r \sin\theta),
	\end{align*}
	\begin{align*}
		f: \R^2\setminus \left( R^+_* \times \{0\} \right) &\longrightarrow \R \\
		(x,y) &\longmapsto f(x,y),
	\end{align*}
	\begin{align*}
		g: \overbrace{\R^+_* \times ]0, 2\pi[}^{=V} &\longrightarrow \R \\
		(r, \theta) &\longmapsto f(r\cos\theta, r\sin\theta).
	\end{align*}

	\begin{align*}
		\forall (r_0,\theta_0) \in V,&\\[5mm]
		\frac{\partial g}{\partial r}(r_0, \theta_0) &= \frac{\partial f}{\partial x}(r_0\cos\theta_0, r_0\sin\theta_0)\cos\theta_0\\
		&+ \frac{\partial f}{\partial y}(r_0 \cos\theta_0, r_0\sin\theta_0)\sin\theta_0\\
		&= 2r_0\cos^2\theta_0 + 2r_0\sin^2(\theta_0) \\
		&= 2r_0 \\[5mm]
		\frac{\partial g}{\partial \theta}(r_0, \theta_0) &= \frac{\partial f}{\partial x}(r_0\cos\theta_0, r_0\sin\theta_0)r_0\sin\theta_0\\
		&+ \frac{\partial f}{\partial y}(r_0 \cos\theta_0, r_0\sin\theta_0)r_0\cos\theta_0\\
		&= -2{r_0}^2\cos(\theta_0)\sin(\theta_0) + 2{r_0}^2 \sin(\theta_0)\cos(\theta_0)\\
		&= 0 \\
	\end{align*}

	Donc, \[
		g(r, \theta) = r^2.
	\]
\end{exm}

\begin{exm}
	Résoudre \[
		\begin{cases}
			\frac{\partial f}{\partial x} = \frac{x}{x^2+y^2},\\
			\frac{\partial f}{\partial y} = \frac{y}{x^2+y^2}.\\
		\end{cases}
	\]

	On pose $g: (r, \theta) \mapsto f(r \cos\theta, r \sin\theta)$.

	\begin{align*}
		&\frac{\partial g}{\partial r} = \frac{1}{r}\cos^2\theta + \frac{1}{r}\sin^2\theta = \frac{1}{r},\\
		&\frac{\partial g}{\partial \theta} = -\cos(\theta) \sin(\theta) + \sin(\theta)\cos(\theta) = 0.
	\end{align*}

	Donc, \[
		\exists C \in \R, g: (r, \theta) \mapsto \ln r + C
	\] d'où,
	\begin{align*}
		\forall (x,y) \in \R^2 \setminus \{(0,0)\}, f(x,y) &= \ln\left(\sqrt{x^2 + y^2} \right)  + C\\
		&= \frac{1}{2}\ln(x^2 + y^2) + C. \\
	\end{align*}
\end{exm}

\begin{rmk}
	Soit $\mathcal{B} = (e_1, e_2)$ la base canonique de $\R^2$, $f: U \to \R$ de classe $\mathcal{C}^1$ avec $U$ un ouvert de $\R^2$.

	Soit $(x,y) \in U$.

	\begin{align*}
		\Mat_{\mathcal{B}}\big(\nabla f(x,y)\big) = \begin{pmatrix}
			\frac{\partial f}{\partial x}(x,y)\\[2mm]
			\frac{\partial f}{\partial y}(x,y)
		\end{pmatrix}
	\end{align*}

	Soit  \begin{align*}
		\varphi: V &\longrightarrow U \\
		(u,v) &\longmapsto \big(x(u,v), y(u,v)\big) 
	\end{align*} avec $x,y$ de classe $\mathcal{C}^1$. Soit $g = f \circ \varphi$.
	\begin{align*}
		\Mat_{\mathcal{B}}\big(\nabla g(u,v)\big)
		&= \begin{pmatrix}
			\frac{\partial g}{\partial u}(u,v) \\[2mm]
			\frac{\partial g}{\partial v}(u,v)
		\end{pmatrix} \\
		&= \begin{pmatrix}
			\frac{\partial x}{\partial u}(u,v) \frac{\partial f}{\partial x}(x,y)
			+ \frac{\partial y}{\partial u}(u,v)\frac{\partial f}{\partial y}(x,y)\\[3mm]
			\frac{\partial x}{\partial v}(u,v) \frac{\partial f}{\partial x}(x,y)
			+ \frac{\partial y}{\partial v}(u,v) \frac{\partial f}{\partial y}(x,y)
		\end{pmatrix}  \\
		&= \underbrace{\begin{pmatrix}
				\frac{\partial x}{\partial u}(u,v)& \frac{\partial y}{\partial u}(u,v)\\[3mm]
				\frac{\partial x}{\partial v}(u,v)& \frac{\partial y}{\partial v}(u,v)
		\end{pmatrix}}_{J(u,v)} \begin{pmatrix}
			\frac{\partial f}{\partial x}(x,y)\\[3mm]
			\frac{\partial f}{\partial y}(x,y)
		\end{pmatrix} \\
		&= J(u,v) \Mat_{\mathcal{B}}\big(\nabla f(x,y)\big) \\
	\end{align*}
	où $J(u,v) = 
	\begin{pNiceArray}{c:c}
		\Mat_{\mathcal{B}}\big(\nabla x(u,v)\big) & \Mat_{\mathcal{B}}\big(\nabla y(u,v)\big)
	\end{pNiceArray}$.

	On dit que $J(u,v)$ est \underline{la jacobienne} de $\varphi$ en $(u,v)$.
	L'application linéaire canoniquement associée à $J(u,v)$ est la \underline{différentielle de $\varphi$} en $(u,v)$ noté $\mathrm{d}\varphi(u,v)$.

	On a $\mathrm{d}\varphi(u,v) \in \mathcal{L}(R^2)$ et $\Mat_{\mathcal{B}}\big(\mathrm{d}\varphi(u,v)\big) = J(u,v)$.

	Par exemple, la jacobienne du changement de coordonnées polaires est \[
		J = \begin{pmatrix}
			\frac{\partial x}{\partial r} & \frac{\partial y}{\partial r}\\[3mm]
			\frac{\partial x}{\partial \theta} & \frac{\partial y}{\partial \theta}
		\end{pmatrix}
		= \begin{pmatrix}
			\cos\theta&\sin\theta\\
			-r\sin\theta&r\cos\theta
		\end{pmatrix}.
	\]
	$\underbrace{\det(J)}_{\text{le jacobien}} = r\cos^2\theta + r\sin^2\theta = r$

	Dans une intégrale double, si $(x,y) = \varphi(u,v)$, alors $\mathrm{d}x\mathrm{d}y = \det(J)\mathrm{d}u\mathrm{d}v$.

	Ici, \[
		\mathrm{d}x\ \mathrm{d}y = r\ \mathrm{d}r\ \mathrm{d}\theta.
	\]
\end{rmk}

\begin{prv}
	On pose $(x_0, y_0) = \varphi(u_0, v_0)$. Pour tout $(h,k) \in \R^2$ tels que $(u_0 + h, v_0 + k) \in V$, en posant $g = f  \circ \varphi$.

	\begin{align*}
		g(u_0 + h, v_0 + h) &= f\big(x(u_0 + h, v_0 + k), y(u_0 + h, v_0 + k)\big) \\
		&= f\left(
			x(u_0,v_0) + h \frac{\partial x}{\partial u}(u_0,v_0) + k \frac{\partial x}{\partial v}(u_0, v_0) + \po\big(\|(h,k)\|\big), \right.\\
		&\phantom{ = f\bigg(\bigg.}\left. y(u_0, v_0) + h \frac{\partial y}{\partial u}(u_0, v_0) + k \frac{\partial y}{\partial v}(u_0, v_0) + \po\big(\|(h,k)\|\big)
		\right)  \\
		&= f(x_0,y_0) \\
		&~+ \left( h \frac{\partial x}{\partial u}(u_0,v_0) + k \frac{\partial x}{\partial v}(u_0, v_0) + \po(\|(h,k)\|) \right) \frac{\partial f}{\partial x}(x_0,y_0)\\
		&~+ \left( h \frac{\partial y}{\partial u}(u_0, v_0) + k\frac{\partial y}{\partial v}(u_0, v_0) + \po(\|(h,k)\|) \right) \frac{\partial f}{\partial y}(x_0, y_0)\\
		&~+ \po(\|(h,k)\|)\\
		&= f(x_0, y_0) \\
		&~+ h \left( \frac{\partial x}{\partial u}(u_0, v_0) \frac{\partial f}{\partial x}(x_0, y_0) + \frac{\partial y}{\partial u}(u_0, v_0) \frac{\partial f}{\partial y}(x_0, y_0) \right)  \\
		&~+ k\left( \frac{\partial x}{\partial v}(u_0, v_0) \frac{\partial f}{\partial x}(x_0, y_0) + \frac{\partial y}{\partial v}(u_0, v_0) \frac{\partial f}{\partial y}(x_0, y_0) \right) 
		&~+ \po(\|(h,k)\|)\\
		&= g(u_0, v_0) + h \frac{\partial g}{\partial u}(u_0, v_0) + k \frac{\partial g}{\partial v}(u_0, v_0) + \po(\|(h,k)\|) \\
	\end{align*}

	Par identification,
	\[
		\frac{\partial g}{\partial u}(u_0, v_0) = \frac{\partial x}{\partial u}(u_0, v_0) \frac{\partial f}{\partial x}(x_0, y_0) + \frac{\partial y}{\partial u}(u_0, v_0) \frac{\partial f}{\partial y}(x_0,y_0)
	\] et \[
		\frac{\partial g}{\partial v}(u_0, v_0) = \frac{\partial x}{\partial v}(u_0,v_0) \frac{\partial f}{\partial x}(x_0, y_0) + \frac{\partial y}{\partial v}(u_0, v_0) \frac{\partial f}{\partial y}(x_0, y_0).
	\] 
\end{prv}

\begin{exm}
	[Régression linéaire]~\\
	\begin{figure}[H]
		\centering
		\begin{asy}
			import graph;
			axes(EndArrow);
			size(5cm);

			real f(real x) { return x + 0.5; }

			real k = 35 / (7 - 0.5);

			for(int i = 0; i < 35; ++i) {
				real mag = exp(sin(100 * pi/exp(1) * i)) * 0.8 + exp(cos(i*40)/3);
				real eps = mag * cos(10 * exp(1)/pi * i) / 3;
				dot((i/k,f(i/k) + eps));
			}

			draw(graph(f, -1, 7), orange);
		\end{asy}
	\end{figure}
	\[
		y = a x + b
	\] 
	On fixe $(a,b) \in \R^2$. \[
		\varepsilon(a,b) = \sum_{i=1}^n\big( y_i - (ax_i + b) \big)^2
	\] l'erreur totale.

	On veut minimiser $\varepsilon(a,b)$. On a 
	\[
		\forall (a,b) \in \R^2,
		\begin{cases}
			\frac{\partial \varepsilon}{\partial a}(a,b) = -2\sum_{i=1}^{n}(y_i - ax_i - b)x_i,\\
			\frac{\partial \varepsilon}{\partial b}(a,b) = -2\sum_{i=1}^{n}(y_i - ax_i - b).
		\end{cases}
	\]

	Donc,
	\begin{align*}
		(a,b) \text{ point critique de } \varepsilon \iff& \begin{cases}
			a \sum_{i=1}^n {x_i}^2 + b\sum_{i=1}^{n}x_i = \sum_{i=1}^{n} y_ix_i\\
			a\sum_{i=1}^{n}x_i + nb = \sum_{i=1}^ny_i
		\end{cases}\\
		\iff& \begin{cases}
			a \left( \frac{1}{n}\sum_{i=1}^n {x_i}^2 - \overline{x}^2\right) = \overline{y} - \overline{x} \overline{y}\\
			b = \frac{1}{n}\sum_{i=1}^ny_i - \frac{a}{n}\sum_{i=1}^nx_i = \frac{1}{n}\sum_{i=1}^n x_i y_i - \overline{x} \overline{y}
		\end{cases}\\
		&\text{ où } \overline{x} = \frac{1}{n} \sum_{i=1}^n x_i,~\overline{y} = \frac{1}{n}\sum_{i=1}^n y_i\\
		\iff& \begin{cases}
			a = \frac{\Cov(x,y)}{V(x)}\\
			b = \overline{y} - a\overline{x}
		\end{cases}
	\end{align*}

	Coefficient de corrélation: $\frac{\Cov(x,y)}{\sigma_x \sigma_y} \in [-1, 1]$
\end{exm}












		\part{Corps}

\begin{exm}[Problème]
	\begin{itemize}
		\item 
			avec $A = \Z / 9 \Z$, résoudre $\overline{x}^2 = \overline{0}$ \\
			\begin{center}
				\begin{tabular}{|c|c|c|c|c|c|c|c|c|c|c|}
					\hline
					$\overline{x}$&$\overline{0}$& $\overline{1}$ &$\overline{2}$&$\overline{3}$ &$\overline{4}$ &$\overline{5}$ &$\overline{6}$ &$\overline{7}$ &$\overline{8}$& $\overline{9}$ \\
					\hline
					$\overline{x}^2$&$\overline{0}$ &$\overline{1}$ &$\overline{4}$ &$\overline{0}$ &$\overline{7}$ &$7$ &$\overline{0}$ &$\overline{4}$ &$\overline{1}$&$\overline{0}$\\
					\hline
				\end{tabular}
			\end{center}
			On a trouvé 3 solutions: $\overline{0}$, $\overline{3}$, $\overline{6}$.
		\item $\Z / 8\Z$
			\begin{center}
				\begin{tabular}{|c|c|c|c|c|c|c|c|c|}
					\hline
					$\overline{x}$& $\overline{0}$& $\overline{1}$& $\overline{2}$& $\overline{3}$& $\overline{4}$& $\overline{5}$& $\overline{6}$& $\overline{7}$\\
					\hline
					$\overline{x^2}$& $\overline{0}$& $\overline{1}$& $\overline{4}$& $\overline{1}$& $\overline{0}$& $\overline{1}$& $\overline{4}$& $\overline{1}$\\
					\hline
				\end{tabular}
			\end{center}
			$\overline{x}^2=7$ a 4 solutions: $\overline{1}, \overline{7}, \overline{3},\text{ et } \overline{5}$
		\item $A = \mathbbm{H} = \{a + bi + cj + dk  \mid  (a,b,c,d) \in \R^4\}$ \\
			$i^2 = j^2 = k^2 = -1$ 
			\begin{align*}
				\begin{array}{c c c}
					ij = k & jk = i & ji = j\\
					ji = -k & kj = -i & ik = -j
				\end{array}
			\end{align*}
			Dans cet anneau, $-1$ a 6 racines!
	\end{itemize}
\end{exm}

\begin{defn}
	Soit $(\mathbbm{K}, +, \times)$ un ensemble muni de deux lois de composition internes. On dit que c'est un \underline{corps} si
	 \begin{enumerate}
		\item $(\mathbbm{K}, \times)$ est un groupe abélien
		\item $(\mathbbm{K}, \times)$ est un monoïde commutatif
		\item $\forall x \in \mathbbm{K}\setminus \{0_\mathbbm{K}\}, \exists y \in \mathbbm{K}, xy = 1_\mathbbm{K}$
		\item $0_\mathbbm{K} \neq  1_\mathbbm{K}$
	\end{enumerate}
	\index{corps}
\end{defn}

\begin{exm}
	\begin{itemize}
		\item $(\C, +, \times)$ est un corps
		\item $(\R, +, \times)$ est un corps
		\item $(\Q, +, \times)$ est un corps
		\item $(\Z, +, \times)$ n'est pas un corps
	\end{itemize}
\end{exm}

\begin{prop}
	$(\Z / n\Z, +, \times)$ est un corps si et seulement si $n$ est premier.
\end{prop}

\begin{prv}
	\[
		\left( \Z / n\Z \right)^\times = \left\{ \overline{k}  \mid k \wedge n = 1 \right\}
	\] 
\end{prv}


\begin{prop}
	Tout corps est un anneau intègre.
\end{prop}

\begin{prv}
	Soit $(\mathbbm{K}, +, \times)$ un corps. Soient $(a,b) \in \mathbbm{K}^2$ tel que $a \times b = 0_\mathbbm{K}$.\\
	On suppose $a \neq  0_\mathbbm{K}$. Alors, $a$ est inversible et donc \[
		b = a^{-1} \times a \times b = a^{-1} \times 0_\mathbbm{K} = 0_\mathbbm{K}
	\] 
\end{prv}

\begin{exm}
	Soit $(\mathbbm{K},+,\times)$ un corps.\\
	Résoudre \[
		\begin{cases}
			x^2 = 1_\mathbbm{K}\\
			x \in \mathbbm{K}
		\end{cases}
	\]

	\begin{align*}
		x^2 = 1_\mathbbm{K} &\iff x^2 - 1_\mathbbm{K} = 0_\mathbbm{K}\\
		&\iff (x - 1_\mathbbm{K})(x+1_\mathbbm{K}) = 0_\mathbbm{K}\\
		&\iff x - 1_\mathbbm{K} = 0_\mathbbm{K} \text{ ou } x + 1_\mathbbm{K} = 0_\mathbbm{K}\\
		&\iff x = 1_\mathbbm{K} \text{ ou } x = -1_\mathbbm{K}
	\end{align*}

	Il y a au plus 2 solutions.
\end{exm}

\begin{prop}
	Soit $(\mathbbm{K},+,\times )$ un corps et $P$ un polynôme à coefficients dans $\mathbbm{K}$ de degré $n$. Alors, l'équation $P(x) = 0_{\mathbbm{K}}$ a au plus $n$ solutions dans $\mathbbm{K}$ 
	\qed
\end{prop}

\begin{crlr}[(Théorème de Wilson)]
	voir exercice 16 du TD 12
\end{crlr}


\begin{defn}
	Soit $(\mathbbm{K}, +, \times)$ un corps et $L\subset \mathbbm{K}$.\\
	On dit que $L$ est un \underline{sous corps} de $\mathbbm{K}$ si
	\begin{enumerate}
		\item $L$ est un anneau de $(\mathbbm{K}, +, \times)$ non nul
		\item $\forall x \in L\setminus \{0_\mathbbm{K}\}, x^{-1} \in L$ 
	\end{enumerate}
	\vspace{2mm}
	en d'autres termes si
	\begin{enumerate}
		\item $\forall (x,y) \in L^2, x - y \in L$
		\item $\forall (x,y) \in L^2, x \times y^{-1} \in L$
	\end{enumerate}
	\vspace{5mm}
	On dit aussi que $\mathbbm{K}$ est une \underline{extension} de $L$.
	\index{sous corps}
	\index{extension}
\end{defn}

\begin{prop}
	Tout sous corps est un corps. \qed
\end{prop}

\begin{defn}
	Soient $(\mathbbm{K}_1,+,\times )$ et $(\mathbbm{K}_2,+, \times)$ deux corps et $f: \mathbbm{K}_1 \to \mathbbm{K}_2$.\\
	On dit que $f$ est un \underline{morphisme de corps} si $f$ est un morphisme d'anneaux.\\
	i.e. si
	\[
		\begin{cases}
			\forall (x,y) \in {\mathbbm{K}_1}^2,& f(x+y) = f(x) + f(y)\\
			\forall (x,y) \in {\mathbbm{K}_1}^2,& f(x \times y) = f(x) \times f(y)\\
		\end{cases}
	\] 
	\index{homomorphisme (de corps)}
	\index{morphisme (de corps)}
\end{defn}

\begin{prop}
	Tout morphisme de corps est injectif.
\end{prop}

\begin{prv}
	Soit $f: \mathbbm{K}_1 \to \mathbbm{K}_2$ un morphisme de corps.\\
	\begin{itemize}
		\item $\Ker(f)$ est un sous groupe de $(\mathbbm{K}_1, +)$ 
		\item Soit $x \in \Ker(f)$ et $y \in \mathbbm{K}_1$ \[
				f(x \times y) = f(x) \times f(y) = 0_{\mathbbm{K}_2} \times f(y) = 0_{\mathbbm{K}_2}
			\]
		\item Soit $x \in \Ker(f) \setminus \{0_{\mathbbm{K}_1}\}$.\\
			Alors, $x$ est inversible.\\
			\begin{align*}
				\begin{rcases*}
					x \in \Ker(f)\\
					x^{-1} \in \mathbbm{K}_1
				\end{rcases*}& \text{ donc } x \times x ^{-1} \in \Ker(f)\\
				&\text{ donc } 1_{\mathbbm{K}_1} \in \Ker(f)\\
				&\text{ donc } f(1_{\mathbbm{K}_1}) = 0_{\mathbbm{K}_2}
			\end{align*}
			Or, $f(1_{\mathbbm{K}_1}) = 1_{\mathbbm{K}_2} \neq 0_{\mathbbm{K}_2}$
	\end{itemize}
	Donc, $\Ker(f) = \{0_{\mathbbm{K}_1}\}$ donc $f$ est injective.
\end{prv}

\begin{exm}
	$\begin{array}{cc}
		\C &\longrightarrow \C\\
		z &\longmapsto \overline{z}\\
	\end{array}$ est un morphisme de corps
\end{exm}



		\part{Opérations sur les séries}

\begin{prop}
	L'ensemble $E = \{u \in \C^\N  \mid \Sigma u_n \text{ converge}\}$ est un sous-espace vectoriel de $\C^\N$ et \begin{align*}
		S: E &\longrightarrow \C \\
		u &\longmapsto \sum_{n=0}^{+\infty} u_n
	\end{align*} est une forme linéaire.
	\qed
\end{prop}

\begin{rmk}
	La somme d'une série convergente et d'une série divergente diverge.
	Le produit d'une série divergente par un scalaire non nul diverge.
\end{rmk}

		\part{Comparaison de suites}

\begin{defn}
	Soient $u$ et $v$ deux suites réelles. On dit que $u$ est \underline{dominée} par  $v$ si \[
	\exists M\in \R, \exists N\in \N,\forall n\ge N,\left| u_n \right| \le M \left| v_n \right| 
	\] Dans ce cas, on note $u = O(v)$ ou $u_n = O(v_n)$ et on dit que "$u$ est un grand o de $v$"
\end{defn}

\begin{exm}
	En informatique, on dit qu'un alogirithme a une \underline{complexité linéaire} si son temps d'éxécution est un $O(n)$ 
	Par exemple, on calcule $a^n$ 

	\begin{itemize}
		\item Approche naïve
			\begin{algorithm}
				\begin{algorithmic}[1]
					\State $p \gets 1$
					\For{$i \in \left\llbracket 0,n-1 \right\rrbracket$}
						\State $p \gets p \times a$
					\EndFor
					\State \Return p
				\end{algorithmic}
			\end{algorithm}
			Complexité linéaire $O(n)$
		\item Exponentiation rapide\\
			On écrit $n$ en binaire: \begin{align*}
				n &= \overline{a_k a_{k-1}\ldots a_0}^{(2)}\\
					&= \sum_{i=0}^{k} a_i 2^i
			\end{align*} avec $(a_i) \in \left\{ 0,1 \right\} ^{k+1}$
			\begin{align*}
				a^n &= a^{\sum_{i=0}^{k} a_i 2^i} \\
				&= \prod_{i=0}^{k} a^{a_i 2^i}  \\
			\end{align*}
			
			\begin{algorithm}
				\begin{algorithmic}
					[1]

					\State $s \gets 0$
					\State $p \gets a$
					\For{ $i \in \left\llbracket 0, \log_2(n) \right\rrbracket$}
						\State $p \gets p \times p$
						\If{$a[i] = 1$}
							\State $s \gets s + p$
						\EndIf
					\EndFor
					\State \Return s
				\end{algorithmic}
			\end{algorithm}
			Compléxité logarithmique $O(\log_2(n))$
	\end{itemize}
\end{exm}


\begin{prop}
	$O$ est une relation réfléctive et transitive.
\end{prop}

\begin{prv}
	\begin{itemize}
		\item Soit $u$ une suite. On pose $M = 1$ et \[
			\forall n \in \N, \left| u_n \right| \le M \left| u_n \right|
			\] Donc $u = O(u)$.
		\item Soient $u, v, w$ trois suites telles que  \[
		\begin{cases}
			u = O(v)\\
			v = O(w)
		\end{cases}
		\] Soient $M_1,M_2 \in \R$ et $N_1,N_2\in \N$ tels que \[
		\begin{cases}
			\forall n \ge  N_1, \left| u_n \right| \le M_1 \left| v_n \right| \\
			\forall n \ge  N_2, \left| v_n \right| \le M_2 \left| w_n \right| \\
		\end{cases}
		\] 

		Nécéssairement, $M_1\ge 0$ et $M_2\ge 0$.\\
		Soit $N = \max(N_1,N_2)$. \[
		\forall n \ge  N, \left| u_n \right| \le M_1 \left| v_n \right| \le  M_1M_2 \left| w_n \right| 
		\] Donc $u = O(w)$
	\end{itemize}
\end{prv}

\begin{defn}
	Soient $u$ et $v$ deux suites. On dit que $u$ est \underline{négligeable} devant $v$ si \[
	\forall \varepsilon>0, \exists N\in \N, \forall n\ge N, \left| u_n \right| \le \varepsilon \left| v_n \right| 
	\] Dans ce cas, on note $u = o(v)$ ou $u_n = o(v_n)$ ou on le lit "$u$ est un petit o de $v$"
\end{defn}

\begin{prop}
	$o$ est une relation transitive, non-réfléctive
\end{prop}

\begin{prv}
	\begin{itemize}
		\item Soient $u$, $v$ et $w$ trois suites telles que \[
			\begin{cases}
				u = o(v)\\
				v = o(w)
			\end{cases}
			\] Soit $\varepsilon>0$. Soit $N_1\in \N$ tel que \[
			\forall n \ge N_1, \left| u_n \right| \le \sqrt{\varepsilon}  \left| v_n \right| 
			\] Soit $N_2\in \N$ tel que \[
			\forall n \ge N_2, \left| v_n \right| \le \sqrt{\varepsilon}  \left| w_n \right| 
			\] On pose $N = \max(N_1,N_2)$, alors \[
			\forall n \ge N, \left| u_n \right| \le \sqrt{\varepsilon}  \left| v_n \right| \le \underbrace{\sqrt{\varepsilon} \times \sqrt{\varepsilon}} _\varepsilon \left| w_n \right| 
			\] donc $u = o(w)$
		\item Soit $u$ une suite tel qu'il existe $N \in \N$ tel que \[
		\forall n \ge N, u_n > 0
		\] On suppose que $u = o(u)$, alors \[
		\forall \varepsilon>0,\exists N \in \N, \forall n \ge N, \left| u_n \right| \le \varepsilon \left| u_n \right| 
		\] On pose $\varepsilon = \frac{1}{2}$ alors \[
		\exists N \in \N, \forall n \ge N, \left| u_n \right| \le \frac{1}{2} \left| u_n \right| 
		\] une contradiction
	\end{itemize}
\end{prv}

\begin{prop}
	Soient $u$ et $v$ deux suites.
	\begin{itemize}
		\item $o(u) + o(u) = o(u)$
		\item $v \times o(u) = o(uv)$
		\item $o(u) \times o(v) = o(uv)$
		\item $o(o(u)) = o(u)$
	\end{itemize}
	\qed
\end{prop}

\begin{defn}
	Soient $u$ et $v$ deux suites. On dit que $u$ et $v$ sont \underline{équivalentes} si \[
	u = v + o(v)
	\] i.e. \[
	\forall \varepsilon >0, \exists N \in \N, \forall n \ge N, \left| u_n-v_n \right| \le \varepsilon\left| v_n \right| 
	\] Dans ce cas, on le note $u \sim v$
\end{defn}

\begin{prop}
	$\sim$ est une relation d'équivalence \qed
\end{prop}

\begin{prop}
	Soient $(u,v) \in \R^\N$. On suppose que $v$ ne s'annule pas à partir d'un certain rang
	\begin{enumerate}
		\item $u = o(v) \iff \left( \frac{u_n}{v_n} \right)$ bornée
		\item $u = o(v) \iff \frac{u_n}{v_n} \tendsto{n \to  +\infty} 0$
		\item $u \sim v \iff \frac{u_n}{v_n} \tendsto{n \to  +\infty} 1$
	\end{enumerate}
	\qed
\end{prop}

\begin{prop}
	[Suites de références]
	\begin{enumerate}
		\item $\ln^\alpha(n) = o(n^\beta)$ avec $(\alpha,\beta) \in \left( \R^+_* \right) ^2$ 
		\item $n^\beta = o(a^n)$ avec $\beta > 0$ et $a > 1$ 
		\item $a^n = o(n!)$ avec $a >1$ 
		\item $n! = o(n^n)$
	\end{enumerate}
\end{prop}


\begin{lem}
	[Exercice 10 du TD]
	Soit $u \in \left(\R^+_*\right)^\N$\\
	Si $\frac{u_{n+1}}{u_n} \tendsto{n \to +\infty} \ell < 1$ avec $\ell\in \R$,\\ alors $u_n \tendsto{n \to +\infty} 0$
\end{lem}

\begin{prv} [de la proposition]
	\begin{enumerate}
		\item par croissance comparée
		\item On pose $\forall n \in \N^*, u_n = \frac{n^\beta}{a^n}$. 
			\begin{align*}
				\forall  n \in \N^*, \frac{u_{n+1}}{u_n} &= \left( \frac{n+1}{n} \right) ^\beta \times \frac{1}{a} \\
				&= \frac{1}{a}\left( 1+\frac{1}{n} \right) ^\beta \\
				&\tendsto{n \to +\infty} \frac{1}{a} < 1
			\end{align*}
			Donc, $u_n \tendsto{n \to  +\infty} 0$
		\item On pose $\forall n \in \N, u_n = \frac{a^n}{n!}$ \[
			\forall n \in \N, \frac{u_{n+1}}{u_n} = \frac{a}{n+1} \tendsto{n \to +\infty} 0 < 1
			\] donc $u_n \tendsto{n \to +\infty} 0$
		\item On pose $\forall  n\in \N^*, u_n = \frac{n!}{n^n}$.
			\begin{align*}
				\forall n \in \N^*, \frac{u_{n+1}}{u_n}
				&= (n+1) {\frac{n^n}{(n+1)^{n+1}}} \\
				&= \left( \frac{n}{n+1} \right) ^n \\
				&= e^{n \ln\left( \frac{n}{n+1} \right) } \\
				&= e^{n \ln\left( 1+\frac{1}{n+1} \right)} \\
				&= e^{n(-\frac{1}{n} + o(\frac{1}{n})} \\
				&= e^{-1 + o(1)} \\
				&\tendsto{n \to  +\infty} e^{-1}<1
			\end{align*}
			donc $u_n \tendsto{n\to +\infty} 0$
	\end{enumerate}
\end{prv}

		\part{Matrices par blocs}

\begin{exm}
	Soit $p$ un projecteur de $E$ : \[
		E = \Ker p \oplus \mathrm{Im}\ p
	\] Soit $\mathcal{B} = (e_1, \ldots, e_k, e_{k+1}, \ldots, e_n)$ une base de $E$ avec $\begin{cases}
		\mathrm{Im}(p) = \Vect(e_1, \ldots, e_k)\\
		\Ker(p) = \Vect(e_{k+1}, \ldots, e_n)\\
	\end{cases}$

	Alors, 
	\begin{align*}
		\Mat_\mathcal{B}(p) =
		\left(\begin{NiceArray}{c c c | c c c}
				1&&&0&\Cdots&0\\
				 &\Ddots&&\Vdots&&\Vdots\\
				&&1&0&\Cdots&0\\\hline
				0&\Cdots&0&0&\Cdots&0\\
				\Vdots&&\Vdots&\Vdots&&\Vdots\\
				0&\Cdots&0&0&\Cdots&0\\
		\end{NiceArray}\right)
		= \left( \begin{array}{c|c}
				I_k & 0\\ \hline
				0&0
		\end{array}\right) \\
	\end{align*}

	De même, si $\s$ est une symétrie de $E$, \[
		E = \Ker(\s - \id_E) \oplus \Ker(\s + \id_E)
	.\] Soit $\mathcal{C} = (e_1', \ldots, e_\ell', e_{\ell+1}', \ldots, e'_n)$ avec $\begin{cases}
		\Vect(e'_1, \ldots, e'_\ell) = \Ker(\s - \id_E),\\
		\Vect(e'_{\ell+1}, \ldots, e'_n) = \Ker(\s + \id_E).\\
	\end{cases}$

	Alors
	\[
		\Mat_\mathcal{C}(\s) = \left(\begin{array}{c|c}
				I_\ell &0\\ \hline
				0&-I_{n-\ell}
		\end{array}\right) 
	\]
\end{exm}

\begin{prop}
	Soient $F$ et $G$ supplémentaires dans $E$ : \[
		E = F \oplus G.
	\] Soit $f \in \mathcal{L}(F)$ et $g \in \mathcal{L}(G)$. Alors \[
	\exists !h \in \mathcal{L}(E) h_{|F} = f,\ h_{|G} = g \et h = f \circ p + g \circ q
	\] où $\begin{cases}
		p \text{ est la projection sur $F$ parallèlement à $G$}\\
		q \text{ est la projection sur $G$ parallèlement à $F$}\\
	\end{cases}$.

	On a aussi $q = \id_E - p$.
\end{prop}

\begin{prv}
	\begin{itemize}
		\item[\sc \underline{Analyse}] Soit $h \in \mathcal{L}(E)$ tel que $\begin{cases}
				h_{|F}=f\\
				h_{|G}=g
			\end{cases}$.

			Soit $x \in E$. Alors \[
				x = \underbrace{p(x)}_{\in F} + \underbrace{q(x)}_{\in G}
			\]

			Donc,
			\begin{align*}
				h(x) &= h\big(p(x)\big) + h\big(q(x)\big)\\
				&= f\big(p(x)\big) + g\big(q(x)\big) \\
				&= (f \circ p + g \circ q)(x) \\
			\end{align*}
			Si $h$ existe, alors \[
				h = f \circ p + g \circ q
			\]
		\item[\underline{\sc Synthèse}] On pose $h = f \circ p + g  \circ q$.

			$p$, $q$, $f$ et $g$ sont linéaires donc $h$ aussi.

			Soit $x \in E$.
			\begin{align*}
				h(x) &= f\big(p(x)\big) + g\big(q(x)\big) \\
				&= f(x) + g(0_E) \\
				&= f(x) \\
			\end{align*}
			donc $h_{|F} = f$ et de même $h_{|G}=g$.
	\end{itemize}
\end{prv}

\begin{prop}
	On reprend les notations et hypothèses précédentes. Soit $(e_1, \ldots, e_p)$ une base de $F$, et $(f_1, \ldots, f_q)$ une base de $G$. Alors, $\mathcal{B} = (e_1, \ldots, e_p, f_1, \ldots, f_q)$ est une base de $E$ et \[
		\Mat_\mathcal{B}(h) = \left(
		\begin{array}{c|c}
			A&0\\ \hline
			0&B
		\end{array}\right)
	\] où $\begin{cases}
		A = \Mat_{(e_1, \ldots e_p)}(f)\\
		B = \Mat_{(f_1, \ldots, f_q)}(g)
	\end{cases}$
	\qed
\end{prop}

\begin{prop}
	Soient $(A,A') \in \mathcal{M}_n(\mathbbm{K})^2$ et $(B,B') \in \mathcal{M}_p(\mathbbm{K})^2$.
	\begin{enumerate}
		\item \[
				\left(\begin{array}{c|c}
					A&0\\ \hline
					0&B
				\end{array}\right)
				\left(\begin{array}{c|c}
					A'&0\\ \hline
					0&B'
				\end{array}\right) = 
				\left(\begin{array}{c|c}
					AA'&0\\ \hline
					0&BB'
				\end{array}\right)
			\]
		\item \[
				\left(\begin{array}{c|c}
					A&0\\ \hline
					0&B
				\end{array}\right) \in \mathrm{GL}_{n+p}(\mathbbm{K})	 \iff \begin{cases}
					 A \in \mathrm{GL}_n(\mathbbm{K})\\
					 B \in \mathrm{GL}_p(\mathbbm{K})
				\end{cases}
			\] et dans ce cas, \[
				\left(\begin{array}{c|c}
					A&0\\ \hline
					0&B
				\end{array}\right)^{-1} =
				\left(\begin{array}{c|c}
					A^{-1}&0\\ \hline
					0&B^{-1}
				\end{array}\right)
			\]
		\item \[
				\tr \left(\begin{array}{c|c}
					A&0\\ \hline
					0&B
				\end{array}\right) = \tr A + \tr B
			\]
	\end{enumerate}
\end{prop}

\begin{prv}
	\begin{enumerate}
		\item Soit $\begin{cases}
				f \in \mathcal{L}(F) \text{ tel que } \Mat_\mathcal{B}(f) = A,
				f' \in \mathcal{L}(F) \text{ tel que } \Mat_\mathcal{B}(f') = A',
				g \in \mathcal{L}(G) \text{ tel que } \Mat_\mathcal{C}(g) = B,
				g' \in \mathcal{L}(G) \text{ tel que } \Mat_\mathcal{C}(g') = B'
			\end{cases}$ où $\begin{cases}
				F \oplus G = \mathbbm{K}^{n+p},\\
				\dim(F) = n, \dim(G) = p,\\
				\mathcal{B} \text{ base de } F,\\
				\mathcal{C} \text{ base de } G.\\
			\end{cases}$
			Soit $\begin{cases}
				h \in \mathcal{L}(\mathbbm{K}^{n+p}) \text{ tel que } \begin{cases}
					h_{|F} = f\\
					h_{|G} = g
				\end{cases}\\
				h' \in \mathcal{L}(\mathbbm{K}^{n+p}) \text{ tel que } \begin{cases}
					h'_{|F} = f'\\
					h'_{|G} = g'\\
				\end{cases}
			\end{cases}$
			Soit $\mathcal{D} = \mathcal{B} \cup \mathcal{C}$ une base de $\mathbbm{K}^{n+p}$.
			\begin{align*}
				\left(\begin{array}{c|c}
					A&0\\ \hline
					0&B
				\end{array}\right)
				\left(\begin{array}{c|c}
					A'&0\\ \hline
					0&B'
				\end{array}\right) &= \Mat_{\mathcal{D}}(h) \Mat_{\mathcal{D}}(h')\\
				&= \Mat_{\mathcal{D}}(h \circ h') \\
			\end{align*}
			Or, $(h \circ h')_{|F} = f \circ f'$ et $(h \circ h')_{|G} = g \circ g'$.

			Donc,
			\begin{align*}
				\Mat_\mathcal{D}(h \circ h') &=
					\left(\begin{array}{c|c}
						\Mat_\mathcal{B}(f \circ f')&0\\ \hline
						0&\Mat_\mathcal{C}(g \circ g')
					\end{array}\right)\\
				&=\left(\begin{array}{c|c}
					AA'&0\\ \hline
					0&BB'
				\end{array}\right).
			\end{align*}
	\end{enumerate}
\end{prv}

\begin{prop}
	Soient $A,A' \in \mathcal{M}_n(\mathbbm{K})$, $B,B' \in \mathcal{M}_{n,p}(\mathbbm{K})$, $C,C' \in \mathcal{M}_{p,n}(\mathbbm{K})$ et $D, D' \in \mathcal{M}_p(\mathbbm{K})$.

	\[
		\left(\begin{array}{c|c}
			A&B\\ \hline
			C&D
		\end{array}\right)
		\left(\begin{array}{c|c}
			A'&B'\\ \hline
			C'&D'
		\end{array}\right) = 
		\left(\begin{array}{c|c}
			AA' + BC'& AB' + BD'\\ \hline
			CA' + DC'&CB' + DD'
		\end{array}\right)
	\] Cette formule se généralise à un nombre quelconque de blocs : \[
		\left(\begin{array}{c|c|c|c}
				A_{11}&A_{12}&\cdots&A_{1,n}\\ \hline
				A_{21}&A_{22}&\cdots&A_{2,n}\\ \hline
				\vdots&\vdots&\ddots&\vdots\\ \hline
				A_{p,1}&A_{p,2}&\cdots&A_{p,n}
		\end{array}\right)
		\left(\begin{array}{c|c|c|c}
				A'_{11}&A'_{12}&\cdots&A'_{1,n}\\ \hline
				A'_{21}&A'_{22}&\cdots&A'_{2,n}\\ \hline
				\vdots&\vdots&\ddots&\vdots\\ \hline
				A'_{p,1}&A'_{p,2}&\cdots&A'_{p,n}
		\end{array}\right)
	\] Cette matrice se calcyle comme on s'y attend si les dimensions des blocs autorisent les produits.
\end{prop}

\begin{prop}
	Le rang d'une matrice $A$, c'est la taille de la plus grande matrice carrée inversible que l'on peut extraire de $A$.
	\qed
\end{prop}




		\part{Trigonométrie hyperbolique}

\begin{defn}
	Pour tout $x \in \R$, on pose \[
		\begin{cases}
			\ch x = \frac{e^x + e^{-x}}{2},\\
			\sh x = \frac{e^x - e^{-x}}{2},\\
			\th x = \frac{\sh x}{\ch x}.
		\end{cases}
	\]

	$\ch$ est appelé \underline{cosinus hyperbolique}, $\sh$ est appelé \underline{sinus hyperbolique} et $\th$ est appelé \underline{tangeante hyperbolique}.
	\index{cosinus hyperbolique}
	\index{sinus hyperbolique}
	\index{tangente hyperbolique}
\end{defn}

\begin{rmk}
	Ces formules rappèlent les formules d'Euler : pour tout $x \in \R$,
	\begin{align*}
		\cos x = \frac{e^{ix} + e^{-ix}}{2}\quad\longleftrightarrow\quad\ch x = \frac{e^x + e^{-x}}{2}\\
		\sin x = \frac{e^{ix} - e^{-ix}}{2i}\quad\longleftrightarrow\quad\sh x = \frac{e^x - e^{-x}}{2}\\
	\end{align*}
\end{rmk}

\begin{figure}[H]
	\centering
	\begin{asy}
		import graph;

		size(12cm);

		pair A = (-2, 0);
		pair B = (2, 0);

		real eps = 0.05;

		draw(shift(A) * ((0, -1.3) -- (0, 1.3)), Arrow(TeXHead));
		draw(shift(A) * ((-1.3, 0) -- (1.3, 0)), Arrow(TeXHead));

		draw(circle(A, 1), magenta);
		
		real theta = 38;
		pair M = dir(theta) + A;
		draw(A -- M, red);
		draw(arc(A, 0.35, 0, theta), red, Arrow(TeXHead));
		draw(M -- (A.x-eps, M.y), dashed);
		draw(M -- (M.x, A.y-eps), dashed);
		label("\small$\theta$", 0.5dir(theta/2) + A, red);
		label("\small$\cos\theta$", (M.x, A.y), align=S);
		label("\small$\sin\theta$", (A.x, M.y), align=1.2W);
		dot("\small$M$", M);

		label("\small$x^2 + y^2 = 1$", A + 1.5dir(45+180));

		draw(shift(B) * ((0, -1.3) -- (0, 1.3)), Arrow(TeXHead));
		draw(shift(B) * ((-1.3, 0) -- (1.3, 0)), Arrow(TeXHead));

		real ch(real x) { return (exp(x) + exp(-x)) / 2.; }
		real sh(real x) { return (exp(x) - exp(-x)) / 2.; }
		real nch(real x) { return -ch(x); }

		real k = 1.9; real r = 1.2;
		real t = 1.4;

		draw(shift(B) * scale(0.35) * graph(ch, sh, -k, k), magenta);
		draw(shift(B) * scale(0.35) * graph(nch, sh, -k, k), magenta);

		label("\small$x^2 - y^2 = 1$", B + 1.5dir(45+180) + (0, -0.2));

		M = B + 0.35(ch(t), sh(t));

		draw(M -- (B.x-eps, M.y), dashed);
		draw(M -- (M.x, B.y-eps), dashed);
		dot("\small$M$", M);
		label("\small$\ch x$", (M.x, B.y), align=S);
		label("\small$\sh x$", (B.x, M.y), align=1.2W);

		draw(shift(B) * ((-r, -r)--(r,r)), gray + dashed);
		draw(shift(B) * ((r, -r)--(-r,r)), gray + dashed);
	\end{asy}
\end{figure}


		\part{Applications}
\section{Formule de Stirling}

\begin{prop}
	On a :
	\[
		n! \simi_{n\to +\infty} \sqrt{2\pi n} \left( \frac{n}{e} \right)^n?.
	\]
\end{prop}

\begin{prv}
	\[
		\forall n \in \N^*, \ln(n!) = \sum_{k=1}^n \ln k.
	\]

	$x \mapsto \ln x$ est strictement croissante sur $[1, +\infty[$ donc \[
		\forall k \in \N^*, \forall x \in [k, k+1], \ln x \ge \ln k
	\] donc \[
		\forall k \in \N^*, \int_{k}^{k+1} \ln x~\mathrm{d}x \ge \int_{k}^{k+1} \ln k~\mathrm{d}x = \ln k
	\] et \[
		\forall k \ge 2, \forall x \in [k - 1, k], \ln x \le \ln k
	\] et docn \[
		\forall k \ge 2, \int_{k-1}^{k}  \ln x~\mathrm{d}x \le \int_{k-1}^{k} \ln k~\mathrm{d}x = \ln k
	\] Ainsi \[
		\forall n \ge 2, 
		\int_{1}^{n} \ln x~\mathrm{d}x \ge \sum_{k=2}^n \le \int_{2}^{n+1} \ln x~\mathrm{d}x
	\] Or
	\begin{align*}
		\forall n \ge 2, \int_{1}^{n} \ln x~\mathrm{d}x &= \left[ x \ln x \right]_0^n\\
		&= n \ln(n) - n + 1 \\
		&\simi_{n\to +\infty} n \ln n\\
		\int_{2}^{n+1} \ln x~\mathrm{d}x &= (n+1) \ln(n+1) - (n+1) - 2 \ln(2) + 2 \\
		&\simi_{n\to +\infty} (n+1) \ln(n+1)\\
		&\simi_{n\to +\infty}n \ln n
	\end{align*}
	car
	\begin{align*}
		\ln(n+1) &= \ln\left( n \left( 1+ \frac{1}{n} \right) \right) \\
		&= \ln n + \ln\left( 1+\frac{1}{n} \right) \\
		&= \ln n + \frac{1}{n} + \po\left( \frac{1}{n} \right) \\
		&\sim \ln n \\
	\end{align*}

	Donc \[
		\ln(n!)) \simi_{n\to +\infty} n \ln n
	\]
	Cependant, on a un problème : {\color{orange}
	\begin{align*}
		&\ln(n!) = n \ln n + \po(n \ln n)\\
		\text{donc } & n! = n^n \underbrace{e^{\po(n \ln n)}}_{?}
	\end{align*}}

	On pose \[
		\forall n \in \N^*, u_n = \ln(n!) - n\ln n
	\] $(u_n)$ a même nature que $\Sigma(u_{n+1} - u_n)$ et
	\begin{align*}
		\forall n \in \N^*,
		u_{n+1} - u_n &= \ln\left( \frac{(n+1)!}{n!} \right) - (n+1) \ln(n+1) + n \ln n \\
		&= n\big(\ln n - \ln(n+1)\big) \\
		&= n\ln\left( \frac{n}{n+1} \right) \\
		&= n \ln \left( 1 - \frac{1}{n+1} \right) \\
		&\sim -\frac{n}{n+1} \sim -1 < 0
	\end{align*}

	$\Sigma(-1)$ diverge donc $(u_n)$ diverge.

	{\color{red}
		\underline{Conjecture}
		\[
			u_n = \sum_{k=1}^{n-1}(u_{k+1} - u_k) \underbrace{\sim}_{\mathclap{\substack{~\\\downarrow\\\text{On n'a absolument pas le droit !}}}} \sum_{k=1}^{n-1} (-1) = -(n-1) \sim -n
		\]
	}

	On pose \[
		\forall n \in \N^*, v_n = u_n + n
	\] et donc 
	\begin{align*}
		\forall n \in \N^*, v_{n+1} - v_n &= n \ln\left( 1 - \frac{1}{n+1} \right) + 1 \\
		&= n\left( -\frac{1}{n+1} - \frac{1}{2(n+1)^2} + \po\left( \left( \frac{1}{n+1} \right)^2 \right) \right) + 1 \\
		&= n \left( -\frac{1}{n\left( 1+\frac{1}{n} \right)} - \frac{1}{2n^2\left( 1+\frac{1}{n^2} \right)} + \po\left( \frac{1}{n^2} \right) \right) + 1 \\
		&= -\left( \frac{1}{1+\frac{1}{n}} - \frac{1}{2n} \times \frac{1}{\left( 1+\frac{1}{n} \right)^2} + \po\left( \frac{1}{n} \right) \right) \\
		&= -\left( 1 - \frac{1}{n} + \frac{1}{2n} + \po\left( \frac{1}{n} \right) \right) + 1 \\
		&= \frac{1}{2n} + \po\left( \frac{1}{n} \right) \\
		&\sim \frac{1}{2n} > 0.
	\end{align*}

	{\color{red}
		\[
			v_n \sim \sum_{k=1}^{n-1}(v_{k-1} - v_k) \sim \sum_{k=1}^{n-1} \frac{1}{2k} \sim \frac{1}{2} \ln(n)
		\]
	}

	On pose \[
		\forall n \in \N^*, w_n = v_n - \frac{1}{2} \ln n
	\] et donc
	\begin{align*}
		\forall n \in \N^*,
		w_{n+1}- w_n &= n\ln\left( 1+\frac{1}{n+1} \right) - \frac{1}{2}\ln(n+1) + \frac{1}{2} \ln(n) + 1 \\
		&= n\left( -\frac{1}{n+1} - \frac{1}{2(n+1)^2} - \frac{1}{3(n+1)^3} + \po\left( \frac{1}{(n+1)^3} \right) \right)\\
		&\phantom{=}\,+ 1 + \frac{1}{2} \ln\left( 1 - \frac{1}{n+1} \right) \\
		&= -1 - \frac{1}{2(n+1)} - \frac{1}{3(n+1)^2} + \po\left( \frac{1}{(n+1)^2} \right) \\
		&\phantom{=}\,+ \frac{1}{n+1} + \frac{1}{2(n+1)^2} + 1\\
		&\phantom{=}\,+ \frac{1}{2} \left( -\frac{1}{n+1} - \frac{1}{2(n+1)^2} + \po\left( \frac{1}{(n+1)^2} \right) \right)
		&\sim -\frac{1}{12(n+1)^2}\\
		&\sim -\frac{1}{12n^2} < 0
	\end{align*}
	donc $\Sigma(w_{n+1} - w_n)$ converge et donc $(w_n)$ converge.

	On pose $\ell = \lim_{n\to +\infty} w_n$. Ainsi, \[
		\forall n \in \N^*, w_n = \ell + \po(1)
	\] et donc \[
		\forall n \in \N^*, \ln(n!) = n \ln n - n + \frac{1}{2} \ln(n) + \ell + \po(1)
	\] et alors
	\begin{align*}
		\forall n \in \N^*, n! &= n^n e^{-n} \sqrt{n} e^{\ell} \underbrace{e^{\po(1)}}_{\mathclap{\tendsto{n\to +\infty} 1}} \\
		&\sim \left( \frac{n}{e} \right)^n \sqrt{n} \times K
	\end{align*} avec $K = e^{\ell}$.

	On pose \[
		\forall n \in \N^*, I_n = \int_{0}^{\frac{\pi}{2}} \sin^n x~\mathrm{d}x \sim \sqrt{\frac{\pi}{2n}}
	\]et \hfill (c.f. TD5 / Exercice 8)\[
		I_{2n} = \frac{(2n)!}{\left( 2^n n! \right)^2} \times \frac{\pi}{2}.
	\]

	\begin{align*}
		I_{2n} &\sim \frac{\pi}{2} \cancel{\left( \frac{2n}{2e} \right)^{2n}} \sqrt{2n} K \cancel{\left( \frac{e}{n} \right)^{2n}} \frac{1}{n} \times \frac{1}{K^2}\\
		&\sim \frac{\pi}{K\sqrt{2n}}.
	\end{align*}
	Or \[
		I_{2n} \sim \sqrt{\frac{\pi}{4n}}.
	\] Donc \[
		\frac{\sqrt{\frac{\pi}{4n}}}{\frac{\pi}{K\sqrt{2n}}} \tendsto{n\to +\infty} 1
	\] donc \[
		\frac{K}{\sqrt{2\pi}} \tendsto{n\to +\infty} 1
	\] et donc $K = \sqrt{2\pi}$.
\end{prv}

\section{Développement décimal}

\begin{exm}
	\begin{itemize}
		\item Avec $x = 0,54\mathunderline{54}\ldots$, que vaut $2x$ ?
		\item Avec $x = 0,333\mathunderline{3}\ldots$, que vaut $3x$ ?
			\begin{itemize}
				\item $0.999\mathunderline{9}\ldots$ ?
				\item $3 \times \frac{1}{3} = 1$ ?
			\end{itemize}
	\end{itemize}
\end{exm}

\begin{prop}
	Soit $(a_n)_{n \in \N}$ telle que \[
		\begin{cases}
			a_0 \in \Z,\\
			\forall n \ge 1, a_n \in \left\llbracket 0,9 \right\rrbracket
		\end{cases}
	\]

	La série $\sum \frac{a_n}{10^n}$ converge.
\end{prop}

\begin{prv}
	\[
		\forall n \ge 1, 0 \le \frac{a_n}{10^n} \le \frac{9}{10^n}
	\] $\sum \frac{1}{10^n}$ converge car $\frac{1}{10} \in [0, 1[$.
	Donc $\sum_{n\ge 1} \frac{a_n}{10^n}$ converge donc $\sum_{n\ge 1} \frac{a_n}{10^n}$ converge.
\end{prv}

\begin{defn}
	Soit $x \in \R$. On dit que $x$ admet un \underline{développement décimal} si \[
		\exists a_0 \in \Z, (a_n)_{n\ge 1} \in \left\llbracket 0,9 \right\rrbracket^N,
		x = \sum_{n=0}^{+\infty} \frac{a_n}{10^n}.
	\]
	\index{développement décimal}
\end{defn}

\begin{thm}
	Tou réel $x \in [0, 1[$ admet un développement décimal : \[
		x = \sum_{n=1}^{+\infty} \frac{\left\lfloor 10^n x \right\rfloor - 10 \left\lfloor 10^{n-1} x \right\rfloor}{10^n}
	\]
\end{thm}

\begin{prv}
	\begin{align*}
		\forall n \ge 1,\kern 5mm &\phantom{-}10^n x - 1 < \left\lfloor 10^n x \right\rfloor \le  10^n x\\
		&-10^n x + 10 > -10 \left\lfloor 10^{n-1} x \right\rfloor \ge -10^n x
	\end{align*}
	donc \[
		-1 < \left\lfloor 10^n x \right\rfloor - 10 \left\lfloor 10^{n-1} x \right\rfloor < 10
	\] et donc \[
		\left\lfloor 10^n x \right\rfloor - 10 \left\lfloor 10^{n-1} x \right\rfloor \in \left\llbracket 0,9 \right\rrbracket.
	\]

	De plus,
	\begin{align*}
		\sum_{k=1}^n \frac{\left\lfloor 10^k x \right\rfloor - 10 \left\lfloor 10^{k-1}x \right\rfloor }{10^k} &= \sum_{k=1}^n \left( \frac{\left\lfloor 10^k x \right\rfloor}{10^k} - \frac{\left\lfloor 10^{k-1}x \right\rfloor}{10^{k-1}} \right) \\
		&= \frac{\left\lfloor 10^n x \right\rfloor}{10^n} - \underbrace{\left\lfloor x \right\rfloor}_{=0}\\
		&\tendsto{n\to +\infty} x. \\
	\end{align*}
\end{prv}

\begin{thm}
	Soit $x \in ]0, 1[$.

	\begin{enumerate}
		\item Si $x$ n'est pas décimal (i.e. on ne peut pas l'écrire comme $\sfrac{p}{10^n}$ avec $p \in \Z$ et $n \in \N$), alors $x$ a un unique développement décimal.
		\item Si $x$ est décimal, alors $x$ a exactement 2 développements décimaux :
			\begin{itemize}
				\item il y en a un où, à partir d'un certain rang, tous les chiffres sont nuls,
				\item et un autre où tous les chiffres sont égaux à 9 à parir d'un certain rang.
			\end{itemize}
	\end{enumerate}
\end{thm}

\begin{prv}
	Soit $(a_n)_{n\ge 1} \in \left\llbracket 0,9 \right\rrbracket^{\N^*}$ et $(b_n)_{n\ge 1} \in \left\llbracket 0,9 \right\rrbracket^{\N^*}$ telles que \[
		x = \sum_{n=1}^{+\infty} \frac{a_n}{10^n} = \sum_{n=1}^{+\infty} \frac{b_n}{10^n}
	\] On pose $n_0 = \min \{n \in \N^*  \mid a_n \neq b_n\}$ : \[
		\begin{cases}
			\forall n < n_0, a_n = b_n,\\
			a_{n_0} \neq b_{n_0}.
		\end{cases}
	\] Sans perte de généralité, on suppose $a_{n_0} < b_{n_0}$. On a donc
	\begin{align*}
		0 < \frac{b_{n_0} - a_{n_0}}{10^{n_0}} &= \sum_{n = n_0 + 1}^{+\infty} \frac{a_n - b_n}{10^n} \\
	\end{align*}
	\[
		\forall n \ge n_0, \begin{cases}
			0 \le a_n \le 9\\
			0 \le b_n \le 9
		\end{cases}
	\] donc \[
		\forall n \ge n_0, -9 \le a_n - b_n \le 9
	\] donc \[
		-9 \sum_{n=n_0+1}^{+\infty} \frac{1}{10^n} \le \sum_{n=n_0 + 1}^{+\infty} \frac{a_n - b_n}{10^n} \le 9 \sum_{n=1}^{+\infty} \frac{1}{10^n}.
	\]
	Or,
	\begin{align*}
		\sum_{n=n_0 + 1}^{+\infty} \frac{1}{10^n} &= \frac{1}{10^{n_0+1}} \sum_{n=0}^{+\infty} \frac{1}{10^n} \\
		&= \frac{1}{10^{n_0+1}} \times \frac{1}{1-\frac{1}{10}} \\
		&= \frac{1}{9 \times 10^{n_0}} \\
	\end{align*}
	D'où, \[
		0 < \frac{b_{n_0} - a_{n_0}}{10^{n_0}} \le  \frac{1}{10^{n_0}}
	\] donc \[
		0 < \underbrace{b_{n_0} - a_{n_0}}_{\in \Z} \le 1
	\] donc $b_{n_0} - a_{n_0} = 1$ et donc \[
	\sum_{n = n_0 + 1}^{+\infty} \frac{a_n - b_n}{10^n} = \frac{1}{10^{n_0}}
	\] donc \[
		\forall n > n_0, a_n - b_n = 9
	\] et donc \[
		\forall n > n_0, \begin{cases}
			a_n = 9\\
			b_n = 0
		\end{cases}
	\] Comme \[
		\forall n > n_0, b_n = 0
	\] $x$ est décimal et les deux développements de $x$ sont alors
	\begin{align*}
		x &= 0,a_1\ldots a_{n_0-1}a_{n_0}\mathunderline{9}\ldots\\
		&= 0,a_1\ldots a_{n_0-1}(a_{n_0}+1)\mathunderline{0}\ldots \\
	\end{align*}
\end{prv}

\begin{rmk}
	Avec $x = 0,\!54\mathunderline{54}\ldots$, $100x = 54,\!54\mathunderline{54}\ldots = 54 + x$. On a donc $x = \frac{54}{99}$.

	Avec $x = 0,\!987\,123\,\mathunderline{123}\ldots$, on a
	\begin{align*}
		x &= \frac{987}{1000} + 0,\!000\,\mathunderline{123}\ldots\\
		&= \frac{987}{1000} + \frac{1}{10^3}\underbrace{(0,\!\mathunderline{123}\ldots)}_y \\
	\end{align*}
	On a $1000 y = 123 + y$ et donc $y = \frac{123}{999}$ et donc $x = \frac{987 + \frac{123}{999}}{1000}$.
\end{rmk}





		\ifsimple\else
	\pagebreak
	\begin{mdframed}
		La suite du cours provient d'Aubin. Je ne suis pas responsable pour les éventuelles bêtises qu'il a pu taper.
	\end{mdframed}
	\pagebreak


\let\cross\times
\let\gt\ge
\let\lt\le
\let\exist\exists



\part{Axiomatique de $\N$}


\begin{axm}[Axiomatique de Von Neumann]

		$(\N, \leq)$ est un ensemble totalement ordonné vérifiant\\
				Toute partie non vide de $\N$ a un plus petit élément\\
				Toute partie non vide majorée de $\N$ a un plus grand élément\\
				$\N$ n’est pas majoré\\

\end{axm}

\begin{defn}[$0$]

		$0 = \text{min}(\N)$\\

\end{defn}

\begin{defn}[$1$]

		$1 = \text{min}(\N \text{\\} \{0\})$\\

\end{defn}

\begin{defn}[$n+1$]

		Soit $n \in \N$\\
		On pose $n+1 = \text{min}(\{k \in \N|k>n\})$\\
		On dit que $n+1$ est le successeur de $n$\\

\end{defn}

\begin{prop}[$+1-1$]

		$\forall n \in \N, (n+1)-1 = n$\\
		$\forall n \in \N, (n-1)+1 = n$\\

\end{prop}

\begin{prv}

		Soient $n \in \N,\ p = n+1,\ q = p-1$ \\

		$n < p$ et $q < p$\\
		Donc $n \leq q$ car $q = \text{max}(\{k \in \N|k<p\})$\\
		Si $q > n$, alors $q \ge p$ car $p = \text{min}(\{k \in \N|k >n\})$\\
		Donc $q = n$\\

\end{prv}

\begin{prop}[Ensemble Ouvert Vide]

		$\forall n \in \N, ]n, n+1[ = \varnothing$\\

\end{prop}

\begin{prv}

		Soit $n \in \N$, on sait que $n+1>n$\\
		Soit $p>n$, on suppose $n < p < n+1$\\
		Comme $p >n, p \ge n+1$		Contradiction\\

\end{prv}

\begin{prop}[Théorème de Récurrence]

		Soit $P$ un prédicat sur $\N$ et $n \in \N$\\
		Si $\left\{\begin{array}{l c r}P(n_0) \text{ est vrai} \\ \forall n \ge n_0n P(n) \Longrightarrow P(n+1) \end{array}\right.$\\
		Alors $ \forall n \ge n_0, P(n)$ est vrai\\

\end{prop}

\begin{prv}

		Soit $A = \{n \in \N|n \ge n_0 \}$ et $P(n)$ faux\\
		Supposons $A \neq \varnothing$\\
		$A$ a donc un plus petit élement, soit $N = \text{min}(A)$\\

		Cas 1 : $N = 0$\\
				Alors, comme $N \in A$, on a $n_0 \leq 0$ et $P(0)$ est faux\\
				Alors $n_0 = 0$		Contradiction avec “$P(n)$ est vrai”\\

		Cas 2 : $N \neq 0$\\
				Alors $N-1 \in \N$\\
				$N-1 \notin A$ car $N-1 < N$\\
				Donc $N-1 < n_0$ ou $P(n)$ vrai\\

				Supposons $N-1 < n_0$\\
				$N \in A$ donc $N \ge n_0$\\
				$N-1 < n_0 < N$\\
				Donc $N = n_0$\\
				Or, $N \in A$ donc $P(n)$ est faux alors que $P(n)$ est vrai\\

				Supposons $\left\{\begin{array}{l c r}P(n-1) \text{ vrai} \\ N-1 \ge n_0\end{array}\right.$		Comme $N-1 \ge n_0, P(N-1) \Longrightarrow P(N)$\\

				Donc $P(N)$ est vrai\\
				Or, $N \in A$ donc $P(N)$ est faux\\

				Donc $A = \varnothing$\\



\end{prv}


\part{Récurrences}


\begin{prop}[Récurrence Double]

		Soient $P$ un prédicat sur $\N$ et $n_0 \in N$\\
		Si $\left\{\begin{array}{l c r} P(n_0)\text{ est vrai} \\ P(n_0 + 1) \text{ est vrai} \\ \forall n > n_0, P(n)\text{ et } P(n+1) \Longrightarrow P(n+2) \end{array}\right.$\\
		Alors $\forall n \ge n_0, P(n)$ est vrai\\

\end{prop}

\begin{prv}

		On pose $\forall n \in \N, Q(n) :$ “$P(n)\text{ et } P(n+1) \text{ vrais}$ ”\\
		$Q(n_0)$ est vrai\\

		Soit $n \ge n_0$, on suppose $Q(n)$ vrai\\
		On sait alors que $P(n+2)$ est vrai\\
		On sait par hypothèse de récurrence que $P(n+1)$ est vrai\\
		Donc $Q(n+1)$ est vrai\\

\end{prv}

\begin{prop}[Récurrence Multiple]

		Soient $P$ un prédicat sur $\N$ et $(p, n_0) \in \N^2$\\
		Si $\left\{\begin{array}{l c r} \forall k \in [\![0,p]\!], P(n_0-k)\text{ est vrai} \\ \forall n \ge n_0, (P(n) \text{ et ... } P(n+p-1)) \Longrightarrow P(n+p) \end{array}\right.$\\
		Alors $\forall n \ge n_0, P(n)$ est vrai\\

\end{prop}

\begin{prop}[Récurrence Forte]

		Soient $P$ un prédicat sur $\N$ et $n_0 \in \N$\\
		Si $\left\{\begin{array}{l c r} P(n_0)\text{ est vrai} \\ \forall n \ge n_0, (P(n_0) \text{ et ... } P(n-1)) \Longrightarrow P(n) \end{array}\right.$\\
		Alors $\forall n \ge n_0, P(n)$ est vrai\\

\end{prop}

\begin{prv}

		On pose $\forall n \in N, Q(n) :$ “$\forall k \in [\![n_0,n]\!], P(k) \text{ vrai}$”\\
		$Q(n_0)$ est vrai car $P(n_0)$ est vrai\\

		Soit $n \ge n_0$, on suppose $Q(n)$ vrai\\
		On sait que $\forall k \in [\![n_0,n]\!], P(k)$ est vrai\\
		Alors $P(n+1)$ est vrai\\
		Donc $\forall k \in [\![n_0,n+1]\!], P(k)$ est vrai\\
		Donc $Q(n+1)$ est vrai\\



\end{prv}


\part{Divisibilité}


\begin{defn}[Divisibilité]

		Soient $(a,b) \in \Z^2$\\
		On dit que $a$ divise $b$ si il existe $k \in \Z$ tel que $b = ka$\\
		On écrit $a|b$ et on dit que $\left\{\begin{array}{l c r}a \text{ est un diviseur de }b \\b \text{ est un multiple de }a\end{array}\right.$\\

\end{defn}

\begin{prop}[Caractéristiques de la Divisibilité]

		$|$ est une relation d’ordre sur $\Z$\\
		Ce n’est pas une relation totale\\

\end{prop}

\begin{prop}[Ordonnancement et Divisibilité]

		Soient $(a,b) \in \Z\cross \Z^*$\\
		Si $a|b, |a| \leq |b|  $\\

\end{prop}

\begin{prop}[Divisibilité et Combinaison Linéaire]

		Soient $(a,b,c) \in (\Z^*)^3$\\
		$\left\{\begin{array}{l c r}a|b\\a|c\end{array}\right. \Longrightarrow \forall (k,l) \in \Z^2, a|(bk+cl)$\\

\end{prop}

\begin{prv}

		$\left\{\begin{array}{l c r}b=au \text{ avec }u \in \Z \\ c = av \text{ avec } v \in \Z \end{array}\right.$\\
		Soient $(k,l) \in \Z^2$\\
		$bk + cl = auk+avl = a(uk+vl)$\\
		Donc $a|(bk+cl)$\\

\end{prv}

\begin{defn}[Nombres Associés]

		Soient $(a,b) \in \Z^2$\\

		$a$ et $b$ sont associés si $a=b$ ou $a=-b$\\

\end{defn}

\begin{prop}[Nombres Associés et Divisibilité]

		Soient $(a,b) \in \Z^2$\\
		$a|b \iff -a|b \iff a |-b \iff-a|-b$\\



\end{prop}


\part{Division Euclidienne}


\begin{prop}[Division Euclidienne dans $\N$]

		Soient $a \in \N$ et $b \in \N^*$\\
		$\exist!(q,r)\in \N^2, \left\{\begin{array}{l c r}a = bq+r \\r \in [0,b[\end{array}\right.$\\

\end{prop}

\begin{prv}

		Existence : On considère| $A = \{q \in \N|qb\leq a\}$ $A$ est non vide car $0 \in A$\\
				$A$ est majoré : $\forall q \in A, q \leq a$ car $a \ge qb \ge q$\\

				Soit $q = \text{max}(A)$, on pose $r = a-bq$\\
				Comme $a,b$ et $q \in \N, r \in \Z$\\
				$q \in A$ donc $qb \leq a$ donc $r \ge 0$\\
				$q+1 > \text{max}(A)$ donc $q+1 \notin A$ donc $(q+1)b > a$\\
				Donc $r < b$\\

		Unicité : Soit $(q',r') \in \N^2$ tel que $\left\{\begin{array}{l c r}a+bq'+r'\\0\leq r'<b\end{array}\right.$\\
				On sait aussi que $a = bq+r$\\
				Donc $0 = b(q'-q) + r'-r$\\
						$-r'+r=b(q'-q)$\\

				De plus, $\left\{\begin{array}{l c r}0\leq r < b \\ -b < -r \leq 0\end{array}\right.$\\
				Donc $-b < r'-r<b$\\

				Le seul multiple de $b$ dans $]\!]-b,b[\![$ est $0$\\
				Donc $r'-r=0$, donc $r=r'$\\
				et $b(q'-q)=0$\\
				Or, $b \neq 0$ donc $q'-q=0$ donc $q=q'$\\

\end{prv}

\begin{prop}[Division Euclidienne dans $\Z$]

		Soient $a \in \Z$ et $b \in \Z^*$\\
		$\exist! (q,r) \in \Z^2, \left\{\begin{array}{l c r}a=bq+r \\ 0 \leq r < |b| \end{array}\right.$\\

\end{prop}

\begin{prv}

		Existence :\\
				Cas 1 : $a \in \N, b \in \N^*$\\
						D’après ce qui précède, $\exist!(q,r) \in \N^2, \left\{\begin{array}{l c r}a=bq+r \\ 0 \leq r <b\end{array}\right.$\\
						Comme $b \in \N^*$, on a bien $0 \leq r < |b| $\\
						$q \in \N \subset \Z$\\

				Cas 2 : $a \in \Z, b \in \N^*$\\
						Comme $-a \in \N, \exist! (q',r') \in \N^2, \left\{\begin{array}{l c r}-a=bq'+r' \\ 0 \leq r' < b\end{array}\right.$\\

						Donc $a = b(-q') - r'$\\
								$=b(-q'-1) - r'+b$\\

						On pose $q = \left\{\begin{array}{l c r}-q'-1 \text{ si } r \neq 0 \\-q' \text{ si }r = 0 \end{array}\right.$	et	$r = \left\{\begin{array}{l c r}-r'+b \text{ si }r' \neq 0 \\ r' \text{ si } r'= 0\end{array}\right.$\\
						On a bien $\left\{\begin{array}{l c r}a = bq+r \\q \in \Z \\ 0 \leq r < |b| \end{array}\right.$\\

				Cas 3 : $a \in \N, b \in \Z^*_-$\\
						$\exist! (q',r') \in \N^2, \left\{\begin{array}{l c r}a=(-b)q'+r' \\0 \leq r' < -b\end{array}\right.$\\

						On pose $\left\{\begin{array}{l c r}q=-q'\\r=r'\end{array}\right.$\\
						Et on a bien $\left\{\begin{array}{l c r}a = bq+r \\0 \leq r < |b| \end{array}\right.$\\

				Cas 4 : $a \in \Z^-, b \in \Z^*_-$\\
						$\exist! (q',r') \in \N^2, \left\{\begin{array}{l c r}-a=-bq'+r' \\ 0 \leq r' < -b\end{array}\right.$\\

						Donc $a = bq' - r'$\\
								$=b(q'+1) - r'-b$\\

						On pose $q = \left\{\begin{array}{l c r}q'+1 \text{ si } r \neq 0 \\q' \text{ si }r = 0 \end{array}\right.$	et	$r = \left\{\begin{array}{l c r}-r-b' \text{ si }r' \neq 0 \\ r' \text{ si } r'= 0\end{array}\right.$\\
						On a bien $\left\{\begin{array}{l c r}a = bq+r \\q \in \Z \\ 0 \leq r < |b| \end{array}\right.$\\

		Unicité :\\
				Soit $(q',r') \in \Z^2$ tel que $\left\{\begin{array}{l c r}a=bq'+r' \\ 0 \leq r' < |b| \end{array}\right.$ et $\left\{\begin{array}{l c r}a = bq+r \\0 \leq r < |b|  \end{array}\right.$\\

				D’où $\left\{\begin{array}{l c r}b(q'-q) = r'-r \\ -|b| < r -r' < |b|   \end{array}\right.$\\
				Donc $r-r'=0$\\

				Donc $r'=r$ et $q'=q$\\

\end{prv}

\begin{defn}[Quotient et Reste]

		Soient $a \in \Z$ et $b \in \Z^*$\\
		D’après le théorème précédent, $\exist! (q,r) \in \Z\cross \N, \left\{\begin{array}{l c r}a=bq+r \\ 0 \leq r < |b| \end{array}\right.$\\
		On dit que $q$ est le quotient et $r$ le reste dans la division (euclidienne) de $a$ par $b$\\

\end{defn}

\begin{prop}[Reste et Divisibilité]

		Soient $a \in \Z, b \in \Z^*$\\
		On note $r$ le reste de la division de $a$ par $b$\\
		$r = 0 \iff b|a$\\

\end{prop}

\begin{prv}

		On pose $a = bq+r, q \in \Z$\\

		“$\Longrightarrow$” : Si $r = 0$, alors $\left\{\begin{array}{l c r}a = bq\\q \in \Z\end{array}\right.$	donc $b|a$\\

		“$\Longleftarrow$” : Si $b|a$, $\exist k \in \Z, a =bk$\\
				Donc $\left\{\begin{array}{l c r}a = bk+0\\0 \leq 0 < |b| \end{array}\right.$\\
				Par unicité du reste, $r=0$\\



\end{prv}


\part{Arithmétique Modulaire}


\begin{defn}[Congruences]

		Soient $(a,b) \in \Z^2, c \in \N^*$\\
		On dit que $a$ et $b$ sont congrus modulo $c$ si $a$ et $b$ ont le même reste dans ma division par $c$\\
		On note $a = b[c]$\\

\end{defn}

\begin{prop}[Congruence et Relation D’Equivalence]

		La relation de congruence modulo $c$ est une relation d’équivalence\\

\end{prop}

\begin{rmk}[Classes d’Equivalence Modulo $c$]

		On note $\Z/c \Z$ l’ensemble des classes d’équivalence modulo $c$\\
		$\Z/5 \Z = \{\bar0, \bar1, \bar2, \bar3, \bar4\}$\\

\end{rmk}

\begin{prop}[Modulo et Divisibilité]

		Soient $(a,b) \in \Z^2$ et $c \in \N^*$\\
		$a \equiv b [c] \iff c | b-a$\\

\end{prop}

\begin{prv}

		“$\Longrightarrow$” : On pose $\left\{\begin{array}{l c r}a = cq+r, q \in \Z, 0\ \leq r < c \\ b = cq' + r, q' \in \Z\end{array}\right.$\\
				Donc $b-a = c(q'-q)$\\
				Donc $c|b-a$\\

		“$\Longleftarrow$” : On pose $\left\{\begin{array}{l c r}a = cq+r, q \in \Z, 0\ \leq r < c \\ b = cq' + r', q' \in \Z, 0\ \leq r' < c\end{array}\right.$\\
				$b-a = c(q'-q) + r'-r$\\
				$-c \leq r'-r \leq c$\\

				Si $r'-r \ge 0, r'-r$ est le reste de la division de $a-b$ par $c$\\
				Donc $r'=r$ donc $a \equiv b [c]$\\

				Si $r'-r < 0, r-r'$ est le reste de la division de $a-b$ par $c$\\
				Donc $r'=r$ donc $a \equiv b[c]$\\

\end{prv}

\begin{prop}[Addition et Multiplication de Congruences]

		Soient $(a,b,x,y) \in \Z^4$ et $c \in \N^*$\\
		On suppose $\left\{\begin{array}{l c r}a \equiv b[c]\\x \equiv y [c]\end{array}\right.$\\

		Alors $\left\{\begin{array}{l c r}a+x \equiv b+ y [c] \\ ax \equiv by [c] \end{array}\right.$\\

\end{prop}

\begin{prv}

		$c|b-a$ et $c|y-x$\\
		Donc $c|(b-a+y-x)$\\
		Donc $c|(b+y-(a+x))$\\
		Donc $a+x \equiv b+y [c]$\\

		On pose $\left\{\begin{array}{l c r}a = ck+b, k \in \Z \\ x = cl+y, l \in \Z\end{array}\right.$\\
		$ax = (ck+b)(cl+y)$\\
				$=by+cky+clk+c^2kl$\\
				$=by+c(ky+bl+clk)$\\
		Donc $ax \equiv by [c]$\\

\end{prv}

\begin{prop}[Critères de Divisibilité en Base 10]

		Soit $N \in \N$, on notera ses chiffres $a_0...a_n$\\
		$N = \overset{n}{\underset{k=0}{\sum}}10^ka_k$\\

		Divisibilité par $2$ :\\
				$N$ pair $\iff N \equiv 0[2]$\\
						$\iff \overset{n}{\underset{k=0}{\sum}} 10^ka_k \equiv 0 [2]$\\
						$\iff a_0 \equiv 0 [2]$ car	$\forall k \ge 1, 10^k \equiv 0 [2] \\10^0 = 1 \equiv 1 [2]\quad\ $\\

		Divisibilité par $3$ :\\
				$\forall k \in \N, 10^k \equiv 1^k \equiv1 [3]$ car $10 \equiv 1 [3]$\\
				$3|N \iff N \equiv 0 [3]$\\
						$\iff \overset{n}{\underset{k=0}{\sum}}10^ka_k \equiv 0 [3]$\\

		Divisibilité par $9$ : \\
				$\forall k \in N, 10^k \equiv 1 [9]$\\
				$9|N \iff N \equiv 0 [9]$\\
						$\iff \overset{n}{\underset{k=0}{\sum}}10^ka_k \equiv 0 [9]$\\

		Divisibilité par $5$ :\\
				$\left\{\begin{array}{l c r}10^0 \equiv 1 [5] \\ \forall k \in \N^*, 10^k \equiv 0 [5] \end{array}\right.$\\
				$5|N \iff a_0 \equiv 0 [5]$\\
						$\iff a_0 \in \{0,5\}$\\

		Divisibilité par $11$ :\\
				$10^0 \equiv -1 [11]$\\
				Donc $\forall k \in \N, 10^k \equiv (-1)^k[11]$\\
				$N \equiv 0 [11] \iff \overset{n}{\underset{k=0}{\sum}}(-1)^ka_k \equiv 0 [11]$\\
						$\iff a_0-a_1+a_2...+(-1)^na_n \equiv 0 [11]$\\

\end{prop}

\begin{rmk}[Réécriture en Classes d’Equivalence]

		On peut réecrire le calcul précédent dans $\Z/11 \Z$ \\
		$\overline N = \overline{\overset{n}{\underset{k=0}{\sum}}10^ka_k} = \overset{n}{\underset{k=0}{\sum}}\overline{10^k}\ \overline{a_k} = \overset{n}{\underset{k=0}{\sum}}\overline{(-1)^k}\ \overline{a_k}$\\

\end{rmk}

\begin{rmk}[Opération dans $\Z / n \Z$]

		Dans $\Z / n \Z$, on dispose $\left\{\begin{array}{l c r}\text{d’une addition} : \quad\quad\ \ \overline{a} + \overline{ b} = \overline{a+b} \\ \text{d'une multiplication} : \overline{a} * \overline{b} = \overline{a*b}\end{array}\right.$\\

		L’addition est commutative, associative, n’élément neutre $\overline{0}$ et l’opposé de $\overline{a}$ est $\overline{-a}$\\
		La multiplication est commutative, associative, d’élément neutre $\overline{1}$ et distributive par rapport à $+$\\



\end{rmk}


\part{PCGD et PPCM}


\begin{defn}[PGCD]

		Soient $(a,b) \in \Z^2$\\
		Le PGCD de $a$ et $b$ est le plus grand diviseur commun à $a$ et $b$\\
		Il existe car $\mathcal{D} = \{d \in \N|d|a \text{ et } d|b\}$ est non vide car $a \in \mathcal{D}$\\
		$\mathcal{D}$ est majoré par $|a| $\\
		On le note $\text{PGCD}(a,b)$ ou $a\wedge b$\\

\end{defn}

\begin{prop}[Théorème d’Euclide]

		Soient $a \in \Z, b \in \N^*$\\
		Soit $r$ le reste de la division de $a$ par $b$\\
		$a \wedge b = b \wedge r$\\

\end{prop}

\begin{prv}

		On pose $\left\{\begin{array}{l c r}d = a \wedge b\\ \varsigma = b \wedge r \\ a = bq+r\end{array}\right.$\\

		$\left\{\begin{array}{l c r}d|a\\d|b\end{array}\right. \Longrightarrow \left\{\begin{array}{l c r}d|a-bq\\d|b\end{array}\right. \Longrightarrow \left\{\begin{array}{l c r}d|r\\d|b\end{array}\right. \Longrightarrow  d \leq \varsigma$\\

		$\left\{\begin{array}{l c r}\varsigma|b\\\varsigma|r\end{array}\right. \Longrightarrow \left\{\begin{array}{l c r}\varsigma|bq+r\\\varsigma|b\end{array}\right. \Longrightarrow \left\{\begin{array}{l c r}\varsigma|a\\\varsigma|b\end{array}\right. \Longrightarrow \varsigma \leq d$\\

		Donc $d = \varsigma$\\

\end{prv}

\begin{prop}[PGCD et Diviseurs]

		Soient $(a,b) \in \Z^2$ et $d = a \wedge b$\\
		$\mathcal{D} = \{k \in \Z|\ k|a, k|b \}$\\
		$\forall k \in \Z, k \in \mathcal{D} \iff k|d$\\

\end{prop}

\begin{prv}

		“$\Longleftarrow$” : Soit $k \in \Z$, on suppose $k|d$\\
				$d|a$ donc $k|a$\\
				$d|b$ donc $k|b$\\

				Donc $k \in \mathcal{D}$\\

		“$\Longrightarrow$” : Soit $k \in \mathcal{D}$\\
				On pose $r_0$ le reste de la division de $a$ par $b$,\\
				$r_1$ le reste de la division de $b$ par $r_0$\\
				et $\forall n \in \N, r_{n+1}$ le reste de la division de $r_{n-1}$ par $r_n$ si $r_n \neq 0$\\

				La suite $(r_n)$ est décroissante, minorée par $0$ et à valeurs entières\\
				Soit $N \in \N$ tel que $r_N = 0$			(si $N=0$, on pose $r_{-1} = b$)\\

				D’après la poposition précédente,\\
				$d =a \wedge b = b \wedge r_0 = r_0 \wedge r_1...r_{N-1} \wedge r_N$\\
						$= r_{N-1} \wedge 0 = r_N$\\

				On pose aussi $\forall n \in [\![1, N-1]\!], r_{n-1} = r_nq_n + r_{n+1}$\\
				On en déduit que $\exists(\alpha_n, \beta_n) \in \Z^2, r_{N-1} = a \alpha_N + b \beta_N$\\
				$\left\{\begin{array}{l c r}k|a\\k|b\end{array}\right. \Longrightarrow k|a \alpha_N + b \beta_N \Longrightarrow k|r_{N-1} \Longrightarrow k|d$\\

\end{prv}

\begin{defn}[PPCM]

		Soit $(a,b) \in \Z^2$, on pose $M = \{k \in \N|\ a|k, b|k\}$\\
		$M \neq \varnothing$ car $ab \in M$\\
		$M \neq \varnothing$ donc admet un plus petit élément noté $\text{PPCM}(a,b)$ ou $a \vee b$\\

\end{defn}

\begin{prop}[Produit PGCD PPCM]

		$\forall (a,b) \in \Z^2, (a \wedge b)(a \vee b) = ab$\\

\end{prop}

\begin{prv}

		Voir paragraphe Facteurs Premiers\\

\end{prv}

\begin{prop}[Propriétés de $\wedge$ et $\vee$]

		$\wedge$ est commutative, associative sur $\Z^*$\\
		$\vee$ est commutative, associative sur $\Z^*$\\

\end{prop}

\begin{prv}

		Soient $(a,b,c) \in (\Z^*)^3, d = (a \wedge b) \wedge c, \varsigma = a \wedge (b \wedge c)$ \\

		$\left\{\begin{array}{l c r}d|c\\d|a \wedge b\end{array}\right. \Longrightarrow \left\{\begin{array}{l c r}d|c\\d|a\\d|b\end{array}\right.$\\

		$\left\{\begin{array}{l c r}\varsigma|a\\\varsigma|b \wedge c\end{array}\right. \Longrightarrow \left\{\begin{array}{l c r}\varsigma|a\\\varsigma|b\\\varsigma|c\end{array}\right.$\\

		On pose $\varepsilon = \text{PGCD}(a,b,c)$\\
		On a $\left\{\begin{array}{l c r}d \leq \varepsilon \\ \varsigma \leq \varepsilon\end{array}\right.$\\

		$\left\{\begin{array}{l c r}\varepsilon|a\\\varepsilon|b\\\varepsilon|c\end{array}\right. \Longrightarrow \left\{\begin{array}{l c r}\varepsilon|a\\\\\varepsilon|b \wedge c\end{array}\right. \Longrightarrow \varepsilon|a \wedge (b \wedge c) \Longrightarrow \varepsilon \leq d$\\

		De même, on $\varepsilon\leq \varsigma$\\
		Donc $d = \varepsilon = \varsigma$\\

\end{prv}

\begin{prop}[Théorème de Bézout]

		Soient $(a,b) \in \Z \cross \Z^*, d = a \wedge b$\\
		$\exists(u,v) \in \Z^2, d = au+bv$\\

\end{prop}

\begin{prv}

		On pose $A = \{au+bv|(u,v) \in \Z^2\}$\\
		On veut montrer que $d \in A$\\
				$a = a*1+b*0$ donc $a \in A$\\
				$b = a*0+b*1$ donc $b \in A$\\
				$0 = a*0+b*0$ donc $0 \in A$\\

		Soit $(x,y) \in A^2$\\
				$x = au_1+bv_1, (u_1,v_1) \in \Z^2$\\
				$y = au_2 + bv_2, (u_2,v_2) \in \Z^2$\\
				$x+y = a(u_1+u_2) + b(v_1+v_2) \in A$\\

		Soit $x \in A, k \in \Z$\\
				$x = au+bv, (u,v) \in \Z^2$\\
				$kx = aku+bkv \in A$\\

		Soit $n = \text{min}(A \cap \N^*)$		$(|b| \in A \cap \N^*)$\\
		Soit $x \in A$\\

		Par division euclidienne de $x$ par $n$ :\\
				$\left\{\begin{array}{l c r}x = nq+r \\ q \in A, 0 \leq r<n \end{array}\right.$\\

		$\left\{\begin{array}{l c r}x \in A\\n \in A\end{array}\right. \Longrightarrow \left\{\begin{array}{l c r}x \in A\\-qn \in A\end{array}\right. \Longrightarrow x-qn \in A \Longrightarrow r \in A$\\

		$\left\{\begin{array}{l c r}r < n \\r \in A\end{array}\right. \Longrightarrow r \leq 0$\\

		Donc $r = 0$\\
		Donc $n|x$\\

		D’où $A = n \Z$\\
		Or, $\left\{\begin{array}{l c r}a \in A\\b \in A\end{array}\right. \Longrightarrow \left\{\begin{array}{l c r}n|a\\n|b\end{array}\right.$\\

		Cas particulier : $a \wedge b = d = 1$, alors $1$ est le seul diviseur positif de $a$ et $b$\\
		Donc $n=1$ donc $A = \Z$ donc $1 \in \Z$\\

		Cas général : On pose $a' = \frac{a}{d} \in \Z,\ b' = \frac{b}{d} \in \Z,\ a' \wedge b' = 1$\\
		D’après le cas particulier, $\exist (u,v) \in \Z^2, a'u+b'v = 1$\\

		D’où $au+bv=d$\\

\end{prv}

\begin{prop}[Réciproque du Théorème de Bézout]

		Soient $(a,b) \in \Z\cross \Z^*$\\
		On suppose qu’il existe $(u,v) \in \Z^2$ tel que $au+bv=1$\\
		Alors $a \wedge b = 1$\\

\end{prop}

\begin{prv}

		On pose $d = a \wedge b$\\
		$\left\{\begin{array}{l c r}d|a\\d|b\end{array}\right. \Longrightarrow d|au+bv \Longrightarrow d|1 \Longrightarrow d=1$\\

\end{prv}

\begin{prop}[Théorème de \Gauss]

		Soient $(a,b,c) \in \Z^3$ tels que $\left\{\begin{array}{l c r}a \wedge b = 1\\a|bc\end{array}\right.$\\
		Alors $a|c$\\

\end{prop}

\begin{prv}

		D’après le théorème de Bézout,\\
		$au+bv = 1$ avec $(u,v) \in \Z^2$\\
		D’où $acu+bcv = c$\\

		$\left\{\begin{array}{l c r}a|acu\\a|bcv\end{array}\right. $ \\
		Donc $a|acu+bcv $\\
		Donc $a|c$\\

\end{prv}

\begin{rmk}[Inversion Modulo $n$]

		Soit $x \in \Z$\\
		$\exist?y \in \Z, xy \equiv 1 [n]$\\
		$(\iff$dans $\Z/n \Z, \overline{x} + \overline{y} = \overline{1} ?)$\\

		Avec $n=4 :$\\
				$\begin{matrix}x\ 0123\\0\ 0000\\1\ 0123 \\2 \ 0202 \\3\ 0321\end{matrix}$		$\left\{\begin{array}{l c r}1 \text{ et }3 \text{ sont inversibles modulo } 4 \\ 2 \text{ n'est pas inversible modulo } 4\end{array}\right.$ \\

		$3x \equiv 2[4]$\\
		$\iff 3 * 3x \equiv 3*2[4]$\\
		$\iff x \equiv 2[4]$\\

		$2x \equiv 1[4]$\\
		$\Longrightarrow 2*2x \equiv 2[4]$\\
		$\Longrightarrow 0 \equiv 2 [4]$\\

\end{rmk}

\begin{prop}[Congruences et Nombres Premiers]

		Soit $p$ un nombre premier\\
		Alors $\forall x \in \Z, x \not \equiv 0[p] \Longrightarrow \exists y \in \Z, xy \equiv 1 [p]$\\

\end{prop}

\begin{prv}

		Soit $x \in \Z$ tel que $x \not \equiv 0[p]$\\
		Soit $y \in \Z$\\
		$xy \equiv 1[p] \iff \exists u \in \Z, xy = 1+pu$\\
				$\iff \exists u \in \Z, xy -pu = 1$\\

		$y$ existe $\iff x \wedge p = 1$\\
				$\iff p \nmid x$\\

\end{prv}

\begin{prop}[Inversibilité Modulo $n$]

		Soit $n \in \N^*, x \in \Z$\\
		$x$ inversible modulo $n$ $\iff x \wedge n = 1$\\

\end{prop}

\begin{prv}

		Voir précédent\\

\end{prv}

\begin{prop}[Petit Théorème de Fermat]

		Soit $p$ premier, $a \in \Z$\\
		$a^p \equiv a[p]$\\

\end{prop}

\begin{prv}

		Cas 1 : $a \equiv 0[p]$\\
				$a^p \equiv 0^p[p]$\\
				$a^p \equiv 0[p] \equiv a [p]$\\

		Cas 2 : $a \not \equiv 0 [p]$\\
				Alors $a \wedge p = 1$\\

				On pose $\forall i \in \N^*, r_i$ le reste de la division de $ia$ par $p$\\
				Soit $i \in [\![1,p-1]\!]$\\
						$r_i = 0 \Longrightarrow p|ia \Longrightarrow p|i$		Contradiction\\
						$\forall i \in \N^*, r_i \neq 0$\\

				Soit $(i,j) \in [\![1,p-1]\!]^2, i \neq j$\\
						On suppose $r_i = r_j$, alors $ia \equiv ja [p]$\\
						Or, $a \wedge p = 1$ donc $a$ est inversible modulo $p$\\
						Donc $a \equiv j[p]$ donc $i = j$		Contradiction\\

				Ainsi, $r_1...r_{p-1} \in [\![1,p-1]\!]$ distincts donc ils prennent toutes les valeurs de $[\![1,p-1]\!]$\\
				$i \longmapsto r_i$ est injective\\
				$\{r_1...r_{p-1}\} = [\![1,p-1]\!]$\\

				Donc $\overset{p-1}{\underset{k=1}{\prod}} r_i = (p-1)!$\\
				Donc $\overset{p-1}{\underset{k=1}{\prod}} ia \equiv (p-1)![p]$\\
				Donc $(p-1)! a^{p-1} \equiv (p-1)![p]$\\

				D’où $(p-1)! \equiv 0[p] \iff p |1*2*3\dots*(p-1)$\\
						$\iff\exists i \in [\![1,p-1]\!], p|i$\\
				Donc $(p-1)! \not \equiv 0[p]$\\
				Donc $(p-1)!$ est inversible modulo $p$\\
				Donc $a^p \equiv 1[p]$\\
				Donc $a^p \equiv a[p]$\\



\end{prv}


\part{Décomposition en Facteurs Premiers}


\begin{defn}[Nombre Premier]

		Soit $n \in \N$\\
		On dit que $n$ est premier si\\
				$n \ge 2$\\
				les seuls diviseurs entiers de $n$ sont $1$ et $n$\\

\end{defn}

\begin{prop}[Infinité de Nombres Premiers]

		Il y a une infinité de nombres premiers\\

\end{prop}

\begin{prv}

		On suppose qu’il n’y a qu’un nombre fini de nombres premiers\\
				$p_1<...<p_n$\\

		On pose $N = p_n*...*p_1+1$\\
		$N>p_n$ donc $N$ n’est pas premier\\
		$N$ a d’autres diviseurs positifs que $1$ et $N$\\
		$N$ est divisible par un nombre entre $2$ et $N-1$\\

		Soit $p = \text{min}(\{k \in [\![2,N-1]\!] | k |N\})$\\
		$p$ est premier		(Tout diviseur de $p$ divise aussi $N$)\\
		$\exists i \in [\![1,n]\!], p_i=p$\\
		$p_i|N$\\
		$p_i|N-p_1...p_n$\\
		$p_i|1$		Contradiction\\

		Donc il y a une infinité de nombres premiers\\

\end{prv}

\begin{prop}[Théorème Fondamental de l’Arithmétique]

		“Tout entier se décompose en un unique produit de nombres premiers”\\

		Soient $n \in \N$ tel que $n \ge 2$ et $\mathcal{P}$ l’ensemble des nombres premiers\\
		$\exist! \nu : \mathcal{P} \longrightarrow \N$ telle que $\left\{\begin{array}{l c r}\{p \in \mathcal{P}|\nu(p) \neq 0\} \text{ est fini} \\ n = {\underset{p \in \mathcal{P}}{\prod}}p^{\nu(p)} \end{array}\right.$\\

\end{prop}

\begin{prv}

		Existence : Déjà vue : Récurrence Forte (Chapitre 9)\\

		Unicité : Soit $n \ge 2$ et $\nu : \mathcal{P} \longrightarrow \N$ telle que\\
		${(*)\underset{p \in \mathcal{ P}}{\prod}}p^{\mu(p)} = {\underset{p \in \mathcal{ P}}{\prod}}p^{\nu(p)}$	avec $\left\{\begin{array}{l c r}\mu\neq \nu \\ M = \{p \in \mathcal{P}|\mu(p)\neq 0\} \text{ fini} \\ M = \{p \in \mathcal{P}|\nu(p)\neq 0\} \text{ fini} \\ n \text{ minimale pour cette propriété} \end{array}\right.$\\

		Soit $p \in M, \mu(p) \neq 0$ donc $p|n$\\
		Si $\nu(p) = 0, \forall q \in \N, p \wedge q = 1$ donc $p|1$		Contradiction avec le théorème de \Gauss\\

		Donc on peut simplifier $(*)$ par $p$\\
		On a alors $2$ décompositions de $\frac{n}{p}<n$		Contradiction\\
		Donc $n$ n’existe pas\\

\end{prv}

\begin{prop}[Divisibilité et Nombres Premiers]

		Soient $(,b,c) \in \N^3$ supérieurs à $2$\\
		On pose $a = {\underset{p \in \mathcal{ P}}{\prod}} p^{\alpha(p)}$ et $b = {\underset{p \in \mathcal{ P}}{\prod}}p^{\beta(p)}$\\
		$a|b \iff \forall p \in \mathcal{P}, \alpha(p) \leq \beta(p)$\\

\end{prop}

\begin{prv}

		“$\Longrightarrow$” : On suppose $a|b, \exist k, b = ak$\\
				On pose $k = {\underset{p \in \mathcal{ P}}{\prod}}p^{\kappa(p)}$\\
				et donc $b = {\underset{p \in \mathcal{ P}}{\prod}}p^{\alpha(p)}{\underset{p \in \mathcal{ P}}{\prod}}p^{\kappa(p)} = {\underset{p \in \mathcal{ P}}{\prod}}p^{\alpha(p) + \kappa(p)}$\\

				Par unicité de la décomposition en facteurs premiers,\\
				$\forall p \in \mathcal{P}, \beta(p) = \alpha(p) + \kappa(p) \ge \alpha(p)$\\

		“$\Longleftarrow$” : On suppose $\forall p \in \mathcal{P}, \beta(p) \ge \alpha(p)$\\
				On pose $\forall p \in \mathcal{P}, \kappa(p) = \beta(p) - \alpha(p) \in \N$\\

				Tous les $\alpha(p)$ et $\beta(p)$ sont nuls à partir d’un certain rang\\
				C'est donc le cas aussi pour les $\kappa(p)$\\
				Donc on a le droit de former le produit\\
				${\underset{p \in \mathcal{ P}}{\prod}}p^{\kappa(p)} \in \N$\\

				On pose $k = {\underset{p \in \mathcal{ P}}{\prod}}p^{\kappa(p)}$ et $ak = {\underset{p \in \mathcal{ P}}{\prod}}p^{\alpha(p)}{\underset{p \in \mathcal{ P}}{\prod}}p^{\kappa(p)} = {\underset{p \in \mathcal{ P}}{\prod}}p^{\alpha(p) + \kappa(p)} = {\underset{p \in \mathcal{ P}}{\prod}}p^{\beta(p)}=b$\\

\end{prv}

\begin{prop}[Produit de Facteurs Premiers, PGCD et PPCM]

		Avec les notations précédentes,\\

		$a \wedge b = {\underset{p \in \mathcal{ P}}{\prod}}p^{\text{min}(\alpha(p), \beta(p))}$	et	$a \vee b = {\underset{p \in \mathcal{ P}}{\prod}}p^{\text{max}(\alpha(p), \beta(p))}$\\
		\\

\end{prop}

\begin{crlr}

$(a \wedge b)(a \vee b) = ab$\\

\end{crlr}

\begin{prv}

		$(a \wedge b)(a \vee b) = {\underset{p \in \mathcal{ P}}{\prod}}p^{\text{min}(\alpha(p), \beta(p))}{\underset{p \in \mathcal{ P}}{\prod}}p^{\text{max}(\alpha(p), \beta(p))}$\\
				$= {\underset{p \in \mathcal{ P}}{\prod}}p^{\text{min}(\alpha(p),\beta(p)) + \text{max}(\alpha(p), \beta(p))}$\\
				$={\underset{p \in \mathcal{ P}}{\prod}}p^{\alpha(p) + \beta(p)} = ab$\\

\end{prv}

\fi
	}

	{
		\chap[10]{Nombres entiers - $\N$}
		\renewcommand{\cwd}{../chap10}
		\begin{defn}
	Soit $E$ un $\mathbbm{K}$-espace vectoriel. On dit que $E$ est de \underline{dimension finie} si $E$ a au moins une famille génératrice finie. On dit que $E$ est de \underline{dimension infinie} sinon.
	\index{dimension finie (espace vectoriel)}
	\index{dimension infinie (espace vectoriel)}
\end{defn}

\begin{thm}
	[Théorème de la base extraite]
	Soit $E$ un $\mathbbm{K}$-espace vectoriel non nul de dimension finie. Soit $\mathcal{G}$ une famille génératrice finie de $E$. Alors, il existe une base $\mathcal{B}$ de $\mathcal{E}$ telle que $\mathcal{B} \subset \mathcal{G}$.
\end{thm}

\begin{prv}
	[par récurrence sur $\#G = \Card(G)$]
	\begin{itemize}
		\item Soit $E$ un $\mathbbm{K}$-espace vectoriel non nul engendré par $\mathcal{G} = (u)$.\\
			Si $u = 0_E$, alors $E = \{0_E\}$: une contradiction $\lightning$ \\
			Donc $u \neq 0_E$ donc $(u)$ est libre. En effet, \[
				\forall \lambda \in \mathbbm{K}, \lambda u = 0_E \implies \lambda = 0_\mathbbm{K}
			\] Donc $\mathcal{G}$ est une base de $E$.\\
		\item Soit $n \in \N_*$. Soit $E$ un $\mathbbm{K}$-espace vectoriel. On suppose que si $E$ a une famille génératrice constituée de $n$ vecteurs, alors on peut extraire de cette famille une base de $E$.\\
			Soit $\mathcal{G}$ une famille génératrice de $E$ avec $n+1$ vecteurs.\\
			Si $\mathcal{G}$ est libre, alors $\mathcal{G}$ est une base de $E$. \\
			Si $\mathcal{G}$ n'est pas libre, alors il existe $u \in \mathcal{G}$ tel que $u \in \Vect(\mathcal{G}\setminus \{u\})$ \\
			Donc $\mathcal{G}\setminus \{u\}$ engendre $E$. Or, $\mathcal{G}\setminus \{u\}$ possède $n$ vecteurs. D'après l'hypothèse de récurrence, il existe une base $\mathcal{B}$ de $E$ telle que \[
				\mathcal{B} \subset \mathcal{G} \setminus \{u\} \subset \mathcal{G}
			\] 
	\end{itemize}
\end{prv}

\begin{crlr}
	Tout espace de dimension finie a une base.
	\qed
\end{crlr}

\begin{thm}
	[Théorème de la base incomplète]
	Soit $E$ un $\mathbbm{K}$-espace vectoriel de dimension finie, $\mathcal{G}$ une famille génératrice finie de $E$. $\mathcal{L}$ une famille libre de $E$. Alors, il existe une base $\mathcal{B}$ de $E$ telle que \[
		\mathcal{L} \subset \mathcal{B} \text{ et } \mathcal{B}\setminus \mathcal{L} \subset \mathcal{G}
	\] 
\end{thm}

\begin{prv}
	[par récurrence sur $\#(\mathcal{G}\setminus\mathcal{L})$]
	\begin{itemize}
		\item Avec les notations précédentes, on suppose que $\mathcal{G}\setminus\mathcal{L} \neq \O$ \[
				\forall u \in \mathcal{G}, u \in \mathcal{L}
			\] Donc $\mathcal{G} \subset \mathcal{L}$ donc $\mathcal{L}$ est génératrice donc $\mathcal{L}$ est une base de $E$. On pose $\mathcal{B} = \mathcal{L}$ et alors \[
				\mathcal{L} \subset  \mathcal{B} \text{ et } \mathcal{B}\setminus\mathcal{L} = \O \subset  \mathcal{G}
			\] 
		\item Soit $n \in \N$. On suppose que si $\mathcal{G}$ est génératrice et $\mathcal{L}$ libre avec $\#(\mathcal{G}\setminus\mathcal{L}) = n$ alors il existe une base $\mathcal{B}$ de $E$ telle que \[
			\mathcal{L}\subset \mathcal{B} \text{ et } \mathcal{B}\setminus\mathcal{L}\subset \mathcal{G}
		\] Soient à présent $\mathcal{G}$ une famille génératrice de $E$ et $\mathcal{L}$ une famille libre de $E$ telles que $\#(\mathcal{G}\setminus\mathcal{L}) = n+1 > 0$\\
		Si $\mathcal{L}$ engendre $E$, alors $\mathcal{L}$ est une base de $E$. On pose $\mathcal{B} = \mathcal{L}$ et on a bien \[
			\mathcal{L} \subset  \mathcal{B} \text{ et } \mathcal{B} \setminus \mathcal{L} = \O \subset  \mathcal{G}
		\] On suppose que $\mathcal{L}$ n'engendre pas $E$. Il existe $u \in \mathcal{G}$ tel que $u \not\in \Vec(\mathcal{L})$ (car sinon, $\mathcal{G} \subset \Vect(\mathcal{L})$ et donc $\underbrace{\Vect(\mathcal{G})}_{= E} \subset  \underbrace{\Vect(\mathcal{L})}_{ \subset E}$\\
		Donc $\mathcal{L} \cup \{u\} $ est libre. On pose $\mathcal{L}' = \mathcal{L} \cup \{u\} $ \[
			\mathcal{G}\setminus \mathcal{L}' = \mathcal{G}\setminus (\mathcal{L} \cup \{u\}) = (\mathcal{G}\setminus\mathcal{L})\setminus \{u\} 
		\] donc $\#(\mathcal{G}\setminus\mathcal{L}') = n+1 -1 = n$\\
		D'après l'hypothèse de récurrence, il existe $\mathcal{B}$ une base de $E$ telle que \[
			\mathcal{L} \subset  \mathcal{L}' \subset \mathcal{B} \text{ et } \mathcal{B}\setminus \mathcal{L}' \subset \mathcal{G}
		\] \[
			\mathcal{B} \setminus \mathcal{L} = \underbrace{\mathcal{B}\setminus\mathcal{L}'}_{\subset \mathcal{G}} \cup \underbrace{\{u\}}_{\subset \mathcal{G} \text{ car } u \in \mathcal{G}}
		\] On a $\mathcal{B}\setminus\mathcal{L}\subset \mathcal{G}$
	\end{itemize}
\end{prv}

\begin{thm}
	Soit $E$ un $\mathbbm{K}$-espace vectoriel de dimension finie. Toutes les bases de $E$ ont le même cardinal.
\end{thm}

\begin{prv}
	Soit $\mathcal{G}$ une famille génératrice finie de $E$ et $\mathcal{B} \subset  \mathcal{G}$ une base de $E$. On note $n = \#\mathcal{B}$ \\
	Soit $\mathcal{B}'$ une base de $E$. On pose $p = n - \#(\mathcal{B} \cap  \mathcal{B}')$. Montrons par récurrence sur  $p$ que $\#\mathcal{B} = \#\mathcal{B}'$ 
	\begin{itemize}
		\item On suppose que $p = 0$. Alors, $\#(\mathcal{B} \cap \mathcal{B}') = n$ \\
			Or, $\mathcal{B}' \cap \mathcal{B} \subset \mathcal{B}$ donc $\mathcal{B} \cap \mathcal{B}' = \mathcal{B}$ donc $\mathcal{B} \subset  \mathcal{B}'$ et donc $\mathcal{B} = \mathcal{B}'$ 
		\item Soit $p \in \N$. On suppose que si $\mathcal{B}'$ est une base de $E$ telle que $n - \#(\mathcal{B} \cap \mathcal{B}') = p$, alors $\#\mathcal{B}' = n$ \\
			Aoit $\mathcal{B}'$ une base de $E$ telle que $n - \#(\mathcal{B}\cap \mathcal{B}') = p+1 > 0$ \\
			Donc $\mathcal{B} \cap \mathcal{B}' \neq \mathcal{B}$. Soit $u \in \mathcal{B}' \setminus \mathcal{B}$. D'après le lemme d'échange, il existe $v \in \mathcal{B}\setminus \mathcal{B}'$ tel que $\mathcal{B}' \setminus \{u\} \cup \{v\}$ est une base de $E$. On pose $\mathcal{B}'' = \mathcal{B}' \setminus \{u\} \cup \{v\}$ 
			\begin{align*}
				\mathcal{B}'' \cap \mathcal{B} &= \left( (\mathcal{B}' \setminus \{u\})  \cap \mathcal{B} \right) \cup \{v\} \\
				&= (\mathcal{B}' \cap \mathcal{B}) \cup \{v\} \\
			\end{align*}
			donc,
			\begin{align*}
				n - \#(\mathcal{B}'' \cap \mathcal{B}) &= n - (\#(\mathcal{B}' \cap \mathcal{B}) + 1) \\
				&= p+1- 1 \\
				&= p \\
			\end{align*}
			D'après l'hypothèse de récurrence, \[
				\#\mathcal{B}'' = n
			\] Or, $\#\mathcal{B}'' = \#\mathcal{B}'$
	\end{itemize}
\end{prv}

\begin{lem}
	Soient $\mathcal{B}$ et $\mathcal{B}'$ deux bases de $E$ telles que $\mathcal{B}\subset \mathcal{B}'$. Alors, $\mathcal{B} = \mathcal{B}'$.
\end{lem}

\begin{prv}
	On suppose $\mathcal{B}' \neq \mathcal{B}$. Soit $u \in \mathcal{B}' \setminus \mathcal{B}$
	$u \in E = \Vect(\mathcal{B})$ donc $\mathcal{B} \cup \{u\}$ n'est pas libre.
	Donc $\mathcal{B}\cup \{u\} \subset \mathcal{B}'$ et $\mathcal{B}'$ est libre donc $\mathcal{B}\cup \{u\}$ est libre: une contradiction $\lightning$
\end{prv}

\begin{lem}
	[Lemme d'échange] Soient $\mathcal{B}_1$ et $\mathcal{B}_2$ deux bases de $E$ et $u \in \mathcal{B}_1 \setminus \mathcal{B}_2$. Alors, il existe $v \in \mathcal{B}_2$ tel que $(\mathcal{B}_1 \setminus \{u\}) \cup \{v\}$ soit une base de $E$.
\end{lem}

\begin{prv}
	[1${}^\text{nde}$ méthode]
	On suppose que pout tout $v \in \mathcal{B}_2$, $(\mathcal{B}_1\setminus \{u\}) \cup \{v\}$ n'est pas une base de $E$
	Soit $v \in \mathcal{B}_2$.
	\begin{itemize}
		\item Supposons $(\mathcal{B}_1\setminus \{u\})\cup \{v\}$ non libre. $\mathcal{B}_1 \setminus \{u\}$ est libre. Donc $v \in \Vect(\mathcal{B}_1 \setminus \{u\})$
		\item Supposons $(\mathcal{B}_1\setminus \{u\}) \cup \{v\}$ non génératrice.
			Comme $\mathcal{B}_1$ engendre $E$, $u \not\in \Vect(\mathcal{B}_1\setminus \{v\})$.
			On suppose que $\mathcal{B}_1 \neq \mathcal{B}_2$.
			$\forall v \in \mathcal{B}_2 \setminus \mathcal{B}_1, \Vect(\mathcal{B}_1 \setminus \{v\}) = \Vect(\mathcal{B}_1) = E \ni u$ 
			donc, $(\mathcal{B}_1\setminus \{u\}) \cup \{v\}$ engendre $E$ et donc \[
				v \in \Vect(\mathcal{B}_1 \setminus \{u\})
			\] On a aussi \[
				\forall v \in \mathcal{B}_1 \setminus \{u\}, v \in \Vect(\mathcal{B}_1\setminus \{u\})
			\] Comme $u \not\in \mathcal{B}_2$, on a \[
				\forall v \in \mathcal{B}_2, v \in \Vect(\mathcal{B}_1\setminus \{u\})
			\] docn \[
				E = \Vect(\mathcal{B}_2) \subset \Vect(\mathcal{B}_1\setminus \{u\})
			\] donc $\mathcal{B}_1\setminus \{u\}$ engendre $E$ donc $\mathcal{B}_1\setminus \{u\}$ est une base de $E$. Or, $\mathcal{B}_1 \setminus \{u\}  \subset  \mathcal{B}_1$, donc $\mathcal{B}_1\setminus \{u\} = \mathcal{B}_1$
	\end{itemize}
\end{prv}

\begin{prv}
	[2${}^\text{nde}$ méthode]
	On suppose que pout tout $v \in \mathcal{B}_2$, $(\mathcal{B}_1\setminus \{u\}) \cup \{v\}$ n'est pas une base de $E$
	\begin{itemize}
		\item Comme $u \in \mathcal{B}_1 \setminus \mathcal{B}_2$, nécéssairement $\mathcal{B}_1 \neq \mathcal{B}_2$ donc $\mathcal{B}_2 \not\subset \mathcal{B}_1$, donc $\mathcal{B}_2\setminus\mathcal{B}_1 \neq \O$ 
		\item Soit $v \in \mathcal{B}_2\setminus\mathcal{B}_1$. Il existe $(\lambda_w)_{w\in\mathcal{B}_1}$ une famille de scalaires presque nulle telle que \[
				v = \sum_{w \in \mathcal{B}_1} \lambda_w w - \lambda_u u + + \sum_{w \in \mathcal{B}_1\setminus \{u\}}\lambda_w w
			\]
			Si $\lambda_u \neq 0_E$, alors
			\begin{align*}
				u &= \lambda_u^{-1}\left( v - \sum_{w \in \mathcal{B}_1 \setminus \{u\}} \lambda_w w \right)\\
					&\in \Vect(\mathcal{B}_1\setminus \{u\} \cup v)
			\end{align*}
			 donc $\mathcal{B}_1 \subset \Vect(\mathcal{B}_1\setminus \{u\} \cup \{v\})$\\
			 et donc $E \subset  \Vect(\mathcal{B}_1 \setminus \{u\} \cup \{v\})$ \\
			 et donc $\mathcal{B}_1 \setminus \{u\} \cup \{v\}$ engendre $E$ \\
			 donc $\mathcal{B}_1 \setminus \{u\} \cup \{v\}$ n'est pas libre\\
			 donc $v \in \Vect(\mathcal{B}_1\setminus \{u\})$ (car $\mathcal{B}_1 \setminus \{u\}$ est libre\\
			 donc $\lambda_u = 0_\mathbbm{K}$ $\lightning$\\`

			 Donc, $\lambda_u = 0_\mathbbm{K}$, docn $v \in \Vect(\mathcal{B}_1\setminus \{u\})$ \\
			 On vient de prouver que
			 \begin{align*}
			 	\mathcal{B}_2 \setminus \mathcal{B}_1 \subset \Vect(\mathcal{B}_1 \setminus \{u\})\\
			 	\mathcal{B}_1 \setminus \{u\} \subset \Vect(\mathcal{B}_1 \setminus \{u\})\\
			 \end{align*}
			 Comme $u \not\in \mathcal{B}_2$, \[
			 	\mathcal{B}_2 \subset \Vect(\mathcal{B}_1 \setminus \{u\})
			 \] donc \[
			 	E = \Vect(\mathcal{B}_2) \subset  \Vect(\mathcal{B}_1 \setminus \{u\})
			 \] donc $\mathcal{B}_1 \setminus \{u\}$ engendre $E$. Donc,  $\mathcal{B}_1 \setminus \{u\}$ est une base de $E$.\\
			 Or, $\mathcal{B}_1 \setminus \{u\} \subset  \mathcal{B}_1$, donc $\mathcal{B}_1 \setminus \{u\} = \mathcal{B}_1$
	\end{itemize}
\end{prv}

\begin{defn}
	Soit $E$ un $\mathbbm{K}$-espace vectoriel de dimension finie. Le cardinal commun à toutes les bases de $E$ est appelé \underline{dimension} de $E$ est notée $\dim(E)$ ou $\dim_\mathbbm{K}(E)$\\
	C'est donc aussi le nombre de coordonnées de n'importe quel vecteur dans n'importe quelle base.
	\index{dimension (espace vectoriel)}
\end{defn}

\begin{exm}
	\begin{enumerate}
		\item $\dim_\R(\C) = 2$ et $\dim_\C(\C) = 1$ 
		\item $\dim_\mathbbm{K}(\mathbbm{K}^{n}) = n$ 
		\item $\dim_{\mathbbm{K}}(\mathcal{M}_{n,p}(\mathbbm{K})) = np$
	\end{enumerate}
\end{exm}

\begin{crlr}
	Soit $E$ un $\mathbbm{K}$-espace vectoriel de dimension finie, $\mathcal{L}$ une famille libre de $E$, $\mathcal{G}$ une famille génératrice de $E$. On note $n = \dim(E)$
	\begin{enumerate}
		\item $\#\mathcal{G} \ge n$ et $(\#\mathcal{G} = n \implies \mathcal{G} \text{ est une base de } E$)
		\item $\#\mathcal{L} \le n$ et $(\#\mathcal{L} = n \implies \mathcal{L} \text{ est une base de } E$)
	\end{enumerate}
\end{crlr}

\begin{crlr}
	$\R^{\R}$ est de dimension infinie.
	$\forall i \in \N, e_i: x \mapsto x^i$\\
	$(e_i)_{i\in\N}$ est libre dans $\R^\R$
\end{crlr}

\begin{prop}
	Soient $E$ et $F$ deux $\mathbbm{K}$-espaces vectoriels de dimension finie. Alors $E\times F$ est de dimension finie et $\dim(E\times F) = \dim(E) + \dim(F)$
\end{prop}

\begin{prv}
	Soit $(e_1,\ldots, e_n)$ une base de $E$, $(f_1, \ldots, f_p)$ une base de $F$.
	On pose \[
		\left\{\begin{array}
			{r c l}
			u_1 &=& (e_1,0_F)\\
			u_2 &=& (e_2,0_F)\\
					&\vdots&\\
			u_n &=& (e_n,0_F)\\
			u_{n+1} &=& (0_E, f_1)\\
			u_{n+2} &=& (0_E, f_2)\\
					&\vdots&\\
			u_{n+p} &=& (0_E,f_p)\\
		\end{array}\right.
	\]
	Soit $(x,y) \in E\times F$. \[
		\begin{cases}
			\exists (x_1,\ldots,x_n)\in \mathbbm{K}^n, x = \sum_{i=1}^{n} x_ie_i
			\exists (y_1,\ldots,y_n)\in \mathbbm{K}^n, x = \sum_{j=1}^{p} y_jf_j
		\end{cases}
	\] 
	\begin{align*}
		(x,y) &= \left( \sum_{i=1}^{n} x_ie_i, \sum_{i=1}^{p} y_jf_j \right)  \\
		&= \sum_{i=1}^{n} x_i (e_i + 0_F) + \sum_{j=1}^{p} y_j (0_E, f_j) \\
		&= \sum_{i=1}^{n} x_i u_i + \sum_{j=1}^{p} y_j u_{n+j} \\
	\end{align*}
	Donc, $E\times F = \Vect(u_1, \ldots, u_{n+p})$ donc $E\times F$ est de dimension finie.\\
	Soit $(\lambda_1, \ldots, \lambda_{n+p}) \in \mathbbm{K}^{n+p}$ tel que \[
		(*): \quad \sum_{k=1}^{n+p} \lambda_ku_k = 0_{E\times F} = (0_E, 0_F)
	\]
	\begin{align*}
		(*) &\iff \sum_{k=1}^{n} \lambda_k (e_k, 0_F) + \sum_{k=n+1}^{p} \lambda_k(0_E, f_{k-n}) = (0_E, 0_F)\\
				&\iff \begin{cases}
					\sum_{k=1}^{n} \lambda_k e_k = 0_E\\
					\sum_{k=n+1}^{p} \lambda_k f_{k-n} = 0_F
				\end{cases}\\
				&\iff \begin{cases}
					\forall k \in \left\llbracket 1,n \right\rrbracket, \lambda_k = 0_\mathbbm{K} \qquad&(\text{car $(e_1,\ldots,e_n)$ est libre})\\
					\forall k \in \left\llbracket n+1,n+p \right\rrbracket, \lambda_k = 0_\mathbbm{K} \qquad&(\text{car $(f_1,\ldots,f_n)$ est libre})\\
				\end{cases}
	\end{align*}
	Donc $(u_1, \ldots, u_{n+p})$ est une base de $E\times F$. Donc, $\dim(E\times F) = n + p = \dim(E) + \dim(F)$
\end{prv}

\begin{rmk}
	[Convention]
	\[\dim\big(\{0_E\}\big) = 0\]
\end{rmk}

\begin{thm}
	Soit $E$ un $\mathbbm{K}$-espace vectoriel de dimension finie, $F$ un sous-espace vectoriel de $E$. Alors, $F$ est de dimension finie et  $\dim(F) \le \dim(E)$\\
	Si $\dim(F) = \dim(E)$, alors $F = E$
\end{thm}

\begin{prv}
	On considère \[
		A = \{k \in \N \mid \text{il existe une famille libre de $F$ à $k$ éléments}\} 
	\]
	On suppose $F \neq \{0_E\}$.
	\begin{itemize}
		\item Soit $u \in F\setminus \{0_E\}$. $(u)$ est libre donc $1 \in A$ et donc $A \neq \O$
		\item Soit $\mathcal{L}$ une famille libre de $F$. Alors, $\mathcal{L}$ est une famille libre de $E$ \\
			donc $\#\mathcal{L} \le \dim(E)$\\
			Donc $A$ est majorée par $\dim(E)$ \\
			On en déduit que $A$ a un plus grand élément $p$.
		\item Soit $\mathcal{L}$ une famille libre de $F$ avec $p$ éléments.\\
			Si $\mathcal{L}$ n'engendre pas $F$, alors il existe $u\in F$ tel que $u\not\in \Vect(\mathcal{L})$ et donc $\mathcal{L} \cup \{u\}$ est une famille libre de $F$, donc $p+1 \in A$ en contradiction avec la maximalité de $p$.\\
			Donc $\mathcal{L}$ est une base de $F$ donc $F$ est de dimension finie et $\dim(F) = p \le \dim(E)$\\
	\end{itemize}

	Soit $\mathcal{B}$ une base de $F$. Alors, $\mathcal{B}$ est aussi une famille de libre de de $E$. Donc $\#\mathcal{B} \le \dim(E)$ donc $\dim(F) = \dim(E)$ \\
	Si $\dim(F) = \dim(E)$, alors $\mathcal{B}$ est une base de $E$, et donc $F = \Vect(\mathcal{B}) = E$
\end{prv}

\begin{prop}
	[Formule de Grassmann]
	Soit $E$ un $\mathbbm{K}$-espace vectoriel de dimension finie, $F$ et $G$ deux sous-espace vectoriels de $E$. Alors, \[
		\dim(F+G) = \dim(F) + \dim(G) - \dim(F\cap G)
	\] 
\end{prop}

\begin{prv}
	Soit $(e_1, \ldots, e_p)$ une base de $F\cap G$. $(e_1,\ldots,e_p)$ est une famille libre de $F$.\\
	On complète $(e_1, \ldots, e_p)$ en une base $(e_1, \ldots, e_p, u_1, \ldots, u_q)$ de $F$.\\
	De même, on complète $(e_1, \ldots, e_p)$ en une base $(e_1, \ldots, e_p, v_1, \ldots, v_r)$ de $G$.\\
	On pose  $\mathcal{B} = (e_1, \ldots, e_p, u_1, \ldots, u_q, v_1, \ldots, v_r)$. Montrons que $\mathcal{B}$ est une base de $F+G$
	\begin{itemize}
		\item Soit $u \in F+G$ \\
			On pose $u = v+w$ avec $\begin{cases}
				v\in F\\
				w \in G
			\end{cases}$.\\
			On pose $v = \sum_{i=1}^p \lambda_i e_i + \sum_{i=1}^q \mu_i u_i$ avec $(\lambda_1, \ldots, \lambda_p, \mu_1, \ldots, \lambda_q) \in \mathbbm{K}^{p+q}$\\
			On pose aussi $w = \sum_{i = 1}^p \lambda'_ie_i + \sum_{j=1}^r \nu_j v_j$ avec $(\lambda_1',\ldots,\lambda_p', \nu_1, \ldots, \nu_r) \in \mathbbm{K}^{p+r}$\\
			D'où, \[
				u = \sum_{i=1}^p (\lambda_i + \lambda'_i)e_i + \sum_{j=1}^q \mu_j u_j + \sum_{k=1}^r \nu_k v_k \in \Vect(\mathcal{B})
			\]
		\item Soient $(\lambda_1, \ldots, \lambda_p, \mu_1, \ldots, \mu_q, \nu_1, \ldots, \nu_r) \in \mathbbm{K}^{p+q+r}$.\\
			On suppose \[
				(*)\quad \sum_{i=1}^{p}\lambda_ie_i + \sum_{j=1}^q\mu_ju_j + \sum_{k=1}^r \nu_k v_k = 0_E
			\] 
			D'où, \[
				\underbrace{\sum_{i=1}^p\lambda_i e_i + \sum_{j=1}^q \mu_ju_j}_{\in F} = \underbrace{-\sum_{k=1}^r\nu_jv_k}_{\in G}
			\] 
			Donc, \[
				f = \sum_{i=1}^p \lambda_i e_i + \sum_{j=1}^q \mu_j u_j \in F\cap G
			\] Comme $(e_1, \ldots, e_p)$ est une base de $F\cap G$, $\exists ! (\lambda_1', \ldots, \lambda_p') \in \mathbbm{K}^p$ tel que \[
				f = \sum_{i=1}^p \lambda'_i e_i = \sum_{i=1}^p \lambda'_i e_i + \sum_{j=1}^q 0_\mathbbm{K}u_j
			\] Comme $(e_1, \ldots, e_p, u_1, \ldots, u_q)$ est une base de $F$, \[
				\forall k \in \left\llbracket 1, q \right\rrbracket, \mu_j = 0_\mathbbm{K}
			\] De même, \[
				\forall k \in \left\llbracket 1,r \right\rrbracket , \nu_k = 0_\mathbbm{K}
			\] On remplace dans $(*)$ pour trouver \[
				\sum_{i=1}^p \lambda_ie_i = 0_E
			\] Comme $(e_1, \ldots, e_p)$ est libre, \[
				\forall i \in \left\llbracket 1,p \right\rrbracket, \lambda_i = 0_\mathbbm{K}
			\] Donc $\mathcal{B}$ est libre.\\
			Donc, 
			\begin{align*}
				\dim(F+G) &=  p +q + r \\
				&= (p+q)+ (p+r) - p \\
				&= \dim(F) + \dim(G) - \dim(F\cap G) \\
			\end{align*}
	\end{itemize}
\end{prv}

\begin{crlr}
	Avec les hypothèse précédentes, \[
		E = F \oplus G \iff \begin{cases}
			F \cap  G = \{0_E\} \\
			\dim(E) = \dim(F) + \dim(G)
		\end{cases}
	\] 
\end{crlr}

\begin{prv}
	\begin{itemize}
		\item[``$\implies$''] On suppose $E = F \oplus G$ \\
			Comme la somme est directe, $F \cap G = \{0_E\}$ 
			\begin{align*}
				\dim(E) &= \dim(F)\\
				&= \dim(F) + \dim(G) - \dim(F\cap G)\\
				&= \dim(F) + \dim(G)\\
			\end{align*}
		\item[``$\impliedby$''] On suppose $F\cap G = \{0_E\}$ et $\dim(E) = \dim(F) + \dim(G)$.\\
			On sait déjà que $F+G = F \oplus G$\\
			 \begin{align*}
				\dim(F+G) = \dim(F) + \dim(G) - \dim(F \cap G) = \dim(E)
			\end{align*}
			Donc $F + G = E$
	\end{itemize}
\end{prv}

\begin{prop}
	Soit $F$ un $\mathbbm{K}$-espace vectoriel de dimension finie $n$. Soit $\mathcal{B} = (e_1, \ldots, e_n)$ une base de $F$. L'application
	\begin{align*}
		f: \mathbbm{K}^n &\longrightarrow F \\
		(\lambda_1, \ldots, \lambda_n) &\longmapsto \sum_{i=1}^n \lambda_i e_i
	\end{align*} est bijective.\\
	Si $\mathbbm{K}$ est infini, $\mathbbm{K}^n$ aussi et donc $F$ aussi.\\
	Si $\#\mathbbm{K} = p \in \N_*$,
	\begin{align*}
		\#&\mathbbm{K}^n = p^n\\
		&\vrt=\\
		\#&F
	\end{align*}
\end{prop}


		\part{Dérivation}

\underline{Motivation}:

{
\begin{wrapfigure}{l}{3cm}
	\centering
	\begin{asy}
		import three;

		size(3cm);
		settings.render=0;
		settings.prc=false;
		currentprojection = obliqueZ;

		draw(unitbox);
		draw(shift(1.1Z + 0.05X) * (O -- X), Arrows3(TeXHead2));
		draw(shift(1.1Z + 0.05Y) * (O -- Y), Arrows3(TeXHead2));
		draw(shift(1.1X + 0.05Z) * (O -- Z), Arrows3(TeXHead2));

		label("$x$", (X/2) + (1.1Z + 0.05X), align=S);
		label("$y$", (Y/2) + (1.1Z + 0.05Y), align=W);
		label("$z$", (Z/2) + X, align=SE);
	\end{asy}
\end{wrapfigure}

\begin{align*}
	&S(x,y,z) = 2(xy + xz + yz)\\
	&V(x,y,z) = xyz
\end{align*}

On cherche à minimiser $S$ avec la contrainte $V = 1$.

Soit $f : \begin{array}{rcl}
	\left( \R_*^+ \right)^2 &\longrightarrow& \R \\
	(x,y) &\longmapsto& S\left( x,y,\frac{1}{xy} \right) = 2\left( xy + \frac{1}{y} + \frac{1}{x} \right).
\end{array}$

On cherche $(a,b) \in \left( \R^+_* \right)^2$ tel que \[
	\forall (x,y) \in (\R^+_*), f(x,y) \ge f(a,b).
\]
}

\begin{defn}
	Soit $f: U \to \R$ où $U$ est un ouvert de $\R^2$. Soit $(a,b) \in U$.
	\vspace{2mm}

	Si $\lim_{x \to a} \frac{f(x,b) - f(a,b)}{x - a} \in \R$, alors on dit que $f$ a une dérivée partielle suivant $x$ en $(a,b)$ et cette limite est notée \[
		\partial f_1(a,b) = \frac{\partial f}{\partial x}(a,b).
	\]

	Si $\lim_{y \to b} \frac{f(a,y) - f(a,b)}{y - b} \in \R$, alors on dit que $f$ a une dérivée partielle suivant $y$ et la limite est notée \[
		\partial f_2(a,b) = \frac{\partial f}{\partial y}(a,b).
	\]
\end{defn}

\begin{exm}
	\begin{enumerate}
		\item $f: (x,y) \mapsto xy + x - y$.

			\begin{align*}
				&\frac{\partial f}{\partial x} : (x,y) \mapsto y + 1,\\
				&\frac{\partial f}{\partial y} : (x,y) \mapsto x - 1.
			\end{align*}

		\item $f: (x,y) \mapsto xy + \frac{1}{y}+ \frac{1}{x}$.

			\begin{align*}
				&\frac{\partial f}{\partial x}: (x,y) \mapsto y - \frac{1}{x^2},\\
				&\frac{\partial f}{\partial y}: (x,y) \mapsto x - \frac{1}{y^2}.
			\end{align*}

		\item Trouver $f$ telle que $\begin{cases}
				(1): \qquad \frac{\partial f}{\partial x}=y,\\[2mm]
				(2): \qquad \frac{\partial f}{\partial y} = x.
			\end{cases}$

			D'après $(1)$ : \[
				\forall (x,y), \exists C(y) \in \R, f(x,y) = xy + C(y)
			\] et donc \[
				\frac{\partial f}{\partial y}(x,y) = x + C'(y)
			\] donc $C'(y) = 0$ et donc $C$ est constante.

		\item Trouver $f$ telle que $\begin{cases}
			\frac{\partial f}{\partial x} = -y,\\[2mm]
			\frac{\partial f}{ƒ\partial y} = x.
		\end{cases}$

		Ce n'est pas possible !
	\end{enumerate}
\end{exm}

\begin{defn}~\\
	\begin{minipage}{\linewidth}
		\begin{wrapfigure}{r}{4cm}
			\centering
			\vspace{-5mm}
			\begin{asy}
				import three;
				import graph3;
				size(4cm);

				settings.render = 0;
				settings.prc = false;
				currentprojection = obliqueX;

				draw(O -- X, Arrow3(TeXHead2));
				draw(O -- Y, Arrow3(TeXHead2));
				draw(O -- Z, Arrow3(TeXHead2));

				triple f(real x, real y, real z = 0) { return (x,y,cos(x - 0.5) * cos(y - 0.5)/1.2 + 0.15); }

				real inc = 1 / 5;

				for(real x = 0; x <= 1; x += inc) {
					draw(graph(
						new real(real t) { return x; }, // x
						new real(real y) { return y; }, // y
						new real(real y) { return f(x,y).z; }, // z
						0, 1
					), gray);
				}

				for(real y = 0; y <= 1; y += inc) {
					draw(graph(
						new real(real x) { return x; }, // x
						new real(real t) { return y; }, // y
						new real(real x) { return f(x,y).z; }, // z
						0, 1
					), gray);
				}

				path3 path1 = (0.8, 0.2, 0) .. (0.5, 0.5, 0) .. (0.3, 0.7, 0);
				path3 path2 = f(0.8, 0.2, 0) .. f(0.5, 0.5, 0) .. f(0.3, 0.7, 0);
				path3 d = (0.2, 0.3, 0) .. (0.3, 0.4, 0) .. (0.2, 0.7, 0) .. (0.8, 0.9, 0) .. (0.6, 0.2, 0) .. cycle;

				draw(path1, red, Arrow3(TeXHead2));
				draw(path2, red, Arrow3(TeXHead2, position=0.8));

				dot((0.5, 0.5, 0));
				dot(f(0.5, 0.5, 0));
				draw((0.5, 0.5, 0) -- f(0.5, 0.5, 0), dashed);
				draw(d);

				label("$w$", (0.3, 0.7, 0), red, align=SE);
				label("$U$", (0.8, 0.9, 0), align=SE);
			\end{asy}
		\end{wrapfigure}

		Soit $f: U \to \R$ où $U$ est un ouvert. Soit $(a,b) \in U$. Soit $w = (w_1, w_2) \in \R^2$.

		Si 
		\[
			\lim_{t\to 0} \frac{f(a + tw_1, b + tw_2) - f(a,b)}{t}
		\] existe et est réelle, alors on dit que $f$ a une dérivée dans la direction de $w$ et la limite est notée \[
			\mathrm{d}f(w)\,(a,b) = D_w(f)\,(a,b).
		\]
	\end{minipage}
\end{defn}

\begin{exm}
	\begin{align*}
		f: \left( \R_*^+ \right)^2 &\longrightarrow \R \\
		(x,y) &\longmapsto xy+\frac{1}{x}+\frac{1}{y}.
	\end{align*}

	On pose $(a,b) = (1,2)$, $w = (w_1, w_2) = (1,1)$.
	\begin{align*}
		\frac{f(1+t, 2+t) - f(1,2)}{t} &= \frac{1}{t} \left( (1+t)(2+t) + \frac{1}{1+t} + \frac{1}{2+t} - 3 - \frac{1}{2} \right) \\
		&= \frac{1}{t}\left(\cancel 2 + 3t + \po(t) + \cancel 1 - t + \po(t) + \frac{1}{2}\left( \cancel 1 - \frac{t}{2} + \po(t) \right) - \cancel3 - \cancel{\frac{1}{2}} \right) \\
		&= \frac{1}{t} \left( \frac{7}{4} t + \po(t) \right)  \\
		&= \frac{7}{4} + \po(1) \tendsto{t \to 0} \frac{7}{4}. \\
	\end{align*}

	Donc, \[
		\mathrm{d}f(1,1)\,(1,2) = \frac{7}{4}.
	\]
\end{exm}

\begin{rmk}~\\
	\begin{figure}[H]
		\centering
		\begin{asy}
			import solids;
			import graph;
			size(5cm);

			settings.render = 0;
			settings.prc = false;

			path3 par = graph(
				new real(real x) { return x; },
				new real(real x) { return 0; },
				new real(real x) { return x^2; },
				0,3);
			revolution r = revolution(par, axis=Z);

			path3 par2 = graph(
				new real(real x) { return x; },
				new real(real x) { return 0; },
				new real(real x) { return x^2; },
				-3,3);

			draw(r,1,longitudinalpen=nullpen);
			draw(r.silhouette());

			draw((-4, 0, -1) -- (-4, 0, 10) -- (4, 0, 10) -- (4, 0, -1) -- cycle, red);
			draw(par2, deepred);

			draw((4,4.5) -- (7, 4.5), black+0.5mm, Arrow(TeXHead));

			path par2d = graph(new real(real x) { return x^2; }, -3, 3);
			draw(shift((11, 0)) * par2d, deepred);

			dot(O);
			dot((11, 0));
		\end{asy}
	\end{figure}
\end{rmk}


%todo ajouter théorème-définition
\begin{thm}
	Soit $f : U \to \R$, $(a,b) \in U$. On suppose que $\frac{\partial f}{\partial x}$ et $\frac{\partial f}{\partial y}$ existent en $(a,b)$ et sont {\bfseries continues} en $(a,b)$. Alors,
	\begin{align*}
		&\forall (h, k) \in \R^2 \text{ tel que } (a +h, b + k) \in U,\\
		&f(a+ h, b + k) = f(a,b) + h \frac{\partial f}{\partial x}(a,b) + k \frac{\partial f}{\partial y}(a,b) + \po_{(h,k)\to (0,0)}\big(\|(h,k)\|\big).
	\end{align*}

	On dit que $f$ est de classe $\mathcal{C}^1$ si $\frac{\partial f}{\partial x}$ et $\frac{\partial f}{\partial y}$ existent et sont continues.

	\qed
\end{thm}

\begin{rmk}
	En physique, cette formule correspond à : \[
		\mathrm{d}f = \frac{\partial f}{\partial x}\mathrm{d}x + \frac{\partial f}{\partial y} \mathrm{d}y.
	\] En effet :
	\begin{align*}
		\mathrm{d}f &= f(x+ \mathrm{d}x, y + \mathrm{d}y) - f(x,y) \\
		&= \frac{\partial f}{\partial x} \mathrm{d}x + \frac{\partial f}{\partial y} \mathrm{d}y.
	\end{align*}
\end{rmk}

\begin{prop}
	Soit $f: U \to \R$ de classe $\mathcal{C}^1$ en $(a,b) \in U$. Alors, \[
		\forall w = (w_1, w_2) \in \R^2, \mathrm{d}f(w)\,(a,b) = w_1 \frac{\partial f}{\partial x}(a,b) + w_2 \frac{\partial f}{\partial y}(a,b).
	\]
\end{prop}

\begin{prv}
	Soit $w = (w_1, w_2) \in \R^2$. Soit $t \in \R^*$.
	\begin{align*}
		\frac{1}{t}\big(f(a + tw_1, b + tw_2) - f(a,b)\big)
		&= \frac{1}{t} \left( tw_1 \frac{\partial f}{\partial x}(a,b) + tw_2 \frac{\partial f}{\partial y}(a,b) + \po_{t \to 0}\big(\|tw\|\big) \right) \\
		&= w_1 \frac{\partial f}{\partial x}(a,b) + w_2 \frac{\partial f}{\partial y}(a,b) + \po_{t\to 0}(1) \\
		&\tendsto{t\to 0} w_1 \frac{\partial f}{\partial x}(a,b) + w_2\frac{\partial f}{\partial y}(a,b).
	\end{align*}
\end{prv}


\begin{defn}
	Avec les hypothèses précédentes, en posant \[
		\nabla f(a,b) = \left( \frac{\partial f}{\partial x}(a,b), \frac{\partial f}{\partial y}(a,b) \right) 
	\]on obtient \[
		\mathrm{d}f(w)\,(a,b) = \left<w  \mid \nabla f(a,b) \right>
	\] où $\left<\cdot|\cdot \right>$ est le produit scalaire.

	Le vecteur $\nabla f(a,b)$ est appelé \underline{gradient de $f$ en $(a,b)$}.

	Le développement limité à l'ordre 1 de $f$ devient \[
		f\big((a,b)+w\big) = f(a,b) + \left<w \mid \nabla f(a,b) \right> + \po_{w\to 0}(\|w\|)
	\]
\end{defn}

\begin{prop}
	Soit $f : U \to \R$ de classe $\mathcal{C}^1$.

	\begin{figure}[H]
    \centering
    \incfig{gradient}
	\end{figure}

	$\nabla f$ est orthogonal au lignes de niveaux de $f$, son orientation va dans le sens d'une augmentation de $f$.
\end{prop}

\begin{prv}
	Soit $\gamma : I \to U$ une courbe de niveau : \[
		\forall t \in I, f\big(\gamma(t)\big) = \text{cste}.
	\] D'après le lemme suivant : \[
		\forall t \in I, 0 = (f \circ \gamma)'(t) = \mathrm{d}f\big(\gamma'(t)\big)\big(\gamma(t)\big) = \left<\gamma'(t)  \mid \nabla f\big(\gamma(t)\big) \right>
	\] Donc $\nabla f\big(\gamma(t)\big)$ est orthogonal à $\gamma'(t)$.

	Pour tout $t \in I$, on pose $w(t) = t\, \nabla f\big(\gamma(t)\big)$. Donc \[
		f\big(\gamma(t) + w(t)\big) = f\big(\gamma(t)\big) + t \|\nabla f(\gamma(t))\|^2 + \po_{t \to 0}(t)
	\] Pour $t$ assez petit, $f\big(\gamma(t) + w(t)\big) - f\big(\gamma(t)\big)$ est du même signe que $t$.
\end{prv}

\begin{rmk}
	\begin{align*}
		V: \R^3 &\longrightarrow \R \\
		(x,y,z) &\longmapsto -mgz
	\end{align*}
	l'énerge potentielle de pesenteur

	On a donc \[
		\nabla V(x,y,z) = \left( \frac{\partial V}{\partial x}, \frac{\partial V}{\partial y}, \frac{\partial V}{\partial z} \right) = (0, 0, -mg) = \vec{P}.
	\]
\end{rmk}

\begin{lem}
	Soit $f : U \to \R$ de classe $\mathcal{C}^1$, $\gamma : \begin{array}{rcl}
		I &\longrightarrow& U \\
		t &\longmapsto& \big(x(t), y(t)\big)
	\end{array}$ où $x$ et $y$ sont dérivables.

	On pose \[
		\forall t \in I, \gamma'(t) = \big(x'(t), y'(t)\big).
	\] Alors $f \circ \gamma : I \to \R$ est dérivable et
	\begin{align*}
		\forall t \in I, (f \circ \gamma)'(t) &= \mathrm{d}f\big(\gamma'(t)\big) \big(\gamma(t)\big)\\
		&= \left<\gamma'(t)  \mid \nabla f\big(\gamma(t)\big)  \right> \\
		&= x'(t) \frac{\partial f}{\partial x}\big(x(t), y(t)\big) + y'(t) \frac{\partial f}{\partial y}\big(x(t),y(t)\big). \\
	\end{align*}
\end{lem}

\begin{prv}
	On fixe $t \in I$.

	\begin{align*}
		\forall h \neq 0, \frac{f \circ \gamma(t + h) - f \circ \gamma(t)}{h}
		&= \frac{1}{h}\big(f(\gamma(t)) + h\gamma'(t) + \po_{h\to 0}(h) - f(\gamma(t))\big) \\
		&= \frac{1}{h}\bigg(\cancel{f(\gamma(t))} + \left<h\gamma'(t) \mid \nabla f(\gamma(t)) \right> + \po_{h\to 0}(\|h\gamma'(t)\|) - \cancel{f(\gamma(t))}\bigg)\\
		&= \left<\gamma'(t) \mid \nabla f(\gamma(t)) \right> + \po_{h\to 0}(1) \\
		&\tendsto{h\to 0} \left<\gamma'(t)  \mid \nabla f(\gamma(t)) \right>
	\end{align*}
\end{prv}

\begin{defn}
	Soit $f : U \to \R$ de classe $\mathcal{C}^1$ et $(a,b) \in U$. On dit que $(a,b)$ est un \underline{point critique} de $f$ si $\nabla f(a,b) = 0$ i.e. $\frac{\partial f}{\partial x}(a,b) = \frac{\partial f}{\partial y}(a,b) = 0$.

	Dans ce cas, $f(a,b)$ est appelé \underline{valeur critique} de $f$.
\end{defn}

\begin{prop}~\\
	\begin{minipage}{\linewidth}
		\begin{wrapfigure}{r}{3cm}
			\centering
			\vspace{-1cm}
			\begin{asy}
				import solids;
				import graph;
				size(3cm);

				settings.render = 0;
				settings.prc = false;

				path3 par = graph(
					new real(real x) { return x; },
					new real(real x) { return 0; },
					new real(real x) { return -x^2; },
					0,3);
				revolution r = revolution(par, axis=Z);

				draw(r,1,longitudinalpen=nullpen);
				draw(r.silhouette());

				dot("$(a,b)$", O, red, align=N);
				real s = sqrt(2.5);
				path3 g=(s,0,-2.5)..(0,s,-2.5)..(-s,0,-2.5)..(0,-s,-2.5)..cycle;
				draw(g, deepcyan);
			\end{asy}
		\end{wrapfigure}
		Soit $f: U \to \R$ de classe $\mathcal{C}^1$ et $(a,b) \in U$ tel que \[
			\exists r > 0, \forall (x,y) \in B_{(a,b)}(r), f(x,y) \le f(a,b)
		\] Alors $\nabla f(a,b) = (0,0)$.
	\end{minipage}
\end{prop}

\begin{prv}
	Soit $g: x \mapsto f(x,b)$. $g(a)$ est un maximum local de $g$ donc $g'(a) = 0$.

	Or, $g'(a) = \frac{\partial f}{\partial x}(a,b)$

	donc $\frac{\partial f}{\partial x}(a,b) = 0$.

	Soit $h : y \mapsto f(a,y)$. On a de même $h'(b) = 0$.

	Or, $h'(b) = \frac{\partial f}{\partial y}(a,b)$.

	Donc, $\nabla f(a,b) = (0,0)$.
\end{prv}

\begin{rmk}
	Un minimum local est aussi une valeur critique.
\end{rmk}

\begin{figure}[H]
	\centering
	\begin{subfigure}{3cm}
		\centering
		\begin{asy}
			import solids;
			import graph;
			size(3cm);

			settings.render = 0;
			settings.prc = false;

			path3 par = graph(
				new real(real x) { return x; },
				new real(real x) { return 0; },
				new real(real x) { return -x^2; },
				0,3);
			revolution r = revolution(par, axis=Z);

			draw(r,1,longitudinalpen=nullpen);
			draw(r.silhouette());

			dot(O, red);
		\end{asy}
		\caption{Maximum local}
	\end{subfigure}
	\begin{subfigure}{3cm}
		\centering
		\begin{asy}
			import solids;
			import graph;
			size(3cm);

			settings.render = 0;
			settings.prc = false;

			path3 par = graph(
				new real(real x) { return x; },
				new real(real x) { return 0; },
				new real(real x) { return x^2; },
				0,3);
			revolution r = revolution(par, axis=Z);

			draw(r,1,longitudinalpen=nullpen);
			draw(r.silhouette());

			dot(O, red);
		\end{asy}
		\caption{Minimum local}
	\end{subfigure}
	\begin{subfigure}{3cm}
		\centering
		\begin{asy}
			import solids;
			import graph;
			size(3cm);

			settings.render = 0;
			settings.prc = false;
			currentprojection = obliqueZ;

			draw(graph(
				new real(real x) { return x; },
				new real(real x) { return -x^2 / 3; },
				new real(real x) { return 3; },
				-3, 3
			));

			draw(graph(
				new real(real x) { return x; },
				new real(real x) { return -x^2 / 3; },
				new real(real x) { return -3; },
				-3, 3
			));

			draw(graph(
				new real(real x) { return x; },
				new real(real x) { return -x^2 / 3 - 1; },
				new real(real x) { return 0; },
				-3, 3
			));

			draw(graph(
				new real(real x) { return 0; },
				new real(real x) { return x^2 / 9 - 1; },
				new real(real x) { return x; },
				-3, 3
			));

			draw(graph(
				new real(real x) { return -3; },
				new real(real x) { return x^2 / 9 - 4; },
				new real(real x) { return x; },
				-3, 3
			));

			draw(graph(
				new real(real x) { return 3; },
				new real(real x) { return x^2 / 9 - 4; },
				new real(real x) { return x; },
				-3, 3
			));

			dot((0,-1,0), red);
		\end{asy}
		\caption{Point de selle / Point col}
	\end{subfigure}
\end{figure}

\begin{exm}
	On revient à l'exemple donné en introduction : 
	\begin{align*}
		f: \left( \R^*_+ \right)^2 &\longrightarrow \R \\
		(x,y) &\longmapsto 2\left( xy + \frac{1}{x} + \frac{1}{y} \right).
	\end{align*}

	$\left( \R^+_* \right)^2$ est un ouvert de $\R^2$. Soit $(x,y) \in \left( \R^+_* \right)^2$.
	
	On a \[
		\begin{cases}
			\frac{\partial f}{\partial x}(x,y) = 2\left( y - \frac{1}{x^2} \right),\\
			\frac{\partial f}{\partial y}(x,y) = 2\left( x - \frac{1}{y^2} \right).
		\end{cases}
	\]

	\begin{align*}
		&\frac{\partial f}{\partial x}(x,y) = \frac{\partial f}{\partial y}(x,y) = 0\\
		\iff& \begin{cases}
			y = \frac{1}{x^2}\\
			x = \frac{1}{y^2}
		\end{cases}\\
		\iff& \begin{cases}
			y = \frac{1}{x^2}\\
			x = x^4
		\end{cases}\\
		\iff& \begin{cases}
			x = 1\\
			y = 1
		\end{cases}
	\end{align*}

	On vérivie que $f$ présente en effet un minium local en $(1,1)$. \[
		f(1,1) = 6
	\] On fixe $y \in \R^+_*$ et \[
		g : x \mapsto 2\left( xy + \frac{1}{x} + \frac{1}{y} \right).
	\] Donc \[
		\forall x \in \R^+_*, g'(x) = 2\left( y - \frac{1}{x^2} \right).
	\]
	\begin{center}
		\begin{tikzpicture}
			\tkzTabInit{$x$/1,$g'(x)$/1,$g$/2.3}{$0$, $\frac{1}{\sqrt{y}}$, $+\infty$}
			\tkzTabLine{,-,z,+,}
			\tkzTabVar{+/{}, -/$2\left( 2\sqrt{y} +\frac{1}{y} \right)$, +/{}}
		\end{tikzpicture}
	\end{center}
	
	Ainsi, \[
		\forall x \in \R^+_*, \forall y \in \R^+_*, f(x,y) \ge 2\left( 2\sqrt{y} + \frac{1}{y} \right)
	\] Soit $h : y \mapsto 2\sqrt{y} + \frac{1}{y}$. On a \[
		\forall y > 0, h'(y) = \frac{1}{\sqrt{y}} - \frac{1}{y^2} = \frac{y\sqrt{y} - 1}{y^2} = \frac{y^{\frac{3}{2}} - 1}{y^2}
	\]

	\begin{center}
		\begin{tikzpicture}
			\tkzTabInit{$y$/0.7,$h'(y)$/0.7,$h$/1.4}{$0$, $1$, $+\infty$}
			\tkzTabLine{,-,z,+,}
			\tkzTabVar{+/{}, -/$3$, +/{}}
		\end{tikzpicture}
	\end{center}

	Donc, \[
		\forall x,y > 0, f(x,y) \ge 2\times 3 = 6 = f(1,1).
	\]
\end{exm}

\begin{prop}
	[règle de la chaîne]

	Soit $f : \begin{array}{rcl}
		U &\longrightarrow& \R^2 \\
		(x,y) &\longmapsto& f(x,y)
	\end{array}$ de classe $\mathcal{C}^1$ et $U, V$ deux ouverts de $\R^2$.

	Soit $\varphi : \begin{array}{rcl}
		V &\longrightarrow& U \\
		(u,v) &\longmapsto& \varphi(u,v) = \big(x(u,v), y(u,v)\big)
	\end{array}$.

	On suppose que $x$ et $y$ sont de classe $\mathcal{C}^1$ sur $V$.

	Alors,  $f \circ \varphi : \begin{array}{rcl}
		V &\longrightarrow& \R \\
		(u,v) &\longmapsto& f\big(\varphi(u,v)\big)
	\end{array}$ est de classe $\mathcal{C}^1$ et
	\begin{align*}
		\forall (u_0, v_0) \in V, \frac{\partial (f \circ \varphi)}{\partial u}(u_0, v_0)
		&= \frac{\partial f}{\partial x}\big(\varphi(u_0, v_0)\big) \times \frac{\partial x}{\partial u}(u_0, v_0)\\
		&+ \frac{\partial f}{\partial y}\big(\varphi(u_0,v_0)\big) \frac{\partial y}{\partial u}(u_0,v_0)
	\end{align*}
	\begin{align*}
		\forall (u_0, v_0) \in V, \frac{\partial (f \circ \varphi)}{\partial v}(u_0, v_0)
		&= \frac{\partial f}{\partial x}\big(\varphi(u_0, v_0)\big) \times \frac{\partial x}{\partial v}(u_0, v_0)\\
		&+ \frac{\partial f}{\partial y}\big(\varphi(u_0,v_0)\big) \frac{\partial y}{\partial v}(u_0,v_0)
	\end{align*}
\end{prop}

\begin{exm}
	[changement de coordonnées polaires]
	On pose \begin{align*}
		\varphi: \R^+_* \times ]0,2\pi[ &\longrightarrow \R^2\setminus \left( R^+_* \times \{0\} \right) \\
		(r, \theta) &\longmapsto (r \cos \theta, r \sin\theta),
	\end{align*}
	\begin{align*}
		f: \R^2\setminus \left( R^+_* \times \{0\} \right) &\longrightarrow \R \\
		(x,y) &\longmapsto f(x,y),
	\end{align*}
	\begin{align*}
		g: \overbrace{\R^+_* \times ]0, 2\pi[}^{=V} &\longrightarrow \R \\
		(r, \theta) &\longmapsto f(r\cos\theta, r\sin\theta).
	\end{align*}

	\begin{align*}
		\forall (r_0,\theta_0) \in V,&\\[5mm]
		\frac{\partial g}{\partial r}(r_0, \theta_0) &= \frac{\partial f}{\partial x}(r_0\cos\theta_0, r_0\sin\theta_0)\cos\theta_0\\
		&+ \frac{\partial f}{\partial y}(r_0 \cos\theta_0, r_0\sin\theta_0)\sin\theta_0\\
		&= 2r_0\cos^2\theta_0 + 2r_0\sin^2(\theta_0) \\
		&= 2r_0 \\[5mm]
		\frac{\partial g}{\partial \theta}(r_0, \theta_0) &= \frac{\partial f}{\partial x}(r_0\cos\theta_0, r_0\sin\theta_0)r_0\sin\theta_0\\
		&+ \frac{\partial f}{\partial y}(r_0 \cos\theta_0, r_0\sin\theta_0)r_0\cos\theta_0\\
		&= -2{r_0}^2\cos(\theta_0)\sin(\theta_0) + 2{r_0}^2 \sin(\theta_0)\cos(\theta_0)\\
		&= 0 \\
	\end{align*}

	Donc, \[
		g(r, \theta) = r^2.
	\]
\end{exm}

\begin{exm}
	Résoudre \[
		\begin{cases}
			\frac{\partial f}{\partial x} = \frac{x}{x^2+y^2},\\
			\frac{\partial f}{\partial y} = \frac{y}{x^2+y^2}.\\
		\end{cases}
	\]

	On pose $g: (r, \theta) \mapsto f(r \cos\theta, r \sin\theta)$.

	\begin{align*}
		&\frac{\partial g}{\partial r} = \frac{1}{r}\cos^2\theta + \frac{1}{r}\sin^2\theta = \frac{1}{r},\\
		&\frac{\partial g}{\partial \theta} = -\cos(\theta) \sin(\theta) + \sin(\theta)\cos(\theta) = 0.
	\end{align*}

	Donc, \[
		\exists C \in \R, g: (r, \theta) \mapsto \ln r + C
	\] d'où,
	\begin{align*}
		\forall (x,y) \in \R^2 \setminus \{(0,0)\}, f(x,y) &= \ln\left(\sqrt{x^2 + y^2} \right)  + C\\
		&= \frac{1}{2}\ln(x^2 + y^2) + C. \\
	\end{align*}
\end{exm}

\begin{rmk}
	Soit $\mathcal{B} = (e_1, e_2)$ la base canonique de $\R^2$, $f: U \to \R$ de classe $\mathcal{C}^1$ avec $U$ un ouvert de $\R^2$.

	Soit $(x,y) \in U$.

	\begin{align*}
		\Mat_{\mathcal{B}}\big(\nabla f(x,y)\big) = \begin{pmatrix}
			\frac{\partial f}{\partial x}(x,y)\\[2mm]
			\frac{\partial f}{\partial y}(x,y)
		\end{pmatrix}
	\end{align*}

	Soit  \begin{align*}
		\varphi: V &\longrightarrow U \\
		(u,v) &\longmapsto \big(x(u,v), y(u,v)\big) 
	\end{align*} avec $x,y$ de classe $\mathcal{C}^1$. Soit $g = f \circ \varphi$.
	\begin{align*}
		\Mat_{\mathcal{B}}\big(\nabla g(u,v)\big)
		&= \begin{pmatrix}
			\frac{\partial g}{\partial u}(u,v) \\[2mm]
			\frac{\partial g}{\partial v}(u,v)
		\end{pmatrix} \\
		&= \begin{pmatrix}
			\frac{\partial x}{\partial u}(u,v) \frac{\partial f}{\partial x}(x,y)
			+ \frac{\partial y}{\partial u}(u,v)\frac{\partial f}{\partial y}(x,y)\\[3mm]
			\frac{\partial x}{\partial v}(u,v) \frac{\partial f}{\partial x}(x,y)
			+ \frac{\partial y}{\partial v}(u,v) \frac{\partial f}{\partial y}(x,y)
		\end{pmatrix}  \\
		&= \underbrace{\begin{pmatrix}
				\frac{\partial x}{\partial u}(u,v)& \frac{\partial y}{\partial u}(u,v)\\[3mm]
				\frac{\partial x}{\partial v}(u,v)& \frac{\partial y}{\partial v}(u,v)
		\end{pmatrix}}_{J(u,v)} \begin{pmatrix}
			\frac{\partial f}{\partial x}(x,y)\\[3mm]
			\frac{\partial f}{\partial y}(x,y)
		\end{pmatrix} \\
		&= J(u,v) \Mat_{\mathcal{B}}\big(\nabla f(x,y)\big) \\
	\end{align*}
	où $J(u,v) = 
	\begin{pNiceArray}{c:c}
		\Mat_{\mathcal{B}}\big(\nabla x(u,v)\big) & \Mat_{\mathcal{B}}\big(\nabla y(u,v)\big)
	\end{pNiceArray}$.

	On dit que $J(u,v)$ est \underline{la jacobienne} de $\varphi$ en $(u,v)$.
	L'application linéaire canoniquement associée à $J(u,v)$ est la \underline{différentielle de $\varphi$} en $(u,v)$ noté $\mathrm{d}\varphi(u,v)$.

	On a $\mathrm{d}\varphi(u,v) \in \mathcal{L}(R^2)$ et $\Mat_{\mathcal{B}}\big(\mathrm{d}\varphi(u,v)\big) = J(u,v)$.

	Par exemple, la jacobienne du changement de coordonnées polaires est \[
		J = \begin{pmatrix}
			\frac{\partial x}{\partial r} & \frac{\partial y}{\partial r}\\[3mm]
			\frac{\partial x}{\partial \theta} & \frac{\partial y}{\partial \theta}
		\end{pmatrix}
		= \begin{pmatrix}
			\cos\theta&\sin\theta\\
			-r\sin\theta&r\cos\theta
		\end{pmatrix}.
	\]
	$\underbrace{\det(J)}_{\text{le jacobien}} = r\cos^2\theta + r\sin^2\theta = r$

	Dans une intégrale double, si $(x,y) = \varphi(u,v)$, alors $\mathrm{d}x\mathrm{d}y = \det(J)\mathrm{d}u\mathrm{d}v$.

	Ici, \[
		\mathrm{d}x\ \mathrm{d}y = r\ \mathrm{d}r\ \mathrm{d}\theta.
	\]
\end{rmk}

\begin{prv}
	On pose $(x_0, y_0) = \varphi(u_0, v_0)$. Pour tout $(h,k) \in \R^2$ tels que $(u_0 + h, v_0 + k) \in V$, en posant $g = f  \circ \varphi$.

	\begin{align*}
		g(u_0 + h, v_0 + h) &= f\big(x(u_0 + h, v_0 + k), y(u_0 + h, v_0 + k)\big) \\
		&= f\left(
			x(u_0,v_0) + h \frac{\partial x}{\partial u}(u_0,v_0) + k \frac{\partial x}{\partial v}(u_0, v_0) + \po\big(\|(h,k)\|\big), \right.\\
		&\phantom{ = f\bigg(\bigg.}\left. y(u_0, v_0) + h \frac{\partial y}{\partial u}(u_0, v_0) + k \frac{\partial y}{\partial v}(u_0, v_0) + \po\big(\|(h,k)\|\big)
		\right)  \\
		&= f(x_0,y_0) \\
		&~+ \left( h \frac{\partial x}{\partial u}(u_0,v_0) + k \frac{\partial x}{\partial v}(u_0, v_0) + \po(\|(h,k)\|) \right) \frac{\partial f}{\partial x}(x_0,y_0)\\
		&~+ \left( h \frac{\partial y}{\partial u}(u_0, v_0) + k\frac{\partial y}{\partial v}(u_0, v_0) + \po(\|(h,k)\|) \right) \frac{\partial f}{\partial y}(x_0, y_0)\\
		&~+ \po(\|(h,k)\|)\\
		&= f(x_0, y_0) \\
		&~+ h \left( \frac{\partial x}{\partial u}(u_0, v_0) \frac{\partial f}{\partial x}(x_0, y_0) + \frac{\partial y}{\partial u}(u_0, v_0) \frac{\partial f}{\partial y}(x_0, y_0) \right)  \\
		&~+ k\left( \frac{\partial x}{\partial v}(u_0, v_0) \frac{\partial f}{\partial x}(x_0, y_0) + \frac{\partial y}{\partial v}(u_0, v_0) \frac{\partial f}{\partial y}(x_0, y_0) \right) 
		&~+ \po(\|(h,k)\|)\\
		&= g(u_0, v_0) + h \frac{\partial g}{\partial u}(u_0, v_0) + k \frac{\partial g}{\partial v}(u_0, v_0) + \po(\|(h,k)\|) \\
	\end{align*}

	Par identification,
	\[
		\frac{\partial g}{\partial u}(u_0, v_0) = \frac{\partial x}{\partial u}(u_0, v_0) \frac{\partial f}{\partial x}(x_0, y_0) + \frac{\partial y}{\partial u}(u_0, v_0) \frac{\partial f}{\partial y}(x_0,y_0)
	\] et \[
		\frac{\partial g}{\partial v}(u_0, v_0) = \frac{\partial x}{\partial v}(u_0,v_0) \frac{\partial f}{\partial x}(x_0, y_0) + \frac{\partial y}{\partial v}(u_0, v_0) \frac{\partial f}{\partial y}(x_0, y_0).
	\] 
\end{prv}

\begin{exm}
	[Régression linéaire]~\\
	\begin{figure}[H]
		\centering
		\begin{asy}
			import graph;
			axes(EndArrow);
			size(5cm);

			real f(real x) { return x + 0.5; }

			real k = 35 / (7 - 0.5);

			for(int i = 0; i < 35; ++i) {
				real mag = exp(sin(100 * pi/exp(1) * i)) * 0.8 + exp(cos(i*40)/3);
				real eps = mag * cos(10 * exp(1)/pi * i) / 3;
				dot((i/k,f(i/k) + eps));
			}

			draw(graph(f, -1, 7), orange);
		\end{asy}
	\end{figure}
	\[
		y = a x + b
	\] 
	On fixe $(a,b) \in \R^2$. \[
		\varepsilon(a,b) = \sum_{i=1}^n\big( y_i - (ax_i + b) \big)^2
	\] l'erreur totale.

	On veut minimiser $\varepsilon(a,b)$. On a 
	\[
		\forall (a,b) \in \R^2,
		\begin{cases}
			\frac{\partial \varepsilon}{\partial a}(a,b) = -2\sum_{i=1}^{n}(y_i - ax_i - b)x_i,\\
			\frac{\partial \varepsilon}{\partial b}(a,b) = -2\sum_{i=1}^{n}(y_i - ax_i - b).
		\end{cases}
	\]

	Donc,
	\begin{align*}
		(a,b) \text{ point critique de } \varepsilon \iff& \begin{cases}
			a \sum_{i=1}^n {x_i}^2 + b\sum_{i=1}^{n}x_i = \sum_{i=1}^{n} y_ix_i\\
			a\sum_{i=1}^{n}x_i + nb = \sum_{i=1}^ny_i
		\end{cases}\\
		\iff& \begin{cases}
			a \left( \frac{1}{n}\sum_{i=1}^n {x_i}^2 - \overline{x}^2\right) = \overline{y} - \overline{x} \overline{y}\\
			b = \frac{1}{n}\sum_{i=1}^ny_i - \frac{a}{n}\sum_{i=1}^nx_i = \frac{1}{n}\sum_{i=1}^n x_i y_i - \overline{x} \overline{y}
		\end{cases}\\
		&\text{ où } \overline{x} = \frac{1}{n} \sum_{i=1}^n x_i,~\overline{y} = \frac{1}{n}\sum_{i=1}^n y_i\\
		\iff& \begin{cases}
			a = \frac{\Cov(x,y)}{V(x)}\\
			b = \overline{y} - a\overline{x}
		\end{cases}
	\end{align*}

	Coefficient de corrélation: $\frac{\Cov(x,y)}{\sigma_x \sigma_y} \in [-1, 1]$
\end{exm}












		\part{Corps}

\begin{exm}[Problème]
	\begin{itemize}
		\item 
			avec $A = \Z / 9 \Z$, résoudre $\overline{x}^2 = \overline{0}$ \\
			\begin{center}
				\begin{tabular}{|c|c|c|c|c|c|c|c|c|c|c|}
					\hline
					$\overline{x}$&$\overline{0}$& $\overline{1}$ &$\overline{2}$&$\overline{3}$ &$\overline{4}$ &$\overline{5}$ &$\overline{6}$ &$\overline{7}$ &$\overline{8}$& $\overline{9}$ \\
					\hline
					$\overline{x}^2$&$\overline{0}$ &$\overline{1}$ &$\overline{4}$ &$\overline{0}$ &$\overline{7}$ &$7$ &$\overline{0}$ &$\overline{4}$ &$\overline{1}$&$\overline{0}$\\
					\hline
				\end{tabular}
			\end{center}
			On a trouvé 3 solutions: $\overline{0}$, $\overline{3}$, $\overline{6}$.
		\item $\Z / 8\Z$
			\begin{center}
				\begin{tabular}{|c|c|c|c|c|c|c|c|c|}
					\hline
					$\overline{x}$& $\overline{0}$& $\overline{1}$& $\overline{2}$& $\overline{3}$& $\overline{4}$& $\overline{5}$& $\overline{6}$& $\overline{7}$\\
					\hline
					$\overline{x^2}$& $\overline{0}$& $\overline{1}$& $\overline{4}$& $\overline{1}$& $\overline{0}$& $\overline{1}$& $\overline{4}$& $\overline{1}$\\
					\hline
				\end{tabular}
			\end{center}
			$\overline{x}^2=7$ a 4 solutions: $\overline{1}, \overline{7}, \overline{3},\text{ et } \overline{5}$
		\item $A = \mathbbm{H} = \{a + bi + cj + dk  \mid  (a,b,c,d) \in \R^4\}$ \\
			$i^2 = j^2 = k^2 = -1$ 
			\begin{align*}
				\begin{array}{c c c}
					ij = k & jk = i & ji = j\\
					ji = -k & kj = -i & ik = -j
				\end{array}
			\end{align*}
			Dans cet anneau, $-1$ a 6 racines!
	\end{itemize}
\end{exm}

\begin{defn}
	Soit $(\mathbbm{K}, +, \times)$ un ensemble muni de deux lois de composition internes. On dit que c'est un \underline{corps} si
	 \begin{enumerate}
		\item $(\mathbbm{K}, \times)$ est un groupe abélien
		\item $(\mathbbm{K}, \times)$ est un monoïde commutatif
		\item $\forall x \in \mathbbm{K}\setminus \{0_\mathbbm{K}\}, \exists y \in \mathbbm{K}, xy = 1_\mathbbm{K}$
		\item $0_\mathbbm{K} \neq  1_\mathbbm{K}$
	\end{enumerate}
	\index{corps}
\end{defn}

\begin{exm}
	\begin{itemize}
		\item $(\C, +, \times)$ est un corps
		\item $(\R, +, \times)$ est un corps
		\item $(\Q, +, \times)$ est un corps
		\item $(\Z, +, \times)$ n'est pas un corps
	\end{itemize}
\end{exm}

\begin{prop}
	$(\Z / n\Z, +, \times)$ est un corps si et seulement si $n$ est premier.
\end{prop}

\begin{prv}
	\[
		\left( \Z / n\Z \right)^\times = \left\{ \overline{k}  \mid k \wedge n = 1 \right\}
	\] 
\end{prv}


\begin{prop}
	Tout corps est un anneau intègre.
\end{prop}

\begin{prv}
	Soit $(\mathbbm{K}, +, \times)$ un corps. Soient $(a,b) \in \mathbbm{K}^2$ tel que $a \times b = 0_\mathbbm{K}$.\\
	On suppose $a \neq  0_\mathbbm{K}$. Alors, $a$ est inversible et donc \[
		b = a^{-1} \times a \times b = a^{-1} \times 0_\mathbbm{K} = 0_\mathbbm{K}
	\] 
\end{prv}

\begin{exm}
	Soit $(\mathbbm{K},+,\times)$ un corps.\\
	Résoudre \[
		\begin{cases}
			x^2 = 1_\mathbbm{K}\\
			x \in \mathbbm{K}
		\end{cases}
	\]

	\begin{align*}
		x^2 = 1_\mathbbm{K} &\iff x^2 - 1_\mathbbm{K} = 0_\mathbbm{K}\\
		&\iff (x - 1_\mathbbm{K})(x+1_\mathbbm{K}) = 0_\mathbbm{K}\\
		&\iff x - 1_\mathbbm{K} = 0_\mathbbm{K} \text{ ou } x + 1_\mathbbm{K} = 0_\mathbbm{K}\\
		&\iff x = 1_\mathbbm{K} \text{ ou } x = -1_\mathbbm{K}
	\end{align*}

	Il y a au plus 2 solutions.
\end{exm}

\begin{prop}
	Soit $(\mathbbm{K},+,\times )$ un corps et $P$ un polynôme à coefficients dans $\mathbbm{K}$ de degré $n$. Alors, l'équation $P(x) = 0_{\mathbbm{K}}$ a au plus $n$ solutions dans $\mathbbm{K}$ 
	\qed
\end{prop}

\begin{crlr}[(Théorème de Wilson)]
	voir exercice 16 du TD 12
\end{crlr}


\begin{defn}
	Soit $(\mathbbm{K}, +, \times)$ un corps et $L\subset \mathbbm{K}$.\\
	On dit que $L$ est un \underline{sous corps} de $\mathbbm{K}$ si
	\begin{enumerate}
		\item $L$ est un anneau de $(\mathbbm{K}, +, \times)$ non nul
		\item $\forall x \in L\setminus \{0_\mathbbm{K}\}, x^{-1} \in L$ 
	\end{enumerate}
	\vspace{2mm}
	en d'autres termes si
	\begin{enumerate}
		\item $\forall (x,y) \in L^2, x - y \in L$
		\item $\forall (x,y) \in L^2, x \times y^{-1} \in L$
	\end{enumerate}
	\vspace{5mm}
	On dit aussi que $\mathbbm{K}$ est une \underline{extension} de $L$.
	\index{sous corps}
	\index{extension}
\end{defn}

\begin{prop}
	Tout sous corps est un corps. \qed
\end{prop}

\begin{defn}
	Soient $(\mathbbm{K}_1,+,\times )$ et $(\mathbbm{K}_2,+, \times)$ deux corps et $f: \mathbbm{K}_1 \to \mathbbm{K}_2$.\\
	On dit que $f$ est un \underline{morphisme de corps} si $f$ est un morphisme d'anneaux.\\
	i.e. si
	\[
		\begin{cases}
			\forall (x,y) \in {\mathbbm{K}_1}^2,& f(x+y) = f(x) + f(y)\\
			\forall (x,y) \in {\mathbbm{K}_1}^2,& f(x \times y) = f(x) \times f(y)\\
		\end{cases}
	\] 
	\index{homomorphisme (de corps)}
	\index{morphisme (de corps)}
\end{defn}

\begin{prop}
	Tout morphisme de corps est injectif.
\end{prop}

\begin{prv}
	Soit $f: \mathbbm{K}_1 \to \mathbbm{K}_2$ un morphisme de corps.\\
	\begin{itemize}
		\item $\Ker(f)$ est un sous groupe de $(\mathbbm{K}_1, +)$ 
		\item Soit $x \in \Ker(f)$ et $y \in \mathbbm{K}_1$ \[
				f(x \times y) = f(x) \times f(y) = 0_{\mathbbm{K}_2} \times f(y) = 0_{\mathbbm{K}_2}
			\]
		\item Soit $x \in \Ker(f) \setminus \{0_{\mathbbm{K}_1}\}$.\\
			Alors, $x$ est inversible.\\
			\begin{align*}
				\begin{rcases*}
					x \in \Ker(f)\\
					x^{-1} \in \mathbbm{K}_1
				\end{rcases*}& \text{ donc } x \times x ^{-1} \in \Ker(f)\\
				&\text{ donc } 1_{\mathbbm{K}_1} \in \Ker(f)\\
				&\text{ donc } f(1_{\mathbbm{K}_1}) = 0_{\mathbbm{K}_2}
			\end{align*}
			Or, $f(1_{\mathbbm{K}_1}) = 1_{\mathbbm{K}_2} \neq 0_{\mathbbm{K}_2}$
	\end{itemize}
	Donc, $\Ker(f) = \{0_{\mathbbm{K}_1}\}$ donc $f$ est injective.
\end{prv}

\begin{exm}
	$\begin{array}{cc}
		\C &\longrightarrow \C\\
		z &\longmapsto \overline{z}\\
	\end{array}$ est un morphisme de corps
\end{exm}



		\part{Opérations sur les séries}

\begin{prop}
	L'ensemble $E = \{u \in \C^\N  \mid \Sigma u_n \text{ converge}\}$ est un sous-espace vectoriel de $\C^\N$ et \begin{align*}
		S: E &\longrightarrow \C \\
		u &\longmapsto \sum_{n=0}^{+\infty} u_n
	\end{align*} est une forme linéaire.
	\qed
\end{prop}

\begin{rmk}
	La somme d'une série convergente et d'une série divergente diverge.
	Le produit d'une série divergente par un scalaire non nul diverge.
\end{rmk}

		\ifsimple\else
	\pagebreak
	\begin{mdframed}
		La suite du cours provient d'Aubin. Je ne suis pas responsable pour les éventuelles bêtises qu'il a pu taper.
	\end{mdframed}
	\pagebreak


\let\cross\times
\let\gt\ge
\let\lt\le
\let\exist\exists



\part{Axiomatique de $\N$}


\begin{axm}[Axiomatique de Von Neumann]

		$(\N, \leq)$ est un ensemble totalement ordonné vérifiant\\
				Toute partie non vide de $\N$ a un plus petit élément\\
				Toute partie non vide majorée de $\N$ a un plus grand élément\\
				$\N$ n’est pas majoré\\

\end{axm}

\begin{defn}[$0$]

		$0 = \text{min}(\N)$\\

\end{defn}

\begin{defn}[$1$]

		$1 = \text{min}(\N \text{\\} \{0\})$\\

\end{defn}

\begin{defn}[$n+1$]

		Soit $n \in \N$\\
		On pose $n+1 = \text{min}(\{k \in \N|k>n\})$\\
		On dit que $n+1$ est le successeur de $n$\\

\end{defn}

\begin{prop}[$+1-1$]

		$\forall n \in \N, (n+1)-1 = n$\\
		$\forall n \in \N, (n-1)+1 = n$\\

\end{prop}

\begin{prv}

		Soient $n \in \N,\ p = n+1,\ q = p-1$ \\

		$n < p$ et $q < p$\\
		Donc $n \leq q$ car $q = \text{max}(\{k \in \N|k<p\})$\\
		Si $q > n$, alors $q \ge p$ car $p = \text{min}(\{k \in \N|k >n\})$\\
		Donc $q = n$\\

\end{prv}

\begin{prop}[Ensemble Ouvert Vide]

		$\forall n \in \N, ]n, n+1[ = \varnothing$\\

\end{prop}

\begin{prv}

		Soit $n \in \N$, on sait que $n+1>n$\\
		Soit $p>n$, on suppose $n < p < n+1$\\
		Comme $p >n, p \ge n+1$		Contradiction\\

\end{prv}

\begin{prop}[Théorème de Récurrence]

		Soit $P$ un prédicat sur $\N$ et $n \in \N$\\
		Si $\left\{\begin{array}{l c r}P(n_0) \text{ est vrai} \\ \forall n \ge n_0n P(n) \Longrightarrow P(n+1) \end{array}\right.$\\
		Alors $ \forall n \ge n_0, P(n)$ est vrai\\

\end{prop}

\begin{prv}

		Soit $A = \{n \in \N|n \ge n_0 \}$ et $P(n)$ faux\\
		Supposons $A \neq \varnothing$\\
		$A$ a donc un plus petit élement, soit $N = \text{min}(A)$\\

		Cas 1 : $N = 0$\\
				Alors, comme $N \in A$, on a $n_0 \leq 0$ et $P(0)$ est faux\\
				Alors $n_0 = 0$		Contradiction avec “$P(n)$ est vrai”\\

		Cas 2 : $N \neq 0$\\
				Alors $N-1 \in \N$\\
				$N-1 \notin A$ car $N-1 < N$\\
				Donc $N-1 < n_0$ ou $P(n)$ vrai\\

				Supposons $N-1 < n_0$\\
				$N \in A$ donc $N \ge n_0$\\
				$N-1 < n_0 < N$\\
				Donc $N = n_0$\\
				Or, $N \in A$ donc $P(n)$ est faux alors que $P(n)$ est vrai\\

				Supposons $\left\{\begin{array}{l c r}P(n-1) \text{ vrai} \\ N-1 \ge n_0\end{array}\right.$		Comme $N-1 \ge n_0, P(N-1) \Longrightarrow P(N)$\\

				Donc $P(N)$ est vrai\\
				Or, $N \in A$ donc $P(N)$ est faux\\

				Donc $A = \varnothing$\\



\end{prv}


\part{Récurrences}


\begin{prop}[Récurrence Double]

		Soient $P$ un prédicat sur $\N$ et $n_0 \in N$\\
		Si $\left\{\begin{array}{l c r} P(n_0)\text{ est vrai} \\ P(n_0 + 1) \text{ est vrai} \\ \forall n > n_0, P(n)\text{ et } P(n+1) \Longrightarrow P(n+2) \end{array}\right.$\\
		Alors $\forall n \ge n_0, P(n)$ est vrai\\

\end{prop}

\begin{prv}

		On pose $\forall n \in \N, Q(n) :$ “$P(n)\text{ et } P(n+1) \text{ vrais}$ ”\\
		$Q(n_0)$ est vrai\\

		Soit $n \ge n_0$, on suppose $Q(n)$ vrai\\
		On sait alors que $P(n+2)$ est vrai\\
		On sait par hypothèse de récurrence que $P(n+1)$ est vrai\\
		Donc $Q(n+1)$ est vrai\\

\end{prv}

\begin{prop}[Récurrence Multiple]

		Soient $P$ un prédicat sur $\N$ et $(p, n_0) \in \N^2$\\
		Si $\left\{\begin{array}{l c r} \forall k \in [\![0,p]\!], P(n_0-k)\text{ est vrai} \\ \forall n \ge n_0, (P(n) \text{ et ... } P(n+p-1)) \Longrightarrow P(n+p) \end{array}\right.$\\
		Alors $\forall n \ge n_0, P(n)$ est vrai\\

\end{prop}

\begin{prop}[Récurrence Forte]

		Soient $P$ un prédicat sur $\N$ et $n_0 \in \N$\\
		Si $\left\{\begin{array}{l c r} P(n_0)\text{ est vrai} \\ \forall n \ge n_0, (P(n_0) \text{ et ... } P(n-1)) \Longrightarrow P(n) \end{array}\right.$\\
		Alors $\forall n \ge n_0, P(n)$ est vrai\\

\end{prop}

\begin{prv}

		On pose $\forall n \in N, Q(n) :$ “$\forall k \in [\![n_0,n]\!], P(k) \text{ vrai}$”\\
		$Q(n_0)$ est vrai car $P(n_0)$ est vrai\\

		Soit $n \ge n_0$, on suppose $Q(n)$ vrai\\
		On sait que $\forall k \in [\![n_0,n]\!], P(k)$ est vrai\\
		Alors $P(n+1)$ est vrai\\
		Donc $\forall k \in [\![n_0,n+1]\!], P(k)$ est vrai\\
		Donc $Q(n+1)$ est vrai\\



\end{prv}


\part{Divisibilité}


\begin{defn}[Divisibilité]

		Soient $(a,b) \in \Z^2$\\
		On dit que $a$ divise $b$ si il existe $k \in \Z$ tel que $b = ka$\\
		On écrit $a|b$ et on dit que $\left\{\begin{array}{l c r}a \text{ est un diviseur de }b \\b \text{ est un multiple de }a\end{array}\right.$\\

\end{defn}

\begin{prop}[Caractéristiques de la Divisibilité]

		$|$ est une relation d’ordre sur $\Z$\\
		Ce n’est pas une relation totale\\

\end{prop}

\begin{prop}[Ordonnancement et Divisibilité]

		Soient $(a,b) \in \Z\cross \Z^*$\\
		Si $a|b, |a| \leq |b|  $\\

\end{prop}

\begin{prop}[Divisibilité et Combinaison Linéaire]

		Soient $(a,b,c) \in (\Z^*)^3$\\
		$\left\{\begin{array}{l c r}a|b\\a|c\end{array}\right. \Longrightarrow \forall (k,l) \in \Z^2, a|(bk+cl)$\\

\end{prop}

\begin{prv}

		$\left\{\begin{array}{l c r}b=au \text{ avec }u \in \Z \\ c = av \text{ avec } v \in \Z \end{array}\right.$\\
		Soient $(k,l) \in \Z^2$\\
		$bk + cl = auk+avl = a(uk+vl)$\\
		Donc $a|(bk+cl)$\\

\end{prv}

\begin{defn}[Nombres Associés]

		Soient $(a,b) \in \Z^2$\\

		$a$ et $b$ sont associés si $a=b$ ou $a=-b$\\

\end{defn}

\begin{prop}[Nombres Associés et Divisibilité]

		Soient $(a,b) \in \Z^2$\\
		$a|b \iff -a|b \iff a |-b \iff-a|-b$\\



\end{prop}


\part{Division Euclidienne}


\begin{prop}[Division Euclidienne dans $\N$]

		Soient $a \in \N$ et $b \in \N^*$\\
		$\exist!(q,r)\in \N^2, \left\{\begin{array}{l c r}a = bq+r \\r \in [0,b[\end{array}\right.$\\

\end{prop}

\begin{prv}

		Existence : On considère| $A = \{q \in \N|qb\leq a\}$ $A$ est non vide car $0 \in A$\\
				$A$ est majoré : $\forall q \in A, q \leq a$ car $a \ge qb \ge q$\\

				Soit $q = \text{max}(A)$, on pose $r = a-bq$\\
				Comme $a,b$ et $q \in \N, r \in \Z$\\
				$q \in A$ donc $qb \leq a$ donc $r \ge 0$\\
				$q+1 > \text{max}(A)$ donc $q+1 \notin A$ donc $(q+1)b > a$\\
				Donc $r < b$\\

		Unicité : Soit $(q',r') \in \N^2$ tel que $\left\{\begin{array}{l c r}a+bq'+r'\\0\leq r'<b\end{array}\right.$\\
				On sait aussi que $a = bq+r$\\
				Donc $0 = b(q'-q) + r'-r$\\
						$-r'+r=b(q'-q)$\\

				De plus, $\left\{\begin{array}{l c r}0\leq r < b \\ -b < -r \leq 0\end{array}\right.$\\
				Donc $-b < r'-r<b$\\

				Le seul multiple de $b$ dans $]\!]-b,b[\![$ est $0$\\
				Donc $r'-r=0$, donc $r=r'$\\
				et $b(q'-q)=0$\\
				Or, $b \neq 0$ donc $q'-q=0$ donc $q=q'$\\

\end{prv}

\begin{prop}[Division Euclidienne dans $\Z$]

		Soient $a \in \Z$ et $b \in \Z^*$\\
		$\exist! (q,r) \in \Z^2, \left\{\begin{array}{l c r}a=bq+r \\ 0 \leq r < |b| \end{array}\right.$\\

\end{prop}

\begin{prv}

		Existence :\\
				Cas 1 : $a \in \N, b \in \N^*$\\
						D’après ce qui précède, $\exist!(q,r) \in \N^2, \left\{\begin{array}{l c r}a=bq+r \\ 0 \leq r <b\end{array}\right.$\\
						Comme $b \in \N^*$, on a bien $0 \leq r < |b| $\\
						$q \in \N \subset \Z$\\

				Cas 2 : $a \in \Z, b \in \N^*$\\
						Comme $-a \in \N, \exist! (q',r') \in \N^2, \left\{\begin{array}{l c r}-a=bq'+r' \\ 0 \leq r' < b\end{array}\right.$\\

						Donc $a = b(-q') - r'$\\
								$=b(-q'-1) - r'+b$\\

						On pose $q = \left\{\begin{array}{l c r}-q'-1 \text{ si } r \neq 0 \\-q' \text{ si }r = 0 \end{array}\right.$	et	$r = \left\{\begin{array}{l c r}-r'+b \text{ si }r' \neq 0 \\ r' \text{ si } r'= 0\end{array}\right.$\\
						On a bien $\left\{\begin{array}{l c r}a = bq+r \\q \in \Z \\ 0 \leq r < |b| \end{array}\right.$\\

				Cas 3 : $a \in \N, b \in \Z^*_-$\\
						$\exist! (q',r') \in \N^2, \left\{\begin{array}{l c r}a=(-b)q'+r' \\0 \leq r' < -b\end{array}\right.$\\

						On pose $\left\{\begin{array}{l c r}q=-q'\\r=r'\end{array}\right.$\\
						Et on a bien $\left\{\begin{array}{l c r}a = bq+r \\0 \leq r < |b| \end{array}\right.$\\

				Cas 4 : $a \in \Z^-, b \in \Z^*_-$\\
						$\exist! (q',r') \in \N^2, \left\{\begin{array}{l c r}-a=-bq'+r' \\ 0 \leq r' < -b\end{array}\right.$\\

						Donc $a = bq' - r'$\\
								$=b(q'+1) - r'-b$\\

						On pose $q = \left\{\begin{array}{l c r}q'+1 \text{ si } r \neq 0 \\q' \text{ si }r = 0 \end{array}\right.$	et	$r = \left\{\begin{array}{l c r}-r-b' \text{ si }r' \neq 0 \\ r' \text{ si } r'= 0\end{array}\right.$\\
						On a bien $\left\{\begin{array}{l c r}a = bq+r \\q \in \Z \\ 0 \leq r < |b| \end{array}\right.$\\

		Unicité :\\
				Soit $(q',r') \in \Z^2$ tel que $\left\{\begin{array}{l c r}a=bq'+r' \\ 0 \leq r' < |b| \end{array}\right.$ et $\left\{\begin{array}{l c r}a = bq+r \\0 \leq r < |b|  \end{array}\right.$\\

				D’où $\left\{\begin{array}{l c r}b(q'-q) = r'-r \\ -|b| < r -r' < |b|   \end{array}\right.$\\
				Donc $r-r'=0$\\

				Donc $r'=r$ et $q'=q$\\

\end{prv}

\begin{defn}[Quotient et Reste]

		Soient $a \in \Z$ et $b \in \Z^*$\\
		D’après le théorème précédent, $\exist! (q,r) \in \Z\cross \N, \left\{\begin{array}{l c r}a=bq+r \\ 0 \leq r < |b| \end{array}\right.$\\
		On dit que $q$ est le quotient et $r$ le reste dans la division (euclidienne) de $a$ par $b$\\

\end{defn}

\begin{prop}[Reste et Divisibilité]

		Soient $a \in \Z, b \in \Z^*$\\
		On note $r$ le reste de la division de $a$ par $b$\\
		$r = 0 \iff b|a$\\

\end{prop}

\begin{prv}

		On pose $a = bq+r, q \in \Z$\\

		“$\Longrightarrow$” : Si $r = 0$, alors $\left\{\begin{array}{l c r}a = bq\\q \in \Z\end{array}\right.$	donc $b|a$\\

		“$\Longleftarrow$” : Si $b|a$, $\exist k \in \Z, a =bk$\\
				Donc $\left\{\begin{array}{l c r}a = bk+0\\0 \leq 0 < |b| \end{array}\right.$\\
				Par unicité du reste, $r=0$\\



\end{prv}


\part{Arithmétique Modulaire}


\begin{defn}[Congruences]

		Soient $(a,b) \in \Z^2, c \in \N^*$\\
		On dit que $a$ et $b$ sont congrus modulo $c$ si $a$ et $b$ ont le même reste dans ma division par $c$\\
		On note $a = b[c]$\\

\end{defn}

\begin{prop}[Congruence et Relation D’Equivalence]

		La relation de congruence modulo $c$ est une relation d’équivalence\\

\end{prop}

\begin{rmk}[Classes d’Equivalence Modulo $c$]

		On note $\Z/c \Z$ l’ensemble des classes d’équivalence modulo $c$\\
		$\Z/5 \Z = \{\bar0, \bar1, \bar2, \bar3, \bar4\}$\\

\end{rmk}

\begin{prop}[Modulo et Divisibilité]

		Soient $(a,b) \in \Z^2$ et $c \in \N^*$\\
		$a \equiv b [c] \iff c | b-a$\\

\end{prop}

\begin{prv}

		“$\Longrightarrow$” : On pose $\left\{\begin{array}{l c r}a = cq+r, q \in \Z, 0\ \leq r < c \\ b = cq' + r, q' \in \Z\end{array}\right.$\\
				Donc $b-a = c(q'-q)$\\
				Donc $c|b-a$\\

		“$\Longleftarrow$” : On pose $\left\{\begin{array}{l c r}a = cq+r, q \in \Z, 0\ \leq r < c \\ b = cq' + r', q' \in \Z, 0\ \leq r' < c\end{array}\right.$\\
				$b-a = c(q'-q) + r'-r$\\
				$-c \leq r'-r \leq c$\\

				Si $r'-r \ge 0, r'-r$ est le reste de la division de $a-b$ par $c$\\
				Donc $r'=r$ donc $a \equiv b [c]$\\

				Si $r'-r < 0, r-r'$ est le reste de la division de $a-b$ par $c$\\
				Donc $r'=r$ donc $a \equiv b[c]$\\

\end{prv}

\begin{prop}[Addition et Multiplication de Congruences]

		Soient $(a,b,x,y) \in \Z^4$ et $c \in \N^*$\\
		On suppose $\left\{\begin{array}{l c r}a \equiv b[c]\\x \equiv y [c]\end{array}\right.$\\

		Alors $\left\{\begin{array}{l c r}a+x \equiv b+ y [c] \\ ax \equiv by [c] \end{array}\right.$\\

\end{prop}

\begin{prv}

		$c|b-a$ et $c|y-x$\\
		Donc $c|(b-a+y-x)$\\
		Donc $c|(b+y-(a+x))$\\
		Donc $a+x \equiv b+y [c]$\\

		On pose $\left\{\begin{array}{l c r}a = ck+b, k \in \Z \\ x = cl+y, l \in \Z\end{array}\right.$\\
		$ax = (ck+b)(cl+y)$\\
				$=by+cky+clk+c^2kl$\\
				$=by+c(ky+bl+clk)$\\
		Donc $ax \equiv by [c]$\\

\end{prv}

\begin{prop}[Critères de Divisibilité en Base 10]

		Soit $N \in \N$, on notera ses chiffres $a_0...a_n$\\
		$N = \overset{n}{\underset{k=0}{\sum}}10^ka_k$\\

		Divisibilité par $2$ :\\
				$N$ pair $\iff N \equiv 0[2]$\\
						$\iff \overset{n}{\underset{k=0}{\sum}} 10^ka_k \equiv 0 [2]$\\
						$\iff a_0 \equiv 0 [2]$ car	$\forall k \ge 1, 10^k \equiv 0 [2] \\10^0 = 1 \equiv 1 [2]\quad\ $\\

		Divisibilité par $3$ :\\
				$\forall k \in \N, 10^k \equiv 1^k \equiv1 [3]$ car $10 \equiv 1 [3]$\\
				$3|N \iff N \equiv 0 [3]$\\
						$\iff \overset{n}{\underset{k=0}{\sum}}10^ka_k \equiv 0 [3]$\\

		Divisibilité par $9$ : \\
				$\forall k \in N, 10^k \equiv 1 [9]$\\
				$9|N \iff N \equiv 0 [9]$\\
						$\iff \overset{n}{\underset{k=0}{\sum}}10^ka_k \equiv 0 [9]$\\

		Divisibilité par $5$ :\\
				$\left\{\begin{array}{l c r}10^0 \equiv 1 [5] \\ \forall k \in \N^*, 10^k \equiv 0 [5] \end{array}\right.$\\
				$5|N \iff a_0 \equiv 0 [5]$\\
						$\iff a_0 \in \{0,5\}$\\

		Divisibilité par $11$ :\\
				$10^0 \equiv -1 [11]$\\
				Donc $\forall k \in \N, 10^k \equiv (-1)^k[11]$\\
				$N \equiv 0 [11] \iff \overset{n}{\underset{k=0}{\sum}}(-1)^ka_k \equiv 0 [11]$\\
						$\iff a_0-a_1+a_2...+(-1)^na_n \equiv 0 [11]$\\

\end{prop}

\begin{rmk}[Réécriture en Classes d’Equivalence]

		On peut réecrire le calcul précédent dans $\Z/11 \Z$ \\
		$\overline N = \overline{\overset{n}{\underset{k=0}{\sum}}10^ka_k} = \overset{n}{\underset{k=0}{\sum}}\overline{10^k}\ \overline{a_k} = \overset{n}{\underset{k=0}{\sum}}\overline{(-1)^k}\ \overline{a_k}$\\

\end{rmk}

\begin{rmk}[Opération dans $\Z / n \Z$]

		Dans $\Z / n \Z$, on dispose $\left\{\begin{array}{l c r}\text{d’une addition} : \quad\quad\ \ \overline{a} + \overline{ b} = \overline{a+b} \\ \text{d'une multiplication} : \overline{a} * \overline{b} = \overline{a*b}\end{array}\right.$\\

		L’addition est commutative, associative, n’élément neutre $\overline{0}$ et l’opposé de $\overline{a}$ est $\overline{-a}$\\
		La multiplication est commutative, associative, d’élément neutre $\overline{1}$ et distributive par rapport à $+$\\



\end{rmk}


\part{PCGD et PPCM}


\begin{defn}[PGCD]

		Soient $(a,b) \in \Z^2$\\
		Le PGCD de $a$ et $b$ est le plus grand diviseur commun à $a$ et $b$\\
		Il existe car $\mathcal{D} = \{d \in \N|d|a \text{ et } d|b\}$ est non vide car $a \in \mathcal{D}$\\
		$\mathcal{D}$ est majoré par $|a| $\\
		On le note $\text{PGCD}(a,b)$ ou $a\wedge b$\\

\end{defn}

\begin{prop}[Théorème d’Euclide]

		Soient $a \in \Z, b \in \N^*$\\
		Soit $r$ le reste de la division de $a$ par $b$\\
		$a \wedge b = b \wedge r$\\

\end{prop}

\begin{prv}

		On pose $\left\{\begin{array}{l c r}d = a \wedge b\\ \varsigma = b \wedge r \\ a = bq+r\end{array}\right.$\\

		$\left\{\begin{array}{l c r}d|a\\d|b\end{array}\right. \Longrightarrow \left\{\begin{array}{l c r}d|a-bq\\d|b\end{array}\right. \Longrightarrow \left\{\begin{array}{l c r}d|r\\d|b\end{array}\right. \Longrightarrow  d \leq \varsigma$\\

		$\left\{\begin{array}{l c r}\varsigma|b\\\varsigma|r\end{array}\right. \Longrightarrow \left\{\begin{array}{l c r}\varsigma|bq+r\\\varsigma|b\end{array}\right. \Longrightarrow \left\{\begin{array}{l c r}\varsigma|a\\\varsigma|b\end{array}\right. \Longrightarrow \varsigma \leq d$\\

		Donc $d = \varsigma$\\

\end{prv}

\begin{prop}[PGCD et Diviseurs]

		Soient $(a,b) \in \Z^2$ et $d = a \wedge b$\\
		$\mathcal{D} = \{k \in \Z|\ k|a, k|b \}$\\
		$\forall k \in \Z, k \in \mathcal{D} \iff k|d$\\

\end{prop}

\begin{prv}

		“$\Longleftarrow$” : Soit $k \in \Z$, on suppose $k|d$\\
				$d|a$ donc $k|a$\\
				$d|b$ donc $k|b$\\

				Donc $k \in \mathcal{D}$\\

		“$\Longrightarrow$” : Soit $k \in \mathcal{D}$\\
				On pose $r_0$ le reste de la division de $a$ par $b$,\\
				$r_1$ le reste de la division de $b$ par $r_0$\\
				et $\forall n \in \N, r_{n+1}$ le reste de la division de $r_{n-1}$ par $r_n$ si $r_n \neq 0$\\

				La suite $(r_n)$ est décroissante, minorée par $0$ et à valeurs entières\\
				Soit $N \in \N$ tel que $r_N = 0$			(si $N=0$, on pose $r_{-1} = b$)\\

				D’après la poposition précédente,\\
				$d =a \wedge b = b \wedge r_0 = r_0 \wedge r_1...r_{N-1} \wedge r_N$\\
						$= r_{N-1} \wedge 0 = r_N$\\

				On pose aussi $\forall n \in [\![1, N-1]\!], r_{n-1} = r_nq_n + r_{n+1}$\\
				On en déduit que $\exists(\alpha_n, \beta_n) \in \Z^2, r_{N-1} = a \alpha_N + b \beta_N$\\
				$\left\{\begin{array}{l c r}k|a\\k|b\end{array}\right. \Longrightarrow k|a \alpha_N + b \beta_N \Longrightarrow k|r_{N-1} \Longrightarrow k|d$\\

\end{prv}

\begin{defn}[PPCM]

		Soit $(a,b) \in \Z^2$, on pose $M = \{k \in \N|\ a|k, b|k\}$\\
		$M \neq \varnothing$ car $ab \in M$\\
		$M \neq \varnothing$ donc admet un plus petit élément noté $\text{PPCM}(a,b)$ ou $a \vee b$\\

\end{defn}

\begin{prop}[Produit PGCD PPCM]

		$\forall (a,b) \in \Z^2, (a \wedge b)(a \vee b) = ab$\\

\end{prop}

\begin{prv}

		Voir paragraphe Facteurs Premiers\\

\end{prv}

\begin{prop}[Propriétés de $\wedge$ et $\vee$]

		$\wedge$ est commutative, associative sur $\Z^*$\\
		$\vee$ est commutative, associative sur $\Z^*$\\

\end{prop}

\begin{prv}

		Soient $(a,b,c) \in (\Z^*)^3, d = (a \wedge b) \wedge c, \varsigma = a \wedge (b \wedge c)$ \\

		$\left\{\begin{array}{l c r}d|c\\d|a \wedge b\end{array}\right. \Longrightarrow \left\{\begin{array}{l c r}d|c\\d|a\\d|b\end{array}\right.$\\

		$\left\{\begin{array}{l c r}\varsigma|a\\\varsigma|b \wedge c\end{array}\right. \Longrightarrow \left\{\begin{array}{l c r}\varsigma|a\\\varsigma|b\\\varsigma|c\end{array}\right.$\\

		On pose $\varepsilon = \text{PGCD}(a,b,c)$\\
		On a $\left\{\begin{array}{l c r}d \leq \varepsilon \\ \varsigma \leq \varepsilon\end{array}\right.$\\

		$\left\{\begin{array}{l c r}\varepsilon|a\\\varepsilon|b\\\varepsilon|c\end{array}\right. \Longrightarrow \left\{\begin{array}{l c r}\varepsilon|a\\\\\varepsilon|b \wedge c\end{array}\right. \Longrightarrow \varepsilon|a \wedge (b \wedge c) \Longrightarrow \varepsilon \leq d$\\

		De même, on $\varepsilon\leq \varsigma$\\
		Donc $d = \varepsilon = \varsigma$\\

\end{prv}

\begin{prop}[Théorème de Bézout]

		Soient $(a,b) \in \Z \cross \Z^*, d = a \wedge b$\\
		$\exists(u,v) \in \Z^2, d = au+bv$\\

\end{prop}

\begin{prv}

		On pose $A = \{au+bv|(u,v) \in \Z^2\}$\\
		On veut montrer que $d \in A$\\
				$a = a*1+b*0$ donc $a \in A$\\
				$b = a*0+b*1$ donc $b \in A$\\
				$0 = a*0+b*0$ donc $0 \in A$\\

		Soit $(x,y) \in A^2$\\
				$x = au_1+bv_1, (u_1,v_1) \in \Z^2$\\
				$y = au_2 + bv_2, (u_2,v_2) \in \Z^2$\\
				$x+y = a(u_1+u_2) + b(v_1+v_2) \in A$\\

		Soit $x \in A, k \in \Z$\\
				$x = au+bv, (u,v) \in \Z^2$\\
				$kx = aku+bkv \in A$\\

		Soit $n = \text{min}(A \cap \N^*)$		$(|b| \in A \cap \N^*)$\\
		Soit $x \in A$\\

		Par division euclidienne de $x$ par $n$ :\\
				$\left\{\begin{array}{l c r}x = nq+r \\ q \in A, 0 \leq r<n \end{array}\right.$\\

		$\left\{\begin{array}{l c r}x \in A\\n \in A\end{array}\right. \Longrightarrow \left\{\begin{array}{l c r}x \in A\\-qn \in A\end{array}\right. \Longrightarrow x-qn \in A \Longrightarrow r \in A$\\

		$\left\{\begin{array}{l c r}r < n \\r \in A\end{array}\right. \Longrightarrow r \leq 0$\\

		Donc $r = 0$\\
		Donc $n|x$\\

		D’où $A = n \Z$\\
		Or, $\left\{\begin{array}{l c r}a \in A\\b \in A\end{array}\right. \Longrightarrow \left\{\begin{array}{l c r}n|a\\n|b\end{array}\right.$\\

		Cas particulier : $a \wedge b = d = 1$, alors $1$ est le seul diviseur positif de $a$ et $b$\\
		Donc $n=1$ donc $A = \Z$ donc $1 \in \Z$\\

		Cas général : On pose $a' = \frac{a}{d} \in \Z,\ b' = \frac{b}{d} \in \Z,\ a' \wedge b' = 1$\\
		D’après le cas particulier, $\exist (u,v) \in \Z^2, a'u+b'v = 1$\\

		D’où $au+bv=d$\\

\end{prv}

\begin{prop}[Réciproque du Théorème de Bézout]

		Soient $(a,b) \in \Z\cross \Z^*$\\
		On suppose qu’il existe $(u,v) \in \Z^2$ tel que $au+bv=1$\\
		Alors $a \wedge b = 1$\\

\end{prop}

\begin{prv}

		On pose $d = a \wedge b$\\
		$\left\{\begin{array}{l c r}d|a\\d|b\end{array}\right. \Longrightarrow d|au+bv \Longrightarrow d|1 \Longrightarrow d=1$\\

\end{prv}

\begin{prop}[Théorème de \Gauss]

		Soient $(a,b,c) \in \Z^3$ tels que $\left\{\begin{array}{l c r}a \wedge b = 1\\a|bc\end{array}\right.$\\
		Alors $a|c$\\

\end{prop}

\begin{prv}

		D’après le théorème de Bézout,\\
		$au+bv = 1$ avec $(u,v) \in \Z^2$\\
		D’où $acu+bcv = c$\\

		$\left\{\begin{array}{l c r}a|acu\\a|bcv\end{array}\right. $ \\
		Donc $a|acu+bcv $\\
		Donc $a|c$\\

\end{prv}

\begin{rmk}[Inversion Modulo $n$]

		Soit $x \in \Z$\\
		$\exist?y \in \Z, xy \equiv 1 [n]$\\
		$(\iff$dans $\Z/n \Z, \overline{x} + \overline{y} = \overline{1} ?)$\\

		Avec $n=4 :$\\
				$\begin{matrix}x\ 0123\\0\ 0000\\1\ 0123 \\2 \ 0202 \\3\ 0321\end{matrix}$		$\left\{\begin{array}{l c r}1 \text{ et }3 \text{ sont inversibles modulo } 4 \\ 2 \text{ n'est pas inversible modulo } 4\end{array}\right.$ \\

		$3x \equiv 2[4]$\\
		$\iff 3 * 3x \equiv 3*2[4]$\\
		$\iff x \equiv 2[4]$\\

		$2x \equiv 1[4]$\\
		$\Longrightarrow 2*2x \equiv 2[4]$\\
		$\Longrightarrow 0 \equiv 2 [4]$\\

\end{rmk}

\begin{prop}[Congruences et Nombres Premiers]

		Soit $p$ un nombre premier\\
		Alors $\forall x \in \Z, x \not \equiv 0[p] \Longrightarrow \exists y \in \Z, xy \equiv 1 [p]$\\

\end{prop}

\begin{prv}

		Soit $x \in \Z$ tel que $x \not \equiv 0[p]$\\
		Soit $y \in \Z$\\
		$xy \equiv 1[p] \iff \exists u \in \Z, xy = 1+pu$\\
				$\iff \exists u \in \Z, xy -pu = 1$\\

		$y$ existe $\iff x \wedge p = 1$\\
				$\iff p \nmid x$\\

\end{prv}

\begin{prop}[Inversibilité Modulo $n$]

		Soit $n \in \N^*, x \in \Z$\\
		$x$ inversible modulo $n$ $\iff x \wedge n = 1$\\

\end{prop}

\begin{prv}

		Voir précédent\\

\end{prv}

\begin{prop}[Petit Théorème de Fermat]

		Soit $p$ premier, $a \in \Z$\\
		$a^p \equiv a[p]$\\

\end{prop}

\begin{prv}

		Cas 1 : $a \equiv 0[p]$\\
				$a^p \equiv 0^p[p]$\\
				$a^p \equiv 0[p] \equiv a [p]$\\

		Cas 2 : $a \not \equiv 0 [p]$\\
				Alors $a \wedge p = 1$\\

				On pose $\forall i \in \N^*, r_i$ le reste de la division de $ia$ par $p$\\
				Soit $i \in [\![1,p-1]\!]$\\
						$r_i = 0 \Longrightarrow p|ia \Longrightarrow p|i$		Contradiction\\
						$\forall i \in \N^*, r_i \neq 0$\\

				Soit $(i,j) \in [\![1,p-1]\!]^2, i \neq j$\\
						On suppose $r_i = r_j$, alors $ia \equiv ja [p]$\\
						Or, $a \wedge p = 1$ donc $a$ est inversible modulo $p$\\
						Donc $a \equiv j[p]$ donc $i = j$		Contradiction\\

				Ainsi, $r_1...r_{p-1} \in [\![1,p-1]\!]$ distincts donc ils prennent toutes les valeurs de $[\![1,p-1]\!]$\\
				$i \longmapsto r_i$ est injective\\
				$\{r_1...r_{p-1}\} = [\![1,p-1]\!]$\\

				Donc $\overset{p-1}{\underset{k=1}{\prod}} r_i = (p-1)!$\\
				Donc $\overset{p-1}{\underset{k=1}{\prod}} ia \equiv (p-1)![p]$\\
				Donc $(p-1)! a^{p-1} \equiv (p-1)![p]$\\

				D’où $(p-1)! \equiv 0[p] \iff p |1*2*3\dots*(p-1)$\\
						$\iff\exists i \in [\![1,p-1]\!], p|i$\\
				Donc $(p-1)! \not \equiv 0[p]$\\
				Donc $(p-1)!$ est inversible modulo $p$\\
				Donc $a^p \equiv 1[p]$\\
				Donc $a^p \equiv a[p]$\\



\end{prv}


\part{Décomposition en Facteurs Premiers}


\begin{defn}[Nombre Premier]

		Soit $n \in \N$\\
		On dit que $n$ est premier si\\
				$n \ge 2$\\
				les seuls diviseurs entiers de $n$ sont $1$ et $n$\\

\end{defn}

\begin{prop}[Infinité de Nombres Premiers]

		Il y a une infinité de nombres premiers\\

\end{prop}

\begin{prv}

		On suppose qu’il n’y a qu’un nombre fini de nombres premiers\\
				$p_1<...<p_n$\\

		On pose $N = p_n*...*p_1+1$\\
		$N>p_n$ donc $N$ n’est pas premier\\
		$N$ a d’autres diviseurs positifs que $1$ et $N$\\
		$N$ est divisible par un nombre entre $2$ et $N-1$\\

		Soit $p = \text{min}(\{k \in [\![2,N-1]\!] | k |N\})$\\
		$p$ est premier		(Tout diviseur de $p$ divise aussi $N$)\\
		$\exists i \in [\![1,n]\!], p_i=p$\\
		$p_i|N$\\
		$p_i|N-p_1...p_n$\\
		$p_i|1$		Contradiction\\

		Donc il y a une infinité de nombres premiers\\

\end{prv}

\begin{prop}[Théorème Fondamental de l’Arithmétique]

		“Tout entier se décompose en un unique produit de nombres premiers”\\

		Soient $n \in \N$ tel que $n \ge 2$ et $\mathcal{P}$ l’ensemble des nombres premiers\\
		$\exist! \nu : \mathcal{P} \longrightarrow \N$ telle que $\left\{\begin{array}{l c r}\{p \in \mathcal{P}|\nu(p) \neq 0\} \text{ est fini} \\ n = {\underset{p \in \mathcal{P}}{\prod}}p^{\nu(p)} \end{array}\right.$\\

\end{prop}

\begin{prv}

		Existence : Déjà vue : Récurrence Forte (Chapitre 9)\\

		Unicité : Soit $n \ge 2$ et $\nu : \mathcal{P} \longrightarrow \N$ telle que\\
		${(*)\underset{p \in \mathcal{ P}}{\prod}}p^{\mu(p)} = {\underset{p \in \mathcal{ P}}{\prod}}p^{\nu(p)}$	avec $\left\{\begin{array}{l c r}\mu\neq \nu \\ M = \{p \in \mathcal{P}|\mu(p)\neq 0\} \text{ fini} \\ M = \{p \in \mathcal{P}|\nu(p)\neq 0\} \text{ fini} \\ n \text{ minimale pour cette propriété} \end{array}\right.$\\

		Soit $p \in M, \mu(p) \neq 0$ donc $p|n$\\
		Si $\nu(p) = 0, \forall q \in \N, p \wedge q = 1$ donc $p|1$		Contradiction avec le théorème de \Gauss\\

		Donc on peut simplifier $(*)$ par $p$\\
		On a alors $2$ décompositions de $\frac{n}{p}<n$		Contradiction\\
		Donc $n$ n’existe pas\\

\end{prv}

\begin{prop}[Divisibilité et Nombres Premiers]

		Soient $(,b,c) \in \N^3$ supérieurs à $2$\\
		On pose $a = {\underset{p \in \mathcal{ P}}{\prod}} p^{\alpha(p)}$ et $b = {\underset{p \in \mathcal{ P}}{\prod}}p^{\beta(p)}$\\
		$a|b \iff \forall p \in \mathcal{P}, \alpha(p) \leq \beta(p)$\\

\end{prop}

\begin{prv}

		“$\Longrightarrow$” : On suppose $a|b, \exist k, b = ak$\\
				On pose $k = {\underset{p \in \mathcal{ P}}{\prod}}p^{\kappa(p)}$\\
				et donc $b = {\underset{p \in \mathcal{ P}}{\prod}}p^{\alpha(p)}{\underset{p \in \mathcal{ P}}{\prod}}p^{\kappa(p)} = {\underset{p \in \mathcal{ P}}{\prod}}p^{\alpha(p) + \kappa(p)}$\\

				Par unicité de la décomposition en facteurs premiers,\\
				$\forall p \in \mathcal{P}, \beta(p) = \alpha(p) + \kappa(p) \ge \alpha(p)$\\

		“$\Longleftarrow$” : On suppose $\forall p \in \mathcal{P}, \beta(p) \ge \alpha(p)$\\
				On pose $\forall p \in \mathcal{P}, \kappa(p) = \beta(p) - \alpha(p) \in \N$\\

				Tous les $\alpha(p)$ et $\beta(p)$ sont nuls à partir d’un certain rang\\
				C'est donc le cas aussi pour les $\kappa(p)$\\
				Donc on a le droit de former le produit\\
				${\underset{p \in \mathcal{ P}}{\prod}}p^{\kappa(p)} \in \N$\\

				On pose $k = {\underset{p \in \mathcal{ P}}{\prod}}p^{\kappa(p)}$ et $ak = {\underset{p \in \mathcal{ P}}{\prod}}p^{\alpha(p)}{\underset{p \in \mathcal{ P}}{\prod}}p^{\kappa(p)} = {\underset{p \in \mathcal{ P}}{\prod}}p^{\alpha(p) + \kappa(p)} = {\underset{p \in \mathcal{ P}}{\prod}}p^{\beta(p)}=b$\\

\end{prv}

\begin{prop}[Produit de Facteurs Premiers, PGCD et PPCM]

		Avec les notations précédentes,\\

		$a \wedge b = {\underset{p \in \mathcal{ P}}{\prod}}p^{\text{min}(\alpha(p), \beta(p))}$	et	$a \vee b = {\underset{p \in \mathcal{ P}}{\prod}}p^{\text{max}(\alpha(p), \beta(p))}$\\
		\\

\end{prop}

\begin{crlr}

$(a \wedge b)(a \vee b) = ab$\\

\end{crlr}

\begin{prv}

		$(a \wedge b)(a \vee b) = {\underset{p \in \mathcal{ P}}{\prod}}p^{\text{min}(\alpha(p), \beta(p))}{\underset{p \in \mathcal{ P}}{\prod}}p^{\text{max}(\alpha(p), \beta(p))}$\\
				$= {\underset{p \in \mathcal{ P}}{\prod}}p^{\text{min}(\alpha(p),\beta(p)) + \text{max}(\alpha(p), \beta(p))}$\\
				$={\underset{p \in \mathcal{ P}}{\prod}}p^{\alpha(p) + \beta(p)} = ab$\\

\end{prv}

\fi
	}

	{
		\chap[11]{Suites numériques}
		\renewcommand{\cwd}{../chap11}
		\begin{defn}
	Un \underline{proposition} est un énoncé qui est soit vrai, soit faux.
\end{defn}

\begin{exm}
	\begin{align*}
		\begin{rcases*}
			A: ``B \text{ est vraie }"\\
			B: ``A \text{ est fausse }"\\
		\end{rcases*} \text{ Le système $\{A,B\}$ est une \underline{auto-contradiction}}
	\end{align*}
\end{exm}

\begin{defn}
	\underline{Démontrer} une proposition revient à prouver qu'elle est vraie
\end{defn}

		\begin{defn}
	Soit $E$ un $\mathbbm{K}$-espace vectoriel. On dit que $E$ est de \underline{dimension finie} si $E$ a au moins une famille génératrice finie. On dit que $E$ est de \underline{dimension infinie} sinon.
	\index{dimension finie (espace vectoriel)}
	\index{dimension infinie (espace vectoriel)}
\end{defn}

\begin{thm}
	[Théorème de la base extraite]
	Soit $E$ un $\mathbbm{K}$-espace vectoriel non nul de dimension finie. Soit $\mathcal{G}$ une famille génératrice finie de $E$. Alors, il existe une base $\mathcal{B}$ de $\mathcal{E}$ telle que $\mathcal{B} \subset \mathcal{G}$.
\end{thm}

\begin{prv}
	[par récurrence sur $\#G = \Card(G)$]
	\begin{itemize}
		\item Soit $E$ un $\mathbbm{K}$-espace vectoriel non nul engendré par $\mathcal{G} = (u)$.\\
			Si $u = 0_E$, alors $E = \{0_E\}$: une contradiction $\lightning$ \\
			Donc $u \neq 0_E$ donc $(u)$ est libre. En effet, \[
				\forall \lambda \in \mathbbm{K}, \lambda u = 0_E \implies \lambda = 0_\mathbbm{K}
			\] Donc $\mathcal{G}$ est une base de $E$.\\
		\item Soit $n \in \N_*$. Soit $E$ un $\mathbbm{K}$-espace vectoriel. On suppose que si $E$ a une famille génératrice constituée de $n$ vecteurs, alors on peut extraire de cette famille une base de $E$.\\
			Soit $\mathcal{G}$ une famille génératrice de $E$ avec $n+1$ vecteurs.\\
			Si $\mathcal{G}$ est libre, alors $\mathcal{G}$ est une base de $E$. \\
			Si $\mathcal{G}$ n'est pas libre, alors il existe $u \in \mathcal{G}$ tel que $u \in \Vect(\mathcal{G}\setminus \{u\})$ \\
			Donc $\mathcal{G}\setminus \{u\}$ engendre $E$. Or, $\mathcal{G}\setminus \{u\}$ possède $n$ vecteurs. D'après l'hypothèse de récurrence, il existe une base $\mathcal{B}$ de $E$ telle que \[
				\mathcal{B} \subset \mathcal{G} \setminus \{u\} \subset \mathcal{G}
			\] 
	\end{itemize}
\end{prv}

\begin{crlr}
	Tout espace de dimension finie a une base.
	\qed
\end{crlr}

\begin{thm}
	[Théorème de la base incomplète]
	Soit $E$ un $\mathbbm{K}$-espace vectoriel de dimension finie, $\mathcal{G}$ une famille génératrice finie de $E$. $\mathcal{L}$ une famille libre de $E$. Alors, il existe une base $\mathcal{B}$ de $E$ telle que \[
		\mathcal{L} \subset \mathcal{B} \text{ et } \mathcal{B}\setminus \mathcal{L} \subset \mathcal{G}
	\] 
\end{thm}

\begin{prv}
	[par récurrence sur $\#(\mathcal{G}\setminus\mathcal{L})$]
	\begin{itemize}
		\item Avec les notations précédentes, on suppose que $\mathcal{G}\setminus\mathcal{L} \neq \O$ \[
				\forall u \in \mathcal{G}, u \in \mathcal{L}
			\] Donc $\mathcal{G} \subset \mathcal{L}$ donc $\mathcal{L}$ est génératrice donc $\mathcal{L}$ est une base de $E$. On pose $\mathcal{B} = \mathcal{L}$ et alors \[
				\mathcal{L} \subset  \mathcal{B} \text{ et } \mathcal{B}\setminus\mathcal{L} = \O \subset  \mathcal{G}
			\] 
		\item Soit $n \in \N$. On suppose que si $\mathcal{G}$ est génératrice et $\mathcal{L}$ libre avec $\#(\mathcal{G}\setminus\mathcal{L}) = n$ alors il existe une base $\mathcal{B}$ de $E$ telle que \[
			\mathcal{L}\subset \mathcal{B} \text{ et } \mathcal{B}\setminus\mathcal{L}\subset \mathcal{G}
		\] Soient à présent $\mathcal{G}$ une famille génératrice de $E$ et $\mathcal{L}$ une famille libre de $E$ telles que $\#(\mathcal{G}\setminus\mathcal{L}) = n+1 > 0$\\
		Si $\mathcal{L}$ engendre $E$, alors $\mathcal{L}$ est une base de $E$. On pose $\mathcal{B} = \mathcal{L}$ et on a bien \[
			\mathcal{L} \subset  \mathcal{B} \text{ et } \mathcal{B} \setminus \mathcal{L} = \O \subset  \mathcal{G}
		\] On suppose que $\mathcal{L}$ n'engendre pas $E$. Il existe $u \in \mathcal{G}$ tel que $u \not\in \Vec(\mathcal{L})$ (car sinon, $\mathcal{G} \subset \Vect(\mathcal{L})$ et donc $\underbrace{\Vect(\mathcal{G})}_{= E} \subset  \underbrace{\Vect(\mathcal{L})}_{ \subset E}$\\
		Donc $\mathcal{L} \cup \{u\} $ est libre. On pose $\mathcal{L}' = \mathcal{L} \cup \{u\} $ \[
			\mathcal{G}\setminus \mathcal{L}' = \mathcal{G}\setminus (\mathcal{L} \cup \{u\}) = (\mathcal{G}\setminus\mathcal{L})\setminus \{u\} 
		\] donc $\#(\mathcal{G}\setminus\mathcal{L}') = n+1 -1 = n$\\
		D'après l'hypothèse de récurrence, il existe $\mathcal{B}$ une base de $E$ telle que \[
			\mathcal{L} \subset  \mathcal{L}' \subset \mathcal{B} \text{ et } \mathcal{B}\setminus \mathcal{L}' \subset \mathcal{G}
		\] \[
			\mathcal{B} \setminus \mathcal{L} = \underbrace{\mathcal{B}\setminus\mathcal{L}'}_{\subset \mathcal{G}} \cup \underbrace{\{u\}}_{\subset \mathcal{G} \text{ car } u \in \mathcal{G}}
		\] On a $\mathcal{B}\setminus\mathcal{L}\subset \mathcal{G}$
	\end{itemize}
\end{prv}

\begin{thm}
	Soit $E$ un $\mathbbm{K}$-espace vectoriel de dimension finie. Toutes les bases de $E$ ont le même cardinal.
\end{thm}

\begin{prv}
	Soit $\mathcal{G}$ une famille génératrice finie de $E$ et $\mathcal{B} \subset  \mathcal{G}$ une base de $E$. On note $n = \#\mathcal{B}$ \\
	Soit $\mathcal{B}'$ une base de $E$. On pose $p = n - \#(\mathcal{B} \cap  \mathcal{B}')$. Montrons par récurrence sur  $p$ que $\#\mathcal{B} = \#\mathcal{B}'$ 
	\begin{itemize}
		\item On suppose que $p = 0$. Alors, $\#(\mathcal{B} \cap \mathcal{B}') = n$ \\
			Or, $\mathcal{B}' \cap \mathcal{B} \subset \mathcal{B}$ donc $\mathcal{B} \cap \mathcal{B}' = \mathcal{B}$ donc $\mathcal{B} \subset  \mathcal{B}'$ et donc $\mathcal{B} = \mathcal{B}'$ 
		\item Soit $p \in \N$. On suppose que si $\mathcal{B}'$ est une base de $E$ telle que $n - \#(\mathcal{B} \cap \mathcal{B}') = p$, alors $\#\mathcal{B}' = n$ \\
			Aoit $\mathcal{B}'$ une base de $E$ telle que $n - \#(\mathcal{B}\cap \mathcal{B}') = p+1 > 0$ \\
			Donc $\mathcal{B} \cap \mathcal{B}' \neq \mathcal{B}$. Soit $u \in \mathcal{B}' \setminus \mathcal{B}$. D'après le lemme d'échange, il existe $v \in \mathcal{B}\setminus \mathcal{B}'$ tel que $\mathcal{B}' \setminus \{u\} \cup \{v\}$ est une base de $E$. On pose $\mathcal{B}'' = \mathcal{B}' \setminus \{u\} \cup \{v\}$ 
			\begin{align*}
				\mathcal{B}'' \cap \mathcal{B} &= \left( (\mathcal{B}' \setminus \{u\})  \cap \mathcal{B} \right) \cup \{v\} \\
				&= (\mathcal{B}' \cap \mathcal{B}) \cup \{v\} \\
			\end{align*}
			donc,
			\begin{align*}
				n - \#(\mathcal{B}'' \cap \mathcal{B}) &= n - (\#(\mathcal{B}' \cap \mathcal{B}) + 1) \\
				&= p+1- 1 \\
				&= p \\
			\end{align*}
			D'après l'hypothèse de récurrence, \[
				\#\mathcal{B}'' = n
			\] Or, $\#\mathcal{B}'' = \#\mathcal{B}'$
	\end{itemize}
\end{prv}

\begin{lem}
	Soient $\mathcal{B}$ et $\mathcal{B}'$ deux bases de $E$ telles que $\mathcal{B}\subset \mathcal{B}'$. Alors, $\mathcal{B} = \mathcal{B}'$.
\end{lem}

\begin{prv}
	On suppose $\mathcal{B}' \neq \mathcal{B}$. Soit $u \in \mathcal{B}' \setminus \mathcal{B}$
	$u \in E = \Vect(\mathcal{B})$ donc $\mathcal{B} \cup \{u\}$ n'est pas libre.
	Donc $\mathcal{B}\cup \{u\} \subset \mathcal{B}'$ et $\mathcal{B}'$ est libre donc $\mathcal{B}\cup \{u\}$ est libre: une contradiction $\lightning$
\end{prv}

\begin{lem}
	[Lemme d'échange] Soient $\mathcal{B}_1$ et $\mathcal{B}_2$ deux bases de $E$ et $u \in \mathcal{B}_1 \setminus \mathcal{B}_2$. Alors, il existe $v \in \mathcal{B}_2$ tel que $(\mathcal{B}_1 \setminus \{u\}) \cup \{v\}$ soit une base de $E$.
\end{lem}

\begin{prv}
	[1${}^\text{nde}$ méthode]
	On suppose que pout tout $v \in \mathcal{B}_2$, $(\mathcal{B}_1\setminus \{u\}) \cup \{v\}$ n'est pas une base de $E$
	Soit $v \in \mathcal{B}_2$.
	\begin{itemize}
		\item Supposons $(\mathcal{B}_1\setminus \{u\})\cup \{v\}$ non libre. $\mathcal{B}_1 \setminus \{u\}$ est libre. Donc $v \in \Vect(\mathcal{B}_1 \setminus \{u\})$
		\item Supposons $(\mathcal{B}_1\setminus \{u\}) \cup \{v\}$ non génératrice.
			Comme $\mathcal{B}_1$ engendre $E$, $u \not\in \Vect(\mathcal{B}_1\setminus \{v\})$.
			On suppose que $\mathcal{B}_1 \neq \mathcal{B}_2$.
			$\forall v \in \mathcal{B}_2 \setminus \mathcal{B}_1, \Vect(\mathcal{B}_1 \setminus \{v\}) = \Vect(\mathcal{B}_1) = E \ni u$ 
			donc, $(\mathcal{B}_1\setminus \{u\}) \cup \{v\}$ engendre $E$ et donc \[
				v \in \Vect(\mathcal{B}_1 \setminus \{u\})
			\] On a aussi \[
				\forall v \in \mathcal{B}_1 \setminus \{u\}, v \in \Vect(\mathcal{B}_1\setminus \{u\})
			\] Comme $u \not\in \mathcal{B}_2$, on a \[
				\forall v \in \mathcal{B}_2, v \in \Vect(\mathcal{B}_1\setminus \{u\})
			\] docn \[
				E = \Vect(\mathcal{B}_2) \subset \Vect(\mathcal{B}_1\setminus \{u\})
			\] donc $\mathcal{B}_1\setminus \{u\}$ engendre $E$ donc $\mathcal{B}_1\setminus \{u\}$ est une base de $E$. Or, $\mathcal{B}_1 \setminus \{u\}  \subset  \mathcal{B}_1$, donc $\mathcal{B}_1\setminus \{u\} = \mathcal{B}_1$
	\end{itemize}
\end{prv}

\begin{prv}
	[2${}^\text{nde}$ méthode]
	On suppose que pout tout $v \in \mathcal{B}_2$, $(\mathcal{B}_1\setminus \{u\}) \cup \{v\}$ n'est pas une base de $E$
	\begin{itemize}
		\item Comme $u \in \mathcal{B}_1 \setminus \mathcal{B}_2$, nécéssairement $\mathcal{B}_1 \neq \mathcal{B}_2$ donc $\mathcal{B}_2 \not\subset \mathcal{B}_1$, donc $\mathcal{B}_2\setminus\mathcal{B}_1 \neq \O$ 
		\item Soit $v \in \mathcal{B}_2\setminus\mathcal{B}_1$. Il existe $(\lambda_w)_{w\in\mathcal{B}_1}$ une famille de scalaires presque nulle telle que \[
				v = \sum_{w \in \mathcal{B}_1} \lambda_w w - \lambda_u u + + \sum_{w \in \mathcal{B}_1\setminus \{u\}}\lambda_w w
			\]
			Si $\lambda_u \neq 0_E$, alors
			\begin{align*}
				u &= \lambda_u^{-1}\left( v - \sum_{w \in \mathcal{B}_1 \setminus \{u\}} \lambda_w w \right)\\
					&\in \Vect(\mathcal{B}_1\setminus \{u\} \cup v)
			\end{align*}
			 donc $\mathcal{B}_1 \subset \Vect(\mathcal{B}_1\setminus \{u\} \cup \{v\})$\\
			 et donc $E \subset  \Vect(\mathcal{B}_1 \setminus \{u\} \cup \{v\})$ \\
			 et donc $\mathcal{B}_1 \setminus \{u\} \cup \{v\}$ engendre $E$ \\
			 donc $\mathcal{B}_1 \setminus \{u\} \cup \{v\}$ n'est pas libre\\
			 donc $v \in \Vect(\mathcal{B}_1\setminus \{u\})$ (car $\mathcal{B}_1 \setminus \{u\}$ est libre\\
			 donc $\lambda_u = 0_\mathbbm{K}$ $\lightning$\\`

			 Donc, $\lambda_u = 0_\mathbbm{K}$, docn $v \in \Vect(\mathcal{B}_1\setminus \{u\})$ \\
			 On vient de prouver que
			 \begin{align*}
			 	\mathcal{B}_2 \setminus \mathcal{B}_1 \subset \Vect(\mathcal{B}_1 \setminus \{u\})\\
			 	\mathcal{B}_1 \setminus \{u\} \subset \Vect(\mathcal{B}_1 \setminus \{u\})\\
			 \end{align*}
			 Comme $u \not\in \mathcal{B}_2$, \[
			 	\mathcal{B}_2 \subset \Vect(\mathcal{B}_1 \setminus \{u\})
			 \] donc \[
			 	E = \Vect(\mathcal{B}_2) \subset  \Vect(\mathcal{B}_1 \setminus \{u\})
			 \] donc $\mathcal{B}_1 \setminus \{u\}$ engendre $E$. Donc,  $\mathcal{B}_1 \setminus \{u\}$ est une base de $E$.\\
			 Or, $\mathcal{B}_1 \setminus \{u\} \subset  \mathcal{B}_1$, donc $\mathcal{B}_1 \setminus \{u\} = \mathcal{B}_1$
	\end{itemize}
\end{prv}

\begin{defn}
	Soit $E$ un $\mathbbm{K}$-espace vectoriel de dimension finie. Le cardinal commun à toutes les bases de $E$ est appelé \underline{dimension} de $E$ est notée $\dim(E)$ ou $\dim_\mathbbm{K}(E)$\\
	C'est donc aussi le nombre de coordonnées de n'importe quel vecteur dans n'importe quelle base.
	\index{dimension (espace vectoriel)}
\end{defn}

\begin{exm}
	\begin{enumerate}
		\item $\dim_\R(\C) = 2$ et $\dim_\C(\C) = 1$ 
		\item $\dim_\mathbbm{K}(\mathbbm{K}^{n}) = n$ 
		\item $\dim_{\mathbbm{K}}(\mathcal{M}_{n,p}(\mathbbm{K})) = np$
	\end{enumerate}
\end{exm}

\begin{crlr}
	Soit $E$ un $\mathbbm{K}$-espace vectoriel de dimension finie, $\mathcal{L}$ une famille libre de $E$, $\mathcal{G}$ une famille génératrice de $E$. On note $n = \dim(E)$
	\begin{enumerate}
		\item $\#\mathcal{G} \ge n$ et $(\#\mathcal{G} = n \implies \mathcal{G} \text{ est une base de } E$)
		\item $\#\mathcal{L} \le n$ et $(\#\mathcal{L} = n \implies \mathcal{L} \text{ est une base de } E$)
	\end{enumerate}
\end{crlr}

\begin{crlr}
	$\R^{\R}$ est de dimension infinie.
	$\forall i \in \N, e_i: x \mapsto x^i$\\
	$(e_i)_{i\in\N}$ est libre dans $\R^\R$
\end{crlr}

\begin{prop}
	Soient $E$ et $F$ deux $\mathbbm{K}$-espaces vectoriels de dimension finie. Alors $E\times F$ est de dimension finie et $\dim(E\times F) = \dim(E) + \dim(F)$
\end{prop}

\begin{prv}
	Soit $(e_1,\ldots, e_n)$ une base de $E$, $(f_1, \ldots, f_p)$ une base de $F$.
	On pose \[
		\left\{\begin{array}
			{r c l}
			u_1 &=& (e_1,0_F)\\
			u_2 &=& (e_2,0_F)\\
					&\vdots&\\
			u_n &=& (e_n,0_F)\\
			u_{n+1} &=& (0_E, f_1)\\
			u_{n+2} &=& (0_E, f_2)\\
					&\vdots&\\
			u_{n+p} &=& (0_E,f_p)\\
		\end{array}\right.
	\]
	Soit $(x,y) \in E\times F$. \[
		\begin{cases}
			\exists (x_1,\ldots,x_n)\in \mathbbm{K}^n, x = \sum_{i=1}^{n} x_ie_i
			\exists (y_1,\ldots,y_n)\in \mathbbm{K}^n, x = \sum_{j=1}^{p} y_jf_j
		\end{cases}
	\] 
	\begin{align*}
		(x,y) &= \left( \sum_{i=1}^{n} x_ie_i, \sum_{i=1}^{p} y_jf_j \right)  \\
		&= \sum_{i=1}^{n} x_i (e_i + 0_F) + \sum_{j=1}^{p} y_j (0_E, f_j) \\
		&= \sum_{i=1}^{n} x_i u_i + \sum_{j=1}^{p} y_j u_{n+j} \\
	\end{align*}
	Donc, $E\times F = \Vect(u_1, \ldots, u_{n+p})$ donc $E\times F$ est de dimension finie.\\
	Soit $(\lambda_1, \ldots, \lambda_{n+p}) \in \mathbbm{K}^{n+p}$ tel que \[
		(*): \quad \sum_{k=1}^{n+p} \lambda_ku_k = 0_{E\times F} = (0_E, 0_F)
	\]
	\begin{align*}
		(*) &\iff \sum_{k=1}^{n} \lambda_k (e_k, 0_F) + \sum_{k=n+1}^{p} \lambda_k(0_E, f_{k-n}) = (0_E, 0_F)\\
				&\iff \begin{cases}
					\sum_{k=1}^{n} \lambda_k e_k = 0_E\\
					\sum_{k=n+1}^{p} \lambda_k f_{k-n} = 0_F
				\end{cases}\\
				&\iff \begin{cases}
					\forall k \in \left\llbracket 1,n \right\rrbracket, \lambda_k = 0_\mathbbm{K} \qquad&(\text{car $(e_1,\ldots,e_n)$ est libre})\\
					\forall k \in \left\llbracket n+1,n+p \right\rrbracket, \lambda_k = 0_\mathbbm{K} \qquad&(\text{car $(f_1,\ldots,f_n)$ est libre})\\
				\end{cases}
	\end{align*}
	Donc $(u_1, \ldots, u_{n+p})$ est une base de $E\times F$. Donc, $\dim(E\times F) = n + p = \dim(E) + \dim(F)$
\end{prv}

\begin{rmk}
	[Convention]
	\[\dim\big(\{0_E\}\big) = 0\]
\end{rmk}

\begin{thm}
	Soit $E$ un $\mathbbm{K}$-espace vectoriel de dimension finie, $F$ un sous-espace vectoriel de $E$. Alors, $F$ est de dimension finie et  $\dim(F) \le \dim(E)$\\
	Si $\dim(F) = \dim(E)$, alors $F = E$
\end{thm}

\begin{prv}
	On considère \[
		A = \{k \in \N \mid \text{il existe une famille libre de $F$ à $k$ éléments}\} 
	\]
	On suppose $F \neq \{0_E\}$.
	\begin{itemize}
		\item Soit $u \in F\setminus \{0_E\}$. $(u)$ est libre donc $1 \in A$ et donc $A \neq \O$
		\item Soit $\mathcal{L}$ une famille libre de $F$. Alors, $\mathcal{L}$ est une famille libre de $E$ \\
			donc $\#\mathcal{L} \le \dim(E)$\\
			Donc $A$ est majorée par $\dim(E)$ \\
			On en déduit que $A$ a un plus grand élément $p$.
		\item Soit $\mathcal{L}$ une famille libre de $F$ avec $p$ éléments.\\
			Si $\mathcal{L}$ n'engendre pas $F$, alors il existe $u\in F$ tel que $u\not\in \Vect(\mathcal{L})$ et donc $\mathcal{L} \cup \{u\}$ est une famille libre de $F$, donc $p+1 \in A$ en contradiction avec la maximalité de $p$.\\
			Donc $\mathcal{L}$ est une base de $F$ donc $F$ est de dimension finie et $\dim(F) = p \le \dim(E)$\\
	\end{itemize}

	Soit $\mathcal{B}$ une base de $F$. Alors, $\mathcal{B}$ est aussi une famille de libre de de $E$. Donc $\#\mathcal{B} \le \dim(E)$ donc $\dim(F) = \dim(E)$ \\
	Si $\dim(F) = \dim(E)$, alors $\mathcal{B}$ est une base de $E$, et donc $F = \Vect(\mathcal{B}) = E$
\end{prv}

\begin{prop}
	[Formule de Grassmann]
	Soit $E$ un $\mathbbm{K}$-espace vectoriel de dimension finie, $F$ et $G$ deux sous-espace vectoriels de $E$. Alors, \[
		\dim(F+G) = \dim(F) + \dim(G) - \dim(F\cap G)
	\] 
\end{prop}

\begin{prv}
	Soit $(e_1, \ldots, e_p)$ une base de $F\cap G$. $(e_1,\ldots,e_p)$ est une famille libre de $F$.\\
	On complète $(e_1, \ldots, e_p)$ en une base $(e_1, \ldots, e_p, u_1, \ldots, u_q)$ de $F$.\\
	De même, on complète $(e_1, \ldots, e_p)$ en une base $(e_1, \ldots, e_p, v_1, \ldots, v_r)$ de $G$.\\
	On pose  $\mathcal{B} = (e_1, \ldots, e_p, u_1, \ldots, u_q, v_1, \ldots, v_r)$. Montrons que $\mathcal{B}$ est une base de $F+G$
	\begin{itemize}
		\item Soit $u \in F+G$ \\
			On pose $u = v+w$ avec $\begin{cases}
				v\in F\\
				w \in G
			\end{cases}$.\\
			On pose $v = \sum_{i=1}^p \lambda_i e_i + \sum_{i=1}^q \mu_i u_i$ avec $(\lambda_1, \ldots, \lambda_p, \mu_1, \ldots, \lambda_q) \in \mathbbm{K}^{p+q}$\\
			On pose aussi $w = \sum_{i = 1}^p \lambda'_ie_i + \sum_{j=1}^r \nu_j v_j$ avec $(\lambda_1',\ldots,\lambda_p', \nu_1, \ldots, \nu_r) \in \mathbbm{K}^{p+r}$\\
			D'où, \[
				u = \sum_{i=1}^p (\lambda_i + \lambda'_i)e_i + \sum_{j=1}^q \mu_j u_j + \sum_{k=1}^r \nu_k v_k \in \Vect(\mathcal{B})
			\]
		\item Soient $(\lambda_1, \ldots, \lambda_p, \mu_1, \ldots, \mu_q, \nu_1, \ldots, \nu_r) \in \mathbbm{K}^{p+q+r}$.\\
			On suppose \[
				(*)\quad \sum_{i=1}^{p}\lambda_ie_i + \sum_{j=1}^q\mu_ju_j + \sum_{k=1}^r \nu_k v_k = 0_E
			\] 
			D'où, \[
				\underbrace{\sum_{i=1}^p\lambda_i e_i + \sum_{j=1}^q \mu_ju_j}_{\in F} = \underbrace{-\sum_{k=1}^r\nu_jv_k}_{\in G}
			\] 
			Donc, \[
				f = \sum_{i=1}^p \lambda_i e_i + \sum_{j=1}^q \mu_j u_j \in F\cap G
			\] Comme $(e_1, \ldots, e_p)$ est une base de $F\cap G$, $\exists ! (\lambda_1', \ldots, \lambda_p') \in \mathbbm{K}^p$ tel que \[
				f = \sum_{i=1}^p \lambda'_i e_i = \sum_{i=1}^p \lambda'_i e_i + \sum_{j=1}^q 0_\mathbbm{K}u_j
			\] Comme $(e_1, \ldots, e_p, u_1, \ldots, u_q)$ est une base de $F$, \[
				\forall k \in \left\llbracket 1, q \right\rrbracket, \mu_j = 0_\mathbbm{K}
			\] De même, \[
				\forall k \in \left\llbracket 1,r \right\rrbracket , \nu_k = 0_\mathbbm{K}
			\] On remplace dans $(*)$ pour trouver \[
				\sum_{i=1}^p \lambda_ie_i = 0_E
			\] Comme $(e_1, \ldots, e_p)$ est libre, \[
				\forall i \in \left\llbracket 1,p \right\rrbracket, \lambda_i = 0_\mathbbm{K}
			\] Donc $\mathcal{B}$ est libre.\\
			Donc, 
			\begin{align*}
				\dim(F+G) &=  p +q + r \\
				&= (p+q)+ (p+r) - p \\
				&= \dim(F) + \dim(G) - \dim(F\cap G) \\
			\end{align*}
	\end{itemize}
\end{prv}

\begin{crlr}
	Avec les hypothèse précédentes, \[
		E = F \oplus G \iff \begin{cases}
			F \cap  G = \{0_E\} \\
			\dim(E) = \dim(F) + \dim(G)
		\end{cases}
	\] 
\end{crlr}

\begin{prv}
	\begin{itemize}
		\item[``$\implies$''] On suppose $E = F \oplus G$ \\
			Comme la somme est directe, $F \cap G = \{0_E\}$ 
			\begin{align*}
				\dim(E) &= \dim(F)\\
				&= \dim(F) + \dim(G) - \dim(F\cap G)\\
				&= \dim(F) + \dim(G)\\
			\end{align*}
		\item[``$\impliedby$''] On suppose $F\cap G = \{0_E\}$ et $\dim(E) = \dim(F) + \dim(G)$.\\
			On sait déjà que $F+G = F \oplus G$\\
			 \begin{align*}
				\dim(F+G) = \dim(F) + \dim(G) - \dim(F \cap G) = \dim(E)
			\end{align*}
			Donc $F + G = E$
	\end{itemize}
\end{prv}

\begin{prop}
	Soit $F$ un $\mathbbm{K}$-espace vectoriel de dimension finie $n$. Soit $\mathcal{B} = (e_1, \ldots, e_n)$ une base de $F$. L'application
	\begin{align*}
		f: \mathbbm{K}^n &\longrightarrow F \\
		(\lambda_1, \ldots, \lambda_n) &\longmapsto \sum_{i=1}^n \lambda_i e_i
	\end{align*} est bijective.\\
	Si $\mathbbm{K}$ est infini, $\mathbbm{K}^n$ aussi et donc $F$ aussi.\\
	Si $\#\mathbbm{K} = p \in \N_*$,
	\begin{align*}
		\#&\mathbbm{K}^n = p^n\\
		&\vrt=\\
		\#&F
	\end{align*}
\end{prop}


		\part{Dérivation}

\underline{Motivation}:

{
\begin{wrapfigure}{l}{3cm}
	\centering
	\begin{asy}
		import three;

		size(3cm);
		settings.render=0;
		settings.prc=false;
		currentprojection = obliqueZ;

		draw(unitbox);
		draw(shift(1.1Z + 0.05X) * (O -- X), Arrows3(TeXHead2));
		draw(shift(1.1Z + 0.05Y) * (O -- Y), Arrows3(TeXHead2));
		draw(shift(1.1X + 0.05Z) * (O -- Z), Arrows3(TeXHead2));

		label("$x$", (X/2) + (1.1Z + 0.05X), align=S);
		label("$y$", (Y/2) + (1.1Z + 0.05Y), align=W);
		label("$z$", (Z/2) + X, align=SE);
	\end{asy}
\end{wrapfigure}

\begin{align*}
	&S(x,y,z) = 2(xy + xz + yz)\\
	&V(x,y,z) = xyz
\end{align*}

On cherche à minimiser $S$ avec la contrainte $V = 1$.

Soit $f : \begin{array}{rcl}
	\left( \R_*^+ \right)^2 &\longrightarrow& \R \\
	(x,y) &\longmapsto& S\left( x,y,\frac{1}{xy} \right) = 2\left( xy + \frac{1}{y} + \frac{1}{x} \right).
\end{array}$

On cherche $(a,b) \in \left( \R^+_* \right)^2$ tel que \[
	\forall (x,y) \in (\R^+_*), f(x,y) \ge f(a,b).
\]
}

\begin{defn}
	Soit $f: U \to \R$ où $U$ est un ouvert de $\R^2$. Soit $(a,b) \in U$.
	\vspace{2mm}

	Si $\lim_{x \to a} \frac{f(x,b) - f(a,b)}{x - a} \in \R$, alors on dit que $f$ a une dérivée partielle suivant $x$ en $(a,b)$ et cette limite est notée \[
		\partial f_1(a,b) = \frac{\partial f}{\partial x}(a,b).
	\]

	Si $\lim_{y \to b} \frac{f(a,y) - f(a,b)}{y - b} \in \R$, alors on dit que $f$ a une dérivée partielle suivant $y$ et la limite est notée \[
		\partial f_2(a,b) = \frac{\partial f}{\partial y}(a,b).
	\]
\end{defn}

\begin{exm}
	\begin{enumerate}
		\item $f: (x,y) \mapsto xy + x - y$.

			\begin{align*}
				&\frac{\partial f}{\partial x} : (x,y) \mapsto y + 1,\\
				&\frac{\partial f}{\partial y} : (x,y) \mapsto x - 1.
			\end{align*}

		\item $f: (x,y) \mapsto xy + \frac{1}{y}+ \frac{1}{x}$.

			\begin{align*}
				&\frac{\partial f}{\partial x}: (x,y) \mapsto y - \frac{1}{x^2},\\
				&\frac{\partial f}{\partial y}: (x,y) \mapsto x - \frac{1}{y^2}.
			\end{align*}

		\item Trouver $f$ telle que $\begin{cases}
				(1): \qquad \frac{\partial f}{\partial x}=y,\\[2mm]
				(2): \qquad \frac{\partial f}{\partial y} = x.
			\end{cases}$

			D'après $(1)$ : \[
				\forall (x,y), \exists C(y) \in \R, f(x,y) = xy + C(y)
			\] et donc \[
				\frac{\partial f}{\partial y}(x,y) = x + C'(y)
			\] donc $C'(y) = 0$ et donc $C$ est constante.

		\item Trouver $f$ telle que $\begin{cases}
			\frac{\partial f}{\partial x} = -y,\\[2mm]
			\frac{\partial f}{ƒ\partial y} = x.
		\end{cases}$

		Ce n'est pas possible !
	\end{enumerate}
\end{exm}

\begin{defn}~\\
	\begin{minipage}{\linewidth}
		\begin{wrapfigure}{r}{4cm}
			\centering
			\vspace{-5mm}
			\begin{asy}
				import three;
				import graph3;
				size(4cm);

				settings.render = 0;
				settings.prc = false;
				currentprojection = obliqueX;

				draw(O -- X, Arrow3(TeXHead2));
				draw(O -- Y, Arrow3(TeXHead2));
				draw(O -- Z, Arrow3(TeXHead2));

				triple f(real x, real y, real z = 0) { return (x,y,cos(x - 0.5) * cos(y - 0.5)/1.2 + 0.15); }

				real inc = 1 / 5;

				for(real x = 0; x <= 1; x += inc) {
					draw(graph(
						new real(real t) { return x; }, // x
						new real(real y) { return y; }, // y
						new real(real y) { return f(x,y).z; }, // z
						0, 1
					), gray);
				}

				for(real y = 0; y <= 1; y += inc) {
					draw(graph(
						new real(real x) { return x; }, // x
						new real(real t) { return y; }, // y
						new real(real x) { return f(x,y).z; }, // z
						0, 1
					), gray);
				}

				path3 path1 = (0.8, 0.2, 0) .. (0.5, 0.5, 0) .. (0.3, 0.7, 0);
				path3 path2 = f(0.8, 0.2, 0) .. f(0.5, 0.5, 0) .. f(0.3, 0.7, 0);
				path3 d = (0.2, 0.3, 0) .. (0.3, 0.4, 0) .. (0.2, 0.7, 0) .. (0.8, 0.9, 0) .. (0.6, 0.2, 0) .. cycle;

				draw(path1, red, Arrow3(TeXHead2));
				draw(path2, red, Arrow3(TeXHead2, position=0.8));

				dot((0.5, 0.5, 0));
				dot(f(0.5, 0.5, 0));
				draw((0.5, 0.5, 0) -- f(0.5, 0.5, 0), dashed);
				draw(d);

				label("$w$", (0.3, 0.7, 0), red, align=SE);
				label("$U$", (0.8, 0.9, 0), align=SE);
			\end{asy}
		\end{wrapfigure}

		Soit $f: U \to \R$ où $U$ est un ouvert. Soit $(a,b) \in U$. Soit $w = (w_1, w_2) \in \R^2$.

		Si 
		\[
			\lim_{t\to 0} \frac{f(a + tw_1, b + tw_2) - f(a,b)}{t}
		\] existe et est réelle, alors on dit que $f$ a une dérivée dans la direction de $w$ et la limite est notée \[
			\mathrm{d}f(w)\,(a,b) = D_w(f)\,(a,b).
		\]
	\end{minipage}
\end{defn}

\begin{exm}
	\begin{align*}
		f: \left( \R_*^+ \right)^2 &\longrightarrow \R \\
		(x,y) &\longmapsto xy+\frac{1}{x}+\frac{1}{y}.
	\end{align*}

	On pose $(a,b) = (1,2)$, $w = (w_1, w_2) = (1,1)$.
	\begin{align*}
		\frac{f(1+t, 2+t) - f(1,2)}{t} &= \frac{1}{t} \left( (1+t)(2+t) + \frac{1}{1+t} + \frac{1}{2+t} - 3 - \frac{1}{2} \right) \\
		&= \frac{1}{t}\left(\cancel 2 + 3t + \po(t) + \cancel 1 - t + \po(t) + \frac{1}{2}\left( \cancel 1 - \frac{t}{2} + \po(t) \right) - \cancel3 - \cancel{\frac{1}{2}} \right) \\
		&= \frac{1}{t} \left( \frac{7}{4} t + \po(t) \right)  \\
		&= \frac{7}{4} + \po(1) \tendsto{t \to 0} \frac{7}{4}. \\
	\end{align*}

	Donc, \[
		\mathrm{d}f(1,1)\,(1,2) = \frac{7}{4}.
	\]
\end{exm}

\begin{rmk}~\\
	\begin{figure}[H]
		\centering
		\begin{asy}
			import solids;
			import graph;
			size(5cm);

			settings.render = 0;
			settings.prc = false;

			path3 par = graph(
				new real(real x) { return x; },
				new real(real x) { return 0; },
				new real(real x) { return x^2; },
				0,3);
			revolution r = revolution(par, axis=Z);

			path3 par2 = graph(
				new real(real x) { return x; },
				new real(real x) { return 0; },
				new real(real x) { return x^2; },
				-3,3);

			draw(r,1,longitudinalpen=nullpen);
			draw(r.silhouette());

			draw((-4, 0, -1) -- (-4, 0, 10) -- (4, 0, 10) -- (4, 0, -1) -- cycle, red);
			draw(par2, deepred);

			draw((4,4.5) -- (7, 4.5), black+0.5mm, Arrow(TeXHead));

			path par2d = graph(new real(real x) { return x^2; }, -3, 3);
			draw(shift((11, 0)) * par2d, deepred);

			dot(O);
			dot((11, 0));
		\end{asy}
	\end{figure}
\end{rmk}


%todo ajouter théorème-définition
\begin{thm}
	Soit $f : U \to \R$, $(a,b) \in U$. On suppose que $\frac{\partial f}{\partial x}$ et $\frac{\partial f}{\partial y}$ existent en $(a,b)$ et sont {\bfseries continues} en $(a,b)$. Alors,
	\begin{align*}
		&\forall (h, k) \in \R^2 \text{ tel que } (a +h, b + k) \in U,\\
		&f(a+ h, b + k) = f(a,b) + h \frac{\partial f}{\partial x}(a,b) + k \frac{\partial f}{\partial y}(a,b) + \po_{(h,k)\to (0,0)}\big(\|(h,k)\|\big).
	\end{align*}

	On dit que $f$ est de classe $\mathcal{C}^1$ si $\frac{\partial f}{\partial x}$ et $\frac{\partial f}{\partial y}$ existent et sont continues.

	\qed
\end{thm}

\begin{rmk}
	En physique, cette formule correspond à : \[
		\mathrm{d}f = \frac{\partial f}{\partial x}\mathrm{d}x + \frac{\partial f}{\partial y} \mathrm{d}y.
	\] En effet :
	\begin{align*}
		\mathrm{d}f &= f(x+ \mathrm{d}x, y + \mathrm{d}y) - f(x,y) \\
		&= \frac{\partial f}{\partial x} \mathrm{d}x + \frac{\partial f}{\partial y} \mathrm{d}y.
	\end{align*}
\end{rmk}

\begin{prop}
	Soit $f: U \to \R$ de classe $\mathcal{C}^1$ en $(a,b) \in U$. Alors, \[
		\forall w = (w_1, w_2) \in \R^2, \mathrm{d}f(w)\,(a,b) = w_1 \frac{\partial f}{\partial x}(a,b) + w_2 \frac{\partial f}{\partial y}(a,b).
	\]
\end{prop}

\begin{prv}
	Soit $w = (w_1, w_2) \in \R^2$. Soit $t \in \R^*$.
	\begin{align*}
		\frac{1}{t}\big(f(a + tw_1, b + tw_2) - f(a,b)\big)
		&= \frac{1}{t} \left( tw_1 \frac{\partial f}{\partial x}(a,b) + tw_2 \frac{\partial f}{\partial y}(a,b) + \po_{t \to 0}\big(\|tw\|\big) \right) \\
		&= w_1 \frac{\partial f}{\partial x}(a,b) + w_2 \frac{\partial f}{\partial y}(a,b) + \po_{t\to 0}(1) \\
		&\tendsto{t\to 0} w_1 \frac{\partial f}{\partial x}(a,b) + w_2\frac{\partial f}{\partial y}(a,b).
	\end{align*}
\end{prv}


\begin{defn}
	Avec les hypothèses précédentes, en posant \[
		\nabla f(a,b) = \left( \frac{\partial f}{\partial x}(a,b), \frac{\partial f}{\partial y}(a,b) \right) 
	\]on obtient \[
		\mathrm{d}f(w)\,(a,b) = \left<w  \mid \nabla f(a,b) \right>
	\] où $\left<\cdot|\cdot \right>$ est le produit scalaire.

	Le vecteur $\nabla f(a,b)$ est appelé \underline{gradient de $f$ en $(a,b)$}.

	Le développement limité à l'ordre 1 de $f$ devient \[
		f\big((a,b)+w\big) = f(a,b) + \left<w \mid \nabla f(a,b) \right> + \po_{w\to 0}(\|w\|)
	\]
\end{defn}

\begin{prop}
	Soit $f : U \to \R$ de classe $\mathcal{C}^1$.

	\begin{figure}[H]
    \centering
    \incfig{gradient}
	\end{figure}

	$\nabla f$ est orthogonal au lignes de niveaux de $f$, son orientation va dans le sens d'une augmentation de $f$.
\end{prop}

\begin{prv}
	Soit $\gamma : I \to U$ une courbe de niveau : \[
		\forall t \in I, f\big(\gamma(t)\big) = \text{cste}.
	\] D'après le lemme suivant : \[
		\forall t \in I, 0 = (f \circ \gamma)'(t) = \mathrm{d}f\big(\gamma'(t)\big)\big(\gamma(t)\big) = \left<\gamma'(t)  \mid \nabla f\big(\gamma(t)\big) \right>
	\] Donc $\nabla f\big(\gamma(t)\big)$ est orthogonal à $\gamma'(t)$.

	Pour tout $t \in I$, on pose $w(t) = t\, \nabla f\big(\gamma(t)\big)$. Donc \[
		f\big(\gamma(t) + w(t)\big) = f\big(\gamma(t)\big) + t \|\nabla f(\gamma(t))\|^2 + \po_{t \to 0}(t)
	\] Pour $t$ assez petit, $f\big(\gamma(t) + w(t)\big) - f\big(\gamma(t)\big)$ est du même signe que $t$.
\end{prv}

\begin{rmk}
	\begin{align*}
		V: \R^3 &\longrightarrow \R \\
		(x,y,z) &\longmapsto -mgz
	\end{align*}
	l'énerge potentielle de pesenteur

	On a donc \[
		\nabla V(x,y,z) = \left( \frac{\partial V}{\partial x}, \frac{\partial V}{\partial y}, \frac{\partial V}{\partial z} \right) = (0, 0, -mg) = \vec{P}.
	\]
\end{rmk}

\begin{lem}
	Soit $f : U \to \R$ de classe $\mathcal{C}^1$, $\gamma : \begin{array}{rcl}
		I &\longrightarrow& U \\
		t &\longmapsto& \big(x(t), y(t)\big)
	\end{array}$ où $x$ et $y$ sont dérivables.

	On pose \[
		\forall t \in I, \gamma'(t) = \big(x'(t), y'(t)\big).
	\] Alors $f \circ \gamma : I \to \R$ est dérivable et
	\begin{align*}
		\forall t \in I, (f \circ \gamma)'(t) &= \mathrm{d}f\big(\gamma'(t)\big) \big(\gamma(t)\big)\\
		&= \left<\gamma'(t)  \mid \nabla f\big(\gamma(t)\big)  \right> \\
		&= x'(t) \frac{\partial f}{\partial x}\big(x(t), y(t)\big) + y'(t) \frac{\partial f}{\partial y}\big(x(t),y(t)\big). \\
	\end{align*}
\end{lem}

\begin{prv}
	On fixe $t \in I$.

	\begin{align*}
		\forall h \neq 0, \frac{f \circ \gamma(t + h) - f \circ \gamma(t)}{h}
		&= \frac{1}{h}\big(f(\gamma(t)) + h\gamma'(t) + \po_{h\to 0}(h) - f(\gamma(t))\big) \\
		&= \frac{1}{h}\bigg(\cancel{f(\gamma(t))} + \left<h\gamma'(t) \mid \nabla f(\gamma(t)) \right> + \po_{h\to 0}(\|h\gamma'(t)\|) - \cancel{f(\gamma(t))}\bigg)\\
		&= \left<\gamma'(t) \mid \nabla f(\gamma(t)) \right> + \po_{h\to 0}(1) \\
		&\tendsto{h\to 0} \left<\gamma'(t)  \mid \nabla f(\gamma(t)) \right>
	\end{align*}
\end{prv}

\begin{defn}
	Soit $f : U \to \R$ de classe $\mathcal{C}^1$ et $(a,b) \in U$. On dit que $(a,b)$ est un \underline{point critique} de $f$ si $\nabla f(a,b) = 0$ i.e. $\frac{\partial f}{\partial x}(a,b) = \frac{\partial f}{\partial y}(a,b) = 0$.

	Dans ce cas, $f(a,b)$ est appelé \underline{valeur critique} de $f$.
\end{defn}

\begin{prop}~\\
	\begin{minipage}{\linewidth}
		\begin{wrapfigure}{r}{3cm}
			\centering
			\vspace{-1cm}
			\begin{asy}
				import solids;
				import graph;
				size(3cm);

				settings.render = 0;
				settings.prc = false;

				path3 par = graph(
					new real(real x) { return x; },
					new real(real x) { return 0; },
					new real(real x) { return -x^2; },
					0,3);
				revolution r = revolution(par, axis=Z);

				draw(r,1,longitudinalpen=nullpen);
				draw(r.silhouette());

				dot("$(a,b)$", O, red, align=N);
				real s = sqrt(2.5);
				path3 g=(s,0,-2.5)..(0,s,-2.5)..(-s,0,-2.5)..(0,-s,-2.5)..cycle;
				draw(g, deepcyan);
			\end{asy}
		\end{wrapfigure}
		Soit $f: U \to \R$ de classe $\mathcal{C}^1$ et $(a,b) \in U$ tel que \[
			\exists r > 0, \forall (x,y) \in B_{(a,b)}(r), f(x,y) \le f(a,b)
		\] Alors $\nabla f(a,b) = (0,0)$.
	\end{minipage}
\end{prop}

\begin{prv}
	Soit $g: x \mapsto f(x,b)$. $g(a)$ est un maximum local de $g$ donc $g'(a) = 0$.

	Or, $g'(a) = \frac{\partial f}{\partial x}(a,b)$

	donc $\frac{\partial f}{\partial x}(a,b) = 0$.

	Soit $h : y \mapsto f(a,y)$. On a de même $h'(b) = 0$.

	Or, $h'(b) = \frac{\partial f}{\partial y}(a,b)$.

	Donc, $\nabla f(a,b) = (0,0)$.
\end{prv}

\begin{rmk}
	Un minimum local est aussi une valeur critique.
\end{rmk}

\begin{figure}[H]
	\centering
	\begin{subfigure}{3cm}
		\centering
		\begin{asy}
			import solids;
			import graph;
			size(3cm);

			settings.render = 0;
			settings.prc = false;

			path3 par = graph(
				new real(real x) { return x; },
				new real(real x) { return 0; },
				new real(real x) { return -x^2; },
				0,3);
			revolution r = revolution(par, axis=Z);

			draw(r,1,longitudinalpen=nullpen);
			draw(r.silhouette());

			dot(O, red);
		\end{asy}
		\caption{Maximum local}
	\end{subfigure}
	\begin{subfigure}{3cm}
		\centering
		\begin{asy}
			import solids;
			import graph;
			size(3cm);

			settings.render = 0;
			settings.prc = false;

			path3 par = graph(
				new real(real x) { return x; },
				new real(real x) { return 0; },
				new real(real x) { return x^2; },
				0,3);
			revolution r = revolution(par, axis=Z);

			draw(r,1,longitudinalpen=nullpen);
			draw(r.silhouette());

			dot(O, red);
		\end{asy}
		\caption{Minimum local}
	\end{subfigure}
	\begin{subfigure}{3cm}
		\centering
		\begin{asy}
			import solids;
			import graph;
			size(3cm);

			settings.render = 0;
			settings.prc = false;
			currentprojection = obliqueZ;

			draw(graph(
				new real(real x) { return x; },
				new real(real x) { return -x^2 / 3; },
				new real(real x) { return 3; },
				-3, 3
			));

			draw(graph(
				new real(real x) { return x; },
				new real(real x) { return -x^2 / 3; },
				new real(real x) { return -3; },
				-3, 3
			));

			draw(graph(
				new real(real x) { return x; },
				new real(real x) { return -x^2 / 3 - 1; },
				new real(real x) { return 0; },
				-3, 3
			));

			draw(graph(
				new real(real x) { return 0; },
				new real(real x) { return x^2 / 9 - 1; },
				new real(real x) { return x; },
				-3, 3
			));

			draw(graph(
				new real(real x) { return -3; },
				new real(real x) { return x^2 / 9 - 4; },
				new real(real x) { return x; },
				-3, 3
			));

			draw(graph(
				new real(real x) { return 3; },
				new real(real x) { return x^2 / 9 - 4; },
				new real(real x) { return x; },
				-3, 3
			));

			dot((0,-1,0), red);
		\end{asy}
		\caption{Point de selle / Point col}
	\end{subfigure}
\end{figure}

\begin{exm}
	On revient à l'exemple donné en introduction : 
	\begin{align*}
		f: \left( \R^*_+ \right)^2 &\longrightarrow \R \\
		(x,y) &\longmapsto 2\left( xy + \frac{1}{x} + \frac{1}{y} \right).
	\end{align*}

	$\left( \R^+_* \right)^2$ est un ouvert de $\R^2$. Soit $(x,y) \in \left( \R^+_* \right)^2$.
	
	On a \[
		\begin{cases}
			\frac{\partial f}{\partial x}(x,y) = 2\left( y - \frac{1}{x^2} \right),\\
			\frac{\partial f}{\partial y}(x,y) = 2\left( x - \frac{1}{y^2} \right).
		\end{cases}
	\]

	\begin{align*}
		&\frac{\partial f}{\partial x}(x,y) = \frac{\partial f}{\partial y}(x,y) = 0\\
		\iff& \begin{cases}
			y = \frac{1}{x^2}\\
			x = \frac{1}{y^2}
		\end{cases}\\
		\iff& \begin{cases}
			y = \frac{1}{x^2}\\
			x = x^4
		\end{cases}\\
		\iff& \begin{cases}
			x = 1\\
			y = 1
		\end{cases}
	\end{align*}

	On vérivie que $f$ présente en effet un minium local en $(1,1)$. \[
		f(1,1) = 6
	\] On fixe $y \in \R^+_*$ et \[
		g : x \mapsto 2\left( xy + \frac{1}{x} + \frac{1}{y} \right).
	\] Donc \[
		\forall x \in \R^+_*, g'(x) = 2\left( y - \frac{1}{x^2} \right).
	\]
	\begin{center}
		\begin{tikzpicture}
			\tkzTabInit{$x$/1,$g'(x)$/1,$g$/2.3}{$0$, $\frac{1}{\sqrt{y}}$, $+\infty$}
			\tkzTabLine{,-,z,+,}
			\tkzTabVar{+/{}, -/$2\left( 2\sqrt{y} +\frac{1}{y} \right)$, +/{}}
		\end{tikzpicture}
	\end{center}
	
	Ainsi, \[
		\forall x \in \R^+_*, \forall y \in \R^+_*, f(x,y) \ge 2\left( 2\sqrt{y} + \frac{1}{y} \right)
	\] Soit $h : y \mapsto 2\sqrt{y} + \frac{1}{y}$. On a \[
		\forall y > 0, h'(y) = \frac{1}{\sqrt{y}} - \frac{1}{y^2} = \frac{y\sqrt{y} - 1}{y^2} = \frac{y^{\frac{3}{2}} - 1}{y^2}
	\]

	\begin{center}
		\begin{tikzpicture}
			\tkzTabInit{$y$/0.7,$h'(y)$/0.7,$h$/1.4}{$0$, $1$, $+\infty$}
			\tkzTabLine{,-,z,+,}
			\tkzTabVar{+/{}, -/$3$, +/{}}
		\end{tikzpicture}
	\end{center}

	Donc, \[
		\forall x,y > 0, f(x,y) \ge 2\times 3 = 6 = f(1,1).
	\]
\end{exm}

\begin{prop}
	[règle de la chaîne]

	Soit $f : \begin{array}{rcl}
		U &\longrightarrow& \R^2 \\
		(x,y) &\longmapsto& f(x,y)
	\end{array}$ de classe $\mathcal{C}^1$ et $U, V$ deux ouverts de $\R^2$.

	Soit $\varphi : \begin{array}{rcl}
		V &\longrightarrow& U \\
		(u,v) &\longmapsto& \varphi(u,v) = \big(x(u,v), y(u,v)\big)
	\end{array}$.

	On suppose que $x$ et $y$ sont de classe $\mathcal{C}^1$ sur $V$.

	Alors,  $f \circ \varphi : \begin{array}{rcl}
		V &\longrightarrow& \R \\
		(u,v) &\longmapsto& f\big(\varphi(u,v)\big)
	\end{array}$ est de classe $\mathcal{C}^1$ et
	\begin{align*}
		\forall (u_0, v_0) \in V, \frac{\partial (f \circ \varphi)}{\partial u}(u_0, v_0)
		&= \frac{\partial f}{\partial x}\big(\varphi(u_0, v_0)\big) \times \frac{\partial x}{\partial u}(u_0, v_0)\\
		&+ \frac{\partial f}{\partial y}\big(\varphi(u_0,v_0)\big) \frac{\partial y}{\partial u}(u_0,v_0)
	\end{align*}
	\begin{align*}
		\forall (u_0, v_0) \in V, \frac{\partial (f \circ \varphi)}{\partial v}(u_0, v_0)
		&= \frac{\partial f}{\partial x}\big(\varphi(u_0, v_0)\big) \times \frac{\partial x}{\partial v}(u_0, v_0)\\
		&+ \frac{\partial f}{\partial y}\big(\varphi(u_0,v_0)\big) \frac{\partial y}{\partial v}(u_0,v_0)
	\end{align*}
\end{prop}

\begin{exm}
	[changement de coordonnées polaires]
	On pose \begin{align*}
		\varphi: \R^+_* \times ]0,2\pi[ &\longrightarrow \R^2\setminus \left( R^+_* \times \{0\} \right) \\
		(r, \theta) &\longmapsto (r \cos \theta, r \sin\theta),
	\end{align*}
	\begin{align*}
		f: \R^2\setminus \left( R^+_* \times \{0\} \right) &\longrightarrow \R \\
		(x,y) &\longmapsto f(x,y),
	\end{align*}
	\begin{align*}
		g: \overbrace{\R^+_* \times ]0, 2\pi[}^{=V} &\longrightarrow \R \\
		(r, \theta) &\longmapsto f(r\cos\theta, r\sin\theta).
	\end{align*}

	\begin{align*}
		\forall (r_0,\theta_0) \in V,&\\[5mm]
		\frac{\partial g}{\partial r}(r_0, \theta_0) &= \frac{\partial f}{\partial x}(r_0\cos\theta_0, r_0\sin\theta_0)\cos\theta_0\\
		&+ \frac{\partial f}{\partial y}(r_0 \cos\theta_0, r_0\sin\theta_0)\sin\theta_0\\
		&= 2r_0\cos^2\theta_0 + 2r_0\sin^2(\theta_0) \\
		&= 2r_0 \\[5mm]
		\frac{\partial g}{\partial \theta}(r_0, \theta_0) &= \frac{\partial f}{\partial x}(r_0\cos\theta_0, r_0\sin\theta_0)r_0\sin\theta_0\\
		&+ \frac{\partial f}{\partial y}(r_0 \cos\theta_0, r_0\sin\theta_0)r_0\cos\theta_0\\
		&= -2{r_0}^2\cos(\theta_0)\sin(\theta_0) + 2{r_0}^2 \sin(\theta_0)\cos(\theta_0)\\
		&= 0 \\
	\end{align*}

	Donc, \[
		g(r, \theta) = r^2.
	\]
\end{exm}

\begin{exm}
	Résoudre \[
		\begin{cases}
			\frac{\partial f}{\partial x} = \frac{x}{x^2+y^2},\\
			\frac{\partial f}{\partial y} = \frac{y}{x^2+y^2}.\\
		\end{cases}
	\]

	On pose $g: (r, \theta) \mapsto f(r \cos\theta, r \sin\theta)$.

	\begin{align*}
		&\frac{\partial g}{\partial r} = \frac{1}{r}\cos^2\theta + \frac{1}{r}\sin^2\theta = \frac{1}{r},\\
		&\frac{\partial g}{\partial \theta} = -\cos(\theta) \sin(\theta) + \sin(\theta)\cos(\theta) = 0.
	\end{align*}

	Donc, \[
		\exists C \in \R, g: (r, \theta) \mapsto \ln r + C
	\] d'où,
	\begin{align*}
		\forall (x,y) \in \R^2 \setminus \{(0,0)\}, f(x,y) &= \ln\left(\sqrt{x^2 + y^2} \right)  + C\\
		&= \frac{1}{2}\ln(x^2 + y^2) + C. \\
	\end{align*}
\end{exm}

\begin{rmk}
	Soit $\mathcal{B} = (e_1, e_2)$ la base canonique de $\R^2$, $f: U \to \R$ de classe $\mathcal{C}^1$ avec $U$ un ouvert de $\R^2$.

	Soit $(x,y) \in U$.

	\begin{align*}
		\Mat_{\mathcal{B}}\big(\nabla f(x,y)\big) = \begin{pmatrix}
			\frac{\partial f}{\partial x}(x,y)\\[2mm]
			\frac{\partial f}{\partial y}(x,y)
		\end{pmatrix}
	\end{align*}

	Soit  \begin{align*}
		\varphi: V &\longrightarrow U \\
		(u,v) &\longmapsto \big(x(u,v), y(u,v)\big) 
	\end{align*} avec $x,y$ de classe $\mathcal{C}^1$. Soit $g = f \circ \varphi$.
	\begin{align*}
		\Mat_{\mathcal{B}}\big(\nabla g(u,v)\big)
		&= \begin{pmatrix}
			\frac{\partial g}{\partial u}(u,v) \\[2mm]
			\frac{\partial g}{\partial v}(u,v)
		\end{pmatrix} \\
		&= \begin{pmatrix}
			\frac{\partial x}{\partial u}(u,v) \frac{\partial f}{\partial x}(x,y)
			+ \frac{\partial y}{\partial u}(u,v)\frac{\partial f}{\partial y}(x,y)\\[3mm]
			\frac{\partial x}{\partial v}(u,v) \frac{\partial f}{\partial x}(x,y)
			+ \frac{\partial y}{\partial v}(u,v) \frac{\partial f}{\partial y}(x,y)
		\end{pmatrix}  \\
		&= \underbrace{\begin{pmatrix}
				\frac{\partial x}{\partial u}(u,v)& \frac{\partial y}{\partial u}(u,v)\\[3mm]
				\frac{\partial x}{\partial v}(u,v)& \frac{\partial y}{\partial v}(u,v)
		\end{pmatrix}}_{J(u,v)} \begin{pmatrix}
			\frac{\partial f}{\partial x}(x,y)\\[3mm]
			\frac{\partial f}{\partial y}(x,y)
		\end{pmatrix} \\
		&= J(u,v) \Mat_{\mathcal{B}}\big(\nabla f(x,y)\big) \\
	\end{align*}
	où $J(u,v) = 
	\begin{pNiceArray}{c:c}
		\Mat_{\mathcal{B}}\big(\nabla x(u,v)\big) & \Mat_{\mathcal{B}}\big(\nabla y(u,v)\big)
	\end{pNiceArray}$.

	On dit que $J(u,v)$ est \underline{la jacobienne} de $\varphi$ en $(u,v)$.
	L'application linéaire canoniquement associée à $J(u,v)$ est la \underline{différentielle de $\varphi$} en $(u,v)$ noté $\mathrm{d}\varphi(u,v)$.

	On a $\mathrm{d}\varphi(u,v) \in \mathcal{L}(R^2)$ et $\Mat_{\mathcal{B}}\big(\mathrm{d}\varphi(u,v)\big) = J(u,v)$.

	Par exemple, la jacobienne du changement de coordonnées polaires est \[
		J = \begin{pmatrix}
			\frac{\partial x}{\partial r} & \frac{\partial y}{\partial r}\\[3mm]
			\frac{\partial x}{\partial \theta} & \frac{\partial y}{\partial \theta}
		\end{pmatrix}
		= \begin{pmatrix}
			\cos\theta&\sin\theta\\
			-r\sin\theta&r\cos\theta
		\end{pmatrix}.
	\]
	$\underbrace{\det(J)}_{\text{le jacobien}} = r\cos^2\theta + r\sin^2\theta = r$

	Dans une intégrale double, si $(x,y) = \varphi(u,v)$, alors $\mathrm{d}x\mathrm{d}y = \det(J)\mathrm{d}u\mathrm{d}v$.

	Ici, \[
		\mathrm{d}x\ \mathrm{d}y = r\ \mathrm{d}r\ \mathrm{d}\theta.
	\]
\end{rmk}

\begin{prv}
	On pose $(x_0, y_0) = \varphi(u_0, v_0)$. Pour tout $(h,k) \in \R^2$ tels que $(u_0 + h, v_0 + k) \in V$, en posant $g = f  \circ \varphi$.

	\begin{align*}
		g(u_0 + h, v_0 + h) &= f\big(x(u_0 + h, v_0 + k), y(u_0 + h, v_0 + k)\big) \\
		&= f\left(
			x(u_0,v_0) + h \frac{\partial x}{\partial u}(u_0,v_0) + k \frac{\partial x}{\partial v}(u_0, v_0) + \po\big(\|(h,k)\|\big), \right.\\
		&\phantom{ = f\bigg(\bigg.}\left. y(u_0, v_0) + h \frac{\partial y}{\partial u}(u_0, v_0) + k \frac{\partial y}{\partial v}(u_0, v_0) + \po\big(\|(h,k)\|\big)
		\right)  \\
		&= f(x_0,y_0) \\
		&~+ \left( h \frac{\partial x}{\partial u}(u_0,v_0) + k \frac{\partial x}{\partial v}(u_0, v_0) + \po(\|(h,k)\|) \right) \frac{\partial f}{\partial x}(x_0,y_0)\\
		&~+ \left( h \frac{\partial y}{\partial u}(u_0, v_0) + k\frac{\partial y}{\partial v}(u_0, v_0) + \po(\|(h,k)\|) \right) \frac{\partial f}{\partial y}(x_0, y_0)\\
		&~+ \po(\|(h,k)\|)\\
		&= f(x_0, y_0) \\
		&~+ h \left( \frac{\partial x}{\partial u}(u_0, v_0) \frac{\partial f}{\partial x}(x_0, y_0) + \frac{\partial y}{\partial u}(u_0, v_0) \frac{\partial f}{\partial y}(x_0, y_0) \right)  \\
		&~+ k\left( \frac{\partial x}{\partial v}(u_0, v_0) \frac{\partial f}{\partial x}(x_0, y_0) + \frac{\partial y}{\partial v}(u_0, v_0) \frac{\partial f}{\partial y}(x_0, y_0) \right) 
		&~+ \po(\|(h,k)\|)\\
		&= g(u_0, v_0) + h \frac{\partial g}{\partial u}(u_0, v_0) + k \frac{\partial g}{\partial v}(u_0, v_0) + \po(\|(h,k)\|) \\
	\end{align*}

	Par identification,
	\[
		\frac{\partial g}{\partial u}(u_0, v_0) = \frac{\partial x}{\partial u}(u_0, v_0) \frac{\partial f}{\partial x}(x_0, y_0) + \frac{\partial y}{\partial u}(u_0, v_0) \frac{\partial f}{\partial y}(x_0,y_0)
	\] et \[
		\frac{\partial g}{\partial v}(u_0, v_0) = \frac{\partial x}{\partial v}(u_0,v_0) \frac{\partial f}{\partial x}(x_0, y_0) + \frac{\partial y}{\partial v}(u_0, v_0) \frac{\partial f}{\partial y}(x_0, y_0).
	\] 
\end{prv}

\begin{exm}
	[Régression linéaire]~\\
	\begin{figure}[H]
		\centering
		\begin{asy}
			import graph;
			axes(EndArrow);
			size(5cm);

			real f(real x) { return x + 0.5; }

			real k = 35 / (7 - 0.5);

			for(int i = 0; i < 35; ++i) {
				real mag = exp(sin(100 * pi/exp(1) * i)) * 0.8 + exp(cos(i*40)/3);
				real eps = mag * cos(10 * exp(1)/pi * i) / 3;
				dot((i/k,f(i/k) + eps));
			}

			draw(graph(f, -1, 7), orange);
		\end{asy}
	\end{figure}
	\[
		y = a x + b
	\] 
	On fixe $(a,b) \in \R^2$. \[
		\varepsilon(a,b) = \sum_{i=1}^n\big( y_i - (ax_i + b) \big)^2
	\] l'erreur totale.

	On veut minimiser $\varepsilon(a,b)$. On a 
	\[
		\forall (a,b) \in \R^2,
		\begin{cases}
			\frac{\partial \varepsilon}{\partial a}(a,b) = -2\sum_{i=1}^{n}(y_i - ax_i - b)x_i,\\
			\frac{\partial \varepsilon}{\partial b}(a,b) = -2\sum_{i=1}^{n}(y_i - ax_i - b).
		\end{cases}
	\]

	Donc,
	\begin{align*}
		(a,b) \text{ point critique de } \varepsilon \iff& \begin{cases}
			a \sum_{i=1}^n {x_i}^2 + b\sum_{i=1}^{n}x_i = \sum_{i=1}^{n} y_ix_i\\
			a\sum_{i=1}^{n}x_i + nb = \sum_{i=1}^ny_i
		\end{cases}\\
		\iff& \begin{cases}
			a \left( \frac{1}{n}\sum_{i=1}^n {x_i}^2 - \overline{x}^2\right) = \overline{y} - \overline{x} \overline{y}\\
			b = \frac{1}{n}\sum_{i=1}^ny_i - \frac{a}{n}\sum_{i=1}^nx_i = \frac{1}{n}\sum_{i=1}^n x_i y_i - \overline{x} \overline{y}
		\end{cases}\\
		&\text{ où } \overline{x} = \frac{1}{n} \sum_{i=1}^n x_i,~\overline{y} = \frac{1}{n}\sum_{i=1}^n y_i\\
		\iff& \begin{cases}
			a = \frac{\Cov(x,y)}{V(x)}\\
			b = \overline{y} - a\overline{x}
		\end{cases}
	\end{align*}

	Coefficient de corrélation: $\frac{\Cov(x,y)}{\sigma_x \sigma_y} \in [-1, 1]$
\end{exm}












		\part{Corps}

\begin{exm}[Problème]
	\begin{itemize}
		\item 
			avec $A = \Z / 9 \Z$, résoudre $\overline{x}^2 = \overline{0}$ \\
			\begin{center}
				\begin{tabular}{|c|c|c|c|c|c|c|c|c|c|c|}
					\hline
					$\overline{x}$&$\overline{0}$& $\overline{1}$ &$\overline{2}$&$\overline{3}$ &$\overline{4}$ &$\overline{5}$ &$\overline{6}$ &$\overline{7}$ &$\overline{8}$& $\overline{9}$ \\
					\hline
					$\overline{x}^2$&$\overline{0}$ &$\overline{1}$ &$\overline{4}$ &$\overline{0}$ &$\overline{7}$ &$7$ &$\overline{0}$ &$\overline{4}$ &$\overline{1}$&$\overline{0}$\\
					\hline
				\end{tabular}
			\end{center}
			On a trouvé 3 solutions: $\overline{0}$, $\overline{3}$, $\overline{6}$.
		\item $\Z / 8\Z$
			\begin{center}
				\begin{tabular}{|c|c|c|c|c|c|c|c|c|}
					\hline
					$\overline{x}$& $\overline{0}$& $\overline{1}$& $\overline{2}$& $\overline{3}$& $\overline{4}$& $\overline{5}$& $\overline{6}$& $\overline{7}$\\
					\hline
					$\overline{x^2}$& $\overline{0}$& $\overline{1}$& $\overline{4}$& $\overline{1}$& $\overline{0}$& $\overline{1}$& $\overline{4}$& $\overline{1}$\\
					\hline
				\end{tabular}
			\end{center}
			$\overline{x}^2=7$ a 4 solutions: $\overline{1}, \overline{7}, \overline{3},\text{ et } \overline{5}$
		\item $A = \mathbbm{H} = \{a + bi + cj + dk  \mid  (a,b,c,d) \in \R^4\}$ \\
			$i^2 = j^2 = k^2 = -1$ 
			\begin{align*}
				\begin{array}{c c c}
					ij = k & jk = i & ji = j\\
					ji = -k & kj = -i & ik = -j
				\end{array}
			\end{align*}
			Dans cet anneau, $-1$ a 6 racines!
	\end{itemize}
\end{exm}

\begin{defn}
	Soit $(\mathbbm{K}, +, \times)$ un ensemble muni de deux lois de composition internes. On dit que c'est un \underline{corps} si
	 \begin{enumerate}
		\item $(\mathbbm{K}, \times)$ est un groupe abélien
		\item $(\mathbbm{K}, \times)$ est un monoïde commutatif
		\item $\forall x \in \mathbbm{K}\setminus \{0_\mathbbm{K}\}, \exists y \in \mathbbm{K}, xy = 1_\mathbbm{K}$
		\item $0_\mathbbm{K} \neq  1_\mathbbm{K}$
	\end{enumerate}
	\index{corps}
\end{defn}

\begin{exm}
	\begin{itemize}
		\item $(\C, +, \times)$ est un corps
		\item $(\R, +, \times)$ est un corps
		\item $(\Q, +, \times)$ est un corps
		\item $(\Z, +, \times)$ n'est pas un corps
	\end{itemize}
\end{exm}

\begin{prop}
	$(\Z / n\Z, +, \times)$ est un corps si et seulement si $n$ est premier.
\end{prop}

\begin{prv}
	\[
		\left( \Z / n\Z \right)^\times = \left\{ \overline{k}  \mid k \wedge n = 1 \right\}
	\] 
\end{prv}


\begin{prop}
	Tout corps est un anneau intègre.
\end{prop}

\begin{prv}
	Soit $(\mathbbm{K}, +, \times)$ un corps. Soient $(a,b) \in \mathbbm{K}^2$ tel que $a \times b = 0_\mathbbm{K}$.\\
	On suppose $a \neq  0_\mathbbm{K}$. Alors, $a$ est inversible et donc \[
		b = a^{-1} \times a \times b = a^{-1} \times 0_\mathbbm{K} = 0_\mathbbm{K}
	\] 
\end{prv}

\begin{exm}
	Soit $(\mathbbm{K},+,\times)$ un corps.\\
	Résoudre \[
		\begin{cases}
			x^2 = 1_\mathbbm{K}\\
			x \in \mathbbm{K}
		\end{cases}
	\]

	\begin{align*}
		x^2 = 1_\mathbbm{K} &\iff x^2 - 1_\mathbbm{K} = 0_\mathbbm{K}\\
		&\iff (x - 1_\mathbbm{K})(x+1_\mathbbm{K}) = 0_\mathbbm{K}\\
		&\iff x - 1_\mathbbm{K} = 0_\mathbbm{K} \text{ ou } x + 1_\mathbbm{K} = 0_\mathbbm{K}\\
		&\iff x = 1_\mathbbm{K} \text{ ou } x = -1_\mathbbm{K}
	\end{align*}

	Il y a au plus 2 solutions.
\end{exm}

\begin{prop}
	Soit $(\mathbbm{K},+,\times )$ un corps et $P$ un polynôme à coefficients dans $\mathbbm{K}$ de degré $n$. Alors, l'équation $P(x) = 0_{\mathbbm{K}}$ a au plus $n$ solutions dans $\mathbbm{K}$ 
	\qed
\end{prop}

\begin{crlr}[(Théorème de Wilson)]
	voir exercice 16 du TD 12
\end{crlr}


\begin{defn}
	Soit $(\mathbbm{K}, +, \times)$ un corps et $L\subset \mathbbm{K}$.\\
	On dit que $L$ est un \underline{sous corps} de $\mathbbm{K}$ si
	\begin{enumerate}
		\item $L$ est un anneau de $(\mathbbm{K}, +, \times)$ non nul
		\item $\forall x \in L\setminus \{0_\mathbbm{K}\}, x^{-1} \in L$ 
	\end{enumerate}
	\vspace{2mm}
	en d'autres termes si
	\begin{enumerate}
		\item $\forall (x,y) \in L^2, x - y \in L$
		\item $\forall (x,y) \in L^2, x \times y^{-1} \in L$
	\end{enumerate}
	\vspace{5mm}
	On dit aussi que $\mathbbm{K}$ est une \underline{extension} de $L$.
	\index{sous corps}
	\index{extension}
\end{defn}

\begin{prop}
	Tout sous corps est un corps. \qed
\end{prop}

\begin{defn}
	Soient $(\mathbbm{K}_1,+,\times )$ et $(\mathbbm{K}_2,+, \times)$ deux corps et $f: \mathbbm{K}_1 \to \mathbbm{K}_2$.\\
	On dit que $f$ est un \underline{morphisme de corps} si $f$ est un morphisme d'anneaux.\\
	i.e. si
	\[
		\begin{cases}
			\forall (x,y) \in {\mathbbm{K}_1}^2,& f(x+y) = f(x) + f(y)\\
			\forall (x,y) \in {\mathbbm{K}_1}^2,& f(x \times y) = f(x) \times f(y)\\
		\end{cases}
	\] 
	\index{homomorphisme (de corps)}
	\index{morphisme (de corps)}
\end{defn}

\begin{prop}
	Tout morphisme de corps est injectif.
\end{prop}

\begin{prv}
	Soit $f: \mathbbm{K}_1 \to \mathbbm{K}_2$ un morphisme de corps.\\
	\begin{itemize}
		\item $\Ker(f)$ est un sous groupe de $(\mathbbm{K}_1, +)$ 
		\item Soit $x \in \Ker(f)$ et $y \in \mathbbm{K}_1$ \[
				f(x \times y) = f(x) \times f(y) = 0_{\mathbbm{K}_2} \times f(y) = 0_{\mathbbm{K}_2}
			\]
		\item Soit $x \in \Ker(f) \setminus \{0_{\mathbbm{K}_1}\}$.\\
			Alors, $x$ est inversible.\\
			\begin{align*}
				\begin{rcases*}
					x \in \Ker(f)\\
					x^{-1} \in \mathbbm{K}_1
				\end{rcases*}& \text{ donc } x \times x ^{-1} \in \Ker(f)\\
				&\text{ donc } 1_{\mathbbm{K}_1} \in \Ker(f)\\
				&\text{ donc } f(1_{\mathbbm{K}_1}) = 0_{\mathbbm{K}_2}
			\end{align*}
			Or, $f(1_{\mathbbm{K}_1}) = 1_{\mathbbm{K}_2} \neq 0_{\mathbbm{K}_2}$
	\end{itemize}
	Donc, $\Ker(f) = \{0_{\mathbbm{K}_1}\}$ donc $f$ est injective.
\end{prv}

\begin{exm}
	$\begin{array}{cc}
		\C &\longrightarrow \C\\
		z &\longmapsto \overline{z}\\
	\end{array}$ est un morphisme de corps
\end{exm}



		\part{Opérations sur les séries}

\begin{prop}
	L'ensemble $E = \{u \in \C^\N  \mid \Sigma u_n \text{ converge}\}$ est un sous-espace vectoriel de $\C^\N$ et \begin{align*}
		S: E &\longrightarrow \C \\
		u &\longmapsto \sum_{n=0}^{+\infty} u_n
	\end{align*} est une forme linéaire.
	\qed
\end{prop}

\begin{rmk}
	La somme d'une série convergente et d'une série divergente diverge.
	Le produit d'une série divergente par un scalaire non nul diverge.
\end{rmk}

		\part{Comparaison de suites}

\begin{defn}
	Soient $u$ et $v$ deux suites réelles. On dit que $u$ est \underline{dominée} par  $v$ si \[
	\exists M\in \R, \exists N\in \N,\forall n\ge N,\left| u_n \right| \le M \left| v_n \right| 
	\] Dans ce cas, on note $u = O(v)$ ou $u_n = O(v_n)$ et on dit que "$u$ est un grand o de $v$"
\end{defn}

\begin{exm}
	En informatique, on dit qu'un alogirithme a une \underline{complexité linéaire} si son temps d'éxécution est un $O(n)$ 
	Par exemple, on calcule $a^n$ 

	\begin{itemize}
		\item Approche naïve
			\begin{algorithm}
				\begin{algorithmic}[1]
					\State $p \gets 1$
					\For{$i \in \left\llbracket 0,n-1 \right\rrbracket$}
						\State $p \gets p \times a$
					\EndFor
					\State \Return p
				\end{algorithmic}
			\end{algorithm}
			Complexité linéaire $O(n)$
		\item Exponentiation rapide\\
			On écrit $n$ en binaire: \begin{align*}
				n &= \overline{a_k a_{k-1}\ldots a_0}^{(2)}\\
					&= \sum_{i=0}^{k} a_i 2^i
			\end{align*} avec $(a_i) \in \left\{ 0,1 \right\} ^{k+1}$
			\begin{align*}
				a^n &= a^{\sum_{i=0}^{k} a_i 2^i} \\
				&= \prod_{i=0}^{k} a^{a_i 2^i}  \\
			\end{align*}
			
			\begin{algorithm}
				\begin{algorithmic}
					[1]

					\State $s \gets 0$
					\State $p \gets a$
					\For{ $i \in \left\llbracket 0, \log_2(n) \right\rrbracket$}
						\State $p \gets p \times p$
						\If{$a[i] = 1$}
							\State $s \gets s + p$
						\EndIf
					\EndFor
					\State \Return s
				\end{algorithmic}
			\end{algorithm}
			Compléxité logarithmique $O(\log_2(n))$
	\end{itemize}
\end{exm}


\begin{prop}
	$O$ est une relation réfléctive et transitive.
\end{prop}

\begin{prv}
	\begin{itemize}
		\item Soit $u$ une suite. On pose $M = 1$ et \[
			\forall n \in \N, \left| u_n \right| \le M \left| u_n \right|
			\] Donc $u = O(u)$.
		\item Soient $u, v, w$ trois suites telles que  \[
		\begin{cases}
			u = O(v)\\
			v = O(w)
		\end{cases}
		\] Soient $M_1,M_2 \in \R$ et $N_1,N_2\in \N$ tels que \[
		\begin{cases}
			\forall n \ge  N_1, \left| u_n \right| \le M_1 \left| v_n \right| \\
			\forall n \ge  N_2, \left| v_n \right| \le M_2 \left| w_n \right| \\
		\end{cases}
		\] 

		Nécéssairement, $M_1\ge 0$ et $M_2\ge 0$.\\
		Soit $N = \max(N_1,N_2)$. \[
		\forall n \ge  N, \left| u_n \right| \le M_1 \left| v_n \right| \le  M_1M_2 \left| w_n \right| 
		\] Donc $u = O(w)$
	\end{itemize}
\end{prv}

\begin{defn}
	Soient $u$ et $v$ deux suites. On dit que $u$ est \underline{négligeable} devant $v$ si \[
	\forall \varepsilon>0, \exists N\in \N, \forall n\ge N, \left| u_n \right| \le \varepsilon \left| v_n \right| 
	\] Dans ce cas, on note $u = o(v)$ ou $u_n = o(v_n)$ ou on le lit "$u$ est un petit o de $v$"
\end{defn}

\begin{prop}
	$o$ est une relation transitive, non-réfléctive
\end{prop}

\begin{prv}
	\begin{itemize}
		\item Soient $u$, $v$ et $w$ trois suites telles que \[
			\begin{cases}
				u = o(v)\\
				v = o(w)
			\end{cases}
			\] Soit $\varepsilon>0$. Soit $N_1\in \N$ tel que \[
			\forall n \ge N_1, \left| u_n \right| \le \sqrt{\varepsilon}  \left| v_n \right| 
			\] Soit $N_2\in \N$ tel que \[
			\forall n \ge N_2, \left| v_n \right| \le \sqrt{\varepsilon}  \left| w_n \right| 
			\] On pose $N = \max(N_1,N_2)$, alors \[
			\forall n \ge N, \left| u_n \right| \le \sqrt{\varepsilon}  \left| v_n \right| \le \underbrace{\sqrt{\varepsilon} \times \sqrt{\varepsilon}} _\varepsilon \left| w_n \right| 
			\] donc $u = o(w)$
		\item Soit $u$ une suite tel qu'il existe $N \in \N$ tel que \[
		\forall n \ge N, u_n > 0
		\] On suppose que $u = o(u)$, alors \[
		\forall \varepsilon>0,\exists N \in \N, \forall n \ge N, \left| u_n \right| \le \varepsilon \left| u_n \right| 
		\] On pose $\varepsilon = \frac{1}{2}$ alors \[
		\exists N \in \N, \forall n \ge N, \left| u_n \right| \le \frac{1}{2} \left| u_n \right| 
		\] une contradiction
	\end{itemize}
\end{prv}

\begin{prop}
	Soient $u$ et $v$ deux suites.
	\begin{itemize}
		\item $o(u) + o(u) = o(u)$
		\item $v \times o(u) = o(uv)$
		\item $o(u) \times o(v) = o(uv)$
		\item $o(o(u)) = o(u)$
	\end{itemize}
	\qed
\end{prop}

\begin{defn}
	Soient $u$ et $v$ deux suites. On dit que $u$ et $v$ sont \underline{équivalentes} si \[
	u = v + o(v)
	\] i.e. \[
	\forall \varepsilon >0, \exists N \in \N, \forall n \ge N, \left| u_n-v_n \right| \le \varepsilon\left| v_n \right| 
	\] Dans ce cas, on le note $u \sim v$
\end{defn}

\begin{prop}
	$\sim$ est une relation d'équivalence \qed
\end{prop}

\begin{prop}
	Soient $(u,v) \in \R^\N$. On suppose que $v$ ne s'annule pas à partir d'un certain rang
	\begin{enumerate}
		\item $u = o(v) \iff \left( \frac{u_n}{v_n} \right)$ bornée
		\item $u = o(v) \iff \frac{u_n}{v_n} \tendsto{n \to  +\infty} 0$
		\item $u \sim v \iff \frac{u_n}{v_n} \tendsto{n \to  +\infty} 1$
	\end{enumerate}
	\qed
\end{prop}

\begin{prop}
	[Suites de références]
	\begin{enumerate}
		\item $\ln^\alpha(n) = o(n^\beta)$ avec $(\alpha,\beta) \in \left( \R^+_* \right) ^2$ 
		\item $n^\beta = o(a^n)$ avec $\beta > 0$ et $a > 1$ 
		\item $a^n = o(n!)$ avec $a >1$ 
		\item $n! = o(n^n)$
	\end{enumerate}
\end{prop}


\begin{lem}
	[Exercice 10 du TD]
	Soit $u \in \left(\R^+_*\right)^\N$\\
	Si $\frac{u_{n+1}}{u_n} \tendsto{n \to +\infty} \ell < 1$ avec $\ell\in \R$,\\ alors $u_n \tendsto{n \to +\infty} 0$
\end{lem}

\begin{prv} [de la proposition]
	\begin{enumerate}
		\item par croissance comparée
		\item On pose $\forall n \in \N^*, u_n = \frac{n^\beta}{a^n}$. 
			\begin{align*}
				\forall  n \in \N^*, \frac{u_{n+1}}{u_n} &= \left( \frac{n+1}{n} \right) ^\beta \times \frac{1}{a} \\
				&= \frac{1}{a}\left( 1+\frac{1}{n} \right) ^\beta \\
				&\tendsto{n \to +\infty} \frac{1}{a} < 1
			\end{align*}
			Donc, $u_n \tendsto{n \to  +\infty} 0$
		\item On pose $\forall n \in \N, u_n = \frac{a^n}{n!}$ \[
			\forall n \in \N, \frac{u_{n+1}}{u_n} = \frac{a}{n+1} \tendsto{n \to +\infty} 0 < 1
			\] donc $u_n \tendsto{n \to +\infty} 0$
		\item On pose $\forall  n\in \N^*, u_n = \frac{n!}{n^n}$.
			\begin{align*}
				\forall n \in \N^*, \frac{u_{n+1}}{u_n}
				&= (n+1) {\frac{n^n}{(n+1)^{n+1}}} \\
				&= \left( \frac{n}{n+1} \right) ^n \\
				&= e^{n \ln\left( \frac{n}{n+1} \right) } \\
				&= e^{n \ln\left( 1+\frac{1}{n+1} \right)} \\
				&= e^{n(-\frac{1}{n} + o(\frac{1}{n})} \\
				&= e^{-1 + o(1)} \\
				&\tendsto{n \to  +\infty} e^{-1}<1
			\end{align*}
			donc $u_n \tendsto{n\to +\infty} 0$
	\end{enumerate}
\end{prv}

		\part{Matrices par blocs}

\begin{exm}
	Soit $p$ un projecteur de $E$ : \[
		E = \Ker p \oplus \mathrm{Im}\ p
	\] Soit $\mathcal{B} = (e_1, \ldots, e_k, e_{k+1}, \ldots, e_n)$ une base de $E$ avec $\begin{cases}
		\mathrm{Im}(p) = \Vect(e_1, \ldots, e_k)\\
		\Ker(p) = \Vect(e_{k+1}, \ldots, e_n)\\
	\end{cases}$

	Alors, 
	\begin{align*}
		\Mat_\mathcal{B}(p) =
		\left(\begin{NiceArray}{c c c | c c c}
				1&&&0&\Cdots&0\\
				 &\Ddots&&\Vdots&&\Vdots\\
				&&1&0&\Cdots&0\\\hline
				0&\Cdots&0&0&\Cdots&0\\
				\Vdots&&\Vdots&\Vdots&&\Vdots\\
				0&\Cdots&0&0&\Cdots&0\\
		\end{NiceArray}\right)
		= \left( \begin{array}{c|c}
				I_k & 0\\ \hline
				0&0
		\end{array}\right) \\
	\end{align*}

	De même, si $\s$ est une symétrie de $E$, \[
		E = \Ker(\s - \id_E) \oplus \Ker(\s + \id_E)
	.\] Soit $\mathcal{C} = (e_1', \ldots, e_\ell', e_{\ell+1}', \ldots, e'_n)$ avec $\begin{cases}
		\Vect(e'_1, \ldots, e'_\ell) = \Ker(\s - \id_E),\\
		\Vect(e'_{\ell+1}, \ldots, e'_n) = \Ker(\s + \id_E).\\
	\end{cases}$

	Alors
	\[
		\Mat_\mathcal{C}(\s) = \left(\begin{array}{c|c}
				I_\ell &0\\ \hline
				0&-I_{n-\ell}
		\end{array}\right) 
	\]
\end{exm}

\begin{prop}
	Soient $F$ et $G$ supplémentaires dans $E$ : \[
		E = F \oplus G.
	\] Soit $f \in \mathcal{L}(F)$ et $g \in \mathcal{L}(G)$. Alors \[
	\exists !h \in \mathcal{L}(E) h_{|F} = f,\ h_{|G} = g \et h = f \circ p + g \circ q
	\] où $\begin{cases}
		p \text{ est la projection sur $F$ parallèlement à $G$}\\
		q \text{ est la projection sur $G$ parallèlement à $F$}\\
	\end{cases}$.

	On a aussi $q = \id_E - p$.
\end{prop}

\begin{prv}
	\begin{itemize}
		\item[\sc \underline{Analyse}] Soit $h \in \mathcal{L}(E)$ tel que $\begin{cases}
				h_{|F}=f\\
				h_{|G}=g
			\end{cases}$.

			Soit $x \in E$. Alors \[
				x = \underbrace{p(x)}_{\in F} + \underbrace{q(x)}_{\in G}
			\]

			Donc,
			\begin{align*}
				h(x) &= h\big(p(x)\big) + h\big(q(x)\big)\\
				&= f\big(p(x)\big) + g\big(q(x)\big) \\
				&= (f \circ p + g \circ q)(x) \\
			\end{align*}
			Si $h$ existe, alors \[
				h = f \circ p + g \circ q
			\]
		\item[\underline{\sc Synthèse}] On pose $h = f \circ p + g  \circ q$.

			$p$, $q$, $f$ et $g$ sont linéaires donc $h$ aussi.

			Soit $x \in E$.
			\begin{align*}
				h(x) &= f\big(p(x)\big) + g\big(q(x)\big) \\
				&= f(x) + g(0_E) \\
				&= f(x) \\
			\end{align*}
			donc $h_{|F} = f$ et de même $h_{|G}=g$.
	\end{itemize}
\end{prv}

\begin{prop}
	On reprend les notations et hypothèses précédentes. Soit $(e_1, \ldots, e_p)$ une base de $F$, et $(f_1, \ldots, f_q)$ une base de $G$. Alors, $\mathcal{B} = (e_1, \ldots, e_p, f_1, \ldots, f_q)$ est une base de $E$ et \[
		\Mat_\mathcal{B}(h) = \left(
		\begin{array}{c|c}
			A&0\\ \hline
			0&B
		\end{array}\right)
	\] où $\begin{cases}
		A = \Mat_{(e_1, \ldots e_p)}(f)\\
		B = \Mat_{(f_1, \ldots, f_q)}(g)
	\end{cases}$
	\qed
\end{prop}

\begin{prop}
	Soient $(A,A') \in \mathcal{M}_n(\mathbbm{K})^2$ et $(B,B') \in \mathcal{M}_p(\mathbbm{K})^2$.
	\begin{enumerate}
		\item \[
				\left(\begin{array}{c|c}
					A&0\\ \hline
					0&B
				\end{array}\right)
				\left(\begin{array}{c|c}
					A'&0\\ \hline
					0&B'
				\end{array}\right) = 
				\left(\begin{array}{c|c}
					AA'&0\\ \hline
					0&BB'
				\end{array}\right)
			\]
		\item \[
				\left(\begin{array}{c|c}
					A&0\\ \hline
					0&B
				\end{array}\right) \in \mathrm{GL}_{n+p}(\mathbbm{K})	 \iff \begin{cases}
					 A \in \mathrm{GL}_n(\mathbbm{K})\\
					 B \in \mathrm{GL}_p(\mathbbm{K})
				\end{cases}
			\] et dans ce cas, \[
				\left(\begin{array}{c|c}
					A&0\\ \hline
					0&B
				\end{array}\right)^{-1} =
				\left(\begin{array}{c|c}
					A^{-1}&0\\ \hline
					0&B^{-1}
				\end{array}\right)
			\]
		\item \[
				\tr \left(\begin{array}{c|c}
					A&0\\ \hline
					0&B
				\end{array}\right) = \tr A + \tr B
			\]
	\end{enumerate}
\end{prop}

\begin{prv}
	\begin{enumerate}
		\item Soit $\begin{cases}
				f \in \mathcal{L}(F) \text{ tel que } \Mat_\mathcal{B}(f) = A,
				f' \in \mathcal{L}(F) \text{ tel que } \Mat_\mathcal{B}(f') = A',
				g \in \mathcal{L}(G) \text{ tel que } \Mat_\mathcal{C}(g) = B,
				g' \in \mathcal{L}(G) \text{ tel que } \Mat_\mathcal{C}(g') = B'
			\end{cases}$ où $\begin{cases}
				F \oplus G = \mathbbm{K}^{n+p},\\
				\dim(F) = n, \dim(G) = p,\\
				\mathcal{B} \text{ base de } F,\\
				\mathcal{C} \text{ base de } G.\\
			\end{cases}$
			Soit $\begin{cases}
				h \in \mathcal{L}(\mathbbm{K}^{n+p}) \text{ tel que } \begin{cases}
					h_{|F} = f\\
					h_{|G} = g
				\end{cases}\\
				h' \in \mathcal{L}(\mathbbm{K}^{n+p}) \text{ tel que } \begin{cases}
					h'_{|F} = f'\\
					h'_{|G} = g'\\
				\end{cases}
			\end{cases}$
			Soit $\mathcal{D} = \mathcal{B} \cup \mathcal{C}$ une base de $\mathbbm{K}^{n+p}$.
			\begin{align*}
				\left(\begin{array}{c|c}
					A&0\\ \hline
					0&B
				\end{array}\right)
				\left(\begin{array}{c|c}
					A'&0\\ \hline
					0&B'
				\end{array}\right) &= \Mat_{\mathcal{D}}(h) \Mat_{\mathcal{D}}(h')\\
				&= \Mat_{\mathcal{D}}(h \circ h') \\
			\end{align*}
			Or, $(h \circ h')_{|F} = f \circ f'$ et $(h \circ h')_{|G} = g \circ g'$.

			Donc,
			\begin{align*}
				\Mat_\mathcal{D}(h \circ h') &=
					\left(\begin{array}{c|c}
						\Mat_\mathcal{B}(f \circ f')&0\\ \hline
						0&\Mat_\mathcal{C}(g \circ g')
					\end{array}\right)\\
				&=\left(\begin{array}{c|c}
					AA'&0\\ \hline
					0&BB'
				\end{array}\right).
			\end{align*}
	\end{enumerate}
\end{prv}

\begin{prop}
	Soient $A,A' \in \mathcal{M}_n(\mathbbm{K})$, $B,B' \in \mathcal{M}_{n,p}(\mathbbm{K})$, $C,C' \in \mathcal{M}_{p,n}(\mathbbm{K})$ et $D, D' \in \mathcal{M}_p(\mathbbm{K})$.

	\[
		\left(\begin{array}{c|c}
			A&B\\ \hline
			C&D
		\end{array}\right)
		\left(\begin{array}{c|c}
			A'&B'\\ \hline
			C'&D'
		\end{array}\right) = 
		\left(\begin{array}{c|c}
			AA' + BC'& AB' + BD'\\ \hline
			CA' + DC'&CB' + DD'
		\end{array}\right)
	\] Cette formule se généralise à un nombre quelconque de blocs : \[
		\left(\begin{array}{c|c|c|c}
				A_{11}&A_{12}&\cdots&A_{1,n}\\ \hline
				A_{21}&A_{22}&\cdots&A_{2,n}\\ \hline
				\vdots&\vdots&\ddots&\vdots\\ \hline
				A_{p,1}&A_{p,2}&\cdots&A_{p,n}
		\end{array}\right)
		\left(\begin{array}{c|c|c|c}
				A'_{11}&A'_{12}&\cdots&A'_{1,n}\\ \hline
				A'_{21}&A'_{22}&\cdots&A'_{2,n}\\ \hline
				\vdots&\vdots&\ddots&\vdots\\ \hline
				A'_{p,1}&A'_{p,2}&\cdots&A'_{p,n}
		\end{array}\right)
	\] Cette matrice se calcyle comme on s'y attend si les dimensions des blocs autorisent les produits.
\end{prop}

\begin{prop}
	Le rang d'une matrice $A$, c'est la taille de la plus grande matrice carrée inversible que l'on peut extraire de $A$.
	\qed
\end{prop}




		\part{Trigonométrie hyperbolique}

\begin{defn}
	Pour tout $x \in \R$, on pose \[
		\begin{cases}
			\ch x = \frac{e^x + e^{-x}}{2},\\
			\sh x = \frac{e^x - e^{-x}}{2},\\
			\th x = \frac{\sh x}{\ch x}.
		\end{cases}
	\]

	$\ch$ est appelé \underline{cosinus hyperbolique}, $\sh$ est appelé \underline{sinus hyperbolique} et $\th$ est appelé \underline{tangeante hyperbolique}.
	\index{cosinus hyperbolique}
	\index{sinus hyperbolique}
	\index{tangente hyperbolique}
\end{defn}

\begin{rmk}
	Ces formules rappèlent les formules d'Euler : pour tout $x \in \R$,
	\begin{align*}
		\cos x = \frac{e^{ix} + e^{-ix}}{2}\quad\longleftrightarrow\quad\ch x = \frac{e^x + e^{-x}}{2}\\
		\sin x = \frac{e^{ix} - e^{-ix}}{2i}\quad\longleftrightarrow\quad\sh x = \frac{e^x - e^{-x}}{2}\\
	\end{align*}
\end{rmk}

\begin{figure}[H]
	\centering
	\begin{asy}
		import graph;

		size(12cm);

		pair A = (-2, 0);
		pair B = (2, 0);

		real eps = 0.05;

		draw(shift(A) * ((0, -1.3) -- (0, 1.3)), Arrow(TeXHead));
		draw(shift(A) * ((-1.3, 0) -- (1.3, 0)), Arrow(TeXHead));

		draw(circle(A, 1), magenta);
		
		real theta = 38;
		pair M = dir(theta) + A;
		draw(A -- M, red);
		draw(arc(A, 0.35, 0, theta), red, Arrow(TeXHead));
		draw(M -- (A.x-eps, M.y), dashed);
		draw(M -- (M.x, A.y-eps), dashed);
		label("\small$\theta$", 0.5dir(theta/2) + A, red);
		label("\small$\cos\theta$", (M.x, A.y), align=S);
		label("\small$\sin\theta$", (A.x, M.y), align=1.2W);
		dot("\small$M$", M);

		label("\small$x^2 + y^2 = 1$", A + 1.5dir(45+180));

		draw(shift(B) * ((0, -1.3) -- (0, 1.3)), Arrow(TeXHead));
		draw(shift(B) * ((-1.3, 0) -- (1.3, 0)), Arrow(TeXHead));

		real ch(real x) { return (exp(x) + exp(-x)) / 2.; }
		real sh(real x) { return (exp(x) - exp(-x)) / 2.; }
		real nch(real x) { return -ch(x); }

		real k = 1.9; real r = 1.2;
		real t = 1.4;

		draw(shift(B) * scale(0.35) * graph(ch, sh, -k, k), magenta);
		draw(shift(B) * scale(0.35) * graph(nch, sh, -k, k), magenta);

		label("\small$x^2 - y^2 = 1$", B + 1.5dir(45+180) + (0, -0.2));

		M = B + 0.35(ch(t), sh(t));

		draw(M -- (B.x-eps, M.y), dashed);
		draw(M -- (M.x, B.y-eps), dashed);
		dot("\small$M$", M);
		label("\small$\ch x$", (M.x, B.y), align=S);
		label("\small$\sh x$", (B.x, M.y), align=1.2W);

		draw(shift(B) * ((-r, -r)--(r,r)), gray + dashed);
		draw(shift(B) * ((r, -r)--(-r,r)), gray + dashed);
	\end{asy}
\end{figure}


	}

	{
		\chap[12]{Structures algébriques usuelles}
		\renewcommand{\cwd}{../chap12}
		\begin{defn}
	Soit $E$ un $\mathbbm{K}$-espace vectoriel. On dit que $E$ est de \underline{dimension finie} si $E$ a au moins une famille génératrice finie. On dit que $E$ est de \underline{dimension infinie} sinon.
	\index{dimension finie (espace vectoriel)}
	\index{dimension infinie (espace vectoriel)}
\end{defn}

\begin{thm}
	[Théorème de la base extraite]
	Soit $E$ un $\mathbbm{K}$-espace vectoriel non nul de dimension finie. Soit $\mathcal{G}$ une famille génératrice finie de $E$. Alors, il existe une base $\mathcal{B}$ de $\mathcal{E}$ telle que $\mathcal{B} \subset \mathcal{G}$.
\end{thm}

\begin{prv}
	[par récurrence sur $\#G = \Card(G)$]
	\begin{itemize}
		\item Soit $E$ un $\mathbbm{K}$-espace vectoriel non nul engendré par $\mathcal{G} = (u)$.\\
			Si $u = 0_E$, alors $E = \{0_E\}$: une contradiction $\lightning$ \\
			Donc $u \neq 0_E$ donc $(u)$ est libre. En effet, \[
				\forall \lambda \in \mathbbm{K}, \lambda u = 0_E \implies \lambda = 0_\mathbbm{K}
			\] Donc $\mathcal{G}$ est une base de $E$.\\
		\item Soit $n \in \N_*$. Soit $E$ un $\mathbbm{K}$-espace vectoriel. On suppose que si $E$ a une famille génératrice constituée de $n$ vecteurs, alors on peut extraire de cette famille une base de $E$.\\
			Soit $\mathcal{G}$ une famille génératrice de $E$ avec $n+1$ vecteurs.\\
			Si $\mathcal{G}$ est libre, alors $\mathcal{G}$ est une base de $E$. \\
			Si $\mathcal{G}$ n'est pas libre, alors il existe $u \in \mathcal{G}$ tel que $u \in \Vect(\mathcal{G}\setminus \{u\})$ \\
			Donc $\mathcal{G}\setminus \{u\}$ engendre $E$. Or, $\mathcal{G}\setminus \{u\}$ possède $n$ vecteurs. D'après l'hypothèse de récurrence, il existe une base $\mathcal{B}$ de $E$ telle que \[
				\mathcal{B} \subset \mathcal{G} \setminus \{u\} \subset \mathcal{G}
			\] 
	\end{itemize}
\end{prv}

\begin{crlr}
	Tout espace de dimension finie a une base.
	\qed
\end{crlr}

\begin{thm}
	[Théorème de la base incomplète]
	Soit $E$ un $\mathbbm{K}$-espace vectoriel de dimension finie, $\mathcal{G}$ une famille génératrice finie de $E$. $\mathcal{L}$ une famille libre de $E$. Alors, il existe une base $\mathcal{B}$ de $E$ telle que \[
		\mathcal{L} \subset \mathcal{B} \text{ et } \mathcal{B}\setminus \mathcal{L} \subset \mathcal{G}
	\] 
\end{thm}

\begin{prv}
	[par récurrence sur $\#(\mathcal{G}\setminus\mathcal{L})$]
	\begin{itemize}
		\item Avec les notations précédentes, on suppose que $\mathcal{G}\setminus\mathcal{L} \neq \O$ \[
				\forall u \in \mathcal{G}, u \in \mathcal{L}
			\] Donc $\mathcal{G} \subset \mathcal{L}$ donc $\mathcal{L}$ est génératrice donc $\mathcal{L}$ est une base de $E$. On pose $\mathcal{B} = \mathcal{L}$ et alors \[
				\mathcal{L} \subset  \mathcal{B} \text{ et } \mathcal{B}\setminus\mathcal{L} = \O \subset  \mathcal{G}
			\] 
		\item Soit $n \in \N$. On suppose que si $\mathcal{G}$ est génératrice et $\mathcal{L}$ libre avec $\#(\mathcal{G}\setminus\mathcal{L}) = n$ alors il existe une base $\mathcal{B}$ de $E$ telle que \[
			\mathcal{L}\subset \mathcal{B} \text{ et } \mathcal{B}\setminus\mathcal{L}\subset \mathcal{G}
		\] Soient à présent $\mathcal{G}$ une famille génératrice de $E$ et $\mathcal{L}$ une famille libre de $E$ telles que $\#(\mathcal{G}\setminus\mathcal{L}) = n+1 > 0$\\
		Si $\mathcal{L}$ engendre $E$, alors $\mathcal{L}$ est une base de $E$. On pose $\mathcal{B} = \mathcal{L}$ et on a bien \[
			\mathcal{L} \subset  \mathcal{B} \text{ et } \mathcal{B} \setminus \mathcal{L} = \O \subset  \mathcal{G}
		\] On suppose que $\mathcal{L}$ n'engendre pas $E$. Il existe $u \in \mathcal{G}$ tel que $u \not\in \Vec(\mathcal{L})$ (car sinon, $\mathcal{G} \subset \Vect(\mathcal{L})$ et donc $\underbrace{\Vect(\mathcal{G})}_{= E} \subset  \underbrace{\Vect(\mathcal{L})}_{ \subset E}$\\
		Donc $\mathcal{L} \cup \{u\} $ est libre. On pose $\mathcal{L}' = \mathcal{L} \cup \{u\} $ \[
			\mathcal{G}\setminus \mathcal{L}' = \mathcal{G}\setminus (\mathcal{L} \cup \{u\}) = (\mathcal{G}\setminus\mathcal{L})\setminus \{u\} 
		\] donc $\#(\mathcal{G}\setminus\mathcal{L}') = n+1 -1 = n$\\
		D'après l'hypothèse de récurrence, il existe $\mathcal{B}$ une base de $E$ telle que \[
			\mathcal{L} \subset  \mathcal{L}' \subset \mathcal{B} \text{ et } \mathcal{B}\setminus \mathcal{L}' \subset \mathcal{G}
		\] \[
			\mathcal{B} \setminus \mathcal{L} = \underbrace{\mathcal{B}\setminus\mathcal{L}'}_{\subset \mathcal{G}} \cup \underbrace{\{u\}}_{\subset \mathcal{G} \text{ car } u \in \mathcal{G}}
		\] On a $\mathcal{B}\setminus\mathcal{L}\subset \mathcal{G}$
	\end{itemize}
\end{prv}

\begin{thm}
	Soit $E$ un $\mathbbm{K}$-espace vectoriel de dimension finie. Toutes les bases de $E$ ont le même cardinal.
\end{thm}

\begin{prv}
	Soit $\mathcal{G}$ une famille génératrice finie de $E$ et $\mathcal{B} \subset  \mathcal{G}$ une base de $E$. On note $n = \#\mathcal{B}$ \\
	Soit $\mathcal{B}'$ une base de $E$. On pose $p = n - \#(\mathcal{B} \cap  \mathcal{B}')$. Montrons par récurrence sur  $p$ que $\#\mathcal{B} = \#\mathcal{B}'$ 
	\begin{itemize}
		\item On suppose que $p = 0$. Alors, $\#(\mathcal{B} \cap \mathcal{B}') = n$ \\
			Or, $\mathcal{B}' \cap \mathcal{B} \subset \mathcal{B}$ donc $\mathcal{B} \cap \mathcal{B}' = \mathcal{B}$ donc $\mathcal{B} \subset  \mathcal{B}'$ et donc $\mathcal{B} = \mathcal{B}'$ 
		\item Soit $p \in \N$. On suppose que si $\mathcal{B}'$ est une base de $E$ telle que $n - \#(\mathcal{B} \cap \mathcal{B}') = p$, alors $\#\mathcal{B}' = n$ \\
			Aoit $\mathcal{B}'$ une base de $E$ telle que $n - \#(\mathcal{B}\cap \mathcal{B}') = p+1 > 0$ \\
			Donc $\mathcal{B} \cap \mathcal{B}' \neq \mathcal{B}$. Soit $u \in \mathcal{B}' \setminus \mathcal{B}$. D'après le lemme d'échange, il existe $v \in \mathcal{B}\setminus \mathcal{B}'$ tel que $\mathcal{B}' \setminus \{u\} \cup \{v\}$ est une base de $E$. On pose $\mathcal{B}'' = \mathcal{B}' \setminus \{u\} \cup \{v\}$ 
			\begin{align*}
				\mathcal{B}'' \cap \mathcal{B} &= \left( (\mathcal{B}' \setminus \{u\})  \cap \mathcal{B} \right) \cup \{v\} \\
				&= (\mathcal{B}' \cap \mathcal{B}) \cup \{v\} \\
			\end{align*}
			donc,
			\begin{align*}
				n - \#(\mathcal{B}'' \cap \mathcal{B}) &= n - (\#(\mathcal{B}' \cap \mathcal{B}) + 1) \\
				&= p+1- 1 \\
				&= p \\
			\end{align*}
			D'après l'hypothèse de récurrence, \[
				\#\mathcal{B}'' = n
			\] Or, $\#\mathcal{B}'' = \#\mathcal{B}'$
	\end{itemize}
\end{prv}

\begin{lem}
	Soient $\mathcal{B}$ et $\mathcal{B}'$ deux bases de $E$ telles que $\mathcal{B}\subset \mathcal{B}'$. Alors, $\mathcal{B} = \mathcal{B}'$.
\end{lem}

\begin{prv}
	On suppose $\mathcal{B}' \neq \mathcal{B}$. Soit $u \in \mathcal{B}' \setminus \mathcal{B}$
	$u \in E = \Vect(\mathcal{B})$ donc $\mathcal{B} \cup \{u\}$ n'est pas libre.
	Donc $\mathcal{B}\cup \{u\} \subset \mathcal{B}'$ et $\mathcal{B}'$ est libre donc $\mathcal{B}\cup \{u\}$ est libre: une contradiction $\lightning$
\end{prv}

\begin{lem}
	[Lemme d'échange] Soient $\mathcal{B}_1$ et $\mathcal{B}_2$ deux bases de $E$ et $u \in \mathcal{B}_1 \setminus \mathcal{B}_2$. Alors, il existe $v \in \mathcal{B}_2$ tel que $(\mathcal{B}_1 \setminus \{u\}) \cup \{v\}$ soit une base de $E$.
\end{lem}

\begin{prv}
	[1${}^\text{nde}$ méthode]
	On suppose que pout tout $v \in \mathcal{B}_2$, $(\mathcal{B}_1\setminus \{u\}) \cup \{v\}$ n'est pas une base de $E$
	Soit $v \in \mathcal{B}_2$.
	\begin{itemize}
		\item Supposons $(\mathcal{B}_1\setminus \{u\})\cup \{v\}$ non libre. $\mathcal{B}_1 \setminus \{u\}$ est libre. Donc $v \in \Vect(\mathcal{B}_1 \setminus \{u\})$
		\item Supposons $(\mathcal{B}_1\setminus \{u\}) \cup \{v\}$ non génératrice.
			Comme $\mathcal{B}_1$ engendre $E$, $u \not\in \Vect(\mathcal{B}_1\setminus \{v\})$.
			On suppose que $\mathcal{B}_1 \neq \mathcal{B}_2$.
			$\forall v \in \mathcal{B}_2 \setminus \mathcal{B}_1, \Vect(\mathcal{B}_1 \setminus \{v\}) = \Vect(\mathcal{B}_1) = E \ni u$ 
			donc, $(\mathcal{B}_1\setminus \{u\}) \cup \{v\}$ engendre $E$ et donc \[
				v \in \Vect(\mathcal{B}_1 \setminus \{u\})
			\] On a aussi \[
				\forall v \in \mathcal{B}_1 \setminus \{u\}, v \in \Vect(\mathcal{B}_1\setminus \{u\})
			\] Comme $u \not\in \mathcal{B}_2$, on a \[
				\forall v \in \mathcal{B}_2, v \in \Vect(\mathcal{B}_1\setminus \{u\})
			\] docn \[
				E = \Vect(\mathcal{B}_2) \subset \Vect(\mathcal{B}_1\setminus \{u\})
			\] donc $\mathcal{B}_1\setminus \{u\}$ engendre $E$ donc $\mathcal{B}_1\setminus \{u\}$ est une base de $E$. Or, $\mathcal{B}_1 \setminus \{u\}  \subset  \mathcal{B}_1$, donc $\mathcal{B}_1\setminus \{u\} = \mathcal{B}_1$
	\end{itemize}
\end{prv}

\begin{prv}
	[2${}^\text{nde}$ méthode]
	On suppose que pout tout $v \in \mathcal{B}_2$, $(\mathcal{B}_1\setminus \{u\}) \cup \{v\}$ n'est pas une base de $E$
	\begin{itemize}
		\item Comme $u \in \mathcal{B}_1 \setminus \mathcal{B}_2$, nécéssairement $\mathcal{B}_1 \neq \mathcal{B}_2$ donc $\mathcal{B}_2 \not\subset \mathcal{B}_1$, donc $\mathcal{B}_2\setminus\mathcal{B}_1 \neq \O$ 
		\item Soit $v \in \mathcal{B}_2\setminus\mathcal{B}_1$. Il existe $(\lambda_w)_{w\in\mathcal{B}_1}$ une famille de scalaires presque nulle telle que \[
				v = \sum_{w \in \mathcal{B}_1} \lambda_w w - \lambda_u u + + \sum_{w \in \mathcal{B}_1\setminus \{u\}}\lambda_w w
			\]
			Si $\lambda_u \neq 0_E$, alors
			\begin{align*}
				u &= \lambda_u^{-1}\left( v - \sum_{w \in \mathcal{B}_1 \setminus \{u\}} \lambda_w w \right)\\
					&\in \Vect(\mathcal{B}_1\setminus \{u\} \cup v)
			\end{align*}
			 donc $\mathcal{B}_1 \subset \Vect(\mathcal{B}_1\setminus \{u\} \cup \{v\})$\\
			 et donc $E \subset  \Vect(\mathcal{B}_1 \setminus \{u\} \cup \{v\})$ \\
			 et donc $\mathcal{B}_1 \setminus \{u\} \cup \{v\}$ engendre $E$ \\
			 donc $\mathcal{B}_1 \setminus \{u\} \cup \{v\}$ n'est pas libre\\
			 donc $v \in \Vect(\mathcal{B}_1\setminus \{u\})$ (car $\mathcal{B}_1 \setminus \{u\}$ est libre\\
			 donc $\lambda_u = 0_\mathbbm{K}$ $\lightning$\\`

			 Donc, $\lambda_u = 0_\mathbbm{K}$, docn $v \in \Vect(\mathcal{B}_1\setminus \{u\})$ \\
			 On vient de prouver que
			 \begin{align*}
			 	\mathcal{B}_2 \setminus \mathcal{B}_1 \subset \Vect(\mathcal{B}_1 \setminus \{u\})\\
			 	\mathcal{B}_1 \setminus \{u\} \subset \Vect(\mathcal{B}_1 \setminus \{u\})\\
			 \end{align*}
			 Comme $u \not\in \mathcal{B}_2$, \[
			 	\mathcal{B}_2 \subset \Vect(\mathcal{B}_1 \setminus \{u\})
			 \] donc \[
			 	E = \Vect(\mathcal{B}_2) \subset  \Vect(\mathcal{B}_1 \setminus \{u\})
			 \] donc $\mathcal{B}_1 \setminus \{u\}$ engendre $E$. Donc,  $\mathcal{B}_1 \setminus \{u\}$ est une base de $E$.\\
			 Or, $\mathcal{B}_1 \setminus \{u\} \subset  \mathcal{B}_1$, donc $\mathcal{B}_1 \setminus \{u\} = \mathcal{B}_1$
	\end{itemize}
\end{prv}

\begin{defn}
	Soit $E$ un $\mathbbm{K}$-espace vectoriel de dimension finie. Le cardinal commun à toutes les bases de $E$ est appelé \underline{dimension} de $E$ est notée $\dim(E)$ ou $\dim_\mathbbm{K}(E)$\\
	C'est donc aussi le nombre de coordonnées de n'importe quel vecteur dans n'importe quelle base.
	\index{dimension (espace vectoriel)}
\end{defn}

\begin{exm}
	\begin{enumerate}
		\item $\dim_\R(\C) = 2$ et $\dim_\C(\C) = 1$ 
		\item $\dim_\mathbbm{K}(\mathbbm{K}^{n}) = n$ 
		\item $\dim_{\mathbbm{K}}(\mathcal{M}_{n,p}(\mathbbm{K})) = np$
	\end{enumerate}
\end{exm}

\begin{crlr}
	Soit $E$ un $\mathbbm{K}$-espace vectoriel de dimension finie, $\mathcal{L}$ une famille libre de $E$, $\mathcal{G}$ une famille génératrice de $E$. On note $n = \dim(E)$
	\begin{enumerate}
		\item $\#\mathcal{G} \ge n$ et $(\#\mathcal{G} = n \implies \mathcal{G} \text{ est une base de } E$)
		\item $\#\mathcal{L} \le n$ et $(\#\mathcal{L} = n \implies \mathcal{L} \text{ est une base de } E$)
	\end{enumerate}
\end{crlr}

\begin{crlr}
	$\R^{\R}$ est de dimension infinie.
	$\forall i \in \N, e_i: x \mapsto x^i$\\
	$(e_i)_{i\in\N}$ est libre dans $\R^\R$
\end{crlr}

\begin{prop}
	Soient $E$ et $F$ deux $\mathbbm{K}$-espaces vectoriels de dimension finie. Alors $E\times F$ est de dimension finie et $\dim(E\times F) = \dim(E) + \dim(F)$
\end{prop}

\begin{prv}
	Soit $(e_1,\ldots, e_n)$ une base de $E$, $(f_1, \ldots, f_p)$ une base de $F$.
	On pose \[
		\left\{\begin{array}
			{r c l}
			u_1 &=& (e_1,0_F)\\
			u_2 &=& (e_2,0_F)\\
					&\vdots&\\
			u_n &=& (e_n,0_F)\\
			u_{n+1} &=& (0_E, f_1)\\
			u_{n+2} &=& (0_E, f_2)\\
					&\vdots&\\
			u_{n+p} &=& (0_E,f_p)\\
		\end{array}\right.
	\]
	Soit $(x,y) \in E\times F$. \[
		\begin{cases}
			\exists (x_1,\ldots,x_n)\in \mathbbm{K}^n, x = \sum_{i=1}^{n} x_ie_i
			\exists (y_1,\ldots,y_n)\in \mathbbm{K}^n, x = \sum_{j=1}^{p} y_jf_j
		\end{cases}
	\] 
	\begin{align*}
		(x,y) &= \left( \sum_{i=1}^{n} x_ie_i, \sum_{i=1}^{p} y_jf_j \right)  \\
		&= \sum_{i=1}^{n} x_i (e_i + 0_F) + \sum_{j=1}^{p} y_j (0_E, f_j) \\
		&= \sum_{i=1}^{n} x_i u_i + \sum_{j=1}^{p} y_j u_{n+j} \\
	\end{align*}
	Donc, $E\times F = \Vect(u_1, \ldots, u_{n+p})$ donc $E\times F$ est de dimension finie.\\
	Soit $(\lambda_1, \ldots, \lambda_{n+p}) \in \mathbbm{K}^{n+p}$ tel que \[
		(*): \quad \sum_{k=1}^{n+p} \lambda_ku_k = 0_{E\times F} = (0_E, 0_F)
	\]
	\begin{align*}
		(*) &\iff \sum_{k=1}^{n} \lambda_k (e_k, 0_F) + \sum_{k=n+1}^{p} \lambda_k(0_E, f_{k-n}) = (0_E, 0_F)\\
				&\iff \begin{cases}
					\sum_{k=1}^{n} \lambda_k e_k = 0_E\\
					\sum_{k=n+1}^{p} \lambda_k f_{k-n} = 0_F
				\end{cases}\\
				&\iff \begin{cases}
					\forall k \in \left\llbracket 1,n \right\rrbracket, \lambda_k = 0_\mathbbm{K} \qquad&(\text{car $(e_1,\ldots,e_n)$ est libre})\\
					\forall k \in \left\llbracket n+1,n+p \right\rrbracket, \lambda_k = 0_\mathbbm{K} \qquad&(\text{car $(f_1,\ldots,f_n)$ est libre})\\
				\end{cases}
	\end{align*}
	Donc $(u_1, \ldots, u_{n+p})$ est une base de $E\times F$. Donc, $\dim(E\times F) = n + p = \dim(E) + \dim(F)$
\end{prv}

\begin{rmk}
	[Convention]
	\[\dim\big(\{0_E\}\big) = 0\]
\end{rmk}

\begin{thm}
	Soit $E$ un $\mathbbm{K}$-espace vectoriel de dimension finie, $F$ un sous-espace vectoriel de $E$. Alors, $F$ est de dimension finie et  $\dim(F) \le \dim(E)$\\
	Si $\dim(F) = \dim(E)$, alors $F = E$
\end{thm}

\begin{prv}
	On considère \[
		A = \{k \in \N \mid \text{il existe une famille libre de $F$ à $k$ éléments}\} 
	\]
	On suppose $F \neq \{0_E\}$.
	\begin{itemize}
		\item Soit $u \in F\setminus \{0_E\}$. $(u)$ est libre donc $1 \in A$ et donc $A \neq \O$
		\item Soit $\mathcal{L}$ une famille libre de $F$. Alors, $\mathcal{L}$ est une famille libre de $E$ \\
			donc $\#\mathcal{L} \le \dim(E)$\\
			Donc $A$ est majorée par $\dim(E)$ \\
			On en déduit que $A$ a un plus grand élément $p$.
		\item Soit $\mathcal{L}$ une famille libre de $F$ avec $p$ éléments.\\
			Si $\mathcal{L}$ n'engendre pas $F$, alors il existe $u\in F$ tel que $u\not\in \Vect(\mathcal{L})$ et donc $\mathcal{L} \cup \{u\}$ est une famille libre de $F$, donc $p+1 \in A$ en contradiction avec la maximalité de $p$.\\
			Donc $\mathcal{L}$ est une base de $F$ donc $F$ est de dimension finie et $\dim(F) = p \le \dim(E)$\\
	\end{itemize}

	Soit $\mathcal{B}$ une base de $F$. Alors, $\mathcal{B}$ est aussi une famille de libre de de $E$. Donc $\#\mathcal{B} \le \dim(E)$ donc $\dim(F) = \dim(E)$ \\
	Si $\dim(F) = \dim(E)$, alors $\mathcal{B}$ est une base de $E$, et donc $F = \Vect(\mathcal{B}) = E$
\end{prv}

\begin{prop}
	[Formule de Grassmann]
	Soit $E$ un $\mathbbm{K}$-espace vectoriel de dimension finie, $F$ et $G$ deux sous-espace vectoriels de $E$. Alors, \[
		\dim(F+G) = \dim(F) + \dim(G) - \dim(F\cap G)
	\] 
\end{prop}

\begin{prv}
	Soit $(e_1, \ldots, e_p)$ une base de $F\cap G$. $(e_1,\ldots,e_p)$ est une famille libre de $F$.\\
	On complète $(e_1, \ldots, e_p)$ en une base $(e_1, \ldots, e_p, u_1, \ldots, u_q)$ de $F$.\\
	De même, on complète $(e_1, \ldots, e_p)$ en une base $(e_1, \ldots, e_p, v_1, \ldots, v_r)$ de $G$.\\
	On pose  $\mathcal{B} = (e_1, \ldots, e_p, u_1, \ldots, u_q, v_1, \ldots, v_r)$. Montrons que $\mathcal{B}$ est une base de $F+G$
	\begin{itemize}
		\item Soit $u \in F+G$ \\
			On pose $u = v+w$ avec $\begin{cases}
				v\in F\\
				w \in G
			\end{cases}$.\\
			On pose $v = \sum_{i=1}^p \lambda_i e_i + \sum_{i=1}^q \mu_i u_i$ avec $(\lambda_1, \ldots, \lambda_p, \mu_1, \ldots, \lambda_q) \in \mathbbm{K}^{p+q}$\\
			On pose aussi $w = \sum_{i = 1}^p \lambda'_ie_i + \sum_{j=1}^r \nu_j v_j$ avec $(\lambda_1',\ldots,\lambda_p', \nu_1, \ldots, \nu_r) \in \mathbbm{K}^{p+r}$\\
			D'où, \[
				u = \sum_{i=1}^p (\lambda_i + \lambda'_i)e_i + \sum_{j=1}^q \mu_j u_j + \sum_{k=1}^r \nu_k v_k \in \Vect(\mathcal{B})
			\]
		\item Soient $(\lambda_1, \ldots, \lambda_p, \mu_1, \ldots, \mu_q, \nu_1, \ldots, \nu_r) \in \mathbbm{K}^{p+q+r}$.\\
			On suppose \[
				(*)\quad \sum_{i=1}^{p}\lambda_ie_i + \sum_{j=1}^q\mu_ju_j + \sum_{k=1}^r \nu_k v_k = 0_E
			\] 
			D'où, \[
				\underbrace{\sum_{i=1}^p\lambda_i e_i + \sum_{j=1}^q \mu_ju_j}_{\in F} = \underbrace{-\sum_{k=1}^r\nu_jv_k}_{\in G}
			\] 
			Donc, \[
				f = \sum_{i=1}^p \lambda_i e_i + \sum_{j=1}^q \mu_j u_j \in F\cap G
			\] Comme $(e_1, \ldots, e_p)$ est une base de $F\cap G$, $\exists ! (\lambda_1', \ldots, \lambda_p') \in \mathbbm{K}^p$ tel que \[
				f = \sum_{i=1}^p \lambda'_i e_i = \sum_{i=1}^p \lambda'_i e_i + \sum_{j=1}^q 0_\mathbbm{K}u_j
			\] Comme $(e_1, \ldots, e_p, u_1, \ldots, u_q)$ est une base de $F$, \[
				\forall k \in \left\llbracket 1, q \right\rrbracket, \mu_j = 0_\mathbbm{K}
			\] De même, \[
				\forall k \in \left\llbracket 1,r \right\rrbracket , \nu_k = 0_\mathbbm{K}
			\] On remplace dans $(*)$ pour trouver \[
				\sum_{i=1}^p \lambda_ie_i = 0_E
			\] Comme $(e_1, \ldots, e_p)$ est libre, \[
				\forall i \in \left\llbracket 1,p \right\rrbracket, \lambda_i = 0_\mathbbm{K}
			\] Donc $\mathcal{B}$ est libre.\\
			Donc, 
			\begin{align*}
				\dim(F+G) &=  p +q + r \\
				&= (p+q)+ (p+r) - p \\
				&= \dim(F) + \dim(G) - \dim(F\cap G) \\
			\end{align*}
	\end{itemize}
\end{prv}

\begin{crlr}
	Avec les hypothèse précédentes, \[
		E = F \oplus G \iff \begin{cases}
			F \cap  G = \{0_E\} \\
			\dim(E) = \dim(F) + \dim(G)
		\end{cases}
	\] 
\end{crlr}

\begin{prv}
	\begin{itemize}
		\item[``$\implies$''] On suppose $E = F \oplus G$ \\
			Comme la somme est directe, $F \cap G = \{0_E\}$ 
			\begin{align*}
				\dim(E) &= \dim(F)\\
				&= \dim(F) + \dim(G) - \dim(F\cap G)\\
				&= \dim(F) + \dim(G)\\
			\end{align*}
		\item[``$\impliedby$''] On suppose $F\cap G = \{0_E\}$ et $\dim(E) = \dim(F) + \dim(G)$.\\
			On sait déjà que $F+G = F \oplus G$\\
			 \begin{align*}
				\dim(F+G) = \dim(F) + \dim(G) - \dim(F \cap G) = \dim(E)
			\end{align*}
			Donc $F + G = E$
	\end{itemize}
\end{prv}

\begin{prop}
	Soit $F$ un $\mathbbm{K}$-espace vectoriel de dimension finie $n$. Soit $\mathcal{B} = (e_1, \ldots, e_n)$ une base de $F$. L'application
	\begin{align*}
		f: \mathbbm{K}^n &\longrightarrow F \\
		(\lambda_1, \ldots, \lambda_n) &\longmapsto \sum_{i=1}^n \lambda_i e_i
	\end{align*} est bijective.\\
	Si $\mathbbm{K}$ est infini, $\mathbbm{K}^n$ aussi et donc $F$ aussi.\\
	Si $\#\mathbbm{K} = p \in \N_*$,
	\begin{align*}
		\#&\mathbbm{K}^n = p^n\\
		&\vrt=\\
		\#&F
	\end{align*}
\end{prop}


		\part{Dérivation}

\underline{Motivation}:

{
\begin{wrapfigure}{l}{3cm}
	\centering
	\begin{asy}
		import three;

		size(3cm);
		settings.render=0;
		settings.prc=false;
		currentprojection = obliqueZ;

		draw(unitbox);
		draw(shift(1.1Z + 0.05X) * (O -- X), Arrows3(TeXHead2));
		draw(shift(1.1Z + 0.05Y) * (O -- Y), Arrows3(TeXHead2));
		draw(shift(1.1X + 0.05Z) * (O -- Z), Arrows3(TeXHead2));

		label("$x$", (X/2) + (1.1Z + 0.05X), align=S);
		label("$y$", (Y/2) + (1.1Z + 0.05Y), align=W);
		label("$z$", (Z/2) + X, align=SE);
	\end{asy}
\end{wrapfigure}

\begin{align*}
	&S(x,y,z) = 2(xy + xz + yz)\\
	&V(x,y,z) = xyz
\end{align*}

On cherche à minimiser $S$ avec la contrainte $V = 1$.

Soit $f : \begin{array}{rcl}
	\left( \R_*^+ \right)^2 &\longrightarrow& \R \\
	(x,y) &\longmapsto& S\left( x,y,\frac{1}{xy} \right) = 2\left( xy + \frac{1}{y} + \frac{1}{x} \right).
\end{array}$

On cherche $(a,b) \in \left( \R^+_* \right)^2$ tel que \[
	\forall (x,y) \in (\R^+_*), f(x,y) \ge f(a,b).
\]
}

\begin{defn}
	Soit $f: U \to \R$ où $U$ est un ouvert de $\R^2$. Soit $(a,b) \in U$.
	\vspace{2mm}

	Si $\lim_{x \to a} \frac{f(x,b) - f(a,b)}{x - a} \in \R$, alors on dit que $f$ a une dérivée partielle suivant $x$ en $(a,b)$ et cette limite est notée \[
		\partial f_1(a,b) = \frac{\partial f}{\partial x}(a,b).
	\]

	Si $\lim_{y \to b} \frac{f(a,y) - f(a,b)}{y - b} \in \R$, alors on dit que $f$ a une dérivée partielle suivant $y$ et la limite est notée \[
		\partial f_2(a,b) = \frac{\partial f}{\partial y}(a,b).
	\]
\end{defn}

\begin{exm}
	\begin{enumerate}
		\item $f: (x,y) \mapsto xy + x - y$.

			\begin{align*}
				&\frac{\partial f}{\partial x} : (x,y) \mapsto y + 1,\\
				&\frac{\partial f}{\partial y} : (x,y) \mapsto x - 1.
			\end{align*}

		\item $f: (x,y) \mapsto xy + \frac{1}{y}+ \frac{1}{x}$.

			\begin{align*}
				&\frac{\partial f}{\partial x}: (x,y) \mapsto y - \frac{1}{x^2},\\
				&\frac{\partial f}{\partial y}: (x,y) \mapsto x - \frac{1}{y^2}.
			\end{align*}

		\item Trouver $f$ telle que $\begin{cases}
				(1): \qquad \frac{\partial f}{\partial x}=y,\\[2mm]
				(2): \qquad \frac{\partial f}{\partial y} = x.
			\end{cases}$

			D'après $(1)$ : \[
				\forall (x,y), \exists C(y) \in \R, f(x,y) = xy + C(y)
			\] et donc \[
				\frac{\partial f}{\partial y}(x,y) = x + C'(y)
			\] donc $C'(y) = 0$ et donc $C$ est constante.

		\item Trouver $f$ telle que $\begin{cases}
			\frac{\partial f}{\partial x} = -y,\\[2mm]
			\frac{\partial f}{ƒ\partial y} = x.
		\end{cases}$

		Ce n'est pas possible !
	\end{enumerate}
\end{exm}

\begin{defn}~\\
	\begin{minipage}{\linewidth}
		\begin{wrapfigure}{r}{4cm}
			\centering
			\vspace{-5mm}
			\begin{asy}
				import three;
				import graph3;
				size(4cm);

				settings.render = 0;
				settings.prc = false;
				currentprojection = obliqueX;

				draw(O -- X, Arrow3(TeXHead2));
				draw(O -- Y, Arrow3(TeXHead2));
				draw(O -- Z, Arrow3(TeXHead2));

				triple f(real x, real y, real z = 0) { return (x,y,cos(x - 0.5) * cos(y - 0.5)/1.2 + 0.15); }

				real inc = 1 / 5;

				for(real x = 0; x <= 1; x += inc) {
					draw(graph(
						new real(real t) { return x; }, // x
						new real(real y) { return y; }, // y
						new real(real y) { return f(x,y).z; }, // z
						0, 1
					), gray);
				}

				for(real y = 0; y <= 1; y += inc) {
					draw(graph(
						new real(real x) { return x; }, // x
						new real(real t) { return y; }, // y
						new real(real x) { return f(x,y).z; }, // z
						0, 1
					), gray);
				}

				path3 path1 = (0.8, 0.2, 0) .. (0.5, 0.5, 0) .. (0.3, 0.7, 0);
				path3 path2 = f(0.8, 0.2, 0) .. f(0.5, 0.5, 0) .. f(0.3, 0.7, 0);
				path3 d = (0.2, 0.3, 0) .. (0.3, 0.4, 0) .. (0.2, 0.7, 0) .. (0.8, 0.9, 0) .. (0.6, 0.2, 0) .. cycle;

				draw(path1, red, Arrow3(TeXHead2));
				draw(path2, red, Arrow3(TeXHead2, position=0.8));

				dot((0.5, 0.5, 0));
				dot(f(0.5, 0.5, 0));
				draw((0.5, 0.5, 0) -- f(0.5, 0.5, 0), dashed);
				draw(d);

				label("$w$", (0.3, 0.7, 0), red, align=SE);
				label("$U$", (0.8, 0.9, 0), align=SE);
			\end{asy}
		\end{wrapfigure}

		Soit $f: U \to \R$ où $U$ est un ouvert. Soit $(a,b) \in U$. Soit $w = (w_1, w_2) \in \R^2$.

		Si 
		\[
			\lim_{t\to 0} \frac{f(a + tw_1, b + tw_2) - f(a,b)}{t}
		\] existe et est réelle, alors on dit que $f$ a une dérivée dans la direction de $w$ et la limite est notée \[
			\mathrm{d}f(w)\,(a,b) = D_w(f)\,(a,b).
		\]
	\end{minipage}
\end{defn}

\begin{exm}
	\begin{align*}
		f: \left( \R_*^+ \right)^2 &\longrightarrow \R \\
		(x,y) &\longmapsto xy+\frac{1}{x}+\frac{1}{y}.
	\end{align*}

	On pose $(a,b) = (1,2)$, $w = (w_1, w_2) = (1,1)$.
	\begin{align*}
		\frac{f(1+t, 2+t) - f(1,2)}{t} &= \frac{1}{t} \left( (1+t)(2+t) + \frac{1}{1+t} + \frac{1}{2+t} - 3 - \frac{1}{2} \right) \\
		&= \frac{1}{t}\left(\cancel 2 + 3t + \po(t) + \cancel 1 - t + \po(t) + \frac{1}{2}\left( \cancel 1 - \frac{t}{2} + \po(t) \right) - \cancel3 - \cancel{\frac{1}{2}} \right) \\
		&= \frac{1}{t} \left( \frac{7}{4} t + \po(t) \right)  \\
		&= \frac{7}{4} + \po(1) \tendsto{t \to 0} \frac{7}{4}. \\
	\end{align*}

	Donc, \[
		\mathrm{d}f(1,1)\,(1,2) = \frac{7}{4}.
	\]
\end{exm}

\begin{rmk}~\\
	\begin{figure}[H]
		\centering
		\begin{asy}
			import solids;
			import graph;
			size(5cm);

			settings.render = 0;
			settings.prc = false;

			path3 par = graph(
				new real(real x) { return x; },
				new real(real x) { return 0; },
				new real(real x) { return x^2; },
				0,3);
			revolution r = revolution(par, axis=Z);

			path3 par2 = graph(
				new real(real x) { return x; },
				new real(real x) { return 0; },
				new real(real x) { return x^2; },
				-3,3);

			draw(r,1,longitudinalpen=nullpen);
			draw(r.silhouette());

			draw((-4, 0, -1) -- (-4, 0, 10) -- (4, 0, 10) -- (4, 0, -1) -- cycle, red);
			draw(par2, deepred);

			draw((4,4.5) -- (7, 4.5), black+0.5mm, Arrow(TeXHead));

			path par2d = graph(new real(real x) { return x^2; }, -3, 3);
			draw(shift((11, 0)) * par2d, deepred);

			dot(O);
			dot((11, 0));
		\end{asy}
	\end{figure}
\end{rmk}


%todo ajouter théorème-définition
\begin{thm}
	Soit $f : U \to \R$, $(a,b) \in U$. On suppose que $\frac{\partial f}{\partial x}$ et $\frac{\partial f}{\partial y}$ existent en $(a,b)$ et sont {\bfseries continues} en $(a,b)$. Alors,
	\begin{align*}
		&\forall (h, k) \in \R^2 \text{ tel que } (a +h, b + k) \in U,\\
		&f(a+ h, b + k) = f(a,b) + h \frac{\partial f}{\partial x}(a,b) + k \frac{\partial f}{\partial y}(a,b) + \po_{(h,k)\to (0,0)}\big(\|(h,k)\|\big).
	\end{align*}

	On dit que $f$ est de classe $\mathcal{C}^1$ si $\frac{\partial f}{\partial x}$ et $\frac{\partial f}{\partial y}$ existent et sont continues.

	\qed
\end{thm}

\begin{rmk}
	En physique, cette formule correspond à : \[
		\mathrm{d}f = \frac{\partial f}{\partial x}\mathrm{d}x + \frac{\partial f}{\partial y} \mathrm{d}y.
	\] En effet :
	\begin{align*}
		\mathrm{d}f &= f(x+ \mathrm{d}x, y + \mathrm{d}y) - f(x,y) \\
		&= \frac{\partial f}{\partial x} \mathrm{d}x + \frac{\partial f}{\partial y} \mathrm{d}y.
	\end{align*}
\end{rmk}

\begin{prop}
	Soit $f: U \to \R$ de classe $\mathcal{C}^1$ en $(a,b) \in U$. Alors, \[
		\forall w = (w_1, w_2) \in \R^2, \mathrm{d}f(w)\,(a,b) = w_1 \frac{\partial f}{\partial x}(a,b) + w_2 \frac{\partial f}{\partial y}(a,b).
	\]
\end{prop}

\begin{prv}
	Soit $w = (w_1, w_2) \in \R^2$. Soit $t \in \R^*$.
	\begin{align*}
		\frac{1}{t}\big(f(a + tw_1, b + tw_2) - f(a,b)\big)
		&= \frac{1}{t} \left( tw_1 \frac{\partial f}{\partial x}(a,b) + tw_2 \frac{\partial f}{\partial y}(a,b) + \po_{t \to 0}\big(\|tw\|\big) \right) \\
		&= w_1 \frac{\partial f}{\partial x}(a,b) + w_2 \frac{\partial f}{\partial y}(a,b) + \po_{t\to 0}(1) \\
		&\tendsto{t\to 0} w_1 \frac{\partial f}{\partial x}(a,b) + w_2\frac{\partial f}{\partial y}(a,b).
	\end{align*}
\end{prv}


\begin{defn}
	Avec les hypothèses précédentes, en posant \[
		\nabla f(a,b) = \left( \frac{\partial f}{\partial x}(a,b), \frac{\partial f}{\partial y}(a,b) \right) 
	\]on obtient \[
		\mathrm{d}f(w)\,(a,b) = \left<w  \mid \nabla f(a,b) \right>
	\] où $\left<\cdot|\cdot \right>$ est le produit scalaire.

	Le vecteur $\nabla f(a,b)$ est appelé \underline{gradient de $f$ en $(a,b)$}.

	Le développement limité à l'ordre 1 de $f$ devient \[
		f\big((a,b)+w\big) = f(a,b) + \left<w \mid \nabla f(a,b) \right> + \po_{w\to 0}(\|w\|)
	\]
\end{defn}

\begin{prop}
	Soit $f : U \to \R$ de classe $\mathcal{C}^1$.

	\begin{figure}[H]
    \centering
    \incfig{gradient}
	\end{figure}

	$\nabla f$ est orthogonal au lignes de niveaux de $f$, son orientation va dans le sens d'une augmentation de $f$.
\end{prop}

\begin{prv}
	Soit $\gamma : I \to U$ une courbe de niveau : \[
		\forall t \in I, f\big(\gamma(t)\big) = \text{cste}.
	\] D'après le lemme suivant : \[
		\forall t \in I, 0 = (f \circ \gamma)'(t) = \mathrm{d}f\big(\gamma'(t)\big)\big(\gamma(t)\big) = \left<\gamma'(t)  \mid \nabla f\big(\gamma(t)\big) \right>
	\] Donc $\nabla f\big(\gamma(t)\big)$ est orthogonal à $\gamma'(t)$.

	Pour tout $t \in I$, on pose $w(t) = t\, \nabla f\big(\gamma(t)\big)$. Donc \[
		f\big(\gamma(t) + w(t)\big) = f\big(\gamma(t)\big) + t \|\nabla f(\gamma(t))\|^2 + \po_{t \to 0}(t)
	\] Pour $t$ assez petit, $f\big(\gamma(t) + w(t)\big) - f\big(\gamma(t)\big)$ est du même signe que $t$.
\end{prv}

\begin{rmk}
	\begin{align*}
		V: \R^3 &\longrightarrow \R \\
		(x,y,z) &\longmapsto -mgz
	\end{align*}
	l'énerge potentielle de pesenteur

	On a donc \[
		\nabla V(x,y,z) = \left( \frac{\partial V}{\partial x}, \frac{\partial V}{\partial y}, \frac{\partial V}{\partial z} \right) = (0, 0, -mg) = \vec{P}.
	\]
\end{rmk}

\begin{lem}
	Soit $f : U \to \R$ de classe $\mathcal{C}^1$, $\gamma : \begin{array}{rcl}
		I &\longrightarrow& U \\
		t &\longmapsto& \big(x(t), y(t)\big)
	\end{array}$ où $x$ et $y$ sont dérivables.

	On pose \[
		\forall t \in I, \gamma'(t) = \big(x'(t), y'(t)\big).
	\] Alors $f \circ \gamma : I \to \R$ est dérivable et
	\begin{align*}
		\forall t \in I, (f \circ \gamma)'(t) &= \mathrm{d}f\big(\gamma'(t)\big) \big(\gamma(t)\big)\\
		&= \left<\gamma'(t)  \mid \nabla f\big(\gamma(t)\big)  \right> \\
		&= x'(t) \frac{\partial f}{\partial x}\big(x(t), y(t)\big) + y'(t) \frac{\partial f}{\partial y}\big(x(t),y(t)\big). \\
	\end{align*}
\end{lem}

\begin{prv}
	On fixe $t \in I$.

	\begin{align*}
		\forall h \neq 0, \frac{f \circ \gamma(t + h) - f \circ \gamma(t)}{h}
		&= \frac{1}{h}\big(f(\gamma(t)) + h\gamma'(t) + \po_{h\to 0}(h) - f(\gamma(t))\big) \\
		&= \frac{1}{h}\bigg(\cancel{f(\gamma(t))} + \left<h\gamma'(t) \mid \nabla f(\gamma(t)) \right> + \po_{h\to 0}(\|h\gamma'(t)\|) - \cancel{f(\gamma(t))}\bigg)\\
		&= \left<\gamma'(t) \mid \nabla f(\gamma(t)) \right> + \po_{h\to 0}(1) \\
		&\tendsto{h\to 0} \left<\gamma'(t)  \mid \nabla f(\gamma(t)) \right>
	\end{align*}
\end{prv}

\begin{defn}
	Soit $f : U \to \R$ de classe $\mathcal{C}^1$ et $(a,b) \in U$. On dit que $(a,b)$ est un \underline{point critique} de $f$ si $\nabla f(a,b) = 0$ i.e. $\frac{\partial f}{\partial x}(a,b) = \frac{\partial f}{\partial y}(a,b) = 0$.

	Dans ce cas, $f(a,b)$ est appelé \underline{valeur critique} de $f$.
\end{defn}

\begin{prop}~\\
	\begin{minipage}{\linewidth}
		\begin{wrapfigure}{r}{3cm}
			\centering
			\vspace{-1cm}
			\begin{asy}
				import solids;
				import graph;
				size(3cm);

				settings.render = 0;
				settings.prc = false;

				path3 par = graph(
					new real(real x) { return x; },
					new real(real x) { return 0; },
					new real(real x) { return -x^2; },
					0,3);
				revolution r = revolution(par, axis=Z);

				draw(r,1,longitudinalpen=nullpen);
				draw(r.silhouette());

				dot("$(a,b)$", O, red, align=N);
				real s = sqrt(2.5);
				path3 g=(s,0,-2.5)..(0,s,-2.5)..(-s,0,-2.5)..(0,-s,-2.5)..cycle;
				draw(g, deepcyan);
			\end{asy}
		\end{wrapfigure}
		Soit $f: U \to \R$ de classe $\mathcal{C}^1$ et $(a,b) \in U$ tel que \[
			\exists r > 0, \forall (x,y) \in B_{(a,b)}(r), f(x,y) \le f(a,b)
		\] Alors $\nabla f(a,b) = (0,0)$.
	\end{minipage}
\end{prop}

\begin{prv}
	Soit $g: x \mapsto f(x,b)$. $g(a)$ est un maximum local de $g$ donc $g'(a) = 0$.

	Or, $g'(a) = \frac{\partial f}{\partial x}(a,b)$

	donc $\frac{\partial f}{\partial x}(a,b) = 0$.

	Soit $h : y \mapsto f(a,y)$. On a de même $h'(b) = 0$.

	Or, $h'(b) = \frac{\partial f}{\partial y}(a,b)$.

	Donc, $\nabla f(a,b) = (0,0)$.
\end{prv}

\begin{rmk}
	Un minimum local est aussi une valeur critique.
\end{rmk}

\begin{figure}[H]
	\centering
	\begin{subfigure}{3cm}
		\centering
		\begin{asy}
			import solids;
			import graph;
			size(3cm);

			settings.render = 0;
			settings.prc = false;

			path3 par = graph(
				new real(real x) { return x; },
				new real(real x) { return 0; },
				new real(real x) { return -x^2; },
				0,3);
			revolution r = revolution(par, axis=Z);

			draw(r,1,longitudinalpen=nullpen);
			draw(r.silhouette());

			dot(O, red);
		\end{asy}
		\caption{Maximum local}
	\end{subfigure}
	\begin{subfigure}{3cm}
		\centering
		\begin{asy}
			import solids;
			import graph;
			size(3cm);

			settings.render = 0;
			settings.prc = false;

			path3 par = graph(
				new real(real x) { return x; },
				new real(real x) { return 0; },
				new real(real x) { return x^2; },
				0,3);
			revolution r = revolution(par, axis=Z);

			draw(r,1,longitudinalpen=nullpen);
			draw(r.silhouette());

			dot(O, red);
		\end{asy}
		\caption{Minimum local}
	\end{subfigure}
	\begin{subfigure}{3cm}
		\centering
		\begin{asy}
			import solids;
			import graph;
			size(3cm);

			settings.render = 0;
			settings.prc = false;
			currentprojection = obliqueZ;

			draw(graph(
				new real(real x) { return x; },
				new real(real x) { return -x^2 / 3; },
				new real(real x) { return 3; },
				-3, 3
			));

			draw(graph(
				new real(real x) { return x; },
				new real(real x) { return -x^2 / 3; },
				new real(real x) { return -3; },
				-3, 3
			));

			draw(graph(
				new real(real x) { return x; },
				new real(real x) { return -x^2 / 3 - 1; },
				new real(real x) { return 0; },
				-3, 3
			));

			draw(graph(
				new real(real x) { return 0; },
				new real(real x) { return x^2 / 9 - 1; },
				new real(real x) { return x; },
				-3, 3
			));

			draw(graph(
				new real(real x) { return -3; },
				new real(real x) { return x^2 / 9 - 4; },
				new real(real x) { return x; },
				-3, 3
			));

			draw(graph(
				new real(real x) { return 3; },
				new real(real x) { return x^2 / 9 - 4; },
				new real(real x) { return x; },
				-3, 3
			));

			dot((0,-1,0), red);
		\end{asy}
		\caption{Point de selle / Point col}
	\end{subfigure}
\end{figure}

\begin{exm}
	On revient à l'exemple donné en introduction : 
	\begin{align*}
		f: \left( \R^*_+ \right)^2 &\longrightarrow \R \\
		(x,y) &\longmapsto 2\left( xy + \frac{1}{x} + \frac{1}{y} \right).
	\end{align*}

	$\left( \R^+_* \right)^2$ est un ouvert de $\R^2$. Soit $(x,y) \in \left( \R^+_* \right)^2$.
	
	On a \[
		\begin{cases}
			\frac{\partial f}{\partial x}(x,y) = 2\left( y - \frac{1}{x^2} \right),\\
			\frac{\partial f}{\partial y}(x,y) = 2\left( x - \frac{1}{y^2} \right).
		\end{cases}
	\]

	\begin{align*}
		&\frac{\partial f}{\partial x}(x,y) = \frac{\partial f}{\partial y}(x,y) = 0\\
		\iff& \begin{cases}
			y = \frac{1}{x^2}\\
			x = \frac{1}{y^2}
		\end{cases}\\
		\iff& \begin{cases}
			y = \frac{1}{x^2}\\
			x = x^4
		\end{cases}\\
		\iff& \begin{cases}
			x = 1\\
			y = 1
		\end{cases}
	\end{align*}

	On vérivie que $f$ présente en effet un minium local en $(1,1)$. \[
		f(1,1) = 6
	\] On fixe $y \in \R^+_*$ et \[
		g : x \mapsto 2\left( xy + \frac{1}{x} + \frac{1}{y} \right).
	\] Donc \[
		\forall x \in \R^+_*, g'(x) = 2\left( y - \frac{1}{x^2} \right).
	\]
	\begin{center}
		\begin{tikzpicture}
			\tkzTabInit{$x$/1,$g'(x)$/1,$g$/2.3}{$0$, $\frac{1}{\sqrt{y}}$, $+\infty$}
			\tkzTabLine{,-,z,+,}
			\tkzTabVar{+/{}, -/$2\left( 2\sqrt{y} +\frac{1}{y} \right)$, +/{}}
		\end{tikzpicture}
	\end{center}
	
	Ainsi, \[
		\forall x \in \R^+_*, \forall y \in \R^+_*, f(x,y) \ge 2\left( 2\sqrt{y} + \frac{1}{y} \right)
	\] Soit $h : y \mapsto 2\sqrt{y} + \frac{1}{y}$. On a \[
		\forall y > 0, h'(y) = \frac{1}{\sqrt{y}} - \frac{1}{y^2} = \frac{y\sqrt{y} - 1}{y^2} = \frac{y^{\frac{3}{2}} - 1}{y^2}
	\]

	\begin{center}
		\begin{tikzpicture}
			\tkzTabInit{$y$/0.7,$h'(y)$/0.7,$h$/1.4}{$0$, $1$, $+\infty$}
			\tkzTabLine{,-,z,+,}
			\tkzTabVar{+/{}, -/$3$, +/{}}
		\end{tikzpicture}
	\end{center}

	Donc, \[
		\forall x,y > 0, f(x,y) \ge 2\times 3 = 6 = f(1,1).
	\]
\end{exm}

\begin{prop}
	[règle de la chaîne]

	Soit $f : \begin{array}{rcl}
		U &\longrightarrow& \R^2 \\
		(x,y) &\longmapsto& f(x,y)
	\end{array}$ de classe $\mathcal{C}^1$ et $U, V$ deux ouverts de $\R^2$.

	Soit $\varphi : \begin{array}{rcl}
		V &\longrightarrow& U \\
		(u,v) &\longmapsto& \varphi(u,v) = \big(x(u,v), y(u,v)\big)
	\end{array}$.

	On suppose que $x$ et $y$ sont de classe $\mathcal{C}^1$ sur $V$.

	Alors,  $f \circ \varphi : \begin{array}{rcl}
		V &\longrightarrow& \R \\
		(u,v) &\longmapsto& f\big(\varphi(u,v)\big)
	\end{array}$ est de classe $\mathcal{C}^1$ et
	\begin{align*}
		\forall (u_0, v_0) \in V, \frac{\partial (f \circ \varphi)}{\partial u}(u_0, v_0)
		&= \frac{\partial f}{\partial x}\big(\varphi(u_0, v_0)\big) \times \frac{\partial x}{\partial u}(u_0, v_0)\\
		&+ \frac{\partial f}{\partial y}\big(\varphi(u_0,v_0)\big) \frac{\partial y}{\partial u}(u_0,v_0)
	\end{align*}
	\begin{align*}
		\forall (u_0, v_0) \in V, \frac{\partial (f \circ \varphi)}{\partial v}(u_0, v_0)
		&= \frac{\partial f}{\partial x}\big(\varphi(u_0, v_0)\big) \times \frac{\partial x}{\partial v}(u_0, v_0)\\
		&+ \frac{\partial f}{\partial y}\big(\varphi(u_0,v_0)\big) \frac{\partial y}{\partial v}(u_0,v_0)
	\end{align*}
\end{prop}

\begin{exm}
	[changement de coordonnées polaires]
	On pose \begin{align*}
		\varphi: \R^+_* \times ]0,2\pi[ &\longrightarrow \R^2\setminus \left( R^+_* \times \{0\} \right) \\
		(r, \theta) &\longmapsto (r \cos \theta, r \sin\theta),
	\end{align*}
	\begin{align*}
		f: \R^2\setminus \left( R^+_* \times \{0\} \right) &\longrightarrow \R \\
		(x,y) &\longmapsto f(x,y),
	\end{align*}
	\begin{align*}
		g: \overbrace{\R^+_* \times ]0, 2\pi[}^{=V} &\longrightarrow \R \\
		(r, \theta) &\longmapsto f(r\cos\theta, r\sin\theta).
	\end{align*}

	\begin{align*}
		\forall (r_0,\theta_0) \in V,&\\[5mm]
		\frac{\partial g}{\partial r}(r_0, \theta_0) &= \frac{\partial f}{\partial x}(r_0\cos\theta_0, r_0\sin\theta_0)\cos\theta_0\\
		&+ \frac{\partial f}{\partial y}(r_0 \cos\theta_0, r_0\sin\theta_0)\sin\theta_0\\
		&= 2r_0\cos^2\theta_0 + 2r_0\sin^2(\theta_0) \\
		&= 2r_0 \\[5mm]
		\frac{\partial g}{\partial \theta}(r_0, \theta_0) &= \frac{\partial f}{\partial x}(r_0\cos\theta_0, r_0\sin\theta_0)r_0\sin\theta_0\\
		&+ \frac{\partial f}{\partial y}(r_0 \cos\theta_0, r_0\sin\theta_0)r_0\cos\theta_0\\
		&= -2{r_0}^2\cos(\theta_0)\sin(\theta_0) + 2{r_0}^2 \sin(\theta_0)\cos(\theta_0)\\
		&= 0 \\
	\end{align*}

	Donc, \[
		g(r, \theta) = r^2.
	\]
\end{exm}

\begin{exm}
	Résoudre \[
		\begin{cases}
			\frac{\partial f}{\partial x} = \frac{x}{x^2+y^2},\\
			\frac{\partial f}{\partial y} = \frac{y}{x^2+y^2}.\\
		\end{cases}
	\]

	On pose $g: (r, \theta) \mapsto f(r \cos\theta, r \sin\theta)$.

	\begin{align*}
		&\frac{\partial g}{\partial r} = \frac{1}{r}\cos^2\theta + \frac{1}{r}\sin^2\theta = \frac{1}{r},\\
		&\frac{\partial g}{\partial \theta} = -\cos(\theta) \sin(\theta) + \sin(\theta)\cos(\theta) = 0.
	\end{align*}

	Donc, \[
		\exists C \in \R, g: (r, \theta) \mapsto \ln r + C
	\] d'où,
	\begin{align*}
		\forall (x,y) \in \R^2 \setminus \{(0,0)\}, f(x,y) &= \ln\left(\sqrt{x^2 + y^2} \right)  + C\\
		&= \frac{1}{2}\ln(x^2 + y^2) + C. \\
	\end{align*}
\end{exm}

\begin{rmk}
	Soit $\mathcal{B} = (e_1, e_2)$ la base canonique de $\R^2$, $f: U \to \R$ de classe $\mathcal{C}^1$ avec $U$ un ouvert de $\R^2$.

	Soit $(x,y) \in U$.

	\begin{align*}
		\Mat_{\mathcal{B}}\big(\nabla f(x,y)\big) = \begin{pmatrix}
			\frac{\partial f}{\partial x}(x,y)\\[2mm]
			\frac{\partial f}{\partial y}(x,y)
		\end{pmatrix}
	\end{align*}

	Soit  \begin{align*}
		\varphi: V &\longrightarrow U \\
		(u,v) &\longmapsto \big(x(u,v), y(u,v)\big) 
	\end{align*} avec $x,y$ de classe $\mathcal{C}^1$. Soit $g = f \circ \varphi$.
	\begin{align*}
		\Mat_{\mathcal{B}}\big(\nabla g(u,v)\big)
		&= \begin{pmatrix}
			\frac{\partial g}{\partial u}(u,v) \\[2mm]
			\frac{\partial g}{\partial v}(u,v)
		\end{pmatrix} \\
		&= \begin{pmatrix}
			\frac{\partial x}{\partial u}(u,v) \frac{\partial f}{\partial x}(x,y)
			+ \frac{\partial y}{\partial u}(u,v)\frac{\partial f}{\partial y}(x,y)\\[3mm]
			\frac{\partial x}{\partial v}(u,v) \frac{\partial f}{\partial x}(x,y)
			+ \frac{\partial y}{\partial v}(u,v) \frac{\partial f}{\partial y}(x,y)
		\end{pmatrix}  \\
		&= \underbrace{\begin{pmatrix}
				\frac{\partial x}{\partial u}(u,v)& \frac{\partial y}{\partial u}(u,v)\\[3mm]
				\frac{\partial x}{\partial v}(u,v)& \frac{\partial y}{\partial v}(u,v)
		\end{pmatrix}}_{J(u,v)} \begin{pmatrix}
			\frac{\partial f}{\partial x}(x,y)\\[3mm]
			\frac{\partial f}{\partial y}(x,y)
		\end{pmatrix} \\
		&= J(u,v) \Mat_{\mathcal{B}}\big(\nabla f(x,y)\big) \\
	\end{align*}
	où $J(u,v) = 
	\begin{pNiceArray}{c:c}
		\Mat_{\mathcal{B}}\big(\nabla x(u,v)\big) & \Mat_{\mathcal{B}}\big(\nabla y(u,v)\big)
	\end{pNiceArray}$.

	On dit que $J(u,v)$ est \underline{la jacobienne} de $\varphi$ en $(u,v)$.
	L'application linéaire canoniquement associée à $J(u,v)$ est la \underline{différentielle de $\varphi$} en $(u,v)$ noté $\mathrm{d}\varphi(u,v)$.

	On a $\mathrm{d}\varphi(u,v) \in \mathcal{L}(R^2)$ et $\Mat_{\mathcal{B}}\big(\mathrm{d}\varphi(u,v)\big) = J(u,v)$.

	Par exemple, la jacobienne du changement de coordonnées polaires est \[
		J = \begin{pmatrix}
			\frac{\partial x}{\partial r} & \frac{\partial y}{\partial r}\\[3mm]
			\frac{\partial x}{\partial \theta} & \frac{\partial y}{\partial \theta}
		\end{pmatrix}
		= \begin{pmatrix}
			\cos\theta&\sin\theta\\
			-r\sin\theta&r\cos\theta
		\end{pmatrix}.
	\]
	$\underbrace{\det(J)}_{\text{le jacobien}} = r\cos^2\theta + r\sin^2\theta = r$

	Dans une intégrale double, si $(x,y) = \varphi(u,v)$, alors $\mathrm{d}x\mathrm{d}y = \det(J)\mathrm{d}u\mathrm{d}v$.

	Ici, \[
		\mathrm{d}x\ \mathrm{d}y = r\ \mathrm{d}r\ \mathrm{d}\theta.
	\]
\end{rmk}

\begin{prv}
	On pose $(x_0, y_0) = \varphi(u_0, v_0)$. Pour tout $(h,k) \in \R^2$ tels que $(u_0 + h, v_0 + k) \in V$, en posant $g = f  \circ \varphi$.

	\begin{align*}
		g(u_0 + h, v_0 + h) &= f\big(x(u_0 + h, v_0 + k), y(u_0 + h, v_0 + k)\big) \\
		&= f\left(
			x(u_0,v_0) + h \frac{\partial x}{\partial u}(u_0,v_0) + k \frac{\partial x}{\partial v}(u_0, v_0) + \po\big(\|(h,k)\|\big), \right.\\
		&\phantom{ = f\bigg(\bigg.}\left. y(u_0, v_0) + h \frac{\partial y}{\partial u}(u_0, v_0) + k \frac{\partial y}{\partial v}(u_0, v_0) + \po\big(\|(h,k)\|\big)
		\right)  \\
		&= f(x_0,y_0) \\
		&~+ \left( h \frac{\partial x}{\partial u}(u_0,v_0) + k \frac{\partial x}{\partial v}(u_0, v_0) + \po(\|(h,k)\|) \right) \frac{\partial f}{\partial x}(x_0,y_0)\\
		&~+ \left( h \frac{\partial y}{\partial u}(u_0, v_0) + k\frac{\partial y}{\partial v}(u_0, v_0) + \po(\|(h,k)\|) \right) \frac{\partial f}{\partial y}(x_0, y_0)\\
		&~+ \po(\|(h,k)\|)\\
		&= f(x_0, y_0) \\
		&~+ h \left( \frac{\partial x}{\partial u}(u_0, v_0) \frac{\partial f}{\partial x}(x_0, y_0) + \frac{\partial y}{\partial u}(u_0, v_0) \frac{\partial f}{\partial y}(x_0, y_0) \right)  \\
		&~+ k\left( \frac{\partial x}{\partial v}(u_0, v_0) \frac{\partial f}{\partial x}(x_0, y_0) + \frac{\partial y}{\partial v}(u_0, v_0) \frac{\partial f}{\partial y}(x_0, y_0) \right) 
		&~+ \po(\|(h,k)\|)\\
		&= g(u_0, v_0) + h \frac{\partial g}{\partial u}(u_0, v_0) + k \frac{\partial g}{\partial v}(u_0, v_0) + \po(\|(h,k)\|) \\
	\end{align*}

	Par identification,
	\[
		\frac{\partial g}{\partial u}(u_0, v_0) = \frac{\partial x}{\partial u}(u_0, v_0) \frac{\partial f}{\partial x}(x_0, y_0) + \frac{\partial y}{\partial u}(u_0, v_0) \frac{\partial f}{\partial y}(x_0,y_0)
	\] et \[
		\frac{\partial g}{\partial v}(u_0, v_0) = \frac{\partial x}{\partial v}(u_0,v_0) \frac{\partial f}{\partial x}(x_0, y_0) + \frac{\partial y}{\partial v}(u_0, v_0) \frac{\partial f}{\partial y}(x_0, y_0).
	\] 
\end{prv}

\begin{exm}
	[Régression linéaire]~\\
	\begin{figure}[H]
		\centering
		\begin{asy}
			import graph;
			axes(EndArrow);
			size(5cm);

			real f(real x) { return x + 0.5; }

			real k = 35 / (7 - 0.5);

			for(int i = 0; i < 35; ++i) {
				real mag = exp(sin(100 * pi/exp(1) * i)) * 0.8 + exp(cos(i*40)/3);
				real eps = mag * cos(10 * exp(1)/pi * i) / 3;
				dot((i/k,f(i/k) + eps));
			}

			draw(graph(f, -1, 7), orange);
		\end{asy}
	\end{figure}
	\[
		y = a x + b
	\] 
	On fixe $(a,b) \in \R^2$. \[
		\varepsilon(a,b) = \sum_{i=1}^n\big( y_i - (ax_i + b) \big)^2
	\] l'erreur totale.

	On veut minimiser $\varepsilon(a,b)$. On a 
	\[
		\forall (a,b) \in \R^2,
		\begin{cases}
			\frac{\partial \varepsilon}{\partial a}(a,b) = -2\sum_{i=1}^{n}(y_i - ax_i - b)x_i,\\
			\frac{\partial \varepsilon}{\partial b}(a,b) = -2\sum_{i=1}^{n}(y_i - ax_i - b).
		\end{cases}
	\]

	Donc,
	\begin{align*}
		(a,b) \text{ point critique de } \varepsilon \iff& \begin{cases}
			a \sum_{i=1}^n {x_i}^2 + b\sum_{i=1}^{n}x_i = \sum_{i=1}^{n} y_ix_i\\
			a\sum_{i=1}^{n}x_i + nb = \sum_{i=1}^ny_i
		\end{cases}\\
		\iff& \begin{cases}
			a \left( \frac{1}{n}\sum_{i=1}^n {x_i}^2 - \overline{x}^2\right) = \overline{y} - \overline{x} \overline{y}\\
			b = \frac{1}{n}\sum_{i=1}^ny_i - \frac{a}{n}\sum_{i=1}^nx_i = \frac{1}{n}\sum_{i=1}^n x_i y_i - \overline{x} \overline{y}
		\end{cases}\\
		&\text{ où } \overline{x} = \frac{1}{n} \sum_{i=1}^n x_i,~\overline{y} = \frac{1}{n}\sum_{i=1}^n y_i\\
		\iff& \begin{cases}
			a = \frac{\Cov(x,y)}{V(x)}\\
			b = \overline{y} - a\overline{x}
		\end{cases}
	\end{align*}

	Coefficient de corrélation: $\frac{\Cov(x,y)}{\sigma_x \sigma_y} \in [-1, 1]$
\end{exm}












		\part{Corps}

\begin{exm}[Problème]
	\begin{itemize}
		\item 
			avec $A = \Z / 9 \Z$, résoudre $\overline{x}^2 = \overline{0}$ \\
			\begin{center}
				\begin{tabular}{|c|c|c|c|c|c|c|c|c|c|c|}
					\hline
					$\overline{x}$&$\overline{0}$& $\overline{1}$ &$\overline{2}$&$\overline{3}$ &$\overline{4}$ &$\overline{5}$ &$\overline{6}$ &$\overline{7}$ &$\overline{8}$& $\overline{9}$ \\
					\hline
					$\overline{x}^2$&$\overline{0}$ &$\overline{1}$ &$\overline{4}$ &$\overline{0}$ &$\overline{7}$ &$7$ &$\overline{0}$ &$\overline{4}$ &$\overline{1}$&$\overline{0}$\\
					\hline
				\end{tabular}
			\end{center}
			On a trouvé 3 solutions: $\overline{0}$, $\overline{3}$, $\overline{6}$.
		\item $\Z / 8\Z$
			\begin{center}
				\begin{tabular}{|c|c|c|c|c|c|c|c|c|}
					\hline
					$\overline{x}$& $\overline{0}$& $\overline{1}$& $\overline{2}$& $\overline{3}$& $\overline{4}$& $\overline{5}$& $\overline{6}$& $\overline{7}$\\
					\hline
					$\overline{x^2}$& $\overline{0}$& $\overline{1}$& $\overline{4}$& $\overline{1}$& $\overline{0}$& $\overline{1}$& $\overline{4}$& $\overline{1}$\\
					\hline
				\end{tabular}
			\end{center}
			$\overline{x}^2=7$ a 4 solutions: $\overline{1}, \overline{7}, \overline{3},\text{ et } \overline{5}$
		\item $A = \mathbbm{H} = \{a + bi + cj + dk  \mid  (a,b,c,d) \in \R^4\}$ \\
			$i^2 = j^2 = k^2 = -1$ 
			\begin{align*}
				\begin{array}{c c c}
					ij = k & jk = i & ji = j\\
					ji = -k & kj = -i & ik = -j
				\end{array}
			\end{align*}
			Dans cet anneau, $-1$ a 6 racines!
	\end{itemize}
\end{exm}

\begin{defn}
	Soit $(\mathbbm{K}, +, \times)$ un ensemble muni de deux lois de composition internes. On dit que c'est un \underline{corps} si
	 \begin{enumerate}
		\item $(\mathbbm{K}, \times)$ est un groupe abélien
		\item $(\mathbbm{K}, \times)$ est un monoïde commutatif
		\item $\forall x \in \mathbbm{K}\setminus \{0_\mathbbm{K}\}, \exists y \in \mathbbm{K}, xy = 1_\mathbbm{K}$
		\item $0_\mathbbm{K} \neq  1_\mathbbm{K}$
	\end{enumerate}
	\index{corps}
\end{defn}

\begin{exm}
	\begin{itemize}
		\item $(\C, +, \times)$ est un corps
		\item $(\R, +, \times)$ est un corps
		\item $(\Q, +, \times)$ est un corps
		\item $(\Z, +, \times)$ n'est pas un corps
	\end{itemize}
\end{exm}

\begin{prop}
	$(\Z / n\Z, +, \times)$ est un corps si et seulement si $n$ est premier.
\end{prop}

\begin{prv}
	\[
		\left( \Z / n\Z \right)^\times = \left\{ \overline{k}  \mid k \wedge n = 1 \right\}
	\] 
\end{prv}


\begin{prop}
	Tout corps est un anneau intègre.
\end{prop}

\begin{prv}
	Soit $(\mathbbm{K}, +, \times)$ un corps. Soient $(a,b) \in \mathbbm{K}^2$ tel que $a \times b = 0_\mathbbm{K}$.\\
	On suppose $a \neq  0_\mathbbm{K}$. Alors, $a$ est inversible et donc \[
		b = a^{-1} \times a \times b = a^{-1} \times 0_\mathbbm{K} = 0_\mathbbm{K}
	\] 
\end{prv}

\begin{exm}
	Soit $(\mathbbm{K},+,\times)$ un corps.\\
	Résoudre \[
		\begin{cases}
			x^2 = 1_\mathbbm{K}\\
			x \in \mathbbm{K}
		\end{cases}
	\]

	\begin{align*}
		x^2 = 1_\mathbbm{K} &\iff x^2 - 1_\mathbbm{K} = 0_\mathbbm{K}\\
		&\iff (x - 1_\mathbbm{K})(x+1_\mathbbm{K}) = 0_\mathbbm{K}\\
		&\iff x - 1_\mathbbm{K} = 0_\mathbbm{K} \text{ ou } x + 1_\mathbbm{K} = 0_\mathbbm{K}\\
		&\iff x = 1_\mathbbm{K} \text{ ou } x = -1_\mathbbm{K}
	\end{align*}

	Il y a au plus 2 solutions.
\end{exm}

\begin{prop}
	Soit $(\mathbbm{K},+,\times )$ un corps et $P$ un polynôme à coefficients dans $\mathbbm{K}$ de degré $n$. Alors, l'équation $P(x) = 0_{\mathbbm{K}}$ a au plus $n$ solutions dans $\mathbbm{K}$ 
	\qed
\end{prop}

\begin{crlr}[(Théorème de Wilson)]
	voir exercice 16 du TD 12
\end{crlr}


\begin{defn}
	Soit $(\mathbbm{K}, +, \times)$ un corps et $L\subset \mathbbm{K}$.\\
	On dit que $L$ est un \underline{sous corps} de $\mathbbm{K}$ si
	\begin{enumerate}
		\item $L$ est un anneau de $(\mathbbm{K}, +, \times)$ non nul
		\item $\forall x \in L\setminus \{0_\mathbbm{K}\}, x^{-1} \in L$ 
	\end{enumerate}
	\vspace{2mm}
	en d'autres termes si
	\begin{enumerate}
		\item $\forall (x,y) \in L^2, x - y \in L$
		\item $\forall (x,y) \in L^2, x \times y^{-1} \in L$
	\end{enumerate}
	\vspace{5mm}
	On dit aussi que $\mathbbm{K}$ est une \underline{extension} de $L$.
	\index{sous corps}
	\index{extension}
\end{defn}

\begin{prop}
	Tout sous corps est un corps. \qed
\end{prop}

\begin{defn}
	Soient $(\mathbbm{K}_1,+,\times )$ et $(\mathbbm{K}_2,+, \times)$ deux corps et $f: \mathbbm{K}_1 \to \mathbbm{K}_2$.\\
	On dit que $f$ est un \underline{morphisme de corps} si $f$ est un morphisme d'anneaux.\\
	i.e. si
	\[
		\begin{cases}
			\forall (x,y) \in {\mathbbm{K}_1}^2,& f(x+y) = f(x) + f(y)\\
			\forall (x,y) \in {\mathbbm{K}_1}^2,& f(x \times y) = f(x) \times f(y)\\
		\end{cases}
	\] 
	\index{homomorphisme (de corps)}
	\index{morphisme (de corps)}
\end{defn}

\begin{prop}
	Tout morphisme de corps est injectif.
\end{prop}

\begin{prv}
	Soit $f: \mathbbm{K}_1 \to \mathbbm{K}_2$ un morphisme de corps.\\
	\begin{itemize}
		\item $\Ker(f)$ est un sous groupe de $(\mathbbm{K}_1, +)$ 
		\item Soit $x \in \Ker(f)$ et $y \in \mathbbm{K}_1$ \[
				f(x \times y) = f(x) \times f(y) = 0_{\mathbbm{K}_2} \times f(y) = 0_{\mathbbm{K}_2}
			\]
		\item Soit $x \in \Ker(f) \setminus \{0_{\mathbbm{K}_1}\}$.\\
			Alors, $x$ est inversible.\\
			\begin{align*}
				\begin{rcases*}
					x \in \Ker(f)\\
					x^{-1} \in \mathbbm{K}_1
				\end{rcases*}& \text{ donc } x \times x ^{-1} \in \Ker(f)\\
				&\text{ donc } 1_{\mathbbm{K}_1} \in \Ker(f)\\
				&\text{ donc } f(1_{\mathbbm{K}_1}) = 0_{\mathbbm{K}_2}
			\end{align*}
			Or, $f(1_{\mathbbm{K}_1}) = 1_{\mathbbm{K}_2} \neq 0_{\mathbbm{K}_2}$
	\end{itemize}
	Donc, $\Ker(f) = \{0_{\mathbbm{K}_1}\}$ donc $f$ est injective.
\end{prv}

\begin{exm}
	$\begin{array}{cc}
		\C &\longrightarrow \C\\
		z &\longmapsto \overline{z}\\
	\end{array}$ est un morphisme de corps
\end{exm}



		\part{Opérations sur les séries}

\begin{prop}
	L'ensemble $E = \{u \in \C^\N  \mid \Sigma u_n \text{ converge}\}$ est un sous-espace vectoriel de $\C^\N$ et \begin{align*}
		S: E &\longrightarrow \C \\
		u &\longmapsto \sum_{n=0}^{+\infty} u_n
	\end{align*} est une forme linéaire.
	\qed
\end{prop}

\begin{rmk}
	La somme d'une série convergente et d'une série divergente diverge.
	Le produit d'une série divergente par un scalaire non nul diverge.
\end{rmk}

		\addrecap
	}

	{
		\chap[13]{Systèmes linéaires et calculs matriciels}
		\renewcommand{\cwd}{../chap13}
		\begin{defn}
	Soit $E$ un $\mathbbm{K}$-espace vectoriel. On dit que $E$ est de \underline{dimension finie} si $E$ a au moins une famille génératrice finie. On dit que $E$ est de \underline{dimension infinie} sinon.
	\index{dimension finie (espace vectoriel)}
	\index{dimension infinie (espace vectoriel)}
\end{defn}

\begin{thm}
	[Théorème de la base extraite]
	Soit $E$ un $\mathbbm{K}$-espace vectoriel non nul de dimension finie. Soit $\mathcal{G}$ une famille génératrice finie de $E$. Alors, il existe une base $\mathcal{B}$ de $\mathcal{E}$ telle que $\mathcal{B} \subset \mathcal{G}$.
\end{thm}

\begin{prv}
	[par récurrence sur $\#G = \Card(G)$]
	\begin{itemize}
		\item Soit $E$ un $\mathbbm{K}$-espace vectoriel non nul engendré par $\mathcal{G} = (u)$.\\
			Si $u = 0_E$, alors $E = \{0_E\}$: une contradiction $\lightning$ \\
			Donc $u \neq 0_E$ donc $(u)$ est libre. En effet, \[
				\forall \lambda \in \mathbbm{K}, \lambda u = 0_E \implies \lambda = 0_\mathbbm{K}
			\] Donc $\mathcal{G}$ est une base de $E$.\\
		\item Soit $n \in \N_*$. Soit $E$ un $\mathbbm{K}$-espace vectoriel. On suppose que si $E$ a une famille génératrice constituée de $n$ vecteurs, alors on peut extraire de cette famille une base de $E$.\\
			Soit $\mathcal{G}$ une famille génératrice de $E$ avec $n+1$ vecteurs.\\
			Si $\mathcal{G}$ est libre, alors $\mathcal{G}$ est une base de $E$. \\
			Si $\mathcal{G}$ n'est pas libre, alors il existe $u \in \mathcal{G}$ tel que $u \in \Vect(\mathcal{G}\setminus \{u\})$ \\
			Donc $\mathcal{G}\setminus \{u\}$ engendre $E$. Or, $\mathcal{G}\setminus \{u\}$ possède $n$ vecteurs. D'après l'hypothèse de récurrence, il existe une base $\mathcal{B}$ de $E$ telle que \[
				\mathcal{B} \subset \mathcal{G} \setminus \{u\} \subset \mathcal{G}
			\] 
	\end{itemize}
\end{prv}

\begin{crlr}
	Tout espace de dimension finie a une base.
	\qed
\end{crlr}

\begin{thm}
	[Théorème de la base incomplète]
	Soit $E$ un $\mathbbm{K}$-espace vectoriel de dimension finie, $\mathcal{G}$ une famille génératrice finie de $E$. $\mathcal{L}$ une famille libre de $E$. Alors, il existe une base $\mathcal{B}$ de $E$ telle que \[
		\mathcal{L} \subset \mathcal{B} \text{ et } \mathcal{B}\setminus \mathcal{L} \subset \mathcal{G}
	\] 
\end{thm}

\begin{prv}
	[par récurrence sur $\#(\mathcal{G}\setminus\mathcal{L})$]
	\begin{itemize}
		\item Avec les notations précédentes, on suppose que $\mathcal{G}\setminus\mathcal{L} \neq \O$ \[
				\forall u \in \mathcal{G}, u \in \mathcal{L}
			\] Donc $\mathcal{G} \subset \mathcal{L}$ donc $\mathcal{L}$ est génératrice donc $\mathcal{L}$ est une base de $E$. On pose $\mathcal{B} = \mathcal{L}$ et alors \[
				\mathcal{L} \subset  \mathcal{B} \text{ et } \mathcal{B}\setminus\mathcal{L} = \O \subset  \mathcal{G}
			\] 
		\item Soit $n \in \N$. On suppose que si $\mathcal{G}$ est génératrice et $\mathcal{L}$ libre avec $\#(\mathcal{G}\setminus\mathcal{L}) = n$ alors il existe une base $\mathcal{B}$ de $E$ telle que \[
			\mathcal{L}\subset \mathcal{B} \text{ et } \mathcal{B}\setminus\mathcal{L}\subset \mathcal{G}
		\] Soient à présent $\mathcal{G}$ une famille génératrice de $E$ et $\mathcal{L}$ une famille libre de $E$ telles que $\#(\mathcal{G}\setminus\mathcal{L}) = n+1 > 0$\\
		Si $\mathcal{L}$ engendre $E$, alors $\mathcal{L}$ est une base de $E$. On pose $\mathcal{B} = \mathcal{L}$ et on a bien \[
			\mathcal{L} \subset  \mathcal{B} \text{ et } \mathcal{B} \setminus \mathcal{L} = \O \subset  \mathcal{G}
		\] On suppose que $\mathcal{L}$ n'engendre pas $E$. Il existe $u \in \mathcal{G}$ tel que $u \not\in \Vec(\mathcal{L})$ (car sinon, $\mathcal{G} \subset \Vect(\mathcal{L})$ et donc $\underbrace{\Vect(\mathcal{G})}_{= E} \subset  \underbrace{\Vect(\mathcal{L})}_{ \subset E}$\\
		Donc $\mathcal{L} \cup \{u\} $ est libre. On pose $\mathcal{L}' = \mathcal{L} \cup \{u\} $ \[
			\mathcal{G}\setminus \mathcal{L}' = \mathcal{G}\setminus (\mathcal{L} \cup \{u\}) = (\mathcal{G}\setminus\mathcal{L})\setminus \{u\} 
		\] donc $\#(\mathcal{G}\setminus\mathcal{L}') = n+1 -1 = n$\\
		D'après l'hypothèse de récurrence, il existe $\mathcal{B}$ une base de $E$ telle que \[
			\mathcal{L} \subset  \mathcal{L}' \subset \mathcal{B} \text{ et } \mathcal{B}\setminus \mathcal{L}' \subset \mathcal{G}
		\] \[
			\mathcal{B} \setminus \mathcal{L} = \underbrace{\mathcal{B}\setminus\mathcal{L}'}_{\subset \mathcal{G}} \cup \underbrace{\{u\}}_{\subset \mathcal{G} \text{ car } u \in \mathcal{G}}
		\] On a $\mathcal{B}\setminus\mathcal{L}\subset \mathcal{G}$
	\end{itemize}
\end{prv}

\begin{thm}
	Soit $E$ un $\mathbbm{K}$-espace vectoriel de dimension finie. Toutes les bases de $E$ ont le même cardinal.
\end{thm}

\begin{prv}
	Soit $\mathcal{G}$ une famille génératrice finie de $E$ et $\mathcal{B} \subset  \mathcal{G}$ une base de $E$. On note $n = \#\mathcal{B}$ \\
	Soit $\mathcal{B}'$ une base de $E$. On pose $p = n - \#(\mathcal{B} \cap  \mathcal{B}')$. Montrons par récurrence sur  $p$ que $\#\mathcal{B} = \#\mathcal{B}'$ 
	\begin{itemize}
		\item On suppose que $p = 0$. Alors, $\#(\mathcal{B} \cap \mathcal{B}') = n$ \\
			Or, $\mathcal{B}' \cap \mathcal{B} \subset \mathcal{B}$ donc $\mathcal{B} \cap \mathcal{B}' = \mathcal{B}$ donc $\mathcal{B} \subset  \mathcal{B}'$ et donc $\mathcal{B} = \mathcal{B}'$ 
		\item Soit $p \in \N$. On suppose que si $\mathcal{B}'$ est une base de $E$ telle que $n - \#(\mathcal{B} \cap \mathcal{B}') = p$, alors $\#\mathcal{B}' = n$ \\
			Aoit $\mathcal{B}'$ une base de $E$ telle que $n - \#(\mathcal{B}\cap \mathcal{B}') = p+1 > 0$ \\
			Donc $\mathcal{B} \cap \mathcal{B}' \neq \mathcal{B}$. Soit $u \in \mathcal{B}' \setminus \mathcal{B}$. D'après le lemme d'échange, il existe $v \in \mathcal{B}\setminus \mathcal{B}'$ tel que $\mathcal{B}' \setminus \{u\} \cup \{v\}$ est une base de $E$. On pose $\mathcal{B}'' = \mathcal{B}' \setminus \{u\} \cup \{v\}$ 
			\begin{align*}
				\mathcal{B}'' \cap \mathcal{B} &= \left( (\mathcal{B}' \setminus \{u\})  \cap \mathcal{B} \right) \cup \{v\} \\
				&= (\mathcal{B}' \cap \mathcal{B}) \cup \{v\} \\
			\end{align*}
			donc,
			\begin{align*}
				n - \#(\mathcal{B}'' \cap \mathcal{B}) &= n - (\#(\mathcal{B}' \cap \mathcal{B}) + 1) \\
				&= p+1- 1 \\
				&= p \\
			\end{align*}
			D'après l'hypothèse de récurrence, \[
				\#\mathcal{B}'' = n
			\] Or, $\#\mathcal{B}'' = \#\mathcal{B}'$
	\end{itemize}
\end{prv}

\begin{lem}
	Soient $\mathcal{B}$ et $\mathcal{B}'$ deux bases de $E$ telles que $\mathcal{B}\subset \mathcal{B}'$. Alors, $\mathcal{B} = \mathcal{B}'$.
\end{lem}

\begin{prv}
	On suppose $\mathcal{B}' \neq \mathcal{B}$. Soit $u \in \mathcal{B}' \setminus \mathcal{B}$
	$u \in E = \Vect(\mathcal{B})$ donc $\mathcal{B} \cup \{u\}$ n'est pas libre.
	Donc $\mathcal{B}\cup \{u\} \subset \mathcal{B}'$ et $\mathcal{B}'$ est libre donc $\mathcal{B}\cup \{u\}$ est libre: une contradiction $\lightning$
\end{prv}

\begin{lem}
	[Lemme d'échange] Soient $\mathcal{B}_1$ et $\mathcal{B}_2$ deux bases de $E$ et $u \in \mathcal{B}_1 \setminus \mathcal{B}_2$. Alors, il existe $v \in \mathcal{B}_2$ tel que $(\mathcal{B}_1 \setminus \{u\}) \cup \{v\}$ soit une base de $E$.
\end{lem}

\begin{prv}
	[1${}^\text{nde}$ méthode]
	On suppose que pout tout $v \in \mathcal{B}_2$, $(\mathcal{B}_1\setminus \{u\}) \cup \{v\}$ n'est pas une base de $E$
	Soit $v \in \mathcal{B}_2$.
	\begin{itemize}
		\item Supposons $(\mathcal{B}_1\setminus \{u\})\cup \{v\}$ non libre. $\mathcal{B}_1 \setminus \{u\}$ est libre. Donc $v \in \Vect(\mathcal{B}_1 \setminus \{u\})$
		\item Supposons $(\mathcal{B}_1\setminus \{u\}) \cup \{v\}$ non génératrice.
			Comme $\mathcal{B}_1$ engendre $E$, $u \not\in \Vect(\mathcal{B}_1\setminus \{v\})$.
			On suppose que $\mathcal{B}_1 \neq \mathcal{B}_2$.
			$\forall v \in \mathcal{B}_2 \setminus \mathcal{B}_1, \Vect(\mathcal{B}_1 \setminus \{v\}) = \Vect(\mathcal{B}_1) = E \ni u$ 
			donc, $(\mathcal{B}_1\setminus \{u\}) \cup \{v\}$ engendre $E$ et donc \[
				v \in \Vect(\mathcal{B}_1 \setminus \{u\})
			\] On a aussi \[
				\forall v \in \mathcal{B}_1 \setminus \{u\}, v \in \Vect(\mathcal{B}_1\setminus \{u\})
			\] Comme $u \not\in \mathcal{B}_2$, on a \[
				\forall v \in \mathcal{B}_2, v \in \Vect(\mathcal{B}_1\setminus \{u\})
			\] docn \[
				E = \Vect(\mathcal{B}_2) \subset \Vect(\mathcal{B}_1\setminus \{u\})
			\] donc $\mathcal{B}_1\setminus \{u\}$ engendre $E$ donc $\mathcal{B}_1\setminus \{u\}$ est une base de $E$. Or, $\mathcal{B}_1 \setminus \{u\}  \subset  \mathcal{B}_1$, donc $\mathcal{B}_1\setminus \{u\} = \mathcal{B}_1$
	\end{itemize}
\end{prv}

\begin{prv}
	[2${}^\text{nde}$ méthode]
	On suppose que pout tout $v \in \mathcal{B}_2$, $(\mathcal{B}_1\setminus \{u\}) \cup \{v\}$ n'est pas une base de $E$
	\begin{itemize}
		\item Comme $u \in \mathcal{B}_1 \setminus \mathcal{B}_2$, nécéssairement $\mathcal{B}_1 \neq \mathcal{B}_2$ donc $\mathcal{B}_2 \not\subset \mathcal{B}_1$, donc $\mathcal{B}_2\setminus\mathcal{B}_1 \neq \O$ 
		\item Soit $v \in \mathcal{B}_2\setminus\mathcal{B}_1$. Il existe $(\lambda_w)_{w\in\mathcal{B}_1}$ une famille de scalaires presque nulle telle que \[
				v = \sum_{w \in \mathcal{B}_1} \lambda_w w - \lambda_u u + + \sum_{w \in \mathcal{B}_1\setminus \{u\}}\lambda_w w
			\]
			Si $\lambda_u \neq 0_E$, alors
			\begin{align*}
				u &= \lambda_u^{-1}\left( v - \sum_{w \in \mathcal{B}_1 \setminus \{u\}} \lambda_w w \right)\\
					&\in \Vect(\mathcal{B}_1\setminus \{u\} \cup v)
			\end{align*}
			 donc $\mathcal{B}_1 \subset \Vect(\mathcal{B}_1\setminus \{u\} \cup \{v\})$\\
			 et donc $E \subset  \Vect(\mathcal{B}_1 \setminus \{u\} \cup \{v\})$ \\
			 et donc $\mathcal{B}_1 \setminus \{u\} \cup \{v\}$ engendre $E$ \\
			 donc $\mathcal{B}_1 \setminus \{u\} \cup \{v\}$ n'est pas libre\\
			 donc $v \in \Vect(\mathcal{B}_1\setminus \{u\})$ (car $\mathcal{B}_1 \setminus \{u\}$ est libre\\
			 donc $\lambda_u = 0_\mathbbm{K}$ $\lightning$\\`

			 Donc, $\lambda_u = 0_\mathbbm{K}$, docn $v \in \Vect(\mathcal{B}_1\setminus \{u\})$ \\
			 On vient de prouver que
			 \begin{align*}
			 	\mathcal{B}_2 \setminus \mathcal{B}_1 \subset \Vect(\mathcal{B}_1 \setminus \{u\})\\
			 	\mathcal{B}_1 \setminus \{u\} \subset \Vect(\mathcal{B}_1 \setminus \{u\})\\
			 \end{align*}
			 Comme $u \not\in \mathcal{B}_2$, \[
			 	\mathcal{B}_2 \subset \Vect(\mathcal{B}_1 \setminus \{u\})
			 \] donc \[
			 	E = \Vect(\mathcal{B}_2) \subset  \Vect(\mathcal{B}_1 \setminus \{u\})
			 \] donc $\mathcal{B}_1 \setminus \{u\}$ engendre $E$. Donc,  $\mathcal{B}_1 \setminus \{u\}$ est une base de $E$.\\
			 Or, $\mathcal{B}_1 \setminus \{u\} \subset  \mathcal{B}_1$, donc $\mathcal{B}_1 \setminus \{u\} = \mathcal{B}_1$
	\end{itemize}
\end{prv}

\begin{defn}
	Soit $E$ un $\mathbbm{K}$-espace vectoriel de dimension finie. Le cardinal commun à toutes les bases de $E$ est appelé \underline{dimension} de $E$ est notée $\dim(E)$ ou $\dim_\mathbbm{K}(E)$\\
	C'est donc aussi le nombre de coordonnées de n'importe quel vecteur dans n'importe quelle base.
	\index{dimension (espace vectoriel)}
\end{defn}

\begin{exm}
	\begin{enumerate}
		\item $\dim_\R(\C) = 2$ et $\dim_\C(\C) = 1$ 
		\item $\dim_\mathbbm{K}(\mathbbm{K}^{n}) = n$ 
		\item $\dim_{\mathbbm{K}}(\mathcal{M}_{n,p}(\mathbbm{K})) = np$
	\end{enumerate}
\end{exm}

\begin{crlr}
	Soit $E$ un $\mathbbm{K}$-espace vectoriel de dimension finie, $\mathcal{L}$ une famille libre de $E$, $\mathcal{G}$ une famille génératrice de $E$. On note $n = \dim(E)$
	\begin{enumerate}
		\item $\#\mathcal{G} \ge n$ et $(\#\mathcal{G} = n \implies \mathcal{G} \text{ est une base de } E$)
		\item $\#\mathcal{L} \le n$ et $(\#\mathcal{L} = n \implies \mathcal{L} \text{ est une base de } E$)
	\end{enumerate}
\end{crlr}

\begin{crlr}
	$\R^{\R}$ est de dimension infinie.
	$\forall i \in \N, e_i: x \mapsto x^i$\\
	$(e_i)_{i\in\N}$ est libre dans $\R^\R$
\end{crlr}

\begin{prop}
	Soient $E$ et $F$ deux $\mathbbm{K}$-espaces vectoriels de dimension finie. Alors $E\times F$ est de dimension finie et $\dim(E\times F) = \dim(E) + \dim(F)$
\end{prop}

\begin{prv}
	Soit $(e_1,\ldots, e_n)$ une base de $E$, $(f_1, \ldots, f_p)$ une base de $F$.
	On pose \[
		\left\{\begin{array}
			{r c l}
			u_1 &=& (e_1,0_F)\\
			u_2 &=& (e_2,0_F)\\
					&\vdots&\\
			u_n &=& (e_n,0_F)\\
			u_{n+1} &=& (0_E, f_1)\\
			u_{n+2} &=& (0_E, f_2)\\
					&\vdots&\\
			u_{n+p} &=& (0_E,f_p)\\
		\end{array}\right.
	\]
	Soit $(x,y) \in E\times F$. \[
		\begin{cases}
			\exists (x_1,\ldots,x_n)\in \mathbbm{K}^n, x = \sum_{i=1}^{n} x_ie_i
			\exists (y_1,\ldots,y_n)\in \mathbbm{K}^n, x = \sum_{j=1}^{p} y_jf_j
		\end{cases}
	\] 
	\begin{align*}
		(x,y) &= \left( \sum_{i=1}^{n} x_ie_i, \sum_{i=1}^{p} y_jf_j \right)  \\
		&= \sum_{i=1}^{n} x_i (e_i + 0_F) + \sum_{j=1}^{p} y_j (0_E, f_j) \\
		&= \sum_{i=1}^{n} x_i u_i + \sum_{j=1}^{p} y_j u_{n+j} \\
	\end{align*}
	Donc, $E\times F = \Vect(u_1, \ldots, u_{n+p})$ donc $E\times F$ est de dimension finie.\\
	Soit $(\lambda_1, \ldots, \lambda_{n+p}) \in \mathbbm{K}^{n+p}$ tel que \[
		(*): \quad \sum_{k=1}^{n+p} \lambda_ku_k = 0_{E\times F} = (0_E, 0_F)
	\]
	\begin{align*}
		(*) &\iff \sum_{k=1}^{n} \lambda_k (e_k, 0_F) + \sum_{k=n+1}^{p} \lambda_k(0_E, f_{k-n}) = (0_E, 0_F)\\
				&\iff \begin{cases}
					\sum_{k=1}^{n} \lambda_k e_k = 0_E\\
					\sum_{k=n+1}^{p} \lambda_k f_{k-n} = 0_F
				\end{cases}\\
				&\iff \begin{cases}
					\forall k \in \left\llbracket 1,n \right\rrbracket, \lambda_k = 0_\mathbbm{K} \qquad&(\text{car $(e_1,\ldots,e_n)$ est libre})\\
					\forall k \in \left\llbracket n+1,n+p \right\rrbracket, \lambda_k = 0_\mathbbm{K} \qquad&(\text{car $(f_1,\ldots,f_n)$ est libre})\\
				\end{cases}
	\end{align*}
	Donc $(u_1, \ldots, u_{n+p})$ est une base de $E\times F$. Donc, $\dim(E\times F) = n + p = \dim(E) + \dim(F)$
\end{prv}

\begin{rmk}
	[Convention]
	\[\dim\big(\{0_E\}\big) = 0\]
\end{rmk}

\begin{thm}
	Soit $E$ un $\mathbbm{K}$-espace vectoriel de dimension finie, $F$ un sous-espace vectoriel de $E$. Alors, $F$ est de dimension finie et  $\dim(F) \le \dim(E)$\\
	Si $\dim(F) = \dim(E)$, alors $F = E$
\end{thm}

\begin{prv}
	On considère \[
		A = \{k \in \N \mid \text{il existe une famille libre de $F$ à $k$ éléments}\} 
	\]
	On suppose $F \neq \{0_E\}$.
	\begin{itemize}
		\item Soit $u \in F\setminus \{0_E\}$. $(u)$ est libre donc $1 \in A$ et donc $A \neq \O$
		\item Soit $\mathcal{L}$ une famille libre de $F$. Alors, $\mathcal{L}$ est une famille libre de $E$ \\
			donc $\#\mathcal{L} \le \dim(E)$\\
			Donc $A$ est majorée par $\dim(E)$ \\
			On en déduit que $A$ a un plus grand élément $p$.
		\item Soit $\mathcal{L}$ une famille libre de $F$ avec $p$ éléments.\\
			Si $\mathcal{L}$ n'engendre pas $F$, alors il existe $u\in F$ tel que $u\not\in \Vect(\mathcal{L})$ et donc $\mathcal{L} \cup \{u\}$ est une famille libre de $F$, donc $p+1 \in A$ en contradiction avec la maximalité de $p$.\\
			Donc $\mathcal{L}$ est une base de $F$ donc $F$ est de dimension finie et $\dim(F) = p \le \dim(E)$\\
	\end{itemize}

	Soit $\mathcal{B}$ une base de $F$. Alors, $\mathcal{B}$ est aussi une famille de libre de de $E$. Donc $\#\mathcal{B} \le \dim(E)$ donc $\dim(F) = \dim(E)$ \\
	Si $\dim(F) = \dim(E)$, alors $\mathcal{B}$ est une base de $E$, et donc $F = \Vect(\mathcal{B}) = E$
\end{prv}

\begin{prop}
	[Formule de Grassmann]
	Soit $E$ un $\mathbbm{K}$-espace vectoriel de dimension finie, $F$ et $G$ deux sous-espace vectoriels de $E$. Alors, \[
		\dim(F+G) = \dim(F) + \dim(G) - \dim(F\cap G)
	\] 
\end{prop}

\begin{prv}
	Soit $(e_1, \ldots, e_p)$ une base de $F\cap G$. $(e_1,\ldots,e_p)$ est une famille libre de $F$.\\
	On complète $(e_1, \ldots, e_p)$ en une base $(e_1, \ldots, e_p, u_1, \ldots, u_q)$ de $F$.\\
	De même, on complète $(e_1, \ldots, e_p)$ en une base $(e_1, \ldots, e_p, v_1, \ldots, v_r)$ de $G$.\\
	On pose  $\mathcal{B} = (e_1, \ldots, e_p, u_1, \ldots, u_q, v_1, \ldots, v_r)$. Montrons que $\mathcal{B}$ est une base de $F+G$
	\begin{itemize}
		\item Soit $u \in F+G$ \\
			On pose $u = v+w$ avec $\begin{cases}
				v\in F\\
				w \in G
			\end{cases}$.\\
			On pose $v = \sum_{i=1}^p \lambda_i e_i + \sum_{i=1}^q \mu_i u_i$ avec $(\lambda_1, \ldots, \lambda_p, \mu_1, \ldots, \lambda_q) \in \mathbbm{K}^{p+q}$\\
			On pose aussi $w = \sum_{i = 1}^p \lambda'_ie_i + \sum_{j=1}^r \nu_j v_j$ avec $(\lambda_1',\ldots,\lambda_p', \nu_1, \ldots, \nu_r) \in \mathbbm{K}^{p+r}$\\
			D'où, \[
				u = \sum_{i=1}^p (\lambda_i + \lambda'_i)e_i + \sum_{j=1}^q \mu_j u_j + \sum_{k=1}^r \nu_k v_k \in \Vect(\mathcal{B})
			\]
		\item Soient $(\lambda_1, \ldots, \lambda_p, \mu_1, \ldots, \mu_q, \nu_1, \ldots, \nu_r) \in \mathbbm{K}^{p+q+r}$.\\
			On suppose \[
				(*)\quad \sum_{i=1}^{p}\lambda_ie_i + \sum_{j=1}^q\mu_ju_j + \sum_{k=1}^r \nu_k v_k = 0_E
			\] 
			D'où, \[
				\underbrace{\sum_{i=1}^p\lambda_i e_i + \sum_{j=1}^q \mu_ju_j}_{\in F} = \underbrace{-\sum_{k=1}^r\nu_jv_k}_{\in G}
			\] 
			Donc, \[
				f = \sum_{i=1}^p \lambda_i e_i + \sum_{j=1}^q \mu_j u_j \in F\cap G
			\] Comme $(e_1, \ldots, e_p)$ est une base de $F\cap G$, $\exists ! (\lambda_1', \ldots, \lambda_p') \in \mathbbm{K}^p$ tel que \[
				f = \sum_{i=1}^p \lambda'_i e_i = \sum_{i=1}^p \lambda'_i e_i + \sum_{j=1}^q 0_\mathbbm{K}u_j
			\] Comme $(e_1, \ldots, e_p, u_1, \ldots, u_q)$ est une base de $F$, \[
				\forall k \in \left\llbracket 1, q \right\rrbracket, \mu_j = 0_\mathbbm{K}
			\] De même, \[
				\forall k \in \left\llbracket 1,r \right\rrbracket , \nu_k = 0_\mathbbm{K}
			\] On remplace dans $(*)$ pour trouver \[
				\sum_{i=1}^p \lambda_ie_i = 0_E
			\] Comme $(e_1, \ldots, e_p)$ est libre, \[
				\forall i \in \left\llbracket 1,p \right\rrbracket, \lambda_i = 0_\mathbbm{K}
			\] Donc $\mathcal{B}$ est libre.\\
			Donc, 
			\begin{align*}
				\dim(F+G) &=  p +q + r \\
				&= (p+q)+ (p+r) - p \\
				&= \dim(F) + \dim(G) - \dim(F\cap G) \\
			\end{align*}
	\end{itemize}
\end{prv}

\begin{crlr}
	Avec les hypothèse précédentes, \[
		E = F \oplus G \iff \begin{cases}
			F \cap  G = \{0_E\} \\
			\dim(E) = \dim(F) + \dim(G)
		\end{cases}
	\] 
\end{crlr}

\begin{prv}
	\begin{itemize}
		\item[``$\implies$''] On suppose $E = F \oplus G$ \\
			Comme la somme est directe, $F \cap G = \{0_E\}$ 
			\begin{align*}
				\dim(E) &= \dim(F)\\
				&= \dim(F) + \dim(G) - \dim(F\cap G)\\
				&= \dim(F) + \dim(G)\\
			\end{align*}
		\item[``$\impliedby$''] On suppose $F\cap G = \{0_E\}$ et $\dim(E) = \dim(F) + \dim(G)$.\\
			On sait déjà que $F+G = F \oplus G$\\
			 \begin{align*}
				\dim(F+G) = \dim(F) + \dim(G) - \dim(F \cap G) = \dim(E)
			\end{align*}
			Donc $F + G = E$
	\end{itemize}
\end{prv}

\begin{prop}
	Soit $F$ un $\mathbbm{K}$-espace vectoriel de dimension finie $n$. Soit $\mathcal{B} = (e_1, \ldots, e_n)$ une base de $F$. L'application
	\begin{align*}
		f: \mathbbm{K}^n &\longrightarrow F \\
		(\lambda_1, \ldots, \lambda_n) &\longmapsto \sum_{i=1}^n \lambda_i e_i
	\end{align*} est bijective.\\
	Si $\mathbbm{K}$ est infini, $\mathbbm{K}^n$ aussi et donc $F$ aussi.\\
	Si $\#\mathbbm{K} = p \in \N_*$,
	\begin{align*}
		\#&\mathbbm{K}^n = p^n\\
		&\vrt=\\
		\#&F
	\end{align*}
\end{prop}


	}

	{
		\chap[14]{Continuité}
		\renewcommand{\cwd}{../chap14}
		\begin{defn}
	Soit $E$ un $\mathbbm{K}$-espace vectoriel. On dit que $E$ est de \underline{dimension finie} si $E$ a au moins une famille génératrice finie. On dit que $E$ est de \underline{dimension infinie} sinon.
	\index{dimension finie (espace vectoriel)}
	\index{dimension infinie (espace vectoriel)}
\end{defn}

\begin{thm}
	[Théorème de la base extraite]
	Soit $E$ un $\mathbbm{K}$-espace vectoriel non nul de dimension finie. Soit $\mathcal{G}$ une famille génératrice finie de $E$. Alors, il existe une base $\mathcal{B}$ de $\mathcal{E}$ telle que $\mathcal{B} \subset \mathcal{G}$.
\end{thm}

\begin{prv}
	[par récurrence sur $\#G = \Card(G)$]
	\begin{itemize}
		\item Soit $E$ un $\mathbbm{K}$-espace vectoriel non nul engendré par $\mathcal{G} = (u)$.\\
			Si $u = 0_E$, alors $E = \{0_E\}$: une contradiction $\lightning$ \\
			Donc $u \neq 0_E$ donc $(u)$ est libre. En effet, \[
				\forall \lambda \in \mathbbm{K}, \lambda u = 0_E \implies \lambda = 0_\mathbbm{K}
			\] Donc $\mathcal{G}$ est une base de $E$.\\
		\item Soit $n \in \N_*$. Soit $E$ un $\mathbbm{K}$-espace vectoriel. On suppose que si $E$ a une famille génératrice constituée de $n$ vecteurs, alors on peut extraire de cette famille une base de $E$.\\
			Soit $\mathcal{G}$ une famille génératrice de $E$ avec $n+1$ vecteurs.\\
			Si $\mathcal{G}$ est libre, alors $\mathcal{G}$ est une base de $E$. \\
			Si $\mathcal{G}$ n'est pas libre, alors il existe $u \in \mathcal{G}$ tel que $u \in \Vect(\mathcal{G}\setminus \{u\})$ \\
			Donc $\mathcal{G}\setminus \{u\}$ engendre $E$. Or, $\mathcal{G}\setminus \{u\}$ possède $n$ vecteurs. D'après l'hypothèse de récurrence, il existe une base $\mathcal{B}$ de $E$ telle que \[
				\mathcal{B} \subset \mathcal{G} \setminus \{u\} \subset \mathcal{G}
			\] 
	\end{itemize}
\end{prv}

\begin{crlr}
	Tout espace de dimension finie a une base.
	\qed
\end{crlr}

\begin{thm}
	[Théorème de la base incomplète]
	Soit $E$ un $\mathbbm{K}$-espace vectoriel de dimension finie, $\mathcal{G}$ une famille génératrice finie de $E$. $\mathcal{L}$ une famille libre de $E$. Alors, il existe une base $\mathcal{B}$ de $E$ telle que \[
		\mathcal{L} \subset \mathcal{B} \text{ et } \mathcal{B}\setminus \mathcal{L} \subset \mathcal{G}
	\] 
\end{thm}

\begin{prv}
	[par récurrence sur $\#(\mathcal{G}\setminus\mathcal{L})$]
	\begin{itemize}
		\item Avec les notations précédentes, on suppose que $\mathcal{G}\setminus\mathcal{L} \neq \O$ \[
				\forall u \in \mathcal{G}, u \in \mathcal{L}
			\] Donc $\mathcal{G} \subset \mathcal{L}$ donc $\mathcal{L}$ est génératrice donc $\mathcal{L}$ est une base de $E$. On pose $\mathcal{B} = \mathcal{L}$ et alors \[
				\mathcal{L} \subset  \mathcal{B} \text{ et } \mathcal{B}\setminus\mathcal{L} = \O \subset  \mathcal{G}
			\] 
		\item Soit $n \in \N$. On suppose que si $\mathcal{G}$ est génératrice et $\mathcal{L}$ libre avec $\#(\mathcal{G}\setminus\mathcal{L}) = n$ alors il existe une base $\mathcal{B}$ de $E$ telle que \[
			\mathcal{L}\subset \mathcal{B} \text{ et } \mathcal{B}\setminus\mathcal{L}\subset \mathcal{G}
		\] Soient à présent $\mathcal{G}$ une famille génératrice de $E$ et $\mathcal{L}$ une famille libre de $E$ telles que $\#(\mathcal{G}\setminus\mathcal{L}) = n+1 > 0$\\
		Si $\mathcal{L}$ engendre $E$, alors $\mathcal{L}$ est une base de $E$. On pose $\mathcal{B} = \mathcal{L}$ et on a bien \[
			\mathcal{L} \subset  \mathcal{B} \text{ et } \mathcal{B} \setminus \mathcal{L} = \O \subset  \mathcal{G}
		\] On suppose que $\mathcal{L}$ n'engendre pas $E$. Il existe $u \in \mathcal{G}$ tel que $u \not\in \Vec(\mathcal{L})$ (car sinon, $\mathcal{G} \subset \Vect(\mathcal{L})$ et donc $\underbrace{\Vect(\mathcal{G})}_{= E} \subset  \underbrace{\Vect(\mathcal{L})}_{ \subset E}$\\
		Donc $\mathcal{L} \cup \{u\} $ est libre. On pose $\mathcal{L}' = \mathcal{L} \cup \{u\} $ \[
			\mathcal{G}\setminus \mathcal{L}' = \mathcal{G}\setminus (\mathcal{L} \cup \{u\}) = (\mathcal{G}\setminus\mathcal{L})\setminus \{u\} 
		\] donc $\#(\mathcal{G}\setminus\mathcal{L}') = n+1 -1 = n$\\
		D'après l'hypothèse de récurrence, il existe $\mathcal{B}$ une base de $E$ telle que \[
			\mathcal{L} \subset  \mathcal{L}' \subset \mathcal{B} \text{ et } \mathcal{B}\setminus \mathcal{L}' \subset \mathcal{G}
		\] \[
			\mathcal{B} \setminus \mathcal{L} = \underbrace{\mathcal{B}\setminus\mathcal{L}'}_{\subset \mathcal{G}} \cup \underbrace{\{u\}}_{\subset \mathcal{G} \text{ car } u \in \mathcal{G}}
		\] On a $\mathcal{B}\setminus\mathcal{L}\subset \mathcal{G}$
	\end{itemize}
\end{prv}

\begin{thm}
	Soit $E$ un $\mathbbm{K}$-espace vectoriel de dimension finie. Toutes les bases de $E$ ont le même cardinal.
\end{thm}

\begin{prv}
	Soit $\mathcal{G}$ une famille génératrice finie de $E$ et $\mathcal{B} \subset  \mathcal{G}$ une base de $E$. On note $n = \#\mathcal{B}$ \\
	Soit $\mathcal{B}'$ une base de $E$. On pose $p = n - \#(\mathcal{B} \cap  \mathcal{B}')$. Montrons par récurrence sur  $p$ que $\#\mathcal{B} = \#\mathcal{B}'$ 
	\begin{itemize}
		\item On suppose que $p = 0$. Alors, $\#(\mathcal{B} \cap \mathcal{B}') = n$ \\
			Or, $\mathcal{B}' \cap \mathcal{B} \subset \mathcal{B}$ donc $\mathcal{B} \cap \mathcal{B}' = \mathcal{B}$ donc $\mathcal{B} \subset  \mathcal{B}'$ et donc $\mathcal{B} = \mathcal{B}'$ 
		\item Soit $p \in \N$. On suppose que si $\mathcal{B}'$ est une base de $E$ telle que $n - \#(\mathcal{B} \cap \mathcal{B}') = p$, alors $\#\mathcal{B}' = n$ \\
			Aoit $\mathcal{B}'$ une base de $E$ telle que $n - \#(\mathcal{B}\cap \mathcal{B}') = p+1 > 0$ \\
			Donc $\mathcal{B} \cap \mathcal{B}' \neq \mathcal{B}$. Soit $u \in \mathcal{B}' \setminus \mathcal{B}$. D'après le lemme d'échange, il existe $v \in \mathcal{B}\setminus \mathcal{B}'$ tel que $\mathcal{B}' \setminus \{u\} \cup \{v\}$ est une base de $E$. On pose $\mathcal{B}'' = \mathcal{B}' \setminus \{u\} \cup \{v\}$ 
			\begin{align*}
				\mathcal{B}'' \cap \mathcal{B} &= \left( (\mathcal{B}' \setminus \{u\})  \cap \mathcal{B} \right) \cup \{v\} \\
				&= (\mathcal{B}' \cap \mathcal{B}) \cup \{v\} \\
			\end{align*}
			donc,
			\begin{align*}
				n - \#(\mathcal{B}'' \cap \mathcal{B}) &= n - (\#(\mathcal{B}' \cap \mathcal{B}) + 1) \\
				&= p+1- 1 \\
				&= p \\
			\end{align*}
			D'après l'hypothèse de récurrence, \[
				\#\mathcal{B}'' = n
			\] Or, $\#\mathcal{B}'' = \#\mathcal{B}'$
	\end{itemize}
\end{prv}

\begin{lem}
	Soient $\mathcal{B}$ et $\mathcal{B}'$ deux bases de $E$ telles que $\mathcal{B}\subset \mathcal{B}'$. Alors, $\mathcal{B} = \mathcal{B}'$.
\end{lem}

\begin{prv}
	On suppose $\mathcal{B}' \neq \mathcal{B}$. Soit $u \in \mathcal{B}' \setminus \mathcal{B}$
	$u \in E = \Vect(\mathcal{B})$ donc $\mathcal{B} \cup \{u\}$ n'est pas libre.
	Donc $\mathcal{B}\cup \{u\} \subset \mathcal{B}'$ et $\mathcal{B}'$ est libre donc $\mathcal{B}\cup \{u\}$ est libre: une contradiction $\lightning$
\end{prv}

\begin{lem}
	[Lemme d'échange] Soient $\mathcal{B}_1$ et $\mathcal{B}_2$ deux bases de $E$ et $u \in \mathcal{B}_1 \setminus \mathcal{B}_2$. Alors, il existe $v \in \mathcal{B}_2$ tel que $(\mathcal{B}_1 \setminus \{u\}) \cup \{v\}$ soit une base de $E$.
\end{lem}

\begin{prv}
	[1${}^\text{nde}$ méthode]
	On suppose que pout tout $v \in \mathcal{B}_2$, $(\mathcal{B}_1\setminus \{u\}) \cup \{v\}$ n'est pas une base de $E$
	Soit $v \in \mathcal{B}_2$.
	\begin{itemize}
		\item Supposons $(\mathcal{B}_1\setminus \{u\})\cup \{v\}$ non libre. $\mathcal{B}_1 \setminus \{u\}$ est libre. Donc $v \in \Vect(\mathcal{B}_1 \setminus \{u\})$
		\item Supposons $(\mathcal{B}_1\setminus \{u\}) \cup \{v\}$ non génératrice.
			Comme $\mathcal{B}_1$ engendre $E$, $u \not\in \Vect(\mathcal{B}_1\setminus \{v\})$.
			On suppose que $\mathcal{B}_1 \neq \mathcal{B}_2$.
			$\forall v \in \mathcal{B}_2 \setminus \mathcal{B}_1, \Vect(\mathcal{B}_1 \setminus \{v\}) = \Vect(\mathcal{B}_1) = E \ni u$ 
			donc, $(\mathcal{B}_1\setminus \{u\}) \cup \{v\}$ engendre $E$ et donc \[
				v \in \Vect(\mathcal{B}_1 \setminus \{u\})
			\] On a aussi \[
				\forall v \in \mathcal{B}_1 \setminus \{u\}, v \in \Vect(\mathcal{B}_1\setminus \{u\})
			\] Comme $u \not\in \mathcal{B}_2$, on a \[
				\forall v \in \mathcal{B}_2, v \in \Vect(\mathcal{B}_1\setminus \{u\})
			\] docn \[
				E = \Vect(\mathcal{B}_2) \subset \Vect(\mathcal{B}_1\setminus \{u\})
			\] donc $\mathcal{B}_1\setminus \{u\}$ engendre $E$ donc $\mathcal{B}_1\setminus \{u\}$ est une base de $E$. Or, $\mathcal{B}_1 \setminus \{u\}  \subset  \mathcal{B}_1$, donc $\mathcal{B}_1\setminus \{u\} = \mathcal{B}_1$
	\end{itemize}
\end{prv}

\begin{prv}
	[2${}^\text{nde}$ méthode]
	On suppose que pout tout $v \in \mathcal{B}_2$, $(\mathcal{B}_1\setminus \{u\}) \cup \{v\}$ n'est pas une base de $E$
	\begin{itemize}
		\item Comme $u \in \mathcal{B}_1 \setminus \mathcal{B}_2$, nécéssairement $\mathcal{B}_1 \neq \mathcal{B}_2$ donc $\mathcal{B}_2 \not\subset \mathcal{B}_1$, donc $\mathcal{B}_2\setminus\mathcal{B}_1 \neq \O$ 
		\item Soit $v \in \mathcal{B}_2\setminus\mathcal{B}_1$. Il existe $(\lambda_w)_{w\in\mathcal{B}_1}$ une famille de scalaires presque nulle telle que \[
				v = \sum_{w \in \mathcal{B}_1} \lambda_w w - \lambda_u u + + \sum_{w \in \mathcal{B}_1\setminus \{u\}}\lambda_w w
			\]
			Si $\lambda_u \neq 0_E$, alors
			\begin{align*}
				u &= \lambda_u^{-1}\left( v - \sum_{w \in \mathcal{B}_1 \setminus \{u\}} \lambda_w w \right)\\
					&\in \Vect(\mathcal{B}_1\setminus \{u\} \cup v)
			\end{align*}
			 donc $\mathcal{B}_1 \subset \Vect(\mathcal{B}_1\setminus \{u\} \cup \{v\})$\\
			 et donc $E \subset  \Vect(\mathcal{B}_1 \setminus \{u\} \cup \{v\})$ \\
			 et donc $\mathcal{B}_1 \setminus \{u\} \cup \{v\}$ engendre $E$ \\
			 donc $\mathcal{B}_1 \setminus \{u\} \cup \{v\}$ n'est pas libre\\
			 donc $v \in \Vect(\mathcal{B}_1\setminus \{u\})$ (car $\mathcal{B}_1 \setminus \{u\}$ est libre\\
			 donc $\lambda_u = 0_\mathbbm{K}$ $\lightning$\\`

			 Donc, $\lambda_u = 0_\mathbbm{K}$, docn $v \in \Vect(\mathcal{B}_1\setminus \{u\})$ \\
			 On vient de prouver que
			 \begin{align*}
			 	\mathcal{B}_2 \setminus \mathcal{B}_1 \subset \Vect(\mathcal{B}_1 \setminus \{u\})\\
			 	\mathcal{B}_1 \setminus \{u\} \subset \Vect(\mathcal{B}_1 \setminus \{u\})\\
			 \end{align*}
			 Comme $u \not\in \mathcal{B}_2$, \[
			 	\mathcal{B}_2 \subset \Vect(\mathcal{B}_1 \setminus \{u\})
			 \] donc \[
			 	E = \Vect(\mathcal{B}_2) \subset  \Vect(\mathcal{B}_1 \setminus \{u\})
			 \] donc $\mathcal{B}_1 \setminus \{u\}$ engendre $E$. Donc,  $\mathcal{B}_1 \setminus \{u\}$ est une base de $E$.\\
			 Or, $\mathcal{B}_1 \setminus \{u\} \subset  \mathcal{B}_1$, donc $\mathcal{B}_1 \setminus \{u\} = \mathcal{B}_1$
	\end{itemize}
\end{prv}

\begin{defn}
	Soit $E$ un $\mathbbm{K}$-espace vectoriel de dimension finie. Le cardinal commun à toutes les bases de $E$ est appelé \underline{dimension} de $E$ est notée $\dim(E)$ ou $\dim_\mathbbm{K}(E)$\\
	C'est donc aussi le nombre de coordonnées de n'importe quel vecteur dans n'importe quelle base.
	\index{dimension (espace vectoriel)}
\end{defn}

\begin{exm}
	\begin{enumerate}
		\item $\dim_\R(\C) = 2$ et $\dim_\C(\C) = 1$ 
		\item $\dim_\mathbbm{K}(\mathbbm{K}^{n}) = n$ 
		\item $\dim_{\mathbbm{K}}(\mathcal{M}_{n,p}(\mathbbm{K})) = np$
	\end{enumerate}
\end{exm}

\begin{crlr}
	Soit $E$ un $\mathbbm{K}$-espace vectoriel de dimension finie, $\mathcal{L}$ une famille libre de $E$, $\mathcal{G}$ une famille génératrice de $E$. On note $n = \dim(E)$
	\begin{enumerate}
		\item $\#\mathcal{G} \ge n$ et $(\#\mathcal{G} = n \implies \mathcal{G} \text{ est une base de } E$)
		\item $\#\mathcal{L} \le n$ et $(\#\mathcal{L} = n \implies \mathcal{L} \text{ est une base de } E$)
	\end{enumerate}
\end{crlr}

\begin{crlr}
	$\R^{\R}$ est de dimension infinie.
	$\forall i \in \N, e_i: x \mapsto x^i$\\
	$(e_i)_{i\in\N}$ est libre dans $\R^\R$
\end{crlr}

\begin{prop}
	Soient $E$ et $F$ deux $\mathbbm{K}$-espaces vectoriels de dimension finie. Alors $E\times F$ est de dimension finie et $\dim(E\times F) = \dim(E) + \dim(F)$
\end{prop}

\begin{prv}
	Soit $(e_1,\ldots, e_n)$ une base de $E$, $(f_1, \ldots, f_p)$ une base de $F$.
	On pose \[
		\left\{\begin{array}
			{r c l}
			u_1 &=& (e_1,0_F)\\
			u_2 &=& (e_2,0_F)\\
					&\vdots&\\
			u_n &=& (e_n,0_F)\\
			u_{n+1} &=& (0_E, f_1)\\
			u_{n+2} &=& (0_E, f_2)\\
					&\vdots&\\
			u_{n+p} &=& (0_E,f_p)\\
		\end{array}\right.
	\]
	Soit $(x,y) \in E\times F$. \[
		\begin{cases}
			\exists (x_1,\ldots,x_n)\in \mathbbm{K}^n, x = \sum_{i=1}^{n} x_ie_i
			\exists (y_1,\ldots,y_n)\in \mathbbm{K}^n, x = \sum_{j=1}^{p} y_jf_j
		\end{cases}
	\] 
	\begin{align*}
		(x,y) &= \left( \sum_{i=1}^{n} x_ie_i, \sum_{i=1}^{p} y_jf_j \right)  \\
		&= \sum_{i=1}^{n} x_i (e_i + 0_F) + \sum_{j=1}^{p} y_j (0_E, f_j) \\
		&= \sum_{i=1}^{n} x_i u_i + \sum_{j=1}^{p} y_j u_{n+j} \\
	\end{align*}
	Donc, $E\times F = \Vect(u_1, \ldots, u_{n+p})$ donc $E\times F$ est de dimension finie.\\
	Soit $(\lambda_1, \ldots, \lambda_{n+p}) \in \mathbbm{K}^{n+p}$ tel que \[
		(*): \quad \sum_{k=1}^{n+p} \lambda_ku_k = 0_{E\times F} = (0_E, 0_F)
	\]
	\begin{align*}
		(*) &\iff \sum_{k=1}^{n} \lambda_k (e_k, 0_F) + \sum_{k=n+1}^{p} \lambda_k(0_E, f_{k-n}) = (0_E, 0_F)\\
				&\iff \begin{cases}
					\sum_{k=1}^{n} \lambda_k e_k = 0_E\\
					\sum_{k=n+1}^{p} \lambda_k f_{k-n} = 0_F
				\end{cases}\\
				&\iff \begin{cases}
					\forall k \in \left\llbracket 1,n \right\rrbracket, \lambda_k = 0_\mathbbm{K} \qquad&(\text{car $(e_1,\ldots,e_n)$ est libre})\\
					\forall k \in \left\llbracket n+1,n+p \right\rrbracket, \lambda_k = 0_\mathbbm{K} \qquad&(\text{car $(f_1,\ldots,f_n)$ est libre})\\
				\end{cases}
	\end{align*}
	Donc $(u_1, \ldots, u_{n+p})$ est une base de $E\times F$. Donc, $\dim(E\times F) = n + p = \dim(E) + \dim(F)$
\end{prv}

\begin{rmk}
	[Convention]
	\[\dim\big(\{0_E\}\big) = 0\]
\end{rmk}

\begin{thm}
	Soit $E$ un $\mathbbm{K}$-espace vectoriel de dimension finie, $F$ un sous-espace vectoriel de $E$. Alors, $F$ est de dimension finie et  $\dim(F) \le \dim(E)$\\
	Si $\dim(F) = \dim(E)$, alors $F = E$
\end{thm}

\begin{prv}
	On considère \[
		A = \{k \in \N \mid \text{il existe une famille libre de $F$ à $k$ éléments}\} 
	\]
	On suppose $F \neq \{0_E\}$.
	\begin{itemize}
		\item Soit $u \in F\setminus \{0_E\}$. $(u)$ est libre donc $1 \in A$ et donc $A \neq \O$
		\item Soit $\mathcal{L}$ une famille libre de $F$. Alors, $\mathcal{L}$ est une famille libre de $E$ \\
			donc $\#\mathcal{L} \le \dim(E)$\\
			Donc $A$ est majorée par $\dim(E)$ \\
			On en déduit que $A$ a un plus grand élément $p$.
		\item Soit $\mathcal{L}$ une famille libre de $F$ avec $p$ éléments.\\
			Si $\mathcal{L}$ n'engendre pas $F$, alors il existe $u\in F$ tel que $u\not\in \Vect(\mathcal{L})$ et donc $\mathcal{L} \cup \{u\}$ est une famille libre de $F$, donc $p+1 \in A$ en contradiction avec la maximalité de $p$.\\
			Donc $\mathcal{L}$ est une base de $F$ donc $F$ est de dimension finie et $\dim(F) = p \le \dim(E)$\\
	\end{itemize}

	Soit $\mathcal{B}$ une base de $F$. Alors, $\mathcal{B}$ est aussi une famille de libre de de $E$. Donc $\#\mathcal{B} \le \dim(E)$ donc $\dim(F) = \dim(E)$ \\
	Si $\dim(F) = \dim(E)$, alors $\mathcal{B}$ est une base de $E$, et donc $F = \Vect(\mathcal{B}) = E$
\end{prv}

\begin{prop}
	[Formule de Grassmann]
	Soit $E$ un $\mathbbm{K}$-espace vectoriel de dimension finie, $F$ et $G$ deux sous-espace vectoriels de $E$. Alors, \[
		\dim(F+G) = \dim(F) + \dim(G) - \dim(F\cap G)
	\] 
\end{prop}

\begin{prv}
	Soit $(e_1, \ldots, e_p)$ une base de $F\cap G$. $(e_1,\ldots,e_p)$ est une famille libre de $F$.\\
	On complète $(e_1, \ldots, e_p)$ en une base $(e_1, \ldots, e_p, u_1, \ldots, u_q)$ de $F$.\\
	De même, on complète $(e_1, \ldots, e_p)$ en une base $(e_1, \ldots, e_p, v_1, \ldots, v_r)$ de $G$.\\
	On pose  $\mathcal{B} = (e_1, \ldots, e_p, u_1, \ldots, u_q, v_1, \ldots, v_r)$. Montrons que $\mathcal{B}$ est une base de $F+G$
	\begin{itemize}
		\item Soit $u \in F+G$ \\
			On pose $u = v+w$ avec $\begin{cases}
				v\in F\\
				w \in G
			\end{cases}$.\\
			On pose $v = \sum_{i=1}^p \lambda_i e_i + \sum_{i=1}^q \mu_i u_i$ avec $(\lambda_1, \ldots, \lambda_p, \mu_1, \ldots, \lambda_q) \in \mathbbm{K}^{p+q}$\\
			On pose aussi $w = \sum_{i = 1}^p \lambda'_ie_i + \sum_{j=1}^r \nu_j v_j$ avec $(\lambda_1',\ldots,\lambda_p', \nu_1, \ldots, \nu_r) \in \mathbbm{K}^{p+r}$\\
			D'où, \[
				u = \sum_{i=1}^p (\lambda_i + \lambda'_i)e_i + \sum_{j=1}^q \mu_j u_j + \sum_{k=1}^r \nu_k v_k \in \Vect(\mathcal{B})
			\]
		\item Soient $(\lambda_1, \ldots, \lambda_p, \mu_1, \ldots, \mu_q, \nu_1, \ldots, \nu_r) \in \mathbbm{K}^{p+q+r}$.\\
			On suppose \[
				(*)\quad \sum_{i=1}^{p}\lambda_ie_i + \sum_{j=1}^q\mu_ju_j + \sum_{k=1}^r \nu_k v_k = 0_E
			\] 
			D'où, \[
				\underbrace{\sum_{i=1}^p\lambda_i e_i + \sum_{j=1}^q \mu_ju_j}_{\in F} = \underbrace{-\sum_{k=1}^r\nu_jv_k}_{\in G}
			\] 
			Donc, \[
				f = \sum_{i=1}^p \lambda_i e_i + \sum_{j=1}^q \mu_j u_j \in F\cap G
			\] Comme $(e_1, \ldots, e_p)$ est une base de $F\cap G$, $\exists ! (\lambda_1', \ldots, \lambda_p') \in \mathbbm{K}^p$ tel que \[
				f = \sum_{i=1}^p \lambda'_i e_i = \sum_{i=1}^p \lambda'_i e_i + \sum_{j=1}^q 0_\mathbbm{K}u_j
			\] Comme $(e_1, \ldots, e_p, u_1, \ldots, u_q)$ est une base de $F$, \[
				\forall k \in \left\llbracket 1, q \right\rrbracket, \mu_j = 0_\mathbbm{K}
			\] De même, \[
				\forall k \in \left\llbracket 1,r \right\rrbracket , \nu_k = 0_\mathbbm{K}
			\] On remplace dans $(*)$ pour trouver \[
				\sum_{i=1}^p \lambda_ie_i = 0_E
			\] Comme $(e_1, \ldots, e_p)$ est libre, \[
				\forall i \in \left\llbracket 1,p \right\rrbracket, \lambda_i = 0_\mathbbm{K}
			\] Donc $\mathcal{B}$ est libre.\\
			Donc, 
			\begin{align*}
				\dim(F+G) &=  p +q + r \\
				&= (p+q)+ (p+r) - p \\
				&= \dim(F) + \dim(G) - \dim(F\cap G) \\
			\end{align*}
	\end{itemize}
\end{prv}

\begin{crlr}
	Avec les hypothèse précédentes, \[
		E = F \oplus G \iff \begin{cases}
			F \cap  G = \{0_E\} \\
			\dim(E) = \dim(F) + \dim(G)
		\end{cases}
	\] 
\end{crlr}

\begin{prv}
	\begin{itemize}
		\item[``$\implies$''] On suppose $E = F \oplus G$ \\
			Comme la somme est directe, $F \cap G = \{0_E\}$ 
			\begin{align*}
				\dim(E) &= \dim(F)\\
				&= \dim(F) + \dim(G) - \dim(F\cap G)\\
				&= \dim(F) + \dim(G)\\
			\end{align*}
		\item[``$\impliedby$''] On suppose $F\cap G = \{0_E\}$ et $\dim(E) = \dim(F) + \dim(G)$.\\
			On sait déjà que $F+G = F \oplus G$\\
			 \begin{align*}
				\dim(F+G) = \dim(F) + \dim(G) - \dim(F \cap G) = \dim(E)
			\end{align*}
			Donc $F + G = E$
	\end{itemize}
\end{prv}

\begin{prop}
	Soit $F$ un $\mathbbm{K}$-espace vectoriel de dimension finie $n$. Soit $\mathcal{B} = (e_1, \ldots, e_n)$ une base de $F$. L'application
	\begin{align*}
		f: \mathbbm{K}^n &\longrightarrow F \\
		(\lambda_1, \ldots, \lambda_n) &\longmapsto \sum_{i=1}^n \lambda_i e_i
	\end{align*} est bijective.\\
	Si $\mathbbm{K}$ est infini, $\mathbbm{K}^n$ aussi et donc $F$ aussi.\\
	Si $\#\mathbbm{K} = p \in \N_*$,
	\begin{align*}
		\#&\mathbbm{K}^n = p^n\\
		&\vrt=\\
		\#&F
	\end{align*}
\end{prop}


		\part{Corps}

\begin{exm}[Problème]
	\begin{itemize}
		\item 
			avec $A = \Z / 9 \Z$, résoudre $\overline{x}^2 = \overline{0}$ \\
			\begin{center}
				\begin{tabular}{|c|c|c|c|c|c|c|c|c|c|c|}
					\hline
					$\overline{x}$&$\overline{0}$& $\overline{1}$ &$\overline{2}$&$\overline{3}$ &$\overline{4}$ &$\overline{5}$ &$\overline{6}$ &$\overline{7}$ &$\overline{8}$& $\overline{9}$ \\
					\hline
					$\overline{x}^2$&$\overline{0}$ &$\overline{1}$ &$\overline{4}$ &$\overline{0}$ &$\overline{7}$ &$7$ &$\overline{0}$ &$\overline{4}$ &$\overline{1}$&$\overline{0}$\\
					\hline
				\end{tabular}
			\end{center}
			On a trouvé 3 solutions: $\overline{0}$, $\overline{3}$, $\overline{6}$.
		\item $\Z / 8\Z$
			\begin{center}
				\begin{tabular}{|c|c|c|c|c|c|c|c|c|}
					\hline
					$\overline{x}$& $\overline{0}$& $\overline{1}$& $\overline{2}$& $\overline{3}$& $\overline{4}$& $\overline{5}$& $\overline{6}$& $\overline{7}$\\
					\hline
					$\overline{x^2}$& $\overline{0}$& $\overline{1}$& $\overline{4}$& $\overline{1}$& $\overline{0}$& $\overline{1}$& $\overline{4}$& $\overline{1}$\\
					\hline
				\end{tabular}
			\end{center}
			$\overline{x}^2=7$ a 4 solutions: $\overline{1}, \overline{7}, \overline{3},\text{ et } \overline{5}$
		\item $A = \mathbbm{H} = \{a + bi + cj + dk  \mid  (a,b,c,d) \in \R^4\}$ \\
			$i^2 = j^2 = k^2 = -1$ 
			\begin{align*}
				\begin{array}{c c c}
					ij = k & jk = i & ji = j\\
					ji = -k & kj = -i & ik = -j
				\end{array}
			\end{align*}
			Dans cet anneau, $-1$ a 6 racines!
	\end{itemize}
\end{exm}

\begin{defn}
	Soit $(\mathbbm{K}, +, \times)$ un ensemble muni de deux lois de composition internes. On dit que c'est un \underline{corps} si
	 \begin{enumerate}
		\item $(\mathbbm{K}, \times)$ est un groupe abélien
		\item $(\mathbbm{K}, \times)$ est un monoïde commutatif
		\item $\forall x \in \mathbbm{K}\setminus \{0_\mathbbm{K}\}, \exists y \in \mathbbm{K}, xy = 1_\mathbbm{K}$
		\item $0_\mathbbm{K} \neq  1_\mathbbm{K}$
	\end{enumerate}
	\index{corps}
\end{defn}

\begin{exm}
	\begin{itemize}
		\item $(\C, +, \times)$ est un corps
		\item $(\R, +, \times)$ est un corps
		\item $(\Q, +, \times)$ est un corps
		\item $(\Z, +, \times)$ n'est pas un corps
	\end{itemize}
\end{exm}

\begin{prop}
	$(\Z / n\Z, +, \times)$ est un corps si et seulement si $n$ est premier.
\end{prop}

\begin{prv}
	\[
		\left( \Z / n\Z \right)^\times = \left\{ \overline{k}  \mid k \wedge n = 1 \right\}
	\] 
\end{prv}


\begin{prop}
	Tout corps est un anneau intègre.
\end{prop}

\begin{prv}
	Soit $(\mathbbm{K}, +, \times)$ un corps. Soient $(a,b) \in \mathbbm{K}^2$ tel que $a \times b = 0_\mathbbm{K}$.\\
	On suppose $a \neq  0_\mathbbm{K}$. Alors, $a$ est inversible et donc \[
		b = a^{-1} \times a \times b = a^{-1} \times 0_\mathbbm{K} = 0_\mathbbm{K}
	\] 
\end{prv}

\begin{exm}
	Soit $(\mathbbm{K},+,\times)$ un corps.\\
	Résoudre \[
		\begin{cases}
			x^2 = 1_\mathbbm{K}\\
			x \in \mathbbm{K}
		\end{cases}
	\]

	\begin{align*}
		x^2 = 1_\mathbbm{K} &\iff x^2 - 1_\mathbbm{K} = 0_\mathbbm{K}\\
		&\iff (x - 1_\mathbbm{K})(x+1_\mathbbm{K}) = 0_\mathbbm{K}\\
		&\iff x - 1_\mathbbm{K} = 0_\mathbbm{K} \text{ ou } x + 1_\mathbbm{K} = 0_\mathbbm{K}\\
		&\iff x = 1_\mathbbm{K} \text{ ou } x = -1_\mathbbm{K}
	\end{align*}

	Il y a au plus 2 solutions.
\end{exm}

\begin{prop}
	Soit $(\mathbbm{K},+,\times )$ un corps et $P$ un polynôme à coefficients dans $\mathbbm{K}$ de degré $n$. Alors, l'équation $P(x) = 0_{\mathbbm{K}}$ a au plus $n$ solutions dans $\mathbbm{K}$ 
	\qed
\end{prop}

\begin{crlr}[(Théorème de Wilson)]
	voir exercice 16 du TD 12
\end{crlr}


\begin{defn}
	Soit $(\mathbbm{K}, +, \times)$ un corps et $L\subset \mathbbm{K}$.\\
	On dit que $L$ est un \underline{sous corps} de $\mathbbm{K}$ si
	\begin{enumerate}
		\item $L$ est un anneau de $(\mathbbm{K}, +, \times)$ non nul
		\item $\forall x \in L\setminus \{0_\mathbbm{K}\}, x^{-1} \in L$ 
	\end{enumerate}
	\vspace{2mm}
	en d'autres termes si
	\begin{enumerate}
		\item $\forall (x,y) \in L^2, x - y \in L$
		\item $\forall (x,y) \in L^2, x \times y^{-1} \in L$
	\end{enumerate}
	\vspace{5mm}
	On dit aussi que $\mathbbm{K}$ est une \underline{extension} de $L$.
	\index{sous corps}
	\index{extension}
\end{defn}

\begin{prop}
	Tout sous corps est un corps. \qed
\end{prop}

\begin{defn}
	Soient $(\mathbbm{K}_1,+,\times )$ et $(\mathbbm{K}_2,+, \times)$ deux corps et $f: \mathbbm{K}_1 \to \mathbbm{K}_2$.\\
	On dit que $f$ est un \underline{morphisme de corps} si $f$ est un morphisme d'anneaux.\\
	i.e. si
	\[
		\begin{cases}
			\forall (x,y) \in {\mathbbm{K}_1}^2,& f(x+y) = f(x) + f(y)\\
			\forall (x,y) \in {\mathbbm{K}_1}^2,& f(x \times y) = f(x) \times f(y)\\
		\end{cases}
	\] 
	\index{homomorphisme (de corps)}
	\index{morphisme (de corps)}
\end{defn}

\begin{prop}
	Tout morphisme de corps est injectif.
\end{prop}

\begin{prv}
	Soit $f: \mathbbm{K}_1 \to \mathbbm{K}_2$ un morphisme de corps.\\
	\begin{itemize}
		\item $\Ker(f)$ est un sous groupe de $(\mathbbm{K}_1, +)$ 
		\item Soit $x \in \Ker(f)$ et $y \in \mathbbm{K}_1$ \[
				f(x \times y) = f(x) \times f(y) = 0_{\mathbbm{K}_2} \times f(y) = 0_{\mathbbm{K}_2}
			\]
		\item Soit $x \in \Ker(f) \setminus \{0_{\mathbbm{K}_1}\}$.\\
			Alors, $x$ est inversible.\\
			\begin{align*}
				\begin{rcases*}
					x \in \Ker(f)\\
					x^{-1} \in \mathbbm{K}_1
				\end{rcases*}& \text{ donc } x \times x ^{-1} \in \Ker(f)\\
				&\text{ donc } 1_{\mathbbm{K}_1} \in \Ker(f)\\
				&\text{ donc } f(1_{\mathbbm{K}_1}) = 0_{\mathbbm{K}_2}
			\end{align*}
			Or, $f(1_{\mathbbm{K}_1}) = 1_{\mathbbm{K}_2} \neq 0_{\mathbbm{K}_2}$
	\end{itemize}
	Donc, $\Ker(f) = \{0_{\mathbbm{K}_1}\}$ donc $f$ est injective.
\end{prv}

\begin{exm}
	$\begin{array}{cc}
		\C &\longrightarrow \C\\
		z &\longmapsto \overline{z}\\
	\end{array}$ est un morphisme de corps
\end{exm}



		\part{Opérations sur les séries}

\begin{prop}
	L'ensemble $E = \{u \in \C^\N  \mid \Sigma u_n \text{ converge}\}$ est un sous-espace vectoriel de $\C^\N$ et \begin{align*}
		S: E &\longrightarrow \C \\
		u &\longmapsto \sum_{n=0}^{+\infty} u_n
	\end{align*} est une forme linéaire.
	\qed
\end{prop}

\begin{rmk}
	La somme d'une série convergente et d'une série divergente diverge.
	Le produit d'une série divergente par un scalaire non nul diverge.
\end{rmk}

		\part{Comparaison de suites}

\begin{defn}
	Soient $u$ et $v$ deux suites réelles. On dit que $u$ est \underline{dominée} par  $v$ si \[
	\exists M\in \R, \exists N\in \N,\forall n\ge N,\left| u_n \right| \le M \left| v_n \right| 
	\] Dans ce cas, on note $u = O(v)$ ou $u_n = O(v_n)$ et on dit que "$u$ est un grand o de $v$"
\end{defn}

\begin{exm}
	En informatique, on dit qu'un alogirithme a une \underline{complexité linéaire} si son temps d'éxécution est un $O(n)$ 
	Par exemple, on calcule $a^n$ 

	\begin{itemize}
		\item Approche naïve
			\begin{algorithm}
				\begin{algorithmic}[1]
					\State $p \gets 1$
					\For{$i \in \left\llbracket 0,n-1 \right\rrbracket$}
						\State $p \gets p \times a$
					\EndFor
					\State \Return p
				\end{algorithmic}
			\end{algorithm}
			Complexité linéaire $O(n)$
		\item Exponentiation rapide\\
			On écrit $n$ en binaire: \begin{align*}
				n &= \overline{a_k a_{k-1}\ldots a_0}^{(2)}\\
					&= \sum_{i=0}^{k} a_i 2^i
			\end{align*} avec $(a_i) \in \left\{ 0,1 \right\} ^{k+1}$
			\begin{align*}
				a^n &= a^{\sum_{i=0}^{k} a_i 2^i} \\
				&= \prod_{i=0}^{k} a^{a_i 2^i}  \\
			\end{align*}
			
			\begin{algorithm}
				\begin{algorithmic}
					[1]

					\State $s \gets 0$
					\State $p \gets a$
					\For{ $i \in \left\llbracket 0, \log_2(n) \right\rrbracket$}
						\State $p \gets p \times p$
						\If{$a[i] = 1$}
							\State $s \gets s + p$
						\EndIf
					\EndFor
					\State \Return s
				\end{algorithmic}
			\end{algorithm}
			Compléxité logarithmique $O(\log_2(n))$
	\end{itemize}
\end{exm}


\begin{prop}
	$O$ est une relation réfléctive et transitive.
\end{prop}

\begin{prv}
	\begin{itemize}
		\item Soit $u$ une suite. On pose $M = 1$ et \[
			\forall n \in \N, \left| u_n \right| \le M \left| u_n \right|
			\] Donc $u = O(u)$.
		\item Soient $u, v, w$ trois suites telles que  \[
		\begin{cases}
			u = O(v)\\
			v = O(w)
		\end{cases}
		\] Soient $M_1,M_2 \in \R$ et $N_1,N_2\in \N$ tels que \[
		\begin{cases}
			\forall n \ge  N_1, \left| u_n \right| \le M_1 \left| v_n \right| \\
			\forall n \ge  N_2, \left| v_n \right| \le M_2 \left| w_n \right| \\
		\end{cases}
		\] 

		Nécéssairement, $M_1\ge 0$ et $M_2\ge 0$.\\
		Soit $N = \max(N_1,N_2)$. \[
		\forall n \ge  N, \left| u_n \right| \le M_1 \left| v_n \right| \le  M_1M_2 \left| w_n \right| 
		\] Donc $u = O(w)$
	\end{itemize}
\end{prv}

\begin{defn}
	Soient $u$ et $v$ deux suites. On dit que $u$ est \underline{négligeable} devant $v$ si \[
	\forall \varepsilon>0, \exists N\in \N, \forall n\ge N, \left| u_n \right| \le \varepsilon \left| v_n \right| 
	\] Dans ce cas, on note $u = o(v)$ ou $u_n = o(v_n)$ ou on le lit "$u$ est un petit o de $v$"
\end{defn}

\begin{prop}
	$o$ est une relation transitive, non-réfléctive
\end{prop}

\begin{prv}
	\begin{itemize}
		\item Soient $u$, $v$ et $w$ trois suites telles que \[
			\begin{cases}
				u = o(v)\\
				v = o(w)
			\end{cases}
			\] Soit $\varepsilon>0$. Soit $N_1\in \N$ tel que \[
			\forall n \ge N_1, \left| u_n \right| \le \sqrt{\varepsilon}  \left| v_n \right| 
			\] Soit $N_2\in \N$ tel que \[
			\forall n \ge N_2, \left| v_n \right| \le \sqrt{\varepsilon}  \left| w_n \right| 
			\] On pose $N = \max(N_1,N_2)$, alors \[
			\forall n \ge N, \left| u_n \right| \le \sqrt{\varepsilon}  \left| v_n \right| \le \underbrace{\sqrt{\varepsilon} \times \sqrt{\varepsilon}} _\varepsilon \left| w_n \right| 
			\] donc $u = o(w)$
		\item Soit $u$ une suite tel qu'il existe $N \in \N$ tel que \[
		\forall n \ge N, u_n > 0
		\] On suppose que $u = o(u)$, alors \[
		\forall \varepsilon>0,\exists N \in \N, \forall n \ge N, \left| u_n \right| \le \varepsilon \left| u_n \right| 
		\] On pose $\varepsilon = \frac{1}{2}$ alors \[
		\exists N \in \N, \forall n \ge N, \left| u_n \right| \le \frac{1}{2} \left| u_n \right| 
		\] une contradiction
	\end{itemize}
\end{prv}

\begin{prop}
	Soient $u$ et $v$ deux suites.
	\begin{itemize}
		\item $o(u) + o(u) = o(u)$
		\item $v \times o(u) = o(uv)$
		\item $o(u) \times o(v) = o(uv)$
		\item $o(o(u)) = o(u)$
	\end{itemize}
	\qed
\end{prop}

\begin{defn}
	Soient $u$ et $v$ deux suites. On dit que $u$ et $v$ sont \underline{équivalentes} si \[
	u = v + o(v)
	\] i.e. \[
	\forall \varepsilon >0, \exists N \in \N, \forall n \ge N, \left| u_n-v_n \right| \le \varepsilon\left| v_n \right| 
	\] Dans ce cas, on le note $u \sim v$
\end{defn}

\begin{prop}
	$\sim$ est une relation d'équivalence \qed
\end{prop}

\begin{prop}
	Soient $(u,v) \in \R^\N$. On suppose que $v$ ne s'annule pas à partir d'un certain rang
	\begin{enumerate}
		\item $u = o(v) \iff \left( \frac{u_n}{v_n} \right)$ bornée
		\item $u = o(v) \iff \frac{u_n}{v_n} \tendsto{n \to  +\infty} 0$
		\item $u \sim v \iff \frac{u_n}{v_n} \tendsto{n \to  +\infty} 1$
	\end{enumerate}
	\qed
\end{prop}

\begin{prop}
	[Suites de références]
	\begin{enumerate}
		\item $\ln^\alpha(n) = o(n^\beta)$ avec $(\alpha,\beta) \in \left( \R^+_* \right) ^2$ 
		\item $n^\beta = o(a^n)$ avec $\beta > 0$ et $a > 1$ 
		\item $a^n = o(n!)$ avec $a >1$ 
		\item $n! = o(n^n)$
	\end{enumerate}
\end{prop}


\begin{lem}
	[Exercice 10 du TD]
	Soit $u \in \left(\R^+_*\right)^\N$\\
	Si $\frac{u_{n+1}}{u_n} \tendsto{n \to +\infty} \ell < 1$ avec $\ell\in \R$,\\ alors $u_n \tendsto{n \to +\infty} 0$
\end{lem}

\begin{prv} [de la proposition]
	\begin{enumerate}
		\item par croissance comparée
		\item On pose $\forall n \in \N^*, u_n = \frac{n^\beta}{a^n}$. 
			\begin{align*}
				\forall  n \in \N^*, \frac{u_{n+1}}{u_n} &= \left( \frac{n+1}{n} \right) ^\beta \times \frac{1}{a} \\
				&= \frac{1}{a}\left( 1+\frac{1}{n} \right) ^\beta \\
				&\tendsto{n \to +\infty} \frac{1}{a} < 1
			\end{align*}
			Donc, $u_n \tendsto{n \to  +\infty} 0$
		\item On pose $\forall n \in \N, u_n = \frac{a^n}{n!}$ \[
			\forall n \in \N, \frac{u_{n+1}}{u_n} = \frac{a}{n+1} \tendsto{n \to +\infty} 0 < 1
			\] donc $u_n \tendsto{n \to +\infty} 0$
		\item On pose $\forall  n\in \N^*, u_n = \frac{n!}{n^n}$.
			\begin{align*}
				\forall n \in \N^*, \frac{u_{n+1}}{u_n}
				&= (n+1) {\frac{n^n}{(n+1)^{n+1}}} \\
				&= \left( \frac{n}{n+1} \right) ^n \\
				&= e^{n \ln\left( \frac{n}{n+1} \right) } \\
				&= e^{n \ln\left( 1+\frac{1}{n+1} \right)} \\
				&= e^{n(-\frac{1}{n} + o(\frac{1}{n})} \\
				&= e^{-1 + o(1)} \\
				&\tendsto{n \to  +\infty} e^{-1}<1
			\end{align*}
			donc $u_n \tendsto{n\to +\infty} 0$
	\end{enumerate}
\end{prv}

	}

	{
		\chap[15]{Espaces vectoriels}
		\renewcommand{\cwd}{../chap15}
		\newcommand{\red}[1]{{\color{red} #1}}
		\begin{defn}
	Soit $E$ un $\mathbbm{K}$-espace vectoriel. On dit que $E$ est de \underline{dimension finie} si $E$ a au moins une famille génératrice finie. On dit que $E$ est de \underline{dimension infinie} sinon.
	\index{dimension finie (espace vectoriel)}
	\index{dimension infinie (espace vectoriel)}
\end{defn}

\begin{thm}
	[Théorème de la base extraite]
	Soit $E$ un $\mathbbm{K}$-espace vectoriel non nul de dimension finie. Soit $\mathcal{G}$ une famille génératrice finie de $E$. Alors, il existe une base $\mathcal{B}$ de $\mathcal{E}$ telle que $\mathcal{B} \subset \mathcal{G}$.
\end{thm}

\begin{prv}
	[par récurrence sur $\#G = \Card(G)$]
	\begin{itemize}
		\item Soit $E$ un $\mathbbm{K}$-espace vectoriel non nul engendré par $\mathcal{G} = (u)$.\\
			Si $u = 0_E$, alors $E = \{0_E\}$: une contradiction $\lightning$ \\
			Donc $u \neq 0_E$ donc $(u)$ est libre. En effet, \[
				\forall \lambda \in \mathbbm{K}, \lambda u = 0_E \implies \lambda = 0_\mathbbm{K}
			\] Donc $\mathcal{G}$ est une base de $E$.\\
		\item Soit $n \in \N_*$. Soit $E$ un $\mathbbm{K}$-espace vectoriel. On suppose que si $E$ a une famille génératrice constituée de $n$ vecteurs, alors on peut extraire de cette famille une base de $E$.\\
			Soit $\mathcal{G}$ une famille génératrice de $E$ avec $n+1$ vecteurs.\\
			Si $\mathcal{G}$ est libre, alors $\mathcal{G}$ est une base de $E$. \\
			Si $\mathcal{G}$ n'est pas libre, alors il existe $u \in \mathcal{G}$ tel que $u \in \Vect(\mathcal{G}\setminus \{u\})$ \\
			Donc $\mathcal{G}\setminus \{u\}$ engendre $E$. Or, $\mathcal{G}\setminus \{u\}$ possède $n$ vecteurs. D'après l'hypothèse de récurrence, il existe une base $\mathcal{B}$ de $E$ telle que \[
				\mathcal{B} \subset \mathcal{G} \setminus \{u\} \subset \mathcal{G}
			\] 
	\end{itemize}
\end{prv}

\begin{crlr}
	Tout espace de dimension finie a une base.
	\qed
\end{crlr}

\begin{thm}
	[Théorème de la base incomplète]
	Soit $E$ un $\mathbbm{K}$-espace vectoriel de dimension finie, $\mathcal{G}$ une famille génératrice finie de $E$. $\mathcal{L}$ une famille libre de $E$. Alors, il existe une base $\mathcal{B}$ de $E$ telle que \[
		\mathcal{L} \subset \mathcal{B} \text{ et } \mathcal{B}\setminus \mathcal{L} \subset \mathcal{G}
	\] 
\end{thm}

\begin{prv}
	[par récurrence sur $\#(\mathcal{G}\setminus\mathcal{L})$]
	\begin{itemize}
		\item Avec les notations précédentes, on suppose que $\mathcal{G}\setminus\mathcal{L} \neq \O$ \[
				\forall u \in \mathcal{G}, u \in \mathcal{L}
			\] Donc $\mathcal{G} \subset \mathcal{L}$ donc $\mathcal{L}$ est génératrice donc $\mathcal{L}$ est une base de $E$. On pose $\mathcal{B} = \mathcal{L}$ et alors \[
				\mathcal{L} \subset  \mathcal{B} \text{ et } \mathcal{B}\setminus\mathcal{L} = \O \subset  \mathcal{G}
			\] 
		\item Soit $n \in \N$. On suppose que si $\mathcal{G}$ est génératrice et $\mathcal{L}$ libre avec $\#(\mathcal{G}\setminus\mathcal{L}) = n$ alors il existe une base $\mathcal{B}$ de $E$ telle que \[
			\mathcal{L}\subset \mathcal{B} \text{ et } \mathcal{B}\setminus\mathcal{L}\subset \mathcal{G}
		\] Soient à présent $\mathcal{G}$ une famille génératrice de $E$ et $\mathcal{L}$ une famille libre de $E$ telles que $\#(\mathcal{G}\setminus\mathcal{L}) = n+1 > 0$\\
		Si $\mathcal{L}$ engendre $E$, alors $\mathcal{L}$ est une base de $E$. On pose $\mathcal{B} = \mathcal{L}$ et on a bien \[
			\mathcal{L} \subset  \mathcal{B} \text{ et } \mathcal{B} \setminus \mathcal{L} = \O \subset  \mathcal{G}
		\] On suppose que $\mathcal{L}$ n'engendre pas $E$. Il existe $u \in \mathcal{G}$ tel que $u \not\in \Vec(\mathcal{L})$ (car sinon, $\mathcal{G} \subset \Vect(\mathcal{L})$ et donc $\underbrace{\Vect(\mathcal{G})}_{= E} \subset  \underbrace{\Vect(\mathcal{L})}_{ \subset E}$\\
		Donc $\mathcal{L} \cup \{u\} $ est libre. On pose $\mathcal{L}' = \mathcal{L} \cup \{u\} $ \[
			\mathcal{G}\setminus \mathcal{L}' = \mathcal{G}\setminus (\mathcal{L} \cup \{u\}) = (\mathcal{G}\setminus\mathcal{L})\setminus \{u\} 
		\] donc $\#(\mathcal{G}\setminus\mathcal{L}') = n+1 -1 = n$\\
		D'après l'hypothèse de récurrence, il existe $\mathcal{B}$ une base de $E$ telle que \[
			\mathcal{L} \subset  \mathcal{L}' \subset \mathcal{B} \text{ et } \mathcal{B}\setminus \mathcal{L}' \subset \mathcal{G}
		\] \[
			\mathcal{B} \setminus \mathcal{L} = \underbrace{\mathcal{B}\setminus\mathcal{L}'}_{\subset \mathcal{G}} \cup \underbrace{\{u\}}_{\subset \mathcal{G} \text{ car } u \in \mathcal{G}}
		\] On a $\mathcal{B}\setminus\mathcal{L}\subset \mathcal{G}$
	\end{itemize}
\end{prv}

\begin{thm}
	Soit $E$ un $\mathbbm{K}$-espace vectoriel de dimension finie. Toutes les bases de $E$ ont le même cardinal.
\end{thm}

\begin{prv}
	Soit $\mathcal{G}$ une famille génératrice finie de $E$ et $\mathcal{B} \subset  \mathcal{G}$ une base de $E$. On note $n = \#\mathcal{B}$ \\
	Soit $\mathcal{B}'$ une base de $E$. On pose $p = n - \#(\mathcal{B} \cap  \mathcal{B}')$. Montrons par récurrence sur  $p$ que $\#\mathcal{B} = \#\mathcal{B}'$ 
	\begin{itemize}
		\item On suppose que $p = 0$. Alors, $\#(\mathcal{B} \cap \mathcal{B}') = n$ \\
			Or, $\mathcal{B}' \cap \mathcal{B} \subset \mathcal{B}$ donc $\mathcal{B} \cap \mathcal{B}' = \mathcal{B}$ donc $\mathcal{B} \subset  \mathcal{B}'$ et donc $\mathcal{B} = \mathcal{B}'$ 
		\item Soit $p \in \N$. On suppose que si $\mathcal{B}'$ est une base de $E$ telle que $n - \#(\mathcal{B} \cap \mathcal{B}') = p$, alors $\#\mathcal{B}' = n$ \\
			Aoit $\mathcal{B}'$ une base de $E$ telle que $n - \#(\mathcal{B}\cap \mathcal{B}') = p+1 > 0$ \\
			Donc $\mathcal{B} \cap \mathcal{B}' \neq \mathcal{B}$. Soit $u \in \mathcal{B}' \setminus \mathcal{B}$. D'après le lemme d'échange, il existe $v \in \mathcal{B}\setminus \mathcal{B}'$ tel que $\mathcal{B}' \setminus \{u\} \cup \{v\}$ est une base de $E$. On pose $\mathcal{B}'' = \mathcal{B}' \setminus \{u\} \cup \{v\}$ 
			\begin{align*}
				\mathcal{B}'' \cap \mathcal{B} &= \left( (\mathcal{B}' \setminus \{u\})  \cap \mathcal{B} \right) \cup \{v\} \\
				&= (\mathcal{B}' \cap \mathcal{B}) \cup \{v\} \\
			\end{align*}
			donc,
			\begin{align*}
				n - \#(\mathcal{B}'' \cap \mathcal{B}) &= n - (\#(\mathcal{B}' \cap \mathcal{B}) + 1) \\
				&= p+1- 1 \\
				&= p \\
			\end{align*}
			D'après l'hypothèse de récurrence, \[
				\#\mathcal{B}'' = n
			\] Or, $\#\mathcal{B}'' = \#\mathcal{B}'$
	\end{itemize}
\end{prv}

\begin{lem}
	Soient $\mathcal{B}$ et $\mathcal{B}'$ deux bases de $E$ telles que $\mathcal{B}\subset \mathcal{B}'$. Alors, $\mathcal{B} = \mathcal{B}'$.
\end{lem}

\begin{prv}
	On suppose $\mathcal{B}' \neq \mathcal{B}$. Soit $u \in \mathcal{B}' \setminus \mathcal{B}$
	$u \in E = \Vect(\mathcal{B})$ donc $\mathcal{B} \cup \{u\}$ n'est pas libre.
	Donc $\mathcal{B}\cup \{u\} \subset \mathcal{B}'$ et $\mathcal{B}'$ est libre donc $\mathcal{B}\cup \{u\}$ est libre: une contradiction $\lightning$
\end{prv}

\begin{lem}
	[Lemme d'échange] Soient $\mathcal{B}_1$ et $\mathcal{B}_2$ deux bases de $E$ et $u \in \mathcal{B}_1 \setminus \mathcal{B}_2$. Alors, il existe $v \in \mathcal{B}_2$ tel que $(\mathcal{B}_1 \setminus \{u\}) \cup \{v\}$ soit une base de $E$.
\end{lem}

\begin{prv}
	[1${}^\text{nde}$ méthode]
	On suppose que pout tout $v \in \mathcal{B}_2$, $(\mathcal{B}_1\setminus \{u\}) \cup \{v\}$ n'est pas une base de $E$
	Soit $v \in \mathcal{B}_2$.
	\begin{itemize}
		\item Supposons $(\mathcal{B}_1\setminus \{u\})\cup \{v\}$ non libre. $\mathcal{B}_1 \setminus \{u\}$ est libre. Donc $v \in \Vect(\mathcal{B}_1 \setminus \{u\})$
		\item Supposons $(\mathcal{B}_1\setminus \{u\}) \cup \{v\}$ non génératrice.
			Comme $\mathcal{B}_1$ engendre $E$, $u \not\in \Vect(\mathcal{B}_1\setminus \{v\})$.
			On suppose que $\mathcal{B}_1 \neq \mathcal{B}_2$.
			$\forall v \in \mathcal{B}_2 \setminus \mathcal{B}_1, \Vect(\mathcal{B}_1 \setminus \{v\}) = \Vect(\mathcal{B}_1) = E \ni u$ 
			donc, $(\mathcal{B}_1\setminus \{u\}) \cup \{v\}$ engendre $E$ et donc \[
				v \in \Vect(\mathcal{B}_1 \setminus \{u\})
			\] On a aussi \[
				\forall v \in \mathcal{B}_1 \setminus \{u\}, v \in \Vect(\mathcal{B}_1\setminus \{u\})
			\] Comme $u \not\in \mathcal{B}_2$, on a \[
				\forall v \in \mathcal{B}_2, v \in \Vect(\mathcal{B}_1\setminus \{u\})
			\] docn \[
				E = \Vect(\mathcal{B}_2) \subset \Vect(\mathcal{B}_1\setminus \{u\})
			\] donc $\mathcal{B}_1\setminus \{u\}$ engendre $E$ donc $\mathcal{B}_1\setminus \{u\}$ est une base de $E$. Or, $\mathcal{B}_1 \setminus \{u\}  \subset  \mathcal{B}_1$, donc $\mathcal{B}_1\setminus \{u\} = \mathcal{B}_1$
	\end{itemize}
\end{prv}

\begin{prv}
	[2${}^\text{nde}$ méthode]
	On suppose que pout tout $v \in \mathcal{B}_2$, $(\mathcal{B}_1\setminus \{u\}) \cup \{v\}$ n'est pas une base de $E$
	\begin{itemize}
		\item Comme $u \in \mathcal{B}_1 \setminus \mathcal{B}_2$, nécéssairement $\mathcal{B}_1 \neq \mathcal{B}_2$ donc $\mathcal{B}_2 \not\subset \mathcal{B}_1$, donc $\mathcal{B}_2\setminus\mathcal{B}_1 \neq \O$ 
		\item Soit $v \in \mathcal{B}_2\setminus\mathcal{B}_1$. Il existe $(\lambda_w)_{w\in\mathcal{B}_1}$ une famille de scalaires presque nulle telle que \[
				v = \sum_{w \in \mathcal{B}_1} \lambda_w w - \lambda_u u + + \sum_{w \in \mathcal{B}_1\setminus \{u\}}\lambda_w w
			\]
			Si $\lambda_u \neq 0_E$, alors
			\begin{align*}
				u &= \lambda_u^{-1}\left( v - \sum_{w \in \mathcal{B}_1 \setminus \{u\}} \lambda_w w \right)\\
					&\in \Vect(\mathcal{B}_1\setminus \{u\} \cup v)
			\end{align*}
			 donc $\mathcal{B}_1 \subset \Vect(\mathcal{B}_1\setminus \{u\} \cup \{v\})$\\
			 et donc $E \subset  \Vect(\mathcal{B}_1 \setminus \{u\} \cup \{v\})$ \\
			 et donc $\mathcal{B}_1 \setminus \{u\} \cup \{v\}$ engendre $E$ \\
			 donc $\mathcal{B}_1 \setminus \{u\} \cup \{v\}$ n'est pas libre\\
			 donc $v \in \Vect(\mathcal{B}_1\setminus \{u\})$ (car $\mathcal{B}_1 \setminus \{u\}$ est libre\\
			 donc $\lambda_u = 0_\mathbbm{K}$ $\lightning$\\`

			 Donc, $\lambda_u = 0_\mathbbm{K}$, docn $v \in \Vect(\mathcal{B}_1\setminus \{u\})$ \\
			 On vient de prouver que
			 \begin{align*}
			 	\mathcal{B}_2 \setminus \mathcal{B}_1 \subset \Vect(\mathcal{B}_1 \setminus \{u\})\\
			 	\mathcal{B}_1 \setminus \{u\} \subset \Vect(\mathcal{B}_1 \setminus \{u\})\\
			 \end{align*}
			 Comme $u \not\in \mathcal{B}_2$, \[
			 	\mathcal{B}_2 \subset \Vect(\mathcal{B}_1 \setminus \{u\})
			 \] donc \[
			 	E = \Vect(\mathcal{B}_2) \subset  \Vect(\mathcal{B}_1 \setminus \{u\})
			 \] donc $\mathcal{B}_1 \setminus \{u\}$ engendre $E$. Donc,  $\mathcal{B}_1 \setminus \{u\}$ est une base de $E$.\\
			 Or, $\mathcal{B}_1 \setminus \{u\} \subset  \mathcal{B}_1$, donc $\mathcal{B}_1 \setminus \{u\} = \mathcal{B}_1$
	\end{itemize}
\end{prv}

\begin{defn}
	Soit $E$ un $\mathbbm{K}$-espace vectoriel de dimension finie. Le cardinal commun à toutes les bases de $E$ est appelé \underline{dimension} de $E$ est notée $\dim(E)$ ou $\dim_\mathbbm{K}(E)$\\
	C'est donc aussi le nombre de coordonnées de n'importe quel vecteur dans n'importe quelle base.
	\index{dimension (espace vectoriel)}
\end{defn}

\begin{exm}
	\begin{enumerate}
		\item $\dim_\R(\C) = 2$ et $\dim_\C(\C) = 1$ 
		\item $\dim_\mathbbm{K}(\mathbbm{K}^{n}) = n$ 
		\item $\dim_{\mathbbm{K}}(\mathcal{M}_{n,p}(\mathbbm{K})) = np$
	\end{enumerate}
\end{exm}

\begin{crlr}
	Soit $E$ un $\mathbbm{K}$-espace vectoriel de dimension finie, $\mathcal{L}$ une famille libre de $E$, $\mathcal{G}$ une famille génératrice de $E$. On note $n = \dim(E)$
	\begin{enumerate}
		\item $\#\mathcal{G} \ge n$ et $(\#\mathcal{G} = n \implies \mathcal{G} \text{ est une base de } E$)
		\item $\#\mathcal{L} \le n$ et $(\#\mathcal{L} = n \implies \mathcal{L} \text{ est une base de } E$)
	\end{enumerate}
\end{crlr}

\begin{crlr}
	$\R^{\R}$ est de dimension infinie.
	$\forall i \in \N, e_i: x \mapsto x^i$\\
	$(e_i)_{i\in\N}$ est libre dans $\R^\R$
\end{crlr}

\begin{prop}
	Soient $E$ et $F$ deux $\mathbbm{K}$-espaces vectoriels de dimension finie. Alors $E\times F$ est de dimension finie et $\dim(E\times F) = \dim(E) + \dim(F)$
\end{prop}

\begin{prv}
	Soit $(e_1,\ldots, e_n)$ une base de $E$, $(f_1, \ldots, f_p)$ une base de $F$.
	On pose \[
		\left\{\begin{array}
			{r c l}
			u_1 &=& (e_1,0_F)\\
			u_2 &=& (e_2,0_F)\\
					&\vdots&\\
			u_n &=& (e_n,0_F)\\
			u_{n+1} &=& (0_E, f_1)\\
			u_{n+2} &=& (0_E, f_2)\\
					&\vdots&\\
			u_{n+p} &=& (0_E,f_p)\\
		\end{array}\right.
	\]
	Soit $(x,y) \in E\times F$. \[
		\begin{cases}
			\exists (x_1,\ldots,x_n)\in \mathbbm{K}^n, x = \sum_{i=1}^{n} x_ie_i
			\exists (y_1,\ldots,y_n)\in \mathbbm{K}^n, x = \sum_{j=1}^{p} y_jf_j
		\end{cases}
	\] 
	\begin{align*}
		(x,y) &= \left( \sum_{i=1}^{n} x_ie_i, \sum_{i=1}^{p} y_jf_j \right)  \\
		&= \sum_{i=1}^{n} x_i (e_i + 0_F) + \sum_{j=1}^{p} y_j (0_E, f_j) \\
		&= \sum_{i=1}^{n} x_i u_i + \sum_{j=1}^{p} y_j u_{n+j} \\
	\end{align*}
	Donc, $E\times F = \Vect(u_1, \ldots, u_{n+p})$ donc $E\times F$ est de dimension finie.\\
	Soit $(\lambda_1, \ldots, \lambda_{n+p}) \in \mathbbm{K}^{n+p}$ tel que \[
		(*): \quad \sum_{k=1}^{n+p} \lambda_ku_k = 0_{E\times F} = (0_E, 0_F)
	\]
	\begin{align*}
		(*) &\iff \sum_{k=1}^{n} \lambda_k (e_k, 0_F) + \sum_{k=n+1}^{p} \lambda_k(0_E, f_{k-n}) = (0_E, 0_F)\\
				&\iff \begin{cases}
					\sum_{k=1}^{n} \lambda_k e_k = 0_E\\
					\sum_{k=n+1}^{p} \lambda_k f_{k-n} = 0_F
				\end{cases}\\
				&\iff \begin{cases}
					\forall k \in \left\llbracket 1,n \right\rrbracket, \lambda_k = 0_\mathbbm{K} \qquad&(\text{car $(e_1,\ldots,e_n)$ est libre})\\
					\forall k \in \left\llbracket n+1,n+p \right\rrbracket, \lambda_k = 0_\mathbbm{K} \qquad&(\text{car $(f_1,\ldots,f_n)$ est libre})\\
				\end{cases}
	\end{align*}
	Donc $(u_1, \ldots, u_{n+p})$ est une base de $E\times F$. Donc, $\dim(E\times F) = n + p = \dim(E) + \dim(F)$
\end{prv}

\begin{rmk}
	[Convention]
	\[\dim\big(\{0_E\}\big) = 0\]
\end{rmk}

\begin{thm}
	Soit $E$ un $\mathbbm{K}$-espace vectoriel de dimension finie, $F$ un sous-espace vectoriel de $E$. Alors, $F$ est de dimension finie et  $\dim(F) \le \dim(E)$\\
	Si $\dim(F) = \dim(E)$, alors $F = E$
\end{thm}

\begin{prv}
	On considère \[
		A = \{k \in \N \mid \text{il existe une famille libre de $F$ à $k$ éléments}\} 
	\]
	On suppose $F \neq \{0_E\}$.
	\begin{itemize}
		\item Soit $u \in F\setminus \{0_E\}$. $(u)$ est libre donc $1 \in A$ et donc $A \neq \O$
		\item Soit $\mathcal{L}$ une famille libre de $F$. Alors, $\mathcal{L}$ est une famille libre de $E$ \\
			donc $\#\mathcal{L} \le \dim(E)$\\
			Donc $A$ est majorée par $\dim(E)$ \\
			On en déduit que $A$ a un plus grand élément $p$.
		\item Soit $\mathcal{L}$ une famille libre de $F$ avec $p$ éléments.\\
			Si $\mathcal{L}$ n'engendre pas $F$, alors il existe $u\in F$ tel que $u\not\in \Vect(\mathcal{L})$ et donc $\mathcal{L} \cup \{u\}$ est une famille libre de $F$, donc $p+1 \in A$ en contradiction avec la maximalité de $p$.\\
			Donc $\mathcal{L}$ est une base de $F$ donc $F$ est de dimension finie et $\dim(F) = p \le \dim(E)$\\
	\end{itemize}

	Soit $\mathcal{B}$ une base de $F$. Alors, $\mathcal{B}$ est aussi une famille de libre de de $E$. Donc $\#\mathcal{B} \le \dim(E)$ donc $\dim(F) = \dim(E)$ \\
	Si $\dim(F) = \dim(E)$, alors $\mathcal{B}$ est une base de $E$, et donc $F = \Vect(\mathcal{B}) = E$
\end{prv}

\begin{prop}
	[Formule de Grassmann]
	Soit $E$ un $\mathbbm{K}$-espace vectoriel de dimension finie, $F$ et $G$ deux sous-espace vectoriels de $E$. Alors, \[
		\dim(F+G) = \dim(F) + \dim(G) - \dim(F\cap G)
	\] 
\end{prop}

\begin{prv}
	Soit $(e_1, \ldots, e_p)$ une base de $F\cap G$. $(e_1,\ldots,e_p)$ est une famille libre de $F$.\\
	On complète $(e_1, \ldots, e_p)$ en une base $(e_1, \ldots, e_p, u_1, \ldots, u_q)$ de $F$.\\
	De même, on complète $(e_1, \ldots, e_p)$ en une base $(e_1, \ldots, e_p, v_1, \ldots, v_r)$ de $G$.\\
	On pose  $\mathcal{B} = (e_1, \ldots, e_p, u_1, \ldots, u_q, v_1, \ldots, v_r)$. Montrons que $\mathcal{B}$ est une base de $F+G$
	\begin{itemize}
		\item Soit $u \in F+G$ \\
			On pose $u = v+w$ avec $\begin{cases}
				v\in F\\
				w \in G
			\end{cases}$.\\
			On pose $v = \sum_{i=1}^p \lambda_i e_i + \sum_{i=1}^q \mu_i u_i$ avec $(\lambda_1, \ldots, \lambda_p, \mu_1, \ldots, \lambda_q) \in \mathbbm{K}^{p+q}$\\
			On pose aussi $w = \sum_{i = 1}^p \lambda'_ie_i + \sum_{j=1}^r \nu_j v_j$ avec $(\lambda_1',\ldots,\lambda_p', \nu_1, \ldots, \nu_r) \in \mathbbm{K}^{p+r}$\\
			D'où, \[
				u = \sum_{i=1}^p (\lambda_i + \lambda'_i)e_i + \sum_{j=1}^q \mu_j u_j + \sum_{k=1}^r \nu_k v_k \in \Vect(\mathcal{B})
			\]
		\item Soient $(\lambda_1, \ldots, \lambda_p, \mu_1, \ldots, \mu_q, \nu_1, \ldots, \nu_r) \in \mathbbm{K}^{p+q+r}$.\\
			On suppose \[
				(*)\quad \sum_{i=1}^{p}\lambda_ie_i + \sum_{j=1}^q\mu_ju_j + \sum_{k=1}^r \nu_k v_k = 0_E
			\] 
			D'où, \[
				\underbrace{\sum_{i=1}^p\lambda_i e_i + \sum_{j=1}^q \mu_ju_j}_{\in F} = \underbrace{-\sum_{k=1}^r\nu_jv_k}_{\in G}
			\] 
			Donc, \[
				f = \sum_{i=1}^p \lambda_i e_i + \sum_{j=1}^q \mu_j u_j \in F\cap G
			\] Comme $(e_1, \ldots, e_p)$ est une base de $F\cap G$, $\exists ! (\lambda_1', \ldots, \lambda_p') \in \mathbbm{K}^p$ tel que \[
				f = \sum_{i=1}^p \lambda'_i e_i = \sum_{i=1}^p \lambda'_i e_i + \sum_{j=1}^q 0_\mathbbm{K}u_j
			\] Comme $(e_1, \ldots, e_p, u_1, \ldots, u_q)$ est une base de $F$, \[
				\forall k \in \left\llbracket 1, q \right\rrbracket, \mu_j = 0_\mathbbm{K}
			\] De même, \[
				\forall k \in \left\llbracket 1,r \right\rrbracket , \nu_k = 0_\mathbbm{K}
			\] On remplace dans $(*)$ pour trouver \[
				\sum_{i=1}^p \lambda_ie_i = 0_E
			\] Comme $(e_1, \ldots, e_p)$ est libre, \[
				\forall i \in \left\llbracket 1,p \right\rrbracket, \lambda_i = 0_\mathbbm{K}
			\] Donc $\mathcal{B}$ est libre.\\
			Donc, 
			\begin{align*}
				\dim(F+G) &=  p +q + r \\
				&= (p+q)+ (p+r) - p \\
				&= \dim(F) + \dim(G) - \dim(F\cap G) \\
			\end{align*}
	\end{itemize}
\end{prv}

\begin{crlr}
	Avec les hypothèse précédentes, \[
		E = F \oplus G \iff \begin{cases}
			F \cap  G = \{0_E\} \\
			\dim(E) = \dim(F) + \dim(G)
		\end{cases}
	\] 
\end{crlr}

\begin{prv}
	\begin{itemize}
		\item[``$\implies$''] On suppose $E = F \oplus G$ \\
			Comme la somme est directe, $F \cap G = \{0_E\}$ 
			\begin{align*}
				\dim(E) &= \dim(F)\\
				&= \dim(F) + \dim(G) - \dim(F\cap G)\\
				&= \dim(F) + \dim(G)\\
			\end{align*}
		\item[``$\impliedby$''] On suppose $F\cap G = \{0_E\}$ et $\dim(E) = \dim(F) + \dim(G)$.\\
			On sait déjà que $F+G = F \oplus G$\\
			 \begin{align*}
				\dim(F+G) = \dim(F) + \dim(G) - \dim(F \cap G) = \dim(E)
			\end{align*}
			Donc $F + G = E$
	\end{itemize}
\end{prv}

\begin{prop}
	Soit $F$ un $\mathbbm{K}$-espace vectoriel de dimension finie $n$. Soit $\mathcal{B} = (e_1, \ldots, e_n)$ une base de $F$. L'application
	\begin{align*}
		f: \mathbbm{K}^n &\longrightarrow F \\
		(\lambda_1, \ldots, \lambda_n) &\longmapsto \sum_{i=1}^n \lambda_i e_i
	\end{align*} est bijective.\\
	Si $\mathbbm{K}$ est infini, $\mathbbm{K}^n$ aussi et donc $F$ aussi.\\
	Si $\#\mathbbm{K} = p \in \N_*$,
	\begin{align*}
		\#&\mathbbm{K}^n = p^n\\
		&\vrt=\\
		\#&F
	\end{align*}
\end{prop}


		\part{Dérivation}

\underline{Motivation}:

{
\begin{wrapfigure}{l}{3cm}
	\centering
	\begin{asy}
		import three;

		size(3cm);
		settings.render=0;
		settings.prc=false;
		currentprojection = obliqueZ;

		draw(unitbox);
		draw(shift(1.1Z + 0.05X) * (O -- X), Arrows3(TeXHead2));
		draw(shift(1.1Z + 0.05Y) * (O -- Y), Arrows3(TeXHead2));
		draw(shift(1.1X + 0.05Z) * (O -- Z), Arrows3(TeXHead2));

		label("$x$", (X/2) + (1.1Z + 0.05X), align=S);
		label("$y$", (Y/2) + (1.1Z + 0.05Y), align=W);
		label("$z$", (Z/2) + X, align=SE);
	\end{asy}
\end{wrapfigure}

\begin{align*}
	&S(x,y,z) = 2(xy + xz + yz)\\
	&V(x,y,z) = xyz
\end{align*}

On cherche à minimiser $S$ avec la contrainte $V = 1$.

Soit $f : \begin{array}{rcl}
	\left( \R_*^+ \right)^2 &\longrightarrow& \R \\
	(x,y) &\longmapsto& S\left( x,y,\frac{1}{xy} \right) = 2\left( xy + \frac{1}{y} + \frac{1}{x} \right).
\end{array}$

On cherche $(a,b) \in \left( \R^+_* \right)^2$ tel que \[
	\forall (x,y) \in (\R^+_*), f(x,y) \ge f(a,b).
\]
}

\begin{defn}
	Soit $f: U \to \R$ où $U$ est un ouvert de $\R^2$. Soit $(a,b) \in U$.
	\vspace{2mm}

	Si $\lim_{x \to a} \frac{f(x,b) - f(a,b)}{x - a} \in \R$, alors on dit que $f$ a une dérivée partielle suivant $x$ en $(a,b)$ et cette limite est notée \[
		\partial f_1(a,b) = \frac{\partial f}{\partial x}(a,b).
	\]

	Si $\lim_{y \to b} \frac{f(a,y) - f(a,b)}{y - b} \in \R$, alors on dit que $f$ a une dérivée partielle suivant $y$ et la limite est notée \[
		\partial f_2(a,b) = \frac{\partial f}{\partial y}(a,b).
	\]
\end{defn}

\begin{exm}
	\begin{enumerate}
		\item $f: (x,y) \mapsto xy + x - y$.

			\begin{align*}
				&\frac{\partial f}{\partial x} : (x,y) \mapsto y + 1,\\
				&\frac{\partial f}{\partial y} : (x,y) \mapsto x - 1.
			\end{align*}

		\item $f: (x,y) \mapsto xy + \frac{1}{y}+ \frac{1}{x}$.

			\begin{align*}
				&\frac{\partial f}{\partial x}: (x,y) \mapsto y - \frac{1}{x^2},\\
				&\frac{\partial f}{\partial y}: (x,y) \mapsto x - \frac{1}{y^2}.
			\end{align*}

		\item Trouver $f$ telle que $\begin{cases}
				(1): \qquad \frac{\partial f}{\partial x}=y,\\[2mm]
				(2): \qquad \frac{\partial f}{\partial y} = x.
			\end{cases}$

			D'après $(1)$ : \[
				\forall (x,y), \exists C(y) \in \R, f(x,y) = xy + C(y)
			\] et donc \[
				\frac{\partial f}{\partial y}(x,y) = x + C'(y)
			\] donc $C'(y) = 0$ et donc $C$ est constante.

		\item Trouver $f$ telle que $\begin{cases}
			\frac{\partial f}{\partial x} = -y,\\[2mm]
			\frac{\partial f}{ƒ\partial y} = x.
		\end{cases}$

		Ce n'est pas possible !
	\end{enumerate}
\end{exm}

\begin{defn}~\\
	\begin{minipage}{\linewidth}
		\begin{wrapfigure}{r}{4cm}
			\centering
			\vspace{-5mm}
			\begin{asy}
				import three;
				import graph3;
				size(4cm);

				settings.render = 0;
				settings.prc = false;
				currentprojection = obliqueX;

				draw(O -- X, Arrow3(TeXHead2));
				draw(O -- Y, Arrow3(TeXHead2));
				draw(O -- Z, Arrow3(TeXHead2));

				triple f(real x, real y, real z = 0) { return (x,y,cos(x - 0.5) * cos(y - 0.5)/1.2 + 0.15); }

				real inc = 1 / 5;

				for(real x = 0; x <= 1; x += inc) {
					draw(graph(
						new real(real t) { return x; }, // x
						new real(real y) { return y; }, // y
						new real(real y) { return f(x,y).z; }, // z
						0, 1
					), gray);
				}

				for(real y = 0; y <= 1; y += inc) {
					draw(graph(
						new real(real x) { return x; }, // x
						new real(real t) { return y; }, // y
						new real(real x) { return f(x,y).z; }, // z
						0, 1
					), gray);
				}

				path3 path1 = (0.8, 0.2, 0) .. (0.5, 0.5, 0) .. (0.3, 0.7, 0);
				path3 path2 = f(0.8, 0.2, 0) .. f(0.5, 0.5, 0) .. f(0.3, 0.7, 0);
				path3 d = (0.2, 0.3, 0) .. (0.3, 0.4, 0) .. (0.2, 0.7, 0) .. (0.8, 0.9, 0) .. (0.6, 0.2, 0) .. cycle;

				draw(path1, red, Arrow3(TeXHead2));
				draw(path2, red, Arrow3(TeXHead2, position=0.8));

				dot((0.5, 0.5, 0));
				dot(f(0.5, 0.5, 0));
				draw((0.5, 0.5, 0) -- f(0.5, 0.5, 0), dashed);
				draw(d);

				label("$w$", (0.3, 0.7, 0), red, align=SE);
				label("$U$", (0.8, 0.9, 0), align=SE);
			\end{asy}
		\end{wrapfigure}

		Soit $f: U \to \R$ où $U$ est un ouvert. Soit $(a,b) \in U$. Soit $w = (w_1, w_2) \in \R^2$.

		Si 
		\[
			\lim_{t\to 0} \frac{f(a + tw_1, b + tw_2) - f(a,b)}{t}
		\] existe et est réelle, alors on dit que $f$ a une dérivée dans la direction de $w$ et la limite est notée \[
			\mathrm{d}f(w)\,(a,b) = D_w(f)\,(a,b).
		\]
	\end{minipage}
\end{defn}

\begin{exm}
	\begin{align*}
		f: \left( \R_*^+ \right)^2 &\longrightarrow \R \\
		(x,y) &\longmapsto xy+\frac{1}{x}+\frac{1}{y}.
	\end{align*}

	On pose $(a,b) = (1,2)$, $w = (w_1, w_2) = (1,1)$.
	\begin{align*}
		\frac{f(1+t, 2+t) - f(1,2)}{t} &= \frac{1}{t} \left( (1+t)(2+t) + \frac{1}{1+t} + \frac{1}{2+t} - 3 - \frac{1}{2} \right) \\
		&= \frac{1}{t}\left(\cancel 2 + 3t + \po(t) + \cancel 1 - t + \po(t) + \frac{1}{2}\left( \cancel 1 - \frac{t}{2} + \po(t) \right) - \cancel3 - \cancel{\frac{1}{2}} \right) \\
		&= \frac{1}{t} \left( \frac{7}{4} t + \po(t) \right)  \\
		&= \frac{7}{4} + \po(1) \tendsto{t \to 0} \frac{7}{4}. \\
	\end{align*}

	Donc, \[
		\mathrm{d}f(1,1)\,(1,2) = \frac{7}{4}.
	\]
\end{exm}

\begin{rmk}~\\
	\begin{figure}[H]
		\centering
		\begin{asy}
			import solids;
			import graph;
			size(5cm);

			settings.render = 0;
			settings.prc = false;

			path3 par = graph(
				new real(real x) { return x; },
				new real(real x) { return 0; },
				new real(real x) { return x^2; },
				0,3);
			revolution r = revolution(par, axis=Z);

			path3 par2 = graph(
				new real(real x) { return x; },
				new real(real x) { return 0; },
				new real(real x) { return x^2; },
				-3,3);

			draw(r,1,longitudinalpen=nullpen);
			draw(r.silhouette());

			draw((-4, 0, -1) -- (-4, 0, 10) -- (4, 0, 10) -- (4, 0, -1) -- cycle, red);
			draw(par2, deepred);

			draw((4,4.5) -- (7, 4.5), black+0.5mm, Arrow(TeXHead));

			path par2d = graph(new real(real x) { return x^2; }, -3, 3);
			draw(shift((11, 0)) * par2d, deepred);

			dot(O);
			dot((11, 0));
		\end{asy}
	\end{figure}
\end{rmk}


%todo ajouter théorème-définition
\begin{thm}
	Soit $f : U \to \R$, $(a,b) \in U$. On suppose que $\frac{\partial f}{\partial x}$ et $\frac{\partial f}{\partial y}$ existent en $(a,b)$ et sont {\bfseries continues} en $(a,b)$. Alors,
	\begin{align*}
		&\forall (h, k) \in \R^2 \text{ tel que } (a +h, b + k) \in U,\\
		&f(a+ h, b + k) = f(a,b) + h \frac{\partial f}{\partial x}(a,b) + k \frac{\partial f}{\partial y}(a,b) + \po_{(h,k)\to (0,0)}\big(\|(h,k)\|\big).
	\end{align*}

	On dit que $f$ est de classe $\mathcal{C}^1$ si $\frac{\partial f}{\partial x}$ et $\frac{\partial f}{\partial y}$ existent et sont continues.

	\qed
\end{thm}

\begin{rmk}
	En physique, cette formule correspond à : \[
		\mathrm{d}f = \frac{\partial f}{\partial x}\mathrm{d}x + \frac{\partial f}{\partial y} \mathrm{d}y.
	\] En effet :
	\begin{align*}
		\mathrm{d}f &= f(x+ \mathrm{d}x, y + \mathrm{d}y) - f(x,y) \\
		&= \frac{\partial f}{\partial x} \mathrm{d}x + \frac{\partial f}{\partial y} \mathrm{d}y.
	\end{align*}
\end{rmk}

\begin{prop}
	Soit $f: U \to \R$ de classe $\mathcal{C}^1$ en $(a,b) \in U$. Alors, \[
		\forall w = (w_1, w_2) \in \R^2, \mathrm{d}f(w)\,(a,b) = w_1 \frac{\partial f}{\partial x}(a,b) + w_2 \frac{\partial f}{\partial y}(a,b).
	\]
\end{prop}

\begin{prv}
	Soit $w = (w_1, w_2) \in \R^2$. Soit $t \in \R^*$.
	\begin{align*}
		\frac{1}{t}\big(f(a + tw_1, b + tw_2) - f(a,b)\big)
		&= \frac{1}{t} \left( tw_1 \frac{\partial f}{\partial x}(a,b) + tw_2 \frac{\partial f}{\partial y}(a,b) + \po_{t \to 0}\big(\|tw\|\big) \right) \\
		&= w_1 \frac{\partial f}{\partial x}(a,b) + w_2 \frac{\partial f}{\partial y}(a,b) + \po_{t\to 0}(1) \\
		&\tendsto{t\to 0} w_1 \frac{\partial f}{\partial x}(a,b) + w_2\frac{\partial f}{\partial y}(a,b).
	\end{align*}
\end{prv}


\begin{defn}
	Avec les hypothèses précédentes, en posant \[
		\nabla f(a,b) = \left( \frac{\partial f}{\partial x}(a,b), \frac{\partial f}{\partial y}(a,b) \right) 
	\]on obtient \[
		\mathrm{d}f(w)\,(a,b) = \left<w  \mid \nabla f(a,b) \right>
	\] où $\left<\cdot|\cdot \right>$ est le produit scalaire.

	Le vecteur $\nabla f(a,b)$ est appelé \underline{gradient de $f$ en $(a,b)$}.

	Le développement limité à l'ordre 1 de $f$ devient \[
		f\big((a,b)+w\big) = f(a,b) + \left<w \mid \nabla f(a,b) \right> + \po_{w\to 0}(\|w\|)
	\]
\end{defn}

\begin{prop}
	Soit $f : U \to \R$ de classe $\mathcal{C}^1$.

	\begin{figure}[H]
    \centering
    \incfig{gradient}
	\end{figure}

	$\nabla f$ est orthogonal au lignes de niveaux de $f$, son orientation va dans le sens d'une augmentation de $f$.
\end{prop}

\begin{prv}
	Soit $\gamma : I \to U$ une courbe de niveau : \[
		\forall t \in I, f\big(\gamma(t)\big) = \text{cste}.
	\] D'après le lemme suivant : \[
		\forall t \in I, 0 = (f \circ \gamma)'(t) = \mathrm{d}f\big(\gamma'(t)\big)\big(\gamma(t)\big) = \left<\gamma'(t)  \mid \nabla f\big(\gamma(t)\big) \right>
	\] Donc $\nabla f\big(\gamma(t)\big)$ est orthogonal à $\gamma'(t)$.

	Pour tout $t \in I$, on pose $w(t) = t\, \nabla f\big(\gamma(t)\big)$. Donc \[
		f\big(\gamma(t) + w(t)\big) = f\big(\gamma(t)\big) + t \|\nabla f(\gamma(t))\|^2 + \po_{t \to 0}(t)
	\] Pour $t$ assez petit, $f\big(\gamma(t) + w(t)\big) - f\big(\gamma(t)\big)$ est du même signe que $t$.
\end{prv}

\begin{rmk}
	\begin{align*}
		V: \R^3 &\longrightarrow \R \\
		(x,y,z) &\longmapsto -mgz
	\end{align*}
	l'énerge potentielle de pesenteur

	On a donc \[
		\nabla V(x,y,z) = \left( \frac{\partial V}{\partial x}, \frac{\partial V}{\partial y}, \frac{\partial V}{\partial z} \right) = (0, 0, -mg) = \vec{P}.
	\]
\end{rmk}

\begin{lem}
	Soit $f : U \to \R$ de classe $\mathcal{C}^1$, $\gamma : \begin{array}{rcl}
		I &\longrightarrow& U \\
		t &\longmapsto& \big(x(t), y(t)\big)
	\end{array}$ où $x$ et $y$ sont dérivables.

	On pose \[
		\forall t \in I, \gamma'(t) = \big(x'(t), y'(t)\big).
	\] Alors $f \circ \gamma : I \to \R$ est dérivable et
	\begin{align*}
		\forall t \in I, (f \circ \gamma)'(t) &= \mathrm{d}f\big(\gamma'(t)\big) \big(\gamma(t)\big)\\
		&= \left<\gamma'(t)  \mid \nabla f\big(\gamma(t)\big)  \right> \\
		&= x'(t) \frac{\partial f}{\partial x}\big(x(t), y(t)\big) + y'(t) \frac{\partial f}{\partial y}\big(x(t),y(t)\big). \\
	\end{align*}
\end{lem}

\begin{prv}
	On fixe $t \in I$.

	\begin{align*}
		\forall h \neq 0, \frac{f \circ \gamma(t + h) - f \circ \gamma(t)}{h}
		&= \frac{1}{h}\big(f(\gamma(t)) + h\gamma'(t) + \po_{h\to 0}(h) - f(\gamma(t))\big) \\
		&= \frac{1}{h}\bigg(\cancel{f(\gamma(t))} + \left<h\gamma'(t) \mid \nabla f(\gamma(t)) \right> + \po_{h\to 0}(\|h\gamma'(t)\|) - \cancel{f(\gamma(t))}\bigg)\\
		&= \left<\gamma'(t) \mid \nabla f(\gamma(t)) \right> + \po_{h\to 0}(1) \\
		&\tendsto{h\to 0} \left<\gamma'(t)  \mid \nabla f(\gamma(t)) \right>
	\end{align*}
\end{prv}

\begin{defn}
	Soit $f : U \to \R$ de classe $\mathcal{C}^1$ et $(a,b) \in U$. On dit que $(a,b)$ est un \underline{point critique} de $f$ si $\nabla f(a,b) = 0$ i.e. $\frac{\partial f}{\partial x}(a,b) = \frac{\partial f}{\partial y}(a,b) = 0$.

	Dans ce cas, $f(a,b)$ est appelé \underline{valeur critique} de $f$.
\end{defn}

\begin{prop}~\\
	\begin{minipage}{\linewidth}
		\begin{wrapfigure}{r}{3cm}
			\centering
			\vspace{-1cm}
			\begin{asy}
				import solids;
				import graph;
				size(3cm);

				settings.render = 0;
				settings.prc = false;

				path3 par = graph(
					new real(real x) { return x; },
					new real(real x) { return 0; },
					new real(real x) { return -x^2; },
					0,3);
				revolution r = revolution(par, axis=Z);

				draw(r,1,longitudinalpen=nullpen);
				draw(r.silhouette());

				dot("$(a,b)$", O, red, align=N);
				real s = sqrt(2.5);
				path3 g=(s,0,-2.5)..(0,s,-2.5)..(-s,0,-2.5)..(0,-s,-2.5)..cycle;
				draw(g, deepcyan);
			\end{asy}
		\end{wrapfigure}
		Soit $f: U \to \R$ de classe $\mathcal{C}^1$ et $(a,b) \in U$ tel que \[
			\exists r > 0, \forall (x,y) \in B_{(a,b)}(r), f(x,y) \le f(a,b)
		\] Alors $\nabla f(a,b) = (0,0)$.
	\end{minipage}
\end{prop}

\begin{prv}
	Soit $g: x \mapsto f(x,b)$. $g(a)$ est un maximum local de $g$ donc $g'(a) = 0$.

	Or, $g'(a) = \frac{\partial f}{\partial x}(a,b)$

	donc $\frac{\partial f}{\partial x}(a,b) = 0$.

	Soit $h : y \mapsto f(a,y)$. On a de même $h'(b) = 0$.

	Or, $h'(b) = \frac{\partial f}{\partial y}(a,b)$.

	Donc, $\nabla f(a,b) = (0,0)$.
\end{prv}

\begin{rmk}
	Un minimum local est aussi une valeur critique.
\end{rmk}

\begin{figure}[H]
	\centering
	\begin{subfigure}{3cm}
		\centering
		\begin{asy}
			import solids;
			import graph;
			size(3cm);

			settings.render = 0;
			settings.prc = false;

			path3 par = graph(
				new real(real x) { return x; },
				new real(real x) { return 0; },
				new real(real x) { return -x^2; },
				0,3);
			revolution r = revolution(par, axis=Z);

			draw(r,1,longitudinalpen=nullpen);
			draw(r.silhouette());

			dot(O, red);
		\end{asy}
		\caption{Maximum local}
	\end{subfigure}
	\begin{subfigure}{3cm}
		\centering
		\begin{asy}
			import solids;
			import graph;
			size(3cm);

			settings.render = 0;
			settings.prc = false;

			path3 par = graph(
				new real(real x) { return x; },
				new real(real x) { return 0; },
				new real(real x) { return x^2; },
				0,3);
			revolution r = revolution(par, axis=Z);

			draw(r,1,longitudinalpen=nullpen);
			draw(r.silhouette());

			dot(O, red);
		\end{asy}
		\caption{Minimum local}
	\end{subfigure}
	\begin{subfigure}{3cm}
		\centering
		\begin{asy}
			import solids;
			import graph;
			size(3cm);

			settings.render = 0;
			settings.prc = false;
			currentprojection = obliqueZ;

			draw(graph(
				new real(real x) { return x; },
				new real(real x) { return -x^2 / 3; },
				new real(real x) { return 3; },
				-3, 3
			));

			draw(graph(
				new real(real x) { return x; },
				new real(real x) { return -x^2 / 3; },
				new real(real x) { return -3; },
				-3, 3
			));

			draw(graph(
				new real(real x) { return x; },
				new real(real x) { return -x^2 / 3 - 1; },
				new real(real x) { return 0; },
				-3, 3
			));

			draw(graph(
				new real(real x) { return 0; },
				new real(real x) { return x^2 / 9 - 1; },
				new real(real x) { return x; },
				-3, 3
			));

			draw(graph(
				new real(real x) { return -3; },
				new real(real x) { return x^2 / 9 - 4; },
				new real(real x) { return x; },
				-3, 3
			));

			draw(graph(
				new real(real x) { return 3; },
				new real(real x) { return x^2 / 9 - 4; },
				new real(real x) { return x; },
				-3, 3
			));

			dot((0,-1,0), red);
		\end{asy}
		\caption{Point de selle / Point col}
	\end{subfigure}
\end{figure}

\begin{exm}
	On revient à l'exemple donné en introduction : 
	\begin{align*}
		f: \left( \R^*_+ \right)^2 &\longrightarrow \R \\
		(x,y) &\longmapsto 2\left( xy + \frac{1}{x} + \frac{1}{y} \right).
	\end{align*}

	$\left( \R^+_* \right)^2$ est un ouvert de $\R^2$. Soit $(x,y) \in \left( \R^+_* \right)^2$.
	
	On a \[
		\begin{cases}
			\frac{\partial f}{\partial x}(x,y) = 2\left( y - \frac{1}{x^2} \right),\\
			\frac{\partial f}{\partial y}(x,y) = 2\left( x - \frac{1}{y^2} \right).
		\end{cases}
	\]

	\begin{align*}
		&\frac{\partial f}{\partial x}(x,y) = \frac{\partial f}{\partial y}(x,y) = 0\\
		\iff& \begin{cases}
			y = \frac{1}{x^2}\\
			x = \frac{1}{y^2}
		\end{cases}\\
		\iff& \begin{cases}
			y = \frac{1}{x^2}\\
			x = x^4
		\end{cases}\\
		\iff& \begin{cases}
			x = 1\\
			y = 1
		\end{cases}
	\end{align*}

	On vérivie que $f$ présente en effet un minium local en $(1,1)$. \[
		f(1,1) = 6
	\] On fixe $y \in \R^+_*$ et \[
		g : x \mapsto 2\left( xy + \frac{1}{x} + \frac{1}{y} \right).
	\] Donc \[
		\forall x \in \R^+_*, g'(x) = 2\left( y - \frac{1}{x^2} \right).
	\]
	\begin{center}
		\begin{tikzpicture}
			\tkzTabInit{$x$/1,$g'(x)$/1,$g$/2.3}{$0$, $\frac{1}{\sqrt{y}}$, $+\infty$}
			\tkzTabLine{,-,z,+,}
			\tkzTabVar{+/{}, -/$2\left( 2\sqrt{y} +\frac{1}{y} \right)$, +/{}}
		\end{tikzpicture}
	\end{center}
	
	Ainsi, \[
		\forall x \in \R^+_*, \forall y \in \R^+_*, f(x,y) \ge 2\left( 2\sqrt{y} + \frac{1}{y} \right)
	\] Soit $h : y \mapsto 2\sqrt{y} + \frac{1}{y}$. On a \[
		\forall y > 0, h'(y) = \frac{1}{\sqrt{y}} - \frac{1}{y^2} = \frac{y\sqrt{y} - 1}{y^2} = \frac{y^{\frac{3}{2}} - 1}{y^2}
	\]

	\begin{center}
		\begin{tikzpicture}
			\tkzTabInit{$y$/0.7,$h'(y)$/0.7,$h$/1.4}{$0$, $1$, $+\infty$}
			\tkzTabLine{,-,z,+,}
			\tkzTabVar{+/{}, -/$3$, +/{}}
		\end{tikzpicture}
	\end{center}

	Donc, \[
		\forall x,y > 0, f(x,y) \ge 2\times 3 = 6 = f(1,1).
	\]
\end{exm}

\begin{prop}
	[règle de la chaîne]

	Soit $f : \begin{array}{rcl}
		U &\longrightarrow& \R^2 \\
		(x,y) &\longmapsto& f(x,y)
	\end{array}$ de classe $\mathcal{C}^1$ et $U, V$ deux ouverts de $\R^2$.

	Soit $\varphi : \begin{array}{rcl}
		V &\longrightarrow& U \\
		(u,v) &\longmapsto& \varphi(u,v) = \big(x(u,v), y(u,v)\big)
	\end{array}$.

	On suppose que $x$ et $y$ sont de classe $\mathcal{C}^1$ sur $V$.

	Alors,  $f \circ \varphi : \begin{array}{rcl}
		V &\longrightarrow& \R \\
		(u,v) &\longmapsto& f\big(\varphi(u,v)\big)
	\end{array}$ est de classe $\mathcal{C}^1$ et
	\begin{align*}
		\forall (u_0, v_0) \in V, \frac{\partial (f \circ \varphi)}{\partial u}(u_0, v_0)
		&= \frac{\partial f}{\partial x}\big(\varphi(u_0, v_0)\big) \times \frac{\partial x}{\partial u}(u_0, v_0)\\
		&+ \frac{\partial f}{\partial y}\big(\varphi(u_0,v_0)\big) \frac{\partial y}{\partial u}(u_0,v_0)
	\end{align*}
	\begin{align*}
		\forall (u_0, v_0) \in V, \frac{\partial (f \circ \varphi)}{\partial v}(u_0, v_0)
		&= \frac{\partial f}{\partial x}\big(\varphi(u_0, v_0)\big) \times \frac{\partial x}{\partial v}(u_0, v_0)\\
		&+ \frac{\partial f}{\partial y}\big(\varphi(u_0,v_0)\big) \frac{\partial y}{\partial v}(u_0,v_0)
	\end{align*}
\end{prop}

\begin{exm}
	[changement de coordonnées polaires]
	On pose \begin{align*}
		\varphi: \R^+_* \times ]0,2\pi[ &\longrightarrow \R^2\setminus \left( R^+_* \times \{0\} \right) \\
		(r, \theta) &\longmapsto (r \cos \theta, r \sin\theta),
	\end{align*}
	\begin{align*}
		f: \R^2\setminus \left( R^+_* \times \{0\} \right) &\longrightarrow \R \\
		(x,y) &\longmapsto f(x,y),
	\end{align*}
	\begin{align*}
		g: \overbrace{\R^+_* \times ]0, 2\pi[}^{=V} &\longrightarrow \R \\
		(r, \theta) &\longmapsto f(r\cos\theta, r\sin\theta).
	\end{align*}

	\begin{align*}
		\forall (r_0,\theta_0) \in V,&\\[5mm]
		\frac{\partial g}{\partial r}(r_0, \theta_0) &= \frac{\partial f}{\partial x}(r_0\cos\theta_0, r_0\sin\theta_0)\cos\theta_0\\
		&+ \frac{\partial f}{\partial y}(r_0 \cos\theta_0, r_0\sin\theta_0)\sin\theta_0\\
		&= 2r_0\cos^2\theta_0 + 2r_0\sin^2(\theta_0) \\
		&= 2r_0 \\[5mm]
		\frac{\partial g}{\partial \theta}(r_0, \theta_0) &= \frac{\partial f}{\partial x}(r_0\cos\theta_0, r_0\sin\theta_0)r_0\sin\theta_0\\
		&+ \frac{\partial f}{\partial y}(r_0 \cos\theta_0, r_0\sin\theta_0)r_0\cos\theta_0\\
		&= -2{r_0}^2\cos(\theta_0)\sin(\theta_0) + 2{r_0}^2 \sin(\theta_0)\cos(\theta_0)\\
		&= 0 \\
	\end{align*}

	Donc, \[
		g(r, \theta) = r^2.
	\]
\end{exm}

\begin{exm}
	Résoudre \[
		\begin{cases}
			\frac{\partial f}{\partial x} = \frac{x}{x^2+y^2},\\
			\frac{\partial f}{\partial y} = \frac{y}{x^2+y^2}.\\
		\end{cases}
	\]

	On pose $g: (r, \theta) \mapsto f(r \cos\theta, r \sin\theta)$.

	\begin{align*}
		&\frac{\partial g}{\partial r} = \frac{1}{r}\cos^2\theta + \frac{1}{r}\sin^2\theta = \frac{1}{r},\\
		&\frac{\partial g}{\partial \theta} = -\cos(\theta) \sin(\theta) + \sin(\theta)\cos(\theta) = 0.
	\end{align*}

	Donc, \[
		\exists C \in \R, g: (r, \theta) \mapsto \ln r + C
	\] d'où,
	\begin{align*}
		\forall (x,y) \in \R^2 \setminus \{(0,0)\}, f(x,y) &= \ln\left(\sqrt{x^2 + y^2} \right)  + C\\
		&= \frac{1}{2}\ln(x^2 + y^2) + C. \\
	\end{align*}
\end{exm}

\begin{rmk}
	Soit $\mathcal{B} = (e_1, e_2)$ la base canonique de $\R^2$, $f: U \to \R$ de classe $\mathcal{C}^1$ avec $U$ un ouvert de $\R^2$.

	Soit $(x,y) \in U$.

	\begin{align*}
		\Mat_{\mathcal{B}}\big(\nabla f(x,y)\big) = \begin{pmatrix}
			\frac{\partial f}{\partial x}(x,y)\\[2mm]
			\frac{\partial f}{\partial y}(x,y)
		\end{pmatrix}
	\end{align*}

	Soit  \begin{align*}
		\varphi: V &\longrightarrow U \\
		(u,v) &\longmapsto \big(x(u,v), y(u,v)\big) 
	\end{align*} avec $x,y$ de classe $\mathcal{C}^1$. Soit $g = f \circ \varphi$.
	\begin{align*}
		\Mat_{\mathcal{B}}\big(\nabla g(u,v)\big)
		&= \begin{pmatrix}
			\frac{\partial g}{\partial u}(u,v) \\[2mm]
			\frac{\partial g}{\partial v}(u,v)
		\end{pmatrix} \\
		&= \begin{pmatrix}
			\frac{\partial x}{\partial u}(u,v) \frac{\partial f}{\partial x}(x,y)
			+ \frac{\partial y}{\partial u}(u,v)\frac{\partial f}{\partial y}(x,y)\\[3mm]
			\frac{\partial x}{\partial v}(u,v) \frac{\partial f}{\partial x}(x,y)
			+ \frac{\partial y}{\partial v}(u,v) \frac{\partial f}{\partial y}(x,y)
		\end{pmatrix}  \\
		&= \underbrace{\begin{pmatrix}
				\frac{\partial x}{\partial u}(u,v)& \frac{\partial y}{\partial u}(u,v)\\[3mm]
				\frac{\partial x}{\partial v}(u,v)& \frac{\partial y}{\partial v}(u,v)
		\end{pmatrix}}_{J(u,v)} \begin{pmatrix}
			\frac{\partial f}{\partial x}(x,y)\\[3mm]
			\frac{\partial f}{\partial y}(x,y)
		\end{pmatrix} \\
		&= J(u,v) \Mat_{\mathcal{B}}\big(\nabla f(x,y)\big) \\
	\end{align*}
	où $J(u,v) = 
	\begin{pNiceArray}{c:c}
		\Mat_{\mathcal{B}}\big(\nabla x(u,v)\big) & \Mat_{\mathcal{B}}\big(\nabla y(u,v)\big)
	\end{pNiceArray}$.

	On dit que $J(u,v)$ est \underline{la jacobienne} de $\varphi$ en $(u,v)$.
	L'application linéaire canoniquement associée à $J(u,v)$ est la \underline{différentielle de $\varphi$} en $(u,v)$ noté $\mathrm{d}\varphi(u,v)$.

	On a $\mathrm{d}\varphi(u,v) \in \mathcal{L}(R^2)$ et $\Mat_{\mathcal{B}}\big(\mathrm{d}\varphi(u,v)\big) = J(u,v)$.

	Par exemple, la jacobienne du changement de coordonnées polaires est \[
		J = \begin{pmatrix}
			\frac{\partial x}{\partial r} & \frac{\partial y}{\partial r}\\[3mm]
			\frac{\partial x}{\partial \theta} & \frac{\partial y}{\partial \theta}
		\end{pmatrix}
		= \begin{pmatrix}
			\cos\theta&\sin\theta\\
			-r\sin\theta&r\cos\theta
		\end{pmatrix}.
	\]
	$\underbrace{\det(J)}_{\text{le jacobien}} = r\cos^2\theta + r\sin^2\theta = r$

	Dans une intégrale double, si $(x,y) = \varphi(u,v)$, alors $\mathrm{d}x\mathrm{d}y = \det(J)\mathrm{d}u\mathrm{d}v$.

	Ici, \[
		\mathrm{d}x\ \mathrm{d}y = r\ \mathrm{d}r\ \mathrm{d}\theta.
	\]
\end{rmk}

\begin{prv}
	On pose $(x_0, y_0) = \varphi(u_0, v_0)$. Pour tout $(h,k) \in \R^2$ tels que $(u_0 + h, v_0 + k) \in V$, en posant $g = f  \circ \varphi$.

	\begin{align*}
		g(u_0 + h, v_0 + h) &= f\big(x(u_0 + h, v_0 + k), y(u_0 + h, v_0 + k)\big) \\
		&= f\left(
			x(u_0,v_0) + h \frac{\partial x}{\partial u}(u_0,v_0) + k \frac{\partial x}{\partial v}(u_0, v_0) + \po\big(\|(h,k)\|\big), \right.\\
		&\phantom{ = f\bigg(\bigg.}\left. y(u_0, v_0) + h \frac{\partial y}{\partial u}(u_0, v_0) + k \frac{\partial y}{\partial v}(u_0, v_0) + \po\big(\|(h,k)\|\big)
		\right)  \\
		&= f(x_0,y_0) \\
		&~+ \left( h \frac{\partial x}{\partial u}(u_0,v_0) + k \frac{\partial x}{\partial v}(u_0, v_0) + \po(\|(h,k)\|) \right) \frac{\partial f}{\partial x}(x_0,y_0)\\
		&~+ \left( h \frac{\partial y}{\partial u}(u_0, v_0) + k\frac{\partial y}{\partial v}(u_0, v_0) + \po(\|(h,k)\|) \right) \frac{\partial f}{\partial y}(x_0, y_0)\\
		&~+ \po(\|(h,k)\|)\\
		&= f(x_0, y_0) \\
		&~+ h \left( \frac{\partial x}{\partial u}(u_0, v_0) \frac{\partial f}{\partial x}(x_0, y_0) + \frac{\partial y}{\partial u}(u_0, v_0) \frac{\partial f}{\partial y}(x_0, y_0) \right)  \\
		&~+ k\left( \frac{\partial x}{\partial v}(u_0, v_0) \frac{\partial f}{\partial x}(x_0, y_0) + \frac{\partial y}{\partial v}(u_0, v_0) \frac{\partial f}{\partial y}(x_0, y_0) \right) 
		&~+ \po(\|(h,k)\|)\\
		&= g(u_0, v_0) + h \frac{\partial g}{\partial u}(u_0, v_0) + k \frac{\partial g}{\partial v}(u_0, v_0) + \po(\|(h,k)\|) \\
	\end{align*}

	Par identification,
	\[
		\frac{\partial g}{\partial u}(u_0, v_0) = \frac{\partial x}{\partial u}(u_0, v_0) \frac{\partial f}{\partial x}(x_0, y_0) + \frac{\partial y}{\partial u}(u_0, v_0) \frac{\partial f}{\partial y}(x_0,y_0)
	\] et \[
		\frac{\partial g}{\partial v}(u_0, v_0) = \frac{\partial x}{\partial v}(u_0,v_0) \frac{\partial f}{\partial x}(x_0, y_0) + \frac{\partial y}{\partial v}(u_0, v_0) \frac{\partial f}{\partial y}(x_0, y_0).
	\] 
\end{prv}

\begin{exm}
	[Régression linéaire]~\\
	\begin{figure}[H]
		\centering
		\begin{asy}
			import graph;
			axes(EndArrow);
			size(5cm);

			real f(real x) { return x + 0.5; }

			real k = 35 / (7 - 0.5);

			for(int i = 0; i < 35; ++i) {
				real mag = exp(sin(100 * pi/exp(1) * i)) * 0.8 + exp(cos(i*40)/3);
				real eps = mag * cos(10 * exp(1)/pi * i) / 3;
				dot((i/k,f(i/k) + eps));
			}

			draw(graph(f, -1, 7), orange);
		\end{asy}
	\end{figure}
	\[
		y = a x + b
	\] 
	On fixe $(a,b) \in \R^2$. \[
		\varepsilon(a,b) = \sum_{i=1}^n\big( y_i - (ax_i + b) \big)^2
	\] l'erreur totale.

	On veut minimiser $\varepsilon(a,b)$. On a 
	\[
		\forall (a,b) \in \R^2,
		\begin{cases}
			\frac{\partial \varepsilon}{\partial a}(a,b) = -2\sum_{i=1}^{n}(y_i - ax_i - b)x_i,\\
			\frac{\partial \varepsilon}{\partial b}(a,b) = -2\sum_{i=1}^{n}(y_i - ax_i - b).
		\end{cases}
	\]

	Donc,
	\begin{align*}
		(a,b) \text{ point critique de } \varepsilon \iff& \begin{cases}
			a \sum_{i=1}^n {x_i}^2 + b\sum_{i=1}^{n}x_i = \sum_{i=1}^{n} y_ix_i\\
			a\sum_{i=1}^{n}x_i + nb = \sum_{i=1}^ny_i
		\end{cases}\\
		\iff& \begin{cases}
			a \left( \frac{1}{n}\sum_{i=1}^n {x_i}^2 - \overline{x}^2\right) = \overline{y} - \overline{x} \overline{y}\\
			b = \frac{1}{n}\sum_{i=1}^ny_i - \frac{a}{n}\sum_{i=1}^nx_i = \frac{1}{n}\sum_{i=1}^n x_i y_i - \overline{x} \overline{y}
		\end{cases}\\
		&\text{ où } \overline{x} = \frac{1}{n} \sum_{i=1}^n x_i,~\overline{y} = \frac{1}{n}\sum_{i=1}^n y_i\\
		\iff& \begin{cases}
			a = \frac{\Cov(x,y)}{V(x)}\\
			b = \overline{y} - a\overline{x}
		\end{cases}
	\end{align*}

	Coefficient de corrélation: $\frac{\Cov(x,y)}{\sigma_x \sigma_y} \in [-1, 1]$
\end{exm}












		\part{Corps}

\begin{exm}[Problème]
	\begin{itemize}
		\item 
			avec $A = \Z / 9 \Z$, résoudre $\overline{x}^2 = \overline{0}$ \\
			\begin{center}
				\begin{tabular}{|c|c|c|c|c|c|c|c|c|c|c|}
					\hline
					$\overline{x}$&$\overline{0}$& $\overline{1}$ &$\overline{2}$&$\overline{3}$ &$\overline{4}$ &$\overline{5}$ &$\overline{6}$ &$\overline{7}$ &$\overline{8}$& $\overline{9}$ \\
					\hline
					$\overline{x}^2$&$\overline{0}$ &$\overline{1}$ &$\overline{4}$ &$\overline{0}$ &$\overline{7}$ &$7$ &$\overline{0}$ &$\overline{4}$ &$\overline{1}$&$\overline{0}$\\
					\hline
				\end{tabular}
			\end{center}
			On a trouvé 3 solutions: $\overline{0}$, $\overline{3}$, $\overline{6}$.
		\item $\Z / 8\Z$
			\begin{center}
				\begin{tabular}{|c|c|c|c|c|c|c|c|c|}
					\hline
					$\overline{x}$& $\overline{0}$& $\overline{1}$& $\overline{2}$& $\overline{3}$& $\overline{4}$& $\overline{5}$& $\overline{6}$& $\overline{7}$\\
					\hline
					$\overline{x^2}$& $\overline{0}$& $\overline{1}$& $\overline{4}$& $\overline{1}$& $\overline{0}$& $\overline{1}$& $\overline{4}$& $\overline{1}$\\
					\hline
				\end{tabular}
			\end{center}
			$\overline{x}^2=7$ a 4 solutions: $\overline{1}, \overline{7}, \overline{3},\text{ et } \overline{5}$
		\item $A = \mathbbm{H} = \{a + bi + cj + dk  \mid  (a,b,c,d) \in \R^4\}$ \\
			$i^2 = j^2 = k^2 = -1$ 
			\begin{align*}
				\begin{array}{c c c}
					ij = k & jk = i & ji = j\\
					ji = -k & kj = -i & ik = -j
				\end{array}
			\end{align*}
			Dans cet anneau, $-1$ a 6 racines!
	\end{itemize}
\end{exm}

\begin{defn}
	Soit $(\mathbbm{K}, +, \times)$ un ensemble muni de deux lois de composition internes. On dit que c'est un \underline{corps} si
	 \begin{enumerate}
		\item $(\mathbbm{K}, \times)$ est un groupe abélien
		\item $(\mathbbm{K}, \times)$ est un monoïde commutatif
		\item $\forall x \in \mathbbm{K}\setminus \{0_\mathbbm{K}\}, \exists y \in \mathbbm{K}, xy = 1_\mathbbm{K}$
		\item $0_\mathbbm{K} \neq  1_\mathbbm{K}$
	\end{enumerate}
	\index{corps}
\end{defn}

\begin{exm}
	\begin{itemize}
		\item $(\C, +, \times)$ est un corps
		\item $(\R, +, \times)$ est un corps
		\item $(\Q, +, \times)$ est un corps
		\item $(\Z, +, \times)$ n'est pas un corps
	\end{itemize}
\end{exm}

\begin{prop}
	$(\Z / n\Z, +, \times)$ est un corps si et seulement si $n$ est premier.
\end{prop}

\begin{prv}
	\[
		\left( \Z / n\Z \right)^\times = \left\{ \overline{k}  \mid k \wedge n = 1 \right\}
	\] 
\end{prv}


\begin{prop}
	Tout corps est un anneau intègre.
\end{prop}

\begin{prv}
	Soit $(\mathbbm{K}, +, \times)$ un corps. Soient $(a,b) \in \mathbbm{K}^2$ tel que $a \times b = 0_\mathbbm{K}$.\\
	On suppose $a \neq  0_\mathbbm{K}$. Alors, $a$ est inversible et donc \[
		b = a^{-1} \times a \times b = a^{-1} \times 0_\mathbbm{K} = 0_\mathbbm{K}
	\] 
\end{prv}

\begin{exm}
	Soit $(\mathbbm{K},+,\times)$ un corps.\\
	Résoudre \[
		\begin{cases}
			x^2 = 1_\mathbbm{K}\\
			x \in \mathbbm{K}
		\end{cases}
	\]

	\begin{align*}
		x^2 = 1_\mathbbm{K} &\iff x^2 - 1_\mathbbm{K} = 0_\mathbbm{K}\\
		&\iff (x - 1_\mathbbm{K})(x+1_\mathbbm{K}) = 0_\mathbbm{K}\\
		&\iff x - 1_\mathbbm{K} = 0_\mathbbm{K} \text{ ou } x + 1_\mathbbm{K} = 0_\mathbbm{K}\\
		&\iff x = 1_\mathbbm{K} \text{ ou } x = -1_\mathbbm{K}
	\end{align*}

	Il y a au plus 2 solutions.
\end{exm}

\begin{prop}
	Soit $(\mathbbm{K},+,\times )$ un corps et $P$ un polynôme à coefficients dans $\mathbbm{K}$ de degré $n$. Alors, l'équation $P(x) = 0_{\mathbbm{K}}$ a au plus $n$ solutions dans $\mathbbm{K}$ 
	\qed
\end{prop}

\begin{crlr}[(Théorème de Wilson)]
	voir exercice 16 du TD 12
\end{crlr}


\begin{defn}
	Soit $(\mathbbm{K}, +, \times)$ un corps et $L\subset \mathbbm{K}$.\\
	On dit que $L$ est un \underline{sous corps} de $\mathbbm{K}$ si
	\begin{enumerate}
		\item $L$ est un anneau de $(\mathbbm{K}, +, \times)$ non nul
		\item $\forall x \in L\setminus \{0_\mathbbm{K}\}, x^{-1} \in L$ 
	\end{enumerate}
	\vspace{2mm}
	en d'autres termes si
	\begin{enumerate}
		\item $\forall (x,y) \in L^2, x - y \in L$
		\item $\forall (x,y) \in L^2, x \times y^{-1} \in L$
	\end{enumerate}
	\vspace{5mm}
	On dit aussi que $\mathbbm{K}$ est une \underline{extension} de $L$.
	\index{sous corps}
	\index{extension}
\end{defn}

\begin{prop}
	Tout sous corps est un corps. \qed
\end{prop}

\begin{defn}
	Soient $(\mathbbm{K}_1,+,\times )$ et $(\mathbbm{K}_2,+, \times)$ deux corps et $f: \mathbbm{K}_1 \to \mathbbm{K}_2$.\\
	On dit que $f$ est un \underline{morphisme de corps} si $f$ est un morphisme d'anneaux.\\
	i.e. si
	\[
		\begin{cases}
			\forall (x,y) \in {\mathbbm{K}_1}^2,& f(x+y) = f(x) + f(y)\\
			\forall (x,y) \in {\mathbbm{K}_1}^2,& f(x \times y) = f(x) \times f(y)\\
		\end{cases}
	\] 
	\index{homomorphisme (de corps)}
	\index{morphisme (de corps)}
\end{defn}

\begin{prop}
	Tout morphisme de corps est injectif.
\end{prop}

\begin{prv}
	Soit $f: \mathbbm{K}_1 \to \mathbbm{K}_2$ un morphisme de corps.\\
	\begin{itemize}
		\item $\Ker(f)$ est un sous groupe de $(\mathbbm{K}_1, +)$ 
		\item Soit $x \in \Ker(f)$ et $y \in \mathbbm{K}_1$ \[
				f(x \times y) = f(x) \times f(y) = 0_{\mathbbm{K}_2} \times f(y) = 0_{\mathbbm{K}_2}
			\]
		\item Soit $x \in \Ker(f) \setminus \{0_{\mathbbm{K}_1}\}$.\\
			Alors, $x$ est inversible.\\
			\begin{align*}
				\begin{rcases*}
					x \in \Ker(f)\\
					x^{-1} \in \mathbbm{K}_1
				\end{rcases*}& \text{ donc } x \times x ^{-1} \in \Ker(f)\\
				&\text{ donc } 1_{\mathbbm{K}_1} \in \Ker(f)\\
				&\text{ donc } f(1_{\mathbbm{K}_1}) = 0_{\mathbbm{K}_2}
			\end{align*}
			Or, $f(1_{\mathbbm{K}_1}) = 1_{\mathbbm{K}_2} \neq 0_{\mathbbm{K}_2}$
	\end{itemize}
	Donc, $\Ker(f) = \{0_{\mathbbm{K}_1}\}$ donc $f$ est injective.
\end{prv}

\begin{exm}
	$\begin{array}{cc}
		\C &\longrightarrow \C\\
		z &\longmapsto \overline{z}\\
	\end{array}$ est un morphisme de corps
\end{exm}



	}

	{
		\chap[16]{Dérivation}
		\renewcommand{\cwd}{../chap16}
		\begin{defn}
	Soit $E$ un $\mathbbm{K}$-espace vectoriel. On dit que $E$ est de \underline{dimension finie} si $E$ a au moins une famille génératrice finie. On dit que $E$ est de \underline{dimension infinie} sinon.
	\index{dimension finie (espace vectoriel)}
	\index{dimension infinie (espace vectoriel)}
\end{defn}

\begin{thm}
	[Théorème de la base extraite]
	Soit $E$ un $\mathbbm{K}$-espace vectoriel non nul de dimension finie. Soit $\mathcal{G}$ une famille génératrice finie de $E$. Alors, il existe une base $\mathcal{B}$ de $\mathcal{E}$ telle que $\mathcal{B} \subset \mathcal{G}$.
\end{thm}

\begin{prv}
	[par récurrence sur $\#G = \Card(G)$]
	\begin{itemize}
		\item Soit $E$ un $\mathbbm{K}$-espace vectoriel non nul engendré par $\mathcal{G} = (u)$.\\
			Si $u = 0_E$, alors $E = \{0_E\}$: une contradiction $\lightning$ \\
			Donc $u \neq 0_E$ donc $(u)$ est libre. En effet, \[
				\forall \lambda \in \mathbbm{K}, \lambda u = 0_E \implies \lambda = 0_\mathbbm{K}
			\] Donc $\mathcal{G}$ est une base de $E$.\\
		\item Soit $n \in \N_*$. Soit $E$ un $\mathbbm{K}$-espace vectoriel. On suppose que si $E$ a une famille génératrice constituée de $n$ vecteurs, alors on peut extraire de cette famille une base de $E$.\\
			Soit $\mathcal{G}$ une famille génératrice de $E$ avec $n+1$ vecteurs.\\
			Si $\mathcal{G}$ est libre, alors $\mathcal{G}$ est une base de $E$. \\
			Si $\mathcal{G}$ n'est pas libre, alors il existe $u \in \mathcal{G}$ tel que $u \in \Vect(\mathcal{G}\setminus \{u\})$ \\
			Donc $\mathcal{G}\setminus \{u\}$ engendre $E$. Or, $\mathcal{G}\setminus \{u\}$ possède $n$ vecteurs. D'après l'hypothèse de récurrence, il existe une base $\mathcal{B}$ de $E$ telle que \[
				\mathcal{B} \subset \mathcal{G} \setminus \{u\} \subset \mathcal{G}
			\] 
	\end{itemize}
\end{prv}

\begin{crlr}
	Tout espace de dimension finie a une base.
	\qed
\end{crlr}

\begin{thm}
	[Théorème de la base incomplète]
	Soit $E$ un $\mathbbm{K}$-espace vectoriel de dimension finie, $\mathcal{G}$ une famille génératrice finie de $E$. $\mathcal{L}$ une famille libre de $E$. Alors, il existe une base $\mathcal{B}$ de $E$ telle que \[
		\mathcal{L} \subset \mathcal{B} \text{ et } \mathcal{B}\setminus \mathcal{L} \subset \mathcal{G}
	\] 
\end{thm}

\begin{prv}
	[par récurrence sur $\#(\mathcal{G}\setminus\mathcal{L})$]
	\begin{itemize}
		\item Avec les notations précédentes, on suppose que $\mathcal{G}\setminus\mathcal{L} \neq \O$ \[
				\forall u \in \mathcal{G}, u \in \mathcal{L}
			\] Donc $\mathcal{G} \subset \mathcal{L}$ donc $\mathcal{L}$ est génératrice donc $\mathcal{L}$ est une base de $E$. On pose $\mathcal{B} = \mathcal{L}$ et alors \[
				\mathcal{L} \subset  \mathcal{B} \text{ et } \mathcal{B}\setminus\mathcal{L} = \O \subset  \mathcal{G}
			\] 
		\item Soit $n \in \N$. On suppose que si $\mathcal{G}$ est génératrice et $\mathcal{L}$ libre avec $\#(\mathcal{G}\setminus\mathcal{L}) = n$ alors il existe une base $\mathcal{B}$ de $E$ telle que \[
			\mathcal{L}\subset \mathcal{B} \text{ et } \mathcal{B}\setminus\mathcal{L}\subset \mathcal{G}
		\] Soient à présent $\mathcal{G}$ une famille génératrice de $E$ et $\mathcal{L}$ une famille libre de $E$ telles que $\#(\mathcal{G}\setminus\mathcal{L}) = n+1 > 0$\\
		Si $\mathcal{L}$ engendre $E$, alors $\mathcal{L}$ est une base de $E$. On pose $\mathcal{B} = \mathcal{L}$ et on a bien \[
			\mathcal{L} \subset  \mathcal{B} \text{ et } \mathcal{B} \setminus \mathcal{L} = \O \subset  \mathcal{G}
		\] On suppose que $\mathcal{L}$ n'engendre pas $E$. Il existe $u \in \mathcal{G}$ tel que $u \not\in \Vec(\mathcal{L})$ (car sinon, $\mathcal{G} \subset \Vect(\mathcal{L})$ et donc $\underbrace{\Vect(\mathcal{G})}_{= E} \subset  \underbrace{\Vect(\mathcal{L})}_{ \subset E}$\\
		Donc $\mathcal{L} \cup \{u\} $ est libre. On pose $\mathcal{L}' = \mathcal{L} \cup \{u\} $ \[
			\mathcal{G}\setminus \mathcal{L}' = \mathcal{G}\setminus (\mathcal{L} \cup \{u\}) = (\mathcal{G}\setminus\mathcal{L})\setminus \{u\} 
		\] donc $\#(\mathcal{G}\setminus\mathcal{L}') = n+1 -1 = n$\\
		D'après l'hypothèse de récurrence, il existe $\mathcal{B}$ une base de $E$ telle que \[
			\mathcal{L} \subset  \mathcal{L}' \subset \mathcal{B} \text{ et } \mathcal{B}\setminus \mathcal{L}' \subset \mathcal{G}
		\] \[
			\mathcal{B} \setminus \mathcal{L} = \underbrace{\mathcal{B}\setminus\mathcal{L}'}_{\subset \mathcal{G}} \cup \underbrace{\{u\}}_{\subset \mathcal{G} \text{ car } u \in \mathcal{G}}
		\] On a $\mathcal{B}\setminus\mathcal{L}\subset \mathcal{G}$
	\end{itemize}
\end{prv}

\begin{thm}
	Soit $E$ un $\mathbbm{K}$-espace vectoriel de dimension finie. Toutes les bases de $E$ ont le même cardinal.
\end{thm}

\begin{prv}
	Soit $\mathcal{G}$ une famille génératrice finie de $E$ et $\mathcal{B} \subset  \mathcal{G}$ une base de $E$. On note $n = \#\mathcal{B}$ \\
	Soit $\mathcal{B}'$ une base de $E$. On pose $p = n - \#(\mathcal{B} \cap  \mathcal{B}')$. Montrons par récurrence sur  $p$ que $\#\mathcal{B} = \#\mathcal{B}'$ 
	\begin{itemize}
		\item On suppose que $p = 0$. Alors, $\#(\mathcal{B} \cap \mathcal{B}') = n$ \\
			Or, $\mathcal{B}' \cap \mathcal{B} \subset \mathcal{B}$ donc $\mathcal{B} \cap \mathcal{B}' = \mathcal{B}$ donc $\mathcal{B} \subset  \mathcal{B}'$ et donc $\mathcal{B} = \mathcal{B}'$ 
		\item Soit $p \in \N$. On suppose que si $\mathcal{B}'$ est une base de $E$ telle que $n - \#(\mathcal{B} \cap \mathcal{B}') = p$, alors $\#\mathcal{B}' = n$ \\
			Aoit $\mathcal{B}'$ une base de $E$ telle que $n - \#(\mathcal{B}\cap \mathcal{B}') = p+1 > 0$ \\
			Donc $\mathcal{B} \cap \mathcal{B}' \neq \mathcal{B}$. Soit $u \in \mathcal{B}' \setminus \mathcal{B}$. D'après le lemme d'échange, il existe $v \in \mathcal{B}\setminus \mathcal{B}'$ tel que $\mathcal{B}' \setminus \{u\} \cup \{v\}$ est une base de $E$. On pose $\mathcal{B}'' = \mathcal{B}' \setminus \{u\} \cup \{v\}$ 
			\begin{align*}
				\mathcal{B}'' \cap \mathcal{B} &= \left( (\mathcal{B}' \setminus \{u\})  \cap \mathcal{B} \right) \cup \{v\} \\
				&= (\mathcal{B}' \cap \mathcal{B}) \cup \{v\} \\
			\end{align*}
			donc,
			\begin{align*}
				n - \#(\mathcal{B}'' \cap \mathcal{B}) &= n - (\#(\mathcal{B}' \cap \mathcal{B}) + 1) \\
				&= p+1- 1 \\
				&= p \\
			\end{align*}
			D'après l'hypothèse de récurrence, \[
				\#\mathcal{B}'' = n
			\] Or, $\#\mathcal{B}'' = \#\mathcal{B}'$
	\end{itemize}
\end{prv}

\begin{lem}
	Soient $\mathcal{B}$ et $\mathcal{B}'$ deux bases de $E$ telles que $\mathcal{B}\subset \mathcal{B}'$. Alors, $\mathcal{B} = \mathcal{B}'$.
\end{lem}

\begin{prv}
	On suppose $\mathcal{B}' \neq \mathcal{B}$. Soit $u \in \mathcal{B}' \setminus \mathcal{B}$
	$u \in E = \Vect(\mathcal{B})$ donc $\mathcal{B} \cup \{u\}$ n'est pas libre.
	Donc $\mathcal{B}\cup \{u\} \subset \mathcal{B}'$ et $\mathcal{B}'$ est libre donc $\mathcal{B}\cup \{u\}$ est libre: une contradiction $\lightning$
\end{prv}

\begin{lem}
	[Lemme d'échange] Soient $\mathcal{B}_1$ et $\mathcal{B}_2$ deux bases de $E$ et $u \in \mathcal{B}_1 \setminus \mathcal{B}_2$. Alors, il existe $v \in \mathcal{B}_2$ tel que $(\mathcal{B}_1 \setminus \{u\}) \cup \{v\}$ soit une base de $E$.
\end{lem}

\begin{prv}
	[1${}^\text{nde}$ méthode]
	On suppose que pout tout $v \in \mathcal{B}_2$, $(\mathcal{B}_1\setminus \{u\}) \cup \{v\}$ n'est pas une base de $E$
	Soit $v \in \mathcal{B}_2$.
	\begin{itemize}
		\item Supposons $(\mathcal{B}_1\setminus \{u\})\cup \{v\}$ non libre. $\mathcal{B}_1 \setminus \{u\}$ est libre. Donc $v \in \Vect(\mathcal{B}_1 \setminus \{u\})$
		\item Supposons $(\mathcal{B}_1\setminus \{u\}) \cup \{v\}$ non génératrice.
			Comme $\mathcal{B}_1$ engendre $E$, $u \not\in \Vect(\mathcal{B}_1\setminus \{v\})$.
			On suppose que $\mathcal{B}_1 \neq \mathcal{B}_2$.
			$\forall v \in \mathcal{B}_2 \setminus \mathcal{B}_1, \Vect(\mathcal{B}_1 \setminus \{v\}) = \Vect(\mathcal{B}_1) = E \ni u$ 
			donc, $(\mathcal{B}_1\setminus \{u\}) \cup \{v\}$ engendre $E$ et donc \[
				v \in \Vect(\mathcal{B}_1 \setminus \{u\})
			\] On a aussi \[
				\forall v \in \mathcal{B}_1 \setminus \{u\}, v \in \Vect(\mathcal{B}_1\setminus \{u\})
			\] Comme $u \not\in \mathcal{B}_2$, on a \[
				\forall v \in \mathcal{B}_2, v \in \Vect(\mathcal{B}_1\setminus \{u\})
			\] docn \[
				E = \Vect(\mathcal{B}_2) \subset \Vect(\mathcal{B}_1\setminus \{u\})
			\] donc $\mathcal{B}_1\setminus \{u\}$ engendre $E$ donc $\mathcal{B}_1\setminus \{u\}$ est une base de $E$. Or, $\mathcal{B}_1 \setminus \{u\}  \subset  \mathcal{B}_1$, donc $\mathcal{B}_1\setminus \{u\} = \mathcal{B}_1$
	\end{itemize}
\end{prv}

\begin{prv}
	[2${}^\text{nde}$ méthode]
	On suppose que pout tout $v \in \mathcal{B}_2$, $(\mathcal{B}_1\setminus \{u\}) \cup \{v\}$ n'est pas une base de $E$
	\begin{itemize}
		\item Comme $u \in \mathcal{B}_1 \setminus \mathcal{B}_2$, nécéssairement $\mathcal{B}_1 \neq \mathcal{B}_2$ donc $\mathcal{B}_2 \not\subset \mathcal{B}_1$, donc $\mathcal{B}_2\setminus\mathcal{B}_1 \neq \O$ 
		\item Soit $v \in \mathcal{B}_2\setminus\mathcal{B}_1$. Il existe $(\lambda_w)_{w\in\mathcal{B}_1}$ une famille de scalaires presque nulle telle que \[
				v = \sum_{w \in \mathcal{B}_1} \lambda_w w - \lambda_u u + + \sum_{w \in \mathcal{B}_1\setminus \{u\}}\lambda_w w
			\]
			Si $\lambda_u \neq 0_E$, alors
			\begin{align*}
				u &= \lambda_u^{-1}\left( v - \sum_{w \in \mathcal{B}_1 \setminus \{u\}} \lambda_w w \right)\\
					&\in \Vect(\mathcal{B}_1\setminus \{u\} \cup v)
			\end{align*}
			 donc $\mathcal{B}_1 \subset \Vect(\mathcal{B}_1\setminus \{u\} \cup \{v\})$\\
			 et donc $E \subset  \Vect(\mathcal{B}_1 \setminus \{u\} \cup \{v\})$ \\
			 et donc $\mathcal{B}_1 \setminus \{u\} \cup \{v\}$ engendre $E$ \\
			 donc $\mathcal{B}_1 \setminus \{u\} \cup \{v\}$ n'est pas libre\\
			 donc $v \in \Vect(\mathcal{B}_1\setminus \{u\})$ (car $\mathcal{B}_1 \setminus \{u\}$ est libre\\
			 donc $\lambda_u = 0_\mathbbm{K}$ $\lightning$\\`

			 Donc, $\lambda_u = 0_\mathbbm{K}$, docn $v \in \Vect(\mathcal{B}_1\setminus \{u\})$ \\
			 On vient de prouver que
			 \begin{align*}
			 	\mathcal{B}_2 \setminus \mathcal{B}_1 \subset \Vect(\mathcal{B}_1 \setminus \{u\})\\
			 	\mathcal{B}_1 \setminus \{u\} \subset \Vect(\mathcal{B}_1 \setminus \{u\})\\
			 \end{align*}
			 Comme $u \not\in \mathcal{B}_2$, \[
			 	\mathcal{B}_2 \subset \Vect(\mathcal{B}_1 \setminus \{u\})
			 \] donc \[
			 	E = \Vect(\mathcal{B}_2) \subset  \Vect(\mathcal{B}_1 \setminus \{u\})
			 \] donc $\mathcal{B}_1 \setminus \{u\}$ engendre $E$. Donc,  $\mathcal{B}_1 \setminus \{u\}$ est une base de $E$.\\
			 Or, $\mathcal{B}_1 \setminus \{u\} \subset  \mathcal{B}_1$, donc $\mathcal{B}_1 \setminus \{u\} = \mathcal{B}_1$
	\end{itemize}
\end{prv}

\begin{defn}
	Soit $E$ un $\mathbbm{K}$-espace vectoriel de dimension finie. Le cardinal commun à toutes les bases de $E$ est appelé \underline{dimension} de $E$ est notée $\dim(E)$ ou $\dim_\mathbbm{K}(E)$\\
	C'est donc aussi le nombre de coordonnées de n'importe quel vecteur dans n'importe quelle base.
	\index{dimension (espace vectoriel)}
\end{defn}

\begin{exm}
	\begin{enumerate}
		\item $\dim_\R(\C) = 2$ et $\dim_\C(\C) = 1$ 
		\item $\dim_\mathbbm{K}(\mathbbm{K}^{n}) = n$ 
		\item $\dim_{\mathbbm{K}}(\mathcal{M}_{n,p}(\mathbbm{K})) = np$
	\end{enumerate}
\end{exm}

\begin{crlr}
	Soit $E$ un $\mathbbm{K}$-espace vectoriel de dimension finie, $\mathcal{L}$ une famille libre de $E$, $\mathcal{G}$ une famille génératrice de $E$. On note $n = \dim(E)$
	\begin{enumerate}
		\item $\#\mathcal{G} \ge n$ et $(\#\mathcal{G} = n \implies \mathcal{G} \text{ est une base de } E$)
		\item $\#\mathcal{L} \le n$ et $(\#\mathcal{L} = n \implies \mathcal{L} \text{ est une base de } E$)
	\end{enumerate}
\end{crlr}

\begin{crlr}
	$\R^{\R}$ est de dimension infinie.
	$\forall i \in \N, e_i: x \mapsto x^i$\\
	$(e_i)_{i\in\N}$ est libre dans $\R^\R$
\end{crlr}

\begin{prop}
	Soient $E$ et $F$ deux $\mathbbm{K}$-espaces vectoriels de dimension finie. Alors $E\times F$ est de dimension finie et $\dim(E\times F) = \dim(E) + \dim(F)$
\end{prop}

\begin{prv}
	Soit $(e_1,\ldots, e_n)$ une base de $E$, $(f_1, \ldots, f_p)$ une base de $F$.
	On pose \[
		\left\{\begin{array}
			{r c l}
			u_1 &=& (e_1,0_F)\\
			u_2 &=& (e_2,0_F)\\
					&\vdots&\\
			u_n &=& (e_n,0_F)\\
			u_{n+1} &=& (0_E, f_1)\\
			u_{n+2} &=& (0_E, f_2)\\
					&\vdots&\\
			u_{n+p} &=& (0_E,f_p)\\
		\end{array}\right.
	\]
	Soit $(x,y) \in E\times F$. \[
		\begin{cases}
			\exists (x_1,\ldots,x_n)\in \mathbbm{K}^n, x = \sum_{i=1}^{n} x_ie_i
			\exists (y_1,\ldots,y_n)\in \mathbbm{K}^n, x = \sum_{j=1}^{p} y_jf_j
		\end{cases}
	\] 
	\begin{align*}
		(x,y) &= \left( \sum_{i=1}^{n} x_ie_i, \sum_{i=1}^{p} y_jf_j \right)  \\
		&= \sum_{i=1}^{n} x_i (e_i + 0_F) + \sum_{j=1}^{p} y_j (0_E, f_j) \\
		&= \sum_{i=1}^{n} x_i u_i + \sum_{j=1}^{p} y_j u_{n+j} \\
	\end{align*}
	Donc, $E\times F = \Vect(u_1, \ldots, u_{n+p})$ donc $E\times F$ est de dimension finie.\\
	Soit $(\lambda_1, \ldots, \lambda_{n+p}) \in \mathbbm{K}^{n+p}$ tel que \[
		(*): \quad \sum_{k=1}^{n+p} \lambda_ku_k = 0_{E\times F} = (0_E, 0_F)
	\]
	\begin{align*}
		(*) &\iff \sum_{k=1}^{n} \lambda_k (e_k, 0_F) + \sum_{k=n+1}^{p} \lambda_k(0_E, f_{k-n}) = (0_E, 0_F)\\
				&\iff \begin{cases}
					\sum_{k=1}^{n} \lambda_k e_k = 0_E\\
					\sum_{k=n+1}^{p} \lambda_k f_{k-n} = 0_F
				\end{cases}\\
				&\iff \begin{cases}
					\forall k \in \left\llbracket 1,n \right\rrbracket, \lambda_k = 0_\mathbbm{K} \qquad&(\text{car $(e_1,\ldots,e_n)$ est libre})\\
					\forall k \in \left\llbracket n+1,n+p \right\rrbracket, \lambda_k = 0_\mathbbm{K} \qquad&(\text{car $(f_1,\ldots,f_n)$ est libre})\\
				\end{cases}
	\end{align*}
	Donc $(u_1, \ldots, u_{n+p})$ est une base de $E\times F$. Donc, $\dim(E\times F) = n + p = \dim(E) + \dim(F)$
\end{prv}

\begin{rmk}
	[Convention]
	\[\dim\big(\{0_E\}\big) = 0\]
\end{rmk}

\begin{thm}
	Soit $E$ un $\mathbbm{K}$-espace vectoriel de dimension finie, $F$ un sous-espace vectoriel de $E$. Alors, $F$ est de dimension finie et  $\dim(F) \le \dim(E)$\\
	Si $\dim(F) = \dim(E)$, alors $F = E$
\end{thm}

\begin{prv}
	On considère \[
		A = \{k \in \N \mid \text{il existe une famille libre de $F$ à $k$ éléments}\} 
	\]
	On suppose $F \neq \{0_E\}$.
	\begin{itemize}
		\item Soit $u \in F\setminus \{0_E\}$. $(u)$ est libre donc $1 \in A$ et donc $A \neq \O$
		\item Soit $\mathcal{L}$ une famille libre de $F$. Alors, $\mathcal{L}$ est une famille libre de $E$ \\
			donc $\#\mathcal{L} \le \dim(E)$\\
			Donc $A$ est majorée par $\dim(E)$ \\
			On en déduit que $A$ a un plus grand élément $p$.
		\item Soit $\mathcal{L}$ une famille libre de $F$ avec $p$ éléments.\\
			Si $\mathcal{L}$ n'engendre pas $F$, alors il existe $u\in F$ tel que $u\not\in \Vect(\mathcal{L})$ et donc $\mathcal{L} \cup \{u\}$ est une famille libre de $F$, donc $p+1 \in A$ en contradiction avec la maximalité de $p$.\\
			Donc $\mathcal{L}$ est une base de $F$ donc $F$ est de dimension finie et $\dim(F) = p \le \dim(E)$\\
	\end{itemize}

	Soit $\mathcal{B}$ une base de $F$. Alors, $\mathcal{B}$ est aussi une famille de libre de de $E$. Donc $\#\mathcal{B} \le \dim(E)$ donc $\dim(F) = \dim(E)$ \\
	Si $\dim(F) = \dim(E)$, alors $\mathcal{B}$ est une base de $E$, et donc $F = \Vect(\mathcal{B}) = E$
\end{prv}

\begin{prop}
	[Formule de Grassmann]
	Soit $E$ un $\mathbbm{K}$-espace vectoriel de dimension finie, $F$ et $G$ deux sous-espace vectoriels de $E$. Alors, \[
		\dim(F+G) = \dim(F) + \dim(G) - \dim(F\cap G)
	\] 
\end{prop}

\begin{prv}
	Soit $(e_1, \ldots, e_p)$ une base de $F\cap G$. $(e_1,\ldots,e_p)$ est une famille libre de $F$.\\
	On complète $(e_1, \ldots, e_p)$ en une base $(e_1, \ldots, e_p, u_1, \ldots, u_q)$ de $F$.\\
	De même, on complète $(e_1, \ldots, e_p)$ en une base $(e_1, \ldots, e_p, v_1, \ldots, v_r)$ de $G$.\\
	On pose  $\mathcal{B} = (e_1, \ldots, e_p, u_1, \ldots, u_q, v_1, \ldots, v_r)$. Montrons que $\mathcal{B}$ est une base de $F+G$
	\begin{itemize}
		\item Soit $u \in F+G$ \\
			On pose $u = v+w$ avec $\begin{cases}
				v\in F\\
				w \in G
			\end{cases}$.\\
			On pose $v = \sum_{i=1}^p \lambda_i e_i + \sum_{i=1}^q \mu_i u_i$ avec $(\lambda_1, \ldots, \lambda_p, \mu_1, \ldots, \lambda_q) \in \mathbbm{K}^{p+q}$\\
			On pose aussi $w = \sum_{i = 1}^p \lambda'_ie_i + \sum_{j=1}^r \nu_j v_j$ avec $(\lambda_1',\ldots,\lambda_p', \nu_1, \ldots, \nu_r) \in \mathbbm{K}^{p+r}$\\
			D'où, \[
				u = \sum_{i=1}^p (\lambda_i + \lambda'_i)e_i + \sum_{j=1}^q \mu_j u_j + \sum_{k=1}^r \nu_k v_k \in \Vect(\mathcal{B})
			\]
		\item Soient $(\lambda_1, \ldots, \lambda_p, \mu_1, \ldots, \mu_q, \nu_1, \ldots, \nu_r) \in \mathbbm{K}^{p+q+r}$.\\
			On suppose \[
				(*)\quad \sum_{i=1}^{p}\lambda_ie_i + \sum_{j=1}^q\mu_ju_j + \sum_{k=1}^r \nu_k v_k = 0_E
			\] 
			D'où, \[
				\underbrace{\sum_{i=1}^p\lambda_i e_i + \sum_{j=1}^q \mu_ju_j}_{\in F} = \underbrace{-\sum_{k=1}^r\nu_jv_k}_{\in G}
			\] 
			Donc, \[
				f = \sum_{i=1}^p \lambda_i e_i + \sum_{j=1}^q \mu_j u_j \in F\cap G
			\] Comme $(e_1, \ldots, e_p)$ est une base de $F\cap G$, $\exists ! (\lambda_1', \ldots, \lambda_p') \in \mathbbm{K}^p$ tel que \[
				f = \sum_{i=1}^p \lambda'_i e_i = \sum_{i=1}^p \lambda'_i e_i + \sum_{j=1}^q 0_\mathbbm{K}u_j
			\] Comme $(e_1, \ldots, e_p, u_1, \ldots, u_q)$ est une base de $F$, \[
				\forall k \in \left\llbracket 1, q \right\rrbracket, \mu_j = 0_\mathbbm{K}
			\] De même, \[
				\forall k \in \left\llbracket 1,r \right\rrbracket , \nu_k = 0_\mathbbm{K}
			\] On remplace dans $(*)$ pour trouver \[
				\sum_{i=1}^p \lambda_ie_i = 0_E
			\] Comme $(e_1, \ldots, e_p)$ est libre, \[
				\forall i \in \left\llbracket 1,p \right\rrbracket, \lambda_i = 0_\mathbbm{K}
			\] Donc $\mathcal{B}$ est libre.\\
			Donc, 
			\begin{align*}
				\dim(F+G) &=  p +q + r \\
				&= (p+q)+ (p+r) - p \\
				&= \dim(F) + \dim(G) - \dim(F\cap G) \\
			\end{align*}
	\end{itemize}
\end{prv}

\begin{crlr}
	Avec les hypothèse précédentes, \[
		E = F \oplus G \iff \begin{cases}
			F \cap  G = \{0_E\} \\
			\dim(E) = \dim(F) + \dim(G)
		\end{cases}
	\] 
\end{crlr}

\begin{prv}
	\begin{itemize}
		\item[``$\implies$''] On suppose $E = F \oplus G$ \\
			Comme la somme est directe, $F \cap G = \{0_E\}$ 
			\begin{align*}
				\dim(E) &= \dim(F)\\
				&= \dim(F) + \dim(G) - \dim(F\cap G)\\
				&= \dim(F) + \dim(G)\\
			\end{align*}
		\item[``$\impliedby$''] On suppose $F\cap G = \{0_E\}$ et $\dim(E) = \dim(F) + \dim(G)$.\\
			On sait déjà que $F+G = F \oplus G$\\
			 \begin{align*}
				\dim(F+G) = \dim(F) + \dim(G) - \dim(F \cap G) = \dim(E)
			\end{align*}
			Donc $F + G = E$
	\end{itemize}
\end{prv}

\begin{prop}
	Soit $F$ un $\mathbbm{K}$-espace vectoriel de dimension finie $n$. Soit $\mathcal{B} = (e_1, \ldots, e_n)$ une base de $F$. L'application
	\begin{align*}
		f: \mathbbm{K}^n &\longrightarrow F \\
		(\lambda_1, \ldots, \lambda_n) &\longmapsto \sum_{i=1}^n \lambda_i e_i
	\end{align*} est bijective.\\
	Si $\mathbbm{K}$ est infini, $\mathbbm{K}^n$ aussi et donc $F$ aussi.\\
	Si $\#\mathbbm{K} = p \in \N_*$,
	\begin{align*}
		\#&\mathbbm{K}^n = p^n\\
		&\vrt=\\
		\#&F
	\end{align*}
\end{prop}


		\part{Dérivation}

\underline{Motivation}:

{
\begin{wrapfigure}{l}{3cm}
	\centering
	\begin{asy}
		import three;

		size(3cm);
		settings.render=0;
		settings.prc=false;
		currentprojection = obliqueZ;

		draw(unitbox);
		draw(shift(1.1Z + 0.05X) * (O -- X), Arrows3(TeXHead2));
		draw(shift(1.1Z + 0.05Y) * (O -- Y), Arrows3(TeXHead2));
		draw(shift(1.1X + 0.05Z) * (O -- Z), Arrows3(TeXHead2));

		label("$x$", (X/2) + (1.1Z + 0.05X), align=S);
		label("$y$", (Y/2) + (1.1Z + 0.05Y), align=W);
		label("$z$", (Z/2) + X, align=SE);
	\end{asy}
\end{wrapfigure}

\begin{align*}
	&S(x,y,z) = 2(xy + xz + yz)\\
	&V(x,y,z) = xyz
\end{align*}

On cherche à minimiser $S$ avec la contrainte $V = 1$.

Soit $f : \begin{array}{rcl}
	\left( \R_*^+ \right)^2 &\longrightarrow& \R \\
	(x,y) &\longmapsto& S\left( x,y,\frac{1}{xy} \right) = 2\left( xy + \frac{1}{y} + \frac{1}{x} \right).
\end{array}$

On cherche $(a,b) \in \left( \R^+_* \right)^2$ tel que \[
	\forall (x,y) \in (\R^+_*), f(x,y) \ge f(a,b).
\]
}

\begin{defn}
	Soit $f: U \to \R$ où $U$ est un ouvert de $\R^2$. Soit $(a,b) \in U$.
	\vspace{2mm}

	Si $\lim_{x \to a} \frac{f(x,b) - f(a,b)}{x - a} \in \R$, alors on dit que $f$ a une dérivée partielle suivant $x$ en $(a,b)$ et cette limite est notée \[
		\partial f_1(a,b) = \frac{\partial f}{\partial x}(a,b).
	\]

	Si $\lim_{y \to b} \frac{f(a,y) - f(a,b)}{y - b} \in \R$, alors on dit que $f$ a une dérivée partielle suivant $y$ et la limite est notée \[
		\partial f_2(a,b) = \frac{\partial f}{\partial y}(a,b).
	\]
\end{defn}

\begin{exm}
	\begin{enumerate}
		\item $f: (x,y) \mapsto xy + x - y$.

			\begin{align*}
				&\frac{\partial f}{\partial x} : (x,y) \mapsto y + 1,\\
				&\frac{\partial f}{\partial y} : (x,y) \mapsto x - 1.
			\end{align*}

		\item $f: (x,y) \mapsto xy + \frac{1}{y}+ \frac{1}{x}$.

			\begin{align*}
				&\frac{\partial f}{\partial x}: (x,y) \mapsto y - \frac{1}{x^2},\\
				&\frac{\partial f}{\partial y}: (x,y) \mapsto x - \frac{1}{y^2}.
			\end{align*}

		\item Trouver $f$ telle que $\begin{cases}
				(1): \qquad \frac{\partial f}{\partial x}=y,\\[2mm]
				(2): \qquad \frac{\partial f}{\partial y} = x.
			\end{cases}$

			D'après $(1)$ : \[
				\forall (x,y), \exists C(y) \in \R, f(x,y) = xy + C(y)
			\] et donc \[
				\frac{\partial f}{\partial y}(x,y) = x + C'(y)
			\] donc $C'(y) = 0$ et donc $C$ est constante.

		\item Trouver $f$ telle que $\begin{cases}
			\frac{\partial f}{\partial x} = -y,\\[2mm]
			\frac{\partial f}{ƒ\partial y} = x.
		\end{cases}$

		Ce n'est pas possible !
	\end{enumerate}
\end{exm}

\begin{defn}~\\
	\begin{minipage}{\linewidth}
		\begin{wrapfigure}{r}{4cm}
			\centering
			\vspace{-5mm}
			\begin{asy}
				import three;
				import graph3;
				size(4cm);

				settings.render = 0;
				settings.prc = false;
				currentprojection = obliqueX;

				draw(O -- X, Arrow3(TeXHead2));
				draw(O -- Y, Arrow3(TeXHead2));
				draw(O -- Z, Arrow3(TeXHead2));

				triple f(real x, real y, real z = 0) { return (x,y,cos(x - 0.5) * cos(y - 0.5)/1.2 + 0.15); }

				real inc = 1 / 5;

				for(real x = 0; x <= 1; x += inc) {
					draw(graph(
						new real(real t) { return x; }, // x
						new real(real y) { return y; }, // y
						new real(real y) { return f(x,y).z; }, // z
						0, 1
					), gray);
				}

				for(real y = 0; y <= 1; y += inc) {
					draw(graph(
						new real(real x) { return x; }, // x
						new real(real t) { return y; }, // y
						new real(real x) { return f(x,y).z; }, // z
						0, 1
					), gray);
				}

				path3 path1 = (0.8, 0.2, 0) .. (0.5, 0.5, 0) .. (0.3, 0.7, 0);
				path3 path2 = f(0.8, 0.2, 0) .. f(0.5, 0.5, 0) .. f(0.3, 0.7, 0);
				path3 d = (0.2, 0.3, 0) .. (0.3, 0.4, 0) .. (0.2, 0.7, 0) .. (0.8, 0.9, 0) .. (0.6, 0.2, 0) .. cycle;

				draw(path1, red, Arrow3(TeXHead2));
				draw(path2, red, Arrow3(TeXHead2, position=0.8));

				dot((0.5, 0.5, 0));
				dot(f(0.5, 0.5, 0));
				draw((0.5, 0.5, 0) -- f(0.5, 0.5, 0), dashed);
				draw(d);

				label("$w$", (0.3, 0.7, 0), red, align=SE);
				label("$U$", (0.8, 0.9, 0), align=SE);
			\end{asy}
		\end{wrapfigure}

		Soit $f: U \to \R$ où $U$ est un ouvert. Soit $(a,b) \in U$. Soit $w = (w_1, w_2) \in \R^2$.

		Si 
		\[
			\lim_{t\to 0} \frac{f(a + tw_1, b + tw_2) - f(a,b)}{t}
		\] existe et est réelle, alors on dit que $f$ a une dérivée dans la direction de $w$ et la limite est notée \[
			\mathrm{d}f(w)\,(a,b) = D_w(f)\,(a,b).
		\]
	\end{minipage}
\end{defn}

\begin{exm}
	\begin{align*}
		f: \left( \R_*^+ \right)^2 &\longrightarrow \R \\
		(x,y) &\longmapsto xy+\frac{1}{x}+\frac{1}{y}.
	\end{align*}

	On pose $(a,b) = (1,2)$, $w = (w_1, w_2) = (1,1)$.
	\begin{align*}
		\frac{f(1+t, 2+t) - f(1,2)}{t} &= \frac{1}{t} \left( (1+t)(2+t) + \frac{1}{1+t} + \frac{1}{2+t} - 3 - \frac{1}{2} \right) \\
		&= \frac{1}{t}\left(\cancel 2 + 3t + \po(t) + \cancel 1 - t + \po(t) + \frac{1}{2}\left( \cancel 1 - \frac{t}{2} + \po(t) \right) - \cancel3 - \cancel{\frac{1}{2}} \right) \\
		&= \frac{1}{t} \left( \frac{7}{4} t + \po(t) \right)  \\
		&= \frac{7}{4} + \po(1) \tendsto{t \to 0} \frac{7}{4}. \\
	\end{align*}

	Donc, \[
		\mathrm{d}f(1,1)\,(1,2) = \frac{7}{4}.
	\]
\end{exm}

\begin{rmk}~\\
	\begin{figure}[H]
		\centering
		\begin{asy}
			import solids;
			import graph;
			size(5cm);

			settings.render = 0;
			settings.prc = false;

			path3 par = graph(
				new real(real x) { return x; },
				new real(real x) { return 0; },
				new real(real x) { return x^2; },
				0,3);
			revolution r = revolution(par, axis=Z);

			path3 par2 = graph(
				new real(real x) { return x; },
				new real(real x) { return 0; },
				new real(real x) { return x^2; },
				-3,3);

			draw(r,1,longitudinalpen=nullpen);
			draw(r.silhouette());

			draw((-4, 0, -1) -- (-4, 0, 10) -- (4, 0, 10) -- (4, 0, -1) -- cycle, red);
			draw(par2, deepred);

			draw((4,4.5) -- (7, 4.5), black+0.5mm, Arrow(TeXHead));

			path par2d = graph(new real(real x) { return x^2; }, -3, 3);
			draw(shift((11, 0)) * par2d, deepred);

			dot(O);
			dot((11, 0));
		\end{asy}
	\end{figure}
\end{rmk}


%todo ajouter théorème-définition
\begin{thm}
	Soit $f : U \to \R$, $(a,b) \in U$. On suppose que $\frac{\partial f}{\partial x}$ et $\frac{\partial f}{\partial y}$ existent en $(a,b)$ et sont {\bfseries continues} en $(a,b)$. Alors,
	\begin{align*}
		&\forall (h, k) \in \R^2 \text{ tel que } (a +h, b + k) \in U,\\
		&f(a+ h, b + k) = f(a,b) + h \frac{\partial f}{\partial x}(a,b) + k \frac{\partial f}{\partial y}(a,b) + \po_{(h,k)\to (0,0)}\big(\|(h,k)\|\big).
	\end{align*}

	On dit que $f$ est de classe $\mathcal{C}^1$ si $\frac{\partial f}{\partial x}$ et $\frac{\partial f}{\partial y}$ existent et sont continues.

	\qed
\end{thm}

\begin{rmk}
	En physique, cette formule correspond à : \[
		\mathrm{d}f = \frac{\partial f}{\partial x}\mathrm{d}x + \frac{\partial f}{\partial y} \mathrm{d}y.
	\] En effet :
	\begin{align*}
		\mathrm{d}f &= f(x+ \mathrm{d}x, y + \mathrm{d}y) - f(x,y) \\
		&= \frac{\partial f}{\partial x} \mathrm{d}x + \frac{\partial f}{\partial y} \mathrm{d}y.
	\end{align*}
\end{rmk}

\begin{prop}
	Soit $f: U \to \R$ de classe $\mathcal{C}^1$ en $(a,b) \in U$. Alors, \[
		\forall w = (w_1, w_2) \in \R^2, \mathrm{d}f(w)\,(a,b) = w_1 \frac{\partial f}{\partial x}(a,b) + w_2 \frac{\partial f}{\partial y}(a,b).
	\]
\end{prop}

\begin{prv}
	Soit $w = (w_1, w_2) \in \R^2$. Soit $t \in \R^*$.
	\begin{align*}
		\frac{1}{t}\big(f(a + tw_1, b + tw_2) - f(a,b)\big)
		&= \frac{1}{t} \left( tw_1 \frac{\partial f}{\partial x}(a,b) + tw_2 \frac{\partial f}{\partial y}(a,b) + \po_{t \to 0}\big(\|tw\|\big) \right) \\
		&= w_1 \frac{\partial f}{\partial x}(a,b) + w_2 \frac{\partial f}{\partial y}(a,b) + \po_{t\to 0}(1) \\
		&\tendsto{t\to 0} w_1 \frac{\partial f}{\partial x}(a,b) + w_2\frac{\partial f}{\partial y}(a,b).
	\end{align*}
\end{prv}


\begin{defn}
	Avec les hypothèses précédentes, en posant \[
		\nabla f(a,b) = \left( \frac{\partial f}{\partial x}(a,b), \frac{\partial f}{\partial y}(a,b) \right) 
	\]on obtient \[
		\mathrm{d}f(w)\,(a,b) = \left<w  \mid \nabla f(a,b) \right>
	\] où $\left<\cdot|\cdot \right>$ est le produit scalaire.

	Le vecteur $\nabla f(a,b)$ est appelé \underline{gradient de $f$ en $(a,b)$}.

	Le développement limité à l'ordre 1 de $f$ devient \[
		f\big((a,b)+w\big) = f(a,b) + \left<w \mid \nabla f(a,b) \right> + \po_{w\to 0}(\|w\|)
	\]
\end{defn}

\begin{prop}
	Soit $f : U \to \R$ de classe $\mathcal{C}^1$.

	\begin{figure}[H]
    \centering
    \incfig{gradient}
	\end{figure}

	$\nabla f$ est orthogonal au lignes de niveaux de $f$, son orientation va dans le sens d'une augmentation de $f$.
\end{prop}

\begin{prv}
	Soit $\gamma : I \to U$ une courbe de niveau : \[
		\forall t \in I, f\big(\gamma(t)\big) = \text{cste}.
	\] D'après le lemme suivant : \[
		\forall t \in I, 0 = (f \circ \gamma)'(t) = \mathrm{d}f\big(\gamma'(t)\big)\big(\gamma(t)\big) = \left<\gamma'(t)  \mid \nabla f\big(\gamma(t)\big) \right>
	\] Donc $\nabla f\big(\gamma(t)\big)$ est orthogonal à $\gamma'(t)$.

	Pour tout $t \in I$, on pose $w(t) = t\, \nabla f\big(\gamma(t)\big)$. Donc \[
		f\big(\gamma(t) + w(t)\big) = f\big(\gamma(t)\big) + t \|\nabla f(\gamma(t))\|^2 + \po_{t \to 0}(t)
	\] Pour $t$ assez petit, $f\big(\gamma(t) + w(t)\big) - f\big(\gamma(t)\big)$ est du même signe que $t$.
\end{prv}

\begin{rmk}
	\begin{align*}
		V: \R^3 &\longrightarrow \R \\
		(x,y,z) &\longmapsto -mgz
	\end{align*}
	l'énerge potentielle de pesenteur

	On a donc \[
		\nabla V(x,y,z) = \left( \frac{\partial V}{\partial x}, \frac{\partial V}{\partial y}, \frac{\partial V}{\partial z} \right) = (0, 0, -mg) = \vec{P}.
	\]
\end{rmk}

\begin{lem}
	Soit $f : U \to \R$ de classe $\mathcal{C}^1$, $\gamma : \begin{array}{rcl}
		I &\longrightarrow& U \\
		t &\longmapsto& \big(x(t), y(t)\big)
	\end{array}$ où $x$ et $y$ sont dérivables.

	On pose \[
		\forall t \in I, \gamma'(t) = \big(x'(t), y'(t)\big).
	\] Alors $f \circ \gamma : I \to \R$ est dérivable et
	\begin{align*}
		\forall t \in I, (f \circ \gamma)'(t) &= \mathrm{d}f\big(\gamma'(t)\big) \big(\gamma(t)\big)\\
		&= \left<\gamma'(t)  \mid \nabla f\big(\gamma(t)\big)  \right> \\
		&= x'(t) \frac{\partial f}{\partial x}\big(x(t), y(t)\big) + y'(t) \frac{\partial f}{\partial y}\big(x(t),y(t)\big). \\
	\end{align*}
\end{lem}

\begin{prv}
	On fixe $t \in I$.

	\begin{align*}
		\forall h \neq 0, \frac{f \circ \gamma(t + h) - f \circ \gamma(t)}{h}
		&= \frac{1}{h}\big(f(\gamma(t)) + h\gamma'(t) + \po_{h\to 0}(h) - f(\gamma(t))\big) \\
		&= \frac{1}{h}\bigg(\cancel{f(\gamma(t))} + \left<h\gamma'(t) \mid \nabla f(\gamma(t)) \right> + \po_{h\to 0}(\|h\gamma'(t)\|) - \cancel{f(\gamma(t))}\bigg)\\
		&= \left<\gamma'(t) \mid \nabla f(\gamma(t)) \right> + \po_{h\to 0}(1) \\
		&\tendsto{h\to 0} \left<\gamma'(t)  \mid \nabla f(\gamma(t)) \right>
	\end{align*}
\end{prv}

\begin{defn}
	Soit $f : U \to \R$ de classe $\mathcal{C}^1$ et $(a,b) \in U$. On dit que $(a,b)$ est un \underline{point critique} de $f$ si $\nabla f(a,b) = 0$ i.e. $\frac{\partial f}{\partial x}(a,b) = \frac{\partial f}{\partial y}(a,b) = 0$.

	Dans ce cas, $f(a,b)$ est appelé \underline{valeur critique} de $f$.
\end{defn}

\begin{prop}~\\
	\begin{minipage}{\linewidth}
		\begin{wrapfigure}{r}{3cm}
			\centering
			\vspace{-1cm}
			\begin{asy}
				import solids;
				import graph;
				size(3cm);

				settings.render = 0;
				settings.prc = false;

				path3 par = graph(
					new real(real x) { return x; },
					new real(real x) { return 0; },
					new real(real x) { return -x^2; },
					0,3);
				revolution r = revolution(par, axis=Z);

				draw(r,1,longitudinalpen=nullpen);
				draw(r.silhouette());

				dot("$(a,b)$", O, red, align=N);
				real s = sqrt(2.5);
				path3 g=(s,0,-2.5)..(0,s,-2.5)..(-s,0,-2.5)..(0,-s,-2.5)..cycle;
				draw(g, deepcyan);
			\end{asy}
		\end{wrapfigure}
		Soit $f: U \to \R$ de classe $\mathcal{C}^1$ et $(a,b) \in U$ tel que \[
			\exists r > 0, \forall (x,y) \in B_{(a,b)}(r), f(x,y) \le f(a,b)
		\] Alors $\nabla f(a,b) = (0,0)$.
	\end{minipage}
\end{prop}

\begin{prv}
	Soit $g: x \mapsto f(x,b)$. $g(a)$ est un maximum local de $g$ donc $g'(a) = 0$.

	Or, $g'(a) = \frac{\partial f}{\partial x}(a,b)$

	donc $\frac{\partial f}{\partial x}(a,b) = 0$.

	Soit $h : y \mapsto f(a,y)$. On a de même $h'(b) = 0$.

	Or, $h'(b) = \frac{\partial f}{\partial y}(a,b)$.

	Donc, $\nabla f(a,b) = (0,0)$.
\end{prv}

\begin{rmk}
	Un minimum local est aussi une valeur critique.
\end{rmk}

\begin{figure}[H]
	\centering
	\begin{subfigure}{3cm}
		\centering
		\begin{asy}
			import solids;
			import graph;
			size(3cm);

			settings.render = 0;
			settings.prc = false;

			path3 par = graph(
				new real(real x) { return x; },
				new real(real x) { return 0; },
				new real(real x) { return -x^2; },
				0,3);
			revolution r = revolution(par, axis=Z);

			draw(r,1,longitudinalpen=nullpen);
			draw(r.silhouette());

			dot(O, red);
		\end{asy}
		\caption{Maximum local}
	\end{subfigure}
	\begin{subfigure}{3cm}
		\centering
		\begin{asy}
			import solids;
			import graph;
			size(3cm);

			settings.render = 0;
			settings.prc = false;

			path3 par = graph(
				new real(real x) { return x; },
				new real(real x) { return 0; },
				new real(real x) { return x^2; },
				0,3);
			revolution r = revolution(par, axis=Z);

			draw(r,1,longitudinalpen=nullpen);
			draw(r.silhouette());

			dot(O, red);
		\end{asy}
		\caption{Minimum local}
	\end{subfigure}
	\begin{subfigure}{3cm}
		\centering
		\begin{asy}
			import solids;
			import graph;
			size(3cm);

			settings.render = 0;
			settings.prc = false;
			currentprojection = obliqueZ;

			draw(graph(
				new real(real x) { return x; },
				new real(real x) { return -x^2 / 3; },
				new real(real x) { return 3; },
				-3, 3
			));

			draw(graph(
				new real(real x) { return x; },
				new real(real x) { return -x^2 / 3; },
				new real(real x) { return -3; },
				-3, 3
			));

			draw(graph(
				new real(real x) { return x; },
				new real(real x) { return -x^2 / 3 - 1; },
				new real(real x) { return 0; },
				-3, 3
			));

			draw(graph(
				new real(real x) { return 0; },
				new real(real x) { return x^2 / 9 - 1; },
				new real(real x) { return x; },
				-3, 3
			));

			draw(graph(
				new real(real x) { return -3; },
				new real(real x) { return x^2 / 9 - 4; },
				new real(real x) { return x; },
				-3, 3
			));

			draw(graph(
				new real(real x) { return 3; },
				new real(real x) { return x^2 / 9 - 4; },
				new real(real x) { return x; },
				-3, 3
			));

			dot((0,-1,0), red);
		\end{asy}
		\caption{Point de selle / Point col}
	\end{subfigure}
\end{figure}

\begin{exm}
	On revient à l'exemple donné en introduction : 
	\begin{align*}
		f: \left( \R^*_+ \right)^2 &\longrightarrow \R \\
		(x,y) &\longmapsto 2\left( xy + \frac{1}{x} + \frac{1}{y} \right).
	\end{align*}

	$\left( \R^+_* \right)^2$ est un ouvert de $\R^2$. Soit $(x,y) \in \left( \R^+_* \right)^2$.
	
	On a \[
		\begin{cases}
			\frac{\partial f}{\partial x}(x,y) = 2\left( y - \frac{1}{x^2} \right),\\
			\frac{\partial f}{\partial y}(x,y) = 2\left( x - \frac{1}{y^2} \right).
		\end{cases}
	\]

	\begin{align*}
		&\frac{\partial f}{\partial x}(x,y) = \frac{\partial f}{\partial y}(x,y) = 0\\
		\iff& \begin{cases}
			y = \frac{1}{x^2}\\
			x = \frac{1}{y^2}
		\end{cases}\\
		\iff& \begin{cases}
			y = \frac{1}{x^2}\\
			x = x^4
		\end{cases}\\
		\iff& \begin{cases}
			x = 1\\
			y = 1
		\end{cases}
	\end{align*}

	On vérivie que $f$ présente en effet un minium local en $(1,1)$. \[
		f(1,1) = 6
	\] On fixe $y \in \R^+_*$ et \[
		g : x \mapsto 2\left( xy + \frac{1}{x} + \frac{1}{y} \right).
	\] Donc \[
		\forall x \in \R^+_*, g'(x) = 2\left( y - \frac{1}{x^2} \right).
	\]
	\begin{center}
		\begin{tikzpicture}
			\tkzTabInit{$x$/1,$g'(x)$/1,$g$/2.3}{$0$, $\frac{1}{\sqrt{y}}$, $+\infty$}
			\tkzTabLine{,-,z,+,}
			\tkzTabVar{+/{}, -/$2\left( 2\sqrt{y} +\frac{1}{y} \right)$, +/{}}
		\end{tikzpicture}
	\end{center}
	
	Ainsi, \[
		\forall x \in \R^+_*, \forall y \in \R^+_*, f(x,y) \ge 2\left( 2\sqrt{y} + \frac{1}{y} \right)
	\] Soit $h : y \mapsto 2\sqrt{y} + \frac{1}{y}$. On a \[
		\forall y > 0, h'(y) = \frac{1}{\sqrt{y}} - \frac{1}{y^2} = \frac{y\sqrt{y} - 1}{y^2} = \frac{y^{\frac{3}{2}} - 1}{y^2}
	\]

	\begin{center}
		\begin{tikzpicture}
			\tkzTabInit{$y$/0.7,$h'(y)$/0.7,$h$/1.4}{$0$, $1$, $+\infty$}
			\tkzTabLine{,-,z,+,}
			\tkzTabVar{+/{}, -/$3$, +/{}}
		\end{tikzpicture}
	\end{center}

	Donc, \[
		\forall x,y > 0, f(x,y) \ge 2\times 3 = 6 = f(1,1).
	\]
\end{exm}

\begin{prop}
	[règle de la chaîne]

	Soit $f : \begin{array}{rcl}
		U &\longrightarrow& \R^2 \\
		(x,y) &\longmapsto& f(x,y)
	\end{array}$ de classe $\mathcal{C}^1$ et $U, V$ deux ouverts de $\R^2$.

	Soit $\varphi : \begin{array}{rcl}
		V &\longrightarrow& U \\
		(u,v) &\longmapsto& \varphi(u,v) = \big(x(u,v), y(u,v)\big)
	\end{array}$.

	On suppose que $x$ et $y$ sont de classe $\mathcal{C}^1$ sur $V$.

	Alors,  $f \circ \varphi : \begin{array}{rcl}
		V &\longrightarrow& \R \\
		(u,v) &\longmapsto& f\big(\varphi(u,v)\big)
	\end{array}$ est de classe $\mathcal{C}^1$ et
	\begin{align*}
		\forall (u_0, v_0) \in V, \frac{\partial (f \circ \varphi)}{\partial u}(u_0, v_0)
		&= \frac{\partial f}{\partial x}\big(\varphi(u_0, v_0)\big) \times \frac{\partial x}{\partial u}(u_0, v_0)\\
		&+ \frac{\partial f}{\partial y}\big(\varphi(u_0,v_0)\big) \frac{\partial y}{\partial u}(u_0,v_0)
	\end{align*}
	\begin{align*}
		\forall (u_0, v_0) \in V, \frac{\partial (f \circ \varphi)}{\partial v}(u_0, v_0)
		&= \frac{\partial f}{\partial x}\big(\varphi(u_0, v_0)\big) \times \frac{\partial x}{\partial v}(u_0, v_0)\\
		&+ \frac{\partial f}{\partial y}\big(\varphi(u_0,v_0)\big) \frac{\partial y}{\partial v}(u_0,v_0)
	\end{align*}
\end{prop}

\begin{exm}
	[changement de coordonnées polaires]
	On pose \begin{align*}
		\varphi: \R^+_* \times ]0,2\pi[ &\longrightarrow \R^2\setminus \left( R^+_* \times \{0\} \right) \\
		(r, \theta) &\longmapsto (r \cos \theta, r \sin\theta),
	\end{align*}
	\begin{align*}
		f: \R^2\setminus \left( R^+_* \times \{0\} \right) &\longrightarrow \R \\
		(x,y) &\longmapsto f(x,y),
	\end{align*}
	\begin{align*}
		g: \overbrace{\R^+_* \times ]0, 2\pi[}^{=V} &\longrightarrow \R \\
		(r, \theta) &\longmapsto f(r\cos\theta, r\sin\theta).
	\end{align*}

	\begin{align*}
		\forall (r_0,\theta_0) \in V,&\\[5mm]
		\frac{\partial g}{\partial r}(r_0, \theta_0) &= \frac{\partial f}{\partial x}(r_0\cos\theta_0, r_0\sin\theta_0)\cos\theta_0\\
		&+ \frac{\partial f}{\partial y}(r_0 \cos\theta_0, r_0\sin\theta_0)\sin\theta_0\\
		&= 2r_0\cos^2\theta_0 + 2r_0\sin^2(\theta_0) \\
		&= 2r_0 \\[5mm]
		\frac{\partial g}{\partial \theta}(r_0, \theta_0) &= \frac{\partial f}{\partial x}(r_0\cos\theta_0, r_0\sin\theta_0)r_0\sin\theta_0\\
		&+ \frac{\partial f}{\partial y}(r_0 \cos\theta_0, r_0\sin\theta_0)r_0\cos\theta_0\\
		&= -2{r_0}^2\cos(\theta_0)\sin(\theta_0) + 2{r_0}^2 \sin(\theta_0)\cos(\theta_0)\\
		&= 0 \\
	\end{align*}

	Donc, \[
		g(r, \theta) = r^2.
	\]
\end{exm}

\begin{exm}
	Résoudre \[
		\begin{cases}
			\frac{\partial f}{\partial x} = \frac{x}{x^2+y^2},\\
			\frac{\partial f}{\partial y} = \frac{y}{x^2+y^2}.\\
		\end{cases}
	\]

	On pose $g: (r, \theta) \mapsto f(r \cos\theta, r \sin\theta)$.

	\begin{align*}
		&\frac{\partial g}{\partial r} = \frac{1}{r}\cos^2\theta + \frac{1}{r}\sin^2\theta = \frac{1}{r},\\
		&\frac{\partial g}{\partial \theta} = -\cos(\theta) \sin(\theta) + \sin(\theta)\cos(\theta) = 0.
	\end{align*}

	Donc, \[
		\exists C \in \R, g: (r, \theta) \mapsto \ln r + C
	\] d'où,
	\begin{align*}
		\forall (x,y) \in \R^2 \setminus \{(0,0)\}, f(x,y) &= \ln\left(\sqrt{x^2 + y^2} \right)  + C\\
		&= \frac{1}{2}\ln(x^2 + y^2) + C. \\
	\end{align*}
\end{exm}

\begin{rmk}
	Soit $\mathcal{B} = (e_1, e_2)$ la base canonique de $\R^2$, $f: U \to \R$ de classe $\mathcal{C}^1$ avec $U$ un ouvert de $\R^2$.

	Soit $(x,y) \in U$.

	\begin{align*}
		\Mat_{\mathcal{B}}\big(\nabla f(x,y)\big) = \begin{pmatrix}
			\frac{\partial f}{\partial x}(x,y)\\[2mm]
			\frac{\partial f}{\partial y}(x,y)
		\end{pmatrix}
	\end{align*}

	Soit  \begin{align*}
		\varphi: V &\longrightarrow U \\
		(u,v) &\longmapsto \big(x(u,v), y(u,v)\big) 
	\end{align*} avec $x,y$ de classe $\mathcal{C}^1$. Soit $g = f \circ \varphi$.
	\begin{align*}
		\Mat_{\mathcal{B}}\big(\nabla g(u,v)\big)
		&= \begin{pmatrix}
			\frac{\partial g}{\partial u}(u,v) \\[2mm]
			\frac{\partial g}{\partial v}(u,v)
		\end{pmatrix} \\
		&= \begin{pmatrix}
			\frac{\partial x}{\partial u}(u,v) \frac{\partial f}{\partial x}(x,y)
			+ \frac{\partial y}{\partial u}(u,v)\frac{\partial f}{\partial y}(x,y)\\[3mm]
			\frac{\partial x}{\partial v}(u,v) \frac{\partial f}{\partial x}(x,y)
			+ \frac{\partial y}{\partial v}(u,v) \frac{\partial f}{\partial y}(x,y)
		\end{pmatrix}  \\
		&= \underbrace{\begin{pmatrix}
				\frac{\partial x}{\partial u}(u,v)& \frac{\partial y}{\partial u}(u,v)\\[3mm]
				\frac{\partial x}{\partial v}(u,v)& \frac{\partial y}{\partial v}(u,v)
		\end{pmatrix}}_{J(u,v)} \begin{pmatrix}
			\frac{\partial f}{\partial x}(x,y)\\[3mm]
			\frac{\partial f}{\partial y}(x,y)
		\end{pmatrix} \\
		&= J(u,v) \Mat_{\mathcal{B}}\big(\nabla f(x,y)\big) \\
	\end{align*}
	où $J(u,v) = 
	\begin{pNiceArray}{c:c}
		\Mat_{\mathcal{B}}\big(\nabla x(u,v)\big) & \Mat_{\mathcal{B}}\big(\nabla y(u,v)\big)
	\end{pNiceArray}$.

	On dit que $J(u,v)$ est \underline{la jacobienne} de $\varphi$ en $(u,v)$.
	L'application linéaire canoniquement associée à $J(u,v)$ est la \underline{différentielle de $\varphi$} en $(u,v)$ noté $\mathrm{d}\varphi(u,v)$.

	On a $\mathrm{d}\varphi(u,v) \in \mathcal{L}(R^2)$ et $\Mat_{\mathcal{B}}\big(\mathrm{d}\varphi(u,v)\big) = J(u,v)$.

	Par exemple, la jacobienne du changement de coordonnées polaires est \[
		J = \begin{pmatrix}
			\frac{\partial x}{\partial r} & \frac{\partial y}{\partial r}\\[3mm]
			\frac{\partial x}{\partial \theta} & \frac{\partial y}{\partial \theta}
		\end{pmatrix}
		= \begin{pmatrix}
			\cos\theta&\sin\theta\\
			-r\sin\theta&r\cos\theta
		\end{pmatrix}.
	\]
	$\underbrace{\det(J)}_{\text{le jacobien}} = r\cos^2\theta + r\sin^2\theta = r$

	Dans une intégrale double, si $(x,y) = \varphi(u,v)$, alors $\mathrm{d}x\mathrm{d}y = \det(J)\mathrm{d}u\mathrm{d}v$.

	Ici, \[
		\mathrm{d}x\ \mathrm{d}y = r\ \mathrm{d}r\ \mathrm{d}\theta.
	\]
\end{rmk}

\begin{prv}
	On pose $(x_0, y_0) = \varphi(u_0, v_0)$. Pour tout $(h,k) \in \R^2$ tels que $(u_0 + h, v_0 + k) \in V$, en posant $g = f  \circ \varphi$.

	\begin{align*}
		g(u_0 + h, v_0 + h) &= f\big(x(u_0 + h, v_0 + k), y(u_0 + h, v_0 + k)\big) \\
		&= f\left(
			x(u_0,v_0) + h \frac{\partial x}{\partial u}(u_0,v_0) + k \frac{\partial x}{\partial v}(u_0, v_0) + \po\big(\|(h,k)\|\big), \right.\\
		&\phantom{ = f\bigg(\bigg.}\left. y(u_0, v_0) + h \frac{\partial y}{\partial u}(u_0, v_0) + k \frac{\partial y}{\partial v}(u_0, v_0) + \po\big(\|(h,k)\|\big)
		\right)  \\
		&= f(x_0,y_0) \\
		&~+ \left( h \frac{\partial x}{\partial u}(u_0,v_0) + k \frac{\partial x}{\partial v}(u_0, v_0) + \po(\|(h,k)\|) \right) \frac{\partial f}{\partial x}(x_0,y_0)\\
		&~+ \left( h \frac{\partial y}{\partial u}(u_0, v_0) + k\frac{\partial y}{\partial v}(u_0, v_0) + \po(\|(h,k)\|) \right) \frac{\partial f}{\partial y}(x_0, y_0)\\
		&~+ \po(\|(h,k)\|)\\
		&= f(x_0, y_0) \\
		&~+ h \left( \frac{\partial x}{\partial u}(u_0, v_0) \frac{\partial f}{\partial x}(x_0, y_0) + \frac{\partial y}{\partial u}(u_0, v_0) \frac{\partial f}{\partial y}(x_0, y_0) \right)  \\
		&~+ k\left( \frac{\partial x}{\partial v}(u_0, v_0) \frac{\partial f}{\partial x}(x_0, y_0) + \frac{\partial y}{\partial v}(u_0, v_0) \frac{\partial f}{\partial y}(x_0, y_0) \right) 
		&~+ \po(\|(h,k)\|)\\
		&= g(u_0, v_0) + h \frac{\partial g}{\partial u}(u_0, v_0) + k \frac{\partial g}{\partial v}(u_0, v_0) + \po(\|(h,k)\|) \\
	\end{align*}

	Par identification,
	\[
		\frac{\partial g}{\partial u}(u_0, v_0) = \frac{\partial x}{\partial u}(u_0, v_0) \frac{\partial f}{\partial x}(x_0, y_0) + \frac{\partial y}{\partial u}(u_0, v_0) \frac{\partial f}{\partial y}(x_0,y_0)
	\] et \[
		\frac{\partial g}{\partial v}(u_0, v_0) = \frac{\partial x}{\partial v}(u_0,v_0) \frac{\partial f}{\partial x}(x_0, y_0) + \frac{\partial y}{\partial v}(u_0, v_0) \frac{\partial f}{\partial y}(x_0, y_0).
	\] 
\end{prv}

\begin{exm}
	[Régression linéaire]~\\
	\begin{figure}[H]
		\centering
		\begin{asy}
			import graph;
			axes(EndArrow);
			size(5cm);

			real f(real x) { return x + 0.5; }

			real k = 35 / (7 - 0.5);

			for(int i = 0; i < 35; ++i) {
				real mag = exp(sin(100 * pi/exp(1) * i)) * 0.8 + exp(cos(i*40)/3);
				real eps = mag * cos(10 * exp(1)/pi * i) / 3;
				dot((i/k,f(i/k) + eps));
			}

			draw(graph(f, -1, 7), orange);
		\end{asy}
	\end{figure}
	\[
		y = a x + b
	\] 
	On fixe $(a,b) \in \R^2$. \[
		\varepsilon(a,b) = \sum_{i=1}^n\big( y_i - (ax_i + b) \big)^2
	\] l'erreur totale.

	On veut minimiser $\varepsilon(a,b)$. On a 
	\[
		\forall (a,b) \in \R^2,
		\begin{cases}
			\frac{\partial \varepsilon}{\partial a}(a,b) = -2\sum_{i=1}^{n}(y_i - ax_i - b)x_i,\\
			\frac{\partial \varepsilon}{\partial b}(a,b) = -2\sum_{i=1}^{n}(y_i - ax_i - b).
		\end{cases}
	\]

	Donc,
	\begin{align*}
		(a,b) \text{ point critique de } \varepsilon \iff& \begin{cases}
			a \sum_{i=1}^n {x_i}^2 + b\sum_{i=1}^{n}x_i = \sum_{i=1}^{n} y_ix_i\\
			a\sum_{i=1}^{n}x_i + nb = \sum_{i=1}^ny_i
		\end{cases}\\
		\iff& \begin{cases}
			a \left( \frac{1}{n}\sum_{i=1}^n {x_i}^2 - \overline{x}^2\right) = \overline{y} - \overline{x} \overline{y}\\
			b = \frac{1}{n}\sum_{i=1}^ny_i - \frac{a}{n}\sum_{i=1}^nx_i = \frac{1}{n}\sum_{i=1}^n x_i y_i - \overline{x} \overline{y}
		\end{cases}\\
		&\text{ où } \overline{x} = \frac{1}{n} \sum_{i=1}^n x_i,~\overline{y} = \frac{1}{n}\sum_{i=1}^n y_i\\
		\iff& \begin{cases}
			a = \frac{\Cov(x,y)}{V(x)}\\
			b = \overline{y} - a\overline{x}
		\end{cases}
	\end{align*}

	Coefficient de corrélation: $\frac{\Cov(x,y)}{\sigma_x \sigma_y} \in [-1, 1]$
\end{exm}












		\part{Corps}

\begin{exm}[Problème]
	\begin{itemize}
		\item 
			avec $A = \Z / 9 \Z$, résoudre $\overline{x}^2 = \overline{0}$ \\
			\begin{center}
				\begin{tabular}{|c|c|c|c|c|c|c|c|c|c|c|}
					\hline
					$\overline{x}$&$\overline{0}$& $\overline{1}$ &$\overline{2}$&$\overline{3}$ &$\overline{4}$ &$\overline{5}$ &$\overline{6}$ &$\overline{7}$ &$\overline{8}$& $\overline{9}$ \\
					\hline
					$\overline{x}^2$&$\overline{0}$ &$\overline{1}$ &$\overline{4}$ &$\overline{0}$ &$\overline{7}$ &$7$ &$\overline{0}$ &$\overline{4}$ &$\overline{1}$&$\overline{0}$\\
					\hline
				\end{tabular}
			\end{center}
			On a trouvé 3 solutions: $\overline{0}$, $\overline{3}$, $\overline{6}$.
		\item $\Z / 8\Z$
			\begin{center}
				\begin{tabular}{|c|c|c|c|c|c|c|c|c|}
					\hline
					$\overline{x}$& $\overline{0}$& $\overline{1}$& $\overline{2}$& $\overline{3}$& $\overline{4}$& $\overline{5}$& $\overline{6}$& $\overline{7}$\\
					\hline
					$\overline{x^2}$& $\overline{0}$& $\overline{1}$& $\overline{4}$& $\overline{1}$& $\overline{0}$& $\overline{1}$& $\overline{4}$& $\overline{1}$\\
					\hline
				\end{tabular}
			\end{center}
			$\overline{x}^2=7$ a 4 solutions: $\overline{1}, \overline{7}, \overline{3},\text{ et } \overline{5}$
		\item $A = \mathbbm{H} = \{a + bi + cj + dk  \mid  (a,b,c,d) \in \R^4\}$ \\
			$i^2 = j^2 = k^2 = -1$ 
			\begin{align*}
				\begin{array}{c c c}
					ij = k & jk = i & ji = j\\
					ji = -k & kj = -i & ik = -j
				\end{array}
			\end{align*}
			Dans cet anneau, $-1$ a 6 racines!
	\end{itemize}
\end{exm}

\begin{defn}
	Soit $(\mathbbm{K}, +, \times)$ un ensemble muni de deux lois de composition internes. On dit que c'est un \underline{corps} si
	 \begin{enumerate}
		\item $(\mathbbm{K}, \times)$ est un groupe abélien
		\item $(\mathbbm{K}, \times)$ est un monoïde commutatif
		\item $\forall x \in \mathbbm{K}\setminus \{0_\mathbbm{K}\}, \exists y \in \mathbbm{K}, xy = 1_\mathbbm{K}$
		\item $0_\mathbbm{K} \neq  1_\mathbbm{K}$
	\end{enumerate}
	\index{corps}
\end{defn}

\begin{exm}
	\begin{itemize}
		\item $(\C, +, \times)$ est un corps
		\item $(\R, +, \times)$ est un corps
		\item $(\Q, +, \times)$ est un corps
		\item $(\Z, +, \times)$ n'est pas un corps
	\end{itemize}
\end{exm}

\begin{prop}
	$(\Z / n\Z, +, \times)$ est un corps si et seulement si $n$ est premier.
\end{prop}

\begin{prv}
	\[
		\left( \Z / n\Z \right)^\times = \left\{ \overline{k}  \mid k \wedge n = 1 \right\}
	\] 
\end{prv}


\begin{prop}
	Tout corps est un anneau intègre.
\end{prop}

\begin{prv}
	Soit $(\mathbbm{K}, +, \times)$ un corps. Soient $(a,b) \in \mathbbm{K}^2$ tel que $a \times b = 0_\mathbbm{K}$.\\
	On suppose $a \neq  0_\mathbbm{K}$. Alors, $a$ est inversible et donc \[
		b = a^{-1} \times a \times b = a^{-1} \times 0_\mathbbm{K} = 0_\mathbbm{K}
	\] 
\end{prv}

\begin{exm}
	Soit $(\mathbbm{K},+,\times)$ un corps.\\
	Résoudre \[
		\begin{cases}
			x^2 = 1_\mathbbm{K}\\
			x \in \mathbbm{K}
		\end{cases}
	\]

	\begin{align*}
		x^2 = 1_\mathbbm{K} &\iff x^2 - 1_\mathbbm{K} = 0_\mathbbm{K}\\
		&\iff (x - 1_\mathbbm{K})(x+1_\mathbbm{K}) = 0_\mathbbm{K}\\
		&\iff x - 1_\mathbbm{K} = 0_\mathbbm{K} \text{ ou } x + 1_\mathbbm{K} = 0_\mathbbm{K}\\
		&\iff x = 1_\mathbbm{K} \text{ ou } x = -1_\mathbbm{K}
	\end{align*}

	Il y a au plus 2 solutions.
\end{exm}

\begin{prop}
	Soit $(\mathbbm{K},+,\times )$ un corps et $P$ un polynôme à coefficients dans $\mathbbm{K}$ de degré $n$. Alors, l'équation $P(x) = 0_{\mathbbm{K}}$ a au plus $n$ solutions dans $\mathbbm{K}$ 
	\qed
\end{prop}

\begin{crlr}[(Théorème de Wilson)]
	voir exercice 16 du TD 12
\end{crlr}


\begin{defn}
	Soit $(\mathbbm{K}, +, \times)$ un corps et $L\subset \mathbbm{K}$.\\
	On dit que $L$ est un \underline{sous corps} de $\mathbbm{K}$ si
	\begin{enumerate}
		\item $L$ est un anneau de $(\mathbbm{K}, +, \times)$ non nul
		\item $\forall x \in L\setminus \{0_\mathbbm{K}\}, x^{-1} \in L$ 
	\end{enumerate}
	\vspace{2mm}
	en d'autres termes si
	\begin{enumerate}
		\item $\forall (x,y) \in L^2, x - y \in L$
		\item $\forall (x,y) \in L^2, x \times y^{-1} \in L$
	\end{enumerate}
	\vspace{5mm}
	On dit aussi que $\mathbbm{K}$ est une \underline{extension} de $L$.
	\index{sous corps}
	\index{extension}
\end{defn}

\begin{prop}
	Tout sous corps est un corps. \qed
\end{prop}

\begin{defn}
	Soient $(\mathbbm{K}_1,+,\times )$ et $(\mathbbm{K}_2,+, \times)$ deux corps et $f: \mathbbm{K}_1 \to \mathbbm{K}_2$.\\
	On dit que $f$ est un \underline{morphisme de corps} si $f$ est un morphisme d'anneaux.\\
	i.e. si
	\[
		\begin{cases}
			\forall (x,y) \in {\mathbbm{K}_1}^2,& f(x+y) = f(x) + f(y)\\
			\forall (x,y) \in {\mathbbm{K}_1}^2,& f(x \times y) = f(x) \times f(y)\\
		\end{cases}
	\] 
	\index{homomorphisme (de corps)}
	\index{morphisme (de corps)}
\end{defn}

\begin{prop}
	Tout morphisme de corps est injectif.
\end{prop}

\begin{prv}
	Soit $f: \mathbbm{K}_1 \to \mathbbm{K}_2$ un morphisme de corps.\\
	\begin{itemize}
		\item $\Ker(f)$ est un sous groupe de $(\mathbbm{K}_1, +)$ 
		\item Soit $x \in \Ker(f)$ et $y \in \mathbbm{K}_1$ \[
				f(x \times y) = f(x) \times f(y) = 0_{\mathbbm{K}_2} \times f(y) = 0_{\mathbbm{K}_2}
			\]
		\item Soit $x \in \Ker(f) \setminus \{0_{\mathbbm{K}_1}\}$.\\
			Alors, $x$ est inversible.\\
			\begin{align*}
				\begin{rcases*}
					x \in \Ker(f)\\
					x^{-1} \in \mathbbm{K}_1
				\end{rcases*}& \text{ donc } x \times x ^{-1} \in \Ker(f)\\
				&\text{ donc } 1_{\mathbbm{K}_1} \in \Ker(f)\\
				&\text{ donc } f(1_{\mathbbm{K}_1}) = 0_{\mathbbm{K}_2}
			\end{align*}
			Or, $f(1_{\mathbbm{K}_1}) = 1_{\mathbbm{K}_2} \neq 0_{\mathbbm{K}_2}$
	\end{itemize}
	Donc, $\Ker(f) = \{0_{\mathbbm{K}_1}\}$ donc $f$ est injective.
\end{prv}

\begin{exm}
	$\begin{array}{cc}
		\C &\longrightarrow \C\\
		z &\longmapsto \overline{z}\\
	\end{array}$ est un morphisme de corps
\end{exm}



		\part{Opérations sur les séries}

\begin{prop}
	L'ensemble $E = \{u \in \C^\N  \mid \Sigma u_n \text{ converge}\}$ est un sous-espace vectoriel de $\C^\N$ et \begin{align*}
		S: E &\longrightarrow \C \\
		u &\longmapsto \sum_{n=0}^{+\infty} u_n
	\end{align*} est une forme linéaire.
	\qed
\end{prop}

\begin{rmk}
	La somme d'une série convergente et d'une série divergente diverge.
	Le produit d'une série divergente par un scalaire non nul diverge.
\end{rmk}

	}

	{
		\chap[17]{Dimension finie}
		\renewcommand{\cwd}{../chap17}
		\begin{defn}
	Soit $E$ un $\mathbbm{K}$-espace vectoriel. On dit que $E$ est de \underline{dimension finie} si $E$ a au moins une famille génératrice finie. On dit que $E$ est de \underline{dimension infinie} sinon.
	\index{dimension finie (espace vectoriel)}
	\index{dimension infinie (espace vectoriel)}
\end{defn}

\begin{thm}
	[Théorème de la base extraite]
	Soit $E$ un $\mathbbm{K}$-espace vectoriel non nul de dimension finie. Soit $\mathcal{G}$ une famille génératrice finie de $E$. Alors, il existe une base $\mathcal{B}$ de $\mathcal{E}$ telle que $\mathcal{B} \subset \mathcal{G}$.
\end{thm}

\begin{prv}
	[par récurrence sur $\#G = \Card(G)$]
	\begin{itemize}
		\item Soit $E$ un $\mathbbm{K}$-espace vectoriel non nul engendré par $\mathcal{G} = (u)$.\\
			Si $u = 0_E$, alors $E = \{0_E\}$: une contradiction $\lightning$ \\
			Donc $u \neq 0_E$ donc $(u)$ est libre. En effet, \[
				\forall \lambda \in \mathbbm{K}, \lambda u = 0_E \implies \lambda = 0_\mathbbm{K}
			\] Donc $\mathcal{G}$ est une base de $E$.\\
		\item Soit $n \in \N_*$. Soit $E$ un $\mathbbm{K}$-espace vectoriel. On suppose que si $E$ a une famille génératrice constituée de $n$ vecteurs, alors on peut extraire de cette famille une base de $E$.\\
			Soit $\mathcal{G}$ une famille génératrice de $E$ avec $n+1$ vecteurs.\\
			Si $\mathcal{G}$ est libre, alors $\mathcal{G}$ est une base de $E$. \\
			Si $\mathcal{G}$ n'est pas libre, alors il existe $u \in \mathcal{G}$ tel que $u \in \Vect(\mathcal{G}\setminus \{u\})$ \\
			Donc $\mathcal{G}\setminus \{u\}$ engendre $E$. Or, $\mathcal{G}\setminus \{u\}$ possède $n$ vecteurs. D'après l'hypothèse de récurrence, il existe une base $\mathcal{B}$ de $E$ telle que \[
				\mathcal{B} \subset \mathcal{G} \setminus \{u\} \subset \mathcal{G}
			\] 
	\end{itemize}
\end{prv}

\begin{crlr}
	Tout espace de dimension finie a une base.
	\qed
\end{crlr}

\begin{thm}
	[Théorème de la base incomplète]
	Soit $E$ un $\mathbbm{K}$-espace vectoriel de dimension finie, $\mathcal{G}$ une famille génératrice finie de $E$. $\mathcal{L}$ une famille libre de $E$. Alors, il existe une base $\mathcal{B}$ de $E$ telle que \[
		\mathcal{L} \subset \mathcal{B} \text{ et } \mathcal{B}\setminus \mathcal{L} \subset \mathcal{G}
	\] 
\end{thm}

\begin{prv}
	[par récurrence sur $\#(\mathcal{G}\setminus\mathcal{L})$]
	\begin{itemize}
		\item Avec les notations précédentes, on suppose que $\mathcal{G}\setminus\mathcal{L} \neq \O$ \[
				\forall u \in \mathcal{G}, u \in \mathcal{L}
			\] Donc $\mathcal{G} \subset \mathcal{L}$ donc $\mathcal{L}$ est génératrice donc $\mathcal{L}$ est une base de $E$. On pose $\mathcal{B} = \mathcal{L}$ et alors \[
				\mathcal{L} \subset  \mathcal{B} \text{ et } \mathcal{B}\setminus\mathcal{L} = \O \subset  \mathcal{G}
			\] 
		\item Soit $n \in \N$. On suppose que si $\mathcal{G}$ est génératrice et $\mathcal{L}$ libre avec $\#(\mathcal{G}\setminus\mathcal{L}) = n$ alors il existe une base $\mathcal{B}$ de $E$ telle que \[
			\mathcal{L}\subset \mathcal{B} \text{ et } \mathcal{B}\setminus\mathcal{L}\subset \mathcal{G}
		\] Soient à présent $\mathcal{G}$ une famille génératrice de $E$ et $\mathcal{L}$ une famille libre de $E$ telles que $\#(\mathcal{G}\setminus\mathcal{L}) = n+1 > 0$\\
		Si $\mathcal{L}$ engendre $E$, alors $\mathcal{L}$ est une base de $E$. On pose $\mathcal{B} = \mathcal{L}$ et on a bien \[
			\mathcal{L} \subset  \mathcal{B} \text{ et } \mathcal{B} \setminus \mathcal{L} = \O \subset  \mathcal{G}
		\] On suppose que $\mathcal{L}$ n'engendre pas $E$. Il existe $u \in \mathcal{G}$ tel que $u \not\in \Vec(\mathcal{L})$ (car sinon, $\mathcal{G} \subset \Vect(\mathcal{L})$ et donc $\underbrace{\Vect(\mathcal{G})}_{= E} \subset  \underbrace{\Vect(\mathcal{L})}_{ \subset E}$\\
		Donc $\mathcal{L} \cup \{u\} $ est libre. On pose $\mathcal{L}' = \mathcal{L} \cup \{u\} $ \[
			\mathcal{G}\setminus \mathcal{L}' = \mathcal{G}\setminus (\mathcal{L} \cup \{u\}) = (\mathcal{G}\setminus\mathcal{L})\setminus \{u\} 
		\] donc $\#(\mathcal{G}\setminus\mathcal{L}') = n+1 -1 = n$\\
		D'après l'hypothèse de récurrence, il existe $\mathcal{B}$ une base de $E$ telle que \[
			\mathcal{L} \subset  \mathcal{L}' \subset \mathcal{B} \text{ et } \mathcal{B}\setminus \mathcal{L}' \subset \mathcal{G}
		\] \[
			\mathcal{B} \setminus \mathcal{L} = \underbrace{\mathcal{B}\setminus\mathcal{L}'}_{\subset \mathcal{G}} \cup \underbrace{\{u\}}_{\subset \mathcal{G} \text{ car } u \in \mathcal{G}}
		\] On a $\mathcal{B}\setminus\mathcal{L}\subset \mathcal{G}$
	\end{itemize}
\end{prv}

\begin{thm}
	Soit $E$ un $\mathbbm{K}$-espace vectoriel de dimension finie. Toutes les bases de $E$ ont le même cardinal.
\end{thm}

\begin{prv}
	Soit $\mathcal{G}$ une famille génératrice finie de $E$ et $\mathcal{B} \subset  \mathcal{G}$ une base de $E$. On note $n = \#\mathcal{B}$ \\
	Soit $\mathcal{B}'$ une base de $E$. On pose $p = n - \#(\mathcal{B} \cap  \mathcal{B}')$. Montrons par récurrence sur  $p$ que $\#\mathcal{B} = \#\mathcal{B}'$ 
	\begin{itemize}
		\item On suppose que $p = 0$. Alors, $\#(\mathcal{B} \cap \mathcal{B}') = n$ \\
			Or, $\mathcal{B}' \cap \mathcal{B} \subset \mathcal{B}$ donc $\mathcal{B} \cap \mathcal{B}' = \mathcal{B}$ donc $\mathcal{B} \subset  \mathcal{B}'$ et donc $\mathcal{B} = \mathcal{B}'$ 
		\item Soit $p \in \N$. On suppose que si $\mathcal{B}'$ est une base de $E$ telle que $n - \#(\mathcal{B} \cap \mathcal{B}') = p$, alors $\#\mathcal{B}' = n$ \\
			Aoit $\mathcal{B}'$ une base de $E$ telle que $n - \#(\mathcal{B}\cap \mathcal{B}') = p+1 > 0$ \\
			Donc $\mathcal{B} \cap \mathcal{B}' \neq \mathcal{B}$. Soit $u \in \mathcal{B}' \setminus \mathcal{B}$. D'après le lemme d'échange, il existe $v \in \mathcal{B}\setminus \mathcal{B}'$ tel que $\mathcal{B}' \setminus \{u\} \cup \{v\}$ est une base de $E$. On pose $\mathcal{B}'' = \mathcal{B}' \setminus \{u\} \cup \{v\}$ 
			\begin{align*}
				\mathcal{B}'' \cap \mathcal{B} &= \left( (\mathcal{B}' \setminus \{u\})  \cap \mathcal{B} \right) \cup \{v\} \\
				&= (\mathcal{B}' \cap \mathcal{B}) \cup \{v\} \\
			\end{align*}
			donc,
			\begin{align*}
				n - \#(\mathcal{B}'' \cap \mathcal{B}) &= n - (\#(\mathcal{B}' \cap \mathcal{B}) + 1) \\
				&= p+1- 1 \\
				&= p \\
			\end{align*}
			D'après l'hypothèse de récurrence, \[
				\#\mathcal{B}'' = n
			\] Or, $\#\mathcal{B}'' = \#\mathcal{B}'$
	\end{itemize}
\end{prv}

\begin{lem}
	Soient $\mathcal{B}$ et $\mathcal{B}'$ deux bases de $E$ telles que $\mathcal{B}\subset \mathcal{B}'$. Alors, $\mathcal{B} = \mathcal{B}'$.
\end{lem}

\begin{prv}
	On suppose $\mathcal{B}' \neq \mathcal{B}$. Soit $u \in \mathcal{B}' \setminus \mathcal{B}$
	$u \in E = \Vect(\mathcal{B})$ donc $\mathcal{B} \cup \{u\}$ n'est pas libre.
	Donc $\mathcal{B}\cup \{u\} \subset \mathcal{B}'$ et $\mathcal{B}'$ est libre donc $\mathcal{B}\cup \{u\}$ est libre: une contradiction $\lightning$
\end{prv}

\begin{lem}
	[Lemme d'échange] Soient $\mathcal{B}_1$ et $\mathcal{B}_2$ deux bases de $E$ et $u \in \mathcal{B}_1 \setminus \mathcal{B}_2$. Alors, il existe $v \in \mathcal{B}_2$ tel que $(\mathcal{B}_1 \setminus \{u\}) \cup \{v\}$ soit une base de $E$.
\end{lem}

\begin{prv}
	[1${}^\text{nde}$ méthode]
	On suppose que pout tout $v \in \mathcal{B}_2$, $(\mathcal{B}_1\setminus \{u\}) \cup \{v\}$ n'est pas une base de $E$
	Soit $v \in \mathcal{B}_2$.
	\begin{itemize}
		\item Supposons $(\mathcal{B}_1\setminus \{u\})\cup \{v\}$ non libre. $\mathcal{B}_1 \setminus \{u\}$ est libre. Donc $v \in \Vect(\mathcal{B}_1 \setminus \{u\})$
		\item Supposons $(\mathcal{B}_1\setminus \{u\}) \cup \{v\}$ non génératrice.
			Comme $\mathcal{B}_1$ engendre $E$, $u \not\in \Vect(\mathcal{B}_1\setminus \{v\})$.
			On suppose que $\mathcal{B}_1 \neq \mathcal{B}_2$.
			$\forall v \in \mathcal{B}_2 \setminus \mathcal{B}_1, \Vect(\mathcal{B}_1 \setminus \{v\}) = \Vect(\mathcal{B}_1) = E \ni u$ 
			donc, $(\mathcal{B}_1\setminus \{u\}) \cup \{v\}$ engendre $E$ et donc \[
				v \in \Vect(\mathcal{B}_1 \setminus \{u\})
			\] On a aussi \[
				\forall v \in \mathcal{B}_1 \setminus \{u\}, v \in \Vect(\mathcal{B}_1\setminus \{u\})
			\] Comme $u \not\in \mathcal{B}_2$, on a \[
				\forall v \in \mathcal{B}_2, v \in \Vect(\mathcal{B}_1\setminus \{u\})
			\] docn \[
				E = \Vect(\mathcal{B}_2) \subset \Vect(\mathcal{B}_1\setminus \{u\})
			\] donc $\mathcal{B}_1\setminus \{u\}$ engendre $E$ donc $\mathcal{B}_1\setminus \{u\}$ est une base de $E$. Or, $\mathcal{B}_1 \setminus \{u\}  \subset  \mathcal{B}_1$, donc $\mathcal{B}_1\setminus \{u\} = \mathcal{B}_1$
	\end{itemize}
\end{prv}

\begin{prv}
	[2${}^\text{nde}$ méthode]
	On suppose que pout tout $v \in \mathcal{B}_2$, $(\mathcal{B}_1\setminus \{u\}) \cup \{v\}$ n'est pas une base de $E$
	\begin{itemize}
		\item Comme $u \in \mathcal{B}_1 \setminus \mathcal{B}_2$, nécéssairement $\mathcal{B}_1 \neq \mathcal{B}_2$ donc $\mathcal{B}_2 \not\subset \mathcal{B}_1$, donc $\mathcal{B}_2\setminus\mathcal{B}_1 \neq \O$ 
		\item Soit $v \in \mathcal{B}_2\setminus\mathcal{B}_1$. Il existe $(\lambda_w)_{w\in\mathcal{B}_1}$ une famille de scalaires presque nulle telle que \[
				v = \sum_{w \in \mathcal{B}_1} \lambda_w w - \lambda_u u + + \sum_{w \in \mathcal{B}_1\setminus \{u\}}\lambda_w w
			\]
			Si $\lambda_u \neq 0_E$, alors
			\begin{align*}
				u &= \lambda_u^{-1}\left( v - \sum_{w \in \mathcal{B}_1 \setminus \{u\}} \lambda_w w \right)\\
					&\in \Vect(\mathcal{B}_1\setminus \{u\} \cup v)
			\end{align*}
			 donc $\mathcal{B}_1 \subset \Vect(\mathcal{B}_1\setminus \{u\} \cup \{v\})$\\
			 et donc $E \subset  \Vect(\mathcal{B}_1 \setminus \{u\} \cup \{v\})$ \\
			 et donc $\mathcal{B}_1 \setminus \{u\} \cup \{v\}$ engendre $E$ \\
			 donc $\mathcal{B}_1 \setminus \{u\} \cup \{v\}$ n'est pas libre\\
			 donc $v \in \Vect(\mathcal{B}_1\setminus \{u\})$ (car $\mathcal{B}_1 \setminus \{u\}$ est libre\\
			 donc $\lambda_u = 0_\mathbbm{K}$ $\lightning$\\`

			 Donc, $\lambda_u = 0_\mathbbm{K}$, docn $v \in \Vect(\mathcal{B}_1\setminus \{u\})$ \\
			 On vient de prouver que
			 \begin{align*}
			 	\mathcal{B}_2 \setminus \mathcal{B}_1 \subset \Vect(\mathcal{B}_1 \setminus \{u\})\\
			 	\mathcal{B}_1 \setminus \{u\} \subset \Vect(\mathcal{B}_1 \setminus \{u\})\\
			 \end{align*}
			 Comme $u \not\in \mathcal{B}_2$, \[
			 	\mathcal{B}_2 \subset \Vect(\mathcal{B}_1 \setminus \{u\})
			 \] donc \[
			 	E = \Vect(\mathcal{B}_2) \subset  \Vect(\mathcal{B}_1 \setminus \{u\})
			 \] donc $\mathcal{B}_1 \setminus \{u\}$ engendre $E$. Donc,  $\mathcal{B}_1 \setminus \{u\}$ est une base de $E$.\\
			 Or, $\mathcal{B}_1 \setminus \{u\} \subset  \mathcal{B}_1$, donc $\mathcal{B}_1 \setminus \{u\} = \mathcal{B}_1$
	\end{itemize}
\end{prv}

\begin{defn}
	Soit $E$ un $\mathbbm{K}$-espace vectoriel de dimension finie. Le cardinal commun à toutes les bases de $E$ est appelé \underline{dimension} de $E$ est notée $\dim(E)$ ou $\dim_\mathbbm{K}(E)$\\
	C'est donc aussi le nombre de coordonnées de n'importe quel vecteur dans n'importe quelle base.
	\index{dimension (espace vectoriel)}
\end{defn}

\begin{exm}
	\begin{enumerate}
		\item $\dim_\R(\C) = 2$ et $\dim_\C(\C) = 1$ 
		\item $\dim_\mathbbm{K}(\mathbbm{K}^{n}) = n$ 
		\item $\dim_{\mathbbm{K}}(\mathcal{M}_{n,p}(\mathbbm{K})) = np$
	\end{enumerate}
\end{exm}

\begin{crlr}
	Soit $E$ un $\mathbbm{K}$-espace vectoriel de dimension finie, $\mathcal{L}$ une famille libre de $E$, $\mathcal{G}$ une famille génératrice de $E$. On note $n = \dim(E)$
	\begin{enumerate}
		\item $\#\mathcal{G} \ge n$ et $(\#\mathcal{G} = n \implies \mathcal{G} \text{ est une base de } E$)
		\item $\#\mathcal{L} \le n$ et $(\#\mathcal{L} = n \implies \mathcal{L} \text{ est une base de } E$)
	\end{enumerate}
\end{crlr}

\begin{crlr}
	$\R^{\R}$ est de dimension infinie.
	$\forall i \in \N, e_i: x \mapsto x^i$\\
	$(e_i)_{i\in\N}$ est libre dans $\R^\R$
\end{crlr}

\begin{prop}
	Soient $E$ et $F$ deux $\mathbbm{K}$-espaces vectoriels de dimension finie. Alors $E\times F$ est de dimension finie et $\dim(E\times F) = \dim(E) + \dim(F)$
\end{prop}

\begin{prv}
	Soit $(e_1,\ldots, e_n)$ une base de $E$, $(f_1, \ldots, f_p)$ une base de $F$.
	On pose \[
		\left\{\begin{array}
			{r c l}
			u_1 &=& (e_1,0_F)\\
			u_2 &=& (e_2,0_F)\\
					&\vdots&\\
			u_n &=& (e_n,0_F)\\
			u_{n+1} &=& (0_E, f_1)\\
			u_{n+2} &=& (0_E, f_2)\\
					&\vdots&\\
			u_{n+p} &=& (0_E,f_p)\\
		\end{array}\right.
	\]
	Soit $(x,y) \in E\times F$. \[
		\begin{cases}
			\exists (x_1,\ldots,x_n)\in \mathbbm{K}^n, x = \sum_{i=1}^{n} x_ie_i
			\exists (y_1,\ldots,y_n)\in \mathbbm{K}^n, x = \sum_{j=1}^{p} y_jf_j
		\end{cases}
	\] 
	\begin{align*}
		(x,y) &= \left( \sum_{i=1}^{n} x_ie_i, \sum_{i=1}^{p} y_jf_j \right)  \\
		&= \sum_{i=1}^{n} x_i (e_i + 0_F) + \sum_{j=1}^{p} y_j (0_E, f_j) \\
		&= \sum_{i=1}^{n} x_i u_i + \sum_{j=1}^{p} y_j u_{n+j} \\
	\end{align*}
	Donc, $E\times F = \Vect(u_1, \ldots, u_{n+p})$ donc $E\times F$ est de dimension finie.\\
	Soit $(\lambda_1, \ldots, \lambda_{n+p}) \in \mathbbm{K}^{n+p}$ tel que \[
		(*): \quad \sum_{k=1}^{n+p} \lambda_ku_k = 0_{E\times F} = (0_E, 0_F)
	\]
	\begin{align*}
		(*) &\iff \sum_{k=1}^{n} \lambda_k (e_k, 0_F) + \sum_{k=n+1}^{p} \lambda_k(0_E, f_{k-n}) = (0_E, 0_F)\\
				&\iff \begin{cases}
					\sum_{k=1}^{n} \lambda_k e_k = 0_E\\
					\sum_{k=n+1}^{p} \lambda_k f_{k-n} = 0_F
				\end{cases}\\
				&\iff \begin{cases}
					\forall k \in \left\llbracket 1,n \right\rrbracket, \lambda_k = 0_\mathbbm{K} \qquad&(\text{car $(e_1,\ldots,e_n)$ est libre})\\
					\forall k \in \left\llbracket n+1,n+p \right\rrbracket, \lambda_k = 0_\mathbbm{K} \qquad&(\text{car $(f_1,\ldots,f_n)$ est libre})\\
				\end{cases}
	\end{align*}
	Donc $(u_1, \ldots, u_{n+p})$ est une base de $E\times F$. Donc, $\dim(E\times F) = n + p = \dim(E) + \dim(F)$
\end{prv}

\begin{rmk}
	[Convention]
	\[\dim\big(\{0_E\}\big) = 0\]
\end{rmk}

\begin{thm}
	Soit $E$ un $\mathbbm{K}$-espace vectoriel de dimension finie, $F$ un sous-espace vectoriel de $E$. Alors, $F$ est de dimension finie et  $\dim(F) \le \dim(E)$\\
	Si $\dim(F) = \dim(E)$, alors $F = E$
\end{thm}

\begin{prv}
	On considère \[
		A = \{k \in \N \mid \text{il existe une famille libre de $F$ à $k$ éléments}\} 
	\]
	On suppose $F \neq \{0_E\}$.
	\begin{itemize}
		\item Soit $u \in F\setminus \{0_E\}$. $(u)$ est libre donc $1 \in A$ et donc $A \neq \O$
		\item Soit $\mathcal{L}$ une famille libre de $F$. Alors, $\mathcal{L}$ est une famille libre de $E$ \\
			donc $\#\mathcal{L} \le \dim(E)$\\
			Donc $A$ est majorée par $\dim(E)$ \\
			On en déduit que $A$ a un plus grand élément $p$.
		\item Soit $\mathcal{L}$ une famille libre de $F$ avec $p$ éléments.\\
			Si $\mathcal{L}$ n'engendre pas $F$, alors il existe $u\in F$ tel que $u\not\in \Vect(\mathcal{L})$ et donc $\mathcal{L} \cup \{u\}$ est une famille libre de $F$, donc $p+1 \in A$ en contradiction avec la maximalité de $p$.\\
			Donc $\mathcal{L}$ est une base de $F$ donc $F$ est de dimension finie et $\dim(F) = p \le \dim(E)$\\
	\end{itemize}

	Soit $\mathcal{B}$ une base de $F$. Alors, $\mathcal{B}$ est aussi une famille de libre de de $E$. Donc $\#\mathcal{B} \le \dim(E)$ donc $\dim(F) = \dim(E)$ \\
	Si $\dim(F) = \dim(E)$, alors $\mathcal{B}$ est une base de $E$, et donc $F = \Vect(\mathcal{B}) = E$
\end{prv}

\begin{prop}
	[Formule de Grassmann]
	Soit $E$ un $\mathbbm{K}$-espace vectoriel de dimension finie, $F$ et $G$ deux sous-espace vectoriels de $E$. Alors, \[
		\dim(F+G) = \dim(F) + \dim(G) - \dim(F\cap G)
	\] 
\end{prop}

\begin{prv}
	Soit $(e_1, \ldots, e_p)$ une base de $F\cap G$. $(e_1,\ldots,e_p)$ est une famille libre de $F$.\\
	On complète $(e_1, \ldots, e_p)$ en une base $(e_1, \ldots, e_p, u_1, \ldots, u_q)$ de $F$.\\
	De même, on complète $(e_1, \ldots, e_p)$ en une base $(e_1, \ldots, e_p, v_1, \ldots, v_r)$ de $G$.\\
	On pose  $\mathcal{B} = (e_1, \ldots, e_p, u_1, \ldots, u_q, v_1, \ldots, v_r)$. Montrons que $\mathcal{B}$ est une base de $F+G$
	\begin{itemize}
		\item Soit $u \in F+G$ \\
			On pose $u = v+w$ avec $\begin{cases}
				v\in F\\
				w \in G
			\end{cases}$.\\
			On pose $v = \sum_{i=1}^p \lambda_i e_i + \sum_{i=1}^q \mu_i u_i$ avec $(\lambda_1, \ldots, \lambda_p, \mu_1, \ldots, \lambda_q) \in \mathbbm{K}^{p+q}$\\
			On pose aussi $w = \sum_{i = 1}^p \lambda'_ie_i + \sum_{j=1}^r \nu_j v_j$ avec $(\lambda_1',\ldots,\lambda_p', \nu_1, \ldots, \nu_r) \in \mathbbm{K}^{p+r}$\\
			D'où, \[
				u = \sum_{i=1}^p (\lambda_i + \lambda'_i)e_i + \sum_{j=1}^q \mu_j u_j + \sum_{k=1}^r \nu_k v_k \in \Vect(\mathcal{B})
			\]
		\item Soient $(\lambda_1, \ldots, \lambda_p, \mu_1, \ldots, \mu_q, \nu_1, \ldots, \nu_r) \in \mathbbm{K}^{p+q+r}$.\\
			On suppose \[
				(*)\quad \sum_{i=1}^{p}\lambda_ie_i + \sum_{j=1}^q\mu_ju_j + \sum_{k=1}^r \nu_k v_k = 0_E
			\] 
			D'où, \[
				\underbrace{\sum_{i=1}^p\lambda_i e_i + \sum_{j=1}^q \mu_ju_j}_{\in F} = \underbrace{-\sum_{k=1}^r\nu_jv_k}_{\in G}
			\] 
			Donc, \[
				f = \sum_{i=1}^p \lambda_i e_i + \sum_{j=1}^q \mu_j u_j \in F\cap G
			\] Comme $(e_1, \ldots, e_p)$ est une base de $F\cap G$, $\exists ! (\lambda_1', \ldots, \lambda_p') \in \mathbbm{K}^p$ tel que \[
				f = \sum_{i=1}^p \lambda'_i e_i = \sum_{i=1}^p \lambda'_i e_i + \sum_{j=1}^q 0_\mathbbm{K}u_j
			\] Comme $(e_1, \ldots, e_p, u_1, \ldots, u_q)$ est une base de $F$, \[
				\forall k \in \left\llbracket 1, q \right\rrbracket, \mu_j = 0_\mathbbm{K}
			\] De même, \[
				\forall k \in \left\llbracket 1,r \right\rrbracket , \nu_k = 0_\mathbbm{K}
			\] On remplace dans $(*)$ pour trouver \[
				\sum_{i=1}^p \lambda_ie_i = 0_E
			\] Comme $(e_1, \ldots, e_p)$ est libre, \[
				\forall i \in \left\llbracket 1,p \right\rrbracket, \lambda_i = 0_\mathbbm{K}
			\] Donc $\mathcal{B}$ est libre.\\
			Donc, 
			\begin{align*}
				\dim(F+G) &=  p +q + r \\
				&= (p+q)+ (p+r) - p \\
				&= \dim(F) + \dim(G) - \dim(F\cap G) \\
			\end{align*}
	\end{itemize}
\end{prv}

\begin{crlr}
	Avec les hypothèse précédentes, \[
		E = F \oplus G \iff \begin{cases}
			F \cap  G = \{0_E\} \\
			\dim(E) = \dim(F) + \dim(G)
		\end{cases}
	\] 
\end{crlr}

\begin{prv}
	\begin{itemize}
		\item[``$\implies$''] On suppose $E = F \oplus G$ \\
			Comme la somme est directe, $F \cap G = \{0_E\}$ 
			\begin{align*}
				\dim(E) &= \dim(F)\\
				&= \dim(F) + \dim(G) - \dim(F\cap G)\\
				&= \dim(F) + \dim(G)\\
			\end{align*}
		\item[``$\impliedby$''] On suppose $F\cap G = \{0_E\}$ et $\dim(E) = \dim(F) + \dim(G)$.\\
			On sait déjà que $F+G = F \oplus G$\\
			 \begin{align*}
				\dim(F+G) = \dim(F) + \dim(G) - \dim(F \cap G) = \dim(E)
			\end{align*}
			Donc $F + G = E$
	\end{itemize}
\end{prv}

\begin{prop}
	Soit $F$ un $\mathbbm{K}$-espace vectoriel de dimension finie $n$. Soit $\mathcal{B} = (e_1, \ldots, e_n)$ une base de $F$. L'application
	\begin{align*}
		f: \mathbbm{K}^n &\longrightarrow F \\
		(\lambda_1, \ldots, \lambda_n) &\longmapsto \sum_{i=1}^n \lambda_i e_i
	\end{align*} est bijective.\\
	Si $\mathbbm{K}$ est infini, $\mathbbm{K}^n$ aussi et donc $F$ aussi.\\
	Si $\#\mathbbm{K} = p \in \N_*$,
	\begin{align*}
		\#&\mathbbm{K}^n = p^n\\
		&\vrt=\\
		\#&F
	\end{align*}
\end{prop}


	}

	{
		\chap[18]{Polynômes formels}
		\renewcommand{\cwd}{../chap18}
		Dans ce chapitre, $\mathbbm{K}$ désigne un corps
		\begin{defn}
	Soit $E$ un $\mathbbm{K}$-espace vectoriel. On dit que $E$ est de \underline{dimension finie} si $E$ a au moins une famille génératrice finie. On dit que $E$ est de \underline{dimension infinie} sinon.
	\index{dimension finie (espace vectoriel)}
	\index{dimension infinie (espace vectoriel)}
\end{defn}

\begin{thm}
	[Théorème de la base extraite]
	Soit $E$ un $\mathbbm{K}$-espace vectoriel non nul de dimension finie. Soit $\mathcal{G}$ une famille génératrice finie de $E$. Alors, il existe une base $\mathcal{B}$ de $\mathcal{E}$ telle que $\mathcal{B} \subset \mathcal{G}$.
\end{thm}

\begin{prv}
	[par récurrence sur $\#G = \Card(G)$]
	\begin{itemize}
		\item Soit $E$ un $\mathbbm{K}$-espace vectoriel non nul engendré par $\mathcal{G} = (u)$.\\
			Si $u = 0_E$, alors $E = \{0_E\}$: une contradiction $\lightning$ \\
			Donc $u \neq 0_E$ donc $(u)$ est libre. En effet, \[
				\forall \lambda \in \mathbbm{K}, \lambda u = 0_E \implies \lambda = 0_\mathbbm{K}
			\] Donc $\mathcal{G}$ est une base de $E$.\\
		\item Soit $n \in \N_*$. Soit $E$ un $\mathbbm{K}$-espace vectoriel. On suppose que si $E$ a une famille génératrice constituée de $n$ vecteurs, alors on peut extraire de cette famille une base de $E$.\\
			Soit $\mathcal{G}$ une famille génératrice de $E$ avec $n+1$ vecteurs.\\
			Si $\mathcal{G}$ est libre, alors $\mathcal{G}$ est une base de $E$. \\
			Si $\mathcal{G}$ n'est pas libre, alors il existe $u \in \mathcal{G}$ tel que $u \in \Vect(\mathcal{G}\setminus \{u\})$ \\
			Donc $\mathcal{G}\setminus \{u\}$ engendre $E$. Or, $\mathcal{G}\setminus \{u\}$ possède $n$ vecteurs. D'après l'hypothèse de récurrence, il existe une base $\mathcal{B}$ de $E$ telle que \[
				\mathcal{B} \subset \mathcal{G} \setminus \{u\} \subset \mathcal{G}
			\] 
	\end{itemize}
\end{prv}

\begin{crlr}
	Tout espace de dimension finie a une base.
	\qed
\end{crlr}

\begin{thm}
	[Théorème de la base incomplète]
	Soit $E$ un $\mathbbm{K}$-espace vectoriel de dimension finie, $\mathcal{G}$ une famille génératrice finie de $E$. $\mathcal{L}$ une famille libre de $E$. Alors, il existe une base $\mathcal{B}$ de $E$ telle que \[
		\mathcal{L} \subset \mathcal{B} \text{ et } \mathcal{B}\setminus \mathcal{L} \subset \mathcal{G}
	\] 
\end{thm}

\begin{prv}
	[par récurrence sur $\#(\mathcal{G}\setminus\mathcal{L})$]
	\begin{itemize}
		\item Avec les notations précédentes, on suppose que $\mathcal{G}\setminus\mathcal{L} \neq \O$ \[
				\forall u \in \mathcal{G}, u \in \mathcal{L}
			\] Donc $\mathcal{G} \subset \mathcal{L}$ donc $\mathcal{L}$ est génératrice donc $\mathcal{L}$ est une base de $E$. On pose $\mathcal{B} = \mathcal{L}$ et alors \[
				\mathcal{L} \subset  \mathcal{B} \text{ et } \mathcal{B}\setminus\mathcal{L} = \O \subset  \mathcal{G}
			\] 
		\item Soit $n \in \N$. On suppose que si $\mathcal{G}$ est génératrice et $\mathcal{L}$ libre avec $\#(\mathcal{G}\setminus\mathcal{L}) = n$ alors il existe une base $\mathcal{B}$ de $E$ telle que \[
			\mathcal{L}\subset \mathcal{B} \text{ et } \mathcal{B}\setminus\mathcal{L}\subset \mathcal{G}
		\] Soient à présent $\mathcal{G}$ une famille génératrice de $E$ et $\mathcal{L}$ une famille libre de $E$ telles que $\#(\mathcal{G}\setminus\mathcal{L}) = n+1 > 0$\\
		Si $\mathcal{L}$ engendre $E$, alors $\mathcal{L}$ est une base de $E$. On pose $\mathcal{B} = \mathcal{L}$ et on a bien \[
			\mathcal{L} \subset  \mathcal{B} \text{ et } \mathcal{B} \setminus \mathcal{L} = \O \subset  \mathcal{G}
		\] On suppose que $\mathcal{L}$ n'engendre pas $E$. Il existe $u \in \mathcal{G}$ tel que $u \not\in \Vec(\mathcal{L})$ (car sinon, $\mathcal{G} \subset \Vect(\mathcal{L})$ et donc $\underbrace{\Vect(\mathcal{G})}_{= E} \subset  \underbrace{\Vect(\mathcal{L})}_{ \subset E}$\\
		Donc $\mathcal{L} \cup \{u\} $ est libre. On pose $\mathcal{L}' = \mathcal{L} \cup \{u\} $ \[
			\mathcal{G}\setminus \mathcal{L}' = \mathcal{G}\setminus (\mathcal{L} \cup \{u\}) = (\mathcal{G}\setminus\mathcal{L})\setminus \{u\} 
		\] donc $\#(\mathcal{G}\setminus\mathcal{L}') = n+1 -1 = n$\\
		D'après l'hypothèse de récurrence, il existe $\mathcal{B}$ une base de $E$ telle que \[
			\mathcal{L} \subset  \mathcal{L}' \subset \mathcal{B} \text{ et } \mathcal{B}\setminus \mathcal{L}' \subset \mathcal{G}
		\] \[
			\mathcal{B} \setminus \mathcal{L} = \underbrace{\mathcal{B}\setminus\mathcal{L}'}_{\subset \mathcal{G}} \cup \underbrace{\{u\}}_{\subset \mathcal{G} \text{ car } u \in \mathcal{G}}
		\] On a $\mathcal{B}\setminus\mathcal{L}\subset \mathcal{G}$
	\end{itemize}
\end{prv}

\begin{thm}
	Soit $E$ un $\mathbbm{K}$-espace vectoriel de dimension finie. Toutes les bases de $E$ ont le même cardinal.
\end{thm}

\begin{prv}
	Soit $\mathcal{G}$ une famille génératrice finie de $E$ et $\mathcal{B} \subset  \mathcal{G}$ une base de $E$. On note $n = \#\mathcal{B}$ \\
	Soit $\mathcal{B}'$ une base de $E$. On pose $p = n - \#(\mathcal{B} \cap  \mathcal{B}')$. Montrons par récurrence sur  $p$ que $\#\mathcal{B} = \#\mathcal{B}'$ 
	\begin{itemize}
		\item On suppose que $p = 0$. Alors, $\#(\mathcal{B} \cap \mathcal{B}') = n$ \\
			Or, $\mathcal{B}' \cap \mathcal{B} \subset \mathcal{B}$ donc $\mathcal{B} \cap \mathcal{B}' = \mathcal{B}$ donc $\mathcal{B} \subset  \mathcal{B}'$ et donc $\mathcal{B} = \mathcal{B}'$ 
		\item Soit $p \in \N$. On suppose que si $\mathcal{B}'$ est une base de $E$ telle que $n - \#(\mathcal{B} \cap \mathcal{B}') = p$, alors $\#\mathcal{B}' = n$ \\
			Aoit $\mathcal{B}'$ une base de $E$ telle que $n - \#(\mathcal{B}\cap \mathcal{B}') = p+1 > 0$ \\
			Donc $\mathcal{B} \cap \mathcal{B}' \neq \mathcal{B}$. Soit $u \in \mathcal{B}' \setminus \mathcal{B}$. D'après le lemme d'échange, il existe $v \in \mathcal{B}\setminus \mathcal{B}'$ tel que $\mathcal{B}' \setminus \{u\} \cup \{v\}$ est une base de $E$. On pose $\mathcal{B}'' = \mathcal{B}' \setminus \{u\} \cup \{v\}$ 
			\begin{align*}
				\mathcal{B}'' \cap \mathcal{B} &= \left( (\mathcal{B}' \setminus \{u\})  \cap \mathcal{B} \right) \cup \{v\} \\
				&= (\mathcal{B}' \cap \mathcal{B}) \cup \{v\} \\
			\end{align*}
			donc,
			\begin{align*}
				n - \#(\mathcal{B}'' \cap \mathcal{B}) &= n - (\#(\mathcal{B}' \cap \mathcal{B}) + 1) \\
				&= p+1- 1 \\
				&= p \\
			\end{align*}
			D'après l'hypothèse de récurrence, \[
				\#\mathcal{B}'' = n
			\] Or, $\#\mathcal{B}'' = \#\mathcal{B}'$
	\end{itemize}
\end{prv}

\begin{lem}
	Soient $\mathcal{B}$ et $\mathcal{B}'$ deux bases de $E$ telles que $\mathcal{B}\subset \mathcal{B}'$. Alors, $\mathcal{B} = \mathcal{B}'$.
\end{lem}

\begin{prv}
	On suppose $\mathcal{B}' \neq \mathcal{B}$. Soit $u \in \mathcal{B}' \setminus \mathcal{B}$
	$u \in E = \Vect(\mathcal{B})$ donc $\mathcal{B} \cup \{u\}$ n'est pas libre.
	Donc $\mathcal{B}\cup \{u\} \subset \mathcal{B}'$ et $\mathcal{B}'$ est libre donc $\mathcal{B}\cup \{u\}$ est libre: une contradiction $\lightning$
\end{prv}

\begin{lem}
	[Lemme d'échange] Soient $\mathcal{B}_1$ et $\mathcal{B}_2$ deux bases de $E$ et $u \in \mathcal{B}_1 \setminus \mathcal{B}_2$. Alors, il existe $v \in \mathcal{B}_2$ tel que $(\mathcal{B}_1 \setminus \{u\}) \cup \{v\}$ soit une base de $E$.
\end{lem}

\begin{prv}
	[1${}^\text{nde}$ méthode]
	On suppose que pout tout $v \in \mathcal{B}_2$, $(\mathcal{B}_1\setminus \{u\}) \cup \{v\}$ n'est pas une base de $E$
	Soit $v \in \mathcal{B}_2$.
	\begin{itemize}
		\item Supposons $(\mathcal{B}_1\setminus \{u\})\cup \{v\}$ non libre. $\mathcal{B}_1 \setminus \{u\}$ est libre. Donc $v \in \Vect(\mathcal{B}_1 \setminus \{u\})$
		\item Supposons $(\mathcal{B}_1\setminus \{u\}) \cup \{v\}$ non génératrice.
			Comme $\mathcal{B}_1$ engendre $E$, $u \not\in \Vect(\mathcal{B}_1\setminus \{v\})$.
			On suppose que $\mathcal{B}_1 \neq \mathcal{B}_2$.
			$\forall v \in \mathcal{B}_2 \setminus \mathcal{B}_1, \Vect(\mathcal{B}_1 \setminus \{v\}) = \Vect(\mathcal{B}_1) = E \ni u$ 
			donc, $(\mathcal{B}_1\setminus \{u\}) \cup \{v\}$ engendre $E$ et donc \[
				v \in \Vect(\mathcal{B}_1 \setminus \{u\})
			\] On a aussi \[
				\forall v \in \mathcal{B}_1 \setminus \{u\}, v \in \Vect(\mathcal{B}_1\setminus \{u\})
			\] Comme $u \not\in \mathcal{B}_2$, on a \[
				\forall v \in \mathcal{B}_2, v \in \Vect(\mathcal{B}_1\setminus \{u\})
			\] docn \[
				E = \Vect(\mathcal{B}_2) \subset \Vect(\mathcal{B}_1\setminus \{u\})
			\] donc $\mathcal{B}_1\setminus \{u\}$ engendre $E$ donc $\mathcal{B}_1\setminus \{u\}$ est une base de $E$. Or, $\mathcal{B}_1 \setminus \{u\}  \subset  \mathcal{B}_1$, donc $\mathcal{B}_1\setminus \{u\} = \mathcal{B}_1$
	\end{itemize}
\end{prv}

\begin{prv}
	[2${}^\text{nde}$ méthode]
	On suppose que pout tout $v \in \mathcal{B}_2$, $(\mathcal{B}_1\setminus \{u\}) \cup \{v\}$ n'est pas une base de $E$
	\begin{itemize}
		\item Comme $u \in \mathcal{B}_1 \setminus \mathcal{B}_2$, nécéssairement $\mathcal{B}_1 \neq \mathcal{B}_2$ donc $\mathcal{B}_2 \not\subset \mathcal{B}_1$, donc $\mathcal{B}_2\setminus\mathcal{B}_1 \neq \O$ 
		\item Soit $v \in \mathcal{B}_2\setminus\mathcal{B}_1$. Il existe $(\lambda_w)_{w\in\mathcal{B}_1}$ une famille de scalaires presque nulle telle que \[
				v = \sum_{w \in \mathcal{B}_1} \lambda_w w - \lambda_u u + + \sum_{w \in \mathcal{B}_1\setminus \{u\}}\lambda_w w
			\]
			Si $\lambda_u \neq 0_E$, alors
			\begin{align*}
				u &= \lambda_u^{-1}\left( v - \sum_{w \in \mathcal{B}_1 \setminus \{u\}} \lambda_w w \right)\\
					&\in \Vect(\mathcal{B}_1\setminus \{u\} \cup v)
			\end{align*}
			 donc $\mathcal{B}_1 \subset \Vect(\mathcal{B}_1\setminus \{u\} \cup \{v\})$\\
			 et donc $E \subset  \Vect(\mathcal{B}_1 \setminus \{u\} \cup \{v\})$ \\
			 et donc $\mathcal{B}_1 \setminus \{u\} \cup \{v\}$ engendre $E$ \\
			 donc $\mathcal{B}_1 \setminus \{u\} \cup \{v\}$ n'est pas libre\\
			 donc $v \in \Vect(\mathcal{B}_1\setminus \{u\})$ (car $\mathcal{B}_1 \setminus \{u\}$ est libre\\
			 donc $\lambda_u = 0_\mathbbm{K}$ $\lightning$\\`

			 Donc, $\lambda_u = 0_\mathbbm{K}$, docn $v \in \Vect(\mathcal{B}_1\setminus \{u\})$ \\
			 On vient de prouver que
			 \begin{align*}
			 	\mathcal{B}_2 \setminus \mathcal{B}_1 \subset \Vect(\mathcal{B}_1 \setminus \{u\})\\
			 	\mathcal{B}_1 \setminus \{u\} \subset \Vect(\mathcal{B}_1 \setminus \{u\})\\
			 \end{align*}
			 Comme $u \not\in \mathcal{B}_2$, \[
			 	\mathcal{B}_2 \subset \Vect(\mathcal{B}_1 \setminus \{u\})
			 \] donc \[
			 	E = \Vect(\mathcal{B}_2) \subset  \Vect(\mathcal{B}_1 \setminus \{u\})
			 \] donc $\mathcal{B}_1 \setminus \{u\}$ engendre $E$. Donc,  $\mathcal{B}_1 \setminus \{u\}$ est une base de $E$.\\
			 Or, $\mathcal{B}_1 \setminus \{u\} \subset  \mathcal{B}_1$, donc $\mathcal{B}_1 \setminus \{u\} = \mathcal{B}_1$
	\end{itemize}
\end{prv}

\begin{defn}
	Soit $E$ un $\mathbbm{K}$-espace vectoriel de dimension finie. Le cardinal commun à toutes les bases de $E$ est appelé \underline{dimension} de $E$ est notée $\dim(E)$ ou $\dim_\mathbbm{K}(E)$\\
	C'est donc aussi le nombre de coordonnées de n'importe quel vecteur dans n'importe quelle base.
	\index{dimension (espace vectoriel)}
\end{defn}

\begin{exm}
	\begin{enumerate}
		\item $\dim_\R(\C) = 2$ et $\dim_\C(\C) = 1$ 
		\item $\dim_\mathbbm{K}(\mathbbm{K}^{n}) = n$ 
		\item $\dim_{\mathbbm{K}}(\mathcal{M}_{n,p}(\mathbbm{K})) = np$
	\end{enumerate}
\end{exm}

\begin{crlr}
	Soit $E$ un $\mathbbm{K}$-espace vectoriel de dimension finie, $\mathcal{L}$ une famille libre de $E$, $\mathcal{G}$ une famille génératrice de $E$. On note $n = \dim(E)$
	\begin{enumerate}
		\item $\#\mathcal{G} \ge n$ et $(\#\mathcal{G} = n \implies \mathcal{G} \text{ est une base de } E$)
		\item $\#\mathcal{L} \le n$ et $(\#\mathcal{L} = n \implies \mathcal{L} \text{ est une base de } E$)
	\end{enumerate}
\end{crlr}

\begin{crlr}
	$\R^{\R}$ est de dimension infinie.
	$\forall i \in \N, e_i: x \mapsto x^i$\\
	$(e_i)_{i\in\N}$ est libre dans $\R^\R$
\end{crlr}

\begin{prop}
	Soient $E$ et $F$ deux $\mathbbm{K}$-espaces vectoriels de dimension finie. Alors $E\times F$ est de dimension finie et $\dim(E\times F) = \dim(E) + \dim(F)$
\end{prop}

\begin{prv}
	Soit $(e_1,\ldots, e_n)$ une base de $E$, $(f_1, \ldots, f_p)$ une base de $F$.
	On pose \[
		\left\{\begin{array}
			{r c l}
			u_1 &=& (e_1,0_F)\\
			u_2 &=& (e_2,0_F)\\
					&\vdots&\\
			u_n &=& (e_n,0_F)\\
			u_{n+1} &=& (0_E, f_1)\\
			u_{n+2} &=& (0_E, f_2)\\
					&\vdots&\\
			u_{n+p} &=& (0_E,f_p)\\
		\end{array}\right.
	\]
	Soit $(x,y) \in E\times F$. \[
		\begin{cases}
			\exists (x_1,\ldots,x_n)\in \mathbbm{K}^n, x = \sum_{i=1}^{n} x_ie_i
			\exists (y_1,\ldots,y_n)\in \mathbbm{K}^n, x = \sum_{j=1}^{p} y_jf_j
		\end{cases}
	\] 
	\begin{align*}
		(x,y) &= \left( \sum_{i=1}^{n} x_ie_i, \sum_{i=1}^{p} y_jf_j \right)  \\
		&= \sum_{i=1}^{n} x_i (e_i + 0_F) + \sum_{j=1}^{p} y_j (0_E, f_j) \\
		&= \sum_{i=1}^{n} x_i u_i + \sum_{j=1}^{p} y_j u_{n+j} \\
	\end{align*}
	Donc, $E\times F = \Vect(u_1, \ldots, u_{n+p})$ donc $E\times F$ est de dimension finie.\\
	Soit $(\lambda_1, \ldots, \lambda_{n+p}) \in \mathbbm{K}^{n+p}$ tel que \[
		(*): \quad \sum_{k=1}^{n+p} \lambda_ku_k = 0_{E\times F} = (0_E, 0_F)
	\]
	\begin{align*}
		(*) &\iff \sum_{k=1}^{n} \lambda_k (e_k, 0_F) + \sum_{k=n+1}^{p} \lambda_k(0_E, f_{k-n}) = (0_E, 0_F)\\
				&\iff \begin{cases}
					\sum_{k=1}^{n} \lambda_k e_k = 0_E\\
					\sum_{k=n+1}^{p} \lambda_k f_{k-n} = 0_F
				\end{cases}\\
				&\iff \begin{cases}
					\forall k \in \left\llbracket 1,n \right\rrbracket, \lambda_k = 0_\mathbbm{K} \qquad&(\text{car $(e_1,\ldots,e_n)$ est libre})\\
					\forall k \in \left\llbracket n+1,n+p \right\rrbracket, \lambda_k = 0_\mathbbm{K} \qquad&(\text{car $(f_1,\ldots,f_n)$ est libre})\\
				\end{cases}
	\end{align*}
	Donc $(u_1, \ldots, u_{n+p})$ est une base de $E\times F$. Donc, $\dim(E\times F) = n + p = \dim(E) + \dim(F)$
\end{prv}

\begin{rmk}
	[Convention]
	\[\dim\big(\{0_E\}\big) = 0\]
\end{rmk}

\begin{thm}
	Soit $E$ un $\mathbbm{K}$-espace vectoriel de dimension finie, $F$ un sous-espace vectoriel de $E$. Alors, $F$ est de dimension finie et  $\dim(F) \le \dim(E)$\\
	Si $\dim(F) = \dim(E)$, alors $F = E$
\end{thm}

\begin{prv}
	On considère \[
		A = \{k \in \N \mid \text{il existe une famille libre de $F$ à $k$ éléments}\} 
	\]
	On suppose $F \neq \{0_E\}$.
	\begin{itemize}
		\item Soit $u \in F\setminus \{0_E\}$. $(u)$ est libre donc $1 \in A$ et donc $A \neq \O$
		\item Soit $\mathcal{L}$ une famille libre de $F$. Alors, $\mathcal{L}$ est une famille libre de $E$ \\
			donc $\#\mathcal{L} \le \dim(E)$\\
			Donc $A$ est majorée par $\dim(E)$ \\
			On en déduit que $A$ a un plus grand élément $p$.
		\item Soit $\mathcal{L}$ une famille libre de $F$ avec $p$ éléments.\\
			Si $\mathcal{L}$ n'engendre pas $F$, alors il existe $u\in F$ tel que $u\not\in \Vect(\mathcal{L})$ et donc $\mathcal{L} \cup \{u\}$ est une famille libre de $F$, donc $p+1 \in A$ en contradiction avec la maximalité de $p$.\\
			Donc $\mathcal{L}$ est une base de $F$ donc $F$ est de dimension finie et $\dim(F) = p \le \dim(E)$\\
	\end{itemize}

	Soit $\mathcal{B}$ une base de $F$. Alors, $\mathcal{B}$ est aussi une famille de libre de de $E$. Donc $\#\mathcal{B} \le \dim(E)$ donc $\dim(F) = \dim(E)$ \\
	Si $\dim(F) = \dim(E)$, alors $\mathcal{B}$ est une base de $E$, et donc $F = \Vect(\mathcal{B}) = E$
\end{prv}

\begin{prop}
	[Formule de Grassmann]
	Soit $E$ un $\mathbbm{K}$-espace vectoriel de dimension finie, $F$ et $G$ deux sous-espace vectoriels de $E$. Alors, \[
		\dim(F+G) = \dim(F) + \dim(G) - \dim(F\cap G)
	\] 
\end{prop}

\begin{prv}
	Soit $(e_1, \ldots, e_p)$ une base de $F\cap G$. $(e_1,\ldots,e_p)$ est une famille libre de $F$.\\
	On complète $(e_1, \ldots, e_p)$ en une base $(e_1, \ldots, e_p, u_1, \ldots, u_q)$ de $F$.\\
	De même, on complète $(e_1, \ldots, e_p)$ en une base $(e_1, \ldots, e_p, v_1, \ldots, v_r)$ de $G$.\\
	On pose  $\mathcal{B} = (e_1, \ldots, e_p, u_1, \ldots, u_q, v_1, \ldots, v_r)$. Montrons que $\mathcal{B}$ est une base de $F+G$
	\begin{itemize}
		\item Soit $u \in F+G$ \\
			On pose $u = v+w$ avec $\begin{cases}
				v\in F\\
				w \in G
			\end{cases}$.\\
			On pose $v = \sum_{i=1}^p \lambda_i e_i + \sum_{i=1}^q \mu_i u_i$ avec $(\lambda_1, \ldots, \lambda_p, \mu_1, \ldots, \lambda_q) \in \mathbbm{K}^{p+q}$\\
			On pose aussi $w = \sum_{i = 1}^p \lambda'_ie_i + \sum_{j=1}^r \nu_j v_j$ avec $(\lambda_1',\ldots,\lambda_p', \nu_1, \ldots, \nu_r) \in \mathbbm{K}^{p+r}$\\
			D'où, \[
				u = \sum_{i=1}^p (\lambda_i + \lambda'_i)e_i + \sum_{j=1}^q \mu_j u_j + \sum_{k=1}^r \nu_k v_k \in \Vect(\mathcal{B})
			\]
		\item Soient $(\lambda_1, \ldots, \lambda_p, \mu_1, \ldots, \mu_q, \nu_1, \ldots, \nu_r) \in \mathbbm{K}^{p+q+r}$.\\
			On suppose \[
				(*)\quad \sum_{i=1}^{p}\lambda_ie_i + \sum_{j=1}^q\mu_ju_j + \sum_{k=1}^r \nu_k v_k = 0_E
			\] 
			D'où, \[
				\underbrace{\sum_{i=1}^p\lambda_i e_i + \sum_{j=1}^q \mu_ju_j}_{\in F} = \underbrace{-\sum_{k=1}^r\nu_jv_k}_{\in G}
			\] 
			Donc, \[
				f = \sum_{i=1}^p \lambda_i e_i + \sum_{j=1}^q \mu_j u_j \in F\cap G
			\] Comme $(e_1, \ldots, e_p)$ est une base de $F\cap G$, $\exists ! (\lambda_1', \ldots, \lambda_p') \in \mathbbm{K}^p$ tel que \[
				f = \sum_{i=1}^p \lambda'_i e_i = \sum_{i=1}^p \lambda'_i e_i + \sum_{j=1}^q 0_\mathbbm{K}u_j
			\] Comme $(e_1, \ldots, e_p, u_1, \ldots, u_q)$ est une base de $F$, \[
				\forall k \in \left\llbracket 1, q \right\rrbracket, \mu_j = 0_\mathbbm{K}
			\] De même, \[
				\forall k \in \left\llbracket 1,r \right\rrbracket , \nu_k = 0_\mathbbm{K}
			\] On remplace dans $(*)$ pour trouver \[
				\sum_{i=1}^p \lambda_ie_i = 0_E
			\] Comme $(e_1, \ldots, e_p)$ est libre, \[
				\forall i \in \left\llbracket 1,p \right\rrbracket, \lambda_i = 0_\mathbbm{K}
			\] Donc $\mathcal{B}$ est libre.\\
			Donc, 
			\begin{align*}
				\dim(F+G) &=  p +q + r \\
				&= (p+q)+ (p+r) - p \\
				&= \dim(F) + \dim(G) - \dim(F\cap G) \\
			\end{align*}
	\end{itemize}
\end{prv}

\begin{crlr}
	Avec les hypothèse précédentes, \[
		E = F \oplus G \iff \begin{cases}
			F \cap  G = \{0_E\} \\
			\dim(E) = \dim(F) + \dim(G)
		\end{cases}
	\] 
\end{crlr}

\begin{prv}
	\begin{itemize}
		\item[``$\implies$''] On suppose $E = F \oplus G$ \\
			Comme la somme est directe, $F \cap G = \{0_E\}$ 
			\begin{align*}
				\dim(E) &= \dim(F)\\
				&= \dim(F) + \dim(G) - \dim(F\cap G)\\
				&= \dim(F) + \dim(G)\\
			\end{align*}
		\item[``$\impliedby$''] On suppose $F\cap G = \{0_E\}$ et $\dim(E) = \dim(F) + \dim(G)$.\\
			On sait déjà que $F+G = F \oplus G$\\
			 \begin{align*}
				\dim(F+G) = \dim(F) + \dim(G) - \dim(F \cap G) = \dim(E)
			\end{align*}
			Donc $F + G = E$
	\end{itemize}
\end{prv}

\begin{prop}
	Soit $F$ un $\mathbbm{K}$-espace vectoriel de dimension finie $n$. Soit $\mathcal{B} = (e_1, \ldots, e_n)$ une base de $F$. L'application
	\begin{align*}
		f: \mathbbm{K}^n &\longrightarrow F \\
		(\lambda_1, \ldots, \lambda_n) &\longmapsto \sum_{i=1}^n \lambda_i e_i
	\end{align*} est bijective.\\
	Si $\mathbbm{K}$ est infini, $\mathbbm{K}^n$ aussi et donc $F$ aussi.\\
	Si $\#\mathbbm{K} = p \in \N_*$,
	\begin{align*}
		\#&\mathbbm{K}^n = p^n\\
		&\vrt=\\
		\#&F
	\end{align*}
\end{prop}


		\part{Dérivation}

\underline{Motivation}:

{
\begin{wrapfigure}{l}{3cm}
	\centering
	\begin{asy}
		import three;

		size(3cm);
		settings.render=0;
		settings.prc=false;
		currentprojection = obliqueZ;

		draw(unitbox);
		draw(shift(1.1Z + 0.05X) * (O -- X), Arrows3(TeXHead2));
		draw(shift(1.1Z + 0.05Y) * (O -- Y), Arrows3(TeXHead2));
		draw(shift(1.1X + 0.05Z) * (O -- Z), Arrows3(TeXHead2));

		label("$x$", (X/2) + (1.1Z + 0.05X), align=S);
		label("$y$", (Y/2) + (1.1Z + 0.05Y), align=W);
		label("$z$", (Z/2) + X, align=SE);
	\end{asy}
\end{wrapfigure}

\begin{align*}
	&S(x,y,z) = 2(xy + xz + yz)\\
	&V(x,y,z) = xyz
\end{align*}

On cherche à minimiser $S$ avec la contrainte $V = 1$.

Soit $f : \begin{array}{rcl}
	\left( \R_*^+ \right)^2 &\longrightarrow& \R \\
	(x,y) &\longmapsto& S\left( x,y,\frac{1}{xy} \right) = 2\left( xy + \frac{1}{y} + \frac{1}{x} \right).
\end{array}$

On cherche $(a,b) \in \left( \R^+_* \right)^2$ tel que \[
	\forall (x,y) \in (\R^+_*), f(x,y) \ge f(a,b).
\]
}

\begin{defn}
	Soit $f: U \to \R$ où $U$ est un ouvert de $\R^2$. Soit $(a,b) \in U$.
	\vspace{2mm}

	Si $\lim_{x \to a} \frac{f(x,b) - f(a,b)}{x - a} \in \R$, alors on dit que $f$ a une dérivée partielle suivant $x$ en $(a,b)$ et cette limite est notée \[
		\partial f_1(a,b) = \frac{\partial f}{\partial x}(a,b).
	\]

	Si $\lim_{y \to b} \frac{f(a,y) - f(a,b)}{y - b} \in \R$, alors on dit que $f$ a une dérivée partielle suivant $y$ et la limite est notée \[
		\partial f_2(a,b) = \frac{\partial f}{\partial y}(a,b).
	\]
\end{defn}

\begin{exm}
	\begin{enumerate}
		\item $f: (x,y) \mapsto xy + x - y$.

			\begin{align*}
				&\frac{\partial f}{\partial x} : (x,y) \mapsto y + 1,\\
				&\frac{\partial f}{\partial y} : (x,y) \mapsto x - 1.
			\end{align*}

		\item $f: (x,y) \mapsto xy + \frac{1}{y}+ \frac{1}{x}$.

			\begin{align*}
				&\frac{\partial f}{\partial x}: (x,y) \mapsto y - \frac{1}{x^2},\\
				&\frac{\partial f}{\partial y}: (x,y) \mapsto x - \frac{1}{y^2}.
			\end{align*}

		\item Trouver $f$ telle que $\begin{cases}
				(1): \qquad \frac{\partial f}{\partial x}=y,\\[2mm]
				(2): \qquad \frac{\partial f}{\partial y} = x.
			\end{cases}$

			D'après $(1)$ : \[
				\forall (x,y), \exists C(y) \in \R, f(x,y) = xy + C(y)
			\] et donc \[
				\frac{\partial f}{\partial y}(x,y) = x + C'(y)
			\] donc $C'(y) = 0$ et donc $C$ est constante.

		\item Trouver $f$ telle que $\begin{cases}
			\frac{\partial f}{\partial x} = -y,\\[2mm]
			\frac{\partial f}{ƒ\partial y} = x.
		\end{cases}$

		Ce n'est pas possible !
	\end{enumerate}
\end{exm}

\begin{defn}~\\
	\begin{minipage}{\linewidth}
		\begin{wrapfigure}{r}{4cm}
			\centering
			\vspace{-5mm}
			\begin{asy}
				import three;
				import graph3;
				size(4cm);

				settings.render = 0;
				settings.prc = false;
				currentprojection = obliqueX;

				draw(O -- X, Arrow3(TeXHead2));
				draw(O -- Y, Arrow3(TeXHead2));
				draw(O -- Z, Arrow3(TeXHead2));

				triple f(real x, real y, real z = 0) { return (x,y,cos(x - 0.5) * cos(y - 0.5)/1.2 + 0.15); }

				real inc = 1 / 5;

				for(real x = 0; x <= 1; x += inc) {
					draw(graph(
						new real(real t) { return x; }, // x
						new real(real y) { return y; }, // y
						new real(real y) { return f(x,y).z; }, // z
						0, 1
					), gray);
				}

				for(real y = 0; y <= 1; y += inc) {
					draw(graph(
						new real(real x) { return x; }, // x
						new real(real t) { return y; }, // y
						new real(real x) { return f(x,y).z; }, // z
						0, 1
					), gray);
				}

				path3 path1 = (0.8, 0.2, 0) .. (0.5, 0.5, 0) .. (0.3, 0.7, 0);
				path3 path2 = f(0.8, 0.2, 0) .. f(0.5, 0.5, 0) .. f(0.3, 0.7, 0);
				path3 d = (0.2, 0.3, 0) .. (0.3, 0.4, 0) .. (0.2, 0.7, 0) .. (0.8, 0.9, 0) .. (0.6, 0.2, 0) .. cycle;

				draw(path1, red, Arrow3(TeXHead2));
				draw(path2, red, Arrow3(TeXHead2, position=0.8));

				dot((0.5, 0.5, 0));
				dot(f(0.5, 0.5, 0));
				draw((0.5, 0.5, 0) -- f(0.5, 0.5, 0), dashed);
				draw(d);

				label("$w$", (0.3, 0.7, 0), red, align=SE);
				label("$U$", (0.8, 0.9, 0), align=SE);
			\end{asy}
		\end{wrapfigure}

		Soit $f: U \to \R$ où $U$ est un ouvert. Soit $(a,b) \in U$. Soit $w = (w_1, w_2) \in \R^2$.

		Si 
		\[
			\lim_{t\to 0} \frac{f(a + tw_1, b + tw_2) - f(a,b)}{t}
		\] existe et est réelle, alors on dit que $f$ a une dérivée dans la direction de $w$ et la limite est notée \[
			\mathrm{d}f(w)\,(a,b) = D_w(f)\,(a,b).
		\]
	\end{minipage}
\end{defn}

\begin{exm}
	\begin{align*}
		f: \left( \R_*^+ \right)^2 &\longrightarrow \R \\
		(x,y) &\longmapsto xy+\frac{1}{x}+\frac{1}{y}.
	\end{align*}

	On pose $(a,b) = (1,2)$, $w = (w_1, w_2) = (1,1)$.
	\begin{align*}
		\frac{f(1+t, 2+t) - f(1,2)}{t} &= \frac{1}{t} \left( (1+t)(2+t) + \frac{1}{1+t} + \frac{1}{2+t} - 3 - \frac{1}{2} \right) \\
		&= \frac{1}{t}\left(\cancel 2 + 3t + \po(t) + \cancel 1 - t + \po(t) + \frac{1}{2}\left( \cancel 1 - \frac{t}{2} + \po(t) \right) - \cancel3 - \cancel{\frac{1}{2}} \right) \\
		&= \frac{1}{t} \left( \frac{7}{4} t + \po(t) \right)  \\
		&= \frac{7}{4} + \po(1) \tendsto{t \to 0} \frac{7}{4}. \\
	\end{align*}

	Donc, \[
		\mathrm{d}f(1,1)\,(1,2) = \frac{7}{4}.
	\]
\end{exm}

\begin{rmk}~\\
	\begin{figure}[H]
		\centering
		\begin{asy}
			import solids;
			import graph;
			size(5cm);

			settings.render = 0;
			settings.prc = false;

			path3 par = graph(
				new real(real x) { return x; },
				new real(real x) { return 0; },
				new real(real x) { return x^2; },
				0,3);
			revolution r = revolution(par, axis=Z);

			path3 par2 = graph(
				new real(real x) { return x; },
				new real(real x) { return 0; },
				new real(real x) { return x^2; },
				-3,3);

			draw(r,1,longitudinalpen=nullpen);
			draw(r.silhouette());

			draw((-4, 0, -1) -- (-4, 0, 10) -- (4, 0, 10) -- (4, 0, -1) -- cycle, red);
			draw(par2, deepred);

			draw((4,4.5) -- (7, 4.5), black+0.5mm, Arrow(TeXHead));

			path par2d = graph(new real(real x) { return x^2; }, -3, 3);
			draw(shift((11, 0)) * par2d, deepred);

			dot(O);
			dot((11, 0));
		\end{asy}
	\end{figure}
\end{rmk}


%todo ajouter théorème-définition
\begin{thm}
	Soit $f : U \to \R$, $(a,b) \in U$. On suppose que $\frac{\partial f}{\partial x}$ et $\frac{\partial f}{\partial y}$ existent en $(a,b)$ et sont {\bfseries continues} en $(a,b)$. Alors,
	\begin{align*}
		&\forall (h, k) \in \R^2 \text{ tel que } (a +h, b + k) \in U,\\
		&f(a+ h, b + k) = f(a,b) + h \frac{\partial f}{\partial x}(a,b) + k \frac{\partial f}{\partial y}(a,b) + \po_{(h,k)\to (0,0)}\big(\|(h,k)\|\big).
	\end{align*}

	On dit que $f$ est de classe $\mathcal{C}^1$ si $\frac{\partial f}{\partial x}$ et $\frac{\partial f}{\partial y}$ existent et sont continues.

	\qed
\end{thm}

\begin{rmk}
	En physique, cette formule correspond à : \[
		\mathrm{d}f = \frac{\partial f}{\partial x}\mathrm{d}x + \frac{\partial f}{\partial y} \mathrm{d}y.
	\] En effet :
	\begin{align*}
		\mathrm{d}f &= f(x+ \mathrm{d}x, y + \mathrm{d}y) - f(x,y) \\
		&= \frac{\partial f}{\partial x} \mathrm{d}x + \frac{\partial f}{\partial y} \mathrm{d}y.
	\end{align*}
\end{rmk}

\begin{prop}
	Soit $f: U \to \R$ de classe $\mathcal{C}^1$ en $(a,b) \in U$. Alors, \[
		\forall w = (w_1, w_2) \in \R^2, \mathrm{d}f(w)\,(a,b) = w_1 \frac{\partial f}{\partial x}(a,b) + w_2 \frac{\partial f}{\partial y}(a,b).
	\]
\end{prop}

\begin{prv}
	Soit $w = (w_1, w_2) \in \R^2$. Soit $t \in \R^*$.
	\begin{align*}
		\frac{1}{t}\big(f(a + tw_1, b + tw_2) - f(a,b)\big)
		&= \frac{1}{t} \left( tw_1 \frac{\partial f}{\partial x}(a,b) + tw_2 \frac{\partial f}{\partial y}(a,b) + \po_{t \to 0}\big(\|tw\|\big) \right) \\
		&= w_1 \frac{\partial f}{\partial x}(a,b) + w_2 \frac{\partial f}{\partial y}(a,b) + \po_{t\to 0}(1) \\
		&\tendsto{t\to 0} w_1 \frac{\partial f}{\partial x}(a,b) + w_2\frac{\partial f}{\partial y}(a,b).
	\end{align*}
\end{prv}


\begin{defn}
	Avec les hypothèses précédentes, en posant \[
		\nabla f(a,b) = \left( \frac{\partial f}{\partial x}(a,b), \frac{\partial f}{\partial y}(a,b) \right) 
	\]on obtient \[
		\mathrm{d}f(w)\,(a,b) = \left<w  \mid \nabla f(a,b) \right>
	\] où $\left<\cdot|\cdot \right>$ est le produit scalaire.

	Le vecteur $\nabla f(a,b)$ est appelé \underline{gradient de $f$ en $(a,b)$}.

	Le développement limité à l'ordre 1 de $f$ devient \[
		f\big((a,b)+w\big) = f(a,b) + \left<w \mid \nabla f(a,b) \right> + \po_{w\to 0}(\|w\|)
	\]
\end{defn}

\begin{prop}
	Soit $f : U \to \R$ de classe $\mathcal{C}^1$.

	\begin{figure}[H]
    \centering
    \incfig{gradient}
	\end{figure}

	$\nabla f$ est orthogonal au lignes de niveaux de $f$, son orientation va dans le sens d'une augmentation de $f$.
\end{prop}

\begin{prv}
	Soit $\gamma : I \to U$ une courbe de niveau : \[
		\forall t \in I, f\big(\gamma(t)\big) = \text{cste}.
	\] D'après le lemme suivant : \[
		\forall t \in I, 0 = (f \circ \gamma)'(t) = \mathrm{d}f\big(\gamma'(t)\big)\big(\gamma(t)\big) = \left<\gamma'(t)  \mid \nabla f\big(\gamma(t)\big) \right>
	\] Donc $\nabla f\big(\gamma(t)\big)$ est orthogonal à $\gamma'(t)$.

	Pour tout $t \in I$, on pose $w(t) = t\, \nabla f\big(\gamma(t)\big)$. Donc \[
		f\big(\gamma(t) + w(t)\big) = f\big(\gamma(t)\big) + t \|\nabla f(\gamma(t))\|^2 + \po_{t \to 0}(t)
	\] Pour $t$ assez petit, $f\big(\gamma(t) + w(t)\big) - f\big(\gamma(t)\big)$ est du même signe que $t$.
\end{prv}

\begin{rmk}
	\begin{align*}
		V: \R^3 &\longrightarrow \R \\
		(x,y,z) &\longmapsto -mgz
	\end{align*}
	l'énerge potentielle de pesenteur

	On a donc \[
		\nabla V(x,y,z) = \left( \frac{\partial V}{\partial x}, \frac{\partial V}{\partial y}, \frac{\partial V}{\partial z} \right) = (0, 0, -mg) = \vec{P}.
	\]
\end{rmk}

\begin{lem}
	Soit $f : U \to \R$ de classe $\mathcal{C}^1$, $\gamma : \begin{array}{rcl}
		I &\longrightarrow& U \\
		t &\longmapsto& \big(x(t), y(t)\big)
	\end{array}$ où $x$ et $y$ sont dérivables.

	On pose \[
		\forall t \in I, \gamma'(t) = \big(x'(t), y'(t)\big).
	\] Alors $f \circ \gamma : I \to \R$ est dérivable et
	\begin{align*}
		\forall t \in I, (f \circ \gamma)'(t) &= \mathrm{d}f\big(\gamma'(t)\big) \big(\gamma(t)\big)\\
		&= \left<\gamma'(t)  \mid \nabla f\big(\gamma(t)\big)  \right> \\
		&= x'(t) \frac{\partial f}{\partial x}\big(x(t), y(t)\big) + y'(t) \frac{\partial f}{\partial y}\big(x(t),y(t)\big). \\
	\end{align*}
\end{lem}

\begin{prv}
	On fixe $t \in I$.

	\begin{align*}
		\forall h \neq 0, \frac{f \circ \gamma(t + h) - f \circ \gamma(t)}{h}
		&= \frac{1}{h}\big(f(\gamma(t)) + h\gamma'(t) + \po_{h\to 0}(h) - f(\gamma(t))\big) \\
		&= \frac{1}{h}\bigg(\cancel{f(\gamma(t))} + \left<h\gamma'(t) \mid \nabla f(\gamma(t)) \right> + \po_{h\to 0}(\|h\gamma'(t)\|) - \cancel{f(\gamma(t))}\bigg)\\
		&= \left<\gamma'(t) \mid \nabla f(\gamma(t)) \right> + \po_{h\to 0}(1) \\
		&\tendsto{h\to 0} \left<\gamma'(t)  \mid \nabla f(\gamma(t)) \right>
	\end{align*}
\end{prv}

\begin{defn}
	Soit $f : U \to \R$ de classe $\mathcal{C}^1$ et $(a,b) \in U$. On dit que $(a,b)$ est un \underline{point critique} de $f$ si $\nabla f(a,b) = 0$ i.e. $\frac{\partial f}{\partial x}(a,b) = \frac{\partial f}{\partial y}(a,b) = 0$.

	Dans ce cas, $f(a,b)$ est appelé \underline{valeur critique} de $f$.
\end{defn}

\begin{prop}~\\
	\begin{minipage}{\linewidth}
		\begin{wrapfigure}{r}{3cm}
			\centering
			\vspace{-1cm}
			\begin{asy}
				import solids;
				import graph;
				size(3cm);

				settings.render = 0;
				settings.prc = false;

				path3 par = graph(
					new real(real x) { return x; },
					new real(real x) { return 0; },
					new real(real x) { return -x^2; },
					0,3);
				revolution r = revolution(par, axis=Z);

				draw(r,1,longitudinalpen=nullpen);
				draw(r.silhouette());

				dot("$(a,b)$", O, red, align=N);
				real s = sqrt(2.5);
				path3 g=(s,0,-2.5)..(0,s,-2.5)..(-s,0,-2.5)..(0,-s,-2.5)..cycle;
				draw(g, deepcyan);
			\end{asy}
		\end{wrapfigure}
		Soit $f: U \to \R$ de classe $\mathcal{C}^1$ et $(a,b) \in U$ tel que \[
			\exists r > 0, \forall (x,y) \in B_{(a,b)}(r), f(x,y) \le f(a,b)
		\] Alors $\nabla f(a,b) = (0,0)$.
	\end{minipage}
\end{prop}

\begin{prv}
	Soit $g: x \mapsto f(x,b)$. $g(a)$ est un maximum local de $g$ donc $g'(a) = 0$.

	Or, $g'(a) = \frac{\partial f}{\partial x}(a,b)$

	donc $\frac{\partial f}{\partial x}(a,b) = 0$.

	Soit $h : y \mapsto f(a,y)$. On a de même $h'(b) = 0$.

	Or, $h'(b) = \frac{\partial f}{\partial y}(a,b)$.

	Donc, $\nabla f(a,b) = (0,0)$.
\end{prv}

\begin{rmk}
	Un minimum local est aussi une valeur critique.
\end{rmk}

\begin{figure}[H]
	\centering
	\begin{subfigure}{3cm}
		\centering
		\begin{asy}
			import solids;
			import graph;
			size(3cm);

			settings.render = 0;
			settings.prc = false;

			path3 par = graph(
				new real(real x) { return x; },
				new real(real x) { return 0; },
				new real(real x) { return -x^2; },
				0,3);
			revolution r = revolution(par, axis=Z);

			draw(r,1,longitudinalpen=nullpen);
			draw(r.silhouette());

			dot(O, red);
		\end{asy}
		\caption{Maximum local}
	\end{subfigure}
	\begin{subfigure}{3cm}
		\centering
		\begin{asy}
			import solids;
			import graph;
			size(3cm);

			settings.render = 0;
			settings.prc = false;

			path3 par = graph(
				new real(real x) { return x; },
				new real(real x) { return 0; },
				new real(real x) { return x^2; },
				0,3);
			revolution r = revolution(par, axis=Z);

			draw(r,1,longitudinalpen=nullpen);
			draw(r.silhouette());

			dot(O, red);
		\end{asy}
		\caption{Minimum local}
	\end{subfigure}
	\begin{subfigure}{3cm}
		\centering
		\begin{asy}
			import solids;
			import graph;
			size(3cm);

			settings.render = 0;
			settings.prc = false;
			currentprojection = obliqueZ;

			draw(graph(
				new real(real x) { return x; },
				new real(real x) { return -x^2 / 3; },
				new real(real x) { return 3; },
				-3, 3
			));

			draw(graph(
				new real(real x) { return x; },
				new real(real x) { return -x^2 / 3; },
				new real(real x) { return -3; },
				-3, 3
			));

			draw(graph(
				new real(real x) { return x; },
				new real(real x) { return -x^2 / 3 - 1; },
				new real(real x) { return 0; },
				-3, 3
			));

			draw(graph(
				new real(real x) { return 0; },
				new real(real x) { return x^2 / 9 - 1; },
				new real(real x) { return x; },
				-3, 3
			));

			draw(graph(
				new real(real x) { return -3; },
				new real(real x) { return x^2 / 9 - 4; },
				new real(real x) { return x; },
				-3, 3
			));

			draw(graph(
				new real(real x) { return 3; },
				new real(real x) { return x^2 / 9 - 4; },
				new real(real x) { return x; },
				-3, 3
			));

			dot((0,-1,0), red);
		\end{asy}
		\caption{Point de selle / Point col}
	\end{subfigure}
\end{figure}

\begin{exm}
	On revient à l'exemple donné en introduction : 
	\begin{align*}
		f: \left( \R^*_+ \right)^2 &\longrightarrow \R \\
		(x,y) &\longmapsto 2\left( xy + \frac{1}{x} + \frac{1}{y} \right).
	\end{align*}

	$\left( \R^+_* \right)^2$ est un ouvert de $\R^2$. Soit $(x,y) \in \left( \R^+_* \right)^2$.
	
	On a \[
		\begin{cases}
			\frac{\partial f}{\partial x}(x,y) = 2\left( y - \frac{1}{x^2} \right),\\
			\frac{\partial f}{\partial y}(x,y) = 2\left( x - \frac{1}{y^2} \right).
		\end{cases}
	\]

	\begin{align*}
		&\frac{\partial f}{\partial x}(x,y) = \frac{\partial f}{\partial y}(x,y) = 0\\
		\iff& \begin{cases}
			y = \frac{1}{x^2}\\
			x = \frac{1}{y^2}
		\end{cases}\\
		\iff& \begin{cases}
			y = \frac{1}{x^2}\\
			x = x^4
		\end{cases}\\
		\iff& \begin{cases}
			x = 1\\
			y = 1
		\end{cases}
	\end{align*}

	On vérivie que $f$ présente en effet un minium local en $(1,1)$. \[
		f(1,1) = 6
	\] On fixe $y \in \R^+_*$ et \[
		g : x \mapsto 2\left( xy + \frac{1}{x} + \frac{1}{y} \right).
	\] Donc \[
		\forall x \in \R^+_*, g'(x) = 2\left( y - \frac{1}{x^2} \right).
	\]
	\begin{center}
		\begin{tikzpicture}
			\tkzTabInit{$x$/1,$g'(x)$/1,$g$/2.3}{$0$, $\frac{1}{\sqrt{y}}$, $+\infty$}
			\tkzTabLine{,-,z,+,}
			\tkzTabVar{+/{}, -/$2\left( 2\sqrt{y} +\frac{1}{y} \right)$, +/{}}
		\end{tikzpicture}
	\end{center}
	
	Ainsi, \[
		\forall x \in \R^+_*, \forall y \in \R^+_*, f(x,y) \ge 2\left( 2\sqrt{y} + \frac{1}{y} \right)
	\] Soit $h : y \mapsto 2\sqrt{y} + \frac{1}{y}$. On a \[
		\forall y > 0, h'(y) = \frac{1}{\sqrt{y}} - \frac{1}{y^2} = \frac{y\sqrt{y} - 1}{y^2} = \frac{y^{\frac{3}{2}} - 1}{y^2}
	\]

	\begin{center}
		\begin{tikzpicture}
			\tkzTabInit{$y$/0.7,$h'(y)$/0.7,$h$/1.4}{$0$, $1$, $+\infty$}
			\tkzTabLine{,-,z,+,}
			\tkzTabVar{+/{}, -/$3$, +/{}}
		\end{tikzpicture}
	\end{center}

	Donc, \[
		\forall x,y > 0, f(x,y) \ge 2\times 3 = 6 = f(1,1).
	\]
\end{exm}

\begin{prop}
	[règle de la chaîne]

	Soit $f : \begin{array}{rcl}
		U &\longrightarrow& \R^2 \\
		(x,y) &\longmapsto& f(x,y)
	\end{array}$ de classe $\mathcal{C}^1$ et $U, V$ deux ouverts de $\R^2$.

	Soit $\varphi : \begin{array}{rcl}
		V &\longrightarrow& U \\
		(u,v) &\longmapsto& \varphi(u,v) = \big(x(u,v), y(u,v)\big)
	\end{array}$.

	On suppose que $x$ et $y$ sont de classe $\mathcal{C}^1$ sur $V$.

	Alors,  $f \circ \varphi : \begin{array}{rcl}
		V &\longrightarrow& \R \\
		(u,v) &\longmapsto& f\big(\varphi(u,v)\big)
	\end{array}$ est de classe $\mathcal{C}^1$ et
	\begin{align*}
		\forall (u_0, v_0) \in V, \frac{\partial (f \circ \varphi)}{\partial u}(u_0, v_0)
		&= \frac{\partial f}{\partial x}\big(\varphi(u_0, v_0)\big) \times \frac{\partial x}{\partial u}(u_0, v_0)\\
		&+ \frac{\partial f}{\partial y}\big(\varphi(u_0,v_0)\big) \frac{\partial y}{\partial u}(u_0,v_0)
	\end{align*}
	\begin{align*}
		\forall (u_0, v_0) \in V, \frac{\partial (f \circ \varphi)}{\partial v}(u_0, v_0)
		&= \frac{\partial f}{\partial x}\big(\varphi(u_0, v_0)\big) \times \frac{\partial x}{\partial v}(u_0, v_0)\\
		&+ \frac{\partial f}{\partial y}\big(\varphi(u_0,v_0)\big) \frac{\partial y}{\partial v}(u_0,v_0)
	\end{align*}
\end{prop}

\begin{exm}
	[changement de coordonnées polaires]
	On pose \begin{align*}
		\varphi: \R^+_* \times ]0,2\pi[ &\longrightarrow \R^2\setminus \left( R^+_* \times \{0\} \right) \\
		(r, \theta) &\longmapsto (r \cos \theta, r \sin\theta),
	\end{align*}
	\begin{align*}
		f: \R^2\setminus \left( R^+_* \times \{0\} \right) &\longrightarrow \R \\
		(x,y) &\longmapsto f(x,y),
	\end{align*}
	\begin{align*}
		g: \overbrace{\R^+_* \times ]0, 2\pi[}^{=V} &\longrightarrow \R \\
		(r, \theta) &\longmapsto f(r\cos\theta, r\sin\theta).
	\end{align*}

	\begin{align*}
		\forall (r_0,\theta_0) \in V,&\\[5mm]
		\frac{\partial g}{\partial r}(r_0, \theta_0) &= \frac{\partial f}{\partial x}(r_0\cos\theta_0, r_0\sin\theta_0)\cos\theta_0\\
		&+ \frac{\partial f}{\partial y}(r_0 \cos\theta_0, r_0\sin\theta_0)\sin\theta_0\\
		&= 2r_0\cos^2\theta_0 + 2r_0\sin^2(\theta_0) \\
		&= 2r_0 \\[5mm]
		\frac{\partial g}{\partial \theta}(r_0, \theta_0) &= \frac{\partial f}{\partial x}(r_0\cos\theta_0, r_0\sin\theta_0)r_0\sin\theta_0\\
		&+ \frac{\partial f}{\partial y}(r_0 \cos\theta_0, r_0\sin\theta_0)r_0\cos\theta_0\\
		&= -2{r_0}^2\cos(\theta_0)\sin(\theta_0) + 2{r_0}^2 \sin(\theta_0)\cos(\theta_0)\\
		&= 0 \\
	\end{align*}

	Donc, \[
		g(r, \theta) = r^2.
	\]
\end{exm}

\begin{exm}
	Résoudre \[
		\begin{cases}
			\frac{\partial f}{\partial x} = \frac{x}{x^2+y^2},\\
			\frac{\partial f}{\partial y} = \frac{y}{x^2+y^2}.\\
		\end{cases}
	\]

	On pose $g: (r, \theta) \mapsto f(r \cos\theta, r \sin\theta)$.

	\begin{align*}
		&\frac{\partial g}{\partial r} = \frac{1}{r}\cos^2\theta + \frac{1}{r}\sin^2\theta = \frac{1}{r},\\
		&\frac{\partial g}{\partial \theta} = -\cos(\theta) \sin(\theta) + \sin(\theta)\cos(\theta) = 0.
	\end{align*}

	Donc, \[
		\exists C \in \R, g: (r, \theta) \mapsto \ln r + C
	\] d'où,
	\begin{align*}
		\forall (x,y) \in \R^2 \setminus \{(0,0)\}, f(x,y) &= \ln\left(\sqrt{x^2 + y^2} \right)  + C\\
		&= \frac{1}{2}\ln(x^2 + y^2) + C. \\
	\end{align*}
\end{exm}

\begin{rmk}
	Soit $\mathcal{B} = (e_1, e_2)$ la base canonique de $\R^2$, $f: U \to \R$ de classe $\mathcal{C}^1$ avec $U$ un ouvert de $\R^2$.

	Soit $(x,y) \in U$.

	\begin{align*}
		\Mat_{\mathcal{B}}\big(\nabla f(x,y)\big) = \begin{pmatrix}
			\frac{\partial f}{\partial x}(x,y)\\[2mm]
			\frac{\partial f}{\partial y}(x,y)
		\end{pmatrix}
	\end{align*}

	Soit  \begin{align*}
		\varphi: V &\longrightarrow U \\
		(u,v) &\longmapsto \big(x(u,v), y(u,v)\big) 
	\end{align*} avec $x,y$ de classe $\mathcal{C}^1$. Soit $g = f \circ \varphi$.
	\begin{align*}
		\Mat_{\mathcal{B}}\big(\nabla g(u,v)\big)
		&= \begin{pmatrix}
			\frac{\partial g}{\partial u}(u,v) \\[2mm]
			\frac{\partial g}{\partial v}(u,v)
		\end{pmatrix} \\
		&= \begin{pmatrix}
			\frac{\partial x}{\partial u}(u,v) \frac{\partial f}{\partial x}(x,y)
			+ \frac{\partial y}{\partial u}(u,v)\frac{\partial f}{\partial y}(x,y)\\[3mm]
			\frac{\partial x}{\partial v}(u,v) \frac{\partial f}{\partial x}(x,y)
			+ \frac{\partial y}{\partial v}(u,v) \frac{\partial f}{\partial y}(x,y)
		\end{pmatrix}  \\
		&= \underbrace{\begin{pmatrix}
				\frac{\partial x}{\partial u}(u,v)& \frac{\partial y}{\partial u}(u,v)\\[3mm]
				\frac{\partial x}{\partial v}(u,v)& \frac{\partial y}{\partial v}(u,v)
		\end{pmatrix}}_{J(u,v)} \begin{pmatrix}
			\frac{\partial f}{\partial x}(x,y)\\[3mm]
			\frac{\partial f}{\partial y}(x,y)
		\end{pmatrix} \\
		&= J(u,v) \Mat_{\mathcal{B}}\big(\nabla f(x,y)\big) \\
	\end{align*}
	où $J(u,v) = 
	\begin{pNiceArray}{c:c}
		\Mat_{\mathcal{B}}\big(\nabla x(u,v)\big) & \Mat_{\mathcal{B}}\big(\nabla y(u,v)\big)
	\end{pNiceArray}$.

	On dit que $J(u,v)$ est \underline{la jacobienne} de $\varphi$ en $(u,v)$.
	L'application linéaire canoniquement associée à $J(u,v)$ est la \underline{différentielle de $\varphi$} en $(u,v)$ noté $\mathrm{d}\varphi(u,v)$.

	On a $\mathrm{d}\varphi(u,v) \in \mathcal{L}(R^2)$ et $\Mat_{\mathcal{B}}\big(\mathrm{d}\varphi(u,v)\big) = J(u,v)$.

	Par exemple, la jacobienne du changement de coordonnées polaires est \[
		J = \begin{pmatrix}
			\frac{\partial x}{\partial r} & \frac{\partial y}{\partial r}\\[3mm]
			\frac{\partial x}{\partial \theta} & \frac{\partial y}{\partial \theta}
		\end{pmatrix}
		= \begin{pmatrix}
			\cos\theta&\sin\theta\\
			-r\sin\theta&r\cos\theta
		\end{pmatrix}.
	\]
	$\underbrace{\det(J)}_{\text{le jacobien}} = r\cos^2\theta + r\sin^2\theta = r$

	Dans une intégrale double, si $(x,y) = \varphi(u,v)$, alors $\mathrm{d}x\mathrm{d}y = \det(J)\mathrm{d}u\mathrm{d}v$.

	Ici, \[
		\mathrm{d}x\ \mathrm{d}y = r\ \mathrm{d}r\ \mathrm{d}\theta.
	\]
\end{rmk}

\begin{prv}
	On pose $(x_0, y_0) = \varphi(u_0, v_0)$. Pour tout $(h,k) \in \R^2$ tels que $(u_0 + h, v_0 + k) \in V$, en posant $g = f  \circ \varphi$.

	\begin{align*}
		g(u_0 + h, v_0 + h) &= f\big(x(u_0 + h, v_0 + k), y(u_0 + h, v_0 + k)\big) \\
		&= f\left(
			x(u_0,v_0) + h \frac{\partial x}{\partial u}(u_0,v_0) + k \frac{\partial x}{\partial v}(u_0, v_0) + \po\big(\|(h,k)\|\big), \right.\\
		&\phantom{ = f\bigg(\bigg.}\left. y(u_0, v_0) + h \frac{\partial y}{\partial u}(u_0, v_0) + k \frac{\partial y}{\partial v}(u_0, v_0) + \po\big(\|(h,k)\|\big)
		\right)  \\
		&= f(x_0,y_0) \\
		&~+ \left( h \frac{\partial x}{\partial u}(u_0,v_0) + k \frac{\partial x}{\partial v}(u_0, v_0) + \po(\|(h,k)\|) \right) \frac{\partial f}{\partial x}(x_0,y_0)\\
		&~+ \left( h \frac{\partial y}{\partial u}(u_0, v_0) + k\frac{\partial y}{\partial v}(u_0, v_0) + \po(\|(h,k)\|) \right) \frac{\partial f}{\partial y}(x_0, y_0)\\
		&~+ \po(\|(h,k)\|)\\
		&= f(x_0, y_0) \\
		&~+ h \left( \frac{\partial x}{\partial u}(u_0, v_0) \frac{\partial f}{\partial x}(x_0, y_0) + \frac{\partial y}{\partial u}(u_0, v_0) \frac{\partial f}{\partial y}(x_0, y_0) \right)  \\
		&~+ k\left( \frac{\partial x}{\partial v}(u_0, v_0) \frac{\partial f}{\partial x}(x_0, y_0) + \frac{\partial y}{\partial v}(u_0, v_0) \frac{\partial f}{\partial y}(x_0, y_0) \right) 
		&~+ \po(\|(h,k)\|)\\
		&= g(u_0, v_0) + h \frac{\partial g}{\partial u}(u_0, v_0) + k \frac{\partial g}{\partial v}(u_0, v_0) + \po(\|(h,k)\|) \\
	\end{align*}

	Par identification,
	\[
		\frac{\partial g}{\partial u}(u_0, v_0) = \frac{\partial x}{\partial u}(u_0, v_0) \frac{\partial f}{\partial x}(x_0, y_0) + \frac{\partial y}{\partial u}(u_0, v_0) \frac{\partial f}{\partial y}(x_0,y_0)
	\] et \[
		\frac{\partial g}{\partial v}(u_0, v_0) = \frac{\partial x}{\partial v}(u_0,v_0) \frac{\partial f}{\partial x}(x_0, y_0) + \frac{\partial y}{\partial v}(u_0, v_0) \frac{\partial f}{\partial y}(x_0, y_0).
	\] 
\end{prv}

\begin{exm}
	[Régression linéaire]~\\
	\begin{figure}[H]
		\centering
		\begin{asy}
			import graph;
			axes(EndArrow);
			size(5cm);

			real f(real x) { return x + 0.5; }

			real k = 35 / (7 - 0.5);

			for(int i = 0; i < 35; ++i) {
				real mag = exp(sin(100 * pi/exp(1) * i)) * 0.8 + exp(cos(i*40)/3);
				real eps = mag * cos(10 * exp(1)/pi * i) / 3;
				dot((i/k,f(i/k) + eps));
			}

			draw(graph(f, -1, 7), orange);
		\end{asy}
	\end{figure}
	\[
		y = a x + b
	\] 
	On fixe $(a,b) \in \R^2$. \[
		\varepsilon(a,b) = \sum_{i=1}^n\big( y_i - (ax_i + b) \big)^2
	\] l'erreur totale.

	On veut minimiser $\varepsilon(a,b)$. On a 
	\[
		\forall (a,b) \in \R^2,
		\begin{cases}
			\frac{\partial \varepsilon}{\partial a}(a,b) = -2\sum_{i=1}^{n}(y_i - ax_i - b)x_i,\\
			\frac{\partial \varepsilon}{\partial b}(a,b) = -2\sum_{i=1}^{n}(y_i - ax_i - b).
		\end{cases}
	\]

	Donc,
	\begin{align*}
		(a,b) \text{ point critique de } \varepsilon \iff& \begin{cases}
			a \sum_{i=1}^n {x_i}^2 + b\sum_{i=1}^{n}x_i = \sum_{i=1}^{n} y_ix_i\\
			a\sum_{i=1}^{n}x_i + nb = \sum_{i=1}^ny_i
		\end{cases}\\
		\iff& \begin{cases}
			a \left( \frac{1}{n}\sum_{i=1}^n {x_i}^2 - \overline{x}^2\right) = \overline{y} - \overline{x} \overline{y}\\
			b = \frac{1}{n}\sum_{i=1}^ny_i - \frac{a}{n}\sum_{i=1}^nx_i = \frac{1}{n}\sum_{i=1}^n x_i y_i - \overline{x} \overline{y}
		\end{cases}\\
		&\text{ où } \overline{x} = \frac{1}{n} \sum_{i=1}^n x_i,~\overline{y} = \frac{1}{n}\sum_{i=1}^n y_i\\
		\iff& \begin{cases}
			a = \frac{\Cov(x,y)}{V(x)}\\
			b = \overline{y} - a\overline{x}
		\end{cases}
	\end{align*}

	Coefficient de corrélation: $\frac{\Cov(x,y)}{\sigma_x \sigma_y} \in [-1, 1]$
\end{exm}












		\part{Corps}

\begin{exm}[Problème]
	\begin{itemize}
		\item 
			avec $A = \Z / 9 \Z$, résoudre $\overline{x}^2 = \overline{0}$ \\
			\begin{center}
				\begin{tabular}{|c|c|c|c|c|c|c|c|c|c|c|}
					\hline
					$\overline{x}$&$\overline{0}$& $\overline{1}$ &$\overline{2}$&$\overline{3}$ &$\overline{4}$ &$\overline{5}$ &$\overline{6}$ &$\overline{7}$ &$\overline{8}$& $\overline{9}$ \\
					\hline
					$\overline{x}^2$&$\overline{0}$ &$\overline{1}$ &$\overline{4}$ &$\overline{0}$ &$\overline{7}$ &$7$ &$\overline{0}$ &$\overline{4}$ &$\overline{1}$&$\overline{0}$\\
					\hline
				\end{tabular}
			\end{center}
			On a trouvé 3 solutions: $\overline{0}$, $\overline{3}$, $\overline{6}$.
		\item $\Z / 8\Z$
			\begin{center}
				\begin{tabular}{|c|c|c|c|c|c|c|c|c|}
					\hline
					$\overline{x}$& $\overline{0}$& $\overline{1}$& $\overline{2}$& $\overline{3}$& $\overline{4}$& $\overline{5}$& $\overline{6}$& $\overline{7}$\\
					\hline
					$\overline{x^2}$& $\overline{0}$& $\overline{1}$& $\overline{4}$& $\overline{1}$& $\overline{0}$& $\overline{1}$& $\overline{4}$& $\overline{1}$\\
					\hline
				\end{tabular}
			\end{center}
			$\overline{x}^2=7$ a 4 solutions: $\overline{1}, \overline{7}, \overline{3},\text{ et } \overline{5}$
		\item $A = \mathbbm{H} = \{a + bi + cj + dk  \mid  (a,b,c,d) \in \R^4\}$ \\
			$i^2 = j^2 = k^2 = -1$ 
			\begin{align*}
				\begin{array}{c c c}
					ij = k & jk = i & ji = j\\
					ji = -k & kj = -i & ik = -j
				\end{array}
			\end{align*}
			Dans cet anneau, $-1$ a 6 racines!
	\end{itemize}
\end{exm}

\begin{defn}
	Soit $(\mathbbm{K}, +, \times)$ un ensemble muni de deux lois de composition internes. On dit que c'est un \underline{corps} si
	 \begin{enumerate}
		\item $(\mathbbm{K}, \times)$ est un groupe abélien
		\item $(\mathbbm{K}, \times)$ est un monoïde commutatif
		\item $\forall x \in \mathbbm{K}\setminus \{0_\mathbbm{K}\}, \exists y \in \mathbbm{K}, xy = 1_\mathbbm{K}$
		\item $0_\mathbbm{K} \neq  1_\mathbbm{K}$
	\end{enumerate}
	\index{corps}
\end{defn}

\begin{exm}
	\begin{itemize}
		\item $(\C, +, \times)$ est un corps
		\item $(\R, +, \times)$ est un corps
		\item $(\Q, +, \times)$ est un corps
		\item $(\Z, +, \times)$ n'est pas un corps
	\end{itemize}
\end{exm}

\begin{prop}
	$(\Z / n\Z, +, \times)$ est un corps si et seulement si $n$ est premier.
\end{prop}

\begin{prv}
	\[
		\left( \Z / n\Z \right)^\times = \left\{ \overline{k}  \mid k \wedge n = 1 \right\}
	\] 
\end{prv}


\begin{prop}
	Tout corps est un anneau intègre.
\end{prop}

\begin{prv}
	Soit $(\mathbbm{K}, +, \times)$ un corps. Soient $(a,b) \in \mathbbm{K}^2$ tel que $a \times b = 0_\mathbbm{K}$.\\
	On suppose $a \neq  0_\mathbbm{K}$. Alors, $a$ est inversible et donc \[
		b = a^{-1} \times a \times b = a^{-1} \times 0_\mathbbm{K} = 0_\mathbbm{K}
	\] 
\end{prv}

\begin{exm}
	Soit $(\mathbbm{K},+,\times)$ un corps.\\
	Résoudre \[
		\begin{cases}
			x^2 = 1_\mathbbm{K}\\
			x \in \mathbbm{K}
		\end{cases}
	\]

	\begin{align*}
		x^2 = 1_\mathbbm{K} &\iff x^2 - 1_\mathbbm{K} = 0_\mathbbm{K}\\
		&\iff (x - 1_\mathbbm{K})(x+1_\mathbbm{K}) = 0_\mathbbm{K}\\
		&\iff x - 1_\mathbbm{K} = 0_\mathbbm{K} \text{ ou } x + 1_\mathbbm{K} = 0_\mathbbm{K}\\
		&\iff x = 1_\mathbbm{K} \text{ ou } x = -1_\mathbbm{K}
	\end{align*}

	Il y a au plus 2 solutions.
\end{exm}

\begin{prop}
	Soit $(\mathbbm{K},+,\times )$ un corps et $P$ un polynôme à coefficients dans $\mathbbm{K}$ de degré $n$. Alors, l'équation $P(x) = 0_{\mathbbm{K}}$ a au plus $n$ solutions dans $\mathbbm{K}$ 
	\qed
\end{prop}

\begin{crlr}[(Théorème de Wilson)]
	voir exercice 16 du TD 12
\end{crlr}


\begin{defn}
	Soit $(\mathbbm{K}, +, \times)$ un corps et $L\subset \mathbbm{K}$.\\
	On dit que $L$ est un \underline{sous corps} de $\mathbbm{K}$ si
	\begin{enumerate}
		\item $L$ est un anneau de $(\mathbbm{K}, +, \times)$ non nul
		\item $\forall x \in L\setminus \{0_\mathbbm{K}\}, x^{-1} \in L$ 
	\end{enumerate}
	\vspace{2mm}
	en d'autres termes si
	\begin{enumerate}
		\item $\forall (x,y) \in L^2, x - y \in L$
		\item $\forall (x,y) \in L^2, x \times y^{-1} \in L$
	\end{enumerate}
	\vspace{5mm}
	On dit aussi que $\mathbbm{K}$ est une \underline{extension} de $L$.
	\index{sous corps}
	\index{extension}
\end{defn}

\begin{prop}
	Tout sous corps est un corps. \qed
\end{prop}

\begin{defn}
	Soient $(\mathbbm{K}_1,+,\times )$ et $(\mathbbm{K}_2,+, \times)$ deux corps et $f: \mathbbm{K}_1 \to \mathbbm{K}_2$.\\
	On dit que $f$ est un \underline{morphisme de corps} si $f$ est un morphisme d'anneaux.\\
	i.e. si
	\[
		\begin{cases}
			\forall (x,y) \in {\mathbbm{K}_1}^2,& f(x+y) = f(x) + f(y)\\
			\forall (x,y) \in {\mathbbm{K}_1}^2,& f(x \times y) = f(x) \times f(y)\\
		\end{cases}
	\] 
	\index{homomorphisme (de corps)}
	\index{morphisme (de corps)}
\end{defn}

\begin{prop}
	Tout morphisme de corps est injectif.
\end{prop}

\begin{prv}
	Soit $f: \mathbbm{K}_1 \to \mathbbm{K}_2$ un morphisme de corps.\\
	\begin{itemize}
		\item $\Ker(f)$ est un sous groupe de $(\mathbbm{K}_1, +)$ 
		\item Soit $x \in \Ker(f)$ et $y \in \mathbbm{K}_1$ \[
				f(x \times y) = f(x) \times f(y) = 0_{\mathbbm{K}_2} \times f(y) = 0_{\mathbbm{K}_2}
			\]
		\item Soit $x \in \Ker(f) \setminus \{0_{\mathbbm{K}_1}\}$.\\
			Alors, $x$ est inversible.\\
			\begin{align*}
				\begin{rcases*}
					x \in \Ker(f)\\
					x^{-1} \in \mathbbm{K}_1
				\end{rcases*}& \text{ donc } x \times x ^{-1} \in \Ker(f)\\
				&\text{ donc } 1_{\mathbbm{K}_1} \in \Ker(f)\\
				&\text{ donc } f(1_{\mathbbm{K}_1}) = 0_{\mathbbm{K}_2}
			\end{align*}
			Or, $f(1_{\mathbbm{K}_1}) = 1_{\mathbbm{K}_2} \neq 0_{\mathbbm{K}_2}$
	\end{itemize}
	Donc, $\Ker(f) = \{0_{\mathbbm{K}_1}\}$ donc $f$ est injective.
\end{prv}

\begin{exm}
	$\begin{array}{cc}
		\C &\longrightarrow \C\\
		z &\longmapsto \overline{z}\\
	\end{array}$ est un morphisme de corps
\end{exm}



		\part{Opérations sur les séries}

\begin{prop}
	L'ensemble $E = \{u \in \C^\N  \mid \Sigma u_n \text{ converge}\}$ est un sous-espace vectoriel de $\C^\N$ et \begin{align*}
		S: E &\longrightarrow \C \\
		u &\longmapsto \sum_{n=0}^{+\infty} u_n
	\end{align*} est une forme linéaire.
	\qed
\end{prop}

\begin{rmk}
	La somme d'une série convergente et d'une série divergente diverge.
	Le produit d'une série divergente par un scalaire non nul diverge.
\end{rmk}

	}

	{
		\chap[19]{Applications linéaires}
		\renewcommand{\cwd}{../chap19}
		\begin{defn}
	Soit $E$ un $\mathbbm{K}$-espace vectoriel. On dit que $E$ est de \underline{dimension finie} si $E$ a au moins une famille génératrice finie. On dit que $E$ est de \underline{dimension infinie} sinon.
	\index{dimension finie (espace vectoriel)}
	\index{dimension infinie (espace vectoriel)}
\end{defn}

\begin{thm}
	[Théorème de la base extraite]
	Soit $E$ un $\mathbbm{K}$-espace vectoriel non nul de dimension finie. Soit $\mathcal{G}$ une famille génératrice finie de $E$. Alors, il existe une base $\mathcal{B}$ de $\mathcal{E}$ telle que $\mathcal{B} \subset \mathcal{G}$.
\end{thm}

\begin{prv}
	[par récurrence sur $\#G = \Card(G)$]
	\begin{itemize}
		\item Soit $E$ un $\mathbbm{K}$-espace vectoriel non nul engendré par $\mathcal{G} = (u)$.\\
			Si $u = 0_E$, alors $E = \{0_E\}$: une contradiction $\lightning$ \\
			Donc $u \neq 0_E$ donc $(u)$ est libre. En effet, \[
				\forall \lambda \in \mathbbm{K}, \lambda u = 0_E \implies \lambda = 0_\mathbbm{K}
			\] Donc $\mathcal{G}$ est une base de $E$.\\
		\item Soit $n \in \N_*$. Soit $E$ un $\mathbbm{K}$-espace vectoriel. On suppose que si $E$ a une famille génératrice constituée de $n$ vecteurs, alors on peut extraire de cette famille une base de $E$.\\
			Soit $\mathcal{G}$ une famille génératrice de $E$ avec $n+1$ vecteurs.\\
			Si $\mathcal{G}$ est libre, alors $\mathcal{G}$ est une base de $E$. \\
			Si $\mathcal{G}$ n'est pas libre, alors il existe $u \in \mathcal{G}$ tel que $u \in \Vect(\mathcal{G}\setminus \{u\})$ \\
			Donc $\mathcal{G}\setminus \{u\}$ engendre $E$. Or, $\mathcal{G}\setminus \{u\}$ possède $n$ vecteurs. D'après l'hypothèse de récurrence, il existe une base $\mathcal{B}$ de $E$ telle que \[
				\mathcal{B} \subset \mathcal{G} \setminus \{u\} \subset \mathcal{G}
			\] 
	\end{itemize}
\end{prv}

\begin{crlr}
	Tout espace de dimension finie a une base.
	\qed
\end{crlr}

\begin{thm}
	[Théorème de la base incomplète]
	Soit $E$ un $\mathbbm{K}$-espace vectoriel de dimension finie, $\mathcal{G}$ une famille génératrice finie de $E$. $\mathcal{L}$ une famille libre de $E$. Alors, il existe une base $\mathcal{B}$ de $E$ telle que \[
		\mathcal{L} \subset \mathcal{B} \text{ et } \mathcal{B}\setminus \mathcal{L} \subset \mathcal{G}
	\] 
\end{thm}

\begin{prv}
	[par récurrence sur $\#(\mathcal{G}\setminus\mathcal{L})$]
	\begin{itemize}
		\item Avec les notations précédentes, on suppose que $\mathcal{G}\setminus\mathcal{L} \neq \O$ \[
				\forall u \in \mathcal{G}, u \in \mathcal{L}
			\] Donc $\mathcal{G} \subset \mathcal{L}$ donc $\mathcal{L}$ est génératrice donc $\mathcal{L}$ est une base de $E$. On pose $\mathcal{B} = \mathcal{L}$ et alors \[
				\mathcal{L} \subset  \mathcal{B} \text{ et } \mathcal{B}\setminus\mathcal{L} = \O \subset  \mathcal{G}
			\] 
		\item Soit $n \in \N$. On suppose que si $\mathcal{G}$ est génératrice et $\mathcal{L}$ libre avec $\#(\mathcal{G}\setminus\mathcal{L}) = n$ alors il existe une base $\mathcal{B}$ de $E$ telle que \[
			\mathcal{L}\subset \mathcal{B} \text{ et } \mathcal{B}\setminus\mathcal{L}\subset \mathcal{G}
		\] Soient à présent $\mathcal{G}$ une famille génératrice de $E$ et $\mathcal{L}$ une famille libre de $E$ telles que $\#(\mathcal{G}\setminus\mathcal{L}) = n+1 > 0$\\
		Si $\mathcal{L}$ engendre $E$, alors $\mathcal{L}$ est une base de $E$. On pose $\mathcal{B} = \mathcal{L}$ et on a bien \[
			\mathcal{L} \subset  \mathcal{B} \text{ et } \mathcal{B} \setminus \mathcal{L} = \O \subset  \mathcal{G}
		\] On suppose que $\mathcal{L}$ n'engendre pas $E$. Il existe $u \in \mathcal{G}$ tel que $u \not\in \Vec(\mathcal{L})$ (car sinon, $\mathcal{G} \subset \Vect(\mathcal{L})$ et donc $\underbrace{\Vect(\mathcal{G})}_{= E} \subset  \underbrace{\Vect(\mathcal{L})}_{ \subset E}$\\
		Donc $\mathcal{L} \cup \{u\} $ est libre. On pose $\mathcal{L}' = \mathcal{L} \cup \{u\} $ \[
			\mathcal{G}\setminus \mathcal{L}' = \mathcal{G}\setminus (\mathcal{L} \cup \{u\}) = (\mathcal{G}\setminus\mathcal{L})\setminus \{u\} 
		\] donc $\#(\mathcal{G}\setminus\mathcal{L}') = n+1 -1 = n$\\
		D'après l'hypothèse de récurrence, il existe $\mathcal{B}$ une base de $E$ telle que \[
			\mathcal{L} \subset  \mathcal{L}' \subset \mathcal{B} \text{ et } \mathcal{B}\setminus \mathcal{L}' \subset \mathcal{G}
		\] \[
			\mathcal{B} \setminus \mathcal{L} = \underbrace{\mathcal{B}\setminus\mathcal{L}'}_{\subset \mathcal{G}} \cup \underbrace{\{u\}}_{\subset \mathcal{G} \text{ car } u \in \mathcal{G}}
		\] On a $\mathcal{B}\setminus\mathcal{L}\subset \mathcal{G}$
	\end{itemize}
\end{prv}

\begin{thm}
	Soit $E$ un $\mathbbm{K}$-espace vectoriel de dimension finie. Toutes les bases de $E$ ont le même cardinal.
\end{thm}

\begin{prv}
	Soit $\mathcal{G}$ une famille génératrice finie de $E$ et $\mathcal{B} \subset  \mathcal{G}$ une base de $E$. On note $n = \#\mathcal{B}$ \\
	Soit $\mathcal{B}'$ une base de $E$. On pose $p = n - \#(\mathcal{B} \cap  \mathcal{B}')$. Montrons par récurrence sur  $p$ que $\#\mathcal{B} = \#\mathcal{B}'$ 
	\begin{itemize}
		\item On suppose que $p = 0$. Alors, $\#(\mathcal{B} \cap \mathcal{B}') = n$ \\
			Or, $\mathcal{B}' \cap \mathcal{B} \subset \mathcal{B}$ donc $\mathcal{B} \cap \mathcal{B}' = \mathcal{B}$ donc $\mathcal{B} \subset  \mathcal{B}'$ et donc $\mathcal{B} = \mathcal{B}'$ 
		\item Soit $p \in \N$. On suppose que si $\mathcal{B}'$ est une base de $E$ telle que $n - \#(\mathcal{B} \cap \mathcal{B}') = p$, alors $\#\mathcal{B}' = n$ \\
			Aoit $\mathcal{B}'$ une base de $E$ telle que $n - \#(\mathcal{B}\cap \mathcal{B}') = p+1 > 0$ \\
			Donc $\mathcal{B} \cap \mathcal{B}' \neq \mathcal{B}$. Soit $u \in \mathcal{B}' \setminus \mathcal{B}$. D'après le lemme d'échange, il existe $v \in \mathcal{B}\setminus \mathcal{B}'$ tel que $\mathcal{B}' \setminus \{u\} \cup \{v\}$ est une base de $E$. On pose $\mathcal{B}'' = \mathcal{B}' \setminus \{u\} \cup \{v\}$ 
			\begin{align*}
				\mathcal{B}'' \cap \mathcal{B} &= \left( (\mathcal{B}' \setminus \{u\})  \cap \mathcal{B} \right) \cup \{v\} \\
				&= (\mathcal{B}' \cap \mathcal{B}) \cup \{v\} \\
			\end{align*}
			donc,
			\begin{align*}
				n - \#(\mathcal{B}'' \cap \mathcal{B}) &= n - (\#(\mathcal{B}' \cap \mathcal{B}) + 1) \\
				&= p+1- 1 \\
				&= p \\
			\end{align*}
			D'après l'hypothèse de récurrence, \[
				\#\mathcal{B}'' = n
			\] Or, $\#\mathcal{B}'' = \#\mathcal{B}'$
	\end{itemize}
\end{prv}

\begin{lem}
	Soient $\mathcal{B}$ et $\mathcal{B}'$ deux bases de $E$ telles que $\mathcal{B}\subset \mathcal{B}'$. Alors, $\mathcal{B} = \mathcal{B}'$.
\end{lem}

\begin{prv}
	On suppose $\mathcal{B}' \neq \mathcal{B}$. Soit $u \in \mathcal{B}' \setminus \mathcal{B}$
	$u \in E = \Vect(\mathcal{B})$ donc $\mathcal{B} \cup \{u\}$ n'est pas libre.
	Donc $\mathcal{B}\cup \{u\} \subset \mathcal{B}'$ et $\mathcal{B}'$ est libre donc $\mathcal{B}\cup \{u\}$ est libre: une contradiction $\lightning$
\end{prv}

\begin{lem}
	[Lemme d'échange] Soient $\mathcal{B}_1$ et $\mathcal{B}_2$ deux bases de $E$ et $u \in \mathcal{B}_1 \setminus \mathcal{B}_2$. Alors, il existe $v \in \mathcal{B}_2$ tel que $(\mathcal{B}_1 \setminus \{u\}) \cup \{v\}$ soit une base de $E$.
\end{lem}

\begin{prv}
	[1${}^\text{nde}$ méthode]
	On suppose que pout tout $v \in \mathcal{B}_2$, $(\mathcal{B}_1\setminus \{u\}) \cup \{v\}$ n'est pas une base de $E$
	Soit $v \in \mathcal{B}_2$.
	\begin{itemize}
		\item Supposons $(\mathcal{B}_1\setminus \{u\})\cup \{v\}$ non libre. $\mathcal{B}_1 \setminus \{u\}$ est libre. Donc $v \in \Vect(\mathcal{B}_1 \setminus \{u\})$
		\item Supposons $(\mathcal{B}_1\setminus \{u\}) \cup \{v\}$ non génératrice.
			Comme $\mathcal{B}_1$ engendre $E$, $u \not\in \Vect(\mathcal{B}_1\setminus \{v\})$.
			On suppose que $\mathcal{B}_1 \neq \mathcal{B}_2$.
			$\forall v \in \mathcal{B}_2 \setminus \mathcal{B}_1, \Vect(\mathcal{B}_1 \setminus \{v\}) = \Vect(\mathcal{B}_1) = E \ni u$ 
			donc, $(\mathcal{B}_1\setminus \{u\}) \cup \{v\}$ engendre $E$ et donc \[
				v \in \Vect(\mathcal{B}_1 \setminus \{u\})
			\] On a aussi \[
				\forall v \in \mathcal{B}_1 \setminus \{u\}, v \in \Vect(\mathcal{B}_1\setminus \{u\})
			\] Comme $u \not\in \mathcal{B}_2$, on a \[
				\forall v \in \mathcal{B}_2, v \in \Vect(\mathcal{B}_1\setminus \{u\})
			\] docn \[
				E = \Vect(\mathcal{B}_2) \subset \Vect(\mathcal{B}_1\setminus \{u\})
			\] donc $\mathcal{B}_1\setminus \{u\}$ engendre $E$ donc $\mathcal{B}_1\setminus \{u\}$ est une base de $E$. Or, $\mathcal{B}_1 \setminus \{u\}  \subset  \mathcal{B}_1$, donc $\mathcal{B}_1\setminus \{u\} = \mathcal{B}_1$
	\end{itemize}
\end{prv}

\begin{prv}
	[2${}^\text{nde}$ méthode]
	On suppose que pout tout $v \in \mathcal{B}_2$, $(\mathcal{B}_1\setminus \{u\}) \cup \{v\}$ n'est pas une base de $E$
	\begin{itemize}
		\item Comme $u \in \mathcal{B}_1 \setminus \mathcal{B}_2$, nécéssairement $\mathcal{B}_1 \neq \mathcal{B}_2$ donc $\mathcal{B}_2 \not\subset \mathcal{B}_1$, donc $\mathcal{B}_2\setminus\mathcal{B}_1 \neq \O$ 
		\item Soit $v \in \mathcal{B}_2\setminus\mathcal{B}_1$. Il existe $(\lambda_w)_{w\in\mathcal{B}_1}$ une famille de scalaires presque nulle telle que \[
				v = \sum_{w \in \mathcal{B}_1} \lambda_w w - \lambda_u u + + \sum_{w \in \mathcal{B}_1\setminus \{u\}}\lambda_w w
			\]
			Si $\lambda_u \neq 0_E$, alors
			\begin{align*}
				u &= \lambda_u^{-1}\left( v - \sum_{w \in \mathcal{B}_1 \setminus \{u\}} \lambda_w w \right)\\
					&\in \Vect(\mathcal{B}_1\setminus \{u\} \cup v)
			\end{align*}
			 donc $\mathcal{B}_1 \subset \Vect(\mathcal{B}_1\setminus \{u\} \cup \{v\})$\\
			 et donc $E \subset  \Vect(\mathcal{B}_1 \setminus \{u\} \cup \{v\})$ \\
			 et donc $\mathcal{B}_1 \setminus \{u\} \cup \{v\}$ engendre $E$ \\
			 donc $\mathcal{B}_1 \setminus \{u\} \cup \{v\}$ n'est pas libre\\
			 donc $v \in \Vect(\mathcal{B}_1\setminus \{u\})$ (car $\mathcal{B}_1 \setminus \{u\}$ est libre\\
			 donc $\lambda_u = 0_\mathbbm{K}$ $\lightning$\\`

			 Donc, $\lambda_u = 0_\mathbbm{K}$, docn $v \in \Vect(\mathcal{B}_1\setminus \{u\})$ \\
			 On vient de prouver que
			 \begin{align*}
			 	\mathcal{B}_2 \setminus \mathcal{B}_1 \subset \Vect(\mathcal{B}_1 \setminus \{u\})\\
			 	\mathcal{B}_1 \setminus \{u\} \subset \Vect(\mathcal{B}_1 \setminus \{u\})\\
			 \end{align*}
			 Comme $u \not\in \mathcal{B}_2$, \[
			 	\mathcal{B}_2 \subset \Vect(\mathcal{B}_1 \setminus \{u\})
			 \] donc \[
			 	E = \Vect(\mathcal{B}_2) \subset  \Vect(\mathcal{B}_1 \setminus \{u\})
			 \] donc $\mathcal{B}_1 \setminus \{u\}$ engendre $E$. Donc,  $\mathcal{B}_1 \setminus \{u\}$ est une base de $E$.\\
			 Or, $\mathcal{B}_1 \setminus \{u\} \subset  \mathcal{B}_1$, donc $\mathcal{B}_1 \setminus \{u\} = \mathcal{B}_1$
	\end{itemize}
\end{prv}

\begin{defn}
	Soit $E$ un $\mathbbm{K}$-espace vectoriel de dimension finie. Le cardinal commun à toutes les bases de $E$ est appelé \underline{dimension} de $E$ est notée $\dim(E)$ ou $\dim_\mathbbm{K}(E)$\\
	C'est donc aussi le nombre de coordonnées de n'importe quel vecteur dans n'importe quelle base.
	\index{dimension (espace vectoriel)}
\end{defn}

\begin{exm}
	\begin{enumerate}
		\item $\dim_\R(\C) = 2$ et $\dim_\C(\C) = 1$ 
		\item $\dim_\mathbbm{K}(\mathbbm{K}^{n}) = n$ 
		\item $\dim_{\mathbbm{K}}(\mathcal{M}_{n,p}(\mathbbm{K})) = np$
	\end{enumerate}
\end{exm}

\begin{crlr}
	Soit $E$ un $\mathbbm{K}$-espace vectoriel de dimension finie, $\mathcal{L}$ une famille libre de $E$, $\mathcal{G}$ une famille génératrice de $E$. On note $n = \dim(E)$
	\begin{enumerate}
		\item $\#\mathcal{G} \ge n$ et $(\#\mathcal{G} = n \implies \mathcal{G} \text{ est une base de } E$)
		\item $\#\mathcal{L} \le n$ et $(\#\mathcal{L} = n \implies \mathcal{L} \text{ est une base de } E$)
	\end{enumerate}
\end{crlr}

\begin{crlr}
	$\R^{\R}$ est de dimension infinie.
	$\forall i \in \N, e_i: x \mapsto x^i$\\
	$(e_i)_{i\in\N}$ est libre dans $\R^\R$
\end{crlr}

\begin{prop}
	Soient $E$ et $F$ deux $\mathbbm{K}$-espaces vectoriels de dimension finie. Alors $E\times F$ est de dimension finie et $\dim(E\times F) = \dim(E) + \dim(F)$
\end{prop}

\begin{prv}
	Soit $(e_1,\ldots, e_n)$ une base de $E$, $(f_1, \ldots, f_p)$ une base de $F$.
	On pose \[
		\left\{\begin{array}
			{r c l}
			u_1 &=& (e_1,0_F)\\
			u_2 &=& (e_2,0_F)\\
					&\vdots&\\
			u_n &=& (e_n,0_F)\\
			u_{n+1} &=& (0_E, f_1)\\
			u_{n+2} &=& (0_E, f_2)\\
					&\vdots&\\
			u_{n+p} &=& (0_E,f_p)\\
		\end{array}\right.
	\]
	Soit $(x,y) \in E\times F$. \[
		\begin{cases}
			\exists (x_1,\ldots,x_n)\in \mathbbm{K}^n, x = \sum_{i=1}^{n} x_ie_i
			\exists (y_1,\ldots,y_n)\in \mathbbm{K}^n, x = \sum_{j=1}^{p} y_jf_j
		\end{cases}
	\] 
	\begin{align*}
		(x,y) &= \left( \sum_{i=1}^{n} x_ie_i, \sum_{i=1}^{p} y_jf_j \right)  \\
		&= \sum_{i=1}^{n} x_i (e_i + 0_F) + \sum_{j=1}^{p} y_j (0_E, f_j) \\
		&= \sum_{i=1}^{n} x_i u_i + \sum_{j=1}^{p} y_j u_{n+j} \\
	\end{align*}
	Donc, $E\times F = \Vect(u_1, \ldots, u_{n+p})$ donc $E\times F$ est de dimension finie.\\
	Soit $(\lambda_1, \ldots, \lambda_{n+p}) \in \mathbbm{K}^{n+p}$ tel que \[
		(*): \quad \sum_{k=1}^{n+p} \lambda_ku_k = 0_{E\times F} = (0_E, 0_F)
	\]
	\begin{align*}
		(*) &\iff \sum_{k=1}^{n} \lambda_k (e_k, 0_F) + \sum_{k=n+1}^{p} \lambda_k(0_E, f_{k-n}) = (0_E, 0_F)\\
				&\iff \begin{cases}
					\sum_{k=1}^{n} \lambda_k e_k = 0_E\\
					\sum_{k=n+1}^{p} \lambda_k f_{k-n} = 0_F
				\end{cases}\\
				&\iff \begin{cases}
					\forall k \in \left\llbracket 1,n \right\rrbracket, \lambda_k = 0_\mathbbm{K} \qquad&(\text{car $(e_1,\ldots,e_n)$ est libre})\\
					\forall k \in \left\llbracket n+1,n+p \right\rrbracket, \lambda_k = 0_\mathbbm{K} \qquad&(\text{car $(f_1,\ldots,f_n)$ est libre})\\
				\end{cases}
	\end{align*}
	Donc $(u_1, \ldots, u_{n+p})$ est une base de $E\times F$. Donc, $\dim(E\times F) = n + p = \dim(E) + \dim(F)$
\end{prv}

\begin{rmk}
	[Convention]
	\[\dim\big(\{0_E\}\big) = 0\]
\end{rmk}

\begin{thm}
	Soit $E$ un $\mathbbm{K}$-espace vectoriel de dimension finie, $F$ un sous-espace vectoriel de $E$. Alors, $F$ est de dimension finie et  $\dim(F) \le \dim(E)$\\
	Si $\dim(F) = \dim(E)$, alors $F = E$
\end{thm}

\begin{prv}
	On considère \[
		A = \{k \in \N \mid \text{il existe une famille libre de $F$ à $k$ éléments}\} 
	\]
	On suppose $F \neq \{0_E\}$.
	\begin{itemize}
		\item Soit $u \in F\setminus \{0_E\}$. $(u)$ est libre donc $1 \in A$ et donc $A \neq \O$
		\item Soit $\mathcal{L}$ une famille libre de $F$. Alors, $\mathcal{L}$ est une famille libre de $E$ \\
			donc $\#\mathcal{L} \le \dim(E)$\\
			Donc $A$ est majorée par $\dim(E)$ \\
			On en déduit que $A$ a un plus grand élément $p$.
		\item Soit $\mathcal{L}$ une famille libre de $F$ avec $p$ éléments.\\
			Si $\mathcal{L}$ n'engendre pas $F$, alors il existe $u\in F$ tel que $u\not\in \Vect(\mathcal{L})$ et donc $\mathcal{L} \cup \{u\}$ est une famille libre de $F$, donc $p+1 \in A$ en contradiction avec la maximalité de $p$.\\
			Donc $\mathcal{L}$ est une base de $F$ donc $F$ est de dimension finie et $\dim(F) = p \le \dim(E)$\\
	\end{itemize}

	Soit $\mathcal{B}$ une base de $F$. Alors, $\mathcal{B}$ est aussi une famille de libre de de $E$. Donc $\#\mathcal{B} \le \dim(E)$ donc $\dim(F) = \dim(E)$ \\
	Si $\dim(F) = \dim(E)$, alors $\mathcal{B}$ est une base de $E$, et donc $F = \Vect(\mathcal{B}) = E$
\end{prv}

\begin{prop}
	[Formule de Grassmann]
	Soit $E$ un $\mathbbm{K}$-espace vectoriel de dimension finie, $F$ et $G$ deux sous-espace vectoriels de $E$. Alors, \[
		\dim(F+G) = \dim(F) + \dim(G) - \dim(F\cap G)
	\] 
\end{prop}

\begin{prv}
	Soit $(e_1, \ldots, e_p)$ une base de $F\cap G$. $(e_1,\ldots,e_p)$ est une famille libre de $F$.\\
	On complète $(e_1, \ldots, e_p)$ en une base $(e_1, \ldots, e_p, u_1, \ldots, u_q)$ de $F$.\\
	De même, on complète $(e_1, \ldots, e_p)$ en une base $(e_1, \ldots, e_p, v_1, \ldots, v_r)$ de $G$.\\
	On pose  $\mathcal{B} = (e_1, \ldots, e_p, u_1, \ldots, u_q, v_1, \ldots, v_r)$. Montrons que $\mathcal{B}$ est une base de $F+G$
	\begin{itemize}
		\item Soit $u \in F+G$ \\
			On pose $u = v+w$ avec $\begin{cases}
				v\in F\\
				w \in G
			\end{cases}$.\\
			On pose $v = \sum_{i=1}^p \lambda_i e_i + \sum_{i=1}^q \mu_i u_i$ avec $(\lambda_1, \ldots, \lambda_p, \mu_1, \ldots, \lambda_q) \in \mathbbm{K}^{p+q}$\\
			On pose aussi $w = \sum_{i = 1}^p \lambda'_ie_i + \sum_{j=1}^r \nu_j v_j$ avec $(\lambda_1',\ldots,\lambda_p', \nu_1, \ldots, \nu_r) \in \mathbbm{K}^{p+r}$\\
			D'où, \[
				u = \sum_{i=1}^p (\lambda_i + \lambda'_i)e_i + \sum_{j=1}^q \mu_j u_j + \sum_{k=1}^r \nu_k v_k \in \Vect(\mathcal{B})
			\]
		\item Soient $(\lambda_1, \ldots, \lambda_p, \mu_1, \ldots, \mu_q, \nu_1, \ldots, \nu_r) \in \mathbbm{K}^{p+q+r}$.\\
			On suppose \[
				(*)\quad \sum_{i=1}^{p}\lambda_ie_i + \sum_{j=1}^q\mu_ju_j + \sum_{k=1}^r \nu_k v_k = 0_E
			\] 
			D'où, \[
				\underbrace{\sum_{i=1}^p\lambda_i e_i + \sum_{j=1}^q \mu_ju_j}_{\in F} = \underbrace{-\sum_{k=1}^r\nu_jv_k}_{\in G}
			\] 
			Donc, \[
				f = \sum_{i=1}^p \lambda_i e_i + \sum_{j=1}^q \mu_j u_j \in F\cap G
			\] Comme $(e_1, \ldots, e_p)$ est une base de $F\cap G$, $\exists ! (\lambda_1', \ldots, \lambda_p') \in \mathbbm{K}^p$ tel que \[
				f = \sum_{i=1}^p \lambda'_i e_i = \sum_{i=1}^p \lambda'_i e_i + \sum_{j=1}^q 0_\mathbbm{K}u_j
			\] Comme $(e_1, \ldots, e_p, u_1, \ldots, u_q)$ est une base de $F$, \[
				\forall k \in \left\llbracket 1, q \right\rrbracket, \mu_j = 0_\mathbbm{K}
			\] De même, \[
				\forall k \in \left\llbracket 1,r \right\rrbracket , \nu_k = 0_\mathbbm{K}
			\] On remplace dans $(*)$ pour trouver \[
				\sum_{i=1}^p \lambda_ie_i = 0_E
			\] Comme $(e_1, \ldots, e_p)$ est libre, \[
				\forall i \in \left\llbracket 1,p \right\rrbracket, \lambda_i = 0_\mathbbm{K}
			\] Donc $\mathcal{B}$ est libre.\\
			Donc, 
			\begin{align*}
				\dim(F+G) &=  p +q + r \\
				&= (p+q)+ (p+r) - p \\
				&= \dim(F) + \dim(G) - \dim(F\cap G) \\
			\end{align*}
	\end{itemize}
\end{prv}

\begin{crlr}
	Avec les hypothèse précédentes, \[
		E = F \oplus G \iff \begin{cases}
			F \cap  G = \{0_E\} \\
			\dim(E) = \dim(F) + \dim(G)
		\end{cases}
	\] 
\end{crlr}

\begin{prv}
	\begin{itemize}
		\item[``$\implies$''] On suppose $E = F \oplus G$ \\
			Comme la somme est directe, $F \cap G = \{0_E\}$ 
			\begin{align*}
				\dim(E) &= \dim(F)\\
				&= \dim(F) + \dim(G) - \dim(F\cap G)\\
				&= \dim(F) + \dim(G)\\
			\end{align*}
		\item[``$\impliedby$''] On suppose $F\cap G = \{0_E\}$ et $\dim(E) = \dim(F) + \dim(G)$.\\
			On sait déjà que $F+G = F \oplus G$\\
			 \begin{align*}
				\dim(F+G) = \dim(F) + \dim(G) - \dim(F \cap G) = \dim(E)
			\end{align*}
			Donc $F + G = E$
	\end{itemize}
\end{prv}

\begin{prop}
	Soit $F$ un $\mathbbm{K}$-espace vectoriel de dimension finie $n$. Soit $\mathcal{B} = (e_1, \ldots, e_n)$ une base de $F$. L'application
	\begin{align*}
		f: \mathbbm{K}^n &\longrightarrow F \\
		(\lambda_1, \ldots, \lambda_n) &\longmapsto \sum_{i=1}^n \lambda_i e_i
	\end{align*} est bijective.\\
	Si $\mathbbm{K}$ est infini, $\mathbbm{K}^n$ aussi et donc $F$ aussi.\\
	Si $\#\mathbbm{K} = p \in \N_*$,
	\begin{align*}
		\#&\mathbbm{K}^n = p^n\\
		&\vrt=\\
		\#&F
	\end{align*}
\end{prop}


		\part{Dérivation}

\underline{Motivation}:

{
\begin{wrapfigure}{l}{3cm}
	\centering
	\begin{asy}
		import three;

		size(3cm);
		settings.render=0;
		settings.prc=false;
		currentprojection = obliqueZ;

		draw(unitbox);
		draw(shift(1.1Z + 0.05X) * (O -- X), Arrows3(TeXHead2));
		draw(shift(1.1Z + 0.05Y) * (O -- Y), Arrows3(TeXHead2));
		draw(shift(1.1X + 0.05Z) * (O -- Z), Arrows3(TeXHead2));

		label("$x$", (X/2) + (1.1Z + 0.05X), align=S);
		label("$y$", (Y/2) + (1.1Z + 0.05Y), align=W);
		label("$z$", (Z/2) + X, align=SE);
	\end{asy}
\end{wrapfigure}

\begin{align*}
	&S(x,y,z) = 2(xy + xz + yz)\\
	&V(x,y,z) = xyz
\end{align*}

On cherche à minimiser $S$ avec la contrainte $V = 1$.

Soit $f : \begin{array}{rcl}
	\left( \R_*^+ \right)^2 &\longrightarrow& \R \\
	(x,y) &\longmapsto& S\left( x,y,\frac{1}{xy} \right) = 2\left( xy + \frac{1}{y} + \frac{1}{x} \right).
\end{array}$

On cherche $(a,b) \in \left( \R^+_* \right)^2$ tel que \[
	\forall (x,y) \in (\R^+_*), f(x,y) \ge f(a,b).
\]
}

\begin{defn}
	Soit $f: U \to \R$ où $U$ est un ouvert de $\R^2$. Soit $(a,b) \in U$.
	\vspace{2mm}

	Si $\lim_{x \to a} \frac{f(x,b) - f(a,b)}{x - a} \in \R$, alors on dit que $f$ a une dérivée partielle suivant $x$ en $(a,b)$ et cette limite est notée \[
		\partial f_1(a,b) = \frac{\partial f}{\partial x}(a,b).
	\]

	Si $\lim_{y \to b} \frac{f(a,y) - f(a,b)}{y - b} \in \R$, alors on dit que $f$ a une dérivée partielle suivant $y$ et la limite est notée \[
		\partial f_2(a,b) = \frac{\partial f}{\partial y}(a,b).
	\]
\end{defn}

\begin{exm}
	\begin{enumerate}
		\item $f: (x,y) \mapsto xy + x - y$.

			\begin{align*}
				&\frac{\partial f}{\partial x} : (x,y) \mapsto y + 1,\\
				&\frac{\partial f}{\partial y} : (x,y) \mapsto x - 1.
			\end{align*}

		\item $f: (x,y) \mapsto xy + \frac{1}{y}+ \frac{1}{x}$.

			\begin{align*}
				&\frac{\partial f}{\partial x}: (x,y) \mapsto y - \frac{1}{x^2},\\
				&\frac{\partial f}{\partial y}: (x,y) \mapsto x - \frac{1}{y^2}.
			\end{align*}

		\item Trouver $f$ telle que $\begin{cases}
				(1): \qquad \frac{\partial f}{\partial x}=y,\\[2mm]
				(2): \qquad \frac{\partial f}{\partial y} = x.
			\end{cases}$

			D'après $(1)$ : \[
				\forall (x,y), \exists C(y) \in \R, f(x,y) = xy + C(y)
			\] et donc \[
				\frac{\partial f}{\partial y}(x,y) = x + C'(y)
			\] donc $C'(y) = 0$ et donc $C$ est constante.

		\item Trouver $f$ telle que $\begin{cases}
			\frac{\partial f}{\partial x} = -y,\\[2mm]
			\frac{\partial f}{ƒ\partial y} = x.
		\end{cases}$

		Ce n'est pas possible !
	\end{enumerate}
\end{exm}

\begin{defn}~\\
	\begin{minipage}{\linewidth}
		\begin{wrapfigure}{r}{4cm}
			\centering
			\vspace{-5mm}
			\begin{asy}
				import three;
				import graph3;
				size(4cm);

				settings.render = 0;
				settings.prc = false;
				currentprojection = obliqueX;

				draw(O -- X, Arrow3(TeXHead2));
				draw(O -- Y, Arrow3(TeXHead2));
				draw(O -- Z, Arrow3(TeXHead2));

				triple f(real x, real y, real z = 0) { return (x,y,cos(x - 0.5) * cos(y - 0.5)/1.2 + 0.15); }

				real inc = 1 / 5;

				for(real x = 0; x <= 1; x += inc) {
					draw(graph(
						new real(real t) { return x; }, // x
						new real(real y) { return y; }, // y
						new real(real y) { return f(x,y).z; }, // z
						0, 1
					), gray);
				}

				for(real y = 0; y <= 1; y += inc) {
					draw(graph(
						new real(real x) { return x; }, // x
						new real(real t) { return y; }, // y
						new real(real x) { return f(x,y).z; }, // z
						0, 1
					), gray);
				}

				path3 path1 = (0.8, 0.2, 0) .. (0.5, 0.5, 0) .. (0.3, 0.7, 0);
				path3 path2 = f(0.8, 0.2, 0) .. f(0.5, 0.5, 0) .. f(0.3, 0.7, 0);
				path3 d = (0.2, 0.3, 0) .. (0.3, 0.4, 0) .. (0.2, 0.7, 0) .. (0.8, 0.9, 0) .. (0.6, 0.2, 0) .. cycle;

				draw(path1, red, Arrow3(TeXHead2));
				draw(path2, red, Arrow3(TeXHead2, position=0.8));

				dot((0.5, 0.5, 0));
				dot(f(0.5, 0.5, 0));
				draw((0.5, 0.5, 0) -- f(0.5, 0.5, 0), dashed);
				draw(d);

				label("$w$", (0.3, 0.7, 0), red, align=SE);
				label("$U$", (0.8, 0.9, 0), align=SE);
			\end{asy}
		\end{wrapfigure}

		Soit $f: U \to \R$ où $U$ est un ouvert. Soit $(a,b) \in U$. Soit $w = (w_1, w_2) \in \R^2$.

		Si 
		\[
			\lim_{t\to 0} \frac{f(a + tw_1, b + tw_2) - f(a,b)}{t}
		\] existe et est réelle, alors on dit que $f$ a une dérivée dans la direction de $w$ et la limite est notée \[
			\mathrm{d}f(w)\,(a,b) = D_w(f)\,(a,b).
		\]
	\end{minipage}
\end{defn}

\begin{exm}
	\begin{align*}
		f: \left( \R_*^+ \right)^2 &\longrightarrow \R \\
		(x,y) &\longmapsto xy+\frac{1}{x}+\frac{1}{y}.
	\end{align*}

	On pose $(a,b) = (1,2)$, $w = (w_1, w_2) = (1,1)$.
	\begin{align*}
		\frac{f(1+t, 2+t) - f(1,2)}{t} &= \frac{1}{t} \left( (1+t)(2+t) + \frac{1}{1+t} + \frac{1}{2+t} - 3 - \frac{1}{2} \right) \\
		&= \frac{1}{t}\left(\cancel 2 + 3t + \po(t) + \cancel 1 - t + \po(t) + \frac{1}{2}\left( \cancel 1 - \frac{t}{2} + \po(t) \right) - \cancel3 - \cancel{\frac{1}{2}} \right) \\
		&= \frac{1}{t} \left( \frac{7}{4} t + \po(t) \right)  \\
		&= \frac{7}{4} + \po(1) \tendsto{t \to 0} \frac{7}{4}. \\
	\end{align*}

	Donc, \[
		\mathrm{d}f(1,1)\,(1,2) = \frac{7}{4}.
	\]
\end{exm}

\begin{rmk}~\\
	\begin{figure}[H]
		\centering
		\begin{asy}
			import solids;
			import graph;
			size(5cm);

			settings.render = 0;
			settings.prc = false;

			path3 par = graph(
				new real(real x) { return x; },
				new real(real x) { return 0; },
				new real(real x) { return x^2; },
				0,3);
			revolution r = revolution(par, axis=Z);

			path3 par2 = graph(
				new real(real x) { return x; },
				new real(real x) { return 0; },
				new real(real x) { return x^2; },
				-3,3);

			draw(r,1,longitudinalpen=nullpen);
			draw(r.silhouette());

			draw((-4, 0, -1) -- (-4, 0, 10) -- (4, 0, 10) -- (4, 0, -1) -- cycle, red);
			draw(par2, deepred);

			draw((4,4.5) -- (7, 4.5), black+0.5mm, Arrow(TeXHead));

			path par2d = graph(new real(real x) { return x^2; }, -3, 3);
			draw(shift((11, 0)) * par2d, deepred);

			dot(O);
			dot((11, 0));
		\end{asy}
	\end{figure}
\end{rmk}


%todo ajouter théorème-définition
\begin{thm}
	Soit $f : U \to \R$, $(a,b) \in U$. On suppose que $\frac{\partial f}{\partial x}$ et $\frac{\partial f}{\partial y}$ existent en $(a,b)$ et sont {\bfseries continues} en $(a,b)$. Alors,
	\begin{align*}
		&\forall (h, k) \in \R^2 \text{ tel que } (a +h, b + k) \in U,\\
		&f(a+ h, b + k) = f(a,b) + h \frac{\partial f}{\partial x}(a,b) + k \frac{\partial f}{\partial y}(a,b) + \po_{(h,k)\to (0,0)}\big(\|(h,k)\|\big).
	\end{align*}

	On dit que $f$ est de classe $\mathcal{C}^1$ si $\frac{\partial f}{\partial x}$ et $\frac{\partial f}{\partial y}$ existent et sont continues.

	\qed
\end{thm}

\begin{rmk}
	En physique, cette formule correspond à : \[
		\mathrm{d}f = \frac{\partial f}{\partial x}\mathrm{d}x + \frac{\partial f}{\partial y} \mathrm{d}y.
	\] En effet :
	\begin{align*}
		\mathrm{d}f &= f(x+ \mathrm{d}x, y + \mathrm{d}y) - f(x,y) \\
		&= \frac{\partial f}{\partial x} \mathrm{d}x + \frac{\partial f}{\partial y} \mathrm{d}y.
	\end{align*}
\end{rmk}

\begin{prop}
	Soit $f: U \to \R$ de classe $\mathcal{C}^1$ en $(a,b) \in U$. Alors, \[
		\forall w = (w_1, w_2) \in \R^2, \mathrm{d}f(w)\,(a,b) = w_1 \frac{\partial f}{\partial x}(a,b) + w_2 \frac{\partial f}{\partial y}(a,b).
	\]
\end{prop}

\begin{prv}
	Soit $w = (w_1, w_2) \in \R^2$. Soit $t \in \R^*$.
	\begin{align*}
		\frac{1}{t}\big(f(a + tw_1, b + tw_2) - f(a,b)\big)
		&= \frac{1}{t} \left( tw_1 \frac{\partial f}{\partial x}(a,b) + tw_2 \frac{\partial f}{\partial y}(a,b) + \po_{t \to 0}\big(\|tw\|\big) \right) \\
		&= w_1 \frac{\partial f}{\partial x}(a,b) + w_2 \frac{\partial f}{\partial y}(a,b) + \po_{t\to 0}(1) \\
		&\tendsto{t\to 0} w_1 \frac{\partial f}{\partial x}(a,b) + w_2\frac{\partial f}{\partial y}(a,b).
	\end{align*}
\end{prv}


\begin{defn}
	Avec les hypothèses précédentes, en posant \[
		\nabla f(a,b) = \left( \frac{\partial f}{\partial x}(a,b), \frac{\partial f}{\partial y}(a,b) \right) 
	\]on obtient \[
		\mathrm{d}f(w)\,(a,b) = \left<w  \mid \nabla f(a,b) \right>
	\] où $\left<\cdot|\cdot \right>$ est le produit scalaire.

	Le vecteur $\nabla f(a,b)$ est appelé \underline{gradient de $f$ en $(a,b)$}.

	Le développement limité à l'ordre 1 de $f$ devient \[
		f\big((a,b)+w\big) = f(a,b) + \left<w \mid \nabla f(a,b) \right> + \po_{w\to 0}(\|w\|)
	\]
\end{defn}

\begin{prop}
	Soit $f : U \to \R$ de classe $\mathcal{C}^1$.

	\begin{figure}[H]
    \centering
    \incfig{gradient}
	\end{figure}

	$\nabla f$ est orthogonal au lignes de niveaux de $f$, son orientation va dans le sens d'une augmentation de $f$.
\end{prop}

\begin{prv}
	Soit $\gamma : I \to U$ une courbe de niveau : \[
		\forall t \in I, f\big(\gamma(t)\big) = \text{cste}.
	\] D'après le lemme suivant : \[
		\forall t \in I, 0 = (f \circ \gamma)'(t) = \mathrm{d}f\big(\gamma'(t)\big)\big(\gamma(t)\big) = \left<\gamma'(t)  \mid \nabla f\big(\gamma(t)\big) \right>
	\] Donc $\nabla f\big(\gamma(t)\big)$ est orthogonal à $\gamma'(t)$.

	Pour tout $t \in I$, on pose $w(t) = t\, \nabla f\big(\gamma(t)\big)$. Donc \[
		f\big(\gamma(t) + w(t)\big) = f\big(\gamma(t)\big) + t \|\nabla f(\gamma(t))\|^2 + \po_{t \to 0}(t)
	\] Pour $t$ assez petit, $f\big(\gamma(t) + w(t)\big) - f\big(\gamma(t)\big)$ est du même signe que $t$.
\end{prv}

\begin{rmk}
	\begin{align*}
		V: \R^3 &\longrightarrow \R \\
		(x,y,z) &\longmapsto -mgz
	\end{align*}
	l'énerge potentielle de pesenteur

	On a donc \[
		\nabla V(x,y,z) = \left( \frac{\partial V}{\partial x}, \frac{\partial V}{\partial y}, \frac{\partial V}{\partial z} \right) = (0, 0, -mg) = \vec{P}.
	\]
\end{rmk}

\begin{lem}
	Soit $f : U \to \R$ de classe $\mathcal{C}^1$, $\gamma : \begin{array}{rcl}
		I &\longrightarrow& U \\
		t &\longmapsto& \big(x(t), y(t)\big)
	\end{array}$ où $x$ et $y$ sont dérivables.

	On pose \[
		\forall t \in I, \gamma'(t) = \big(x'(t), y'(t)\big).
	\] Alors $f \circ \gamma : I \to \R$ est dérivable et
	\begin{align*}
		\forall t \in I, (f \circ \gamma)'(t) &= \mathrm{d}f\big(\gamma'(t)\big) \big(\gamma(t)\big)\\
		&= \left<\gamma'(t)  \mid \nabla f\big(\gamma(t)\big)  \right> \\
		&= x'(t) \frac{\partial f}{\partial x}\big(x(t), y(t)\big) + y'(t) \frac{\partial f}{\partial y}\big(x(t),y(t)\big). \\
	\end{align*}
\end{lem}

\begin{prv}
	On fixe $t \in I$.

	\begin{align*}
		\forall h \neq 0, \frac{f \circ \gamma(t + h) - f \circ \gamma(t)}{h}
		&= \frac{1}{h}\big(f(\gamma(t)) + h\gamma'(t) + \po_{h\to 0}(h) - f(\gamma(t))\big) \\
		&= \frac{1}{h}\bigg(\cancel{f(\gamma(t))} + \left<h\gamma'(t) \mid \nabla f(\gamma(t)) \right> + \po_{h\to 0}(\|h\gamma'(t)\|) - \cancel{f(\gamma(t))}\bigg)\\
		&= \left<\gamma'(t) \mid \nabla f(\gamma(t)) \right> + \po_{h\to 0}(1) \\
		&\tendsto{h\to 0} \left<\gamma'(t)  \mid \nabla f(\gamma(t)) \right>
	\end{align*}
\end{prv}

\begin{defn}
	Soit $f : U \to \R$ de classe $\mathcal{C}^1$ et $(a,b) \in U$. On dit que $(a,b)$ est un \underline{point critique} de $f$ si $\nabla f(a,b) = 0$ i.e. $\frac{\partial f}{\partial x}(a,b) = \frac{\partial f}{\partial y}(a,b) = 0$.

	Dans ce cas, $f(a,b)$ est appelé \underline{valeur critique} de $f$.
\end{defn}

\begin{prop}~\\
	\begin{minipage}{\linewidth}
		\begin{wrapfigure}{r}{3cm}
			\centering
			\vspace{-1cm}
			\begin{asy}
				import solids;
				import graph;
				size(3cm);

				settings.render = 0;
				settings.prc = false;

				path3 par = graph(
					new real(real x) { return x; },
					new real(real x) { return 0; },
					new real(real x) { return -x^2; },
					0,3);
				revolution r = revolution(par, axis=Z);

				draw(r,1,longitudinalpen=nullpen);
				draw(r.silhouette());

				dot("$(a,b)$", O, red, align=N);
				real s = sqrt(2.5);
				path3 g=(s,0,-2.5)..(0,s,-2.5)..(-s,0,-2.5)..(0,-s,-2.5)..cycle;
				draw(g, deepcyan);
			\end{asy}
		\end{wrapfigure}
		Soit $f: U \to \R$ de classe $\mathcal{C}^1$ et $(a,b) \in U$ tel que \[
			\exists r > 0, \forall (x,y) \in B_{(a,b)}(r), f(x,y) \le f(a,b)
		\] Alors $\nabla f(a,b) = (0,0)$.
	\end{minipage}
\end{prop}

\begin{prv}
	Soit $g: x \mapsto f(x,b)$. $g(a)$ est un maximum local de $g$ donc $g'(a) = 0$.

	Or, $g'(a) = \frac{\partial f}{\partial x}(a,b)$

	donc $\frac{\partial f}{\partial x}(a,b) = 0$.

	Soit $h : y \mapsto f(a,y)$. On a de même $h'(b) = 0$.

	Or, $h'(b) = \frac{\partial f}{\partial y}(a,b)$.

	Donc, $\nabla f(a,b) = (0,0)$.
\end{prv}

\begin{rmk}
	Un minimum local est aussi une valeur critique.
\end{rmk}

\begin{figure}[H]
	\centering
	\begin{subfigure}{3cm}
		\centering
		\begin{asy}
			import solids;
			import graph;
			size(3cm);

			settings.render = 0;
			settings.prc = false;

			path3 par = graph(
				new real(real x) { return x; },
				new real(real x) { return 0; },
				new real(real x) { return -x^2; },
				0,3);
			revolution r = revolution(par, axis=Z);

			draw(r,1,longitudinalpen=nullpen);
			draw(r.silhouette());

			dot(O, red);
		\end{asy}
		\caption{Maximum local}
	\end{subfigure}
	\begin{subfigure}{3cm}
		\centering
		\begin{asy}
			import solids;
			import graph;
			size(3cm);

			settings.render = 0;
			settings.prc = false;

			path3 par = graph(
				new real(real x) { return x; },
				new real(real x) { return 0; },
				new real(real x) { return x^2; },
				0,3);
			revolution r = revolution(par, axis=Z);

			draw(r,1,longitudinalpen=nullpen);
			draw(r.silhouette());

			dot(O, red);
		\end{asy}
		\caption{Minimum local}
	\end{subfigure}
	\begin{subfigure}{3cm}
		\centering
		\begin{asy}
			import solids;
			import graph;
			size(3cm);

			settings.render = 0;
			settings.prc = false;
			currentprojection = obliqueZ;

			draw(graph(
				new real(real x) { return x; },
				new real(real x) { return -x^2 / 3; },
				new real(real x) { return 3; },
				-3, 3
			));

			draw(graph(
				new real(real x) { return x; },
				new real(real x) { return -x^2 / 3; },
				new real(real x) { return -3; },
				-3, 3
			));

			draw(graph(
				new real(real x) { return x; },
				new real(real x) { return -x^2 / 3 - 1; },
				new real(real x) { return 0; },
				-3, 3
			));

			draw(graph(
				new real(real x) { return 0; },
				new real(real x) { return x^2 / 9 - 1; },
				new real(real x) { return x; },
				-3, 3
			));

			draw(graph(
				new real(real x) { return -3; },
				new real(real x) { return x^2 / 9 - 4; },
				new real(real x) { return x; },
				-3, 3
			));

			draw(graph(
				new real(real x) { return 3; },
				new real(real x) { return x^2 / 9 - 4; },
				new real(real x) { return x; },
				-3, 3
			));

			dot((0,-1,0), red);
		\end{asy}
		\caption{Point de selle / Point col}
	\end{subfigure}
\end{figure}

\begin{exm}
	On revient à l'exemple donné en introduction : 
	\begin{align*}
		f: \left( \R^*_+ \right)^2 &\longrightarrow \R \\
		(x,y) &\longmapsto 2\left( xy + \frac{1}{x} + \frac{1}{y} \right).
	\end{align*}

	$\left( \R^+_* \right)^2$ est un ouvert de $\R^2$. Soit $(x,y) \in \left( \R^+_* \right)^2$.
	
	On a \[
		\begin{cases}
			\frac{\partial f}{\partial x}(x,y) = 2\left( y - \frac{1}{x^2} \right),\\
			\frac{\partial f}{\partial y}(x,y) = 2\left( x - \frac{1}{y^2} \right).
		\end{cases}
	\]

	\begin{align*}
		&\frac{\partial f}{\partial x}(x,y) = \frac{\partial f}{\partial y}(x,y) = 0\\
		\iff& \begin{cases}
			y = \frac{1}{x^2}\\
			x = \frac{1}{y^2}
		\end{cases}\\
		\iff& \begin{cases}
			y = \frac{1}{x^2}\\
			x = x^4
		\end{cases}\\
		\iff& \begin{cases}
			x = 1\\
			y = 1
		\end{cases}
	\end{align*}

	On vérivie que $f$ présente en effet un minium local en $(1,1)$. \[
		f(1,1) = 6
	\] On fixe $y \in \R^+_*$ et \[
		g : x \mapsto 2\left( xy + \frac{1}{x} + \frac{1}{y} \right).
	\] Donc \[
		\forall x \in \R^+_*, g'(x) = 2\left( y - \frac{1}{x^2} \right).
	\]
	\begin{center}
		\begin{tikzpicture}
			\tkzTabInit{$x$/1,$g'(x)$/1,$g$/2.3}{$0$, $\frac{1}{\sqrt{y}}$, $+\infty$}
			\tkzTabLine{,-,z,+,}
			\tkzTabVar{+/{}, -/$2\left( 2\sqrt{y} +\frac{1}{y} \right)$, +/{}}
		\end{tikzpicture}
	\end{center}
	
	Ainsi, \[
		\forall x \in \R^+_*, \forall y \in \R^+_*, f(x,y) \ge 2\left( 2\sqrt{y} + \frac{1}{y} \right)
	\] Soit $h : y \mapsto 2\sqrt{y} + \frac{1}{y}$. On a \[
		\forall y > 0, h'(y) = \frac{1}{\sqrt{y}} - \frac{1}{y^2} = \frac{y\sqrt{y} - 1}{y^2} = \frac{y^{\frac{3}{2}} - 1}{y^2}
	\]

	\begin{center}
		\begin{tikzpicture}
			\tkzTabInit{$y$/0.7,$h'(y)$/0.7,$h$/1.4}{$0$, $1$, $+\infty$}
			\tkzTabLine{,-,z,+,}
			\tkzTabVar{+/{}, -/$3$, +/{}}
		\end{tikzpicture}
	\end{center}

	Donc, \[
		\forall x,y > 0, f(x,y) \ge 2\times 3 = 6 = f(1,1).
	\]
\end{exm}

\begin{prop}
	[règle de la chaîne]

	Soit $f : \begin{array}{rcl}
		U &\longrightarrow& \R^2 \\
		(x,y) &\longmapsto& f(x,y)
	\end{array}$ de classe $\mathcal{C}^1$ et $U, V$ deux ouverts de $\R^2$.

	Soit $\varphi : \begin{array}{rcl}
		V &\longrightarrow& U \\
		(u,v) &\longmapsto& \varphi(u,v) = \big(x(u,v), y(u,v)\big)
	\end{array}$.

	On suppose que $x$ et $y$ sont de classe $\mathcal{C}^1$ sur $V$.

	Alors,  $f \circ \varphi : \begin{array}{rcl}
		V &\longrightarrow& \R \\
		(u,v) &\longmapsto& f\big(\varphi(u,v)\big)
	\end{array}$ est de classe $\mathcal{C}^1$ et
	\begin{align*}
		\forall (u_0, v_0) \in V, \frac{\partial (f \circ \varphi)}{\partial u}(u_0, v_0)
		&= \frac{\partial f}{\partial x}\big(\varphi(u_0, v_0)\big) \times \frac{\partial x}{\partial u}(u_0, v_0)\\
		&+ \frac{\partial f}{\partial y}\big(\varphi(u_0,v_0)\big) \frac{\partial y}{\partial u}(u_0,v_0)
	\end{align*}
	\begin{align*}
		\forall (u_0, v_0) \in V, \frac{\partial (f \circ \varphi)}{\partial v}(u_0, v_0)
		&= \frac{\partial f}{\partial x}\big(\varphi(u_0, v_0)\big) \times \frac{\partial x}{\partial v}(u_0, v_0)\\
		&+ \frac{\partial f}{\partial y}\big(\varphi(u_0,v_0)\big) \frac{\partial y}{\partial v}(u_0,v_0)
	\end{align*}
\end{prop}

\begin{exm}
	[changement de coordonnées polaires]
	On pose \begin{align*}
		\varphi: \R^+_* \times ]0,2\pi[ &\longrightarrow \R^2\setminus \left( R^+_* \times \{0\} \right) \\
		(r, \theta) &\longmapsto (r \cos \theta, r \sin\theta),
	\end{align*}
	\begin{align*}
		f: \R^2\setminus \left( R^+_* \times \{0\} \right) &\longrightarrow \R \\
		(x,y) &\longmapsto f(x,y),
	\end{align*}
	\begin{align*}
		g: \overbrace{\R^+_* \times ]0, 2\pi[}^{=V} &\longrightarrow \R \\
		(r, \theta) &\longmapsto f(r\cos\theta, r\sin\theta).
	\end{align*}

	\begin{align*}
		\forall (r_0,\theta_0) \in V,&\\[5mm]
		\frac{\partial g}{\partial r}(r_0, \theta_0) &= \frac{\partial f}{\partial x}(r_0\cos\theta_0, r_0\sin\theta_0)\cos\theta_0\\
		&+ \frac{\partial f}{\partial y}(r_0 \cos\theta_0, r_0\sin\theta_0)\sin\theta_0\\
		&= 2r_0\cos^2\theta_0 + 2r_0\sin^2(\theta_0) \\
		&= 2r_0 \\[5mm]
		\frac{\partial g}{\partial \theta}(r_0, \theta_0) &= \frac{\partial f}{\partial x}(r_0\cos\theta_0, r_0\sin\theta_0)r_0\sin\theta_0\\
		&+ \frac{\partial f}{\partial y}(r_0 \cos\theta_0, r_0\sin\theta_0)r_0\cos\theta_0\\
		&= -2{r_0}^2\cos(\theta_0)\sin(\theta_0) + 2{r_0}^2 \sin(\theta_0)\cos(\theta_0)\\
		&= 0 \\
	\end{align*}

	Donc, \[
		g(r, \theta) = r^2.
	\]
\end{exm}

\begin{exm}
	Résoudre \[
		\begin{cases}
			\frac{\partial f}{\partial x} = \frac{x}{x^2+y^2},\\
			\frac{\partial f}{\partial y} = \frac{y}{x^2+y^2}.\\
		\end{cases}
	\]

	On pose $g: (r, \theta) \mapsto f(r \cos\theta, r \sin\theta)$.

	\begin{align*}
		&\frac{\partial g}{\partial r} = \frac{1}{r}\cos^2\theta + \frac{1}{r}\sin^2\theta = \frac{1}{r},\\
		&\frac{\partial g}{\partial \theta} = -\cos(\theta) \sin(\theta) + \sin(\theta)\cos(\theta) = 0.
	\end{align*}

	Donc, \[
		\exists C \in \R, g: (r, \theta) \mapsto \ln r + C
	\] d'où,
	\begin{align*}
		\forall (x,y) \in \R^2 \setminus \{(0,0)\}, f(x,y) &= \ln\left(\sqrt{x^2 + y^2} \right)  + C\\
		&= \frac{1}{2}\ln(x^2 + y^2) + C. \\
	\end{align*}
\end{exm}

\begin{rmk}
	Soit $\mathcal{B} = (e_1, e_2)$ la base canonique de $\R^2$, $f: U \to \R$ de classe $\mathcal{C}^1$ avec $U$ un ouvert de $\R^2$.

	Soit $(x,y) \in U$.

	\begin{align*}
		\Mat_{\mathcal{B}}\big(\nabla f(x,y)\big) = \begin{pmatrix}
			\frac{\partial f}{\partial x}(x,y)\\[2mm]
			\frac{\partial f}{\partial y}(x,y)
		\end{pmatrix}
	\end{align*}

	Soit  \begin{align*}
		\varphi: V &\longrightarrow U \\
		(u,v) &\longmapsto \big(x(u,v), y(u,v)\big) 
	\end{align*} avec $x,y$ de classe $\mathcal{C}^1$. Soit $g = f \circ \varphi$.
	\begin{align*}
		\Mat_{\mathcal{B}}\big(\nabla g(u,v)\big)
		&= \begin{pmatrix}
			\frac{\partial g}{\partial u}(u,v) \\[2mm]
			\frac{\partial g}{\partial v}(u,v)
		\end{pmatrix} \\
		&= \begin{pmatrix}
			\frac{\partial x}{\partial u}(u,v) \frac{\partial f}{\partial x}(x,y)
			+ \frac{\partial y}{\partial u}(u,v)\frac{\partial f}{\partial y}(x,y)\\[3mm]
			\frac{\partial x}{\partial v}(u,v) \frac{\partial f}{\partial x}(x,y)
			+ \frac{\partial y}{\partial v}(u,v) \frac{\partial f}{\partial y}(x,y)
		\end{pmatrix}  \\
		&= \underbrace{\begin{pmatrix}
				\frac{\partial x}{\partial u}(u,v)& \frac{\partial y}{\partial u}(u,v)\\[3mm]
				\frac{\partial x}{\partial v}(u,v)& \frac{\partial y}{\partial v}(u,v)
		\end{pmatrix}}_{J(u,v)} \begin{pmatrix}
			\frac{\partial f}{\partial x}(x,y)\\[3mm]
			\frac{\partial f}{\partial y}(x,y)
		\end{pmatrix} \\
		&= J(u,v) \Mat_{\mathcal{B}}\big(\nabla f(x,y)\big) \\
	\end{align*}
	où $J(u,v) = 
	\begin{pNiceArray}{c:c}
		\Mat_{\mathcal{B}}\big(\nabla x(u,v)\big) & \Mat_{\mathcal{B}}\big(\nabla y(u,v)\big)
	\end{pNiceArray}$.

	On dit que $J(u,v)$ est \underline{la jacobienne} de $\varphi$ en $(u,v)$.
	L'application linéaire canoniquement associée à $J(u,v)$ est la \underline{différentielle de $\varphi$} en $(u,v)$ noté $\mathrm{d}\varphi(u,v)$.

	On a $\mathrm{d}\varphi(u,v) \in \mathcal{L}(R^2)$ et $\Mat_{\mathcal{B}}\big(\mathrm{d}\varphi(u,v)\big) = J(u,v)$.

	Par exemple, la jacobienne du changement de coordonnées polaires est \[
		J = \begin{pmatrix}
			\frac{\partial x}{\partial r} & \frac{\partial y}{\partial r}\\[3mm]
			\frac{\partial x}{\partial \theta} & \frac{\partial y}{\partial \theta}
		\end{pmatrix}
		= \begin{pmatrix}
			\cos\theta&\sin\theta\\
			-r\sin\theta&r\cos\theta
		\end{pmatrix}.
	\]
	$\underbrace{\det(J)}_{\text{le jacobien}} = r\cos^2\theta + r\sin^2\theta = r$

	Dans une intégrale double, si $(x,y) = \varphi(u,v)$, alors $\mathrm{d}x\mathrm{d}y = \det(J)\mathrm{d}u\mathrm{d}v$.

	Ici, \[
		\mathrm{d}x\ \mathrm{d}y = r\ \mathrm{d}r\ \mathrm{d}\theta.
	\]
\end{rmk}

\begin{prv}
	On pose $(x_0, y_0) = \varphi(u_0, v_0)$. Pour tout $(h,k) \in \R^2$ tels que $(u_0 + h, v_0 + k) \in V$, en posant $g = f  \circ \varphi$.

	\begin{align*}
		g(u_0 + h, v_0 + h) &= f\big(x(u_0 + h, v_0 + k), y(u_0 + h, v_0 + k)\big) \\
		&= f\left(
			x(u_0,v_0) + h \frac{\partial x}{\partial u}(u_0,v_0) + k \frac{\partial x}{\partial v}(u_0, v_0) + \po\big(\|(h,k)\|\big), \right.\\
		&\phantom{ = f\bigg(\bigg.}\left. y(u_0, v_0) + h \frac{\partial y}{\partial u}(u_0, v_0) + k \frac{\partial y}{\partial v}(u_0, v_0) + \po\big(\|(h,k)\|\big)
		\right)  \\
		&= f(x_0,y_0) \\
		&~+ \left( h \frac{\partial x}{\partial u}(u_0,v_0) + k \frac{\partial x}{\partial v}(u_0, v_0) + \po(\|(h,k)\|) \right) \frac{\partial f}{\partial x}(x_0,y_0)\\
		&~+ \left( h \frac{\partial y}{\partial u}(u_0, v_0) + k\frac{\partial y}{\partial v}(u_0, v_0) + \po(\|(h,k)\|) \right) \frac{\partial f}{\partial y}(x_0, y_0)\\
		&~+ \po(\|(h,k)\|)\\
		&= f(x_0, y_0) \\
		&~+ h \left( \frac{\partial x}{\partial u}(u_0, v_0) \frac{\partial f}{\partial x}(x_0, y_0) + \frac{\partial y}{\partial u}(u_0, v_0) \frac{\partial f}{\partial y}(x_0, y_0) \right)  \\
		&~+ k\left( \frac{\partial x}{\partial v}(u_0, v_0) \frac{\partial f}{\partial x}(x_0, y_0) + \frac{\partial y}{\partial v}(u_0, v_0) \frac{\partial f}{\partial y}(x_0, y_0) \right) 
		&~+ \po(\|(h,k)\|)\\
		&= g(u_0, v_0) + h \frac{\partial g}{\partial u}(u_0, v_0) + k \frac{\partial g}{\partial v}(u_0, v_0) + \po(\|(h,k)\|) \\
	\end{align*}

	Par identification,
	\[
		\frac{\partial g}{\partial u}(u_0, v_0) = \frac{\partial x}{\partial u}(u_0, v_0) \frac{\partial f}{\partial x}(x_0, y_0) + \frac{\partial y}{\partial u}(u_0, v_0) \frac{\partial f}{\partial y}(x_0,y_0)
	\] et \[
		\frac{\partial g}{\partial v}(u_0, v_0) = \frac{\partial x}{\partial v}(u_0,v_0) \frac{\partial f}{\partial x}(x_0, y_0) + \frac{\partial y}{\partial v}(u_0, v_0) \frac{\partial f}{\partial y}(x_0, y_0).
	\] 
\end{prv}

\begin{exm}
	[Régression linéaire]~\\
	\begin{figure}[H]
		\centering
		\begin{asy}
			import graph;
			axes(EndArrow);
			size(5cm);

			real f(real x) { return x + 0.5; }

			real k = 35 / (7 - 0.5);

			for(int i = 0; i < 35; ++i) {
				real mag = exp(sin(100 * pi/exp(1) * i)) * 0.8 + exp(cos(i*40)/3);
				real eps = mag * cos(10 * exp(1)/pi * i) / 3;
				dot((i/k,f(i/k) + eps));
			}

			draw(graph(f, -1, 7), orange);
		\end{asy}
	\end{figure}
	\[
		y = a x + b
	\] 
	On fixe $(a,b) \in \R^2$. \[
		\varepsilon(a,b) = \sum_{i=1}^n\big( y_i - (ax_i + b) \big)^2
	\] l'erreur totale.

	On veut minimiser $\varepsilon(a,b)$. On a 
	\[
		\forall (a,b) \in \R^2,
		\begin{cases}
			\frac{\partial \varepsilon}{\partial a}(a,b) = -2\sum_{i=1}^{n}(y_i - ax_i - b)x_i,\\
			\frac{\partial \varepsilon}{\partial b}(a,b) = -2\sum_{i=1}^{n}(y_i - ax_i - b).
		\end{cases}
	\]

	Donc,
	\begin{align*}
		(a,b) \text{ point critique de } \varepsilon \iff& \begin{cases}
			a \sum_{i=1}^n {x_i}^2 + b\sum_{i=1}^{n}x_i = \sum_{i=1}^{n} y_ix_i\\
			a\sum_{i=1}^{n}x_i + nb = \sum_{i=1}^ny_i
		\end{cases}\\
		\iff& \begin{cases}
			a \left( \frac{1}{n}\sum_{i=1}^n {x_i}^2 - \overline{x}^2\right) = \overline{y} - \overline{x} \overline{y}\\
			b = \frac{1}{n}\sum_{i=1}^ny_i - \frac{a}{n}\sum_{i=1}^nx_i = \frac{1}{n}\sum_{i=1}^n x_i y_i - \overline{x} \overline{y}
		\end{cases}\\
		&\text{ où } \overline{x} = \frac{1}{n} \sum_{i=1}^n x_i,~\overline{y} = \frac{1}{n}\sum_{i=1}^n y_i\\
		\iff& \begin{cases}
			a = \frac{\Cov(x,y)}{V(x)}\\
			b = \overline{y} - a\overline{x}
		\end{cases}
	\end{align*}

	Coefficient de corrélation: $\frac{\Cov(x,y)}{\sigma_x \sigma_y} \in [-1, 1]$
\end{exm}












		\part{Corps}

\begin{exm}[Problème]
	\begin{itemize}
		\item 
			avec $A = \Z / 9 \Z$, résoudre $\overline{x}^2 = \overline{0}$ \\
			\begin{center}
				\begin{tabular}{|c|c|c|c|c|c|c|c|c|c|c|}
					\hline
					$\overline{x}$&$\overline{0}$& $\overline{1}$ &$\overline{2}$&$\overline{3}$ &$\overline{4}$ &$\overline{5}$ &$\overline{6}$ &$\overline{7}$ &$\overline{8}$& $\overline{9}$ \\
					\hline
					$\overline{x}^2$&$\overline{0}$ &$\overline{1}$ &$\overline{4}$ &$\overline{0}$ &$\overline{7}$ &$7$ &$\overline{0}$ &$\overline{4}$ &$\overline{1}$&$\overline{0}$\\
					\hline
				\end{tabular}
			\end{center}
			On a trouvé 3 solutions: $\overline{0}$, $\overline{3}$, $\overline{6}$.
		\item $\Z / 8\Z$
			\begin{center}
				\begin{tabular}{|c|c|c|c|c|c|c|c|c|}
					\hline
					$\overline{x}$& $\overline{0}$& $\overline{1}$& $\overline{2}$& $\overline{3}$& $\overline{4}$& $\overline{5}$& $\overline{6}$& $\overline{7}$\\
					\hline
					$\overline{x^2}$& $\overline{0}$& $\overline{1}$& $\overline{4}$& $\overline{1}$& $\overline{0}$& $\overline{1}$& $\overline{4}$& $\overline{1}$\\
					\hline
				\end{tabular}
			\end{center}
			$\overline{x}^2=7$ a 4 solutions: $\overline{1}, \overline{7}, \overline{3},\text{ et } \overline{5}$
		\item $A = \mathbbm{H} = \{a + bi + cj + dk  \mid  (a,b,c,d) \in \R^4\}$ \\
			$i^2 = j^2 = k^2 = -1$ 
			\begin{align*}
				\begin{array}{c c c}
					ij = k & jk = i & ji = j\\
					ji = -k & kj = -i & ik = -j
				\end{array}
			\end{align*}
			Dans cet anneau, $-1$ a 6 racines!
	\end{itemize}
\end{exm}

\begin{defn}
	Soit $(\mathbbm{K}, +, \times)$ un ensemble muni de deux lois de composition internes. On dit que c'est un \underline{corps} si
	 \begin{enumerate}
		\item $(\mathbbm{K}, \times)$ est un groupe abélien
		\item $(\mathbbm{K}, \times)$ est un monoïde commutatif
		\item $\forall x \in \mathbbm{K}\setminus \{0_\mathbbm{K}\}, \exists y \in \mathbbm{K}, xy = 1_\mathbbm{K}$
		\item $0_\mathbbm{K} \neq  1_\mathbbm{K}$
	\end{enumerate}
	\index{corps}
\end{defn}

\begin{exm}
	\begin{itemize}
		\item $(\C, +, \times)$ est un corps
		\item $(\R, +, \times)$ est un corps
		\item $(\Q, +, \times)$ est un corps
		\item $(\Z, +, \times)$ n'est pas un corps
	\end{itemize}
\end{exm}

\begin{prop}
	$(\Z / n\Z, +, \times)$ est un corps si et seulement si $n$ est premier.
\end{prop}

\begin{prv}
	\[
		\left( \Z / n\Z \right)^\times = \left\{ \overline{k}  \mid k \wedge n = 1 \right\}
	\] 
\end{prv}


\begin{prop}
	Tout corps est un anneau intègre.
\end{prop}

\begin{prv}
	Soit $(\mathbbm{K}, +, \times)$ un corps. Soient $(a,b) \in \mathbbm{K}^2$ tel que $a \times b = 0_\mathbbm{K}$.\\
	On suppose $a \neq  0_\mathbbm{K}$. Alors, $a$ est inversible et donc \[
		b = a^{-1} \times a \times b = a^{-1} \times 0_\mathbbm{K} = 0_\mathbbm{K}
	\] 
\end{prv}

\begin{exm}
	Soit $(\mathbbm{K},+,\times)$ un corps.\\
	Résoudre \[
		\begin{cases}
			x^2 = 1_\mathbbm{K}\\
			x \in \mathbbm{K}
		\end{cases}
	\]

	\begin{align*}
		x^2 = 1_\mathbbm{K} &\iff x^2 - 1_\mathbbm{K} = 0_\mathbbm{K}\\
		&\iff (x - 1_\mathbbm{K})(x+1_\mathbbm{K}) = 0_\mathbbm{K}\\
		&\iff x - 1_\mathbbm{K} = 0_\mathbbm{K} \text{ ou } x + 1_\mathbbm{K} = 0_\mathbbm{K}\\
		&\iff x = 1_\mathbbm{K} \text{ ou } x = -1_\mathbbm{K}
	\end{align*}

	Il y a au plus 2 solutions.
\end{exm}

\begin{prop}
	Soit $(\mathbbm{K},+,\times )$ un corps et $P$ un polynôme à coefficients dans $\mathbbm{K}$ de degré $n$. Alors, l'équation $P(x) = 0_{\mathbbm{K}}$ a au plus $n$ solutions dans $\mathbbm{K}$ 
	\qed
\end{prop}

\begin{crlr}[(Théorème de Wilson)]
	voir exercice 16 du TD 12
\end{crlr}


\begin{defn}
	Soit $(\mathbbm{K}, +, \times)$ un corps et $L\subset \mathbbm{K}$.\\
	On dit que $L$ est un \underline{sous corps} de $\mathbbm{K}$ si
	\begin{enumerate}
		\item $L$ est un anneau de $(\mathbbm{K}, +, \times)$ non nul
		\item $\forall x \in L\setminus \{0_\mathbbm{K}\}, x^{-1} \in L$ 
	\end{enumerate}
	\vspace{2mm}
	en d'autres termes si
	\begin{enumerate}
		\item $\forall (x,y) \in L^2, x - y \in L$
		\item $\forall (x,y) \in L^2, x \times y^{-1} \in L$
	\end{enumerate}
	\vspace{5mm}
	On dit aussi que $\mathbbm{K}$ est une \underline{extension} de $L$.
	\index{sous corps}
	\index{extension}
\end{defn}

\begin{prop}
	Tout sous corps est un corps. \qed
\end{prop}

\begin{defn}
	Soient $(\mathbbm{K}_1,+,\times )$ et $(\mathbbm{K}_2,+, \times)$ deux corps et $f: \mathbbm{K}_1 \to \mathbbm{K}_2$.\\
	On dit que $f$ est un \underline{morphisme de corps} si $f$ est un morphisme d'anneaux.\\
	i.e. si
	\[
		\begin{cases}
			\forall (x,y) \in {\mathbbm{K}_1}^2,& f(x+y) = f(x) + f(y)\\
			\forall (x,y) \in {\mathbbm{K}_1}^2,& f(x \times y) = f(x) \times f(y)\\
		\end{cases}
	\] 
	\index{homomorphisme (de corps)}
	\index{morphisme (de corps)}
\end{defn}

\begin{prop}
	Tout morphisme de corps est injectif.
\end{prop}

\begin{prv}
	Soit $f: \mathbbm{K}_1 \to \mathbbm{K}_2$ un morphisme de corps.\\
	\begin{itemize}
		\item $\Ker(f)$ est un sous groupe de $(\mathbbm{K}_1, +)$ 
		\item Soit $x \in \Ker(f)$ et $y \in \mathbbm{K}_1$ \[
				f(x \times y) = f(x) \times f(y) = 0_{\mathbbm{K}_2} \times f(y) = 0_{\mathbbm{K}_2}
			\]
		\item Soit $x \in \Ker(f) \setminus \{0_{\mathbbm{K}_1}\}$.\\
			Alors, $x$ est inversible.\\
			\begin{align*}
				\begin{rcases*}
					x \in \Ker(f)\\
					x^{-1} \in \mathbbm{K}_1
				\end{rcases*}& \text{ donc } x \times x ^{-1} \in \Ker(f)\\
				&\text{ donc } 1_{\mathbbm{K}_1} \in \Ker(f)\\
				&\text{ donc } f(1_{\mathbbm{K}_1}) = 0_{\mathbbm{K}_2}
			\end{align*}
			Or, $f(1_{\mathbbm{K}_1}) = 1_{\mathbbm{K}_2} \neq 0_{\mathbbm{K}_2}$
	\end{itemize}
	Donc, $\Ker(f) = \{0_{\mathbbm{K}_1}\}$ donc $f$ est injective.
\end{prv}

\begin{exm}
	$\begin{array}{cc}
		\C &\longrightarrow \C\\
		z &\longmapsto \overline{z}\\
	\end{array}$ est un morphisme de corps
\end{exm}



		\part{Opérations sur les séries}

\begin{prop}
	L'ensemble $E = \{u \in \C^\N  \mid \Sigma u_n \text{ converge}\}$ est un sous-espace vectoriel de $\C^\N$ et \begin{align*}
		S: E &\longrightarrow \C \\
		u &\longmapsto \sum_{n=0}^{+\infty} u_n
	\end{align*} est une forme linéaire.
	\qed
\end{prop}

\begin{rmk}
	La somme d'une série convergente et d'une série divergente diverge.
	Le produit d'une série divergente par un scalaire non nul diverge.
\end{rmk}

		\part{Comparaison de suites}

\begin{defn}
	Soient $u$ et $v$ deux suites réelles. On dit que $u$ est \underline{dominée} par  $v$ si \[
	\exists M\in \R, \exists N\in \N,\forall n\ge N,\left| u_n \right| \le M \left| v_n \right| 
	\] Dans ce cas, on note $u = O(v)$ ou $u_n = O(v_n)$ et on dit que "$u$ est un grand o de $v$"
\end{defn}

\begin{exm}
	En informatique, on dit qu'un alogirithme a une \underline{complexité linéaire} si son temps d'éxécution est un $O(n)$ 
	Par exemple, on calcule $a^n$ 

	\begin{itemize}
		\item Approche naïve
			\begin{algorithm}
				\begin{algorithmic}[1]
					\State $p \gets 1$
					\For{$i \in \left\llbracket 0,n-1 \right\rrbracket$}
						\State $p \gets p \times a$
					\EndFor
					\State \Return p
				\end{algorithmic}
			\end{algorithm}
			Complexité linéaire $O(n)$
		\item Exponentiation rapide\\
			On écrit $n$ en binaire: \begin{align*}
				n &= \overline{a_k a_{k-1}\ldots a_0}^{(2)}\\
					&= \sum_{i=0}^{k} a_i 2^i
			\end{align*} avec $(a_i) \in \left\{ 0,1 \right\} ^{k+1}$
			\begin{align*}
				a^n &= a^{\sum_{i=0}^{k} a_i 2^i} \\
				&= \prod_{i=0}^{k} a^{a_i 2^i}  \\
			\end{align*}
			
			\begin{algorithm}
				\begin{algorithmic}
					[1]

					\State $s \gets 0$
					\State $p \gets a$
					\For{ $i \in \left\llbracket 0, \log_2(n) \right\rrbracket$}
						\State $p \gets p \times p$
						\If{$a[i] = 1$}
							\State $s \gets s + p$
						\EndIf
					\EndFor
					\State \Return s
				\end{algorithmic}
			\end{algorithm}
			Compléxité logarithmique $O(\log_2(n))$
	\end{itemize}
\end{exm}


\begin{prop}
	$O$ est une relation réfléctive et transitive.
\end{prop}

\begin{prv}
	\begin{itemize}
		\item Soit $u$ une suite. On pose $M = 1$ et \[
			\forall n \in \N, \left| u_n \right| \le M \left| u_n \right|
			\] Donc $u = O(u)$.
		\item Soient $u, v, w$ trois suites telles que  \[
		\begin{cases}
			u = O(v)\\
			v = O(w)
		\end{cases}
		\] Soient $M_1,M_2 \in \R$ et $N_1,N_2\in \N$ tels que \[
		\begin{cases}
			\forall n \ge  N_1, \left| u_n \right| \le M_1 \left| v_n \right| \\
			\forall n \ge  N_2, \left| v_n \right| \le M_2 \left| w_n \right| \\
		\end{cases}
		\] 

		Nécéssairement, $M_1\ge 0$ et $M_2\ge 0$.\\
		Soit $N = \max(N_1,N_2)$. \[
		\forall n \ge  N, \left| u_n \right| \le M_1 \left| v_n \right| \le  M_1M_2 \left| w_n \right| 
		\] Donc $u = O(w)$
	\end{itemize}
\end{prv}

\begin{defn}
	Soient $u$ et $v$ deux suites. On dit que $u$ est \underline{négligeable} devant $v$ si \[
	\forall \varepsilon>0, \exists N\in \N, \forall n\ge N, \left| u_n \right| \le \varepsilon \left| v_n \right| 
	\] Dans ce cas, on note $u = o(v)$ ou $u_n = o(v_n)$ ou on le lit "$u$ est un petit o de $v$"
\end{defn}

\begin{prop}
	$o$ est une relation transitive, non-réfléctive
\end{prop}

\begin{prv}
	\begin{itemize}
		\item Soient $u$, $v$ et $w$ trois suites telles que \[
			\begin{cases}
				u = o(v)\\
				v = o(w)
			\end{cases}
			\] Soit $\varepsilon>0$. Soit $N_1\in \N$ tel que \[
			\forall n \ge N_1, \left| u_n \right| \le \sqrt{\varepsilon}  \left| v_n \right| 
			\] Soit $N_2\in \N$ tel que \[
			\forall n \ge N_2, \left| v_n \right| \le \sqrt{\varepsilon}  \left| w_n \right| 
			\] On pose $N = \max(N_1,N_2)$, alors \[
			\forall n \ge N, \left| u_n \right| \le \sqrt{\varepsilon}  \left| v_n \right| \le \underbrace{\sqrt{\varepsilon} \times \sqrt{\varepsilon}} _\varepsilon \left| w_n \right| 
			\] donc $u = o(w)$
		\item Soit $u$ une suite tel qu'il existe $N \in \N$ tel que \[
		\forall n \ge N, u_n > 0
		\] On suppose que $u = o(u)$, alors \[
		\forall \varepsilon>0,\exists N \in \N, \forall n \ge N, \left| u_n \right| \le \varepsilon \left| u_n \right| 
		\] On pose $\varepsilon = \frac{1}{2}$ alors \[
		\exists N \in \N, \forall n \ge N, \left| u_n \right| \le \frac{1}{2} \left| u_n \right| 
		\] une contradiction
	\end{itemize}
\end{prv}

\begin{prop}
	Soient $u$ et $v$ deux suites.
	\begin{itemize}
		\item $o(u) + o(u) = o(u)$
		\item $v \times o(u) = o(uv)$
		\item $o(u) \times o(v) = o(uv)$
		\item $o(o(u)) = o(u)$
	\end{itemize}
	\qed
\end{prop}

\begin{defn}
	Soient $u$ et $v$ deux suites. On dit que $u$ et $v$ sont \underline{équivalentes} si \[
	u = v + o(v)
	\] i.e. \[
	\forall \varepsilon >0, \exists N \in \N, \forall n \ge N, \left| u_n-v_n \right| \le \varepsilon\left| v_n \right| 
	\] Dans ce cas, on le note $u \sim v$
\end{defn}

\begin{prop}
	$\sim$ est une relation d'équivalence \qed
\end{prop}

\begin{prop}
	Soient $(u,v) \in \R^\N$. On suppose que $v$ ne s'annule pas à partir d'un certain rang
	\begin{enumerate}
		\item $u = o(v) \iff \left( \frac{u_n}{v_n} \right)$ bornée
		\item $u = o(v) \iff \frac{u_n}{v_n} \tendsto{n \to  +\infty} 0$
		\item $u \sim v \iff \frac{u_n}{v_n} \tendsto{n \to  +\infty} 1$
	\end{enumerate}
	\qed
\end{prop}

\begin{prop}
	[Suites de références]
	\begin{enumerate}
		\item $\ln^\alpha(n) = o(n^\beta)$ avec $(\alpha,\beta) \in \left( \R^+_* \right) ^2$ 
		\item $n^\beta = o(a^n)$ avec $\beta > 0$ et $a > 1$ 
		\item $a^n = o(n!)$ avec $a >1$ 
		\item $n! = o(n^n)$
	\end{enumerate}
\end{prop}


\begin{lem}
	[Exercice 10 du TD]
	Soit $u \in \left(\R^+_*\right)^\N$\\
	Si $\frac{u_{n+1}}{u_n} \tendsto{n \to +\infty} \ell < 1$ avec $\ell\in \R$,\\ alors $u_n \tendsto{n \to +\infty} 0$
\end{lem}

\begin{prv} [de la proposition]
	\begin{enumerate}
		\item par croissance comparée
		\item On pose $\forall n \in \N^*, u_n = \frac{n^\beta}{a^n}$. 
			\begin{align*}
				\forall  n \in \N^*, \frac{u_{n+1}}{u_n} &= \left( \frac{n+1}{n} \right) ^\beta \times \frac{1}{a} \\
				&= \frac{1}{a}\left( 1+\frac{1}{n} \right) ^\beta \\
				&\tendsto{n \to +\infty} \frac{1}{a} < 1
			\end{align*}
			Donc, $u_n \tendsto{n \to  +\infty} 0$
		\item On pose $\forall n \in \N, u_n = \frac{a^n}{n!}$ \[
			\forall n \in \N, \frac{u_{n+1}}{u_n} = \frac{a}{n+1} \tendsto{n \to +\infty} 0 < 1
			\] donc $u_n \tendsto{n \to +\infty} 0$
		\item On pose $\forall  n\in \N^*, u_n = \frac{n!}{n^n}$.
			\begin{align*}
				\forall n \in \N^*, \frac{u_{n+1}}{u_n}
				&= (n+1) {\frac{n^n}{(n+1)^{n+1}}} \\
				&= \left( \frac{n}{n+1} \right) ^n \\
				&= e^{n \ln\left( \frac{n}{n+1} \right) } \\
				&= e^{n \ln\left( 1+\frac{1}{n+1} \right)} \\
				&= e^{n(-\frac{1}{n} + o(\frac{1}{n})} \\
				&= e^{-1 + o(1)} \\
				&\tendsto{n \to  +\infty} e^{-1}<1
			\end{align*}
			donc $u_n \tendsto{n\to +\infty} 0$
	\end{enumerate}
\end{prv}

	}

	{
		\chap[20]{Fractions rationnelles}
		\renewcommand{\cwd}{../chap20}
		\begin{defn}
	Soit $E$ un $\mathbbm{K}$-espace vectoriel. On dit que $E$ est de \underline{dimension finie} si $E$ a au moins une famille génératrice finie. On dit que $E$ est de \underline{dimension infinie} sinon.
	\index{dimension finie (espace vectoriel)}
	\index{dimension infinie (espace vectoriel)}
\end{defn}

\begin{thm}
	[Théorème de la base extraite]
	Soit $E$ un $\mathbbm{K}$-espace vectoriel non nul de dimension finie. Soit $\mathcal{G}$ une famille génératrice finie de $E$. Alors, il existe une base $\mathcal{B}$ de $\mathcal{E}$ telle que $\mathcal{B} \subset \mathcal{G}$.
\end{thm}

\begin{prv}
	[par récurrence sur $\#G = \Card(G)$]
	\begin{itemize}
		\item Soit $E$ un $\mathbbm{K}$-espace vectoriel non nul engendré par $\mathcal{G} = (u)$.\\
			Si $u = 0_E$, alors $E = \{0_E\}$: une contradiction $\lightning$ \\
			Donc $u \neq 0_E$ donc $(u)$ est libre. En effet, \[
				\forall \lambda \in \mathbbm{K}, \lambda u = 0_E \implies \lambda = 0_\mathbbm{K}
			\] Donc $\mathcal{G}$ est une base de $E$.\\
		\item Soit $n \in \N_*$. Soit $E$ un $\mathbbm{K}$-espace vectoriel. On suppose que si $E$ a une famille génératrice constituée de $n$ vecteurs, alors on peut extraire de cette famille une base de $E$.\\
			Soit $\mathcal{G}$ une famille génératrice de $E$ avec $n+1$ vecteurs.\\
			Si $\mathcal{G}$ est libre, alors $\mathcal{G}$ est une base de $E$. \\
			Si $\mathcal{G}$ n'est pas libre, alors il existe $u \in \mathcal{G}$ tel que $u \in \Vect(\mathcal{G}\setminus \{u\})$ \\
			Donc $\mathcal{G}\setminus \{u\}$ engendre $E$. Or, $\mathcal{G}\setminus \{u\}$ possède $n$ vecteurs. D'après l'hypothèse de récurrence, il existe une base $\mathcal{B}$ de $E$ telle que \[
				\mathcal{B} \subset \mathcal{G} \setminus \{u\} \subset \mathcal{G}
			\] 
	\end{itemize}
\end{prv}

\begin{crlr}
	Tout espace de dimension finie a une base.
	\qed
\end{crlr}

\begin{thm}
	[Théorème de la base incomplète]
	Soit $E$ un $\mathbbm{K}$-espace vectoriel de dimension finie, $\mathcal{G}$ une famille génératrice finie de $E$. $\mathcal{L}$ une famille libre de $E$. Alors, il existe une base $\mathcal{B}$ de $E$ telle que \[
		\mathcal{L} \subset \mathcal{B} \text{ et } \mathcal{B}\setminus \mathcal{L} \subset \mathcal{G}
	\] 
\end{thm}

\begin{prv}
	[par récurrence sur $\#(\mathcal{G}\setminus\mathcal{L})$]
	\begin{itemize}
		\item Avec les notations précédentes, on suppose que $\mathcal{G}\setminus\mathcal{L} \neq \O$ \[
				\forall u \in \mathcal{G}, u \in \mathcal{L}
			\] Donc $\mathcal{G} \subset \mathcal{L}$ donc $\mathcal{L}$ est génératrice donc $\mathcal{L}$ est une base de $E$. On pose $\mathcal{B} = \mathcal{L}$ et alors \[
				\mathcal{L} \subset  \mathcal{B} \text{ et } \mathcal{B}\setminus\mathcal{L} = \O \subset  \mathcal{G}
			\] 
		\item Soit $n \in \N$. On suppose que si $\mathcal{G}$ est génératrice et $\mathcal{L}$ libre avec $\#(\mathcal{G}\setminus\mathcal{L}) = n$ alors il existe une base $\mathcal{B}$ de $E$ telle que \[
			\mathcal{L}\subset \mathcal{B} \text{ et } \mathcal{B}\setminus\mathcal{L}\subset \mathcal{G}
		\] Soient à présent $\mathcal{G}$ une famille génératrice de $E$ et $\mathcal{L}$ une famille libre de $E$ telles que $\#(\mathcal{G}\setminus\mathcal{L}) = n+1 > 0$\\
		Si $\mathcal{L}$ engendre $E$, alors $\mathcal{L}$ est une base de $E$. On pose $\mathcal{B} = \mathcal{L}$ et on a bien \[
			\mathcal{L} \subset  \mathcal{B} \text{ et } \mathcal{B} \setminus \mathcal{L} = \O \subset  \mathcal{G}
		\] On suppose que $\mathcal{L}$ n'engendre pas $E$. Il existe $u \in \mathcal{G}$ tel que $u \not\in \Vec(\mathcal{L})$ (car sinon, $\mathcal{G} \subset \Vect(\mathcal{L})$ et donc $\underbrace{\Vect(\mathcal{G})}_{= E} \subset  \underbrace{\Vect(\mathcal{L})}_{ \subset E}$\\
		Donc $\mathcal{L} \cup \{u\} $ est libre. On pose $\mathcal{L}' = \mathcal{L} \cup \{u\} $ \[
			\mathcal{G}\setminus \mathcal{L}' = \mathcal{G}\setminus (\mathcal{L} \cup \{u\}) = (\mathcal{G}\setminus\mathcal{L})\setminus \{u\} 
		\] donc $\#(\mathcal{G}\setminus\mathcal{L}') = n+1 -1 = n$\\
		D'après l'hypothèse de récurrence, il existe $\mathcal{B}$ une base de $E$ telle que \[
			\mathcal{L} \subset  \mathcal{L}' \subset \mathcal{B} \text{ et } \mathcal{B}\setminus \mathcal{L}' \subset \mathcal{G}
		\] \[
			\mathcal{B} \setminus \mathcal{L} = \underbrace{\mathcal{B}\setminus\mathcal{L}'}_{\subset \mathcal{G}} \cup \underbrace{\{u\}}_{\subset \mathcal{G} \text{ car } u \in \mathcal{G}}
		\] On a $\mathcal{B}\setminus\mathcal{L}\subset \mathcal{G}$
	\end{itemize}
\end{prv}

\begin{thm}
	Soit $E$ un $\mathbbm{K}$-espace vectoriel de dimension finie. Toutes les bases de $E$ ont le même cardinal.
\end{thm}

\begin{prv}
	Soit $\mathcal{G}$ une famille génératrice finie de $E$ et $\mathcal{B} \subset  \mathcal{G}$ une base de $E$. On note $n = \#\mathcal{B}$ \\
	Soit $\mathcal{B}'$ une base de $E$. On pose $p = n - \#(\mathcal{B} \cap  \mathcal{B}')$. Montrons par récurrence sur  $p$ que $\#\mathcal{B} = \#\mathcal{B}'$ 
	\begin{itemize}
		\item On suppose que $p = 0$. Alors, $\#(\mathcal{B} \cap \mathcal{B}') = n$ \\
			Or, $\mathcal{B}' \cap \mathcal{B} \subset \mathcal{B}$ donc $\mathcal{B} \cap \mathcal{B}' = \mathcal{B}$ donc $\mathcal{B} \subset  \mathcal{B}'$ et donc $\mathcal{B} = \mathcal{B}'$ 
		\item Soit $p \in \N$. On suppose que si $\mathcal{B}'$ est une base de $E$ telle que $n - \#(\mathcal{B} \cap \mathcal{B}') = p$, alors $\#\mathcal{B}' = n$ \\
			Aoit $\mathcal{B}'$ une base de $E$ telle que $n - \#(\mathcal{B}\cap \mathcal{B}') = p+1 > 0$ \\
			Donc $\mathcal{B} \cap \mathcal{B}' \neq \mathcal{B}$. Soit $u \in \mathcal{B}' \setminus \mathcal{B}$. D'après le lemme d'échange, il existe $v \in \mathcal{B}\setminus \mathcal{B}'$ tel que $\mathcal{B}' \setminus \{u\} \cup \{v\}$ est une base de $E$. On pose $\mathcal{B}'' = \mathcal{B}' \setminus \{u\} \cup \{v\}$ 
			\begin{align*}
				\mathcal{B}'' \cap \mathcal{B} &= \left( (\mathcal{B}' \setminus \{u\})  \cap \mathcal{B} \right) \cup \{v\} \\
				&= (\mathcal{B}' \cap \mathcal{B}) \cup \{v\} \\
			\end{align*}
			donc,
			\begin{align*}
				n - \#(\mathcal{B}'' \cap \mathcal{B}) &= n - (\#(\mathcal{B}' \cap \mathcal{B}) + 1) \\
				&= p+1- 1 \\
				&= p \\
			\end{align*}
			D'après l'hypothèse de récurrence, \[
				\#\mathcal{B}'' = n
			\] Or, $\#\mathcal{B}'' = \#\mathcal{B}'$
	\end{itemize}
\end{prv}

\begin{lem}
	Soient $\mathcal{B}$ et $\mathcal{B}'$ deux bases de $E$ telles que $\mathcal{B}\subset \mathcal{B}'$. Alors, $\mathcal{B} = \mathcal{B}'$.
\end{lem}

\begin{prv}
	On suppose $\mathcal{B}' \neq \mathcal{B}$. Soit $u \in \mathcal{B}' \setminus \mathcal{B}$
	$u \in E = \Vect(\mathcal{B})$ donc $\mathcal{B} \cup \{u\}$ n'est pas libre.
	Donc $\mathcal{B}\cup \{u\} \subset \mathcal{B}'$ et $\mathcal{B}'$ est libre donc $\mathcal{B}\cup \{u\}$ est libre: une contradiction $\lightning$
\end{prv}

\begin{lem}
	[Lemme d'échange] Soient $\mathcal{B}_1$ et $\mathcal{B}_2$ deux bases de $E$ et $u \in \mathcal{B}_1 \setminus \mathcal{B}_2$. Alors, il existe $v \in \mathcal{B}_2$ tel que $(\mathcal{B}_1 \setminus \{u\}) \cup \{v\}$ soit une base de $E$.
\end{lem}

\begin{prv}
	[1${}^\text{nde}$ méthode]
	On suppose que pout tout $v \in \mathcal{B}_2$, $(\mathcal{B}_1\setminus \{u\}) \cup \{v\}$ n'est pas une base de $E$
	Soit $v \in \mathcal{B}_2$.
	\begin{itemize}
		\item Supposons $(\mathcal{B}_1\setminus \{u\})\cup \{v\}$ non libre. $\mathcal{B}_1 \setminus \{u\}$ est libre. Donc $v \in \Vect(\mathcal{B}_1 \setminus \{u\})$
		\item Supposons $(\mathcal{B}_1\setminus \{u\}) \cup \{v\}$ non génératrice.
			Comme $\mathcal{B}_1$ engendre $E$, $u \not\in \Vect(\mathcal{B}_1\setminus \{v\})$.
			On suppose que $\mathcal{B}_1 \neq \mathcal{B}_2$.
			$\forall v \in \mathcal{B}_2 \setminus \mathcal{B}_1, \Vect(\mathcal{B}_1 \setminus \{v\}) = \Vect(\mathcal{B}_1) = E \ni u$ 
			donc, $(\mathcal{B}_1\setminus \{u\}) \cup \{v\}$ engendre $E$ et donc \[
				v \in \Vect(\mathcal{B}_1 \setminus \{u\})
			\] On a aussi \[
				\forall v \in \mathcal{B}_1 \setminus \{u\}, v \in \Vect(\mathcal{B}_1\setminus \{u\})
			\] Comme $u \not\in \mathcal{B}_2$, on a \[
				\forall v \in \mathcal{B}_2, v \in \Vect(\mathcal{B}_1\setminus \{u\})
			\] docn \[
				E = \Vect(\mathcal{B}_2) \subset \Vect(\mathcal{B}_1\setminus \{u\})
			\] donc $\mathcal{B}_1\setminus \{u\}$ engendre $E$ donc $\mathcal{B}_1\setminus \{u\}$ est une base de $E$. Or, $\mathcal{B}_1 \setminus \{u\}  \subset  \mathcal{B}_1$, donc $\mathcal{B}_1\setminus \{u\} = \mathcal{B}_1$
	\end{itemize}
\end{prv}

\begin{prv}
	[2${}^\text{nde}$ méthode]
	On suppose que pout tout $v \in \mathcal{B}_2$, $(\mathcal{B}_1\setminus \{u\}) \cup \{v\}$ n'est pas une base de $E$
	\begin{itemize}
		\item Comme $u \in \mathcal{B}_1 \setminus \mathcal{B}_2$, nécéssairement $\mathcal{B}_1 \neq \mathcal{B}_2$ donc $\mathcal{B}_2 \not\subset \mathcal{B}_1$, donc $\mathcal{B}_2\setminus\mathcal{B}_1 \neq \O$ 
		\item Soit $v \in \mathcal{B}_2\setminus\mathcal{B}_1$. Il existe $(\lambda_w)_{w\in\mathcal{B}_1}$ une famille de scalaires presque nulle telle que \[
				v = \sum_{w \in \mathcal{B}_1} \lambda_w w - \lambda_u u + + \sum_{w \in \mathcal{B}_1\setminus \{u\}}\lambda_w w
			\]
			Si $\lambda_u \neq 0_E$, alors
			\begin{align*}
				u &= \lambda_u^{-1}\left( v - \sum_{w \in \mathcal{B}_1 \setminus \{u\}} \lambda_w w \right)\\
					&\in \Vect(\mathcal{B}_1\setminus \{u\} \cup v)
			\end{align*}
			 donc $\mathcal{B}_1 \subset \Vect(\mathcal{B}_1\setminus \{u\} \cup \{v\})$\\
			 et donc $E \subset  \Vect(\mathcal{B}_1 \setminus \{u\} \cup \{v\})$ \\
			 et donc $\mathcal{B}_1 \setminus \{u\} \cup \{v\}$ engendre $E$ \\
			 donc $\mathcal{B}_1 \setminus \{u\} \cup \{v\}$ n'est pas libre\\
			 donc $v \in \Vect(\mathcal{B}_1\setminus \{u\})$ (car $\mathcal{B}_1 \setminus \{u\}$ est libre\\
			 donc $\lambda_u = 0_\mathbbm{K}$ $\lightning$\\`

			 Donc, $\lambda_u = 0_\mathbbm{K}$, docn $v \in \Vect(\mathcal{B}_1\setminus \{u\})$ \\
			 On vient de prouver que
			 \begin{align*}
			 	\mathcal{B}_2 \setminus \mathcal{B}_1 \subset \Vect(\mathcal{B}_1 \setminus \{u\})\\
			 	\mathcal{B}_1 \setminus \{u\} \subset \Vect(\mathcal{B}_1 \setminus \{u\})\\
			 \end{align*}
			 Comme $u \not\in \mathcal{B}_2$, \[
			 	\mathcal{B}_2 \subset \Vect(\mathcal{B}_1 \setminus \{u\})
			 \] donc \[
			 	E = \Vect(\mathcal{B}_2) \subset  \Vect(\mathcal{B}_1 \setminus \{u\})
			 \] donc $\mathcal{B}_1 \setminus \{u\}$ engendre $E$. Donc,  $\mathcal{B}_1 \setminus \{u\}$ est une base de $E$.\\
			 Or, $\mathcal{B}_1 \setminus \{u\} \subset  \mathcal{B}_1$, donc $\mathcal{B}_1 \setminus \{u\} = \mathcal{B}_1$
	\end{itemize}
\end{prv}

\begin{defn}
	Soit $E$ un $\mathbbm{K}$-espace vectoriel de dimension finie. Le cardinal commun à toutes les bases de $E$ est appelé \underline{dimension} de $E$ est notée $\dim(E)$ ou $\dim_\mathbbm{K}(E)$\\
	C'est donc aussi le nombre de coordonnées de n'importe quel vecteur dans n'importe quelle base.
	\index{dimension (espace vectoriel)}
\end{defn}

\begin{exm}
	\begin{enumerate}
		\item $\dim_\R(\C) = 2$ et $\dim_\C(\C) = 1$ 
		\item $\dim_\mathbbm{K}(\mathbbm{K}^{n}) = n$ 
		\item $\dim_{\mathbbm{K}}(\mathcal{M}_{n,p}(\mathbbm{K})) = np$
	\end{enumerate}
\end{exm}

\begin{crlr}
	Soit $E$ un $\mathbbm{K}$-espace vectoriel de dimension finie, $\mathcal{L}$ une famille libre de $E$, $\mathcal{G}$ une famille génératrice de $E$. On note $n = \dim(E)$
	\begin{enumerate}
		\item $\#\mathcal{G} \ge n$ et $(\#\mathcal{G} = n \implies \mathcal{G} \text{ est une base de } E$)
		\item $\#\mathcal{L} \le n$ et $(\#\mathcal{L} = n \implies \mathcal{L} \text{ est une base de } E$)
	\end{enumerate}
\end{crlr}

\begin{crlr}
	$\R^{\R}$ est de dimension infinie.
	$\forall i \in \N, e_i: x \mapsto x^i$\\
	$(e_i)_{i\in\N}$ est libre dans $\R^\R$
\end{crlr}

\begin{prop}
	Soient $E$ et $F$ deux $\mathbbm{K}$-espaces vectoriels de dimension finie. Alors $E\times F$ est de dimension finie et $\dim(E\times F) = \dim(E) + \dim(F)$
\end{prop}

\begin{prv}
	Soit $(e_1,\ldots, e_n)$ une base de $E$, $(f_1, \ldots, f_p)$ une base de $F$.
	On pose \[
		\left\{\begin{array}
			{r c l}
			u_1 &=& (e_1,0_F)\\
			u_2 &=& (e_2,0_F)\\
					&\vdots&\\
			u_n &=& (e_n,0_F)\\
			u_{n+1} &=& (0_E, f_1)\\
			u_{n+2} &=& (0_E, f_2)\\
					&\vdots&\\
			u_{n+p} &=& (0_E,f_p)\\
		\end{array}\right.
	\]
	Soit $(x,y) \in E\times F$. \[
		\begin{cases}
			\exists (x_1,\ldots,x_n)\in \mathbbm{K}^n, x = \sum_{i=1}^{n} x_ie_i
			\exists (y_1,\ldots,y_n)\in \mathbbm{K}^n, x = \sum_{j=1}^{p} y_jf_j
		\end{cases}
	\] 
	\begin{align*}
		(x,y) &= \left( \sum_{i=1}^{n} x_ie_i, \sum_{i=1}^{p} y_jf_j \right)  \\
		&= \sum_{i=1}^{n} x_i (e_i + 0_F) + \sum_{j=1}^{p} y_j (0_E, f_j) \\
		&= \sum_{i=1}^{n} x_i u_i + \sum_{j=1}^{p} y_j u_{n+j} \\
	\end{align*}
	Donc, $E\times F = \Vect(u_1, \ldots, u_{n+p})$ donc $E\times F$ est de dimension finie.\\
	Soit $(\lambda_1, \ldots, \lambda_{n+p}) \in \mathbbm{K}^{n+p}$ tel que \[
		(*): \quad \sum_{k=1}^{n+p} \lambda_ku_k = 0_{E\times F} = (0_E, 0_F)
	\]
	\begin{align*}
		(*) &\iff \sum_{k=1}^{n} \lambda_k (e_k, 0_F) + \sum_{k=n+1}^{p} \lambda_k(0_E, f_{k-n}) = (0_E, 0_F)\\
				&\iff \begin{cases}
					\sum_{k=1}^{n} \lambda_k e_k = 0_E\\
					\sum_{k=n+1}^{p} \lambda_k f_{k-n} = 0_F
				\end{cases}\\
				&\iff \begin{cases}
					\forall k \in \left\llbracket 1,n \right\rrbracket, \lambda_k = 0_\mathbbm{K} \qquad&(\text{car $(e_1,\ldots,e_n)$ est libre})\\
					\forall k \in \left\llbracket n+1,n+p \right\rrbracket, \lambda_k = 0_\mathbbm{K} \qquad&(\text{car $(f_1,\ldots,f_n)$ est libre})\\
				\end{cases}
	\end{align*}
	Donc $(u_1, \ldots, u_{n+p})$ est une base de $E\times F$. Donc, $\dim(E\times F) = n + p = \dim(E) + \dim(F)$
\end{prv}

\begin{rmk}
	[Convention]
	\[\dim\big(\{0_E\}\big) = 0\]
\end{rmk}

\begin{thm}
	Soit $E$ un $\mathbbm{K}$-espace vectoriel de dimension finie, $F$ un sous-espace vectoriel de $E$. Alors, $F$ est de dimension finie et  $\dim(F) \le \dim(E)$\\
	Si $\dim(F) = \dim(E)$, alors $F = E$
\end{thm}

\begin{prv}
	On considère \[
		A = \{k \in \N \mid \text{il existe une famille libre de $F$ à $k$ éléments}\} 
	\]
	On suppose $F \neq \{0_E\}$.
	\begin{itemize}
		\item Soit $u \in F\setminus \{0_E\}$. $(u)$ est libre donc $1 \in A$ et donc $A \neq \O$
		\item Soit $\mathcal{L}$ une famille libre de $F$. Alors, $\mathcal{L}$ est une famille libre de $E$ \\
			donc $\#\mathcal{L} \le \dim(E)$\\
			Donc $A$ est majorée par $\dim(E)$ \\
			On en déduit que $A$ a un plus grand élément $p$.
		\item Soit $\mathcal{L}$ une famille libre de $F$ avec $p$ éléments.\\
			Si $\mathcal{L}$ n'engendre pas $F$, alors il existe $u\in F$ tel que $u\not\in \Vect(\mathcal{L})$ et donc $\mathcal{L} \cup \{u\}$ est une famille libre de $F$, donc $p+1 \in A$ en contradiction avec la maximalité de $p$.\\
			Donc $\mathcal{L}$ est une base de $F$ donc $F$ est de dimension finie et $\dim(F) = p \le \dim(E)$\\
	\end{itemize}

	Soit $\mathcal{B}$ une base de $F$. Alors, $\mathcal{B}$ est aussi une famille de libre de de $E$. Donc $\#\mathcal{B} \le \dim(E)$ donc $\dim(F) = \dim(E)$ \\
	Si $\dim(F) = \dim(E)$, alors $\mathcal{B}$ est une base de $E$, et donc $F = \Vect(\mathcal{B}) = E$
\end{prv}

\begin{prop}
	[Formule de Grassmann]
	Soit $E$ un $\mathbbm{K}$-espace vectoriel de dimension finie, $F$ et $G$ deux sous-espace vectoriels de $E$. Alors, \[
		\dim(F+G) = \dim(F) + \dim(G) - \dim(F\cap G)
	\] 
\end{prop}

\begin{prv}
	Soit $(e_1, \ldots, e_p)$ une base de $F\cap G$. $(e_1,\ldots,e_p)$ est une famille libre de $F$.\\
	On complète $(e_1, \ldots, e_p)$ en une base $(e_1, \ldots, e_p, u_1, \ldots, u_q)$ de $F$.\\
	De même, on complète $(e_1, \ldots, e_p)$ en une base $(e_1, \ldots, e_p, v_1, \ldots, v_r)$ de $G$.\\
	On pose  $\mathcal{B} = (e_1, \ldots, e_p, u_1, \ldots, u_q, v_1, \ldots, v_r)$. Montrons que $\mathcal{B}$ est une base de $F+G$
	\begin{itemize}
		\item Soit $u \in F+G$ \\
			On pose $u = v+w$ avec $\begin{cases}
				v\in F\\
				w \in G
			\end{cases}$.\\
			On pose $v = \sum_{i=1}^p \lambda_i e_i + \sum_{i=1}^q \mu_i u_i$ avec $(\lambda_1, \ldots, \lambda_p, \mu_1, \ldots, \lambda_q) \in \mathbbm{K}^{p+q}$\\
			On pose aussi $w = \sum_{i = 1}^p \lambda'_ie_i + \sum_{j=1}^r \nu_j v_j$ avec $(\lambda_1',\ldots,\lambda_p', \nu_1, \ldots, \nu_r) \in \mathbbm{K}^{p+r}$\\
			D'où, \[
				u = \sum_{i=1}^p (\lambda_i + \lambda'_i)e_i + \sum_{j=1}^q \mu_j u_j + \sum_{k=1}^r \nu_k v_k \in \Vect(\mathcal{B})
			\]
		\item Soient $(\lambda_1, \ldots, \lambda_p, \mu_1, \ldots, \mu_q, \nu_1, \ldots, \nu_r) \in \mathbbm{K}^{p+q+r}$.\\
			On suppose \[
				(*)\quad \sum_{i=1}^{p}\lambda_ie_i + \sum_{j=1}^q\mu_ju_j + \sum_{k=1}^r \nu_k v_k = 0_E
			\] 
			D'où, \[
				\underbrace{\sum_{i=1}^p\lambda_i e_i + \sum_{j=1}^q \mu_ju_j}_{\in F} = \underbrace{-\sum_{k=1}^r\nu_jv_k}_{\in G}
			\] 
			Donc, \[
				f = \sum_{i=1}^p \lambda_i e_i + \sum_{j=1}^q \mu_j u_j \in F\cap G
			\] Comme $(e_1, \ldots, e_p)$ est une base de $F\cap G$, $\exists ! (\lambda_1', \ldots, \lambda_p') \in \mathbbm{K}^p$ tel que \[
				f = \sum_{i=1}^p \lambda'_i e_i = \sum_{i=1}^p \lambda'_i e_i + \sum_{j=1}^q 0_\mathbbm{K}u_j
			\] Comme $(e_1, \ldots, e_p, u_1, \ldots, u_q)$ est une base de $F$, \[
				\forall k \in \left\llbracket 1, q \right\rrbracket, \mu_j = 0_\mathbbm{K}
			\] De même, \[
				\forall k \in \left\llbracket 1,r \right\rrbracket , \nu_k = 0_\mathbbm{K}
			\] On remplace dans $(*)$ pour trouver \[
				\sum_{i=1}^p \lambda_ie_i = 0_E
			\] Comme $(e_1, \ldots, e_p)$ est libre, \[
				\forall i \in \left\llbracket 1,p \right\rrbracket, \lambda_i = 0_\mathbbm{K}
			\] Donc $\mathcal{B}$ est libre.\\
			Donc, 
			\begin{align*}
				\dim(F+G) &=  p +q + r \\
				&= (p+q)+ (p+r) - p \\
				&= \dim(F) + \dim(G) - \dim(F\cap G) \\
			\end{align*}
	\end{itemize}
\end{prv}

\begin{crlr}
	Avec les hypothèse précédentes, \[
		E = F \oplus G \iff \begin{cases}
			F \cap  G = \{0_E\} \\
			\dim(E) = \dim(F) + \dim(G)
		\end{cases}
	\] 
\end{crlr}

\begin{prv}
	\begin{itemize}
		\item[``$\implies$''] On suppose $E = F \oplus G$ \\
			Comme la somme est directe, $F \cap G = \{0_E\}$ 
			\begin{align*}
				\dim(E) &= \dim(F)\\
				&= \dim(F) + \dim(G) - \dim(F\cap G)\\
				&= \dim(F) + \dim(G)\\
			\end{align*}
		\item[``$\impliedby$''] On suppose $F\cap G = \{0_E\}$ et $\dim(E) = \dim(F) + \dim(G)$.\\
			On sait déjà que $F+G = F \oplus G$\\
			 \begin{align*}
				\dim(F+G) = \dim(F) + \dim(G) - \dim(F \cap G) = \dim(E)
			\end{align*}
			Donc $F + G = E$
	\end{itemize}
\end{prv}

\begin{prop}
	Soit $F$ un $\mathbbm{K}$-espace vectoriel de dimension finie $n$. Soit $\mathcal{B} = (e_1, \ldots, e_n)$ une base de $F$. L'application
	\begin{align*}
		f: \mathbbm{K}^n &\longrightarrow F \\
		(\lambda_1, \ldots, \lambda_n) &\longmapsto \sum_{i=1}^n \lambda_i e_i
	\end{align*} est bijective.\\
	Si $\mathbbm{K}$ est infini, $\mathbbm{K}^n$ aussi et donc $F$ aussi.\\
	Si $\#\mathbbm{K} = p \in \N_*$,
	\begin{align*}
		\#&\mathbbm{K}^n = p^n\\
		&\vrt=\\
		\#&F
	\end{align*}
\end{prop}


		\part{Dérivation}

\underline{Motivation}:

{
\begin{wrapfigure}{l}{3cm}
	\centering
	\begin{asy}
		import three;

		size(3cm);
		settings.render=0;
		settings.prc=false;
		currentprojection = obliqueZ;

		draw(unitbox);
		draw(shift(1.1Z + 0.05X) * (O -- X), Arrows3(TeXHead2));
		draw(shift(1.1Z + 0.05Y) * (O -- Y), Arrows3(TeXHead2));
		draw(shift(1.1X + 0.05Z) * (O -- Z), Arrows3(TeXHead2));

		label("$x$", (X/2) + (1.1Z + 0.05X), align=S);
		label("$y$", (Y/2) + (1.1Z + 0.05Y), align=W);
		label("$z$", (Z/2) + X, align=SE);
	\end{asy}
\end{wrapfigure}

\begin{align*}
	&S(x,y,z) = 2(xy + xz + yz)\\
	&V(x,y,z) = xyz
\end{align*}

On cherche à minimiser $S$ avec la contrainte $V = 1$.

Soit $f : \begin{array}{rcl}
	\left( \R_*^+ \right)^2 &\longrightarrow& \R \\
	(x,y) &\longmapsto& S\left( x,y,\frac{1}{xy} \right) = 2\left( xy + \frac{1}{y} + \frac{1}{x} \right).
\end{array}$

On cherche $(a,b) \in \left( \R^+_* \right)^2$ tel que \[
	\forall (x,y) \in (\R^+_*), f(x,y) \ge f(a,b).
\]
}

\begin{defn}
	Soit $f: U \to \R$ où $U$ est un ouvert de $\R^2$. Soit $(a,b) \in U$.
	\vspace{2mm}

	Si $\lim_{x \to a} \frac{f(x,b) - f(a,b)}{x - a} \in \R$, alors on dit que $f$ a une dérivée partielle suivant $x$ en $(a,b)$ et cette limite est notée \[
		\partial f_1(a,b) = \frac{\partial f}{\partial x}(a,b).
	\]

	Si $\lim_{y \to b} \frac{f(a,y) - f(a,b)}{y - b} \in \R$, alors on dit que $f$ a une dérivée partielle suivant $y$ et la limite est notée \[
		\partial f_2(a,b) = \frac{\partial f}{\partial y}(a,b).
	\]
\end{defn}

\begin{exm}
	\begin{enumerate}
		\item $f: (x,y) \mapsto xy + x - y$.

			\begin{align*}
				&\frac{\partial f}{\partial x} : (x,y) \mapsto y + 1,\\
				&\frac{\partial f}{\partial y} : (x,y) \mapsto x - 1.
			\end{align*}

		\item $f: (x,y) \mapsto xy + \frac{1}{y}+ \frac{1}{x}$.

			\begin{align*}
				&\frac{\partial f}{\partial x}: (x,y) \mapsto y - \frac{1}{x^2},\\
				&\frac{\partial f}{\partial y}: (x,y) \mapsto x - \frac{1}{y^2}.
			\end{align*}

		\item Trouver $f$ telle que $\begin{cases}
				(1): \qquad \frac{\partial f}{\partial x}=y,\\[2mm]
				(2): \qquad \frac{\partial f}{\partial y} = x.
			\end{cases}$

			D'après $(1)$ : \[
				\forall (x,y), \exists C(y) \in \R, f(x,y) = xy + C(y)
			\] et donc \[
				\frac{\partial f}{\partial y}(x,y) = x + C'(y)
			\] donc $C'(y) = 0$ et donc $C$ est constante.

		\item Trouver $f$ telle que $\begin{cases}
			\frac{\partial f}{\partial x} = -y,\\[2mm]
			\frac{\partial f}{ƒ\partial y} = x.
		\end{cases}$

		Ce n'est pas possible !
	\end{enumerate}
\end{exm}

\begin{defn}~\\
	\begin{minipage}{\linewidth}
		\begin{wrapfigure}{r}{4cm}
			\centering
			\vspace{-5mm}
			\begin{asy}
				import three;
				import graph3;
				size(4cm);

				settings.render = 0;
				settings.prc = false;
				currentprojection = obliqueX;

				draw(O -- X, Arrow3(TeXHead2));
				draw(O -- Y, Arrow3(TeXHead2));
				draw(O -- Z, Arrow3(TeXHead2));

				triple f(real x, real y, real z = 0) { return (x,y,cos(x - 0.5) * cos(y - 0.5)/1.2 + 0.15); }

				real inc = 1 / 5;

				for(real x = 0; x <= 1; x += inc) {
					draw(graph(
						new real(real t) { return x; }, // x
						new real(real y) { return y; }, // y
						new real(real y) { return f(x,y).z; }, // z
						0, 1
					), gray);
				}

				for(real y = 0; y <= 1; y += inc) {
					draw(graph(
						new real(real x) { return x; }, // x
						new real(real t) { return y; }, // y
						new real(real x) { return f(x,y).z; }, // z
						0, 1
					), gray);
				}

				path3 path1 = (0.8, 0.2, 0) .. (0.5, 0.5, 0) .. (0.3, 0.7, 0);
				path3 path2 = f(0.8, 0.2, 0) .. f(0.5, 0.5, 0) .. f(0.3, 0.7, 0);
				path3 d = (0.2, 0.3, 0) .. (0.3, 0.4, 0) .. (0.2, 0.7, 0) .. (0.8, 0.9, 0) .. (0.6, 0.2, 0) .. cycle;

				draw(path1, red, Arrow3(TeXHead2));
				draw(path2, red, Arrow3(TeXHead2, position=0.8));

				dot((0.5, 0.5, 0));
				dot(f(0.5, 0.5, 0));
				draw((0.5, 0.5, 0) -- f(0.5, 0.5, 0), dashed);
				draw(d);

				label("$w$", (0.3, 0.7, 0), red, align=SE);
				label("$U$", (0.8, 0.9, 0), align=SE);
			\end{asy}
		\end{wrapfigure}

		Soit $f: U \to \R$ où $U$ est un ouvert. Soit $(a,b) \in U$. Soit $w = (w_1, w_2) \in \R^2$.

		Si 
		\[
			\lim_{t\to 0} \frac{f(a + tw_1, b + tw_2) - f(a,b)}{t}
		\] existe et est réelle, alors on dit que $f$ a une dérivée dans la direction de $w$ et la limite est notée \[
			\mathrm{d}f(w)\,(a,b) = D_w(f)\,(a,b).
		\]
	\end{minipage}
\end{defn}

\begin{exm}
	\begin{align*}
		f: \left( \R_*^+ \right)^2 &\longrightarrow \R \\
		(x,y) &\longmapsto xy+\frac{1}{x}+\frac{1}{y}.
	\end{align*}

	On pose $(a,b) = (1,2)$, $w = (w_1, w_2) = (1,1)$.
	\begin{align*}
		\frac{f(1+t, 2+t) - f(1,2)}{t} &= \frac{1}{t} \left( (1+t)(2+t) + \frac{1}{1+t} + \frac{1}{2+t} - 3 - \frac{1}{2} \right) \\
		&= \frac{1}{t}\left(\cancel 2 + 3t + \po(t) + \cancel 1 - t + \po(t) + \frac{1}{2}\left( \cancel 1 - \frac{t}{2} + \po(t) \right) - \cancel3 - \cancel{\frac{1}{2}} \right) \\
		&= \frac{1}{t} \left( \frac{7}{4} t + \po(t) \right)  \\
		&= \frac{7}{4} + \po(1) \tendsto{t \to 0} \frac{7}{4}. \\
	\end{align*}

	Donc, \[
		\mathrm{d}f(1,1)\,(1,2) = \frac{7}{4}.
	\]
\end{exm}

\begin{rmk}~\\
	\begin{figure}[H]
		\centering
		\begin{asy}
			import solids;
			import graph;
			size(5cm);

			settings.render = 0;
			settings.prc = false;

			path3 par = graph(
				new real(real x) { return x; },
				new real(real x) { return 0; },
				new real(real x) { return x^2; },
				0,3);
			revolution r = revolution(par, axis=Z);

			path3 par2 = graph(
				new real(real x) { return x; },
				new real(real x) { return 0; },
				new real(real x) { return x^2; },
				-3,3);

			draw(r,1,longitudinalpen=nullpen);
			draw(r.silhouette());

			draw((-4, 0, -1) -- (-4, 0, 10) -- (4, 0, 10) -- (4, 0, -1) -- cycle, red);
			draw(par2, deepred);

			draw((4,4.5) -- (7, 4.5), black+0.5mm, Arrow(TeXHead));

			path par2d = graph(new real(real x) { return x^2; }, -3, 3);
			draw(shift((11, 0)) * par2d, deepred);

			dot(O);
			dot((11, 0));
		\end{asy}
	\end{figure}
\end{rmk}


%todo ajouter théorème-définition
\begin{thm}
	Soit $f : U \to \R$, $(a,b) \in U$. On suppose que $\frac{\partial f}{\partial x}$ et $\frac{\partial f}{\partial y}$ existent en $(a,b)$ et sont {\bfseries continues} en $(a,b)$. Alors,
	\begin{align*}
		&\forall (h, k) \in \R^2 \text{ tel que } (a +h, b + k) \in U,\\
		&f(a+ h, b + k) = f(a,b) + h \frac{\partial f}{\partial x}(a,b) + k \frac{\partial f}{\partial y}(a,b) + \po_{(h,k)\to (0,0)}\big(\|(h,k)\|\big).
	\end{align*}

	On dit que $f$ est de classe $\mathcal{C}^1$ si $\frac{\partial f}{\partial x}$ et $\frac{\partial f}{\partial y}$ existent et sont continues.

	\qed
\end{thm}

\begin{rmk}
	En physique, cette formule correspond à : \[
		\mathrm{d}f = \frac{\partial f}{\partial x}\mathrm{d}x + \frac{\partial f}{\partial y} \mathrm{d}y.
	\] En effet :
	\begin{align*}
		\mathrm{d}f &= f(x+ \mathrm{d}x, y + \mathrm{d}y) - f(x,y) \\
		&= \frac{\partial f}{\partial x} \mathrm{d}x + \frac{\partial f}{\partial y} \mathrm{d}y.
	\end{align*}
\end{rmk}

\begin{prop}
	Soit $f: U \to \R$ de classe $\mathcal{C}^1$ en $(a,b) \in U$. Alors, \[
		\forall w = (w_1, w_2) \in \R^2, \mathrm{d}f(w)\,(a,b) = w_1 \frac{\partial f}{\partial x}(a,b) + w_2 \frac{\partial f}{\partial y}(a,b).
	\]
\end{prop}

\begin{prv}
	Soit $w = (w_1, w_2) \in \R^2$. Soit $t \in \R^*$.
	\begin{align*}
		\frac{1}{t}\big(f(a + tw_1, b + tw_2) - f(a,b)\big)
		&= \frac{1}{t} \left( tw_1 \frac{\partial f}{\partial x}(a,b) + tw_2 \frac{\partial f}{\partial y}(a,b) + \po_{t \to 0}\big(\|tw\|\big) \right) \\
		&= w_1 \frac{\partial f}{\partial x}(a,b) + w_2 \frac{\partial f}{\partial y}(a,b) + \po_{t\to 0}(1) \\
		&\tendsto{t\to 0} w_1 \frac{\partial f}{\partial x}(a,b) + w_2\frac{\partial f}{\partial y}(a,b).
	\end{align*}
\end{prv}


\begin{defn}
	Avec les hypothèses précédentes, en posant \[
		\nabla f(a,b) = \left( \frac{\partial f}{\partial x}(a,b), \frac{\partial f}{\partial y}(a,b) \right) 
	\]on obtient \[
		\mathrm{d}f(w)\,(a,b) = \left<w  \mid \nabla f(a,b) \right>
	\] où $\left<\cdot|\cdot \right>$ est le produit scalaire.

	Le vecteur $\nabla f(a,b)$ est appelé \underline{gradient de $f$ en $(a,b)$}.

	Le développement limité à l'ordre 1 de $f$ devient \[
		f\big((a,b)+w\big) = f(a,b) + \left<w \mid \nabla f(a,b) \right> + \po_{w\to 0}(\|w\|)
	\]
\end{defn}

\begin{prop}
	Soit $f : U \to \R$ de classe $\mathcal{C}^1$.

	\begin{figure}[H]
    \centering
    \incfig{gradient}
	\end{figure}

	$\nabla f$ est orthogonal au lignes de niveaux de $f$, son orientation va dans le sens d'une augmentation de $f$.
\end{prop}

\begin{prv}
	Soit $\gamma : I \to U$ une courbe de niveau : \[
		\forall t \in I, f\big(\gamma(t)\big) = \text{cste}.
	\] D'après le lemme suivant : \[
		\forall t \in I, 0 = (f \circ \gamma)'(t) = \mathrm{d}f\big(\gamma'(t)\big)\big(\gamma(t)\big) = \left<\gamma'(t)  \mid \nabla f\big(\gamma(t)\big) \right>
	\] Donc $\nabla f\big(\gamma(t)\big)$ est orthogonal à $\gamma'(t)$.

	Pour tout $t \in I$, on pose $w(t) = t\, \nabla f\big(\gamma(t)\big)$. Donc \[
		f\big(\gamma(t) + w(t)\big) = f\big(\gamma(t)\big) + t \|\nabla f(\gamma(t))\|^2 + \po_{t \to 0}(t)
	\] Pour $t$ assez petit, $f\big(\gamma(t) + w(t)\big) - f\big(\gamma(t)\big)$ est du même signe que $t$.
\end{prv}

\begin{rmk}
	\begin{align*}
		V: \R^3 &\longrightarrow \R \\
		(x,y,z) &\longmapsto -mgz
	\end{align*}
	l'énerge potentielle de pesenteur

	On a donc \[
		\nabla V(x,y,z) = \left( \frac{\partial V}{\partial x}, \frac{\partial V}{\partial y}, \frac{\partial V}{\partial z} \right) = (0, 0, -mg) = \vec{P}.
	\]
\end{rmk}

\begin{lem}
	Soit $f : U \to \R$ de classe $\mathcal{C}^1$, $\gamma : \begin{array}{rcl}
		I &\longrightarrow& U \\
		t &\longmapsto& \big(x(t), y(t)\big)
	\end{array}$ où $x$ et $y$ sont dérivables.

	On pose \[
		\forall t \in I, \gamma'(t) = \big(x'(t), y'(t)\big).
	\] Alors $f \circ \gamma : I \to \R$ est dérivable et
	\begin{align*}
		\forall t \in I, (f \circ \gamma)'(t) &= \mathrm{d}f\big(\gamma'(t)\big) \big(\gamma(t)\big)\\
		&= \left<\gamma'(t)  \mid \nabla f\big(\gamma(t)\big)  \right> \\
		&= x'(t) \frac{\partial f}{\partial x}\big(x(t), y(t)\big) + y'(t) \frac{\partial f}{\partial y}\big(x(t),y(t)\big). \\
	\end{align*}
\end{lem}

\begin{prv}
	On fixe $t \in I$.

	\begin{align*}
		\forall h \neq 0, \frac{f \circ \gamma(t + h) - f \circ \gamma(t)}{h}
		&= \frac{1}{h}\big(f(\gamma(t)) + h\gamma'(t) + \po_{h\to 0}(h) - f(\gamma(t))\big) \\
		&= \frac{1}{h}\bigg(\cancel{f(\gamma(t))} + \left<h\gamma'(t) \mid \nabla f(\gamma(t)) \right> + \po_{h\to 0}(\|h\gamma'(t)\|) - \cancel{f(\gamma(t))}\bigg)\\
		&= \left<\gamma'(t) \mid \nabla f(\gamma(t)) \right> + \po_{h\to 0}(1) \\
		&\tendsto{h\to 0} \left<\gamma'(t)  \mid \nabla f(\gamma(t)) \right>
	\end{align*}
\end{prv}

\begin{defn}
	Soit $f : U \to \R$ de classe $\mathcal{C}^1$ et $(a,b) \in U$. On dit que $(a,b)$ est un \underline{point critique} de $f$ si $\nabla f(a,b) = 0$ i.e. $\frac{\partial f}{\partial x}(a,b) = \frac{\partial f}{\partial y}(a,b) = 0$.

	Dans ce cas, $f(a,b)$ est appelé \underline{valeur critique} de $f$.
\end{defn}

\begin{prop}~\\
	\begin{minipage}{\linewidth}
		\begin{wrapfigure}{r}{3cm}
			\centering
			\vspace{-1cm}
			\begin{asy}
				import solids;
				import graph;
				size(3cm);

				settings.render = 0;
				settings.prc = false;

				path3 par = graph(
					new real(real x) { return x; },
					new real(real x) { return 0; },
					new real(real x) { return -x^2; },
					0,3);
				revolution r = revolution(par, axis=Z);

				draw(r,1,longitudinalpen=nullpen);
				draw(r.silhouette());

				dot("$(a,b)$", O, red, align=N);
				real s = sqrt(2.5);
				path3 g=(s,0,-2.5)..(0,s,-2.5)..(-s,0,-2.5)..(0,-s,-2.5)..cycle;
				draw(g, deepcyan);
			\end{asy}
		\end{wrapfigure}
		Soit $f: U \to \R$ de classe $\mathcal{C}^1$ et $(a,b) \in U$ tel que \[
			\exists r > 0, \forall (x,y) \in B_{(a,b)}(r), f(x,y) \le f(a,b)
		\] Alors $\nabla f(a,b) = (0,0)$.
	\end{minipage}
\end{prop}

\begin{prv}
	Soit $g: x \mapsto f(x,b)$. $g(a)$ est un maximum local de $g$ donc $g'(a) = 0$.

	Or, $g'(a) = \frac{\partial f}{\partial x}(a,b)$

	donc $\frac{\partial f}{\partial x}(a,b) = 0$.

	Soit $h : y \mapsto f(a,y)$. On a de même $h'(b) = 0$.

	Or, $h'(b) = \frac{\partial f}{\partial y}(a,b)$.

	Donc, $\nabla f(a,b) = (0,0)$.
\end{prv}

\begin{rmk}
	Un minimum local est aussi une valeur critique.
\end{rmk}

\begin{figure}[H]
	\centering
	\begin{subfigure}{3cm}
		\centering
		\begin{asy}
			import solids;
			import graph;
			size(3cm);

			settings.render = 0;
			settings.prc = false;

			path3 par = graph(
				new real(real x) { return x; },
				new real(real x) { return 0; },
				new real(real x) { return -x^2; },
				0,3);
			revolution r = revolution(par, axis=Z);

			draw(r,1,longitudinalpen=nullpen);
			draw(r.silhouette());

			dot(O, red);
		\end{asy}
		\caption{Maximum local}
	\end{subfigure}
	\begin{subfigure}{3cm}
		\centering
		\begin{asy}
			import solids;
			import graph;
			size(3cm);

			settings.render = 0;
			settings.prc = false;

			path3 par = graph(
				new real(real x) { return x; },
				new real(real x) { return 0; },
				new real(real x) { return x^2; },
				0,3);
			revolution r = revolution(par, axis=Z);

			draw(r,1,longitudinalpen=nullpen);
			draw(r.silhouette());

			dot(O, red);
		\end{asy}
		\caption{Minimum local}
	\end{subfigure}
	\begin{subfigure}{3cm}
		\centering
		\begin{asy}
			import solids;
			import graph;
			size(3cm);

			settings.render = 0;
			settings.prc = false;
			currentprojection = obliqueZ;

			draw(graph(
				new real(real x) { return x; },
				new real(real x) { return -x^2 / 3; },
				new real(real x) { return 3; },
				-3, 3
			));

			draw(graph(
				new real(real x) { return x; },
				new real(real x) { return -x^2 / 3; },
				new real(real x) { return -3; },
				-3, 3
			));

			draw(graph(
				new real(real x) { return x; },
				new real(real x) { return -x^2 / 3 - 1; },
				new real(real x) { return 0; },
				-3, 3
			));

			draw(graph(
				new real(real x) { return 0; },
				new real(real x) { return x^2 / 9 - 1; },
				new real(real x) { return x; },
				-3, 3
			));

			draw(graph(
				new real(real x) { return -3; },
				new real(real x) { return x^2 / 9 - 4; },
				new real(real x) { return x; },
				-3, 3
			));

			draw(graph(
				new real(real x) { return 3; },
				new real(real x) { return x^2 / 9 - 4; },
				new real(real x) { return x; },
				-3, 3
			));

			dot((0,-1,0), red);
		\end{asy}
		\caption{Point de selle / Point col}
	\end{subfigure}
\end{figure}

\begin{exm}
	On revient à l'exemple donné en introduction : 
	\begin{align*}
		f: \left( \R^*_+ \right)^2 &\longrightarrow \R \\
		(x,y) &\longmapsto 2\left( xy + \frac{1}{x} + \frac{1}{y} \right).
	\end{align*}

	$\left( \R^+_* \right)^2$ est un ouvert de $\R^2$. Soit $(x,y) \in \left( \R^+_* \right)^2$.
	
	On a \[
		\begin{cases}
			\frac{\partial f}{\partial x}(x,y) = 2\left( y - \frac{1}{x^2} \right),\\
			\frac{\partial f}{\partial y}(x,y) = 2\left( x - \frac{1}{y^2} \right).
		\end{cases}
	\]

	\begin{align*}
		&\frac{\partial f}{\partial x}(x,y) = \frac{\partial f}{\partial y}(x,y) = 0\\
		\iff& \begin{cases}
			y = \frac{1}{x^2}\\
			x = \frac{1}{y^2}
		\end{cases}\\
		\iff& \begin{cases}
			y = \frac{1}{x^2}\\
			x = x^4
		\end{cases}\\
		\iff& \begin{cases}
			x = 1\\
			y = 1
		\end{cases}
	\end{align*}

	On vérivie que $f$ présente en effet un minium local en $(1,1)$. \[
		f(1,1) = 6
	\] On fixe $y \in \R^+_*$ et \[
		g : x \mapsto 2\left( xy + \frac{1}{x} + \frac{1}{y} \right).
	\] Donc \[
		\forall x \in \R^+_*, g'(x) = 2\left( y - \frac{1}{x^2} \right).
	\]
	\begin{center}
		\begin{tikzpicture}
			\tkzTabInit{$x$/1,$g'(x)$/1,$g$/2.3}{$0$, $\frac{1}{\sqrt{y}}$, $+\infty$}
			\tkzTabLine{,-,z,+,}
			\tkzTabVar{+/{}, -/$2\left( 2\sqrt{y} +\frac{1}{y} \right)$, +/{}}
		\end{tikzpicture}
	\end{center}
	
	Ainsi, \[
		\forall x \in \R^+_*, \forall y \in \R^+_*, f(x,y) \ge 2\left( 2\sqrt{y} + \frac{1}{y} \right)
	\] Soit $h : y \mapsto 2\sqrt{y} + \frac{1}{y}$. On a \[
		\forall y > 0, h'(y) = \frac{1}{\sqrt{y}} - \frac{1}{y^2} = \frac{y\sqrt{y} - 1}{y^2} = \frac{y^{\frac{3}{2}} - 1}{y^2}
	\]

	\begin{center}
		\begin{tikzpicture}
			\tkzTabInit{$y$/0.7,$h'(y)$/0.7,$h$/1.4}{$0$, $1$, $+\infty$}
			\tkzTabLine{,-,z,+,}
			\tkzTabVar{+/{}, -/$3$, +/{}}
		\end{tikzpicture}
	\end{center}

	Donc, \[
		\forall x,y > 0, f(x,y) \ge 2\times 3 = 6 = f(1,1).
	\]
\end{exm}

\begin{prop}
	[règle de la chaîne]

	Soit $f : \begin{array}{rcl}
		U &\longrightarrow& \R^2 \\
		(x,y) &\longmapsto& f(x,y)
	\end{array}$ de classe $\mathcal{C}^1$ et $U, V$ deux ouverts de $\R^2$.

	Soit $\varphi : \begin{array}{rcl}
		V &\longrightarrow& U \\
		(u,v) &\longmapsto& \varphi(u,v) = \big(x(u,v), y(u,v)\big)
	\end{array}$.

	On suppose que $x$ et $y$ sont de classe $\mathcal{C}^1$ sur $V$.

	Alors,  $f \circ \varphi : \begin{array}{rcl}
		V &\longrightarrow& \R \\
		(u,v) &\longmapsto& f\big(\varphi(u,v)\big)
	\end{array}$ est de classe $\mathcal{C}^1$ et
	\begin{align*}
		\forall (u_0, v_0) \in V, \frac{\partial (f \circ \varphi)}{\partial u}(u_0, v_0)
		&= \frac{\partial f}{\partial x}\big(\varphi(u_0, v_0)\big) \times \frac{\partial x}{\partial u}(u_0, v_0)\\
		&+ \frac{\partial f}{\partial y}\big(\varphi(u_0,v_0)\big) \frac{\partial y}{\partial u}(u_0,v_0)
	\end{align*}
	\begin{align*}
		\forall (u_0, v_0) \in V, \frac{\partial (f \circ \varphi)}{\partial v}(u_0, v_0)
		&= \frac{\partial f}{\partial x}\big(\varphi(u_0, v_0)\big) \times \frac{\partial x}{\partial v}(u_0, v_0)\\
		&+ \frac{\partial f}{\partial y}\big(\varphi(u_0,v_0)\big) \frac{\partial y}{\partial v}(u_0,v_0)
	\end{align*}
\end{prop}

\begin{exm}
	[changement de coordonnées polaires]
	On pose \begin{align*}
		\varphi: \R^+_* \times ]0,2\pi[ &\longrightarrow \R^2\setminus \left( R^+_* \times \{0\} \right) \\
		(r, \theta) &\longmapsto (r \cos \theta, r \sin\theta),
	\end{align*}
	\begin{align*}
		f: \R^2\setminus \left( R^+_* \times \{0\} \right) &\longrightarrow \R \\
		(x,y) &\longmapsto f(x,y),
	\end{align*}
	\begin{align*}
		g: \overbrace{\R^+_* \times ]0, 2\pi[}^{=V} &\longrightarrow \R \\
		(r, \theta) &\longmapsto f(r\cos\theta, r\sin\theta).
	\end{align*}

	\begin{align*}
		\forall (r_0,\theta_0) \in V,&\\[5mm]
		\frac{\partial g}{\partial r}(r_0, \theta_0) &= \frac{\partial f}{\partial x}(r_0\cos\theta_0, r_0\sin\theta_0)\cos\theta_0\\
		&+ \frac{\partial f}{\partial y}(r_0 \cos\theta_0, r_0\sin\theta_0)\sin\theta_0\\
		&= 2r_0\cos^2\theta_0 + 2r_0\sin^2(\theta_0) \\
		&= 2r_0 \\[5mm]
		\frac{\partial g}{\partial \theta}(r_0, \theta_0) &= \frac{\partial f}{\partial x}(r_0\cos\theta_0, r_0\sin\theta_0)r_0\sin\theta_0\\
		&+ \frac{\partial f}{\partial y}(r_0 \cos\theta_0, r_0\sin\theta_0)r_0\cos\theta_0\\
		&= -2{r_0}^2\cos(\theta_0)\sin(\theta_0) + 2{r_0}^2 \sin(\theta_0)\cos(\theta_0)\\
		&= 0 \\
	\end{align*}

	Donc, \[
		g(r, \theta) = r^2.
	\]
\end{exm}

\begin{exm}
	Résoudre \[
		\begin{cases}
			\frac{\partial f}{\partial x} = \frac{x}{x^2+y^2},\\
			\frac{\partial f}{\partial y} = \frac{y}{x^2+y^2}.\\
		\end{cases}
	\]

	On pose $g: (r, \theta) \mapsto f(r \cos\theta, r \sin\theta)$.

	\begin{align*}
		&\frac{\partial g}{\partial r} = \frac{1}{r}\cos^2\theta + \frac{1}{r}\sin^2\theta = \frac{1}{r},\\
		&\frac{\partial g}{\partial \theta} = -\cos(\theta) \sin(\theta) + \sin(\theta)\cos(\theta) = 0.
	\end{align*}

	Donc, \[
		\exists C \in \R, g: (r, \theta) \mapsto \ln r + C
	\] d'où,
	\begin{align*}
		\forall (x,y) \in \R^2 \setminus \{(0,0)\}, f(x,y) &= \ln\left(\sqrt{x^2 + y^2} \right)  + C\\
		&= \frac{1}{2}\ln(x^2 + y^2) + C. \\
	\end{align*}
\end{exm}

\begin{rmk}
	Soit $\mathcal{B} = (e_1, e_2)$ la base canonique de $\R^2$, $f: U \to \R$ de classe $\mathcal{C}^1$ avec $U$ un ouvert de $\R^2$.

	Soit $(x,y) \in U$.

	\begin{align*}
		\Mat_{\mathcal{B}}\big(\nabla f(x,y)\big) = \begin{pmatrix}
			\frac{\partial f}{\partial x}(x,y)\\[2mm]
			\frac{\partial f}{\partial y}(x,y)
		\end{pmatrix}
	\end{align*}

	Soit  \begin{align*}
		\varphi: V &\longrightarrow U \\
		(u,v) &\longmapsto \big(x(u,v), y(u,v)\big) 
	\end{align*} avec $x,y$ de classe $\mathcal{C}^1$. Soit $g = f \circ \varphi$.
	\begin{align*}
		\Mat_{\mathcal{B}}\big(\nabla g(u,v)\big)
		&= \begin{pmatrix}
			\frac{\partial g}{\partial u}(u,v) \\[2mm]
			\frac{\partial g}{\partial v}(u,v)
		\end{pmatrix} \\
		&= \begin{pmatrix}
			\frac{\partial x}{\partial u}(u,v) \frac{\partial f}{\partial x}(x,y)
			+ \frac{\partial y}{\partial u}(u,v)\frac{\partial f}{\partial y}(x,y)\\[3mm]
			\frac{\partial x}{\partial v}(u,v) \frac{\partial f}{\partial x}(x,y)
			+ \frac{\partial y}{\partial v}(u,v) \frac{\partial f}{\partial y}(x,y)
		\end{pmatrix}  \\
		&= \underbrace{\begin{pmatrix}
				\frac{\partial x}{\partial u}(u,v)& \frac{\partial y}{\partial u}(u,v)\\[3mm]
				\frac{\partial x}{\partial v}(u,v)& \frac{\partial y}{\partial v}(u,v)
		\end{pmatrix}}_{J(u,v)} \begin{pmatrix}
			\frac{\partial f}{\partial x}(x,y)\\[3mm]
			\frac{\partial f}{\partial y}(x,y)
		\end{pmatrix} \\
		&= J(u,v) \Mat_{\mathcal{B}}\big(\nabla f(x,y)\big) \\
	\end{align*}
	où $J(u,v) = 
	\begin{pNiceArray}{c:c}
		\Mat_{\mathcal{B}}\big(\nabla x(u,v)\big) & \Mat_{\mathcal{B}}\big(\nabla y(u,v)\big)
	\end{pNiceArray}$.

	On dit que $J(u,v)$ est \underline{la jacobienne} de $\varphi$ en $(u,v)$.
	L'application linéaire canoniquement associée à $J(u,v)$ est la \underline{différentielle de $\varphi$} en $(u,v)$ noté $\mathrm{d}\varphi(u,v)$.

	On a $\mathrm{d}\varphi(u,v) \in \mathcal{L}(R^2)$ et $\Mat_{\mathcal{B}}\big(\mathrm{d}\varphi(u,v)\big) = J(u,v)$.

	Par exemple, la jacobienne du changement de coordonnées polaires est \[
		J = \begin{pmatrix}
			\frac{\partial x}{\partial r} & \frac{\partial y}{\partial r}\\[3mm]
			\frac{\partial x}{\partial \theta} & \frac{\partial y}{\partial \theta}
		\end{pmatrix}
		= \begin{pmatrix}
			\cos\theta&\sin\theta\\
			-r\sin\theta&r\cos\theta
		\end{pmatrix}.
	\]
	$\underbrace{\det(J)}_{\text{le jacobien}} = r\cos^2\theta + r\sin^2\theta = r$

	Dans une intégrale double, si $(x,y) = \varphi(u,v)$, alors $\mathrm{d}x\mathrm{d}y = \det(J)\mathrm{d}u\mathrm{d}v$.

	Ici, \[
		\mathrm{d}x\ \mathrm{d}y = r\ \mathrm{d}r\ \mathrm{d}\theta.
	\]
\end{rmk}

\begin{prv}
	On pose $(x_0, y_0) = \varphi(u_0, v_0)$. Pour tout $(h,k) \in \R^2$ tels que $(u_0 + h, v_0 + k) \in V$, en posant $g = f  \circ \varphi$.

	\begin{align*}
		g(u_0 + h, v_0 + h) &= f\big(x(u_0 + h, v_0 + k), y(u_0 + h, v_0 + k)\big) \\
		&= f\left(
			x(u_0,v_0) + h \frac{\partial x}{\partial u}(u_0,v_0) + k \frac{\partial x}{\partial v}(u_0, v_0) + \po\big(\|(h,k)\|\big), \right.\\
		&\phantom{ = f\bigg(\bigg.}\left. y(u_0, v_0) + h \frac{\partial y}{\partial u}(u_0, v_0) + k \frac{\partial y}{\partial v}(u_0, v_0) + \po\big(\|(h,k)\|\big)
		\right)  \\
		&= f(x_0,y_0) \\
		&~+ \left( h \frac{\partial x}{\partial u}(u_0,v_0) + k \frac{\partial x}{\partial v}(u_0, v_0) + \po(\|(h,k)\|) \right) \frac{\partial f}{\partial x}(x_0,y_0)\\
		&~+ \left( h \frac{\partial y}{\partial u}(u_0, v_0) + k\frac{\partial y}{\partial v}(u_0, v_0) + \po(\|(h,k)\|) \right) \frac{\partial f}{\partial y}(x_0, y_0)\\
		&~+ \po(\|(h,k)\|)\\
		&= f(x_0, y_0) \\
		&~+ h \left( \frac{\partial x}{\partial u}(u_0, v_0) \frac{\partial f}{\partial x}(x_0, y_0) + \frac{\partial y}{\partial u}(u_0, v_0) \frac{\partial f}{\partial y}(x_0, y_0) \right)  \\
		&~+ k\left( \frac{\partial x}{\partial v}(u_0, v_0) \frac{\partial f}{\partial x}(x_0, y_0) + \frac{\partial y}{\partial v}(u_0, v_0) \frac{\partial f}{\partial y}(x_0, y_0) \right) 
		&~+ \po(\|(h,k)\|)\\
		&= g(u_0, v_0) + h \frac{\partial g}{\partial u}(u_0, v_0) + k \frac{\partial g}{\partial v}(u_0, v_0) + \po(\|(h,k)\|) \\
	\end{align*}

	Par identification,
	\[
		\frac{\partial g}{\partial u}(u_0, v_0) = \frac{\partial x}{\partial u}(u_0, v_0) \frac{\partial f}{\partial x}(x_0, y_0) + \frac{\partial y}{\partial u}(u_0, v_0) \frac{\partial f}{\partial y}(x_0,y_0)
	\] et \[
		\frac{\partial g}{\partial v}(u_0, v_0) = \frac{\partial x}{\partial v}(u_0,v_0) \frac{\partial f}{\partial x}(x_0, y_0) + \frac{\partial y}{\partial v}(u_0, v_0) \frac{\partial f}{\partial y}(x_0, y_0).
	\] 
\end{prv}

\begin{exm}
	[Régression linéaire]~\\
	\begin{figure}[H]
		\centering
		\begin{asy}
			import graph;
			axes(EndArrow);
			size(5cm);

			real f(real x) { return x + 0.5; }

			real k = 35 / (7 - 0.5);

			for(int i = 0; i < 35; ++i) {
				real mag = exp(sin(100 * pi/exp(1) * i)) * 0.8 + exp(cos(i*40)/3);
				real eps = mag * cos(10 * exp(1)/pi * i) / 3;
				dot((i/k,f(i/k) + eps));
			}

			draw(graph(f, -1, 7), orange);
		\end{asy}
	\end{figure}
	\[
		y = a x + b
	\] 
	On fixe $(a,b) \in \R^2$. \[
		\varepsilon(a,b) = \sum_{i=1}^n\big( y_i - (ax_i + b) \big)^2
	\] l'erreur totale.

	On veut minimiser $\varepsilon(a,b)$. On a 
	\[
		\forall (a,b) \in \R^2,
		\begin{cases}
			\frac{\partial \varepsilon}{\partial a}(a,b) = -2\sum_{i=1}^{n}(y_i - ax_i - b)x_i,\\
			\frac{\partial \varepsilon}{\partial b}(a,b) = -2\sum_{i=1}^{n}(y_i - ax_i - b).
		\end{cases}
	\]

	Donc,
	\begin{align*}
		(a,b) \text{ point critique de } \varepsilon \iff& \begin{cases}
			a \sum_{i=1}^n {x_i}^2 + b\sum_{i=1}^{n}x_i = \sum_{i=1}^{n} y_ix_i\\
			a\sum_{i=1}^{n}x_i + nb = \sum_{i=1}^ny_i
		\end{cases}\\
		\iff& \begin{cases}
			a \left( \frac{1}{n}\sum_{i=1}^n {x_i}^2 - \overline{x}^2\right) = \overline{y} - \overline{x} \overline{y}\\
			b = \frac{1}{n}\sum_{i=1}^ny_i - \frac{a}{n}\sum_{i=1}^nx_i = \frac{1}{n}\sum_{i=1}^n x_i y_i - \overline{x} \overline{y}
		\end{cases}\\
		&\text{ où } \overline{x} = \frac{1}{n} \sum_{i=1}^n x_i,~\overline{y} = \frac{1}{n}\sum_{i=1}^n y_i\\
		\iff& \begin{cases}
			a = \frac{\Cov(x,y)}{V(x)}\\
			b = \overline{y} - a\overline{x}
		\end{cases}
	\end{align*}

	Coefficient de corrélation: $\frac{\Cov(x,y)}{\sigma_x \sigma_y} \in [-1, 1]$
\end{exm}












	}

	{
		\chap[21]{Matrices et applications linéaires}
		\renewcommand{\cwd}{../chap21}
		\begin{defn}
	Soit $E$ un $\mathbbm{K}$-espace vectoriel. On dit que $E$ est de \underline{dimension finie} si $E$ a au moins une famille génératrice finie. On dit que $E$ est de \underline{dimension infinie} sinon.
	\index{dimension finie (espace vectoriel)}
	\index{dimension infinie (espace vectoriel)}
\end{defn}

\begin{thm}
	[Théorème de la base extraite]
	Soit $E$ un $\mathbbm{K}$-espace vectoriel non nul de dimension finie. Soit $\mathcal{G}$ une famille génératrice finie de $E$. Alors, il existe une base $\mathcal{B}$ de $\mathcal{E}$ telle que $\mathcal{B} \subset \mathcal{G}$.
\end{thm}

\begin{prv}
	[par récurrence sur $\#G = \Card(G)$]
	\begin{itemize}
		\item Soit $E$ un $\mathbbm{K}$-espace vectoriel non nul engendré par $\mathcal{G} = (u)$.\\
			Si $u = 0_E$, alors $E = \{0_E\}$: une contradiction $\lightning$ \\
			Donc $u \neq 0_E$ donc $(u)$ est libre. En effet, \[
				\forall \lambda \in \mathbbm{K}, \lambda u = 0_E \implies \lambda = 0_\mathbbm{K}
			\] Donc $\mathcal{G}$ est une base de $E$.\\
		\item Soit $n \in \N_*$. Soit $E$ un $\mathbbm{K}$-espace vectoriel. On suppose que si $E$ a une famille génératrice constituée de $n$ vecteurs, alors on peut extraire de cette famille une base de $E$.\\
			Soit $\mathcal{G}$ une famille génératrice de $E$ avec $n+1$ vecteurs.\\
			Si $\mathcal{G}$ est libre, alors $\mathcal{G}$ est une base de $E$. \\
			Si $\mathcal{G}$ n'est pas libre, alors il existe $u \in \mathcal{G}$ tel que $u \in \Vect(\mathcal{G}\setminus \{u\})$ \\
			Donc $\mathcal{G}\setminus \{u\}$ engendre $E$. Or, $\mathcal{G}\setminus \{u\}$ possède $n$ vecteurs. D'après l'hypothèse de récurrence, il existe une base $\mathcal{B}$ de $E$ telle que \[
				\mathcal{B} \subset \mathcal{G} \setminus \{u\} \subset \mathcal{G}
			\] 
	\end{itemize}
\end{prv}

\begin{crlr}
	Tout espace de dimension finie a une base.
	\qed
\end{crlr}

\begin{thm}
	[Théorème de la base incomplète]
	Soit $E$ un $\mathbbm{K}$-espace vectoriel de dimension finie, $\mathcal{G}$ une famille génératrice finie de $E$. $\mathcal{L}$ une famille libre de $E$. Alors, il existe une base $\mathcal{B}$ de $E$ telle que \[
		\mathcal{L} \subset \mathcal{B} \text{ et } \mathcal{B}\setminus \mathcal{L} \subset \mathcal{G}
	\] 
\end{thm}

\begin{prv}
	[par récurrence sur $\#(\mathcal{G}\setminus\mathcal{L})$]
	\begin{itemize}
		\item Avec les notations précédentes, on suppose que $\mathcal{G}\setminus\mathcal{L} \neq \O$ \[
				\forall u \in \mathcal{G}, u \in \mathcal{L}
			\] Donc $\mathcal{G} \subset \mathcal{L}$ donc $\mathcal{L}$ est génératrice donc $\mathcal{L}$ est une base de $E$. On pose $\mathcal{B} = \mathcal{L}$ et alors \[
				\mathcal{L} \subset  \mathcal{B} \text{ et } \mathcal{B}\setminus\mathcal{L} = \O \subset  \mathcal{G}
			\] 
		\item Soit $n \in \N$. On suppose que si $\mathcal{G}$ est génératrice et $\mathcal{L}$ libre avec $\#(\mathcal{G}\setminus\mathcal{L}) = n$ alors il existe une base $\mathcal{B}$ de $E$ telle que \[
			\mathcal{L}\subset \mathcal{B} \text{ et } \mathcal{B}\setminus\mathcal{L}\subset \mathcal{G}
		\] Soient à présent $\mathcal{G}$ une famille génératrice de $E$ et $\mathcal{L}$ une famille libre de $E$ telles que $\#(\mathcal{G}\setminus\mathcal{L}) = n+1 > 0$\\
		Si $\mathcal{L}$ engendre $E$, alors $\mathcal{L}$ est une base de $E$. On pose $\mathcal{B} = \mathcal{L}$ et on a bien \[
			\mathcal{L} \subset  \mathcal{B} \text{ et } \mathcal{B} \setminus \mathcal{L} = \O \subset  \mathcal{G}
		\] On suppose que $\mathcal{L}$ n'engendre pas $E$. Il existe $u \in \mathcal{G}$ tel que $u \not\in \Vec(\mathcal{L})$ (car sinon, $\mathcal{G} \subset \Vect(\mathcal{L})$ et donc $\underbrace{\Vect(\mathcal{G})}_{= E} \subset  \underbrace{\Vect(\mathcal{L})}_{ \subset E}$\\
		Donc $\mathcal{L} \cup \{u\} $ est libre. On pose $\mathcal{L}' = \mathcal{L} \cup \{u\} $ \[
			\mathcal{G}\setminus \mathcal{L}' = \mathcal{G}\setminus (\mathcal{L} \cup \{u\}) = (\mathcal{G}\setminus\mathcal{L})\setminus \{u\} 
		\] donc $\#(\mathcal{G}\setminus\mathcal{L}') = n+1 -1 = n$\\
		D'après l'hypothèse de récurrence, il existe $\mathcal{B}$ une base de $E$ telle que \[
			\mathcal{L} \subset  \mathcal{L}' \subset \mathcal{B} \text{ et } \mathcal{B}\setminus \mathcal{L}' \subset \mathcal{G}
		\] \[
			\mathcal{B} \setminus \mathcal{L} = \underbrace{\mathcal{B}\setminus\mathcal{L}'}_{\subset \mathcal{G}} \cup \underbrace{\{u\}}_{\subset \mathcal{G} \text{ car } u \in \mathcal{G}}
		\] On a $\mathcal{B}\setminus\mathcal{L}\subset \mathcal{G}$
	\end{itemize}
\end{prv}

\begin{thm}
	Soit $E$ un $\mathbbm{K}$-espace vectoriel de dimension finie. Toutes les bases de $E$ ont le même cardinal.
\end{thm}

\begin{prv}
	Soit $\mathcal{G}$ une famille génératrice finie de $E$ et $\mathcal{B} \subset  \mathcal{G}$ une base de $E$. On note $n = \#\mathcal{B}$ \\
	Soit $\mathcal{B}'$ une base de $E$. On pose $p = n - \#(\mathcal{B} \cap  \mathcal{B}')$. Montrons par récurrence sur  $p$ que $\#\mathcal{B} = \#\mathcal{B}'$ 
	\begin{itemize}
		\item On suppose que $p = 0$. Alors, $\#(\mathcal{B} \cap \mathcal{B}') = n$ \\
			Or, $\mathcal{B}' \cap \mathcal{B} \subset \mathcal{B}$ donc $\mathcal{B} \cap \mathcal{B}' = \mathcal{B}$ donc $\mathcal{B} \subset  \mathcal{B}'$ et donc $\mathcal{B} = \mathcal{B}'$ 
		\item Soit $p \in \N$. On suppose que si $\mathcal{B}'$ est une base de $E$ telle que $n - \#(\mathcal{B} \cap \mathcal{B}') = p$, alors $\#\mathcal{B}' = n$ \\
			Aoit $\mathcal{B}'$ une base de $E$ telle que $n - \#(\mathcal{B}\cap \mathcal{B}') = p+1 > 0$ \\
			Donc $\mathcal{B} \cap \mathcal{B}' \neq \mathcal{B}$. Soit $u \in \mathcal{B}' \setminus \mathcal{B}$. D'après le lemme d'échange, il existe $v \in \mathcal{B}\setminus \mathcal{B}'$ tel que $\mathcal{B}' \setminus \{u\} \cup \{v\}$ est une base de $E$. On pose $\mathcal{B}'' = \mathcal{B}' \setminus \{u\} \cup \{v\}$ 
			\begin{align*}
				\mathcal{B}'' \cap \mathcal{B} &= \left( (\mathcal{B}' \setminus \{u\})  \cap \mathcal{B} \right) \cup \{v\} \\
				&= (\mathcal{B}' \cap \mathcal{B}) \cup \{v\} \\
			\end{align*}
			donc,
			\begin{align*}
				n - \#(\mathcal{B}'' \cap \mathcal{B}) &= n - (\#(\mathcal{B}' \cap \mathcal{B}) + 1) \\
				&= p+1- 1 \\
				&= p \\
			\end{align*}
			D'après l'hypothèse de récurrence, \[
				\#\mathcal{B}'' = n
			\] Or, $\#\mathcal{B}'' = \#\mathcal{B}'$
	\end{itemize}
\end{prv}

\begin{lem}
	Soient $\mathcal{B}$ et $\mathcal{B}'$ deux bases de $E$ telles que $\mathcal{B}\subset \mathcal{B}'$. Alors, $\mathcal{B} = \mathcal{B}'$.
\end{lem}

\begin{prv}
	On suppose $\mathcal{B}' \neq \mathcal{B}$. Soit $u \in \mathcal{B}' \setminus \mathcal{B}$
	$u \in E = \Vect(\mathcal{B})$ donc $\mathcal{B} \cup \{u\}$ n'est pas libre.
	Donc $\mathcal{B}\cup \{u\} \subset \mathcal{B}'$ et $\mathcal{B}'$ est libre donc $\mathcal{B}\cup \{u\}$ est libre: une contradiction $\lightning$
\end{prv}

\begin{lem}
	[Lemme d'échange] Soient $\mathcal{B}_1$ et $\mathcal{B}_2$ deux bases de $E$ et $u \in \mathcal{B}_1 \setminus \mathcal{B}_2$. Alors, il existe $v \in \mathcal{B}_2$ tel que $(\mathcal{B}_1 \setminus \{u\}) \cup \{v\}$ soit une base de $E$.
\end{lem}

\begin{prv}
	[1${}^\text{nde}$ méthode]
	On suppose que pout tout $v \in \mathcal{B}_2$, $(\mathcal{B}_1\setminus \{u\}) \cup \{v\}$ n'est pas une base de $E$
	Soit $v \in \mathcal{B}_2$.
	\begin{itemize}
		\item Supposons $(\mathcal{B}_1\setminus \{u\})\cup \{v\}$ non libre. $\mathcal{B}_1 \setminus \{u\}$ est libre. Donc $v \in \Vect(\mathcal{B}_1 \setminus \{u\})$
		\item Supposons $(\mathcal{B}_1\setminus \{u\}) \cup \{v\}$ non génératrice.
			Comme $\mathcal{B}_1$ engendre $E$, $u \not\in \Vect(\mathcal{B}_1\setminus \{v\})$.
			On suppose que $\mathcal{B}_1 \neq \mathcal{B}_2$.
			$\forall v \in \mathcal{B}_2 \setminus \mathcal{B}_1, \Vect(\mathcal{B}_1 \setminus \{v\}) = \Vect(\mathcal{B}_1) = E \ni u$ 
			donc, $(\mathcal{B}_1\setminus \{u\}) \cup \{v\}$ engendre $E$ et donc \[
				v \in \Vect(\mathcal{B}_1 \setminus \{u\})
			\] On a aussi \[
				\forall v \in \mathcal{B}_1 \setminus \{u\}, v \in \Vect(\mathcal{B}_1\setminus \{u\})
			\] Comme $u \not\in \mathcal{B}_2$, on a \[
				\forall v \in \mathcal{B}_2, v \in \Vect(\mathcal{B}_1\setminus \{u\})
			\] docn \[
				E = \Vect(\mathcal{B}_2) \subset \Vect(\mathcal{B}_1\setminus \{u\})
			\] donc $\mathcal{B}_1\setminus \{u\}$ engendre $E$ donc $\mathcal{B}_1\setminus \{u\}$ est une base de $E$. Or, $\mathcal{B}_1 \setminus \{u\}  \subset  \mathcal{B}_1$, donc $\mathcal{B}_1\setminus \{u\} = \mathcal{B}_1$
	\end{itemize}
\end{prv}

\begin{prv}
	[2${}^\text{nde}$ méthode]
	On suppose que pout tout $v \in \mathcal{B}_2$, $(\mathcal{B}_1\setminus \{u\}) \cup \{v\}$ n'est pas une base de $E$
	\begin{itemize}
		\item Comme $u \in \mathcal{B}_1 \setminus \mathcal{B}_2$, nécéssairement $\mathcal{B}_1 \neq \mathcal{B}_2$ donc $\mathcal{B}_2 \not\subset \mathcal{B}_1$, donc $\mathcal{B}_2\setminus\mathcal{B}_1 \neq \O$ 
		\item Soit $v \in \mathcal{B}_2\setminus\mathcal{B}_1$. Il existe $(\lambda_w)_{w\in\mathcal{B}_1}$ une famille de scalaires presque nulle telle que \[
				v = \sum_{w \in \mathcal{B}_1} \lambda_w w - \lambda_u u + + \sum_{w \in \mathcal{B}_1\setminus \{u\}}\lambda_w w
			\]
			Si $\lambda_u \neq 0_E$, alors
			\begin{align*}
				u &= \lambda_u^{-1}\left( v - \sum_{w \in \mathcal{B}_1 \setminus \{u\}} \lambda_w w \right)\\
					&\in \Vect(\mathcal{B}_1\setminus \{u\} \cup v)
			\end{align*}
			 donc $\mathcal{B}_1 \subset \Vect(\mathcal{B}_1\setminus \{u\} \cup \{v\})$\\
			 et donc $E \subset  \Vect(\mathcal{B}_1 \setminus \{u\} \cup \{v\})$ \\
			 et donc $\mathcal{B}_1 \setminus \{u\} \cup \{v\}$ engendre $E$ \\
			 donc $\mathcal{B}_1 \setminus \{u\} \cup \{v\}$ n'est pas libre\\
			 donc $v \in \Vect(\mathcal{B}_1\setminus \{u\})$ (car $\mathcal{B}_1 \setminus \{u\}$ est libre\\
			 donc $\lambda_u = 0_\mathbbm{K}$ $\lightning$\\`

			 Donc, $\lambda_u = 0_\mathbbm{K}$, docn $v \in \Vect(\mathcal{B}_1\setminus \{u\})$ \\
			 On vient de prouver que
			 \begin{align*}
			 	\mathcal{B}_2 \setminus \mathcal{B}_1 \subset \Vect(\mathcal{B}_1 \setminus \{u\})\\
			 	\mathcal{B}_1 \setminus \{u\} \subset \Vect(\mathcal{B}_1 \setminus \{u\})\\
			 \end{align*}
			 Comme $u \not\in \mathcal{B}_2$, \[
			 	\mathcal{B}_2 \subset \Vect(\mathcal{B}_1 \setminus \{u\})
			 \] donc \[
			 	E = \Vect(\mathcal{B}_2) \subset  \Vect(\mathcal{B}_1 \setminus \{u\})
			 \] donc $\mathcal{B}_1 \setminus \{u\}$ engendre $E$. Donc,  $\mathcal{B}_1 \setminus \{u\}$ est une base de $E$.\\
			 Or, $\mathcal{B}_1 \setminus \{u\} \subset  \mathcal{B}_1$, donc $\mathcal{B}_1 \setminus \{u\} = \mathcal{B}_1$
	\end{itemize}
\end{prv}

\begin{defn}
	Soit $E$ un $\mathbbm{K}$-espace vectoriel de dimension finie. Le cardinal commun à toutes les bases de $E$ est appelé \underline{dimension} de $E$ est notée $\dim(E)$ ou $\dim_\mathbbm{K}(E)$\\
	C'est donc aussi le nombre de coordonnées de n'importe quel vecteur dans n'importe quelle base.
	\index{dimension (espace vectoriel)}
\end{defn}

\begin{exm}
	\begin{enumerate}
		\item $\dim_\R(\C) = 2$ et $\dim_\C(\C) = 1$ 
		\item $\dim_\mathbbm{K}(\mathbbm{K}^{n}) = n$ 
		\item $\dim_{\mathbbm{K}}(\mathcal{M}_{n,p}(\mathbbm{K})) = np$
	\end{enumerate}
\end{exm}

\begin{crlr}
	Soit $E$ un $\mathbbm{K}$-espace vectoriel de dimension finie, $\mathcal{L}$ une famille libre de $E$, $\mathcal{G}$ une famille génératrice de $E$. On note $n = \dim(E)$
	\begin{enumerate}
		\item $\#\mathcal{G} \ge n$ et $(\#\mathcal{G} = n \implies \mathcal{G} \text{ est une base de } E$)
		\item $\#\mathcal{L} \le n$ et $(\#\mathcal{L} = n \implies \mathcal{L} \text{ est une base de } E$)
	\end{enumerate}
\end{crlr}

\begin{crlr}
	$\R^{\R}$ est de dimension infinie.
	$\forall i \in \N, e_i: x \mapsto x^i$\\
	$(e_i)_{i\in\N}$ est libre dans $\R^\R$
\end{crlr}

\begin{prop}
	Soient $E$ et $F$ deux $\mathbbm{K}$-espaces vectoriels de dimension finie. Alors $E\times F$ est de dimension finie et $\dim(E\times F) = \dim(E) + \dim(F)$
\end{prop}

\begin{prv}
	Soit $(e_1,\ldots, e_n)$ une base de $E$, $(f_1, \ldots, f_p)$ une base de $F$.
	On pose \[
		\left\{\begin{array}
			{r c l}
			u_1 &=& (e_1,0_F)\\
			u_2 &=& (e_2,0_F)\\
					&\vdots&\\
			u_n &=& (e_n,0_F)\\
			u_{n+1} &=& (0_E, f_1)\\
			u_{n+2} &=& (0_E, f_2)\\
					&\vdots&\\
			u_{n+p} &=& (0_E,f_p)\\
		\end{array}\right.
	\]
	Soit $(x,y) \in E\times F$. \[
		\begin{cases}
			\exists (x_1,\ldots,x_n)\in \mathbbm{K}^n, x = \sum_{i=1}^{n} x_ie_i
			\exists (y_1,\ldots,y_n)\in \mathbbm{K}^n, x = \sum_{j=1}^{p} y_jf_j
		\end{cases}
	\] 
	\begin{align*}
		(x,y) &= \left( \sum_{i=1}^{n} x_ie_i, \sum_{i=1}^{p} y_jf_j \right)  \\
		&= \sum_{i=1}^{n} x_i (e_i + 0_F) + \sum_{j=1}^{p} y_j (0_E, f_j) \\
		&= \sum_{i=1}^{n} x_i u_i + \sum_{j=1}^{p} y_j u_{n+j} \\
	\end{align*}
	Donc, $E\times F = \Vect(u_1, \ldots, u_{n+p})$ donc $E\times F$ est de dimension finie.\\
	Soit $(\lambda_1, \ldots, \lambda_{n+p}) \in \mathbbm{K}^{n+p}$ tel que \[
		(*): \quad \sum_{k=1}^{n+p} \lambda_ku_k = 0_{E\times F} = (0_E, 0_F)
	\]
	\begin{align*}
		(*) &\iff \sum_{k=1}^{n} \lambda_k (e_k, 0_F) + \sum_{k=n+1}^{p} \lambda_k(0_E, f_{k-n}) = (0_E, 0_F)\\
				&\iff \begin{cases}
					\sum_{k=1}^{n} \lambda_k e_k = 0_E\\
					\sum_{k=n+1}^{p} \lambda_k f_{k-n} = 0_F
				\end{cases}\\
				&\iff \begin{cases}
					\forall k \in \left\llbracket 1,n \right\rrbracket, \lambda_k = 0_\mathbbm{K} \qquad&(\text{car $(e_1,\ldots,e_n)$ est libre})\\
					\forall k \in \left\llbracket n+1,n+p \right\rrbracket, \lambda_k = 0_\mathbbm{K} \qquad&(\text{car $(f_1,\ldots,f_n)$ est libre})\\
				\end{cases}
	\end{align*}
	Donc $(u_1, \ldots, u_{n+p})$ est une base de $E\times F$. Donc, $\dim(E\times F) = n + p = \dim(E) + \dim(F)$
\end{prv}

\begin{rmk}
	[Convention]
	\[\dim\big(\{0_E\}\big) = 0\]
\end{rmk}

\begin{thm}
	Soit $E$ un $\mathbbm{K}$-espace vectoriel de dimension finie, $F$ un sous-espace vectoriel de $E$. Alors, $F$ est de dimension finie et  $\dim(F) \le \dim(E)$\\
	Si $\dim(F) = \dim(E)$, alors $F = E$
\end{thm}

\begin{prv}
	On considère \[
		A = \{k \in \N \mid \text{il existe une famille libre de $F$ à $k$ éléments}\} 
	\]
	On suppose $F \neq \{0_E\}$.
	\begin{itemize}
		\item Soit $u \in F\setminus \{0_E\}$. $(u)$ est libre donc $1 \in A$ et donc $A \neq \O$
		\item Soit $\mathcal{L}$ une famille libre de $F$. Alors, $\mathcal{L}$ est une famille libre de $E$ \\
			donc $\#\mathcal{L} \le \dim(E)$\\
			Donc $A$ est majorée par $\dim(E)$ \\
			On en déduit que $A$ a un plus grand élément $p$.
		\item Soit $\mathcal{L}$ une famille libre de $F$ avec $p$ éléments.\\
			Si $\mathcal{L}$ n'engendre pas $F$, alors il existe $u\in F$ tel que $u\not\in \Vect(\mathcal{L})$ et donc $\mathcal{L} \cup \{u\}$ est une famille libre de $F$, donc $p+1 \in A$ en contradiction avec la maximalité de $p$.\\
			Donc $\mathcal{L}$ est une base de $F$ donc $F$ est de dimension finie et $\dim(F) = p \le \dim(E)$\\
	\end{itemize}

	Soit $\mathcal{B}$ une base de $F$. Alors, $\mathcal{B}$ est aussi une famille de libre de de $E$. Donc $\#\mathcal{B} \le \dim(E)$ donc $\dim(F) = \dim(E)$ \\
	Si $\dim(F) = \dim(E)$, alors $\mathcal{B}$ est une base de $E$, et donc $F = \Vect(\mathcal{B}) = E$
\end{prv}

\begin{prop}
	[Formule de Grassmann]
	Soit $E$ un $\mathbbm{K}$-espace vectoriel de dimension finie, $F$ et $G$ deux sous-espace vectoriels de $E$. Alors, \[
		\dim(F+G) = \dim(F) + \dim(G) - \dim(F\cap G)
	\] 
\end{prop}

\begin{prv}
	Soit $(e_1, \ldots, e_p)$ une base de $F\cap G$. $(e_1,\ldots,e_p)$ est une famille libre de $F$.\\
	On complète $(e_1, \ldots, e_p)$ en une base $(e_1, \ldots, e_p, u_1, \ldots, u_q)$ de $F$.\\
	De même, on complète $(e_1, \ldots, e_p)$ en une base $(e_1, \ldots, e_p, v_1, \ldots, v_r)$ de $G$.\\
	On pose  $\mathcal{B} = (e_1, \ldots, e_p, u_1, \ldots, u_q, v_1, \ldots, v_r)$. Montrons que $\mathcal{B}$ est une base de $F+G$
	\begin{itemize}
		\item Soit $u \in F+G$ \\
			On pose $u = v+w$ avec $\begin{cases}
				v\in F\\
				w \in G
			\end{cases}$.\\
			On pose $v = \sum_{i=1}^p \lambda_i e_i + \sum_{i=1}^q \mu_i u_i$ avec $(\lambda_1, \ldots, \lambda_p, \mu_1, \ldots, \lambda_q) \in \mathbbm{K}^{p+q}$\\
			On pose aussi $w = \sum_{i = 1}^p \lambda'_ie_i + \sum_{j=1}^r \nu_j v_j$ avec $(\lambda_1',\ldots,\lambda_p', \nu_1, \ldots, \nu_r) \in \mathbbm{K}^{p+r}$\\
			D'où, \[
				u = \sum_{i=1}^p (\lambda_i + \lambda'_i)e_i + \sum_{j=1}^q \mu_j u_j + \sum_{k=1}^r \nu_k v_k \in \Vect(\mathcal{B})
			\]
		\item Soient $(\lambda_1, \ldots, \lambda_p, \mu_1, \ldots, \mu_q, \nu_1, \ldots, \nu_r) \in \mathbbm{K}^{p+q+r}$.\\
			On suppose \[
				(*)\quad \sum_{i=1}^{p}\lambda_ie_i + \sum_{j=1}^q\mu_ju_j + \sum_{k=1}^r \nu_k v_k = 0_E
			\] 
			D'où, \[
				\underbrace{\sum_{i=1}^p\lambda_i e_i + \sum_{j=1}^q \mu_ju_j}_{\in F} = \underbrace{-\sum_{k=1}^r\nu_jv_k}_{\in G}
			\] 
			Donc, \[
				f = \sum_{i=1}^p \lambda_i e_i + \sum_{j=1}^q \mu_j u_j \in F\cap G
			\] Comme $(e_1, \ldots, e_p)$ est une base de $F\cap G$, $\exists ! (\lambda_1', \ldots, \lambda_p') \in \mathbbm{K}^p$ tel que \[
				f = \sum_{i=1}^p \lambda'_i e_i = \sum_{i=1}^p \lambda'_i e_i + \sum_{j=1}^q 0_\mathbbm{K}u_j
			\] Comme $(e_1, \ldots, e_p, u_1, \ldots, u_q)$ est une base de $F$, \[
				\forall k \in \left\llbracket 1, q \right\rrbracket, \mu_j = 0_\mathbbm{K}
			\] De même, \[
				\forall k \in \left\llbracket 1,r \right\rrbracket , \nu_k = 0_\mathbbm{K}
			\] On remplace dans $(*)$ pour trouver \[
				\sum_{i=1}^p \lambda_ie_i = 0_E
			\] Comme $(e_1, \ldots, e_p)$ est libre, \[
				\forall i \in \left\llbracket 1,p \right\rrbracket, \lambda_i = 0_\mathbbm{K}
			\] Donc $\mathcal{B}$ est libre.\\
			Donc, 
			\begin{align*}
				\dim(F+G) &=  p +q + r \\
				&= (p+q)+ (p+r) - p \\
				&= \dim(F) + \dim(G) - \dim(F\cap G) \\
			\end{align*}
	\end{itemize}
\end{prv}

\begin{crlr}
	Avec les hypothèse précédentes, \[
		E = F \oplus G \iff \begin{cases}
			F \cap  G = \{0_E\} \\
			\dim(E) = \dim(F) + \dim(G)
		\end{cases}
	\] 
\end{crlr}

\begin{prv}
	\begin{itemize}
		\item[``$\implies$''] On suppose $E = F \oplus G$ \\
			Comme la somme est directe, $F \cap G = \{0_E\}$ 
			\begin{align*}
				\dim(E) &= \dim(F)\\
				&= \dim(F) + \dim(G) - \dim(F\cap G)\\
				&= \dim(F) + \dim(G)\\
			\end{align*}
		\item[``$\impliedby$''] On suppose $F\cap G = \{0_E\}$ et $\dim(E) = \dim(F) + \dim(G)$.\\
			On sait déjà que $F+G = F \oplus G$\\
			 \begin{align*}
				\dim(F+G) = \dim(F) + \dim(G) - \dim(F \cap G) = \dim(E)
			\end{align*}
			Donc $F + G = E$
	\end{itemize}
\end{prv}

\begin{prop}
	Soit $F$ un $\mathbbm{K}$-espace vectoriel de dimension finie $n$. Soit $\mathcal{B} = (e_1, \ldots, e_n)$ une base de $F$. L'application
	\begin{align*}
		f: \mathbbm{K}^n &\longrightarrow F \\
		(\lambda_1, \ldots, \lambda_n) &\longmapsto \sum_{i=1}^n \lambda_i e_i
	\end{align*} est bijective.\\
	Si $\mathbbm{K}$ est infini, $\mathbbm{K}^n$ aussi et donc $F$ aussi.\\
	Si $\#\mathbbm{K} = p \in \N_*$,
	\begin{align*}
		\#&\mathbbm{K}^n = p^n\\
		&\vrt=\\
		\#&F
	\end{align*}
\end{prop}


		\part{Dérivation}

\underline{Motivation}:

{
\begin{wrapfigure}{l}{3cm}
	\centering
	\begin{asy}
		import three;

		size(3cm);
		settings.render=0;
		settings.prc=false;
		currentprojection = obliqueZ;

		draw(unitbox);
		draw(shift(1.1Z + 0.05X) * (O -- X), Arrows3(TeXHead2));
		draw(shift(1.1Z + 0.05Y) * (O -- Y), Arrows3(TeXHead2));
		draw(shift(1.1X + 0.05Z) * (O -- Z), Arrows3(TeXHead2));

		label("$x$", (X/2) + (1.1Z + 0.05X), align=S);
		label("$y$", (Y/2) + (1.1Z + 0.05Y), align=W);
		label("$z$", (Z/2) + X, align=SE);
	\end{asy}
\end{wrapfigure}

\begin{align*}
	&S(x,y,z) = 2(xy + xz + yz)\\
	&V(x,y,z) = xyz
\end{align*}

On cherche à minimiser $S$ avec la contrainte $V = 1$.

Soit $f : \begin{array}{rcl}
	\left( \R_*^+ \right)^2 &\longrightarrow& \R \\
	(x,y) &\longmapsto& S\left( x,y,\frac{1}{xy} \right) = 2\left( xy + \frac{1}{y} + \frac{1}{x} \right).
\end{array}$

On cherche $(a,b) \in \left( \R^+_* \right)^2$ tel que \[
	\forall (x,y) \in (\R^+_*), f(x,y) \ge f(a,b).
\]
}

\begin{defn}
	Soit $f: U \to \R$ où $U$ est un ouvert de $\R^2$. Soit $(a,b) \in U$.
	\vspace{2mm}

	Si $\lim_{x \to a} \frac{f(x,b) - f(a,b)}{x - a} \in \R$, alors on dit que $f$ a une dérivée partielle suivant $x$ en $(a,b)$ et cette limite est notée \[
		\partial f_1(a,b) = \frac{\partial f}{\partial x}(a,b).
	\]

	Si $\lim_{y \to b} \frac{f(a,y) - f(a,b)}{y - b} \in \R$, alors on dit que $f$ a une dérivée partielle suivant $y$ et la limite est notée \[
		\partial f_2(a,b) = \frac{\partial f}{\partial y}(a,b).
	\]
\end{defn}

\begin{exm}
	\begin{enumerate}
		\item $f: (x,y) \mapsto xy + x - y$.

			\begin{align*}
				&\frac{\partial f}{\partial x} : (x,y) \mapsto y + 1,\\
				&\frac{\partial f}{\partial y} : (x,y) \mapsto x - 1.
			\end{align*}

		\item $f: (x,y) \mapsto xy + \frac{1}{y}+ \frac{1}{x}$.

			\begin{align*}
				&\frac{\partial f}{\partial x}: (x,y) \mapsto y - \frac{1}{x^2},\\
				&\frac{\partial f}{\partial y}: (x,y) \mapsto x - \frac{1}{y^2}.
			\end{align*}

		\item Trouver $f$ telle que $\begin{cases}
				(1): \qquad \frac{\partial f}{\partial x}=y,\\[2mm]
				(2): \qquad \frac{\partial f}{\partial y} = x.
			\end{cases}$

			D'après $(1)$ : \[
				\forall (x,y), \exists C(y) \in \R, f(x,y) = xy + C(y)
			\] et donc \[
				\frac{\partial f}{\partial y}(x,y) = x + C'(y)
			\] donc $C'(y) = 0$ et donc $C$ est constante.

		\item Trouver $f$ telle que $\begin{cases}
			\frac{\partial f}{\partial x} = -y,\\[2mm]
			\frac{\partial f}{ƒ\partial y} = x.
		\end{cases}$

		Ce n'est pas possible !
	\end{enumerate}
\end{exm}

\begin{defn}~\\
	\begin{minipage}{\linewidth}
		\begin{wrapfigure}{r}{4cm}
			\centering
			\vspace{-5mm}
			\begin{asy}
				import three;
				import graph3;
				size(4cm);

				settings.render = 0;
				settings.prc = false;
				currentprojection = obliqueX;

				draw(O -- X, Arrow3(TeXHead2));
				draw(O -- Y, Arrow3(TeXHead2));
				draw(O -- Z, Arrow3(TeXHead2));

				triple f(real x, real y, real z = 0) { return (x,y,cos(x - 0.5) * cos(y - 0.5)/1.2 + 0.15); }

				real inc = 1 / 5;

				for(real x = 0; x <= 1; x += inc) {
					draw(graph(
						new real(real t) { return x; }, // x
						new real(real y) { return y; }, // y
						new real(real y) { return f(x,y).z; }, // z
						0, 1
					), gray);
				}

				for(real y = 0; y <= 1; y += inc) {
					draw(graph(
						new real(real x) { return x; }, // x
						new real(real t) { return y; }, // y
						new real(real x) { return f(x,y).z; }, // z
						0, 1
					), gray);
				}

				path3 path1 = (0.8, 0.2, 0) .. (0.5, 0.5, 0) .. (0.3, 0.7, 0);
				path3 path2 = f(0.8, 0.2, 0) .. f(0.5, 0.5, 0) .. f(0.3, 0.7, 0);
				path3 d = (0.2, 0.3, 0) .. (0.3, 0.4, 0) .. (0.2, 0.7, 0) .. (0.8, 0.9, 0) .. (0.6, 0.2, 0) .. cycle;

				draw(path1, red, Arrow3(TeXHead2));
				draw(path2, red, Arrow3(TeXHead2, position=0.8));

				dot((0.5, 0.5, 0));
				dot(f(0.5, 0.5, 0));
				draw((0.5, 0.5, 0) -- f(0.5, 0.5, 0), dashed);
				draw(d);

				label("$w$", (0.3, 0.7, 0), red, align=SE);
				label("$U$", (0.8, 0.9, 0), align=SE);
			\end{asy}
		\end{wrapfigure}

		Soit $f: U \to \R$ où $U$ est un ouvert. Soit $(a,b) \in U$. Soit $w = (w_1, w_2) \in \R^2$.

		Si 
		\[
			\lim_{t\to 0} \frac{f(a + tw_1, b + tw_2) - f(a,b)}{t}
		\] existe et est réelle, alors on dit que $f$ a une dérivée dans la direction de $w$ et la limite est notée \[
			\mathrm{d}f(w)\,(a,b) = D_w(f)\,(a,b).
		\]
	\end{minipage}
\end{defn}

\begin{exm}
	\begin{align*}
		f: \left( \R_*^+ \right)^2 &\longrightarrow \R \\
		(x,y) &\longmapsto xy+\frac{1}{x}+\frac{1}{y}.
	\end{align*}

	On pose $(a,b) = (1,2)$, $w = (w_1, w_2) = (1,1)$.
	\begin{align*}
		\frac{f(1+t, 2+t) - f(1,2)}{t} &= \frac{1}{t} \left( (1+t)(2+t) + \frac{1}{1+t} + \frac{1}{2+t} - 3 - \frac{1}{2} \right) \\
		&= \frac{1}{t}\left(\cancel 2 + 3t + \po(t) + \cancel 1 - t + \po(t) + \frac{1}{2}\left( \cancel 1 - \frac{t}{2} + \po(t) \right) - \cancel3 - \cancel{\frac{1}{2}} \right) \\
		&= \frac{1}{t} \left( \frac{7}{4} t + \po(t) \right)  \\
		&= \frac{7}{4} + \po(1) \tendsto{t \to 0} \frac{7}{4}. \\
	\end{align*}

	Donc, \[
		\mathrm{d}f(1,1)\,(1,2) = \frac{7}{4}.
	\]
\end{exm}

\begin{rmk}~\\
	\begin{figure}[H]
		\centering
		\begin{asy}
			import solids;
			import graph;
			size(5cm);

			settings.render = 0;
			settings.prc = false;

			path3 par = graph(
				new real(real x) { return x; },
				new real(real x) { return 0; },
				new real(real x) { return x^2; },
				0,3);
			revolution r = revolution(par, axis=Z);

			path3 par2 = graph(
				new real(real x) { return x; },
				new real(real x) { return 0; },
				new real(real x) { return x^2; },
				-3,3);

			draw(r,1,longitudinalpen=nullpen);
			draw(r.silhouette());

			draw((-4, 0, -1) -- (-4, 0, 10) -- (4, 0, 10) -- (4, 0, -1) -- cycle, red);
			draw(par2, deepred);

			draw((4,4.5) -- (7, 4.5), black+0.5mm, Arrow(TeXHead));

			path par2d = graph(new real(real x) { return x^2; }, -3, 3);
			draw(shift((11, 0)) * par2d, deepred);

			dot(O);
			dot((11, 0));
		\end{asy}
	\end{figure}
\end{rmk}


%todo ajouter théorème-définition
\begin{thm}
	Soit $f : U \to \R$, $(a,b) \in U$. On suppose que $\frac{\partial f}{\partial x}$ et $\frac{\partial f}{\partial y}$ existent en $(a,b)$ et sont {\bfseries continues} en $(a,b)$. Alors,
	\begin{align*}
		&\forall (h, k) \in \R^2 \text{ tel que } (a +h, b + k) \in U,\\
		&f(a+ h, b + k) = f(a,b) + h \frac{\partial f}{\partial x}(a,b) + k \frac{\partial f}{\partial y}(a,b) + \po_{(h,k)\to (0,0)}\big(\|(h,k)\|\big).
	\end{align*}

	On dit que $f$ est de classe $\mathcal{C}^1$ si $\frac{\partial f}{\partial x}$ et $\frac{\partial f}{\partial y}$ existent et sont continues.

	\qed
\end{thm}

\begin{rmk}
	En physique, cette formule correspond à : \[
		\mathrm{d}f = \frac{\partial f}{\partial x}\mathrm{d}x + \frac{\partial f}{\partial y} \mathrm{d}y.
	\] En effet :
	\begin{align*}
		\mathrm{d}f &= f(x+ \mathrm{d}x, y + \mathrm{d}y) - f(x,y) \\
		&= \frac{\partial f}{\partial x} \mathrm{d}x + \frac{\partial f}{\partial y} \mathrm{d}y.
	\end{align*}
\end{rmk}

\begin{prop}
	Soit $f: U \to \R$ de classe $\mathcal{C}^1$ en $(a,b) \in U$. Alors, \[
		\forall w = (w_1, w_2) \in \R^2, \mathrm{d}f(w)\,(a,b) = w_1 \frac{\partial f}{\partial x}(a,b) + w_2 \frac{\partial f}{\partial y}(a,b).
	\]
\end{prop}

\begin{prv}
	Soit $w = (w_1, w_2) \in \R^2$. Soit $t \in \R^*$.
	\begin{align*}
		\frac{1}{t}\big(f(a + tw_1, b + tw_2) - f(a,b)\big)
		&= \frac{1}{t} \left( tw_1 \frac{\partial f}{\partial x}(a,b) + tw_2 \frac{\partial f}{\partial y}(a,b) + \po_{t \to 0}\big(\|tw\|\big) \right) \\
		&= w_1 \frac{\partial f}{\partial x}(a,b) + w_2 \frac{\partial f}{\partial y}(a,b) + \po_{t\to 0}(1) \\
		&\tendsto{t\to 0} w_1 \frac{\partial f}{\partial x}(a,b) + w_2\frac{\partial f}{\partial y}(a,b).
	\end{align*}
\end{prv}


\begin{defn}
	Avec les hypothèses précédentes, en posant \[
		\nabla f(a,b) = \left( \frac{\partial f}{\partial x}(a,b), \frac{\partial f}{\partial y}(a,b) \right) 
	\]on obtient \[
		\mathrm{d}f(w)\,(a,b) = \left<w  \mid \nabla f(a,b) \right>
	\] où $\left<\cdot|\cdot \right>$ est le produit scalaire.

	Le vecteur $\nabla f(a,b)$ est appelé \underline{gradient de $f$ en $(a,b)$}.

	Le développement limité à l'ordre 1 de $f$ devient \[
		f\big((a,b)+w\big) = f(a,b) + \left<w \mid \nabla f(a,b) \right> + \po_{w\to 0}(\|w\|)
	\]
\end{defn}

\begin{prop}
	Soit $f : U \to \R$ de classe $\mathcal{C}^1$.

	\begin{figure}[H]
    \centering
    \incfig{gradient}
	\end{figure}

	$\nabla f$ est orthogonal au lignes de niveaux de $f$, son orientation va dans le sens d'une augmentation de $f$.
\end{prop}

\begin{prv}
	Soit $\gamma : I \to U$ une courbe de niveau : \[
		\forall t \in I, f\big(\gamma(t)\big) = \text{cste}.
	\] D'après le lemme suivant : \[
		\forall t \in I, 0 = (f \circ \gamma)'(t) = \mathrm{d}f\big(\gamma'(t)\big)\big(\gamma(t)\big) = \left<\gamma'(t)  \mid \nabla f\big(\gamma(t)\big) \right>
	\] Donc $\nabla f\big(\gamma(t)\big)$ est orthogonal à $\gamma'(t)$.

	Pour tout $t \in I$, on pose $w(t) = t\, \nabla f\big(\gamma(t)\big)$. Donc \[
		f\big(\gamma(t) + w(t)\big) = f\big(\gamma(t)\big) + t \|\nabla f(\gamma(t))\|^2 + \po_{t \to 0}(t)
	\] Pour $t$ assez petit, $f\big(\gamma(t) + w(t)\big) - f\big(\gamma(t)\big)$ est du même signe que $t$.
\end{prv}

\begin{rmk}
	\begin{align*}
		V: \R^3 &\longrightarrow \R \\
		(x,y,z) &\longmapsto -mgz
	\end{align*}
	l'énerge potentielle de pesenteur

	On a donc \[
		\nabla V(x,y,z) = \left( \frac{\partial V}{\partial x}, \frac{\partial V}{\partial y}, \frac{\partial V}{\partial z} \right) = (0, 0, -mg) = \vec{P}.
	\]
\end{rmk}

\begin{lem}
	Soit $f : U \to \R$ de classe $\mathcal{C}^1$, $\gamma : \begin{array}{rcl}
		I &\longrightarrow& U \\
		t &\longmapsto& \big(x(t), y(t)\big)
	\end{array}$ où $x$ et $y$ sont dérivables.

	On pose \[
		\forall t \in I, \gamma'(t) = \big(x'(t), y'(t)\big).
	\] Alors $f \circ \gamma : I \to \R$ est dérivable et
	\begin{align*}
		\forall t \in I, (f \circ \gamma)'(t) &= \mathrm{d}f\big(\gamma'(t)\big) \big(\gamma(t)\big)\\
		&= \left<\gamma'(t)  \mid \nabla f\big(\gamma(t)\big)  \right> \\
		&= x'(t) \frac{\partial f}{\partial x}\big(x(t), y(t)\big) + y'(t) \frac{\partial f}{\partial y}\big(x(t),y(t)\big). \\
	\end{align*}
\end{lem}

\begin{prv}
	On fixe $t \in I$.

	\begin{align*}
		\forall h \neq 0, \frac{f \circ \gamma(t + h) - f \circ \gamma(t)}{h}
		&= \frac{1}{h}\big(f(\gamma(t)) + h\gamma'(t) + \po_{h\to 0}(h) - f(\gamma(t))\big) \\
		&= \frac{1}{h}\bigg(\cancel{f(\gamma(t))} + \left<h\gamma'(t) \mid \nabla f(\gamma(t)) \right> + \po_{h\to 0}(\|h\gamma'(t)\|) - \cancel{f(\gamma(t))}\bigg)\\
		&= \left<\gamma'(t) \mid \nabla f(\gamma(t)) \right> + \po_{h\to 0}(1) \\
		&\tendsto{h\to 0} \left<\gamma'(t)  \mid \nabla f(\gamma(t)) \right>
	\end{align*}
\end{prv}

\begin{defn}
	Soit $f : U \to \R$ de classe $\mathcal{C}^1$ et $(a,b) \in U$. On dit que $(a,b)$ est un \underline{point critique} de $f$ si $\nabla f(a,b) = 0$ i.e. $\frac{\partial f}{\partial x}(a,b) = \frac{\partial f}{\partial y}(a,b) = 0$.

	Dans ce cas, $f(a,b)$ est appelé \underline{valeur critique} de $f$.
\end{defn}

\begin{prop}~\\
	\begin{minipage}{\linewidth}
		\begin{wrapfigure}{r}{3cm}
			\centering
			\vspace{-1cm}
			\begin{asy}
				import solids;
				import graph;
				size(3cm);

				settings.render = 0;
				settings.prc = false;

				path3 par = graph(
					new real(real x) { return x; },
					new real(real x) { return 0; },
					new real(real x) { return -x^2; },
					0,3);
				revolution r = revolution(par, axis=Z);

				draw(r,1,longitudinalpen=nullpen);
				draw(r.silhouette());

				dot("$(a,b)$", O, red, align=N);
				real s = sqrt(2.5);
				path3 g=(s,0,-2.5)..(0,s,-2.5)..(-s,0,-2.5)..(0,-s,-2.5)..cycle;
				draw(g, deepcyan);
			\end{asy}
		\end{wrapfigure}
		Soit $f: U \to \R$ de classe $\mathcal{C}^1$ et $(a,b) \in U$ tel que \[
			\exists r > 0, \forall (x,y) \in B_{(a,b)}(r), f(x,y) \le f(a,b)
		\] Alors $\nabla f(a,b) = (0,0)$.
	\end{minipage}
\end{prop}

\begin{prv}
	Soit $g: x \mapsto f(x,b)$. $g(a)$ est un maximum local de $g$ donc $g'(a) = 0$.

	Or, $g'(a) = \frac{\partial f}{\partial x}(a,b)$

	donc $\frac{\partial f}{\partial x}(a,b) = 0$.

	Soit $h : y \mapsto f(a,y)$. On a de même $h'(b) = 0$.

	Or, $h'(b) = \frac{\partial f}{\partial y}(a,b)$.

	Donc, $\nabla f(a,b) = (0,0)$.
\end{prv}

\begin{rmk}
	Un minimum local est aussi une valeur critique.
\end{rmk}

\begin{figure}[H]
	\centering
	\begin{subfigure}{3cm}
		\centering
		\begin{asy}
			import solids;
			import graph;
			size(3cm);

			settings.render = 0;
			settings.prc = false;

			path3 par = graph(
				new real(real x) { return x; },
				new real(real x) { return 0; },
				new real(real x) { return -x^2; },
				0,3);
			revolution r = revolution(par, axis=Z);

			draw(r,1,longitudinalpen=nullpen);
			draw(r.silhouette());

			dot(O, red);
		\end{asy}
		\caption{Maximum local}
	\end{subfigure}
	\begin{subfigure}{3cm}
		\centering
		\begin{asy}
			import solids;
			import graph;
			size(3cm);

			settings.render = 0;
			settings.prc = false;

			path3 par = graph(
				new real(real x) { return x; },
				new real(real x) { return 0; },
				new real(real x) { return x^2; },
				0,3);
			revolution r = revolution(par, axis=Z);

			draw(r,1,longitudinalpen=nullpen);
			draw(r.silhouette());

			dot(O, red);
		\end{asy}
		\caption{Minimum local}
	\end{subfigure}
	\begin{subfigure}{3cm}
		\centering
		\begin{asy}
			import solids;
			import graph;
			size(3cm);

			settings.render = 0;
			settings.prc = false;
			currentprojection = obliqueZ;

			draw(graph(
				new real(real x) { return x; },
				new real(real x) { return -x^2 / 3; },
				new real(real x) { return 3; },
				-3, 3
			));

			draw(graph(
				new real(real x) { return x; },
				new real(real x) { return -x^2 / 3; },
				new real(real x) { return -3; },
				-3, 3
			));

			draw(graph(
				new real(real x) { return x; },
				new real(real x) { return -x^2 / 3 - 1; },
				new real(real x) { return 0; },
				-3, 3
			));

			draw(graph(
				new real(real x) { return 0; },
				new real(real x) { return x^2 / 9 - 1; },
				new real(real x) { return x; },
				-3, 3
			));

			draw(graph(
				new real(real x) { return -3; },
				new real(real x) { return x^2 / 9 - 4; },
				new real(real x) { return x; },
				-3, 3
			));

			draw(graph(
				new real(real x) { return 3; },
				new real(real x) { return x^2 / 9 - 4; },
				new real(real x) { return x; },
				-3, 3
			));

			dot((0,-1,0), red);
		\end{asy}
		\caption{Point de selle / Point col}
	\end{subfigure}
\end{figure}

\begin{exm}
	On revient à l'exemple donné en introduction : 
	\begin{align*}
		f: \left( \R^*_+ \right)^2 &\longrightarrow \R \\
		(x,y) &\longmapsto 2\left( xy + \frac{1}{x} + \frac{1}{y} \right).
	\end{align*}

	$\left( \R^+_* \right)^2$ est un ouvert de $\R^2$. Soit $(x,y) \in \left( \R^+_* \right)^2$.
	
	On a \[
		\begin{cases}
			\frac{\partial f}{\partial x}(x,y) = 2\left( y - \frac{1}{x^2} \right),\\
			\frac{\partial f}{\partial y}(x,y) = 2\left( x - \frac{1}{y^2} \right).
		\end{cases}
	\]

	\begin{align*}
		&\frac{\partial f}{\partial x}(x,y) = \frac{\partial f}{\partial y}(x,y) = 0\\
		\iff& \begin{cases}
			y = \frac{1}{x^2}\\
			x = \frac{1}{y^2}
		\end{cases}\\
		\iff& \begin{cases}
			y = \frac{1}{x^2}\\
			x = x^4
		\end{cases}\\
		\iff& \begin{cases}
			x = 1\\
			y = 1
		\end{cases}
	\end{align*}

	On vérivie que $f$ présente en effet un minium local en $(1,1)$. \[
		f(1,1) = 6
	\] On fixe $y \in \R^+_*$ et \[
		g : x \mapsto 2\left( xy + \frac{1}{x} + \frac{1}{y} \right).
	\] Donc \[
		\forall x \in \R^+_*, g'(x) = 2\left( y - \frac{1}{x^2} \right).
	\]
	\begin{center}
		\begin{tikzpicture}
			\tkzTabInit{$x$/1,$g'(x)$/1,$g$/2.3}{$0$, $\frac{1}{\sqrt{y}}$, $+\infty$}
			\tkzTabLine{,-,z,+,}
			\tkzTabVar{+/{}, -/$2\left( 2\sqrt{y} +\frac{1}{y} \right)$, +/{}}
		\end{tikzpicture}
	\end{center}
	
	Ainsi, \[
		\forall x \in \R^+_*, \forall y \in \R^+_*, f(x,y) \ge 2\left( 2\sqrt{y} + \frac{1}{y} \right)
	\] Soit $h : y \mapsto 2\sqrt{y} + \frac{1}{y}$. On a \[
		\forall y > 0, h'(y) = \frac{1}{\sqrt{y}} - \frac{1}{y^2} = \frac{y\sqrt{y} - 1}{y^2} = \frac{y^{\frac{3}{2}} - 1}{y^2}
	\]

	\begin{center}
		\begin{tikzpicture}
			\tkzTabInit{$y$/0.7,$h'(y)$/0.7,$h$/1.4}{$0$, $1$, $+\infty$}
			\tkzTabLine{,-,z,+,}
			\tkzTabVar{+/{}, -/$3$, +/{}}
		\end{tikzpicture}
	\end{center}

	Donc, \[
		\forall x,y > 0, f(x,y) \ge 2\times 3 = 6 = f(1,1).
	\]
\end{exm}

\begin{prop}
	[règle de la chaîne]

	Soit $f : \begin{array}{rcl}
		U &\longrightarrow& \R^2 \\
		(x,y) &\longmapsto& f(x,y)
	\end{array}$ de classe $\mathcal{C}^1$ et $U, V$ deux ouverts de $\R^2$.

	Soit $\varphi : \begin{array}{rcl}
		V &\longrightarrow& U \\
		(u,v) &\longmapsto& \varphi(u,v) = \big(x(u,v), y(u,v)\big)
	\end{array}$.

	On suppose que $x$ et $y$ sont de classe $\mathcal{C}^1$ sur $V$.

	Alors,  $f \circ \varphi : \begin{array}{rcl}
		V &\longrightarrow& \R \\
		(u,v) &\longmapsto& f\big(\varphi(u,v)\big)
	\end{array}$ est de classe $\mathcal{C}^1$ et
	\begin{align*}
		\forall (u_0, v_0) \in V, \frac{\partial (f \circ \varphi)}{\partial u}(u_0, v_0)
		&= \frac{\partial f}{\partial x}\big(\varphi(u_0, v_0)\big) \times \frac{\partial x}{\partial u}(u_0, v_0)\\
		&+ \frac{\partial f}{\partial y}\big(\varphi(u_0,v_0)\big) \frac{\partial y}{\partial u}(u_0,v_0)
	\end{align*}
	\begin{align*}
		\forall (u_0, v_0) \in V, \frac{\partial (f \circ \varphi)}{\partial v}(u_0, v_0)
		&= \frac{\partial f}{\partial x}\big(\varphi(u_0, v_0)\big) \times \frac{\partial x}{\partial v}(u_0, v_0)\\
		&+ \frac{\partial f}{\partial y}\big(\varphi(u_0,v_0)\big) \frac{\partial y}{\partial v}(u_0,v_0)
	\end{align*}
\end{prop}

\begin{exm}
	[changement de coordonnées polaires]
	On pose \begin{align*}
		\varphi: \R^+_* \times ]0,2\pi[ &\longrightarrow \R^2\setminus \left( R^+_* \times \{0\} \right) \\
		(r, \theta) &\longmapsto (r \cos \theta, r \sin\theta),
	\end{align*}
	\begin{align*}
		f: \R^2\setminus \left( R^+_* \times \{0\} \right) &\longrightarrow \R \\
		(x,y) &\longmapsto f(x,y),
	\end{align*}
	\begin{align*}
		g: \overbrace{\R^+_* \times ]0, 2\pi[}^{=V} &\longrightarrow \R \\
		(r, \theta) &\longmapsto f(r\cos\theta, r\sin\theta).
	\end{align*}

	\begin{align*}
		\forall (r_0,\theta_0) \in V,&\\[5mm]
		\frac{\partial g}{\partial r}(r_0, \theta_0) &= \frac{\partial f}{\partial x}(r_0\cos\theta_0, r_0\sin\theta_0)\cos\theta_0\\
		&+ \frac{\partial f}{\partial y}(r_0 \cos\theta_0, r_0\sin\theta_0)\sin\theta_0\\
		&= 2r_0\cos^2\theta_0 + 2r_0\sin^2(\theta_0) \\
		&= 2r_0 \\[5mm]
		\frac{\partial g}{\partial \theta}(r_0, \theta_0) &= \frac{\partial f}{\partial x}(r_0\cos\theta_0, r_0\sin\theta_0)r_0\sin\theta_0\\
		&+ \frac{\partial f}{\partial y}(r_0 \cos\theta_0, r_0\sin\theta_0)r_0\cos\theta_0\\
		&= -2{r_0}^2\cos(\theta_0)\sin(\theta_0) + 2{r_0}^2 \sin(\theta_0)\cos(\theta_0)\\
		&= 0 \\
	\end{align*}

	Donc, \[
		g(r, \theta) = r^2.
	\]
\end{exm}

\begin{exm}
	Résoudre \[
		\begin{cases}
			\frac{\partial f}{\partial x} = \frac{x}{x^2+y^2},\\
			\frac{\partial f}{\partial y} = \frac{y}{x^2+y^2}.\\
		\end{cases}
	\]

	On pose $g: (r, \theta) \mapsto f(r \cos\theta, r \sin\theta)$.

	\begin{align*}
		&\frac{\partial g}{\partial r} = \frac{1}{r}\cos^2\theta + \frac{1}{r}\sin^2\theta = \frac{1}{r},\\
		&\frac{\partial g}{\partial \theta} = -\cos(\theta) \sin(\theta) + \sin(\theta)\cos(\theta) = 0.
	\end{align*}

	Donc, \[
		\exists C \in \R, g: (r, \theta) \mapsto \ln r + C
	\] d'où,
	\begin{align*}
		\forall (x,y) \in \R^2 \setminus \{(0,0)\}, f(x,y) &= \ln\left(\sqrt{x^2 + y^2} \right)  + C\\
		&= \frac{1}{2}\ln(x^2 + y^2) + C. \\
	\end{align*}
\end{exm}

\begin{rmk}
	Soit $\mathcal{B} = (e_1, e_2)$ la base canonique de $\R^2$, $f: U \to \R$ de classe $\mathcal{C}^1$ avec $U$ un ouvert de $\R^2$.

	Soit $(x,y) \in U$.

	\begin{align*}
		\Mat_{\mathcal{B}}\big(\nabla f(x,y)\big) = \begin{pmatrix}
			\frac{\partial f}{\partial x}(x,y)\\[2mm]
			\frac{\partial f}{\partial y}(x,y)
		\end{pmatrix}
	\end{align*}

	Soit  \begin{align*}
		\varphi: V &\longrightarrow U \\
		(u,v) &\longmapsto \big(x(u,v), y(u,v)\big) 
	\end{align*} avec $x,y$ de classe $\mathcal{C}^1$. Soit $g = f \circ \varphi$.
	\begin{align*}
		\Mat_{\mathcal{B}}\big(\nabla g(u,v)\big)
		&= \begin{pmatrix}
			\frac{\partial g}{\partial u}(u,v) \\[2mm]
			\frac{\partial g}{\partial v}(u,v)
		\end{pmatrix} \\
		&= \begin{pmatrix}
			\frac{\partial x}{\partial u}(u,v) \frac{\partial f}{\partial x}(x,y)
			+ \frac{\partial y}{\partial u}(u,v)\frac{\partial f}{\partial y}(x,y)\\[3mm]
			\frac{\partial x}{\partial v}(u,v) \frac{\partial f}{\partial x}(x,y)
			+ \frac{\partial y}{\partial v}(u,v) \frac{\partial f}{\partial y}(x,y)
		\end{pmatrix}  \\
		&= \underbrace{\begin{pmatrix}
				\frac{\partial x}{\partial u}(u,v)& \frac{\partial y}{\partial u}(u,v)\\[3mm]
				\frac{\partial x}{\partial v}(u,v)& \frac{\partial y}{\partial v}(u,v)
		\end{pmatrix}}_{J(u,v)} \begin{pmatrix}
			\frac{\partial f}{\partial x}(x,y)\\[3mm]
			\frac{\partial f}{\partial y}(x,y)
		\end{pmatrix} \\
		&= J(u,v) \Mat_{\mathcal{B}}\big(\nabla f(x,y)\big) \\
	\end{align*}
	où $J(u,v) = 
	\begin{pNiceArray}{c:c}
		\Mat_{\mathcal{B}}\big(\nabla x(u,v)\big) & \Mat_{\mathcal{B}}\big(\nabla y(u,v)\big)
	\end{pNiceArray}$.

	On dit que $J(u,v)$ est \underline{la jacobienne} de $\varphi$ en $(u,v)$.
	L'application linéaire canoniquement associée à $J(u,v)$ est la \underline{différentielle de $\varphi$} en $(u,v)$ noté $\mathrm{d}\varphi(u,v)$.

	On a $\mathrm{d}\varphi(u,v) \in \mathcal{L}(R^2)$ et $\Mat_{\mathcal{B}}\big(\mathrm{d}\varphi(u,v)\big) = J(u,v)$.

	Par exemple, la jacobienne du changement de coordonnées polaires est \[
		J = \begin{pmatrix}
			\frac{\partial x}{\partial r} & \frac{\partial y}{\partial r}\\[3mm]
			\frac{\partial x}{\partial \theta} & \frac{\partial y}{\partial \theta}
		\end{pmatrix}
		= \begin{pmatrix}
			\cos\theta&\sin\theta\\
			-r\sin\theta&r\cos\theta
		\end{pmatrix}.
	\]
	$\underbrace{\det(J)}_{\text{le jacobien}} = r\cos^2\theta + r\sin^2\theta = r$

	Dans une intégrale double, si $(x,y) = \varphi(u,v)$, alors $\mathrm{d}x\mathrm{d}y = \det(J)\mathrm{d}u\mathrm{d}v$.

	Ici, \[
		\mathrm{d}x\ \mathrm{d}y = r\ \mathrm{d}r\ \mathrm{d}\theta.
	\]
\end{rmk}

\begin{prv}
	On pose $(x_0, y_0) = \varphi(u_0, v_0)$. Pour tout $(h,k) \in \R^2$ tels que $(u_0 + h, v_0 + k) \in V$, en posant $g = f  \circ \varphi$.

	\begin{align*}
		g(u_0 + h, v_0 + h) &= f\big(x(u_0 + h, v_0 + k), y(u_0 + h, v_0 + k)\big) \\
		&= f\left(
			x(u_0,v_0) + h \frac{\partial x}{\partial u}(u_0,v_0) + k \frac{\partial x}{\partial v}(u_0, v_0) + \po\big(\|(h,k)\|\big), \right.\\
		&\phantom{ = f\bigg(\bigg.}\left. y(u_0, v_0) + h \frac{\partial y}{\partial u}(u_0, v_0) + k \frac{\partial y}{\partial v}(u_0, v_0) + \po\big(\|(h,k)\|\big)
		\right)  \\
		&= f(x_0,y_0) \\
		&~+ \left( h \frac{\partial x}{\partial u}(u_0,v_0) + k \frac{\partial x}{\partial v}(u_0, v_0) + \po(\|(h,k)\|) \right) \frac{\partial f}{\partial x}(x_0,y_0)\\
		&~+ \left( h \frac{\partial y}{\partial u}(u_0, v_0) + k\frac{\partial y}{\partial v}(u_0, v_0) + \po(\|(h,k)\|) \right) \frac{\partial f}{\partial y}(x_0, y_0)\\
		&~+ \po(\|(h,k)\|)\\
		&= f(x_0, y_0) \\
		&~+ h \left( \frac{\partial x}{\partial u}(u_0, v_0) \frac{\partial f}{\partial x}(x_0, y_0) + \frac{\partial y}{\partial u}(u_0, v_0) \frac{\partial f}{\partial y}(x_0, y_0) \right)  \\
		&~+ k\left( \frac{\partial x}{\partial v}(u_0, v_0) \frac{\partial f}{\partial x}(x_0, y_0) + \frac{\partial y}{\partial v}(u_0, v_0) \frac{\partial f}{\partial y}(x_0, y_0) \right) 
		&~+ \po(\|(h,k)\|)\\
		&= g(u_0, v_0) + h \frac{\partial g}{\partial u}(u_0, v_0) + k \frac{\partial g}{\partial v}(u_0, v_0) + \po(\|(h,k)\|) \\
	\end{align*}

	Par identification,
	\[
		\frac{\partial g}{\partial u}(u_0, v_0) = \frac{\partial x}{\partial u}(u_0, v_0) \frac{\partial f}{\partial x}(x_0, y_0) + \frac{\partial y}{\partial u}(u_0, v_0) \frac{\partial f}{\partial y}(x_0,y_0)
	\] et \[
		\frac{\partial g}{\partial v}(u_0, v_0) = \frac{\partial x}{\partial v}(u_0,v_0) \frac{\partial f}{\partial x}(x_0, y_0) + \frac{\partial y}{\partial v}(u_0, v_0) \frac{\partial f}{\partial y}(x_0, y_0).
	\] 
\end{prv}

\begin{exm}
	[Régression linéaire]~\\
	\begin{figure}[H]
		\centering
		\begin{asy}
			import graph;
			axes(EndArrow);
			size(5cm);

			real f(real x) { return x + 0.5; }

			real k = 35 / (7 - 0.5);

			for(int i = 0; i < 35; ++i) {
				real mag = exp(sin(100 * pi/exp(1) * i)) * 0.8 + exp(cos(i*40)/3);
				real eps = mag * cos(10 * exp(1)/pi * i) / 3;
				dot((i/k,f(i/k) + eps));
			}

			draw(graph(f, -1, 7), orange);
		\end{asy}
	\end{figure}
	\[
		y = a x + b
	\] 
	On fixe $(a,b) \in \R^2$. \[
		\varepsilon(a,b) = \sum_{i=1}^n\big( y_i - (ax_i + b) \big)^2
	\] l'erreur totale.

	On veut minimiser $\varepsilon(a,b)$. On a 
	\[
		\forall (a,b) \in \R^2,
		\begin{cases}
			\frac{\partial \varepsilon}{\partial a}(a,b) = -2\sum_{i=1}^{n}(y_i - ax_i - b)x_i,\\
			\frac{\partial \varepsilon}{\partial b}(a,b) = -2\sum_{i=1}^{n}(y_i - ax_i - b).
		\end{cases}
	\]

	Donc,
	\begin{align*}
		(a,b) \text{ point critique de } \varepsilon \iff& \begin{cases}
			a \sum_{i=1}^n {x_i}^2 + b\sum_{i=1}^{n}x_i = \sum_{i=1}^{n} y_ix_i\\
			a\sum_{i=1}^{n}x_i + nb = \sum_{i=1}^ny_i
		\end{cases}\\
		\iff& \begin{cases}
			a \left( \frac{1}{n}\sum_{i=1}^n {x_i}^2 - \overline{x}^2\right) = \overline{y} - \overline{x} \overline{y}\\
			b = \frac{1}{n}\sum_{i=1}^ny_i - \frac{a}{n}\sum_{i=1}^nx_i = \frac{1}{n}\sum_{i=1}^n x_i y_i - \overline{x} \overline{y}
		\end{cases}\\
		&\text{ où } \overline{x} = \frac{1}{n} \sum_{i=1}^n x_i,~\overline{y} = \frac{1}{n}\sum_{i=1}^n y_i\\
		\iff& \begin{cases}
			a = \frac{\Cov(x,y)}{V(x)}\\
			b = \overline{y} - a\overline{x}
		\end{cases}
	\end{align*}

	Coefficient de corrélation: $\frac{\Cov(x,y)}{\sigma_x \sigma_y} \in [-1, 1]$
\end{exm}












		\part{Corps}

\begin{exm}[Problème]
	\begin{itemize}
		\item 
			avec $A = \Z / 9 \Z$, résoudre $\overline{x}^2 = \overline{0}$ \\
			\begin{center}
				\begin{tabular}{|c|c|c|c|c|c|c|c|c|c|c|}
					\hline
					$\overline{x}$&$\overline{0}$& $\overline{1}$ &$\overline{2}$&$\overline{3}$ &$\overline{4}$ &$\overline{5}$ &$\overline{6}$ &$\overline{7}$ &$\overline{8}$& $\overline{9}$ \\
					\hline
					$\overline{x}^2$&$\overline{0}$ &$\overline{1}$ &$\overline{4}$ &$\overline{0}$ &$\overline{7}$ &$7$ &$\overline{0}$ &$\overline{4}$ &$\overline{1}$&$\overline{0}$\\
					\hline
				\end{tabular}
			\end{center}
			On a trouvé 3 solutions: $\overline{0}$, $\overline{3}$, $\overline{6}$.
		\item $\Z / 8\Z$
			\begin{center}
				\begin{tabular}{|c|c|c|c|c|c|c|c|c|}
					\hline
					$\overline{x}$& $\overline{0}$& $\overline{1}$& $\overline{2}$& $\overline{3}$& $\overline{4}$& $\overline{5}$& $\overline{6}$& $\overline{7}$\\
					\hline
					$\overline{x^2}$& $\overline{0}$& $\overline{1}$& $\overline{4}$& $\overline{1}$& $\overline{0}$& $\overline{1}$& $\overline{4}$& $\overline{1}$\\
					\hline
				\end{tabular}
			\end{center}
			$\overline{x}^2=7$ a 4 solutions: $\overline{1}, \overline{7}, \overline{3},\text{ et } \overline{5}$
		\item $A = \mathbbm{H} = \{a + bi + cj + dk  \mid  (a,b,c,d) \in \R^4\}$ \\
			$i^2 = j^2 = k^2 = -1$ 
			\begin{align*}
				\begin{array}{c c c}
					ij = k & jk = i & ji = j\\
					ji = -k & kj = -i & ik = -j
				\end{array}
			\end{align*}
			Dans cet anneau, $-1$ a 6 racines!
	\end{itemize}
\end{exm}

\begin{defn}
	Soit $(\mathbbm{K}, +, \times)$ un ensemble muni de deux lois de composition internes. On dit que c'est un \underline{corps} si
	 \begin{enumerate}
		\item $(\mathbbm{K}, \times)$ est un groupe abélien
		\item $(\mathbbm{K}, \times)$ est un monoïde commutatif
		\item $\forall x \in \mathbbm{K}\setminus \{0_\mathbbm{K}\}, \exists y \in \mathbbm{K}, xy = 1_\mathbbm{K}$
		\item $0_\mathbbm{K} \neq  1_\mathbbm{K}$
	\end{enumerate}
	\index{corps}
\end{defn}

\begin{exm}
	\begin{itemize}
		\item $(\C, +, \times)$ est un corps
		\item $(\R, +, \times)$ est un corps
		\item $(\Q, +, \times)$ est un corps
		\item $(\Z, +, \times)$ n'est pas un corps
	\end{itemize}
\end{exm}

\begin{prop}
	$(\Z / n\Z, +, \times)$ est un corps si et seulement si $n$ est premier.
\end{prop}

\begin{prv}
	\[
		\left( \Z / n\Z \right)^\times = \left\{ \overline{k}  \mid k \wedge n = 1 \right\}
	\] 
\end{prv}


\begin{prop}
	Tout corps est un anneau intègre.
\end{prop}

\begin{prv}
	Soit $(\mathbbm{K}, +, \times)$ un corps. Soient $(a,b) \in \mathbbm{K}^2$ tel que $a \times b = 0_\mathbbm{K}$.\\
	On suppose $a \neq  0_\mathbbm{K}$. Alors, $a$ est inversible et donc \[
		b = a^{-1} \times a \times b = a^{-1} \times 0_\mathbbm{K} = 0_\mathbbm{K}
	\] 
\end{prv}

\begin{exm}
	Soit $(\mathbbm{K},+,\times)$ un corps.\\
	Résoudre \[
		\begin{cases}
			x^2 = 1_\mathbbm{K}\\
			x \in \mathbbm{K}
		\end{cases}
	\]

	\begin{align*}
		x^2 = 1_\mathbbm{K} &\iff x^2 - 1_\mathbbm{K} = 0_\mathbbm{K}\\
		&\iff (x - 1_\mathbbm{K})(x+1_\mathbbm{K}) = 0_\mathbbm{K}\\
		&\iff x - 1_\mathbbm{K} = 0_\mathbbm{K} \text{ ou } x + 1_\mathbbm{K} = 0_\mathbbm{K}\\
		&\iff x = 1_\mathbbm{K} \text{ ou } x = -1_\mathbbm{K}
	\end{align*}

	Il y a au plus 2 solutions.
\end{exm}

\begin{prop}
	Soit $(\mathbbm{K},+,\times )$ un corps et $P$ un polynôme à coefficients dans $\mathbbm{K}$ de degré $n$. Alors, l'équation $P(x) = 0_{\mathbbm{K}}$ a au plus $n$ solutions dans $\mathbbm{K}$ 
	\qed
\end{prop}

\begin{crlr}[(Théorème de Wilson)]
	voir exercice 16 du TD 12
\end{crlr}


\begin{defn}
	Soit $(\mathbbm{K}, +, \times)$ un corps et $L\subset \mathbbm{K}$.\\
	On dit que $L$ est un \underline{sous corps} de $\mathbbm{K}$ si
	\begin{enumerate}
		\item $L$ est un anneau de $(\mathbbm{K}, +, \times)$ non nul
		\item $\forall x \in L\setminus \{0_\mathbbm{K}\}, x^{-1} \in L$ 
	\end{enumerate}
	\vspace{2mm}
	en d'autres termes si
	\begin{enumerate}
		\item $\forall (x,y) \in L^2, x - y \in L$
		\item $\forall (x,y) \in L^2, x \times y^{-1} \in L$
	\end{enumerate}
	\vspace{5mm}
	On dit aussi que $\mathbbm{K}$ est une \underline{extension} de $L$.
	\index{sous corps}
	\index{extension}
\end{defn}

\begin{prop}
	Tout sous corps est un corps. \qed
\end{prop}

\begin{defn}
	Soient $(\mathbbm{K}_1,+,\times )$ et $(\mathbbm{K}_2,+, \times)$ deux corps et $f: \mathbbm{K}_1 \to \mathbbm{K}_2$.\\
	On dit que $f$ est un \underline{morphisme de corps} si $f$ est un morphisme d'anneaux.\\
	i.e. si
	\[
		\begin{cases}
			\forall (x,y) \in {\mathbbm{K}_1}^2,& f(x+y) = f(x) + f(y)\\
			\forall (x,y) \in {\mathbbm{K}_1}^2,& f(x \times y) = f(x) \times f(y)\\
		\end{cases}
	\] 
	\index{homomorphisme (de corps)}
	\index{morphisme (de corps)}
\end{defn}

\begin{prop}
	Tout morphisme de corps est injectif.
\end{prop}

\begin{prv}
	Soit $f: \mathbbm{K}_1 \to \mathbbm{K}_2$ un morphisme de corps.\\
	\begin{itemize}
		\item $\Ker(f)$ est un sous groupe de $(\mathbbm{K}_1, +)$ 
		\item Soit $x \in \Ker(f)$ et $y \in \mathbbm{K}_1$ \[
				f(x \times y) = f(x) \times f(y) = 0_{\mathbbm{K}_2} \times f(y) = 0_{\mathbbm{K}_2}
			\]
		\item Soit $x \in \Ker(f) \setminus \{0_{\mathbbm{K}_1}\}$.\\
			Alors, $x$ est inversible.\\
			\begin{align*}
				\begin{rcases*}
					x \in \Ker(f)\\
					x^{-1} \in \mathbbm{K}_1
				\end{rcases*}& \text{ donc } x \times x ^{-1} \in \Ker(f)\\
				&\text{ donc } 1_{\mathbbm{K}_1} \in \Ker(f)\\
				&\text{ donc } f(1_{\mathbbm{K}_1}) = 0_{\mathbbm{K}_2}
			\end{align*}
			Or, $f(1_{\mathbbm{K}_1}) = 1_{\mathbbm{K}_2} \neq 0_{\mathbbm{K}_2}$
	\end{itemize}
	Donc, $\Ker(f) = \{0_{\mathbbm{K}_1}\}$ donc $f$ est injective.
\end{prv}

\begin{exm}
	$\begin{array}{cc}
		\C &\longrightarrow \C\\
		z &\longmapsto \overline{z}\\
	\end{array}$ est un morphisme de corps
\end{exm}



		\part{Opérations sur les séries}

\begin{prop}
	L'ensemble $E = \{u \in \C^\N  \mid \Sigma u_n \text{ converge}\}$ est un sous-espace vectoriel de $\C^\N$ et \begin{align*}
		S: E &\longrightarrow \C \\
		u &\longmapsto \sum_{n=0}^{+\infty} u_n
	\end{align*} est une forme linéaire.
	\qed
\end{prop}

\begin{rmk}
	La somme d'une série convergente et d'une série divergente diverge.
	Le produit d'une série divergente par un scalaire non nul diverge.
\end{rmk}

		\part{Comparaison de suites}

\begin{defn}
	Soient $u$ et $v$ deux suites réelles. On dit que $u$ est \underline{dominée} par  $v$ si \[
	\exists M\in \R, \exists N\in \N,\forall n\ge N,\left| u_n \right| \le M \left| v_n \right| 
	\] Dans ce cas, on note $u = O(v)$ ou $u_n = O(v_n)$ et on dit que "$u$ est un grand o de $v$"
\end{defn}

\begin{exm}
	En informatique, on dit qu'un alogirithme a une \underline{complexité linéaire} si son temps d'éxécution est un $O(n)$ 
	Par exemple, on calcule $a^n$ 

	\begin{itemize}
		\item Approche naïve
			\begin{algorithm}
				\begin{algorithmic}[1]
					\State $p \gets 1$
					\For{$i \in \left\llbracket 0,n-1 \right\rrbracket$}
						\State $p \gets p \times a$
					\EndFor
					\State \Return p
				\end{algorithmic}
			\end{algorithm}
			Complexité linéaire $O(n)$
		\item Exponentiation rapide\\
			On écrit $n$ en binaire: \begin{align*}
				n &= \overline{a_k a_{k-1}\ldots a_0}^{(2)}\\
					&= \sum_{i=0}^{k} a_i 2^i
			\end{align*} avec $(a_i) \in \left\{ 0,1 \right\} ^{k+1}$
			\begin{align*}
				a^n &= a^{\sum_{i=0}^{k} a_i 2^i} \\
				&= \prod_{i=0}^{k} a^{a_i 2^i}  \\
			\end{align*}
			
			\begin{algorithm}
				\begin{algorithmic}
					[1]

					\State $s \gets 0$
					\State $p \gets a$
					\For{ $i \in \left\llbracket 0, \log_2(n) \right\rrbracket$}
						\State $p \gets p \times p$
						\If{$a[i] = 1$}
							\State $s \gets s + p$
						\EndIf
					\EndFor
					\State \Return s
				\end{algorithmic}
			\end{algorithm}
			Compléxité logarithmique $O(\log_2(n))$
	\end{itemize}
\end{exm}


\begin{prop}
	$O$ est une relation réfléctive et transitive.
\end{prop}

\begin{prv}
	\begin{itemize}
		\item Soit $u$ une suite. On pose $M = 1$ et \[
			\forall n \in \N, \left| u_n \right| \le M \left| u_n \right|
			\] Donc $u = O(u)$.
		\item Soient $u, v, w$ trois suites telles que  \[
		\begin{cases}
			u = O(v)\\
			v = O(w)
		\end{cases}
		\] Soient $M_1,M_2 \in \R$ et $N_1,N_2\in \N$ tels que \[
		\begin{cases}
			\forall n \ge  N_1, \left| u_n \right| \le M_1 \left| v_n \right| \\
			\forall n \ge  N_2, \left| v_n \right| \le M_2 \left| w_n \right| \\
		\end{cases}
		\] 

		Nécéssairement, $M_1\ge 0$ et $M_2\ge 0$.\\
		Soit $N = \max(N_1,N_2)$. \[
		\forall n \ge  N, \left| u_n \right| \le M_1 \left| v_n \right| \le  M_1M_2 \left| w_n \right| 
		\] Donc $u = O(w)$
	\end{itemize}
\end{prv}

\begin{defn}
	Soient $u$ et $v$ deux suites. On dit que $u$ est \underline{négligeable} devant $v$ si \[
	\forall \varepsilon>0, \exists N\in \N, \forall n\ge N, \left| u_n \right| \le \varepsilon \left| v_n \right| 
	\] Dans ce cas, on note $u = o(v)$ ou $u_n = o(v_n)$ ou on le lit "$u$ est un petit o de $v$"
\end{defn}

\begin{prop}
	$o$ est une relation transitive, non-réfléctive
\end{prop}

\begin{prv}
	\begin{itemize}
		\item Soient $u$, $v$ et $w$ trois suites telles que \[
			\begin{cases}
				u = o(v)\\
				v = o(w)
			\end{cases}
			\] Soit $\varepsilon>0$. Soit $N_1\in \N$ tel que \[
			\forall n \ge N_1, \left| u_n \right| \le \sqrt{\varepsilon}  \left| v_n \right| 
			\] Soit $N_2\in \N$ tel que \[
			\forall n \ge N_2, \left| v_n \right| \le \sqrt{\varepsilon}  \left| w_n \right| 
			\] On pose $N = \max(N_1,N_2)$, alors \[
			\forall n \ge N, \left| u_n \right| \le \sqrt{\varepsilon}  \left| v_n \right| \le \underbrace{\sqrt{\varepsilon} \times \sqrt{\varepsilon}} _\varepsilon \left| w_n \right| 
			\] donc $u = o(w)$
		\item Soit $u$ une suite tel qu'il existe $N \in \N$ tel que \[
		\forall n \ge N, u_n > 0
		\] On suppose que $u = o(u)$, alors \[
		\forall \varepsilon>0,\exists N \in \N, \forall n \ge N, \left| u_n \right| \le \varepsilon \left| u_n \right| 
		\] On pose $\varepsilon = \frac{1}{2}$ alors \[
		\exists N \in \N, \forall n \ge N, \left| u_n \right| \le \frac{1}{2} \left| u_n \right| 
		\] une contradiction
	\end{itemize}
\end{prv}

\begin{prop}
	Soient $u$ et $v$ deux suites.
	\begin{itemize}
		\item $o(u) + o(u) = o(u)$
		\item $v \times o(u) = o(uv)$
		\item $o(u) \times o(v) = o(uv)$
		\item $o(o(u)) = o(u)$
	\end{itemize}
	\qed
\end{prop}

\begin{defn}
	Soient $u$ et $v$ deux suites. On dit que $u$ et $v$ sont \underline{équivalentes} si \[
	u = v + o(v)
	\] i.e. \[
	\forall \varepsilon >0, \exists N \in \N, \forall n \ge N, \left| u_n-v_n \right| \le \varepsilon\left| v_n \right| 
	\] Dans ce cas, on le note $u \sim v$
\end{defn}

\begin{prop}
	$\sim$ est une relation d'équivalence \qed
\end{prop}

\begin{prop}
	Soient $(u,v) \in \R^\N$. On suppose que $v$ ne s'annule pas à partir d'un certain rang
	\begin{enumerate}
		\item $u = o(v) \iff \left( \frac{u_n}{v_n} \right)$ bornée
		\item $u = o(v) \iff \frac{u_n}{v_n} \tendsto{n \to  +\infty} 0$
		\item $u \sim v \iff \frac{u_n}{v_n} \tendsto{n \to  +\infty} 1$
	\end{enumerate}
	\qed
\end{prop}

\begin{prop}
	[Suites de références]
	\begin{enumerate}
		\item $\ln^\alpha(n) = o(n^\beta)$ avec $(\alpha,\beta) \in \left( \R^+_* \right) ^2$ 
		\item $n^\beta = o(a^n)$ avec $\beta > 0$ et $a > 1$ 
		\item $a^n = o(n!)$ avec $a >1$ 
		\item $n! = o(n^n)$
	\end{enumerate}
\end{prop}


\begin{lem}
	[Exercice 10 du TD]
	Soit $u \in \left(\R^+_*\right)^\N$\\
	Si $\frac{u_{n+1}}{u_n} \tendsto{n \to +\infty} \ell < 1$ avec $\ell\in \R$,\\ alors $u_n \tendsto{n \to +\infty} 0$
\end{lem}

\begin{prv} [de la proposition]
	\begin{enumerate}
		\item par croissance comparée
		\item On pose $\forall n \in \N^*, u_n = \frac{n^\beta}{a^n}$. 
			\begin{align*}
				\forall  n \in \N^*, \frac{u_{n+1}}{u_n} &= \left( \frac{n+1}{n} \right) ^\beta \times \frac{1}{a} \\
				&= \frac{1}{a}\left( 1+\frac{1}{n} \right) ^\beta \\
				&\tendsto{n \to +\infty} \frac{1}{a} < 1
			\end{align*}
			Donc, $u_n \tendsto{n \to  +\infty} 0$
		\item On pose $\forall n \in \N, u_n = \frac{a^n}{n!}$ \[
			\forall n \in \N, \frac{u_{n+1}}{u_n} = \frac{a}{n+1} \tendsto{n \to +\infty} 0 < 1
			\] donc $u_n \tendsto{n \to +\infty} 0$
		\item On pose $\forall  n\in \N^*, u_n = \frac{n!}{n^n}$.
			\begin{align*}
				\forall n \in \N^*, \frac{u_{n+1}}{u_n}
				&= (n+1) {\frac{n^n}{(n+1)^{n+1}}} \\
				&= \left( \frac{n}{n+1} \right) ^n \\
				&= e^{n \ln\left( \frac{n}{n+1} \right) } \\
				&= e^{n \ln\left( 1+\frac{1}{n+1} \right)} \\
				&= e^{n(-\frac{1}{n} + o(\frac{1}{n})} \\
				&= e^{-1 + o(1)} \\
				&\tendsto{n \to  +\infty} e^{-1}<1
			\end{align*}
			donc $u_n \tendsto{n\to +\infty} 0$
	\end{enumerate}
\end{prv}

		\part{Matrices par blocs}

\begin{exm}
	Soit $p$ un projecteur de $E$ : \[
		E = \Ker p \oplus \mathrm{Im}\ p
	\] Soit $\mathcal{B} = (e_1, \ldots, e_k, e_{k+1}, \ldots, e_n)$ une base de $E$ avec $\begin{cases}
		\mathrm{Im}(p) = \Vect(e_1, \ldots, e_k)\\
		\Ker(p) = \Vect(e_{k+1}, \ldots, e_n)\\
	\end{cases}$

	Alors, 
	\begin{align*}
		\Mat_\mathcal{B}(p) =
		\left(\begin{NiceArray}{c c c | c c c}
				1&&&0&\Cdots&0\\
				 &\Ddots&&\Vdots&&\Vdots\\
				&&1&0&\Cdots&0\\\hline
				0&\Cdots&0&0&\Cdots&0\\
				\Vdots&&\Vdots&\Vdots&&\Vdots\\
				0&\Cdots&0&0&\Cdots&0\\
		\end{NiceArray}\right)
		= \left( \begin{array}{c|c}
				I_k & 0\\ \hline
				0&0
		\end{array}\right) \\
	\end{align*}

	De même, si $\s$ est une symétrie de $E$, \[
		E = \Ker(\s - \id_E) \oplus \Ker(\s + \id_E)
	.\] Soit $\mathcal{C} = (e_1', \ldots, e_\ell', e_{\ell+1}', \ldots, e'_n)$ avec $\begin{cases}
		\Vect(e'_1, \ldots, e'_\ell) = \Ker(\s - \id_E),\\
		\Vect(e'_{\ell+1}, \ldots, e'_n) = \Ker(\s + \id_E).\\
	\end{cases}$

	Alors
	\[
		\Mat_\mathcal{C}(\s) = \left(\begin{array}{c|c}
				I_\ell &0\\ \hline
				0&-I_{n-\ell}
		\end{array}\right) 
	\]
\end{exm}

\begin{prop}
	Soient $F$ et $G$ supplémentaires dans $E$ : \[
		E = F \oplus G.
	\] Soit $f \in \mathcal{L}(F)$ et $g \in \mathcal{L}(G)$. Alors \[
	\exists !h \in \mathcal{L}(E) h_{|F} = f,\ h_{|G} = g \et h = f \circ p + g \circ q
	\] où $\begin{cases}
		p \text{ est la projection sur $F$ parallèlement à $G$}\\
		q \text{ est la projection sur $G$ parallèlement à $F$}\\
	\end{cases}$.

	On a aussi $q = \id_E - p$.
\end{prop}

\begin{prv}
	\begin{itemize}
		\item[\sc \underline{Analyse}] Soit $h \in \mathcal{L}(E)$ tel que $\begin{cases}
				h_{|F}=f\\
				h_{|G}=g
			\end{cases}$.

			Soit $x \in E$. Alors \[
				x = \underbrace{p(x)}_{\in F} + \underbrace{q(x)}_{\in G}
			\]

			Donc,
			\begin{align*}
				h(x) &= h\big(p(x)\big) + h\big(q(x)\big)\\
				&= f\big(p(x)\big) + g\big(q(x)\big) \\
				&= (f \circ p + g \circ q)(x) \\
			\end{align*}
			Si $h$ existe, alors \[
				h = f \circ p + g \circ q
			\]
		\item[\underline{\sc Synthèse}] On pose $h = f \circ p + g  \circ q$.

			$p$, $q$, $f$ et $g$ sont linéaires donc $h$ aussi.

			Soit $x \in E$.
			\begin{align*}
				h(x) &= f\big(p(x)\big) + g\big(q(x)\big) \\
				&= f(x) + g(0_E) \\
				&= f(x) \\
			\end{align*}
			donc $h_{|F} = f$ et de même $h_{|G}=g$.
	\end{itemize}
\end{prv}

\begin{prop}
	On reprend les notations et hypothèses précédentes. Soit $(e_1, \ldots, e_p)$ une base de $F$, et $(f_1, \ldots, f_q)$ une base de $G$. Alors, $\mathcal{B} = (e_1, \ldots, e_p, f_1, \ldots, f_q)$ est une base de $E$ et \[
		\Mat_\mathcal{B}(h) = \left(
		\begin{array}{c|c}
			A&0\\ \hline
			0&B
		\end{array}\right)
	\] où $\begin{cases}
		A = \Mat_{(e_1, \ldots e_p)}(f)\\
		B = \Mat_{(f_1, \ldots, f_q)}(g)
	\end{cases}$
	\qed
\end{prop}

\begin{prop}
	Soient $(A,A') \in \mathcal{M}_n(\mathbbm{K})^2$ et $(B,B') \in \mathcal{M}_p(\mathbbm{K})^2$.
	\begin{enumerate}
		\item \[
				\left(\begin{array}{c|c}
					A&0\\ \hline
					0&B
				\end{array}\right)
				\left(\begin{array}{c|c}
					A'&0\\ \hline
					0&B'
				\end{array}\right) = 
				\left(\begin{array}{c|c}
					AA'&0\\ \hline
					0&BB'
				\end{array}\right)
			\]
		\item \[
				\left(\begin{array}{c|c}
					A&0\\ \hline
					0&B
				\end{array}\right) \in \mathrm{GL}_{n+p}(\mathbbm{K})	 \iff \begin{cases}
					 A \in \mathrm{GL}_n(\mathbbm{K})\\
					 B \in \mathrm{GL}_p(\mathbbm{K})
				\end{cases}
			\] et dans ce cas, \[
				\left(\begin{array}{c|c}
					A&0\\ \hline
					0&B
				\end{array}\right)^{-1} =
				\left(\begin{array}{c|c}
					A^{-1}&0\\ \hline
					0&B^{-1}
				\end{array}\right)
			\]
		\item \[
				\tr \left(\begin{array}{c|c}
					A&0\\ \hline
					0&B
				\end{array}\right) = \tr A + \tr B
			\]
	\end{enumerate}
\end{prop}

\begin{prv}
	\begin{enumerate}
		\item Soit $\begin{cases}
				f \in \mathcal{L}(F) \text{ tel que } \Mat_\mathcal{B}(f) = A,
				f' \in \mathcal{L}(F) \text{ tel que } \Mat_\mathcal{B}(f') = A',
				g \in \mathcal{L}(G) \text{ tel que } \Mat_\mathcal{C}(g) = B,
				g' \in \mathcal{L}(G) \text{ tel que } \Mat_\mathcal{C}(g') = B'
			\end{cases}$ où $\begin{cases}
				F \oplus G = \mathbbm{K}^{n+p},\\
				\dim(F) = n, \dim(G) = p,\\
				\mathcal{B} \text{ base de } F,\\
				\mathcal{C} \text{ base de } G.\\
			\end{cases}$
			Soit $\begin{cases}
				h \in \mathcal{L}(\mathbbm{K}^{n+p}) \text{ tel que } \begin{cases}
					h_{|F} = f\\
					h_{|G} = g
				\end{cases}\\
				h' \in \mathcal{L}(\mathbbm{K}^{n+p}) \text{ tel que } \begin{cases}
					h'_{|F} = f'\\
					h'_{|G} = g'\\
				\end{cases}
			\end{cases}$
			Soit $\mathcal{D} = \mathcal{B} \cup \mathcal{C}$ une base de $\mathbbm{K}^{n+p}$.
			\begin{align*}
				\left(\begin{array}{c|c}
					A&0\\ \hline
					0&B
				\end{array}\right)
				\left(\begin{array}{c|c}
					A'&0\\ \hline
					0&B'
				\end{array}\right) &= \Mat_{\mathcal{D}}(h) \Mat_{\mathcal{D}}(h')\\
				&= \Mat_{\mathcal{D}}(h \circ h') \\
			\end{align*}
			Or, $(h \circ h')_{|F} = f \circ f'$ et $(h \circ h')_{|G} = g \circ g'$.

			Donc,
			\begin{align*}
				\Mat_\mathcal{D}(h \circ h') &=
					\left(\begin{array}{c|c}
						\Mat_\mathcal{B}(f \circ f')&0\\ \hline
						0&\Mat_\mathcal{C}(g \circ g')
					\end{array}\right)\\
				&=\left(\begin{array}{c|c}
					AA'&0\\ \hline
					0&BB'
				\end{array}\right).
			\end{align*}
	\end{enumerate}
\end{prv}

\begin{prop}
	Soient $A,A' \in \mathcal{M}_n(\mathbbm{K})$, $B,B' \in \mathcal{M}_{n,p}(\mathbbm{K})$, $C,C' \in \mathcal{M}_{p,n}(\mathbbm{K})$ et $D, D' \in \mathcal{M}_p(\mathbbm{K})$.

	\[
		\left(\begin{array}{c|c}
			A&B\\ \hline
			C&D
		\end{array}\right)
		\left(\begin{array}{c|c}
			A'&B'\\ \hline
			C'&D'
		\end{array}\right) = 
		\left(\begin{array}{c|c}
			AA' + BC'& AB' + BD'\\ \hline
			CA' + DC'&CB' + DD'
		\end{array}\right)
	\] Cette formule se généralise à un nombre quelconque de blocs : \[
		\left(\begin{array}{c|c|c|c}
				A_{11}&A_{12}&\cdots&A_{1,n}\\ \hline
				A_{21}&A_{22}&\cdots&A_{2,n}\\ \hline
				\vdots&\vdots&\ddots&\vdots\\ \hline
				A_{p,1}&A_{p,2}&\cdots&A_{p,n}
		\end{array}\right)
		\left(\begin{array}{c|c|c|c}
				A'_{11}&A'_{12}&\cdots&A'_{1,n}\\ \hline
				A'_{21}&A'_{22}&\cdots&A'_{2,n}\\ \hline
				\vdots&\vdots&\ddots&\vdots\\ \hline
				A'_{p,1}&A'_{p,2}&\cdots&A'_{p,n}
		\end{array}\right)
	\] Cette matrice se calcyle comme on s'y attend si les dimensions des blocs autorisent les produits.
\end{prop}

\begin{prop}
	Le rang d'une matrice $A$, c'est la taille de la plus grande matrice carrée inversible que l'on peut extraire de $A$.
	\qed
\end{prop}




	}

	{
		\chap[22]{Fonctions de deux variables}
		\renewcommand{\cwd}{../chap22}
		\begin{defn}
	Un \underline{proposition} est un énoncé qui est soit vrai, soit faux.
\end{defn}

\begin{exm}
	\begin{align*}
		\begin{rcases*}
			A: ``B \text{ est vraie }"\\
			B: ``A \text{ est fausse }"\\
		\end{rcases*} \text{ Le système $\{A,B\}$ est une \underline{auto-contradiction}}
	\end{align*}
\end{exm}

\begin{defn}
	\underline{Démontrer} une proposition revient à prouver qu'elle est vraie
\end{defn}

		\begin{defn}
	Soit $E$ un $\mathbbm{K}$-espace vectoriel. On dit que $E$ est de \underline{dimension finie} si $E$ a au moins une famille génératrice finie. On dit que $E$ est de \underline{dimension infinie} sinon.
	\index{dimension finie (espace vectoriel)}
	\index{dimension infinie (espace vectoriel)}
\end{defn}

\begin{thm}
	[Théorème de la base extraite]
	Soit $E$ un $\mathbbm{K}$-espace vectoriel non nul de dimension finie. Soit $\mathcal{G}$ une famille génératrice finie de $E$. Alors, il existe une base $\mathcal{B}$ de $\mathcal{E}$ telle que $\mathcal{B} \subset \mathcal{G}$.
\end{thm}

\begin{prv}
	[par récurrence sur $\#G = \Card(G)$]
	\begin{itemize}
		\item Soit $E$ un $\mathbbm{K}$-espace vectoriel non nul engendré par $\mathcal{G} = (u)$.\\
			Si $u = 0_E$, alors $E = \{0_E\}$: une contradiction $\lightning$ \\
			Donc $u \neq 0_E$ donc $(u)$ est libre. En effet, \[
				\forall \lambda \in \mathbbm{K}, \lambda u = 0_E \implies \lambda = 0_\mathbbm{K}
			\] Donc $\mathcal{G}$ est une base de $E$.\\
		\item Soit $n \in \N_*$. Soit $E$ un $\mathbbm{K}$-espace vectoriel. On suppose que si $E$ a une famille génératrice constituée de $n$ vecteurs, alors on peut extraire de cette famille une base de $E$.\\
			Soit $\mathcal{G}$ une famille génératrice de $E$ avec $n+1$ vecteurs.\\
			Si $\mathcal{G}$ est libre, alors $\mathcal{G}$ est une base de $E$. \\
			Si $\mathcal{G}$ n'est pas libre, alors il existe $u \in \mathcal{G}$ tel que $u \in \Vect(\mathcal{G}\setminus \{u\})$ \\
			Donc $\mathcal{G}\setminus \{u\}$ engendre $E$. Or, $\mathcal{G}\setminus \{u\}$ possède $n$ vecteurs. D'après l'hypothèse de récurrence, il existe une base $\mathcal{B}$ de $E$ telle que \[
				\mathcal{B} \subset \mathcal{G} \setminus \{u\} \subset \mathcal{G}
			\] 
	\end{itemize}
\end{prv}

\begin{crlr}
	Tout espace de dimension finie a une base.
	\qed
\end{crlr}

\begin{thm}
	[Théorème de la base incomplète]
	Soit $E$ un $\mathbbm{K}$-espace vectoriel de dimension finie, $\mathcal{G}$ une famille génératrice finie de $E$. $\mathcal{L}$ une famille libre de $E$. Alors, il existe une base $\mathcal{B}$ de $E$ telle que \[
		\mathcal{L} \subset \mathcal{B} \text{ et } \mathcal{B}\setminus \mathcal{L} \subset \mathcal{G}
	\] 
\end{thm}

\begin{prv}
	[par récurrence sur $\#(\mathcal{G}\setminus\mathcal{L})$]
	\begin{itemize}
		\item Avec les notations précédentes, on suppose que $\mathcal{G}\setminus\mathcal{L} \neq \O$ \[
				\forall u \in \mathcal{G}, u \in \mathcal{L}
			\] Donc $\mathcal{G} \subset \mathcal{L}$ donc $\mathcal{L}$ est génératrice donc $\mathcal{L}$ est une base de $E$. On pose $\mathcal{B} = \mathcal{L}$ et alors \[
				\mathcal{L} \subset  \mathcal{B} \text{ et } \mathcal{B}\setminus\mathcal{L} = \O \subset  \mathcal{G}
			\] 
		\item Soit $n \in \N$. On suppose que si $\mathcal{G}$ est génératrice et $\mathcal{L}$ libre avec $\#(\mathcal{G}\setminus\mathcal{L}) = n$ alors il existe une base $\mathcal{B}$ de $E$ telle que \[
			\mathcal{L}\subset \mathcal{B} \text{ et } \mathcal{B}\setminus\mathcal{L}\subset \mathcal{G}
		\] Soient à présent $\mathcal{G}$ une famille génératrice de $E$ et $\mathcal{L}$ une famille libre de $E$ telles que $\#(\mathcal{G}\setminus\mathcal{L}) = n+1 > 0$\\
		Si $\mathcal{L}$ engendre $E$, alors $\mathcal{L}$ est une base de $E$. On pose $\mathcal{B} = \mathcal{L}$ et on a bien \[
			\mathcal{L} \subset  \mathcal{B} \text{ et } \mathcal{B} \setminus \mathcal{L} = \O \subset  \mathcal{G}
		\] On suppose que $\mathcal{L}$ n'engendre pas $E$. Il existe $u \in \mathcal{G}$ tel que $u \not\in \Vec(\mathcal{L})$ (car sinon, $\mathcal{G} \subset \Vect(\mathcal{L})$ et donc $\underbrace{\Vect(\mathcal{G})}_{= E} \subset  \underbrace{\Vect(\mathcal{L})}_{ \subset E}$\\
		Donc $\mathcal{L} \cup \{u\} $ est libre. On pose $\mathcal{L}' = \mathcal{L} \cup \{u\} $ \[
			\mathcal{G}\setminus \mathcal{L}' = \mathcal{G}\setminus (\mathcal{L} \cup \{u\}) = (\mathcal{G}\setminus\mathcal{L})\setminus \{u\} 
		\] donc $\#(\mathcal{G}\setminus\mathcal{L}') = n+1 -1 = n$\\
		D'après l'hypothèse de récurrence, il existe $\mathcal{B}$ une base de $E$ telle que \[
			\mathcal{L} \subset  \mathcal{L}' \subset \mathcal{B} \text{ et } \mathcal{B}\setminus \mathcal{L}' \subset \mathcal{G}
		\] \[
			\mathcal{B} \setminus \mathcal{L} = \underbrace{\mathcal{B}\setminus\mathcal{L}'}_{\subset \mathcal{G}} \cup \underbrace{\{u\}}_{\subset \mathcal{G} \text{ car } u \in \mathcal{G}}
		\] On a $\mathcal{B}\setminus\mathcal{L}\subset \mathcal{G}$
	\end{itemize}
\end{prv}

\begin{thm}
	Soit $E$ un $\mathbbm{K}$-espace vectoriel de dimension finie. Toutes les bases de $E$ ont le même cardinal.
\end{thm}

\begin{prv}
	Soit $\mathcal{G}$ une famille génératrice finie de $E$ et $\mathcal{B} \subset  \mathcal{G}$ une base de $E$. On note $n = \#\mathcal{B}$ \\
	Soit $\mathcal{B}'$ une base de $E$. On pose $p = n - \#(\mathcal{B} \cap  \mathcal{B}')$. Montrons par récurrence sur  $p$ que $\#\mathcal{B} = \#\mathcal{B}'$ 
	\begin{itemize}
		\item On suppose que $p = 0$. Alors, $\#(\mathcal{B} \cap \mathcal{B}') = n$ \\
			Or, $\mathcal{B}' \cap \mathcal{B} \subset \mathcal{B}$ donc $\mathcal{B} \cap \mathcal{B}' = \mathcal{B}$ donc $\mathcal{B} \subset  \mathcal{B}'$ et donc $\mathcal{B} = \mathcal{B}'$ 
		\item Soit $p \in \N$. On suppose que si $\mathcal{B}'$ est une base de $E$ telle que $n - \#(\mathcal{B} \cap \mathcal{B}') = p$, alors $\#\mathcal{B}' = n$ \\
			Aoit $\mathcal{B}'$ une base de $E$ telle que $n - \#(\mathcal{B}\cap \mathcal{B}') = p+1 > 0$ \\
			Donc $\mathcal{B} \cap \mathcal{B}' \neq \mathcal{B}$. Soit $u \in \mathcal{B}' \setminus \mathcal{B}$. D'après le lemme d'échange, il existe $v \in \mathcal{B}\setminus \mathcal{B}'$ tel que $\mathcal{B}' \setminus \{u\} \cup \{v\}$ est une base de $E$. On pose $\mathcal{B}'' = \mathcal{B}' \setminus \{u\} \cup \{v\}$ 
			\begin{align*}
				\mathcal{B}'' \cap \mathcal{B} &= \left( (\mathcal{B}' \setminus \{u\})  \cap \mathcal{B} \right) \cup \{v\} \\
				&= (\mathcal{B}' \cap \mathcal{B}) \cup \{v\} \\
			\end{align*}
			donc,
			\begin{align*}
				n - \#(\mathcal{B}'' \cap \mathcal{B}) &= n - (\#(\mathcal{B}' \cap \mathcal{B}) + 1) \\
				&= p+1- 1 \\
				&= p \\
			\end{align*}
			D'après l'hypothèse de récurrence, \[
				\#\mathcal{B}'' = n
			\] Or, $\#\mathcal{B}'' = \#\mathcal{B}'$
	\end{itemize}
\end{prv}

\begin{lem}
	Soient $\mathcal{B}$ et $\mathcal{B}'$ deux bases de $E$ telles que $\mathcal{B}\subset \mathcal{B}'$. Alors, $\mathcal{B} = \mathcal{B}'$.
\end{lem}

\begin{prv}
	On suppose $\mathcal{B}' \neq \mathcal{B}$. Soit $u \in \mathcal{B}' \setminus \mathcal{B}$
	$u \in E = \Vect(\mathcal{B})$ donc $\mathcal{B} \cup \{u\}$ n'est pas libre.
	Donc $\mathcal{B}\cup \{u\} \subset \mathcal{B}'$ et $\mathcal{B}'$ est libre donc $\mathcal{B}\cup \{u\}$ est libre: une contradiction $\lightning$
\end{prv}

\begin{lem}
	[Lemme d'échange] Soient $\mathcal{B}_1$ et $\mathcal{B}_2$ deux bases de $E$ et $u \in \mathcal{B}_1 \setminus \mathcal{B}_2$. Alors, il existe $v \in \mathcal{B}_2$ tel que $(\mathcal{B}_1 \setminus \{u\}) \cup \{v\}$ soit une base de $E$.
\end{lem}

\begin{prv}
	[1${}^\text{nde}$ méthode]
	On suppose que pout tout $v \in \mathcal{B}_2$, $(\mathcal{B}_1\setminus \{u\}) \cup \{v\}$ n'est pas une base de $E$
	Soit $v \in \mathcal{B}_2$.
	\begin{itemize}
		\item Supposons $(\mathcal{B}_1\setminus \{u\})\cup \{v\}$ non libre. $\mathcal{B}_1 \setminus \{u\}$ est libre. Donc $v \in \Vect(\mathcal{B}_1 \setminus \{u\})$
		\item Supposons $(\mathcal{B}_1\setminus \{u\}) \cup \{v\}$ non génératrice.
			Comme $\mathcal{B}_1$ engendre $E$, $u \not\in \Vect(\mathcal{B}_1\setminus \{v\})$.
			On suppose que $\mathcal{B}_1 \neq \mathcal{B}_2$.
			$\forall v \in \mathcal{B}_2 \setminus \mathcal{B}_1, \Vect(\mathcal{B}_1 \setminus \{v\}) = \Vect(\mathcal{B}_1) = E \ni u$ 
			donc, $(\mathcal{B}_1\setminus \{u\}) \cup \{v\}$ engendre $E$ et donc \[
				v \in \Vect(\mathcal{B}_1 \setminus \{u\})
			\] On a aussi \[
				\forall v \in \mathcal{B}_1 \setminus \{u\}, v \in \Vect(\mathcal{B}_1\setminus \{u\})
			\] Comme $u \not\in \mathcal{B}_2$, on a \[
				\forall v \in \mathcal{B}_2, v \in \Vect(\mathcal{B}_1\setminus \{u\})
			\] docn \[
				E = \Vect(\mathcal{B}_2) \subset \Vect(\mathcal{B}_1\setminus \{u\})
			\] donc $\mathcal{B}_1\setminus \{u\}$ engendre $E$ donc $\mathcal{B}_1\setminus \{u\}$ est une base de $E$. Or, $\mathcal{B}_1 \setminus \{u\}  \subset  \mathcal{B}_1$, donc $\mathcal{B}_1\setminus \{u\} = \mathcal{B}_1$
	\end{itemize}
\end{prv}

\begin{prv}
	[2${}^\text{nde}$ méthode]
	On suppose que pout tout $v \in \mathcal{B}_2$, $(\mathcal{B}_1\setminus \{u\}) \cup \{v\}$ n'est pas une base de $E$
	\begin{itemize}
		\item Comme $u \in \mathcal{B}_1 \setminus \mathcal{B}_2$, nécéssairement $\mathcal{B}_1 \neq \mathcal{B}_2$ donc $\mathcal{B}_2 \not\subset \mathcal{B}_1$, donc $\mathcal{B}_2\setminus\mathcal{B}_1 \neq \O$ 
		\item Soit $v \in \mathcal{B}_2\setminus\mathcal{B}_1$. Il existe $(\lambda_w)_{w\in\mathcal{B}_1}$ une famille de scalaires presque nulle telle que \[
				v = \sum_{w \in \mathcal{B}_1} \lambda_w w - \lambda_u u + + \sum_{w \in \mathcal{B}_1\setminus \{u\}}\lambda_w w
			\]
			Si $\lambda_u \neq 0_E$, alors
			\begin{align*}
				u &= \lambda_u^{-1}\left( v - \sum_{w \in \mathcal{B}_1 \setminus \{u\}} \lambda_w w \right)\\
					&\in \Vect(\mathcal{B}_1\setminus \{u\} \cup v)
			\end{align*}
			 donc $\mathcal{B}_1 \subset \Vect(\mathcal{B}_1\setminus \{u\} \cup \{v\})$\\
			 et donc $E \subset  \Vect(\mathcal{B}_1 \setminus \{u\} \cup \{v\})$ \\
			 et donc $\mathcal{B}_1 \setminus \{u\} \cup \{v\}$ engendre $E$ \\
			 donc $\mathcal{B}_1 \setminus \{u\} \cup \{v\}$ n'est pas libre\\
			 donc $v \in \Vect(\mathcal{B}_1\setminus \{u\})$ (car $\mathcal{B}_1 \setminus \{u\}$ est libre\\
			 donc $\lambda_u = 0_\mathbbm{K}$ $\lightning$\\`

			 Donc, $\lambda_u = 0_\mathbbm{K}$, docn $v \in \Vect(\mathcal{B}_1\setminus \{u\})$ \\
			 On vient de prouver que
			 \begin{align*}
			 	\mathcal{B}_2 \setminus \mathcal{B}_1 \subset \Vect(\mathcal{B}_1 \setminus \{u\})\\
			 	\mathcal{B}_1 \setminus \{u\} \subset \Vect(\mathcal{B}_1 \setminus \{u\})\\
			 \end{align*}
			 Comme $u \not\in \mathcal{B}_2$, \[
			 	\mathcal{B}_2 \subset \Vect(\mathcal{B}_1 \setminus \{u\})
			 \] donc \[
			 	E = \Vect(\mathcal{B}_2) \subset  \Vect(\mathcal{B}_1 \setminus \{u\})
			 \] donc $\mathcal{B}_1 \setminus \{u\}$ engendre $E$. Donc,  $\mathcal{B}_1 \setminus \{u\}$ est une base de $E$.\\
			 Or, $\mathcal{B}_1 \setminus \{u\} \subset  \mathcal{B}_1$, donc $\mathcal{B}_1 \setminus \{u\} = \mathcal{B}_1$
	\end{itemize}
\end{prv}

\begin{defn}
	Soit $E$ un $\mathbbm{K}$-espace vectoriel de dimension finie. Le cardinal commun à toutes les bases de $E$ est appelé \underline{dimension} de $E$ est notée $\dim(E)$ ou $\dim_\mathbbm{K}(E)$\\
	C'est donc aussi le nombre de coordonnées de n'importe quel vecteur dans n'importe quelle base.
	\index{dimension (espace vectoriel)}
\end{defn}

\begin{exm}
	\begin{enumerate}
		\item $\dim_\R(\C) = 2$ et $\dim_\C(\C) = 1$ 
		\item $\dim_\mathbbm{K}(\mathbbm{K}^{n}) = n$ 
		\item $\dim_{\mathbbm{K}}(\mathcal{M}_{n,p}(\mathbbm{K})) = np$
	\end{enumerate}
\end{exm}

\begin{crlr}
	Soit $E$ un $\mathbbm{K}$-espace vectoriel de dimension finie, $\mathcal{L}$ une famille libre de $E$, $\mathcal{G}$ une famille génératrice de $E$. On note $n = \dim(E)$
	\begin{enumerate}
		\item $\#\mathcal{G} \ge n$ et $(\#\mathcal{G} = n \implies \mathcal{G} \text{ est une base de } E$)
		\item $\#\mathcal{L} \le n$ et $(\#\mathcal{L} = n \implies \mathcal{L} \text{ est une base de } E$)
	\end{enumerate}
\end{crlr}

\begin{crlr}
	$\R^{\R}$ est de dimension infinie.
	$\forall i \in \N, e_i: x \mapsto x^i$\\
	$(e_i)_{i\in\N}$ est libre dans $\R^\R$
\end{crlr}

\begin{prop}
	Soient $E$ et $F$ deux $\mathbbm{K}$-espaces vectoriels de dimension finie. Alors $E\times F$ est de dimension finie et $\dim(E\times F) = \dim(E) + \dim(F)$
\end{prop}

\begin{prv}
	Soit $(e_1,\ldots, e_n)$ une base de $E$, $(f_1, \ldots, f_p)$ une base de $F$.
	On pose \[
		\left\{\begin{array}
			{r c l}
			u_1 &=& (e_1,0_F)\\
			u_2 &=& (e_2,0_F)\\
					&\vdots&\\
			u_n &=& (e_n,0_F)\\
			u_{n+1} &=& (0_E, f_1)\\
			u_{n+2} &=& (0_E, f_2)\\
					&\vdots&\\
			u_{n+p} &=& (0_E,f_p)\\
		\end{array}\right.
	\]
	Soit $(x,y) \in E\times F$. \[
		\begin{cases}
			\exists (x_1,\ldots,x_n)\in \mathbbm{K}^n, x = \sum_{i=1}^{n} x_ie_i
			\exists (y_1,\ldots,y_n)\in \mathbbm{K}^n, x = \sum_{j=1}^{p} y_jf_j
		\end{cases}
	\] 
	\begin{align*}
		(x,y) &= \left( \sum_{i=1}^{n} x_ie_i, \sum_{i=1}^{p} y_jf_j \right)  \\
		&= \sum_{i=1}^{n} x_i (e_i + 0_F) + \sum_{j=1}^{p} y_j (0_E, f_j) \\
		&= \sum_{i=1}^{n} x_i u_i + \sum_{j=1}^{p} y_j u_{n+j} \\
	\end{align*}
	Donc, $E\times F = \Vect(u_1, \ldots, u_{n+p})$ donc $E\times F$ est de dimension finie.\\
	Soit $(\lambda_1, \ldots, \lambda_{n+p}) \in \mathbbm{K}^{n+p}$ tel que \[
		(*): \quad \sum_{k=1}^{n+p} \lambda_ku_k = 0_{E\times F} = (0_E, 0_F)
	\]
	\begin{align*}
		(*) &\iff \sum_{k=1}^{n} \lambda_k (e_k, 0_F) + \sum_{k=n+1}^{p} \lambda_k(0_E, f_{k-n}) = (0_E, 0_F)\\
				&\iff \begin{cases}
					\sum_{k=1}^{n} \lambda_k e_k = 0_E\\
					\sum_{k=n+1}^{p} \lambda_k f_{k-n} = 0_F
				\end{cases}\\
				&\iff \begin{cases}
					\forall k \in \left\llbracket 1,n \right\rrbracket, \lambda_k = 0_\mathbbm{K} \qquad&(\text{car $(e_1,\ldots,e_n)$ est libre})\\
					\forall k \in \left\llbracket n+1,n+p \right\rrbracket, \lambda_k = 0_\mathbbm{K} \qquad&(\text{car $(f_1,\ldots,f_n)$ est libre})\\
				\end{cases}
	\end{align*}
	Donc $(u_1, \ldots, u_{n+p})$ est une base de $E\times F$. Donc, $\dim(E\times F) = n + p = \dim(E) + \dim(F)$
\end{prv}

\begin{rmk}
	[Convention]
	\[\dim\big(\{0_E\}\big) = 0\]
\end{rmk}

\begin{thm}
	Soit $E$ un $\mathbbm{K}$-espace vectoriel de dimension finie, $F$ un sous-espace vectoriel de $E$. Alors, $F$ est de dimension finie et  $\dim(F) \le \dim(E)$\\
	Si $\dim(F) = \dim(E)$, alors $F = E$
\end{thm}

\begin{prv}
	On considère \[
		A = \{k \in \N \mid \text{il existe une famille libre de $F$ à $k$ éléments}\} 
	\]
	On suppose $F \neq \{0_E\}$.
	\begin{itemize}
		\item Soit $u \in F\setminus \{0_E\}$. $(u)$ est libre donc $1 \in A$ et donc $A \neq \O$
		\item Soit $\mathcal{L}$ une famille libre de $F$. Alors, $\mathcal{L}$ est une famille libre de $E$ \\
			donc $\#\mathcal{L} \le \dim(E)$\\
			Donc $A$ est majorée par $\dim(E)$ \\
			On en déduit que $A$ a un plus grand élément $p$.
		\item Soit $\mathcal{L}$ une famille libre de $F$ avec $p$ éléments.\\
			Si $\mathcal{L}$ n'engendre pas $F$, alors il existe $u\in F$ tel que $u\not\in \Vect(\mathcal{L})$ et donc $\mathcal{L} \cup \{u\}$ est une famille libre de $F$, donc $p+1 \in A$ en contradiction avec la maximalité de $p$.\\
			Donc $\mathcal{L}$ est une base de $F$ donc $F$ est de dimension finie et $\dim(F) = p \le \dim(E)$\\
	\end{itemize}

	Soit $\mathcal{B}$ une base de $F$. Alors, $\mathcal{B}$ est aussi une famille de libre de de $E$. Donc $\#\mathcal{B} \le \dim(E)$ donc $\dim(F) = \dim(E)$ \\
	Si $\dim(F) = \dim(E)$, alors $\mathcal{B}$ est une base de $E$, et donc $F = \Vect(\mathcal{B}) = E$
\end{prv}

\begin{prop}
	[Formule de Grassmann]
	Soit $E$ un $\mathbbm{K}$-espace vectoriel de dimension finie, $F$ et $G$ deux sous-espace vectoriels de $E$. Alors, \[
		\dim(F+G) = \dim(F) + \dim(G) - \dim(F\cap G)
	\] 
\end{prop}

\begin{prv}
	Soit $(e_1, \ldots, e_p)$ une base de $F\cap G$. $(e_1,\ldots,e_p)$ est une famille libre de $F$.\\
	On complète $(e_1, \ldots, e_p)$ en une base $(e_1, \ldots, e_p, u_1, \ldots, u_q)$ de $F$.\\
	De même, on complète $(e_1, \ldots, e_p)$ en une base $(e_1, \ldots, e_p, v_1, \ldots, v_r)$ de $G$.\\
	On pose  $\mathcal{B} = (e_1, \ldots, e_p, u_1, \ldots, u_q, v_1, \ldots, v_r)$. Montrons que $\mathcal{B}$ est une base de $F+G$
	\begin{itemize}
		\item Soit $u \in F+G$ \\
			On pose $u = v+w$ avec $\begin{cases}
				v\in F\\
				w \in G
			\end{cases}$.\\
			On pose $v = \sum_{i=1}^p \lambda_i e_i + \sum_{i=1}^q \mu_i u_i$ avec $(\lambda_1, \ldots, \lambda_p, \mu_1, \ldots, \lambda_q) \in \mathbbm{K}^{p+q}$\\
			On pose aussi $w = \sum_{i = 1}^p \lambda'_ie_i + \sum_{j=1}^r \nu_j v_j$ avec $(\lambda_1',\ldots,\lambda_p', \nu_1, \ldots, \nu_r) \in \mathbbm{K}^{p+r}$\\
			D'où, \[
				u = \sum_{i=1}^p (\lambda_i + \lambda'_i)e_i + \sum_{j=1}^q \mu_j u_j + \sum_{k=1}^r \nu_k v_k \in \Vect(\mathcal{B})
			\]
		\item Soient $(\lambda_1, \ldots, \lambda_p, \mu_1, \ldots, \mu_q, \nu_1, \ldots, \nu_r) \in \mathbbm{K}^{p+q+r}$.\\
			On suppose \[
				(*)\quad \sum_{i=1}^{p}\lambda_ie_i + \sum_{j=1}^q\mu_ju_j + \sum_{k=1}^r \nu_k v_k = 0_E
			\] 
			D'où, \[
				\underbrace{\sum_{i=1}^p\lambda_i e_i + \sum_{j=1}^q \mu_ju_j}_{\in F} = \underbrace{-\sum_{k=1}^r\nu_jv_k}_{\in G}
			\] 
			Donc, \[
				f = \sum_{i=1}^p \lambda_i e_i + \sum_{j=1}^q \mu_j u_j \in F\cap G
			\] Comme $(e_1, \ldots, e_p)$ est une base de $F\cap G$, $\exists ! (\lambda_1', \ldots, \lambda_p') \in \mathbbm{K}^p$ tel que \[
				f = \sum_{i=1}^p \lambda'_i e_i = \sum_{i=1}^p \lambda'_i e_i + \sum_{j=1}^q 0_\mathbbm{K}u_j
			\] Comme $(e_1, \ldots, e_p, u_1, \ldots, u_q)$ est une base de $F$, \[
				\forall k \in \left\llbracket 1, q \right\rrbracket, \mu_j = 0_\mathbbm{K}
			\] De même, \[
				\forall k \in \left\llbracket 1,r \right\rrbracket , \nu_k = 0_\mathbbm{K}
			\] On remplace dans $(*)$ pour trouver \[
				\sum_{i=1}^p \lambda_ie_i = 0_E
			\] Comme $(e_1, \ldots, e_p)$ est libre, \[
				\forall i \in \left\llbracket 1,p \right\rrbracket, \lambda_i = 0_\mathbbm{K}
			\] Donc $\mathcal{B}$ est libre.\\
			Donc, 
			\begin{align*}
				\dim(F+G) &=  p +q + r \\
				&= (p+q)+ (p+r) - p \\
				&= \dim(F) + \dim(G) - \dim(F\cap G) \\
			\end{align*}
	\end{itemize}
\end{prv}

\begin{crlr}
	Avec les hypothèse précédentes, \[
		E = F \oplus G \iff \begin{cases}
			F \cap  G = \{0_E\} \\
			\dim(E) = \dim(F) + \dim(G)
		\end{cases}
	\] 
\end{crlr}

\begin{prv}
	\begin{itemize}
		\item[``$\implies$''] On suppose $E = F \oplus G$ \\
			Comme la somme est directe, $F \cap G = \{0_E\}$ 
			\begin{align*}
				\dim(E) &= \dim(F)\\
				&= \dim(F) + \dim(G) - \dim(F\cap G)\\
				&= \dim(F) + \dim(G)\\
			\end{align*}
		\item[``$\impliedby$''] On suppose $F\cap G = \{0_E\}$ et $\dim(E) = \dim(F) + \dim(G)$.\\
			On sait déjà que $F+G = F \oplus G$\\
			 \begin{align*}
				\dim(F+G) = \dim(F) + \dim(G) - \dim(F \cap G) = \dim(E)
			\end{align*}
			Donc $F + G = E$
	\end{itemize}
\end{prv}

\begin{prop}
	Soit $F$ un $\mathbbm{K}$-espace vectoriel de dimension finie $n$. Soit $\mathcal{B} = (e_1, \ldots, e_n)$ une base de $F$. L'application
	\begin{align*}
		f: \mathbbm{K}^n &\longrightarrow F \\
		(\lambda_1, \ldots, \lambda_n) &\longmapsto \sum_{i=1}^n \lambda_i e_i
	\end{align*} est bijective.\\
	Si $\mathbbm{K}$ est infini, $\mathbbm{K}^n$ aussi et donc $F$ aussi.\\
	Si $\#\mathbbm{K} = p \in \N_*$,
	\begin{align*}
		\#&\mathbbm{K}^n = p^n\\
		&\vrt=\\
		\#&F
	\end{align*}
\end{prop}


		\part{Dérivation}

\underline{Motivation}:

{
\begin{wrapfigure}{l}{3cm}
	\centering
	\begin{asy}
		import three;

		size(3cm);
		settings.render=0;
		settings.prc=false;
		currentprojection = obliqueZ;

		draw(unitbox);
		draw(shift(1.1Z + 0.05X) * (O -- X), Arrows3(TeXHead2));
		draw(shift(1.1Z + 0.05Y) * (O -- Y), Arrows3(TeXHead2));
		draw(shift(1.1X + 0.05Z) * (O -- Z), Arrows3(TeXHead2));

		label("$x$", (X/2) + (1.1Z + 0.05X), align=S);
		label("$y$", (Y/2) + (1.1Z + 0.05Y), align=W);
		label("$z$", (Z/2) + X, align=SE);
	\end{asy}
\end{wrapfigure}

\begin{align*}
	&S(x,y,z) = 2(xy + xz + yz)\\
	&V(x,y,z) = xyz
\end{align*}

On cherche à minimiser $S$ avec la contrainte $V = 1$.

Soit $f : \begin{array}{rcl}
	\left( \R_*^+ \right)^2 &\longrightarrow& \R \\
	(x,y) &\longmapsto& S\left( x,y,\frac{1}{xy} \right) = 2\left( xy + \frac{1}{y} + \frac{1}{x} \right).
\end{array}$

On cherche $(a,b) \in \left( \R^+_* \right)^2$ tel que \[
	\forall (x,y) \in (\R^+_*), f(x,y) \ge f(a,b).
\]
}

\begin{defn}
	Soit $f: U \to \R$ où $U$ est un ouvert de $\R^2$. Soit $(a,b) \in U$.
	\vspace{2mm}

	Si $\lim_{x \to a} \frac{f(x,b) - f(a,b)}{x - a} \in \R$, alors on dit que $f$ a une dérivée partielle suivant $x$ en $(a,b)$ et cette limite est notée \[
		\partial f_1(a,b) = \frac{\partial f}{\partial x}(a,b).
	\]

	Si $\lim_{y \to b} \frac{f(a,y) - f(a,b)}{y - b} \in \R$, alors on dit que $f$ a une dérivée partielle suivant $y$ et la limite est notée \[
		\partial f_2(a,b) = \frac{\partial f}{\partial y}(a,b).
	\]
\end{defn}

\begin{exm}
	\begin{enumerate}
		\item $f: (x,y) \mapsto xy + x - y$.

			\begin{align*}
				&\frac{\partial f}{\partial x} : (x,y) \mapsto y + 1,\\
				&\frac{\partial f}{\partial y} : (x,y) \mapsto x - 1.
			\end{align*}

		\item $f: (x,y) \mapsto xy + \frac{1}{y}+ \frac{1}{x}$.

			\begin{align*}
				&\frac{\partial f}{\partial x}: (x,y) \mapsto y - \frac{1}{x^2},\\
				&\frac{\partial f}{\partial y}: (x,y) \mapsto x - \frac{1}{y^2}.
			\end{align*}

		\item Trouver $f$ telle que $\begin{cases}
				(1): \qquad \frac{\partial f}{\partial x}=y,\\[2mm]
				(2): \qquad \frac{\partial f}{\partial y} = x.
			\end{cases}$

			D'après $(1)$ : \[
				\forall (x,y), \exists C(y) \in \R, f(x,y) = xy + C(y)
			\] et donc \[
				\frac{\partial f}{\partial y}(x,y) = x + C'(y)
			\] donc $C'(y) = 0$ et donc $C$ est constante.

		\item Trouver $f$ telle que $\begin{cases}
			\frac{\partial f}{\partial x} = -y,\\[2mm]
			\frac{\partial f}{ƒ\partial y} = x.
		\end{cases}$

		Ce n'est pas possible !
	\end{enumerate}
\end{exm}

\begin{defn}~\\
	\begin{minipage}{\linewidth}
		\begin{wrapfigure}{r}{4cm}
			\centering
			\vspace{-5mm}
			\begin{asy}
				import three;
				import graph3;
				size(4cm);

				settings.render = 0;
				settings.prc = false;
				currentprojection = obliqueX;

				draw(O -- X, Arrow3(TeXHead2));
				draw(O -- Y, Arrow3(TeXHead2));
				draw(O -- Z, Arrow3(TeXHead2));

				triple f(real x, real y, real z = 0) { return (x,y,cos(x - 0.5) * cos(y - 0.5)/1.2 + 0.15); }

				real inc = 1 / 5;

				for(real x = 0; x <= 1; x += inc) {
					draw(graph(
						new real(real t) { return x; }, // x
						new real(real y) { return y; }, // y
						new real(real y) { return f(x,y).z; }, // z
						0, 1
					), gray);
				}

				for(real y = 0; y <= 1; y += inc) {
					draw(graph(
						new real(real x) { return x; }, // x
						new real(real t) { return y; }, // y
						new real(real x) { return f(x,y).z; }, // z
						0, 1
					), gray);
				}

				path3 path1 = (0.8, 0.2, 0) .. (0.5, 0.5, 0) .. (0.3, 0.7, 0);
				path3 path2 = f(0.8, 0.2, 0) .. f(0.5, 0.5, 0) .. f(0.3, 0.7, 0);
				path3 d = (0.2, 0.3, 0) .. (0.3, 0.4, 0) .. (0.2, 0.7, 0) .. (0.8, 0.9, 0) .. (0.6, 0.2, 0) .. cycle;

				draw(path1, red, Arrow3(TeXHead2));
				draw(path2, red, Arrow3(TeXHead2, position=0.8));

				dot((0.5, 0.5, 0));
				dot(f(0.5, 0.5, 0));
				draw((0.5, 0.5, 0) -- f(0.5, 0.5, 0), dashed);
				draw(d);

				label("$w$", (0.3, 0.7, 0), red, align=SE);
				label("$U$", (0.8, 0.9, 0), align=SE);
			\end{asy}
		\end{wrapfigure}

		Soit $f: U \to \R$ où $U$ est un ouvert. Soit $(a,b) \in U$. Soit $w = (w_1, w_2) \in \R^2$.

		Si 
		\[
			\lim_{t\to 0} \frac{f(a + tw_1, b + tw_2) - f(a,b)}{t}
		\] existe et est réelle, alors on dit que $f$ a une dérivée dans la direction de $w$ et la limite est notée \[
			\mathrm{d}f(w)\,(a,b) = D_w(f)\,(a,b).
		\]
	\end{minipage}
\end{defn}

\begin{exm}
	\begin{align*}
		f: \left( \R_*^+ \right)^2 &\longrightarrow \R \\
		(x,y) &\longmapsto xy+\frac{1}{x}+\frac{1}{y}.
	\end{align*}

	On pose $(a,b) = (1,2)$, $w = (w_1, w_2) = (1,1)$.
	\begin{align*}
		\frac{f(1+t, 2+t) - f(1,2)}{t} &= \frac{1}{t} \left( (1+t)(2+t) + \frac{1}{1+t} + \frac{1}{2+t} - 3 - \frac{1}{2} \right) \\
		&= \frac{1}{t}\left(\cancel 2 + 3t + \po(t) + \cancel 1 - t + \po(t) + \frac{1}{2}\left( \cancel 1 - \frac{t}{2} + \po(t) \right) - \cancel3 - \cancel{\frac{1}{2}} \right) \\
		&= \frac{1}{t} \left( \frac{7}{4} t + \po(t) \right)  \\
		&= \frac{7}{4} + \po(1) \tendsto{t \to 0} \frac{7}{4}. \\
	\end{align*}

	Donc, \[
		\mathrm{d}f(1,1)\,(1,2) = \frac{7}{4}.
	\]
\end{exm}

\begin{rmk}~\\
	\begin{figure}[H]
		\centering
		\begin{asy}
			import solids;
			import graph;
			size(5cm);

			settings.render = 0;
			settings.prc = false;

			path3 par = graph(
				new real(real x) { return x; },
				new real(real x) { return 0; },
				new real(real x) { return x^2; },
				0,3);
			revolution r = revolution(par, axis=Z);

			path3 par2 = graph(
				new real(real x) { return x; },
				new real(real x) { return 0; },
				new real(real x) { return x^2; },
				-3,3);

			draw(r,1,longitudinalpen=nullpen);
			draw(r.silhouette());

			draw((-4, 0, -1) -- (-4, 0, 10) -- (4, 0, 10) -- (4, 0, -1) -- cycle, red);
			draw(par2, deepred);

			draw((4,4.5) -- (7, 4.5), black+0.5mm, Arrow(TeXHead));

			path par2d = graph(new real(real x) { return x^2; }, -3, 3);
			draw(shift((11, 0)) * par2d, deepred);

			dot(O);
			dot((11, 0));
		\end{asy}
	\end{figure}
\end{rmk}


%todo ajouter théorème-définition
\begin{thm}
	Soit $f : U \to \R$, $(a,b) \in U$. On suppose que $\frac{\partial f}{\partial x}$ et $\frac{\partial f}{\partial y}$ existent en $(a,b)$ et sont {\bfseries continues} en $(a,b)$. Alors,
	\begin{align*}
		&\forall (h, k) \in \R^2 \text{ tel que } (a +h, b + k) \in U,\\
		&f(a+ h, b + k) = f(a,b) + h \frac{\partial f}{\partial x}(a,b) + k \frac{\partial f}{\partial y}(a,b) + \po_{(h,k)\to (0,0)}\big(\|(h,k)\|\big).
	\end{align*}

	On dit que $f$ est de classe $\mathcal{C}^1$ si $\frac{\partial f}{\partial x}$ et $\frac{\partial f}{\partial y}$ existent et sont continues.

	\qed
\end{thm}

\begin{rmk}
	En physique, cette formule correspond à : \[
		\mathrm{d}f = \frac{\partial f}{\partial x}\mathrm{d}x + \frac{\partial f}{\partial y} \mathrm{d}y.
	\] En effet :
	\begin{align*}
		\mathrm{d}f &= f(x+ \mathrm{d}x, y + \mathrm{d}y) - f(x,y) \\
		&= \frac{\partial f}{\partial x} \mathrm{d}x + \frac{\partial f}{\partial y} \mathrm{d}y.
	\end{align*}
\end{rmk}

\begin{prop}
	Soit $f: U \to \R$ de classe $\mathcal{C}^1$ en $(a,b) \in U$. Alors, \[
		\forall w = (w_1, w_2) \in \R^2, \mathrm{d}f(w)\,(a,b) = w_1 \frac{\partial f}{\partial x}(a,b) + w_2 \frac{\partial f}{\partial y}(a,b).
	\]
\end{prop}

\begin{prv}
	Soit $w = (w_1, w_2) \in \R^2$. Soit $t \in \R^*$.
	\begin{align*}
		\frac{1}{t}\big(f(a + tw_1, b + tw_2) - f(a,b)\big)
		&= \frac{1}{t} \left( tw_1 \frac{\partial f}{\partial x}(a,b) + tw_2 \frac{\partial f}{\partial y}(a,b) + \po_{t \to 0}\big(\|tw\|\big) \right) \\
		&= w_1 \frac{\partial f}{\partial x}(a,b) + w_2 \frac{\partial f}{\partial y}(a,b) + \po_{t\to 0}(1) \\
		&\tendsto{t\to 0} w_1 \frac{\partial f}{\partial x}(a,b) + w_2\frac{\partial f}{\partial y}(a,b).
	\end{align*}
\end{prv}


\begin{defn}
	Avec les hypothèses précédentes, en posant \[
		\nabla f(a,b) = \left( \frac{\partial f}{\partial x}(a,b), \frac{\partial f}{\partial y}(a,b) \right) 
	\]on obtient \[
		\mathrm{d}f(w)\,(a,b) = \left<w  \mid \nabla f(a,b) \right>
	\] où $\left<\cdot|\cdot \right>$ est le produit scalaire.

	Le vecteur $\nabla f(a,b)$ est appelé \underline{gradient de $f$ en $(a,b)$}.

	Le développement limité à l'ordre 1 de $f$ devient \[
		f\big((a,b)+w\big) = f(a,b) + \left<w \mid \nabla f(a,b) \right> + \po_{w\to 0}(\|w\|)
	\]
\end{defn}

\begin{prop}
	Soit $f : U \to \R$ de classe $\mathcal{C}^1$.

	\begin{figure}[H]
    \centering
    \incfig{gradient}
	\end{figure}

	$\nabla f$ est orthogonal au lignes de niveaux de $f$, son orientation va dans le sens d'une augmentation de $f$.
\end{prop}

\begin{prv}
	Soit $\gamma : I \to U$ une courbe de niveau : \[
		\forall t \in I, f\big(\gamma(t)\big) = \text{cste}.
	\] D'après le lemme suivant : \[
		\forall t \in I, 0 = (f \circ \gamma)'(t) = \mathrm{d}f\big(\gamma'(t)\big)\big(\gamma(t)\big) = \left<\gamma'(t)  \mid \nabla f\big(\gamma(t)\big) \right>
	\] Donc $\nabla f\big(\gamma(t)\big)$ est orthogonal à $\gamma'(t)$.

	Pour tout $t \in I$, on pose $w(t) = t\, \nabla f\big(\gamma(t)\big)$. Donc \[
		f\big(\gamma(t) + w(t)\big) = f\big(\gamma(t)\big) + t \|\nabla f(\gamma(t))\|^2 + \po_{t \to 0}(t)
	\] Pour $t$ assez petit, $f\big(\gamma(t) + w(t)\big) - f\big(\gamma(t)\big)$ est du même signe que $t$.
\end{prv}

\begin{rmk}
	\begin{align*}
		V: \R^3 &\longrightarrow \R \\
		(x,y,z) &\longmapsto -mgz
	\end{align*}
	l'énerge potentielle de pesenteur

	On a donc \[
		\nabla V(x,y,z) = \left( \frac{\partial V}{\partial x}, \frac{\partial V}{\partial y}, \frac{\partial V}{\partial z} \right) = (0, 0, -mg) = \vec{P}.
	\]
\end{rmk}

\begin{lem}
	Soit $f : U \to \R$ de classe $\mathcal{C}^1$, $\gamma : \begin{array}{rcl}
		I &\longrightarrow& U \\
		t &\longmapsto& \big(x(t), y(t)\big)
	\end{array}$ où $x$ et $y$ sont dérivables.

	On pose \[
		\forall t \in I, \gamma'(t) = \big(x'(t), y'(t)\big).
	\] Alors $f \circ \gamma : I \to \R$ est dérivable et
	\begin{align*}
		\forall t \in I, (f \circ \gamma)'(t) &= \mathrm{d}f\big(\gamma'(t)\big) \big(\gamma(t)\big)\\
		&= \left<\gamma'(t)  \mid \nabla f\big(\gamma(t)\big)  \right> \\
		&= x'(t) \frac{\partial f}{\partial x}\big(x(t), y(t)\big) + y'(t) \frac{\partial f}{\partial y}\big(x(t),y(t)\big). \\
	\end{align*}
\end{lem}

\begin{prv}
	On fixe $t \in I$.

	\begin{align*}
		\forall h \neq 0, \frac{f \circ \gamma(t + h) - f \circ \gamma(t)}{h}
		&= \frac{1}{h}\big(f(\gamma(t)) + h\gamma'(t) + \po_{h\to 0}(h) - f(\gamma(t))\big) \\
		&= \frac{1}{h}\bigg(\cancel{f(\gamma(t))} + \left<h\gamma'(t) \mid \nabla f(\gamma(t)) \right> + \po_{h\to 0}(\|h\gamma'(t)\|) - \cancel{f(\gamma(t))}\bigg)\\
		&= \left<\gamma'(t) \mid \nabla f(\gamma(t)) \right> + \po_{h\to 0}(1) \\
		&\tendsto{h\to 0} \left<\gamma'(t)  \mid \nabla f(\gamma(t)) \right>
	\end{align*}
\end{prv}

\begin{defn}
	Soit $f : U \to \R$ de classe $\mathcal{C}^1$ et $(a,b) \in U$. On dit que $(a,b)$ est un \underline{point critique} de $f$ si $\nabla f(a,b) = 0$ i.e. $\frac{\partial f}{\partial x}(a,b) = \frac{\partial f}{\partial y}(a,b) = 0$.

	Dans ce cas, $f(a,b)$ est appelé \underline{valeur critique} de $f$.
\end{defn}

\begin{prop}~\\
	\begin{minipage}{\linewidth}
		\begin{wrapfigure}{r}{3cm}
			\centering
			\vspace{-1cm}
			\begin{asy}
				import solids;
				import graph;
				size(3cm);

				settings.render = 0;
				settings.prc = false;

				path3 par = graph(
					new real(real x) { return x; },
					new real(real x) { return 0; },
					new real(real x) { return -x^2; },
					0,3);
				revolution r = revolution(par, axis=Z);

				draw(r,1,longitudinalpen=nullpen);
				draw(r.silhouette());

				dot("$(a,b)$", O, red, align=N);
				real s = sqrt(2.5);
				path3 g=(s,0,-2.5)..(0,s,-2.5)..(-s,0,-2.5)..(0,-s,-2.5)..cycle;
				draw(g, deepcyan);
			\end{asy}
		\end{wrapfigure}
		Soit $f: U \to \R$ de classe $\mathcal{C}^1$ et $(a,b) \in U$ tel que \[
			\exists r > 0, \forall (x,y) \in B_{(a,b)}(r), f(x,y) \le f(a,b)
		\] Alors $\nabla f(a,b) = (0,0)$.
	\end{minipage}
\end{prop}

\begin{prv}
	Soit $g: x \mapsto f(x,b)$. $g(a)$ est un maximum local de $g$ donc $g'(a) = 0$.

	Or, $g'(a) = \frac{\partial f}{\partial x}(a,b)$

	donc $\frac{\partial f}{\partial x}(a,b) = 0$.

	Soit $h : y \mapsto f(a,y)$. On a de même $h'(b) = 0$.

	Or, $h'(b) = \frac{\partial f}{\partial y}(a,b)$.

	Donc, $\nabla f(a,b) = (0,0)$.
\end{prv}

\begin{rmk}
	Un minimum local est aussi une valeur critique.
\end{rmk}

\begin{figure}[H]
	\centering
	\begin{subfigure}{3cm}
		\centering
		\begin{asy}
			import solids;
			import graph;
			size(3cm);

			settings.render = 0;
			settings.prc = false;

			path3 par = graph(
				new real(real x) { return x; },
				new real(real x) { return 0; },
				new real(real x) { return -x^2; },
				0,3);
			revolution r = revolution(par, axis=Z);

			draw(r,1,longitudinalpen=nullpen);
			draw(r.silhouette());

			dot(O, red);
		\end{asy}
		\caption{Maximum local}
	\end{subfigure}
	\begin{subfigure}{3cm}
		\centering
		\begin{asy}
			import solids;
			import graph;
			size(3cm);

			settings.render = 0;
			settings.prc = false;

			path3 par = graph(
				new real(real x) { return x; },
				new real(real x) { return 0; },
				new real(real x) { return x^2; },
				0,3);
			revolution r = revolution(par, axis=Z);

			draw(r,1,longitudinalpen=nullpen);
			draw(r.silhouette());

			dot(O, red);
		\end{asy}
		\caption{Minimum local}
	\end{subfigure}
	\begin{subfigure}{3cm}
		\centering
		\begin{asy}
			import solids;
			import graph;
			size(3cm);

			settings.render = 0;
			settings.prc = false;
			currentprojection = obliqueZ;

			draw(graph(
				new real(real x) { return x; },
				new real(real x) { return -x^2 / 3; },
				new real(real x) { return 3; },
				-3, 3
			));

			draw(graph(
				new real(real x) { return x; },
				new real(real x) { return -x^2 / 3; },
				new real(real x) { return -3; },
				-3, 3
			));

			draw(graph(
				new real(real x) { return x; },
				new real(real x) { return -x^2 / 3 - 1; },
				new real(real x) { return 0; },
				-3, 3
			));

			draw(graph(
				new real(real x) { return 0; },
				new real(real x) { return x^2 / 9 - 1; },
				new real(real x) { return x; },
				-3, 3
			));

			draw(graph(
				new real(real x) { return -3; },
				new real(real x) { return x^2 / 9 - 4; },
				new real(real x) { return x; },
				-3, 3
			));

			draw(graph(
				new real(real x) { return 3; },
				new real(real x) { return x^2 / 9 - 4; },
				new real(real x) { return x; },
				-3, 3
			));

			dot((0,-1,0), red);
		\end{asy}
		\caption{Point de selle / Point col}
	\end{subfigure}
\end{figure}

\begin{exm}
	On revient à l'exemple donné en introduction : 
	\begin{align*}
		f: \left( \R^*_+ \right)^2 &\longrightarrow \R \\
		(x,y) &\longmapsto 2\left( xy + \frac{1}{x} + \frac{1}{y} \right).
	\end{align*}

	$\left( \R^+_* \right)^2$ est un ouvert de $\R^2$. Soit $(x,y) \in \left( \R^+_* \right)^2$.
	
	On a \[
		\begin{cases}
			\frac{\partial f}{\partial x}(x,y) = 2\left( y - \frac{1}{x^2} \right),\\
			\frac{\partial f}{\partial y}(x,y) = 2\left( x - \frac{1}{y^2} \right).
		\end{cases}
	\]

	\begin{align*}
		&\frac{\partial f}{\partial x}(x,y) = \frac{\partial f}{\partial y}(x,y) = 0\\
		\iff& \begin{cases}
			y = \frac{1}{x^2}\\
			x = \frac{1}{y^2}
		\end{cases}\\
		\iff& \begin{cases}
			y = \frac{1}{x^2}\\
			x = x^4
		\end{cases}\\
		\iff& \begin{cases}
			x = 1\\
			y = 1
		\end{cases}
	\end{align*}

	On vérivie que $f$ présente en effet un minium local en $(1,1)$. \[
		f(1,1) = 6
	\] On fixe $y \in \R^+_*$ et \[
		g : x \mapsto 2\left( xy + \frac{1}{x} + \frac{1}{y} \right).
	\] Donc \[
		\forall x \in \R^+_*, g'(x) = 2\left( y - \frac{1}{x^2} \right).
	\]
	\begin{center}
		\begin{tikzpicture}
			\tkzTabInit{$x$/1,$g'(x)$/1,$g$/2.3}{$0$, $\frac{1}{\sqrt{y}}$, $+\infty$}
			\tkzTabLine{,-,z,+,}
			\tkzTabVar{+/{}, -/$2\left( 2\sqrt{y} +\frac{1}{y} \right)$, +/{}}
		\end{tikzpicture}
	\end{center}
	
	Ainsi, \[
		\forall x \in \R^+_*, \forall y \in \R^+_*, f(x,y) \ge 2\left( 2\sqrt{y} + \frac{1}{y} \right)
	\] Soit $h : y \mapsto 2\sqrt{y} + \frac{1}{y}$. On a \[
		\forall y > 0, h'(y) = \frac{1}{\sqrt{y}} - \frac{1}{y^2} = \frac{y\sqrt{y} - 1}{y^2} = \frac{y^{\frac{3}{2}} - 1}{y^2}
	\]

	\begin{center}
		\begin{tikzpicture}
			\tkzTabInit{$y$/0.7,$h'(y)$/0.7,$h$/1.4}{$0$, $1$, $+\infty$}
			\tkzTabLine{,-,z,+,}
			\tkzTabVar{+/{}, -/$3$, +/{}}
		\end{tikzpicture}
	\end{center}

	Donc, \[
		\forall x,y > 0, f(x,y) \ge 2\times 3 = 6 = f(1,1).
	\]
\end{exm}

\begin{prop}
	[règle de la chaîne]

	Soit $f : \begin{array}{rcl}
		U &\longrightarrow& \R^2 \\
		(x,y) &\longmapsto& f(x,y)
	\end{array}$ de classe $\mathcal{C}^1$ et $U, V$ deux ouverts de $\R^2$.

	Soit $\varphi : \begin{array}{rcl}
		V &\longrightarrow& U \\
		(u,v) &\longmapsto& \varphi(u,v) = \big(x(u,v), y(u,v)\big)
	\end{array}$.

	On suppose que $x$ et $y$ sont de classe $\mathcal{C}^1$ sur $V$.

	Alors,  $f \circ \varphi : \begin{array}{rcl}
		V &\longrightarrow& \R \\
		(u,v) &\longmapsto& f\big(\varphi(u,v)\big)
	\end{array}$ est de classe $\mathcal{C}^1$ et
	\begin{align*}
		\forall (u_0, v_0) \in V, \frac{\partial (f \circ \varphi)}{\partial u}(u_0, v_0)
		&= \frac{\partial f}{\partial x}\big(\varphi(u_0, v_0)\big) \times \frac{\partial x}{\partial u}(u_0, v_0)\\
		&+ \frac{\partial f}{\partial y}\big(\varphi(u_0,v_0)\big) \frac{\partial y}{\partial u}(u_0,v_0)
	\end{align*}
	\begin{align*}
		\forall (u_0, v_0) \in V, \frac{\partial (f \circ \varphi)}{\partial v}(u_0, v_0)
		&= \frac{\partial f}{\partial x}\big(\varphi(u_0, v_0)\big) \times \frac{\partial x}{\partial v}(u_0, v_0)\\
		&+ \frac{\partial f}{\partial y}\big(\varphi(u_0,v_0)\big) \frac{\partial y}{\partial v}(u_0,v_0)
	\end{align*}
\end{prop}

\begin{exm}
	[changement de coordonnées polaires]
	On pose \begin{align*}
		\varphi: \R^+_* \times ]0,2\pi[ &\longrightarrow \R^2\setminus \left( R^+_* \times \{0\} \right) \\
		(r, \theta) &\longmapsto (r \cos \theta, r \sin\theta),
	\end{align*}
	\begin{align*}
		f: \R^2\setminus \left( R^+_* \times \{0\} \right) &\longrightarrow \R \\
		(x,y) &\longmapsto f(x,y),
	\end{align*}
	\begin{align*}
		g: \overbrace{\R^+_* \times ]0, 2\pi[}^{=V} &\longrightarrow \R \\
		(r, \theta) &\longmapsto f(r\cos\theta, r\sin\theta).
	\end{align*}

	\begin{align*}
		\forall (r_0,\theta_0) \in V,&\\[5mm]
		\frac{\partial g}{\partial r}(r_0, \theta_0) &= \frac{\partial f}{\partial x}(r_0\cos\theta_0, r_0\sin\theta_0)\cos\theta_0\\
		&+ \frac{\partial f}{\partial y}(r_0 \cos\theta_0, r_0\sin\theta_0)\sin\theta_0\\
		&= 2r_0\cos^2\theta_0 + 2r_0\sin^2(\theta_0) \\
		&= 2r_0 \\[5mm]
		\frac{\partial g}{\partial \theta}(r_0, \theta_0) &= \frac{\partial f}{\partial x}(r_0\cos\theta_0, r_0\sin\theta_0)r_0\sin\theta_0\\
		&+ \frac{\partial f}{\partial y}(r_0 \cos\theta_0, r_0\sin\theta_0)r_0\cos\theta_0\\
		&= -2{r_0}^2\cos(\theta_0)\sin(\theta_0) + 2{r_0}^2 \sin(\theta_0)\cos(\theta_0)\\
		&= 0 \\
	\end{align*}

	Donc, \[
		g(r, \theta) = r^2.
	\]
\end{exm}

\begin{exm}
	Résoudre \[
		\begin{cases}
			\frac{\partial f}{\partial x} = \frac{x}{x^2+y^2},\\
			\frac{\partial f}{\partial y} = \frac{y}{x^2+y^2}.\\
		\end{cases}
	\]

	On pose $g: (r, \theta) \mapsto f(r \cos\theta, r \sin\theta)$.

	\begin{align*}
		&\frac{\partial g}{\partial r} = \frac{1}{r}\cos^2\theta + \frac{1}{r}\sin^2\theta = \frac{1}{r},\\
		&\frac{\partial g}{\partial \theta} = -\cos(\theta) \sin(\theta) + \sin(\theta)\cos(\theta) = 0.
	\end{align*}

	Donc, \[
		\exists C \in \R, g: (r, \theta) \mapsto \ln r + C
	\] d'où,
	\begin{align*}
		\forall (x,y) \in \R^2 \setminus \{(0,0)\}, f(x,y) &= \ln\left(\sqrt{x^2 + y^2} \right)  + C\\
		&= \frac{1}{2}\ln(x^2 + y^2) + C. \\
	\end{align*}
\end{exm}

\begin{rmk}
	Soit $\mathcal{B} = (e_1, e_2)$ la base canonique de $\R^2$, $f: U \to \R$ de classe $\mathcal{C}^1$ avec $U$ un ouvert de $\R^2$.

	Soit $(x,y) \in U$.

	\begin{align*}
		\Mat_{\mathcal{B}}\big(\nabla f(x,y)\big) = \begin{pmatrix}
			\frac{\partial f}{\partial x}(x,y)\\[2mm]
			\frac{\partial f}{\partial y}(x,y)
		\end{pmatrix}
	\end{align*}

	Soit  \begin{align*}
		\varphi: V &\longrightarrow U \\
		(u,v) &\longmapsto \big(x(u,v), y(u,v)\big) 
	\end{align*} avec $x,y$ de classe $\mathcal{C}^1$. Soit $g = f \circ \varphi$.
	\begin{align*}
		\Mat_{\mathcal{B}}\big(\nabla g(u,v)\big)
		&= \begin{pmatrix}
			\frac{\partial g}{\partial u}(u,v) \\[2mm]
			\frac{\partial g}{\partial v}(u,v)
		\end{pmatrix} \\
		&= \begin{pmatrix}
			\frac{\partial x}{\partial u}(u,v) \frac{\partial f}{\partial x}(x,y)
			+ \frac{\partial y}{\partial u}(u,v)\frac{\partial f}{\partial y}(x,y)\\[3mm]
			\frac{\partial x}{\partial v}(u,v) \frac{\partial f}{\partial x}(x,y)
			+ \frac{\partial y}{\partial v}(u,v) \frac{\partial f}{\partial y}(x,y)
		\end{pmatrix}  \\
		&= \underbrace{\begin{pmatrix}
				\frac{\partial x}{\partial u}(u,v)& \frac{\partial y}{\partial u}(u,v)\\[3mm]
				\frac{\partial x}{\partial v}(u,v)& \frac{\partial y}{\partial v}(u,v)
		\end{pmatrix}}_{J(u,v)} \begin{pmatrix}
			\frac{\partial f}{\partial x}(x,y)\\[3mm]
			\frac{\partial f}{\partial y}(x,y)
		\end{pmatrix} \\
		&= J(u,v) \Mat_{\mathcal{B}}\big(\nabla f(x,y)\big) \\
	\end{align*}
	où $J(u,v) = 
	\begin{pNiceArray}{c:c}
		\Mat_{\mathcal{B}}\big(\nabla x(u,v)\big) & \Mat_{\mathcal{B}}\big(\nabla y(u,v)\big)
	\end{pNiceArray}$.

	On dit que $J(u,v)$ est \underline{la jacobienne} de $\varphi$ en $(u,v)$.
	L'application linéaire canoniquement associée à $J(u,v)$ est la \underline{différentielle de $\varphi$} en $(u,v)$ noté $\mathrm{d}\varphi(u,v)$.

	On a $\mathrm{d}\varphi(u,v) \in \mathcal{L}(R^2)$ et $\Mat_{\mathcal{B}}\big(\mathrm{d}\varphi(u,v)\big) = J(u,v)$.

	Par exemple, la jacobienne du changement de coordonnées polaires est \[
		J = \begin{pmatrix}
			\frac{\partial x}{\partial r} & \frac{\partial y}{\partial r}\\[3mm]
			\frac{\partial x}{\partial \theta} & \frac{\partial y}{\partial \theta}
		\end{pmatrix}
		= \begin{pmatrix}
			\cos\theta&\sin\theta\\
			-r\sin\theta&r\cos\theta
		\end{pmatrix}.
	\]
	$\underbrace{\det(J)}_{\text{le jacobien}} = r\cos^2\theta + r\sin^2\theta = r$

	Dans une intégrale double, si $(x,y) = \varphi(u,v)$, alors $\mathrm{d}x\mathrm{d}y = \det(J)\mathrm{d}u\mathrm{d}v$.

	Ici, \[
		\mathrm{d}x\ \mathrm{d}y = r\ \mathrm{d}r\ \mathrm{d}\theta.
	\]
\end{rmk}

\begin{prv}
	On pose $(x_0, y_0) = \varphi(u_0, v_0)$. Pour tout $(h,k) \in \R^2$ tels que $(u_0 + h, v_0 + k) \in V$, en posant $g = f  \circ \varphi$.

	\begin{align*}
		g(u_0 + h, v_0 + h) &= f\big(x(u_0 + h, v_0 + k), y(u_0 + h, v_0 + k)\big) \\
		&= f\left(
			x(u_0,v_0) + h \frac{\partial x}{\partial u}(u_0,v_0) + k \frac{\partial x}{\partial v}(u_0, v_0) + \po\big(\|(h,k)\|\big), \right.\\
		&\phantom{ = f\bigg(\bigg.}\left. y(u_0, v_0) + h \frac{\partial y}{\partial u}(u_0, v_0) + k \frac{\partial y}{\partial v}(u_0, v_0) + \po\big(\|(h,k)\|\big)
		\right)  \\
		&= f(x_0,y_0) \\
		&~+ \left( h \frac{\partial x}{\partial u}(u_0,v_0) + k \frac{\partial x}{\partial v}(u_0, v_0) + \po(\|(h,k)\|) \right) \frac{\partial f}{\partial x}(x_0,y_0)\\
		&~+ \left( h \frac{\partial y}{\partial u}(u_0, v_0) + k\frac{\partial y}{\partial v}(u_0, v_0) + \po(\|(h,k)\|) \right) \frac{\partial f}{\partial y}(x_0, y_0)\\
		&~+ \po(\|(h,k)\|)\\
		&= f(x_0, y_0) \\
		&~+ h \left( \frac{\partial x}{\partial u}(u_0, v_0) \frac{\partial f}{\partial x}(x_0, y_0) + \frac{\partial y}{\partial u}(u_0, v_0) \frac{\partial f}{\partial y}(x_0, y_0) \right)  \\
		&~+ k\left( \frac{\partial x}{\partial v}(u_0, v_0) \frac{\partial f}{\partial x}(x_0, y_0) + \frac{\partial y}{\partial v}(u_0, v_0) \frac{\partial f}{\partial y}(x_0, y_0) \right) 
		&~+ \po(\|(h,k)\|)\\
		&= g(u_0, v_0) + h \frac{\partial g}{\partial u}(u_0, v_0) + k \frac{\partial g}{\partial v}(u_0, v_0) + \po(\|(h,k)\|) \\
	\end{align*}

	Par identification,
	\[
		\frac{\partial g}{\partial u}(u_0, v_0) = \frac{\partial x}{\partial u}(u_0, v_0) \frac{\partial f}{\partial x}(x_0, y_0) + \frac{\partial y}{\partial u}(u_0, v_0) \frac{\partial f}{\partial y}(x_0,y_0)
	\] et \[
		\frac{\partial g}{\partial v}(u_0, v_0) = \frac{\partial x}{\partial v}(u_0,v_0) \frac{\partial f}{\partial x}(x_0, y_0) + \frac{\partial y}{\partial v}(u_0, v_0) \frac{\partial f}{\partial y}(x_0, y_0).
	\] 
\end{prv}

\begin{exm}
	[Régression linéaire]~\\
	\begin{figure}[H]
		\centering
		\begin{asy}
			import graph;
			axes(EndArrow);
			size(5cm);

			real f(real x) { return x + 0.5; }

			real k = 35 / (7 - 0.5);

			for(int i = 0; i < 35; ++i) {
				real mag = exp(sin(100 * pi/exp(1) * i)) * 0.8 + exp(cos(i*40)/3);
				real eps = mag * cos(10 * exp(1)/pi * i) / 3;
				dot((i/k,f(i/k) + eps));
			}

			draw(graph(f, -1, 7), orange);
		\end{asy}
	\end{figure}
	\[
		y = a x + b
	\] 
	On fixe $(a,b) \in \R^2$. \[
		\varepsilon(a,b) = \sum_{i=1}^n\big( y_i - (ax_i + b) \big)^2
	\] l'erreur totale.

	On veut minimiser $\varepsilon(a,b)$. On a 
	\[
		\forall (a,b) \in \R^2,
		\begin{cases}
			\frac{\partial \varepsilon}{\partial a}(a,b) = -2\sum_{i=1}^{n}(y_i - ax_i - b)x_i,\\
			\frac{\partial \varepsilon}{\partial b}(a,b) = -2\sum_{i=1}^{n}(y_i - ax_i - b).
		\end{cases}
	\]

	Donc,
	\begin{align*}
		(a,b) \text{ point critique de } \varepsilon \iff& \begin{cases}
			a \sum_{i=1}^n {x_i}^2 + b\sum_{i=1}^{n}x_i = \sum_{i=1}^{n} y_ix_i\\
			a\sum_{i=1}^{n}x_i + nb = \sum_{i=1}^ny_i
		\end{cases}\\
		\iff& \begin{cases}
			a \left( \frac{1}{n}\sum_{i=1}^n {x_i}^2 - \overline{x}^2\right) = \overline{y} - \overline{x} \overline{y}\\
			b = \frac{1}{n}\sum_{i=1}^ny_i - \frac{a}{n}\sum_{i=1}^nx_i = \frac{1}{n}\sum_{i=1}^n x_i y_i - \overline{x} \overline{y}
		\end{cases}\\
		&\text{ où } \overline{x} = \frac{1}{n} \sum_{i=1}^n x_i,~\overline{y} = \frac{1}{n}\sum_{i=1}^n y_i\\
		\iff& \begin{cases}
			a = \frac{\Cov(x,y)}{V(x)}\\
			b = \overline{y} - a\overline{x}
		\end{cases}
	\end{align*}

	Coefficient de corrélation: $\frac{\Cov(x,y)}{\sigma_x \sigma_y} \in [-1, 1]$
\end{exm}












	}

	{
		\chap[23]{Dénombrement}
		\renewcommand{\cwd}{../chap23}
		\begin{defn}
	Soit $E$ un $\mathbbm{K}$-espace vectoriel. On dit que $E$ est de \underline{dimension finie} si $E$ a au moins une famille génératrice finie. On dit que $E$ est de \underline{dimension infinie} sinon.
	\index{dimension finie (espace vectoriel)}
	\index{dimension infinie (espace vectoriel)}
\end{defn}

\begin{thm}
	[Théorème de la base extraite]
	Soit $E$ un $\mathbbm{K}$-espace vectoriel non nul de dimension finie. Soit $\mathcal{G}$ une famille génératrice finie de $E$. Alors, il existe une base $\mathcal{B}$ de $\mathcal{E}$ telle que $\mathcal{B} \subset \mathcal{G}$.
\end{thm}

\begin{prv}
	[par récurrence sur $\#G = \Card(G)$]
	\begin{itemize}
		\item Soit $E$ un $\mathbbm{K}$-espace vectoriel non nul engendré par $\mathcal{G} = (u)$.\\
			Si $u = 0_E$, alors $E = \{0_E\}$: une contradiction $\lightning$ \\
			Donc $u \neq 0_E$ donc $(u)$ est libre. En effet, \[
				\forall \lambda \in \mathbbm{K}, \lambda u = 0_E \implies \lambda = 0_\mathbbm{K}
			\] Donc $\mathcal{G}$ est une base de $E$.\\
		\item Soit $n \in \N_*$. Soit $E$ un $\mathbbm{K}$-espace vectoriel. On suppose que si $E$ a une famille génératrice constituée de $n$ vecteurs, alors on peut extraire de cette famille une base de $E$.\\
			Soit $\mathcal{G}$ une famille génératrice de $E$ avec $n+1$ vecteurs.\\
			Si $\mathcal{G}$ est libre, alors $\mathcal{G}$ est une base de $E$. \\
			Si $\mathcal{G}$ n'est pas libre, alors il existe $u \in \mathcal{G}$ tel que $u \in \Vect(\mathcal{G}\setminus \{u\})$ \\
			Donc $\mathcal{G}\setminus \{u\}$ engendre $E$. Or, $\mathcal{G}\setminus \{u\}$ possède $n$ vecteurs. D'après l'hypothèse de récurrence, il existe une base $\mathcal{B}$ de $E$ telle que \[
				\mathcal{B} \subset \mathcal{G} \setminus \{u\} \subset \mathcal{G}
			\] 
	\end{itemize}
\end{prv}

\begin{crlr}
	Tout espace de dimension finie a une base.
	\qed
\end{crlr}

\begin{thm}
	[Théorème de la base incomplète]
	Soit $E$ un $\mathbbm{K}$-espace vectoriel de dimension finie, $\mathcal{G}$ une famille génératrice finie de $E$. $\mathcal{L}$ une famille libre de $E$. Alors, il existe une base $\mathcal{B}$ de $E$ telle que \[
		\mathcal{L} \subset \mathcal{B} \text{ et } \mathcal{B}\setminus \mathcal{L} \subset \mathcal{G}
	\] 
\end{thm}

\begin{prv}
	[par récurrence sur $\#(\mathcal{G}\setminus\mathcal{L})$]
	\begin{itemize}
		\item Avec les notations précédentes, on suppose que $\mathcal{G}\setminus\mathcal{L} \neq \O$ \[
				\forall u \in \mathcal{G}, u \in \mathcal{L}
			\] Donc $\mathcal{G} \subset \mathcal{L}$ donc $\mathcal{L}$ est génératrice donc $\mathcal{L}$ est une base de $E$. On pose $\mathcal{B} = \mathcal{L}$ et alors \[
				\mathcal{L} \subset  \mathcal{B} \text{ et } \mathcal{B}\setminus\mathcal{L} = \O \subset  \mathcal{G}
			\] 
		\item Soit $n \in \N$. On suppose que si $\mathcal{G}$ est génératrice et $\mathcal{L}$ libre avec $\#(\mathcal{G}\setminus\mathcal{L}) = n$ alors il existe une base $\mathcal{B}$ de $E$ telle que \[
			\mathcal{L}\subset \mathcal{B} \text{ et } \mathcal{B}\setminus\mathcal{L}\subset \mathcal{G}
		\] Soient à présent $\mathcal{G}$ une famille génératrice de $E$ et $\mathcal{L}$ une famille libre de $E$ telles que $\#(\mathcal{G}\setminus\mathcal{L}) = n+1 > 0$\\
		Si $\mathcal{L}$ engendre $E$, alors $\mathcal{L}$ est une base de $E$. On pose $\mathcal{B} = \mathcal{L}$ et on a bien \[
			\mathcal{L} \subset  \mathcal{B} \text{ et } \mathcal{B} \setminus \mathcal{L} = \O \subset  \mathcal{G}
		\] On suppose que $\mathcal{L}$ n'engendre pas $E$. Il existe $u \in \mathcal{G}$ tel que $u \not\in \Vec(\mathcal{L})$ (car sinon, $\mathcal{G} \subset \Vect(\mathcal{L})$ et donc $\underbrace{\Vect(\mathcal{G})}_{= E} \subset  \underbrace{\Vect(\mathcal{L})}_{ \subset E}$\\
		Donc $\mathcal{L} \cup \{u\} $ est libre. On pose $\mathcal{L}' = \mathcal{L} \cup \{u\} $ \[
			\mathcal{G}\setminus \mathcal{L}' = \mathcal{G}\setminus (\mathcal{L} \cup \{u\}) = (\mathcal{G}\setminus\mathcal{L})\setminus \{u\} 
		\] donc $\#(\mathcal{G}\setminus\mathcal{L}') = n+1 -1 = n$\\
		D'après l'hypothèse de récurrence, il existe $\mathcal{B}$ une base de $E$ telle que \[
			\mathcal{L} \subset  \mathcal{L}' \subset \mathcal{B} \text{ et } \mathcal{B}\setminus \mathcal{L}' \subset \mathcal{G}
		\] \[
			\mathcal{B} \setminus \mathcal{L} = \underbrace{\mathcal{B}\setminus\mathcal{L}'}_{\subset \mathcal{G}} \cup \underbrace{\{u\}}_{\subset \mathcal{G} \text{ car } u \in \mathcal{G}}
		\] On a $\mathcal{B}\setminus\mathcal{L}\subset \mathcal{G}$
	\end{itemize}
\end{prv}

\begin{thm}
	Soit $E$ un $\mathbbm{K}$-espace vectoriel de dimension finie. Toutes les bases de $E$ ont le même cardinal.
\end{thm}

\begin{prv}
	Soit $\mathcal{G}$ une famille génératrice finie de $E$ et $\mathcal{B} \subset  \mathcal{G}$ une base de $E$. On note $n = \#\mathcal{B}$ \\
	Soit $\mathcal{B}'$ une base de $E$. On pose $p = n - \#(\mathcal{B} \cap  \mathcal{B}')$. Montrons par récurrence sur  $p$ que $\#\mathcal{B} = \#\mathcal{B}'$ 
	\begin{itemize}
		\item On suppose que $p = 0$. Alors, $\#(\mathcal{B} \cap \mathcal{B}') = n$ \\
			Or, $\mathcal{B}' \cap \mathcal{B} \subset \mathcal{B}$ donc $\mathcal{B} \cap \mathcal{B}' = \mathcal{B}$ donc $\mathcal{B} \subset  \mathcal{B}'$ et donc $\mathcal{B} = \mathcal{B}'$ 
		\item Soit $p \in \N$. On suppose que si $\mathcal{B}'$ est une base de $E$ telle que $n - \#(\mathcal{B} \cap \mathcal{B}') = p$, alors $\#\mathcal{B}' = n$ \\
			Aoit $\mathcal{B}'$ une base de $E$ telle que $n - \#(\mathcal{B}\cap \mathcal{B}') = p+1 > 0$ \\
			Donc $\mathcal{B} \cap \mathcal{B}' \neq \mathcal{B}$. Soit $u \in \mathcal{B}' \setminus \mathcal{B}$. D'après le lemme d'échange, il existe $v \in \mathcal{B}\setminus \mathcal{B}'$ tel que $\mathcal{B}' \setminus \{u\} \cup \{v\}$ est une base de $E$. On pose $\mathcal{B}'' = \mathcal{B}' \setminus \{u\} \cup \{v\}$ 
			\begin{align*}
				\mathcal{B}'' \cap \mathcal{B} &= \left( (\mathcal{B}' \setminus \{u\})  \cap \mathcal{B} \right) \cup \{v\} \\
				&= (\mathcal{B}' \cap \mathcal{B}) \cup \{v\} \\
			\end{align*}
			donc,
			\begin{align*}
				n - \#(\mathcal{B}'' \cap \mathcal{B}) &= n - (\#(\mathcal{B}' \cap \mathcal{B}) + 1) \\
				&= p+1- 1 \\
				&= p \\
			\end{align*}
			D'après l'hypothèse de récurrence, \[
				\#\mathcal{B}'' = n
			\] Or, $\#\mathcal{B}'' = \#\mathcal{B}'$
	\end{itemize}
\end{prv}

\begin{lem}
	Soient $\mathcal{B}$ et $\mathcal{B}'$ deux bases de $E$ telles que $\mathcal{B}\subset \mathcal{B}'$. Alors, $\mathcal{B} = \mathcal{B}'$.
\end{lem}

\begin{prv}
	On suppose $\mathcal{B}' \neq \mathcal{B}$. Soit $u \in \mathcal{B}' \setminus \mathcal{B}$
	$u \in E = \Vect(\mathcal{B})$ donc $\mathcal{B} \cup \{u\}$ n'est pas libre.
	Donc $\mathcal{B}\cup \{u\} \subset \mathcal{B}'$ et $\mathcal{B}'$ est libre donc $\mathcal{B}\cup \{u\}$ est libre: une contradiction $\lightning$
\end{prv}

\begin{lem}
	[Lemme d'échange] Soient $\mathcal{B}_1$ et $\mathcal{B}_2$ deux bases de $E$ et $u \in \mathcal{B}_1 \setminus \mathcal{B}_2$. Alors, il existe $v \in \mathcal{B}_2$ tel que $(\mathcal{B}_1 \setminus \{u\}) \cup \{v\}$ soit une base de $E$.
\end{lem}

\begin{prv}
	[1${}^\text{nde}$ méthode]
	On suppose que pout tout $v \in \mathcal{B}_2$, $(\mathcal{B}_1\setminus \{u\}) \cup \{v\}$ n'est pas une base de $E$
	Soit $v \in \mathcal{B}_2$.
	\begin{itemize}
		\item Supposons $(\mathcal{B}_1\setminus \{u\})\cup \{v\}$ non libre. $\mathcal{B}_1 \setminus \{u\}$ est libre. Donc $v \in \Vect(\mathcal{B}_1 \setminus \{u\})$
		\item Supposons $(\mathcal{B}_1\setminus \{u\}) \cup \{v\}$ non génératrice.
			Comme $\mathcal{B}_1$ engendre $E$, $u \not\in \Vect(\mathcal{B}_1\setminus \{v\})$.
			On suppose que $\mathcal{B}_1 \neq \mathcal{B}_2$.
			$\forall v \in \mathcal{B}_2 \setminus \mathcal{B}_1, \Vect(\mathcal{B}_1 \setminus \{v\}) = \Vect(\mathcal{B}_1) = E \ni u$ 
			donc, $(\mathcal{B}_1\setminus \{u\}) \cup \{v\}$ engendre $E$ et donc \[
				v \in \Vect(\mathcal{B}_1 \setminus \{u\})
			\] On a aussi \[
				\forall v \in \mathcal{B}_1 \setminus \{u\}, v \in \Vect(\mathcal{B}_1\setminus \{u\})
			\] Comme $u \not\in \mathcal{B}_2$, on a \[
				\forall v \in \mathcal{B}_2, v \in \Vect(\mathcal{B}_1\setminus \{u\})
			\] docn \[
				E = \Vect(\mathcal{B}_2) \subset \Vect(\mathcal{B}_1\setminus \{u\})
			\] donc $\mathcal{B}_1\setminus \{u\}$ engendre $E$ donc $\mathcal{B}_1\setminus \{u\}$ est une base de $E$. Or, $\mathcal{B}_1 \setminus \{u\}  \subset  \mathcal{B}_1$, donc $\mathcal{B}_1\setminus \{u\} = \mathcal{B}_1$
	\end{itemize}
\end{prv}

\begin{prv}
	[2${}^\text{nde}$ méthode]
	On suppose que pout tout $v \in \mathcal{B}_2$, $(\mathcal{B}_1\setminus \{u\}) \cup \{v\}$ n'est pas une base de $E$
	\begin{itemize}
		\item Comme $u \in \mathcal{B}_1 \setminus \mathcal{B}_2$, nécéssairement $\mathcal{B}_1 \neq \mathcal{B}_2$ donc $\mathcal{B}_2 \not\subset \mathcal{B}_1$, donc $\mathcal{B}_2\setminus\mathcal{B}_1 \neq \O$ 
		\item Soit $v \in \mathcal{B}_2\setminus\mathcal{B}_1$. Il existe $(\lambda_w)_{w\in\mathcal{B}_1}$ une famille de scalaires presque nulle telle que \[
				v = \sum_{w \in \mathcal{B}_1} \lambda_w w - \lambda_u u + + \sum_{w \in \mathcal{B}_1\setminus \{u\}}\lambda_w w
			\]
			Si $\lambda_u \neq 0_E$, alors
			\begin{align*}
				u &= \lambda_u^{-1}\left( v - \sum_{w \in \mathcal{B}_1 \setminus \{u\}} \lambda_w w \right)\\
					&\in \Vect(\mathcal{B}_1\setminus \{u\} \cup v)
			\end{align*}
			 donc $\mathcal{B}_1 \subset \Vect(\mathcal{B}_1\setminus \{u\} \cup \{v\})$\\
			 et donc $E \subset  \Vect(\mathcal{B}_1 \setminus \{u\} \cup \{v\})$ \\
			 et donc $\mathcal{B}_1 \setminus \{u\} \cup \{v\}$ engendre $E$ \\
			 donc $\mathcal{B}_1 \setminus \{u\} \cup \{v\}$ n'est pas libre\\
			 donc $v \in \Vect(\mathcal{B}_1\setminus \{u\})$ (car $\mathcal{B}_1 \setminus \{u\}$ est libre\\
			 donc $\lambda_u = 0_\mathbbm{K}$ $\lightning$\\`

			 Donc, $\lambda_u = 0_\mathbbm{K}$, docn $v \in \Vect(\mathcal{B}_1\setminus \{u\})$ \\
			 On vient de prouver que
			 \begin{align*}
			 	\mathcal{B}_2 \setminus \mathcal{B}_1 \subset \Vect(\mathcal{B}_1 \setminus \{u\})\\
			 	\mathcal{B}_1 \setminus \{u\} \subset \Vect(\mathcal{B}_1 \setminus \{u\})\\
			 \end{align*}
			 Comme $u \not\in \mathcal{B}_2$, \[
			 	\mathcal{B}_2 \subset \Vect(\mathcal{B}_1 \setminus \{u\})
			 \] donc \[
			 	E = \Vect(\mathcal{B}_2) \subset  \Vect(\mathcal{B}_1 \setminus \{u\})
			 \] donc $\mathcal{B}_1 \setminus \{u\}$ engendre $E$. Donc,  $\mathcal{B}_1 \setminus \{u\}$ est une base de $E$.\\
			 Or, $\mathcal{B}_1 \setminus \{u\} \subset  \mathcal{B}_1$, donc $\mathcal{B}_1 \setminus \{u\} = \mathcal{B}_1$
	\end{itemize}
\end{prv}

\begin{defn}
	Soit $E$ un $\mathbbm{K}$-espace vectoriel de dimension finie. Le cardinal commun à toutes les bases de $E$ est appelé \underline{dimension} de $E$ est notée $\dim(E)$ ou $\dim_\mathbbm{K}(E)$\\
	C'est donc aussi le nombre de coordonnées de n'importe quel vecteur dans n'importe quelle base.
	\index{dimension (espace vectoriel)}
\end{defn}

\begin{exm}
	\begin{enumerate}
		\item $\dim_\R(\C) = 2$ et $\dim_\C(\C) = 1$ 
		\item $\dim_\mathbbm{K}(\mathbbm{K}^{n}) = n$ 
		\item $\dim_{\mathbbm{K}}(\mathcal{M}_{n,p}(\mathbbm{K})) = np$
	\end{enumerate}
\end{exm}

\begin{crlr}
	Soit $E$ un $\mathbbm{K}$-espace vectoriel de dimension finie, $\mathcal{L}$ une famille libre de $E$, $\mathcal{G}$ une famille génératrice de $E$. On note $n = \dim(E)$
	\begin{enumerate}
		\item $\#\mathcal{G} \ge n$ et $(\#\mathcal{G} = n \implies \mathcal{G} \text{ est une base de } E$)
		\item $\#\mathcal{L} \le n$ et $(\#\mathcal{L} = n \implies \mathcal{L} \text{ est une base de } E$)
	\end{enumerate}
\end{crlr}

\begin{crlr}
	$\R^{\R}$ est de dimension infinie.
	$\forall i \in \N, e_i: x \mapsto x^i$\\
	$(e_i)_{i\in\N}$ est libre dans $\R^\R$
\end{crlr}

\begin{prop}
	Soient $E$ et $F$ deux $\mathbbm{K}$-espaces vectoriels de dimension finie. Alors $E\times F$ est de dimension finie et $\dim(E\times F) = \dim(E) + \dim(F)$
\end{prop}

\begin{prv}
	Soit $(e_1,\ldots, e_n)$ une base de $E$, $(f_1, \ldots, f_p)$ une base de $F$.
	On pose \[
		\left\{\begin{array}
			{r c l}
			u_1 &=& (e_1,0_F)\\
			u_2 &=& (e_2,0_F)\\
					&\vdots&\\
			u_n &=& (e_n,0_F)\\
			u_{n+1} &=& (0_E, f_1)\\
			u_{n+2} &=& (0_E, f_2)\\
					&\vdots&\\
			u_{n+p} &=& (0_E,f_p)\\
		\end{array}\right.
	\]
	Soit $(x,y) \in E\times F$. \[
		\begin{cases}
			\exists (x_1,\ldots,x_n)\in \mathbbm{K}^n, x = \sum_{i=1}^{n} x_ie_i
			\exists (y_1,\ldots,y_n)\in \mathbbm{K}^n, x = \sum_{j=1}^{p} y_jf_j
		\end{cases}
	\] 
	\begin{align*}
		(x,y) &= \left( \sum_{i=1}^{n} x_ie_i, \sum_{i=1}^{p} y_jf_j \right)  \\
		&= \sum_{i=1}^{n} x_i (e_i + 0_F) + \sum_{j=1}^{p} y_j (0_E, f_j) \\
		&= \sum_{i=1}^{n} x_i u_i + \sum_{j=1}^{p} y_j u_{n+j} \\
	\end{align*}
	Donc, $E\times F = \Vect(u_1, \ldots, u_{n+p})$ donc $E\times F$ est de dimension finie.\\
	Soit $(\lambda_1, \ldots, \lambda_{n+p}) \in \mathbbm{K}^{n+p}$ tel que \[
		(*): \quad \sum_{k=1}^{n+p} \lambda_ku_k = 0_{E\times F} = (0_E, 0_F)
	\]
	\begin{align*}
		(*) &\iff \sum_{k=1}^{n} \lambda_k (e_k, 0_F) + \sum_{k=n+1}^{p} \lambda_k(0_E, f_{k-n}) = (0_E, 0_F)\\
				&\iff \begin{cases}
					\sum_{k=1}^{n} \lambda_k e_k = 0_E\\
					\sum_{k=n+1}^{p} \lambda_k f_{k-n} = 0_F
				\end{cases}\\
				&\iff \begin{cases}
					\forall k \in \left\llbracket 1,n \right\rrbracket, \lambda_k = 0_\mathbbm{K} \qquad&(\text{car $(e_1,\ldots,e_n)$ est libre})\\
					\forall k \in \left\llbracket n+1,n+p \right\rrbracket, \lambda_k = 0_\mathbbm{K} \qquad&(\text{car $(f_1,\ldots,f_n)$ est libre})\\
				\end{cases}
	\end{align*}
	Donc $(u_1, \ldots, u_{n+p})$ est une base de $E\times F$. Donc, $\dim(E\times F) = n + p = \dim(E) + \dim(F)$
\end{prv}

\begin{rmk}
	[Convention]
	\[\dim\big(\{0_E\}\big) = 0\]
\end{rmk}

\begin{thm}
	Soit $E$ un $\mathbbm{K}$-espace vectoriel de dimension finie, $F$ un sous-espace vectoriel de $E$. Alors, $F$ est de dimension finie et  $\dim(F) \le \dim(E)$\\
	Si $\dim(F) = \dim(E)$, alors $F = E$
\end{thm}

\begin{prv}
	On considère \[
		A = \{k \in \N \mid \text{il existe une famille libre de $F$ à $k$ éléments}\} 
	\]
	On suppose $F \neq \{0_E\}$.
	\begin{itemize}
		\item Soit $u \in F\setminus \{0_E\}$. $(u)$ est libre donc $1 \in A$ et donc $A \neq \O$
		\item Soit $\mathcal{L}$ une famille libre de $F$. Alors, $\mathcal{L}$ est une famille libre de $E$ \\
			donc $\#\mathcal{L} \le \dim(E)$\\
			Donc $A$ est majorée par $\dim(E)$ \\
			On en déduit que $A$ a un plus grand élément $p$.
		\item Soit $\mathcal{L}$ une famille libre de $F$ avec $p$ éléments.\\
			Si $\mathcal{L}$ n'engendre pas $F$, alors il existe $u\in F$ tel que $u\not\in \Vect(\mathcal{L})$ et donc $\mathcal{L} \cup \{u\}$ est une famille libre de $F$, donc $p+1 \in A$ en contradiction avec la maximalité de $p$.\\
			Donc $\mathcal{L}$ est une base de $F$ donc $F$ est de dimension finie et $\dim(F) = p \le \dim(E)$\\
	\end{itemize}

	Soit $\mathcal{B}$ une base de $F$. Alors, $\mathcal{B}$ est aussi une famille de libre de de $E$. Donc $\#\mathcal{B} \le \dim(E)$ donc $\dim(F) = \dim(E)$ \\
	Si $\dim(F) = \dim(E)$, alors $\mathcal{B}$ est une base de $E$, et donc $F = \Vect(\mathcal{B}) = E$
\end{prv}

\begin{prop}
	[Formule de Grassmann]
	Soit $E$ un $\mathbbm{K}$-espace vectoriel de dimension finie, $F$ et $G$ deux sous-espace vectoriels de $E$. Alors, \[
		\dim(F+G) = \dim(F) + \dim(G) - \dim(F\cap G)
	\] 
\end{prop}

\begin{prv}
	Soit $(e_1, \ldots, e_p)$ une base de $F\cap G$. $(e_1,\ldots,e_p)$ est une famille libre de $F$.\\
	On complète $(e_1, \ldots, e_p)$ en une base $(e_1, \ldots, e_p, u_1, \ldots, u_q)$ de $F$.\\
	De même, on complète $(e_1, \ldots, e_p)$ en une base $(e_1, \ldots, e_p, v_1, \ldots, v_r)$ de $G$.\\
	On pose  $\mathcal{B} = (e_1, \ldots, e_p, u_1, \ldots, u_q, v_1, \ldots, v_r)$. Montrons que $\mathcal{B}$ est une base de $F+G$
	\begin{itemize}
		\item Soit $u \in F+G$ \\
			On pose $u = v+w$ avec $\begin{cases}
				v\in F\\
				w \in G
			\end{cases}$.\\
			On pose $v = \sum_{i=1}^p \lambda_i e_i + \sum_{i=1}^q \mu_i u_i$ avec $(\lambda_1, \ldots, \lambda_p, \mu_1, \ldots, \lambda_q) \in \mathbbm{K}^{p+q}$\\
			On pose aussi $w = \sum_{i = 1}^p \lambda'_ie_i + \sum_{j=1}^r \nu_j v_j$ avec $(\lambda_1',\ldots,\lambda_p', \nu_1, \ldots, \nu_r) \in \mathbbm{K}^{p+r}$\\
			D'où, \[
				u = \sum_{i=1}^p (\lambda_i + \lambda'_i)e_i + \sum_{j=1}^q \mu_j u_j + \sum_{k=1}^r \nu_k v_k \in \Vect(\mathcal{B})
			\]
		\item Soient $(\lambda_1, \ldots, \lambda_p, \mu_1, \ldots, \mu_q, \nu_1, \ldots, \nu_r) \in \mathbbm{K}^{p+q+r}$.\\
			On suppose \[
				(*)\quad \sum_{i=1}^{p}\lambda_ie_i + \sum_{j=1}^q\mu_ju_j + \sum_{k=1}^r \nu_k v_k = 0_E
			\] 
			D'où, \[
				\underbrace{\sum_{i=1}^p\lambda_i e_i + \sum_{j=1}^q \mu_ju_j}_{\in F} = \underbrace{-\sum_{k=1}^r\nu_jv_k}_{\in G}
			\] 
			Donc, \[
				f = \sum_{i=1}^p \lambda_i e_i + \sum_{j=1}^q \mu_j u_j \in F\cap G
			\] Comme $(e_1, \ldots, e_p)$ est une base de $F\cap G$, $\exists ! (\lambda_1', \ldots, \lambda_p') \in \mathbbm{K}^p$ tel que \[
				f = \sum_{i=1}^p \lambda'_i e_i = \sum_{i=1}^p \lambda'_i e_i + \sum_{j=1}^q 0_\mathbbm{K}u_j
			\] Comme $(e_1, \ldots, e_p, u_1, \ldots, u_q)$ est une base de $F$, \[
				\forall k \in \left\llbracket 1, q \right\rrbracket, \mu_j = 0_\mathbbm{K}
			\] De même, \[
				\forall k \in \left\llbracket 1,r \right\rrbracket , \nu_k = 0_\mathbbm{K}
			\] On remplace dans $(*)$ pour trouver \[
				\sum_{i=1}^p \lambda_ie_i = 0_E
			\] Comme $(e_1, \ldots, e_p)$ est libre, \[
				\forall i \in \left\llbracket 1,p \right\rrbracket, \lambda_i = 0_\mathbbm{K}
			\] Donc $\mathcal{B}$ est libre.\\
			Donc, 
			\begin{align*}
				\dim(F+G) &=  p +q + r \\
				&= (p+q)+ (p+r) - p \\
				&= \dim(F) + \dim(G) - \dim(F\cap G) \\
			\end{align*}
	\end{itemize}
\end{prv}

\begin{crlr}
	Avec les hypothèse précédentes, \[
		E = F \oplus G \iff \begin{cases}
			F \cap  G = \{0_E\} \\
			\dim(E) = \dim(F) + \dim(G)
		\end{cases}
	\] 
\end{crlr}

\begin{prv}
	\begin{itemize}
		\item[``$\implies$''] On suppose $E = F \oplus G$ \\
			Comme la somme est directe, $F \cap G = \{0_E\}$ 
			\begin{align*}
				\dim(E) &= \dim(F)\\
				&= \dim(F) + \dim(G) - \dim(F\cap G)\\
				&= \dim(F) + \dim(G)\\
			\end{align*}
		\item[``$\impliedby$''] On suppose $F\cap G = \{0_E\}$ et $\dim(E) = \dim(F) + \dim(G)$.\\
			On sait déjà que $F+G = F \oplus G$\\
			 \begin{align*}
				\dim(F+G) = \dim(F) + \dim(G) - \dim(F \cap G) = \dim(E)
			\end{align*}
			Donc $F + G = E$
	\end{itemize}
\end{prv}

\begin{prop}
	Soit $F$ un $\mathbbm{K}$-espace vectoriel de dimension finie $n$. Soit $\mathcal{B} = (e_1, \ldots, e_n)$ une base de $F$. L'application
	\begin{align*}
		f: \mathbbm{K}^n &\longrightarrow F \\
		(\lambda_1, \ldots, \lambda_n) &\longmapsto \sum_{i=1}^n \lambda_i e_i
	\end{align*} est bijective.\\
	Si $\mathbbm{K}$ est infini, $\mathbbm{K}^n$ aussi et donc $F$ aussi.\\
	Si $\#\mathbbm{K} = p \in \N_*$,
	\begin{align*}
		\#&\mathbbm{K}^n = p^n\\
		&\vrt=\\
		\#&F
	\end{align*}
\end{prop}


		\part{Dérivation}

\underline{Motivation}:

{
\begin{wrapfigure}{l}{3cm}
	\centering
	\begin{asy}
		import three;

		size(3cm);
		settings.render=0;
		settings.prc=false;
		currentprojection = obliqueZ;

		draw(unitbox);
		draw(shift(1.1Z + 0.05X) * (O -- X), Arrows3(TeXHead2));
		draw(shift(1.1Z + 0.05Y) * (O -- Y), Arrows3(TeXHead2));
		draw(shift(1.1X + 0.05Z) * (O -- Z), Arrows3(TeXHead2));

		label("$x$", (X/2) + (1.1Z + 0.05X), align=S);
		label("$y$", (Y/2) + (1.1Z + 0.05Y), align=W);
		label("$z$", (Z/2) + X, align=SE);
	\end{asy}
\end{wrapfigure}

\begin{align*}
	&S(x,y,z) = 2(xy + xz + yz)\\
	&V(x,y,z) = xyz
\end{align*}

On cherche à minimiser $S$ avec la contrainte $V = 1$.

Soit $f : \begin{array}{rcl}
	\left( \R_*^+ \right)^2 &\longrightarrow& \R \\
	(x,y) &\longmapsto& S\left( x,y,\frac{1}{xy} \right) = 2\left( xy + \frac{1}{y} + \frac{1}{x} \right).
\end{array}$

On cherche $(a,b) \in \left( \R^+_* \right)^2$ tel que \[
	\forall (x,y) \in (\R^+_*), f(x,y) \ge f(a,b).
\]
}

\begin{defn}
	Soit $f: U \to \R$ où $U$ est un ouvert de $\R^2$. Soit $(a,b) \in U$.
	\vspace{2mm}

	Si $\lim_{x \to a} \frac{f(x,b) - f(a,b)}{x - a} \in \R$, alors on dit que $f$ a une dérivée partielle suivant $x$ en $(a,b)$ et cette limite est notée \[
		\partial f_1(a,b) = \frac{\partial f}{\partial x}(a,b).
	\]

	Si $\lim_{y \to b} \frac{f(a,y) - f(a,b)}{y - b} \in \R$, alors on dit que $f$ a une dérivée partielle suivant $y$ et la limite est notée \[
		\partial f_2(a,b) = \frac{\partial f}{\partial y}(a,b).
	\]
\end{defn}

\begin{exm}
	\begin{enumerate}
		\item $f: (x,y) \mapsto xy + x - y$.

			\begin{align*}
				&\frac{\partial f}{\partial x} : (x,y) \mapsto y + 1,\\
				&\frac{\partial f}{\partial y} : (x,y) \mapsto x - 1.
			\end{align*}

		\item $f: (x,y) \mapsto xy + \frac{1}{y}+ \frac{1}{x}$.

			\begin{align*}
				&\frac{\partial f}{\partial x}: (x,y) \mapsto y - \frac{1}{x^2},\\
				&\frac{\partial f}{\partial y}: (x,y) \mapsto x - \frac{1}{y^2}.
			\end{align*}

		\item Trouver $f$ telle que $\begin{cases}
				(1): \qquad \frac{\partial f}{\partial x}=y,\\[2mm]
				(2): \qquad \frac{\partial f}{\partial y} = x.
			\end{cases}$

			D'après $(1)$ : \[
				\forall (x,y), \exists C(y) \in \R, f(x,y) = xy + C(y)
			\] et donc \[
				\frac{\partial f}{\partial y}(x,y) = x + C'(y)
			\] donc $C'(y) = 0$ et donc $C$ est constante.

		\item Trouver $f$ telle que $\begin{cases}
			\frac{\partial f}{\partial x} = -y,\\[2mm]
			\frac{\partial f}{ƒ\partial y} = x.
		\end{cases}$

		Ce n'est pas possible !
	\end{enumerate}
\end{exm}

\begin{defn}~\\
	\begin{minipage}{\linewidth}
		\begin{wrapfigure}{r}{4cm}
			\centering
			\vspace{-5mm}
			\begin{asy}
				import three;
				import graph3;
				size(4cm);

				settings.render = 0;
				settings.prc = false;
				currentprojection = obliqueX;

				draw(O -- X, Arrow3(TeXHead2));
				draw(O -- Y, Arrow3(TeXHead2));
				draw(O -- Z, Arrow3(TeXHead2));

				triple f(real x, real y, real z = 0) { return (x,y,cos(x - 0.5) * cos(y - 0.5)/1.2 + 0.15); }

				real inc = 1 / 5;

				for(real x = 0; x <= 1; x += inc) {
					draw(graph(
						new real(real t) { return x; }, // x
						new real(real y) { return y; }, // y
						new real(real y) { return f(x,y).z; }, // z
						0, 1
					), gray);
				}

				for(real y = 0; y <= 1; y += inc) {
					draw(graph(
						new real(real x) { return x; }, // x
						new real(real t) { return y; }, // y
						new real(real x) { return f(x,y).z; }, // z
						0, 1
					), gray);
				}

				path3 path1 = (0.8, 0.2, 0) .. (0.5, 0.5, 0) .. (0.3, 0.7, 0);
				path3 path2 = f(0.8, 0.2, 0) .. f(0.5, 0.5, 0) .. f(0.3, 0.7, 0);
				path3 d = (0.2, 0.3, 0) .. (0.3, 0.4, 0) .. (0.2, 0.7, 0) .. (0.8, 0.9, 0) .. (0.6, 0.2, 0) .. cycle;

				draw(path1, red, Arrow3(TeXHead2));
				draw(path2, red, Arrow3(TeXHead2, position=0.8));

				dot((0.5, 0.5, 0));
				dot(f(0.5, 0.5, 0));
				draw((0.5, 0.5, 0) -- f(0.5, 0.5, 0), dashed);
				draw(d);

				label("$w$", (0.3, 0.7, 0), red, align=SE);
				label("$U$", (0.8, 0.9, 0), align=SE);
			\end{asy}
		\end{wrapfigure}

		Soit $f: U \to \R$ où $U$ est un ouvert. Soit $(a,b) \in U$. Soit $w = (w_1, w_2) \in \R^2$.

		Si 
		\[
			\lim_{t\to 0} \frac{f(a + tw_1, b + tw_2) - f(a,b)}{t}
		\] existe et est réelle, alors on dit que $f$ a une dérivée dans la direction de $w$ et la limite est notée \[
			\mathrm{d}f(w)\,(a,b) = D_w(f)\,(a,b).
		\]
	\end{minipage}
\end{defn}

\begin{exm}
	\begin{align*}
		f: \left( \R_*^+ \right)^2 &\longrightarrow \R \\
		(x,y) &\longmapsto xy+\frac{1}{x}+\frac{1}{y}.
	\end{align*}

	On pose $(a,b) = (1,2)$, $w = (w_1, w_2) = (1,1)$.
	\begin{align*}
		\frac{f(1+t, 2+t) - f(1,2)}{t} &= \frac{1}{t} \left( (1+t)(2+t) + \frac{1}{1+t} + \frac{1}{2+t} - 3 - \frac{1}{2} \right) \\
		&= \frac{1}{t}\left(\cancel 2 + 3t + \po(t) + \cancel 1 - t + \po(t) + \frac{1}{2}\left( \cancel 1 - \frac{t}{2} + \po(t) \right) - \cancel3 - \cancel{\frac{1}{2}} \right) \\
		&= \frac{1}{t} \left( \frac{7}{4} t + \po(t) \right)  \\
		&= \frac{7}{4} + \po(1) \tendsto{t \to 0} \frac{7}{4}. \\
	\end{align*}

	Donc, \[
		\mathrm{d}f(1,1)\,(1,2) = \frac{7}{4}.
	\]
\end{exm}

\begin{rmk}~\\
	\begin{figure}[H]
		\centering
		\begin{asy}
			import solids;
			import graph;
			size(5cm);

			settings.render = 0;
			settings.prc = false;

			path3 par = graph(
				new real(real x) { return x; },
				new real(real x) { return 0; },
				new real(real x) { return x^2; },
				0,3);
			revolution r = revolution(par, axis=Z);

			path3 par2 = graph(
				new real(real x) { return x; },
				new real(real x) { return 0; },
				new real(real x) { return x^2; },
				-3,3);

			draw(r,1,longitudinalpen=nullpen);
			draw(r.silhouette());

			draw((-4, 0, -1) -- (-4, 0, 10) -- (4, 0, 10) -- (4, 0, -1) -- cycle, red);
			draw(par2, deepred);

			draw((4,4.5) -- (7, 4.5), black+0.5mm, Arrow(TeXHead));

			path par2d = graph(new real(real x) { return x^2; }, -3, 3);
			draw(shift((11, 0)) * par2d, deepred);

			dot(O);
			dot((11, 0));
		\end{asy}
	\end{figure}
\end{rmk}


%todo ajouter théorème-définition
\begin{thm}
	Soit $f : U \to \R$, $(a,b) \in U$. On suppose que $\frac{\partial f}{\partial x}$ et $\frac{\partial f}{\partial y}$ existent en $(a,b)$ et sont {\bfseries continues} en $(a,b)$. Alors,
	\begin{align*}
		&\forall (h, k) \in \R^2 \text{ tel que } (a +h, b + k) \in U,\\
		&f(a+ h, b + k) = f(a,b) + h \frac{\partial f}{\partial x}(a,b) + k \frac{\partial f}{\partial y}(a,b) + \po_{(h,k)\to (0,0)}\big(\|(h,k)\|\big).
	\end{align*}

	On dit que $f$ est de classe $\mathcal{C}^1$ si $\frac{\partial f}{\partial x}$ et $\frac{\partial f}{\partial y}$ existent et sont continues.

	\qed
\end{thm}

\begin{rmk}
	En physique, cette formule correspond à : \[
		\mathrm{d}f = \frac{\partial f}{\partial x}\mathrm{d}x + \frac{\partial f}{\partial y} \mathrm{d}y.
	\] En effet :
	\begin{align*}
		\mathrm{d}f &= f(x+ \mathrm{d}x, y + \mathrm{d}y) - f(x,y) \\
		&= \frac{\partial f}{\partial x} \mathrm{d}x + \frac{\partial f}{\partial y} \mathrm{d}y.
	\end{align*}
\end{rmk}

\begin{prop}
	Soit $f: U \to \R$ de classe $\mathcal{C}^1$ en $(a,b) \in U$. Alors, \[
		\forall w = (w_1, w_2) \in \R^2, \mathrm{d}f(w)\,(a,b) = w_1 \frac{\partial f}{\partial x}(a,b) + w_2 \frac{\partial f}{\partial y}(a,b).
	\]
\end{prop}

\begin{prv}
	Soit $w = (w_1, w_2) \in \R^2$. Soit $t \in \R^*$.
	\begin{align*}
		\frac{1}{t}\big(f(a + tw_1, b + tw_2) - f(a,b)\big)
		&= \frac{1}{t} \left( tw_1 \frac{\partial f}{\partial x}(a,b) + tw_2 \frac{\partial f}{\partial y}(a,b) + \po_{t \to 0}\big(\|tw\|\big) \right) \\
		&= w_1 \frac{\partial f}{\partial x}(a,b) + w_2 \frac{\partial f}{\partial y}(a,b) + \po_{t\to 0}(1) \\
		&\tendsto{t\to 0} w_1 \frac{\partial f}{\partial x}(a,b) + w_2\frac{\partial f}{\partial y}(a,b).
	\end{align*}
\end{prv}


\begin{defn}
	Avec les hypothèses précédentes, en posant \[
		\nabla f(a,b) = \left( \frac{\partial f}{\partial x}(a,b), \frac{\partial f}{\partial y}(a,b) \right) 
	\]on obtient \[
		\mathrm{d}f(w)\,(a,b) = \left<w  \mid \nabla f(a,b) \right>
	\] où $\left<\cdot|\cdot \right>$ est le produit scalaire.

	Le vecteur $\nabla f(a,b)$ est appelé \underline{gradient de $f$ en $(a,b)$}.

	Le développement limité à l'ordre 1 de $f$ devient \[
		f\big((a,b)+w\big) = f(a,b) + \left<w \mid \nabla f(a,b) \right> + \po_{w\to 0}(\|w\|)
	\]
\end{defn}

\begin{prop}
	Soit $f : U \to \R$ de classe $\mathcal{C}^1$.

	\begin{figure}[H]
    \centering
    \incfig{gradient}
	\end{figure}

	$\nabla f$ est orthogonal au lignes de niveaux de $f$, son orientation va dans le sens d'une augmentation de $f$.
\end{prop}

\begin{prv}
	Soit $\gamma : I \to U$ une courbe de niveau : \[
		\forall t \in I, f\big(\gamma(t)\big) = \text{cste}.
	\] D'après le lemme suivant : \[
		\forall t \in I, 0 = (f \circ \gamma)'(t) = \mathrm{d}f\big(\gamma'(t)\big)\big(\gamma(t)\big) = \left<\gamma'(t)  \mid \nabla f\big(\gamma(t)\big) \right>
	\] Donc $\nabla f\big(\gamma(t)\big)$ est orthogonal à $\gamma'(t)$.

	Pour tout $t \in I$, on pose $w(t) = t\, \nabla f\big(\gamma(t)\big)$. Donc \[
		f\big(\gamma(t) + w(t)\big) = f\big(\gamma(t)\big) + t \|\nabla f(\gamma(t))\|^2 + \po_{t \to 0}(t)
	\] Pour $t$ assez petit, $f\big(\gamma(t) + w(t)\big) - f\big(\gamma(t)\big)$ est du même signe que $t$.
\end{prv}

\begin{rmk}
	\begin{align*}
		V: \R^3 &\longrightarrow \R \\
		(x,y,z) &\longmapsto -mgz
	\end{align*}
	l'énerge potentielle de pesenteur

	On a donc \[
		\nabla V(x,y,z) = \left( \frac{\partial V}{\partial x}, \frac{\partial V}{\partial y}, \frac{\partial V}{\partial z} \right) = (0, 0, -mg) = \vec{P}.
	\]
\end{rmk}

\begin{lem}
	Soit $f : U \to \R$ de classe $\mathcal{C}^1$, $\gamma : \begin{array}{rcl}
		I &\longrightarrow& U \\
		t &\longmapsto& \big(x(t), y(t)\big)
	\end{array}$ où $x$ et $y$ sont dérivables.

	On pose \[
		\forall t \in I, \gamma'(t) = \big(x'(t), y'(t)\big).
	\] Alors $f \circ \gamma : I \to \R$ est dérivable et
	\begin{align*}
		\forall t \in I, (f \circ \gamma)'(t) &= \mathrm{d}f\big(\gamma'(t)\big) \big(\gamma(t)\big)\\
		&= \left<\gamma'(t)  \mid \nabla f\big(\gamma(t)\big)  \right> \\
		&= x'(t) \frac{\partial f}{\partial x}\big(x(t), y(t)\big) + y'(t) \frac{\partial f}{\partial y}\big(x(t),y(t)\big). \\
	\end{align*}
\end{lem}

\begin{prv}
	On fixe $t \in I$.

	\begin{align*}
		\forall h \neq 0, \frac{f \circ \gamma(t + h) - f \circ \gamma(t)}{h}
		&= \frac{1}{h}\big(f(\gamma(t)) + h\gamma'(t) + \po_{h\to 0}(h) - f(\gamma(t))\big) \\
		&= \frac{1}{h}\bigg(\cancel{f(\gamma(t))} + \left<h\gamma'(t) \mid \nabla f(\gamma(t)) \right> + \po_{h\to 0}(\|h\gamma'(t)\|) - \cancel{f(\gamma(t))}\bigg)\\
		&= \left<\gamma'(t) \mid \nabla f(\gamma(t)) \right> + \po_{h\to 0}(1) \\
		&\tendsto{h\to 0} \left<\gamma'(t)  \mid \nabla f(\gamma(t)) \right>
	\end{align*}
\end{prv}

\begin{defn}
	Soit $f : U \to \R$ de classe $\mathcal{C}^1$ et $(a,b) \in U$. On dit que $(a,b)$ est un \underline{point critique} de $f$ si $\nabla f(a,b) = 0$ i.e. $\frac{\partial f}{\partial x}(a,b) = \frac{\partial f}{\partial y}(a,b) = 0$.

	Dans ce cas, $f(a,b)$ est appelé \underline{valeur critique} de $f$.
\end{defn}

\begin{prop}~\\
	\begin{minipage}{\linewidth}
		\begin{wrapfigure}{r}{3cm}
			\centering
			\vspace{-1cm}
			\begin{asy}
				import solids;
				import graph;
				size(3cm);

				settings.render = 0;
				settings.prc = false;

				path3 par = graph(
					new real(real x) { return x; },
					new real(real x) { return 0; },
					new real(real x) { return -x^2; },
					0,3);
				revolution r = revolution(par, axis=Z);

				draw(r,1,longitudinalpen=nullpen);
				draw(r.silhouette());

				dot("$(a,b)$", O, red, align=N);
				real s = sqrt(2.5);
				path3 g=(s,0,-2.5)..(0,s,-2.5)..(-s,0,-2.5)..(0,-s,-2.5)..cycle;
				draw(g, deepcyan);
			\end{asy}
		\end{wrapfigure}
		Soit $f: U \to \R$ de classe $\mathcal{C}^1$ et $(a,b) \in U$ tel que \[
			\exists r > 0, \forall (x,y) \in B_{(a,b)}(r), f(x,y) \le f(a,b)
		\] Alors $\nabla f(a,b) = (0,0)$.
	\end{minipage}
\end{prop}

\begin{prv}
	Soit $g: x \mapsto f(x,b)$. $g(a)$ est un maximum local de $g$ donc $g'(a) = 0$.

	Or, $g'(a) = \frac{\partial f}{\partial x}(a,b)$

	donc $\frac{\partial f}{\partial x}(a,b) = 0$.

	Soit $h : y \mapsto f(a,y)$. On a de même $h'(b) = 0$.

	Or, $h'(b) = \frac{\partial f}{\partial y}(a,b)$.

	Donc, $\nabla f(a,b) = (0,0)$.
\end{prv}

\begin{rmk}
	Un minimum local est aussi une valeur critique.
\end{rmk}

\begin{figure}[H]
	\centering
	\begin{subfigure}{3cm}
		\centering
		\begin{asy}
			import solids;
			import graph;
			size(3cm);

			settings.render = 0;
			settings.prc = false;

			path3 par = graph(
				new real(real x) { return x; },
				new real(real x) { return 0; },
				new real(real x) { return -x^2; },
				0,3);
			revolution r = revolution(par, axis=Z);

			draw(r,1,longitudinalpen=nullpen);
			draw(r.silhouette());

			dot(O, red);
		\end{asy}
		\caption{Maximum local}
	\end{subfigure}
	\begin{subfigure}{3cm}
		\centering
		\begin{asy}
			import solids;
			import graph;
			size(3cm);

			settings.render = 0;
			settings.prc = false;

			path3 par = graph(
				new real(real x) { return x; },
				new real(real x) { return 0; },
				new real(real x) { return x^2; },
				0,3);
			revolution r = revolution(par, axis=Z);

			draw(r,1,longitudinalpen=nullpen);
			draw(r.silhouette());

			dot(O, red);
		\end{asy}
		\caption{Minimum local}
	\end{subfigure}
	\begin{subfigure}{3cm}
		\centering
		\begin{asy}
			import solids;
			import graph;
			size(3cm);

			settings.render = 0;
			settings.prc = false;
			currentprojection = obliqueZ;

			draw(graph(
				new real(real x) { return x; },
				new real(real x) { return -x^2 / 3; },
				new real(real x) { return 3; },
				-3, 3
			));

			draw(graph(
				new real(real x) { return x; },
				new real(real x) { return -x^2 / 3; },
				new real(real x) { return -3; },
				-3, 3
			));

			draw(graph(
				new real(real x) { return x; },
				new real(real x) { return -x^2 / 3 - 1; },
				new real(real x) { return 0; },
				-3, 3
			));

			draw(graph(
				new real(real x) { return 0; },
				new real(real x) { return x^2 / 9 - 1; },
				new real(real x) { return x; },
				-3, 3
			));

			draw(graph(
				new real(real x) { return -3; },
				new real(real x) { return x^2 / 9 - 4; },
				new real(real x) { return x; },
				-3, 3
			));

			draw(graph(
				new real(real x) { return 3; },
				new real(real x) { return x^2 / 9 - 4; },
				new real(real x) { return x; },
				-3, 3
			));

			dot((0,-1,0), red);
		\end{asy}
		\caption{Point de selle / Point col}
	\end{subfigure}
\end{figure}

\begin{exm}
	On revient à l'exemple donné en introduction : 
	\begin{align*}
		f: \left( \R^*_+ \right)^2 &\longrightarrow \R \\
		(x,y) &\longmapsto 2\left( xy + \frac{1}{x} + \frac{1}{y} \right).
	\end{align*}

	$\left( \R^+_* \right)^2$ est un ouvert de $\R^2$. Soit $(x,y) \in \left( \R^+_* \right)^2$.
	
	On a \[
		\begin{cases}
			\frac{\partial f}{\partial x}(x,y) = 2\left( y - \frac{1}{x^2} \right),\\
			\frac{\partial f}{\partial y}(x,y) = 2\left( x - \frac{1}{y^2} \right).
		\end{cases}
	\]

	\begin{align*}
		&\frac{\partial f}{\partial x}(x,y) = \frac{\partial f}{\partial y}(x,y) = 0\\
		\iff& \begin{cases}
			y = \frac{1}{x^2}\\
			x = \frac{1}{y^2}
		\end{cases}\\
		\iff& \begin{cases}
			y = \frac{1}{x^2}\\
			x = x^4
		\end{cases}\\
		\iff& \begin{cases}
			x = 1\\
			y = 1
		\end{cases}
	\end{align*}

	On vérivie que $f$ présente en effet un minium local en $(1,1)$. \[
		f(1,1) = 6
	\] On fixe $y \in \R^+_*$ et \[
		g : x \mapsto 2\left( xy + \frac{1}{x} + \frac{1}{y} \right).
	\] Donc \[
		\forall x \in \R^+_*, g'(x) = 2\left( y - \frac{1}{x^2} \right).
	\]
	\begin{center}
		\begin{tikzpicture}
			\tkzTabInit{$x$/1,$g'(x)$/1,$g$/2.3}{$0$, $\frac{1}{\sqrt{y}}$, $+\infty$}
			\tkzTabLine{,-,z,+,}
			\tkzTabVar{+/{}, -/$2\left( 2\sqrt{y} +\frac{1}{y} \right)$, +/{}}
		\end{tikzpicture}
	\end{center}
	
	Ainsi, \[
		\forall x \in \R^+_*, \forall y \in \R^+_*, f(x,y) \ge 2\left( 2\sqrt{y} + \frac{1}{y} \right)
	\] Soit $h : y \mapsto 2\sqrt{y} + \frac{1}{y}$. On a \[
		\forall y > 0, h'(y) = \frac{1}{\sqrt{y}} - \frac{1}{y^2} = \frac{y\sqrt{y} - 1}{y^2} = \frac{y^{\frac{3}{2}} - 1}{y^2}
	\]

	\begin{center}
		\begin{tikzpicture}
			\tkzTabInit{$y$/0.7,$h'(y)$/0.7,$h$/1.4}{$0$, $1$, $+\infty$}
			\tkzTabLine{,-,z,+,}
			\tkzTabVar{+/{}, -/$3$, +/{}}
		\end{tikzpicture}
	\end{center}

	Donc, \[
		\forall x,y > 0, f(x,y) \ge 2\times 3 = 6 = f(1,1).
	\]
\end{exm}

\begin{prop}
	[règle de la chaîne]

	Soit $f : \begin{array}{rcl}
		U &\longrightarrow& \R^2 \\
		(x,y) &\longmapsto& f(x,y)
	\end{array}$ de classe $\mathcal{C}^1$ et $U, V$ deux ouverts de $\R^2$.

	Soit $\varphi : \begin{array}{rcl}
		V &\longrightarrow& U \\
		(u,v) &\longmapsto& \varphi(u,v) = \big(x(u,v), y(u,v)\big)
	\end{array}$.

	On suppose que $x$ et $y$ sont de classe $\mathcal{C}^1$ sur $V$.

	Alors,  $f \circ \varphi : \begin{array}{rcl}
		V &\longrightarrow& \R \\
		(u,v) &\longmapsto& f\big(\varphi(u,v)\big)
	\end{array}$ est de classe $\mathcal{C}^1$ et
	\begin{align*}
		\forall (u_0, v_0) \in V, \frac{\partial (f \circ \varphi)}{\partial u}(u_0, v_0)
		&= \frac{\partial f}{\partial x}\big(\varphi(u_0, v_0)\big) \times \frac{\partial x}{\partial u}(u_0, v_0)\\
		&+ \frac{\partial f}{\partial y}\big(\varphi(u_0,v_0)\big) \frac{\partial y}{\partial u}(u_0,v_0)
	\end{align*}
	\begin{align*}
		\forall (u_0, v_0) \in V, \frac{\partial (f \circ \varphi)}{\partial v}(u_0, v_0)
		&= \frac{\partial f}{\partial x}\big(\varphi(u_0, v_0)\big) \times \frac{\partial x}{\partial v}(u_0, v_0)\\
		&+ \frac{\partial f}{\partial y}\big(\varphi(u_0,v_0)\big) \frac{\partial y}{\partial v}(u_0,v_0)
	\end{align*}
\end{prop}

\begin{exm}
	[changement de coordonnées polaires]
	On pose \begin{align*}
		\varphi: \R^+_* \times ]0,2\pi[ &\longrightarrow \R^2\setminus \left( R^+_* \times \{0\} \right) \\
		(r, \theta) &\longmapsto (r \cos \theta, r \sin\theta),
	\end{align*}
	\begin{align*}
		f: \R^2\setminus \left( R^+_* \times \{0\} \right) &\longrightarrow \R \\
		(x,y) &\longmapsto f(x,y),
	\end{align*}
	\begin{align*}
		g: \overbrace{\R^+_* \times ]0, 2\pi[}^{=V} &\longrightarrow \R \\
		(r, \theta) &\longmapsto f(r\cos\theta, r\sin\theta).
	\end{align*}

	\begin{align*}
		\forall (r_0,\theta_0) \in V,&\\[5mm]
		\frac{\partial g}{\partial r}(r_0, \theta_0) &= \frac{\partial f}{\partial x}(r_0\cos\theta_0, r_0\sin\theta_0)\cos\theta_0\\
		&+ \frac{\partial f}{\partial y}(r_0 \cos\theta_0, r_0\sin\theta_0)\sin\theta_0\\
		&= 2r_0\cos^2\theta_0 + 2r_0\sin^2(\theta_0) \\
		&= 2r_0 \\[5mm]
		\frac{\partial g}{\partial \theta}(r_0, \theta_0) &= \frac{\partial f}{\partial x}(r_0\cos\theta_0, r_0\sin\theta_0)r_0\sin\theta_0\\
		&+ \frac{\partial f}{\partial y}(r_0 \cos\theta_0, r_0\sin\theta_0)r_0\cos\theta_0\\
		&= -2{r_0}^2\cos(\theta_0)\sin(\theta_0) + 2{r_0}^2 \sin(\theta_0)\cos(\theta_0)\\
		&= 0 \\
	\end{align*}

	Donc, \[
		g(r, \theta) = r^2.
	\]
\end{exm}

\begin{exm}
	Résoudre \[
		\begin{cases}
			\frac{\partial f}{\partial x} = \frac{x}{x^2+y^2},\\
			\frac{\partial f}{\partial y} = \frac{y}{x^2+y^2}.\\
		\end{cases}
	\]

	On pose $g: (r, \theta) \mapsto f(r \cos\theta, r \sin\theta)$.

	\begin{align*}
		&\frac{\partial g}{\partial r} = \frac{1}{r}\cos^2\theta + \frac{1}{r}\sin^2\theta = \frac{1}{r},\\
		&\frac{\partial g}{\partial \theta} = -\cos(\theta) \sin(\theta) + \sin(\theta)\cos(\theta) = 0.
	\end{align*}

	Donc, \[
		\exists C \in \R, g: (r, \theta) \mapsto \ln r + C
	\] d'où,
	\begin{align*}
		\forall (x,y) \in \R^2 \setminus \{(0,0)\}, f(x,y) &= \ln\left(\sqrt{x^2 + y^2} \right)  + C\\
		&= \frac{1}{2}\ln(x^2 + y^2) + C. \\
	\end{align*}
\end{exm}

\begin{rmk}
	Soit $\mathcal{B} = (e_1, e_2)$ la base canonique de $\R^2$, $f: U \to \R$ de classe $\mathcal{C}^1$ avec $U$ un ouvert de $\R^2$.

	Soit $(x,y) \in U$.

	\begin{align*}
		\Mat_{\mathcal{B}}\big(\nabla f(x,y)\big) = \begin{pmatrix}
			\frac{\partial f}{\partial x}(x,y)\\[2mm]
			\frac{\partial f}{\partial y}(x,y)
		\end{pmatrix}
	\end{align*}

	Soit  \begin{align*}
		\varphi: V &\longrightarrow U \\
		(u,v) &\longmapsto \big(x(u,v), y(u,v)\big) 
	\end{align*} avec $x,y$ de classe $\mathcal{C}^1$. Soit $g = f \circ \varphi$.
	\begin{align*}
		\Mat_{\mathcal{B}}\big(\nabla g(u,v)\big)
		&= \begin{pmatrix}
			\frac{\partial g}{\partial u}(u,v) \\[2mm]
			\frac{\partial g}{\partial v}(u,v)
		\end{pmatrix} \\
		&= \begin{pmatrix}
			\frac{\partial x}{\partial u}(u,v) \frac{\partial f}{\partial x}(x,y)
			+ \frac{\partial y}{\partial u}(u,v)\frac{\partial f}{\partial y}(x,y)\\[3mm]
			\frac{\partial x}{\partial v}(u,v) \frac{\partial f}{\partial x}(x,y)
			+ \frac{\partial y}{\partial v}(u,v) \frac{\partial f}{\partial y}(x,y)
		\end{pmatrix}  \\
		&= \underbrace{\begin{pmatrix}
				\frac{\partial x}{\partial u}(u,v)& \frac{\partial y}{\partial u}(u,v)\\[3mm]
				\frac{\partial x}{\partial v}(u,v)& \frac{\partial y}{\partial v}(u,v)
		\end{pmatrix}}_{J(u,v)} \begin{pmatrix}
			\frac{\partial f}{\partial x}(x,y)\\[3mm]
			\frac{\partial f}{\partial y}(x,y)
		\end{pmatrix} \\
		&= J(u,v) \Mat_{\mathcal{B}}\big(\nabla f(x,y)\big) \\
	\end{align*}
	où $J(u,v) = 
	\begin{pNiceArray}{c:c}
		\Mat_{\mathcal{B}}\big(\nabla x(u,v)\big) & \Mat_{\mathcal{B}}\big(\nabla y(u,v)\big)
	\end{pNiceArray}$.

	On dit que $J(u,v)$ est \underline{la jacobienne} de $\varphi$ en $(u,v)$.
	L'application linéaire canoniquement associée à $J(u,v)$ est la \underline{différentielle de $\varphi$} en $(u,v)$ noté $\mathrm{d}\varphi(u,v)$.

	On a $\mathrm{d}\varphi(u,v) \in \mathcal{L}(R^2)$ et $\Mat_{\mathcal{B}}\big(\mathrm{d}\varphi(u,v)\big) = J(u,v)$.

	Par exemple, la jacobienne du changement de coordonnées polaires est \[
		J = \begin{pmatrix}
			\frac{\partial x}{\partial r} & \frac{\partial y}{\partial r}\\[3mm]
			\frac{\partial x}{\partial \theta} & \frac{\partial y}{\partial \theta}
		\end{pmatrix}
		= \begin{pmatrix}
			\cos\theta&\sin\theta\\
			-r\sin\theta&r\cos\theta
		\end{pmatrix}.
	\]
	$\underbrace{\det(J)}_{\text{le jacobien}} = r\cos^2\theta + r\sin^2\theta = r$

	Dans une intégrale double, si $(x,y) = \varphi(u,v)$, alors $\mathrm{d}x\mathrm{d}y = \det(J)\mathrm{d}u\mathrm{d}v$.

	Ici, \[
		\mathrm{d}x\ \mathrm{d}y = r\ \mathrm{d}r\ \mathrm{d}\theta.
	\]
\end{rmk}

\begin{prv}
	On pose $(x_0, y_0) = \varphi(u_0, v_0)$. Pour tout $(h,k) \in \R^2$ tels que $(u_0 + h, v_0 + k) \in V$, en posant $g = f  \circ \varphi$.

	\begin{align*}
		g(u_0 + h, v_0 + h) &= f\big(x(u_0 + h, v_0 + k), y(u_0 + h, v_0 + k)\big) \\
		&= f\left(
			x(u_0,v_0) + h \frac{\partial x}{\partial u}(u_0,v_0) + k \frac{\partial x}{\partial v}(u_0, v_0) + \po\big(\|(h,k)\|\big), \right.\\
		&\phantom{ = f\bigg(\bigg.}\left. y(u_0, v_0) + h \frac{\partial y}{\partial u}(u_0, v_0) + k \frac{\partial y}{\partial v}(u_0, v_0) + \po\big(\|(h,k)\|\big)
		\right)  \\
		&= f(x_0,y_0) \\
		&~+ \left( h \frac{\partial x}{\partial u}(u_0,v_0) + k \frac{\partial x}{\partial v}(u_0, v_0) + \po(\|(h,k)\|) \right) \frac{\partial f}{\partial x}(x_0,y_0)\\
		&~+ \left( h \frac{\partial y}{\partial u}(u_0, v_0) + k\frac{\partial y}{\partial v}(u_0, v_0) + \po(\|(h,k)\|) \right) \frac{\partial f}{\partial y}(x_0, y_0)\\
		&~+ \po(\|(h,k)\|)\\
		&= f(x_0, y_0) \\
		&~+ h \left( \frac{\partial x}{\partial u}(u_0, v_0) \frac{\partial f}{\partial x}(x_0, y_0) + \frac{\partial y}{\partial u}(u_0, v_0) \frac{\partial f}{\partial y}(x_0, y_0) \right)  \\
		&~+ k\left( \frac{\partial x}{\partial v}(u_0, v_0) \frac{\partial f}{\partial x}(x_0, y_0) + \frac{\partial y}{\partial v}(u_0, v_0) \frac{\partial f}{\partial y}(x_0, y_0) \right) 
		&~+ \po(\|(h,k)\|)\\
		&= g(u_0, v_0) + h \frac{\partial g}{\partial u}(u_0, v_0) + k \frac{\partial g}{\partial v}(u_0, v_0) + \po(\|(h,k)\|) \\
	\end{align*}

	Par identification,
	\[
		\frac{\partial g}{\partial u}(u_0, v_0) = \frac{\partial x}{\partial u}(u_0, v_0) \frac{\partial f}{\partial x}(x_0, y_0) + \frac{\partial y}{\partial u}(u_0, v_0) \frac{\partial f}{\partial y}(x_0,y_0)
	\] et \[
		\frac{\partial g}{\partial v}(u_0, v_0) = \frac{\partial x}{\partial v}(u_0,v_0) \frac{\partial f}{\partial x}(x_0, y_0) + \frac{\partial y}{\partial v}(u_0, v_0) \frac{\partial f}{\partial y}(x_0, y_0).
	\] 
\end{prv}

\begin{exm}
	[Régression linéaire]~\\
	\begin{figure}[H]
		\centering
		\begin{asy}
			import graph;
			axes(EndArrow);
			size(5cm);

			real f(real x) { return x + 0.5; }

			real k = 35 / (7 - 0.5);

			for(int i = 0; i < 35; ++i) {
				real mag = exp(sin(100 * pi/exp(1) * i)) * 0.8 + exp(cos(i*40)/3);
				real eps = mag * cos(10 * exp(1)/pi * i) / 3;
				dot((i/k,f(i/k) + eps));
			}

			draw(graph(f, -1, 7), orange);
		\end{asy}
	\end{figure}
	\[
		y = a x + b
	\] 
	On fixe $(a,b) \in \R^2$. \[
		\varepsilon(a,b) = \sum_{i=1}^n\big( y_i - (ax_i + b) \big)^2
	\] l'erreur totale.

	On veut minimiser $\varepsilon(a,b)$. On a 
	\[
		\forall (a,b) \in \R^2,
		\begin{cases}
			\frac{\partial \varepsilon}{\partial a}(a,b) = -2\sum_{i=1}^{n}(y_i - ax_i - b)x_i,\\
			\frac{\partial \varepsilon}{\partial b}(a,b) = -2\sum_{i=1}^{n}(y_i - ax_i - b).
		\end{cases}
	\]

	Donc,
	\begin{align*}
		(a,b) \text{ point critique de } \varepsilon \iff& \begin{cases}
			a \sum_{i=1}^n {x_i}^2 + b\sum_{i=1}^{n}x_i = \sum_{i=1}^{n} y_ix_i\\
			a\sum_{i=1}^{n}x_i + nb = \sum_{i=1}^ny_i
		\end{cases}\\
		\iff& \begin{cases}
			a \left( \frac{1}{n}\sum_{i=1}^n {x_i}^2 - \overline{x}^2\right) = \overline{y} - \overline{x} \overline{y}\\
			b = \frac{1}{n}\sum_{i=1}^ny_i - \frac{a}{n}\sum_{i=1}^nx_i = \frac{1}{n}\sum_{i=1}^n x_i y_i - \overline{x} \overline{y}
		\end{cases}\\
		&\text{ où } \overline{x} = \frac{1}{n} \sum_{i=1}^n x_i,~\overline{y} = \frac{1}{n}\sum_{i=1}^n y_i\\
		\iff& \begin{cases}
			a = \frac{\Cov(x,y)}{V(x)}\\
			b = \overline{y} - a\overline{x}
		\end{cases}
	\end{align*}

	Coefficient de corrélation: $\frac{\Cov(x,y)}{\sigma_x \sigma_y} \in [-1, 1]$
\end{exm}












		\part{Corps}

\begin{exm}[Problème]
	\begin{itemize}
		\item 
			avec $A = \Z / 9 \Z$, résoudre $\overline{x}^2 = \overline{0}$ \\
			\begin{center}
				\begin{tabular}{|c|c|c|c|c|c|c|c|c|c|c|}
					\hline
					$\overline{x}$&$\overline{0}$& $\overline{1}$ &$\overline{2}$&$\overline{3}$ &$\overline{4}$ &$\overline{5}$ &$\overline{6}$ &$\overline{7}$ &$\overline{8}$& $\overline{9}$ \\
					\hline
					$\overline{x}^2$&$\overline{0}$ &$\overline{1}$ &$\overline{4}$ &$\overline{0}$ &$\overline{7}$ &$7$ &$\overline{0}$ &$\overline{4}$ &$\overline{1}$&$\overline{0}$\\
					\hline
				\end{tabular}
			\end{center}
			On a trouvé 3 solutions: $\overline{0}$, $\overline{3}$, $\overline{6}$.
		\item $\Z / 8\Z$
			\begin{center}
				\begin{tabular}{|c|c|c|c|c|c|c|c|c|}
					\hline
					$\overline{x}$& $\overline{0}$& $\overline{1}$& $\overline{2}$& $\overline{3}$& $\overline{4}$& $\overline{5}$& $\overline{6}$& $\overline{7}$\\
					\hline
					$\overline{x^2}$& $\overline{0}$& $\overline{1}$& $\overline{4}$& $\overline{1}$& $\overline{0}$& $\overline{1}$& $\overline{4}$& $\overline{1}$\\
					\hline
				\end{tabular}
			\end{center}
			$\overline{x}^2=7$ a 4 solutions: $\overline{1}, \overline{7}, \overline{3},\text{ et } \overline{5}$
		\item $A = \mathbbm{H} = \{a + bi + cj + dk  \mid  (a,b,c,d) \in \R^4\}$ \\
			$i^2 = j^2 = k^2 = -1$ 
			\begin{align*}
				\begin{array}{c c c}
					ij = k & jk = i & ji = j\\
					ji = -k & kj = -i & ik = -j
				\end{array}
			\end{align*}
			Dans cet anneau, $-1$ a 6 racines!
	\end{itemize}
\end{exm}

\begin{defn}
	Soit $(\mathbbm{K}, +, \times)$ un ensemble muni de deux lois de composition internes. On dit que c'est un \underline{corps} si
	 \begin{enumerate}
		\item $(\mathbbm{K}, \times)$ est un groupe abélien
		\item $(\mathbbm{K}, \times)$ est un monoïde commutatif
		\item $\forall x \in \mathbbm{K}\setminus \{0_\mathbbm{K}\}, \exists y \in \mathbbm{K}, xy = 1_\mathbbm{K}$
		\item $0_\mathbbm{K} \neq  1_\mathbbm{K}$
	\end{enumerate}
	\index{corps}
\end{defn}

\begin{exm}
	\begin{itemize}
		\item $(\C, +, \times)$ est un corps
		\item $(\R, +, \times)$ est un corps
		\item $(\Q, +, \times)$ est un corps
		\item $(\Z, +, \times)$ n'est pas un corps
	\end{itemize}
\end{exm}

\begin{prop}
	$(\Z / n\Z, +, \times)$ est un corps si et seulement si $n$ est premier.
\end{prop}

\begin{prv}
	\[
		\left( \Z / n\Z \right)^\times = \left\{ \overline{k}  \mid k \wedge n = 1 \right\}
	\] 
\end{prv}


\begin{prop}
	Tout corps est un anneau intègre.
\end{prop}

\begin{prv}
	Soit $(\mathbbm{K}, +, \times)$ un corps. Soient $(a,b) \in \mathbbm{K}^2$ tel que $a \times b = 0_\mathbbm{K}$.\\
	On suppose $a \neq  0_\mathbbm{K}$. Alors, $a$ est inversible et donc \[
		b = a^{-1} \times a \times b = a^{-1} \times 0_\mathbbm{K} = 0_\mathbbm{K}
	\] 
\end{prv}

\begin{exm}
	Soit $(\mathbbm{K},+,\times)$ un corps.\\
	Résoudre \[
		\begin{cases}
			x^2 = 1_\mathbbm{K}\\
			x \in \mathbbm{K}
		\end{cases}
	\]

	\begin{align*}
		x^2 = 1_\mathbbm{K} &\iff x^2 - 1_\mathbbm{K} = 0_\mathbbm{K}\\
		&\iff (x - 1_\mathbbm{K})(x+1_\mathbbm{K}) = 0_\mathbbm{K}\\
		&\iff x - 1_\mathbbm{K} = 0_\mathbbm{K} \text{ ou } x + 1_\mathbbm{K} = 0_\mathbbm{K}\\
		&\iff x = 1_\mathbbm{K} \text{ ou } x = -1_\mathbbm{K}
	\end{align*}

	Il y a au plus 2 solutions.
\end{exm}

\begin{prop}
	Soit $(\mathbbm{K},+,\times )$ un corps et $P$ un polynôme à coefficients dans $\mathbbm{K}$ de degré $n$. Alors, l'équation $P(x) = 0_{\mathbbm{K}}$ a au plus $n$ solutions dans $\mathbbm{K}$ 
	\qed
\end{prop}

\begin{crlr}[(Théorème de Wilson)]
	voir exercice 16 du TD 12
\end{crlr}


\begin{defn}
	Soit $(\mathbbm{K}, +, \times)$ un corps et $L\subset \mathbbm{K}$.\\
	On dit que $L$ est un \underline{sous corps} de $\mathbbm{K}$ si
	\begin{enumerate}
		\item $L$ est un anneau de $(\mathbbm{K}, +, \times)$ non nul
		\item $\forall x \in L\setminus \{0_\mathbbm{K}\}, x^{-1} \in L$ 
	\end{enumerate}
	\vspace{2mm}
	en d'autres termes si
	\begin{enumerate}
		\item $\forall (x,y) \in L^2, x - y \in L$
		\item $\forall (x,y) \in L^2, x \times y^{-1} \in L$
	\end{enumerate}
	\vspace{5mm}
	On dit aussi que $\mathbbm{K}$ est une \underline{extension} de $L$.
	\index{sous corps}
	\index{extension}
\end{defn}

\begin{prop}
	Tout sous corps est un corps. \qed
\end{prop}

\begin{defn}
	Soient $(\mathbbm{K}_1,+,\times )$ et $(\mathbbm{K}_2,+, \times)$ deux corps et $f: \mathbbm{K}_1 \to \mathbbm{K}_2$.\\
	On dit que $f$ est un \underline{morphisme de corps} si $f$ est un morphisme d'anneaux.\\
	i.e. si
	\[
		\begin{cases}
			\forall (x,y) \in {\mathbbm{K}_1}^2,& f(x+y) = f(x) + f(y)\\
			\forall (x,y) \in {\mathbbm{K}_1}^2,& f(x \times y) = f(x) \times f(y)\\
		\end{cases}
	\] 
	\index{homomorphisme (de corps)}
	\index{morphisme (de corps)}
\end{defn}

\begin{prop}
	Tout morphisme de corps est injectif.
\end{prop}

\begin{prv}
	Soit $f: \mathbbm{K}_1 \to \mathbbm{K}_2$ un morphisme de corps.\\
	\begin{itemize}
		\item $\Ker(f)$ est un sous groupe de $(\mathbbm{K}_1, +)$ 
		\item Soit $x \in \Ker(f)$ et $y \in \mathbbm{K}_1$ \[
				f(x \times y) = f(x) \times f(y) = 0_{\mathbbm{K}_2} \times f(y) = 0_{\mathbbm{K}_2}
			\]
		\item Soit $x \in \Ker(f) \setminus \{0_{\mathbbm{K}_1}\}$.\\
			Alors, $x$ est inversible.\\
			\begin{align*}
				\begin{rcases*}
					x \in \Ker(f)\\
					x^{-1} \in \mathbbm{K}_1
				\end{rcases*}& \text{ donc } x \times x ^{-1} \in \Ker(f)\\
				&\text{ donc } 1_{\mathbbm{K}_1} \in \Ker(f)\\
				&\text{ donc } f(1_{\mathbbm{K}_1}) = 0_{\mathbbm{K}_2}
			\end{align*}
			Or, $f(1_{\mathbbm{K}_1}) = 1_{\mathbbm{K}_2} \neq 0_{\mathbbm{K}_2}$
	\end{itemize}
	Donc, $\Ker(f) = \{0_{\mathbbm{K}_1}\}$ donc $f$ est injective.
\end{prv}

\begin{exm}
	$\begin{array}{cc}
		\C &\longrightarrow \C\\
		z &\longmapsto \overline{z}\\
	\end{array}$ est un morphisme de corps
\end{exm}



		\addrecap
	}

	{
		\chap[24]{Groupe symétrique}
		\renewcommand{\cwd}{../chap24}
		\begin{defn}
	Un \underline{proposition} est un énoncé qui est soit vrai, soit faux.
\end{defn}

\begin{exm}
	\begin{align*}
		\begin{rcases*}
			A: ``B \text{ est vraie }"\\
			B: ``A \text{ est fausse }"\\
		\end{rcases*} \text{ Le système $\{A,B\}$ est une \underline{auto-contradiction}}
	\end{align*}
\end{exm}

\begin{defn}
	\underline{Démontrer} une proposition revient à prouver qu'elle est vraie
\end{defn}

		\begin{defn}
	Soit $E$ un $\mathbbm{K}$-espace vectoriel. On dit que $E$ est de \underline{dimension finie} si $E$ a au moins une famille génératrice finie. On dit que $E$ est de \underline{dimension infinie} sinon.
	\index{dimension finie (espace vectoriel)}
	\index{dimension infinie (espace vectoriel)}
\end{defn}

\begin{thm}
	[Théorème de la base extraite]
	Soit $E$ un $\mathbbm{K}$-espace vectoriel non nul de dimension finie. Soit $\mathcal{G}$ une famille génératrice finie de $E$. Alors, il existe une base $\mathcal{B}$ de $\mathcal{E}$ telle que $\mathcal{B} \subset \mathcal{G}$.
\end{thm}

\begin{prv}
	[par récurrence sur $\#G = \Card(G)$]
	\begin{itemize}
		\item Soit $E$ un $\mathbbm{K}$-espace vectoriel non nul engendré par $\mathcal{G} = (u)$.\\
			Si $u = 0_E$, alors $E = \{0_E\}$: une contradiction $\lightning$ \\
			Donc $u \neq 0_E$ donc $(u)$ est libre. En effet, \[
				\forall \lambda \in \mathbbm{K}, \lambda u = 0_E \implies \lambda = 0_\mathbbm{K}
			\] Donc $\mathcal{G}$ est une base de $E$.\\
		\item Soit $n \in \N_*$. Soit $E$ un $\mathbbm{K}$-espace vectoriel. On suppose que si $E$ a une famille génératrice constituée de $n$ vecteurs, alors on peut extraire de cette famille une base de $E$.\\
			Soit $\mathcal{G}$ une famille génératrice de $E$ avec $n+1$ vecteurs.\\
			Si $\mathcal{G}$ est libre, alors $\mathcal{G}$ est une base de $E$. \\
			Si $\mathcal{G}$ n'est pas libre, alors il existe $u \in \mathcal{G}$ tel que $u \in \Vect(\mathcal{G}\setminus \{u\})$ \\
			Donc $\mathcal{G}\setminus \{u\}$ engendre $E$. Or, $\mathcal{G}\setminus \{u\}$ possède $n$ vecteurs. D'après l'hypothèse de récurrence, il existe une base $\mathcal{B}$ de $E$ telle que \[
				\mathcal{B} \subset \mathcal{G} \setminus \{u\} \subset \mathcal{G}
			\] 
	\end{itemize}
\end{prv}

\begin{crlr}
	Tout espace de dimension finie a une base.
	\qed
\end{crlr}

\begin{thm}
	[Théorème de la base incomplète]
	Soit $E$ un $\mathbbm{K}$-espace vectoriel de dimension finie, $\mathcal{G}$ une famille génératrice finie de $E$. $\mathcal{L}$ une famille libre de $E$. Alors, il existe une base $\mathcal{B}$ de $E$ telle que \[
		\mathcal{L} \subset \mathcal{B} \text{ et } \mathcal{B}\setminus \mathcal{L} \subset \mathcal{G}
	\] 
\end{thm}

\begin{prv}
	[par récurrence sur $\#(\mathcal{G}\setminus\mathcal{L})$]
	\begin{itemize}
		\item Avec les notations précédentes, on suppose que $\mathcal{G}\setminus\mathcal{L} \neq \O$ \[
				\forall u \in \mathcal{G}, u \in \mathcal{L}
			\] Donc $\mathcal{G} \subset \mathcal{L}$ donc $\mathcal{L}$ est génératrice donc $\mathcal{L}$ est une base de $E$. On pose $\mathcal{B} = \mathcal{L}$ et alors \[
				\mathcal{L} \subset  \mathcal{B} \text{ et } \mathcal{B}\setminus\mathcal{L} = \O \subset  \mathcal{G}
			\] 
		\item Soit $n \in \N$. On suppose que si $\mathcal{G}$ est génératrice et $\mathcal{L}$ libre avec $\#(\mathcal{G}\setminus\mathcal{L}) = n$ alors il existe une base $\mathcal{B}$ de $E$ telle que \[
			\mathcal{L}\subset \mathcal{B} \text{ et } \mathcal{B}\setminus\mathcal{L}\subset \mathcal{G}
		\] Soient à présent $\mathcal{G}$ une famille génératrice de $E$ et $\mathcal{L}$ une famille libre de $E$ telles que $\#(\mathcal{G}\setminus\mathcal{L}) = n+1 > 0$\\
		Si $\mathcal{L}$ engendre $E$, alors $\mathcal{L}$ est une base de $E$. On pose $\mathcal{B} = \mathcal{L}$ et on a bien \[
			\mathcal{L} \subset  \mathcal{B} \text{ et } \mathcal{B} \setminus \mathcal{L} = \O \subset  \mathcal{G}
		\] On suppose que $\mathcal{L}$ n'engendre pas $E$. Il existe $u \in \mathcal{G}$ tel que $u \not\in \Vec(\mathcal{L})$ (car sinon, $\mathcal{G} \subset \Vect(\mathcal{L})$ et donc $\underbrace{\Vect(\mathcal{G})}_{= E} \subset  \underbrace{\Vect(\mathcal{L})}_{ \subset E}$\\
		Donc $\mathcal{L} \cup \{u\} $ est libre. On pose $\mathcal{L}' = \mathcal{L} \cup \{u\} $ \[
			\mathcal{G}\setminus \mathcal{L}' = \mathcal{G}\setminus (\mathcal{L} \cup \{u\}) = (\mathcal{G}\setminus\mathcal{L})\setminus \{u\} 
		\] donc $\#(\mathcal{G}\setminus\mathcal{L}') = n+1 -1 = n$\\
		D'après l'hypothèse de récurrence, il existe $\mathcal{B}$ une base de $E$ telle que \[
			\mathcal{L} \subset  \mathcal{L}' \subset \mathcal{B} \text{ et } \mathcal{B}\setminus \mathcal{L}' \subset \mathcal{G}
		\] \[
			\mathcal{B} \setminus \mathcal{L} = \underbrace{\mathcal{B}\setminus\mathcal{L}'}_{\subset \mathcal{G}} \cup \underbrace{\{u\}}_{\subset \mathcal{G} \text{ car } u \in \mathcal{G}}
		\] On a $\mathcal{B}\setminus\mathcal{L}\subset \mathcal{G}$
	\end{itemize}
\end{prv}

\begin{thm}
	Soit $E$ un $\mathbbm{K}$-espace vectoriel de dimension finie. Toutes les bases de $E$ ont le même cardinal.
\end{thm}

\begin{prv}
	Soit $\mathcal{G}$ une famille génératrice finie de $E$ et $\mathcal{B} \subset  \mathcal{G}$ une base de $E$. On note $n = \#\mathcal{B}$ \\
	Soit $\mathcal{B}'$ une base de $E$. On pose $p = n - \#(\mathcal{B} \cap  \mathcal{B}')$. Montrons par récurrence sur  $p$ que $\#\mathcal{B} = \#\mathcal{B}'$ 
	\begin{itemize}
		\item On suppose que $p = 0$. Alors, $\#(\mathcal{B} \cap \mathcal{B}') = n$ \\
			Or, $\mathcal{B}' \cap \mathcal{B} \subset \mathcal{B}$ donc $\mathcal{B} \cap \mathcal{B}' = \mathcal{B}$ donc $\mathcal{B} \subset  \mathcal{B}'$ et donc $\mathcal{B} = \mathcal{B}'$ 
		\item Soit $p \in \N$. On suppose que si $\mathcal{B}'$ est une base de $E$ telle que $n - \#(\mathcal{B} \cap \mathcal{B}') = p$, alors $\#\mathcal{B}' = n$ \\
			Aoit $\mathcal{B}'$ une base de $E$ telle que $n - \#(\mathcal{B}\cap \mathcal{B}') = p+1 > 0$ \\
			Donc $\mathcal{B} \cap \mathcal{B}' \neq \mathcal{B}$. Soit $u \in \mathcal{B}' \setminus \mathcal{B}$. D'après le lemme d'échange, il existe $v \in \mathcal{B}\setminus \mathcal{B}'$ tel que $\mathcal{B}' \setminus \{u\} \cup \{v\}$ est une base de $E$. On pose $\mathcal{B}'' = \mathcal{B}' \setminus \{u\} \cup \{v\}$ 
			\begin{align*}
				\mathcal{B}'' \cap \mathcal{B} &= \left( (\mathcal{B}' \setminus \{u\})  \cap \mathcal{B} \right) \cup \{v\} \\
				&= (\mathcal{B}' \cap \mathcal{B}) \cup \{v\} \\
			\end{align*}
			donc,
			\begin{align*}
				n - \#(\mathcal{B}'' \cap \mathcal{B}) &= n - (\#(\mathcal{B}' \cap \mathcal{B}) + 1) \\
				&= p+1- 1 \\
				&= p \\
			\end{align*}
			D'après l'hypothèse de récurrence, \[
				\#\mathcal{B}'' = n
			\] Or, $\#\mathcal{B}'' = \#\mathcal{B}'$
	\end{itemize}
\end{prv}

\begin{lem}
	Soient $\mathcal{B}$ et $\mathcal{B}'$ deux bases de $E$ telles que $\mathcal{B}\subset \mathcal{B}'$. Alors, $\mathcal{B} = \mathcal{B}'$.
\end{lem}

\begin{prv}
	On suppose $\mathcal{B}' \neq \mathcal{B}$. Soit $u \in \mathcal{B}' \setminus \mathcal{B}$
	$u \in E = \Vect(\mathcal{B})$ donc $\mathcal{B} \cup \{u\}$ n'est pas libre.
	Donc $\mathcal{B}\cup \{u\} \subset \mathcal{B}'$ et $\mathcal{B}'$ est libre donc $\mathcal{B}\cup \{u\}$ est libre: une contradiction $\lightning$
\end{prv}

\begin{lem}
	[Lemme d'échange] Soient $\mathcal{B}_1$ et $\mathcal{B}_2$ deux bases de $E$ et $u \in \mathcal{B}_1 \setminus \mathcal{B}_2$. Alors, il existe $v \in \mathcal{B}_2$ tel que $(\mathcal{B}_1 \setminus \{u\}) \cup \{v\}$ soit une base de $E$.
\end{lem}

\begin{prv}
	[1${}^\text{nde}$ méthode]
	On suppose que pout tout $v \in \mathcal{B}_2$, $(\mathcal{B}_1\setminus \{u\}) \cup \{v\}$ n'est pas une base de $E$
	Soit $v \in \mathcal{B}_2$.
	\begin{itemize}
		\item Supposons $(\mathcal{B}_1\setminus \{u\})\cup \{v\}$ non libre. $\mathcal{B}_1 \setminus \{u\}$ est libre. Donc $v \in \Vect(\mathcal{B}_1 \setminus \{u\})$
		\item Supposons $(\mathcal{B}_1\setminus \{u\}) \cup \{v\}$ non génératrice.
			Comme $\mathcal{B}_1$ engendre $E$, $u \not\in \Vect(\mathcal{B}_1\setminus \{v\})$.
			On suppose que $\mathcal{B}_1 \neq \mathcal{B}_2$.
			$\forall v \in \mathcal{B}_2 \setminus \mathcal{B}_1, \Vect(\mathcal{B}_1 \setminus \{v\}) = \Vect(\mathcal{B}_1) = E \ni u$ 
			donc, $(\mathcal{B}_1\setminus \{u\}) \cup \{v\}$ engendre $E$ et donc \[
				v \in \Vect(\mathcal{B}_1 \setminus \{u\})
			\] On a aussi \[
				\forall v \in \mathcal{B}_1 \setminus \{u\}, v \in \Vect(\mathcal{B}_1\setminus \{u\})
			\] Comme $u \not\in \mathcal{B}_2$, on a \[
				\forall v \in \mathcal{B}_2, v \in \Vect(\mathcal{B}_1\setminus \{u\})
			\] docn \[
				E = \Vect(\mathcal{B}_2) \subset \Vect(\mathcal{B}_1\setminus \{u\})
			\] donc $\mathcal{B}_1\setminus \{u\}$ engendre $E$ donc $\mathcal{B}_1\setminus \{u\}$ est une base de $E$. Or, $\mathcal{B}_1 \setminus \{u\}  \subset  \mathcal{B}_1$, donc $\mathcal{B}_1\setminus \{u\} = \mathcal{B}_1$
	\end{itemize}
\end{prv}

\begin{prv}
	[2${}^\text{nde}$ méthode]
	On suppose que pout tout $v \in \mathcal{B}_2$, $(\mathcal{B}_1\setminus \{u\}) \cup \{v\}$ n'est pas une base de $E$
	\begin{itemize}
		\item Comme $u \in \mathcal{B}_1 \setminus \mathcal{B}_2$, nécéssairement $\mathcal{B}_1 \neq \mathcal{B}_2$ donc $\mathcal{B}_2 \not\subset \mathcal{B}_1$, donc $\mathcal{B}_2\setminus\mathcal{B}_1 \neq \O$ 
		\item Soit $v \in \mathcal{B}_2\setminus\mathcal{B}_1$. Il existe $(\lambda_w)_{w\in\mathcal{B}_1}$ une famille de scalaires presque nulle telle que \[
				v = \sum_{w \in \mathcal{B}_1} \lambda_w w - \lambda_u u + + \sum_{w \in \mathcal{B}_1\setminus \{u\}}\lambda_w w
			\]
			Si $\lambda_u \neq 0_E$, alors
			\begin{align*}
				u &= \lambda_u^{-1}\left( v - \sum_{w \in \mathcal{B}_1 \setminus \{u\}} \lambda_w w \right)\\
					&\in \Vect(\mathcal{B}_1\setminus \{u\} \cup v)
			\end{align*}
			 donc $\mathcal{B}_1 \subset \Vect(\mathcal{B}_1\setminus \{u\} \cup \{v\})$\\
			 et donc $E \subset  \Vect(\mathcal{B}_1 \setminus \{u\} \cup \{v\})$ \\
			 et donc $\mathcal{B}_1 \setminus \{u\} \cup \{v\}$ engendre $E$ \\
			 donc $\mathcal{B}_1 \setminus \{u\} \cup \{v\}$ n'est pas libre\\
			 donc $v \in \Vect(\mathcal{B}_1\setminus \{u\})$ (car $\mathcal{B}_1 \setminus \{u\}$ est libre\\
			 donc $\lambda_u = 0_\mathbbm{K}$ $\lightning$\\`

			 Donc, $\lambda_u = 0_\mathbbm{K}$, docn $v \in \Vect(\mathcal{B}_1\setminus \{u\})$ \\
			 On vient de prouver que
			 \begin{align*}
			 	\mathcal{B}_2 \setminus \mathcal{B}_1 \subset \Vect(\mathcal{B}_1 \setminus \{u\})\\
			 	\mathcal{B}_1 \setminus \{u\} \subset \Vect(\mathcal{B}_1 \setminus \{u\})\\
			 \end{align*}
			 Comme $u \not\in \mathcal{B}_2$, \[
			 	\mathcal{B}_2 \subset \Vect(\mathcal{B}_1 \setminus \{u\})
			 \] donc \[
			 	E = \Vect(\mathcal{B}_2) \subset  \Vect(\mathcal{B}_1 \setminus \{u\})
			 \] donc $\mathcal{B}_1 \setminus \{u\}$ engendre $E$. Donc,  $\mathcal{B}_1 \setminus \{u\}$ est une base de $E$.\\
			 Or, $\mathcal{B}_1 \setminus \{u\} \subset  \mathcal{B}_1$, donc $\mathcal{B}_1 \setminus \{u\} = \mathcal{B}_1$
	\end{itemize}
\end{prv}

\begin{defn}
	Soit $E$ un $\mathbbm{K}$-espace vectoriel de dimension finie. Le cardinal commun à toutes les bases de $E$ est appelé \underline{dimension} de $E$ est notée $\dim(E)$ ou $\dim_\mathbbm{K}(E)$\\
	C'est donc aussi le nombre de coordonnées de n'importe quel vecteur dans n'importe quelle base.
	\index{dimension (espace vectoriel)}
\end{defn}

\begin{exm}
	\begin{enumerate}
		\item $\dim_\R(\C) = 2$ et $\dim_\C(\C) = 1$ 
		\item $\dim_\mathbbm{K}(\mathbbm{K}^{n}) = n$ 
		\item $\dim_{\mathbbm{K}}(\mathcal{M}_{n,p}(\mathbbm{K})) = np$
	\end{enumerate}
\end{exm}

\begin{crlr}
	Soit $E$ un $\mathbbm{K}$-espace vectoriel de dimension finie, $\mathcal{L}$ une famille libre de $E$, $\mathcal{G}$ une famille génératrice de $E$. On note $n = \dim(E)$
	\begin{enumerate}
		\item $\#\mathcal{G} \ge n$ et $(\#\mathcal{G} = n \implies \mathcal{G} \text{ est une base de } E$)
		\item $\#\mathcal{L} \le n$ et $(\#\mathcal{L} = n \implies \mathcal{L} \text{ est une base de } E$)
	\end{enumerate}
\end{crlr}

\begin{crlr}
	$\R^{\R}$ est de dimension infinie.
	$\forall i \in \N, e_i: x \mapsto x^i$\\
	$(e_i)_{i\in\N}$ est libre dans $\R^\R$
\end{crlr}

\begin{prop}
	Soient $E$ et $F$ deux $\mathbbm{K}$-espaces vectoriels de dimension finie. Alors $E\times F$ est de dimension finie et $\dim(E\times F) = \dim(E) + \dim(F)$
\end{prop}

\begin{prv}
	Soit $(e_1,\ldots, e_n)$ une base de $E$, $(f_1, \ldots, f_p)$ une base de $F$.
	On pose \[
		\left\{\begin{array}
			{r c l}
			u_1 &=& (e_1,0_F)\\
			u_2 &=& (e_2,0_F)\\
					&\vdots&\\
			u_n &=& (e_n,0_F)\\
			u_{n+1} &=& (0_E, f_1)\\
			u_{n+2} &=& (0_E, f_2)\\
					&\vdots&\\
			u_{n+p} &=& (0_E,f_p)\\
		\end{array}\right.
	\]
	Soit $(x,y) \in E\times F$. \[
		\begin{cases}
			\exists (x_1,\ldots,x_n)\in \mathbbm{K}^n, x = \sum_{i=1}^{n} x_ie_i
			\exists (y_1,\ldots,y_n)\in \mathbbm{K}^n, x = \sum_{j=1}^{p} y_jf_j
		\end{cases}
	\] 
	\begin{align*}
		(x,y) &= \left( \sum_{i=1}^{n} x_ie_i, \sum_{i=1}^{p} y_jf_j \right)  \\
		&= \sum_{i=1}^{n} x_i (e_i + 0_F) + \sum_{j=1}^{p} y_j (0_E, f_j) \\
		&= \sum_{i=1}^{n} x_i u_i + \sum_{j=1}^{p} y_j u_{n+j} \\
	\end{align*}
	Donc, $E\times F = \Vect(u_1, \ldots, u_{n+p})$ donc $E\times F$ est de dimension finie.\\
	Soit $(\lambda_1, \ldots, \lambda_{n+p}) \in \mathbbm{K}^{n+p}$ tel que \[
		(*): \quad \sum_{k=1}^{n+p} \lambda_ku_k = 0_{E\times F} = (0_E, 0_F)
	\]
	\begin{align*}
		(*) &\iff \sum_{k=1}^{n} \lambda_k (e_k, 0_F) + \sum_{k=n+1}^{p} \lambda_k(0_E, f_{k-n}) = (0_E, 0_F)\\
				&\iff \begin{cases}
					\sum_{k=1}^{n} \lambda_k e_k = 0_E\\
					\sum_{k=n+1}^{p} \lambda_k f_{k-n} = 0_F
				\end{cases}\\
				&\iff \begin{cases}
					\forall k \in \left\llbracket 1,n \right\rrbracket, \lambda_k = 0_\mathbbm{K} \qquad&(\text{car $(e_1,\ldots,e_n)$ est libre})\\
					\forall k \in \left\llbracket n+1,n+p \right\rrbracket, \lambda_k = 0_\mathbbm{K} \qquad&(\text{car $(f_1,\ldots,f_n)$ est libre})\\
				\end{cases}
	\end{align*}
	Donc $(u_1, \ldots, u_{n+p})$ est une base de $E\times F$. Donc, $\dim(E\times F) = n + p = \dim(E) + \dim(F)$
\end{prv}

\begin{rmk}
	[Convention]
	\[\dim\big(\{0_E\}\big) = 0\]
\end{rmk}

\begin{thm}
	Soit $E$ un $\mathbbm{K}$-espace vectoriel de dimension finie, $F$ un sous-espace vectoriel de $E$. Alors, $F$ est de dimension finie et  $\dim(F) \le \dim(E)$\\
	Si $\dim(F) = \dim(E)$, alors $F = E$
\end{thm}

\begin{prv}
	On considère \[
		A = \{k \in \N \mid \text{il existe une famille libre de $F$ à $k$ éléments}\} 
	\]
	On suppose $F \neq \{0_E\}$.
	\begin{itemize}
		\item Soit $u \in F\setminus \{0_E\}$. $(u)$ est libre donc $1 \in A$ et donc $A \neq \O$
		\item Soit $\mathcal{L}$ une famille libre de $F$. Alors, $\mathcal{L}$ est une famille libre de $E$ \\
			donc $\#\mathcal{L} \le \dim(E)$\\
			Donc $A$ est majorée par $\dim(E)$ \\
			On en déduit que $A$ a un plus grand élément $p$.
		\item Soit $\mathcal{L}$ une famille libre de $F$ avec $p$ éléments.\\
			Si $\mathcal{L}$ n'engendre pas $F$, alors il existe $u\in F$ tel que $u\not\in \Vect(\mathcal{L})$ et donc $\mathcal{L} \cup \{u\}$ est une famille libre de $F$, donc $p+1 \in A$ en contradiction avec la maximalité de $p$.\\
			Donc $\mathcal{L}$ est une base de $F$ donc $F$ est de dimension finie et $\dim(F) = p \le \dim(E)$\\
	\end{itemize}

	Soit $\mathcal{B}$ une base de $F$. Alors, $\mathcal{B}$ est aussi une famille de libre de de $E$. Donc $\#\mathcal{B} \le \dim(E)$ donc $\dim(F) = \dim(E)$ \\
	Si $\dim(F) = \dim(E)$, alors $\mathcal{B}$ est une base de $E$, et donc $F = \Vect(\mathcal{B}) = E$
\end{prv}

\begin{prop}
	[Formule de Grassmann]
	Soit $E$ un $\mathbbm{K}$-espace vectoriel de dimension finie, $F$ et $G$ deux sous-espace vectoriels de $E$. Alors, \[
		\dim(F+G) = \dim(F) + \dim(G) - \dim(F\cap G)
	\] 
\end{prop}

\begin{prv}
	Soit $(e_1, \ldots, e_p)$ une base de $F\cap G$. $(e_1,\ldots,e_p)$ est une famille libre de $F$.\\
	On complète $(e_1, \ldots, e_p)$ en une base $(e_1, \ldots, e_p, u_1, \ldots, u_q)$ de $F$.\\
	De même, on complète $(e_1, \ldots, e_p)$ en une base $(e_1, \ldots, e_p, v_1, \ldots, v_r)$ de $G$.\\
	On pose  $\mathcal{B} = (e_1, \ldots, e_p, u_1, \ldots, u_q, v_1, \ldots, v_r)$. Montrons que $\mathcal{B}$ est une base de $F+G$
	\begin{itemize}
		\item Soit $u \in F+G$ \\
			On pose $u = v+w$ avec $\begin{cases}
				v\in F\\
				w \in G
			\end{cases}$.\\
			On pose $v = \sum_{i=1}^p \lambda_i e_i + \sum_{i=1}^q \mu_i u_i$ avec $(\lambda_1, \ldots, \lambda_p, \mu_1, \ldots, \lambda_q) \in \mathbbm{K}^{p+q}$\\
			On pose aussi $w = \sum_{i = 1}^p \lambda'_ie_i + \sum_{j=1}^r \nu_j v_j$ avec $(\lambda_1',\ldots,\lambda_p', \nu_1, \ldots, \nu_r) \in \mathbbm{K}^{p+r}$\\
			D'où, \[
				u = \sum_{i=1}^p (\lambda_i + \lambda'_i)e_i + \sum_{j=1}^q \mu_j u_j + \sum_{k=1}^r \nu_k v_k \in \Vect(\mathcal{B})
			\]
		\item Soient $(\lambda_1, \ldots, \lambda_p, \mu_1, \ldots, \mu_q, \nu_1, \ldots, \nu_r) \in \mathbbm{K}^{p+q+r}$.\\
			On suppose \[
				(*)\quad \sum_{i=1}^{p}\lambda_ie_i + \sum_{j=1}^q\mu_ju_j + \sum_{k=1}^r \nu_k v_k = 0_E
			\] 
			D'où, \[
				\underbrace{\sum_{i=1}^p\lambda_i e_i + \sum_{j=1}^q \mu_ju_j}_{\in F} = \underbrace{-\sum_{k=1}^r\nu_jv_k}_{\in G}
			\] 
			Donc, \[
				f = \sum_{i=1}^p \lambda_i e_i + \sum_{j=1}^q \mu_j u_j \in F\cap G
			\] Comme $(e_1, \ldots, e_p)$ est une base de $F\cap G$, $\exists ! (\lambda_1', \ldots, \lambda_p') \in \mathbbm{K}^p$ tel que \[
				f = \sum_{i=1}^p \lambda'_i e_i = \sum_{i=1}^p \lambda'_i e_i + \sum_{j=1}^q 0_\mathbbm{K}u_j
			\] Comme $(e_1, \ldots, e_p, u_1, \ldots, u_q)$ est une base de $F$, \[
				\forall k \in \left\llbracket 1, q \right\rrbracket, \mu_j = 0_\mathbbm{K}
			\] De même, \[
				\forall k \in \left\llbracket 1,r \right\rrbracket , \nu_k = 0_\mathbbm{K}
			\] On remplace dans $(*)$ pour trouver \[
				\sum_{i=1}^p \lambda_ie_i = 0_E
			\] Comme $(e_1, \ldots, e_p)$ est libre, \[
				\forall i \in \left\llbracket 1,p \right\rrbracket, \lambda_i = 0_\mathbbm{K}
			\] Donc $\mathcal{B}$ est libre.\\
			Donc, 
			\begin{align*}
				\dim(F+G) &=  p +q + r \\
				&= (p+q)+ (p+r) - p \\
				&= \dim(F) + \dim(G) - \dim(F\cap G) \\
			\end{align*}
	\end{itemize}
\end{prv}

\begin{crlr}
	Avec les hypothèse précédentes, \[
		E = F \oplus G \iff \begin{cases}
			F \cap  G = \{0_E\} \\
			\dim(E) = \dim(F) + \dim(G)
		\end{cases}
	\] 
\end{crlr}

\begin{prv}
	\begin{itemize}
		\item[``$\implies$''] On suppose $E = F \oplus G$ \\
			Comme la somme est directe, $F \cap G = \{0_E\}$ 
			\begin{align*}
				\dim(E) &= \dim(F)\\
				&= \dim(F) + \dim(G) - \dim(F\cap G)\\
				&= \dim(F) + \dim(G)\\
			\end{align*}
		\item[``$\impliedby$''] On suppose $F\cap G = \{0_E\}$ et $\dim(E) = \dim(F) + \dim(G)$.\\
			On sait déjà que $F+G = F \oplus G$\\
			 \begin{align*}
				\dim(F+G) = \dim(F) + \dim(G) - \dim(F \cap G) = \dim(E)
			\end{align*}
			Donc $F + G = E$
	\end{itemize}
\end{prv}

\begin{prop}
	Soit $F$ un $\mathbbm{K}$-espace vectoriel de dimension finie $n$. Soit $\mathcal{B} = (e_1, \ldots, e_n)$ une base de $F$. L'application
	\begin{align*}
		f: \mathbbm{K}^n &\longrightarrow F \\
		(\lambda_1, \ldots, \lambda_n) &\longmapsto \sum_{i=1}^n \lambda_i e_i
	\end{align*} est bijective.\\
	Si $\mathbbm{K}$ est infini, $\mathbbm{K}^n$ aussi et donc $F$ aussi.\\
	Si $\#\mathbbm{K} = p \in \N_*$,
	\begin{align*}
		\#&\mathbbm{K}^n = p^n\\
		&\vrt=\\
		\#&F
	\end{align*}
\end{prop}


		\part{Dérivation}

\underline{Motivation}:

{
\begin{wrapfigure}{l}{3cm}
	\centering
	\begin{asy}
		import three;

		size(3cm);
		settings.render=0;
		settings.prc=false;
		currentprojection = obliqueZ;

		draw(unitbox);
		draw(shift(1.1Z + 0.05X) * (O -- X), Arrows3(TeXHead2));
		draw(shift(1.1Z + 0.05Y) * (O -- Y), Arrows3(TeXHead2));
		draw(shift(1.1X + 0.05Z) * (O -- Z), Arrows3(TeXHead2));

		label("$x$", (X/2) + (1.1Z + 0.05X), align=S);
		label("$y$", (Y/2) + (1.1Z + 0.05Y), align=W);
		label("$z$", (Z/2) + X, align=SE);
	\end{asy}
\end{wrapfigure}

\begin{align*}
	&S(x,y,z) = 2(xy + xz + yz)\\
	&V(x,y,z) = xyz
\end{align*}

On cherche à minimiser $S$ avec la contrainte $V = 1$.

Soit $f : \begin{array}{rcl}
	\left( \R_*^+ \right)^2 &\longrightarrow& \R \\
	(x,y) &\longmapsto& S\left( x,y,\frac{1}{xy} \right) = 2\left( xy + \frac{1}{y} + \frac{1}{x} \right).
\end{array}$

On cherche $(a,b) \in \left( \R^+_* \right)^2$ tel que \[
	\forall (x,y) \in (\R^+_*), f(x,y) \ge f(a,b).
\]
}

\begin{defn}
	Soit $f: U \to \R$ où $U$ est un ouvert de $\R^2$. Soit $(a,b) \in U$.
	\vspace{2mm}

	Si $\lim_{x \to a} \frac{f(x,b) - f(a,b)}{x - a} \in \R$, alors on dit que $f$ a une dérivée partielle suivant $x$ en $(a,b)$ et cette limite est notée \[
		\partial f_1(a,b) = \frac{\partial f}{\partial x}(a,b).
	\]

	Si $\lim_{y \to b} \frac{f(a,y) - f(a,b)}{y - b} \in \R$, alors on dit que $f$ a une dérivée partielle suivant $y$ et la limite est notée \[
		\partial f_2(a,b) = \frac{\partial f}{\partial y}(a,b).
	\]
\end{defn}

\begin{exm}
	\begin{enumerate}
		\item $f: (x,y) \mapsto xy + x - y$.

			\begin{align*}
				&\frac{\partial f}{\partial x} : (x,y) \mapsto y + 1,\\
				&\frac{\partial f}{\partial y} : (x,y) \mapsto x - 1.
			\end{align*}

		\item $f: (x,y) \mapsto xy + \frac{1}{y}+ \frac{1}{x}$.

			\begin{align*}
				&\frac{\partial f}{\partial x}: (x,y) \mapsto y - \frac{1}{x^2},\\
				&\frac{\partial f}{\partial y}: (x,y) \mapsto x - \frac{1}{y^2}.
			\end{align*}

		\item Trouver $f$ telle que $\begin{cases}
				(1): \qquad \frac{\partial f}{\partial x}=y,\\[2mm]
				(2): \qquad \frac{\partial f}{\partial y} = x.
			\end{cases}$

			D'après $(1)$ : \[
				\forall (x,y), \exists C(y) \in \R, f(x,y) = xy + C(y)
			\] et donc \[
				\frac{\partial f}{\partial y}(x,y) = x + C'(y)
			\] donc $C'(y) = 0$ et donc $C$ est constante.

		\item Trouver $f$ telle que $\begin{cases}
			\frac{\partial f}{\partial x} = -y,\\[2mm]
			\frac{\partial f}{ƒ\partial y} = x.
		\end{cases}$

		Ce n'est pas possible !
	\end{enumerate}
\end{exm}

\begin{defn}~\\
	\begin{minipage}{\linewidth}
		\begin{wrapfigure}{r}{4cm}
			\centering
			\vspace{-5mm}
			\begin{asy}
				import three;
				import graph3;
				size(4cm);

				settings.render = 0;
				settings.prc = false;
				currentprojection = obliqueX;

				draw(O -- X, Arrow3(TeXHead2));
				draw(O -- Y, Arrow3(TeXHead2));
				draw(O -- Z, Arrow3(TeXHead2));

				triple f(real x, real y, real z = 0) { return (x,y,cos(x - 0.5) * cos(y - 0.5)/1.2 + 0.15); }

				real inc = 1 / 5;

				for(real x = 0; x <= 1; x += inc) {
					draw(graph(
						new real(real t) { return x; }, // x
						new real(real y) { return y; }, // y
						new real(real y) { return f(x,y).z; }, // z
						0, 1
					), gray);
				}

				for(real y = 0; y <= 1; y += inc) {
					draw(graph(
						new real(real x) { return x; }, // x
						new real(real t) { return y; }, // y
						new real(real x) { return f(x,y).z; }, // z
						0, 1
					), gray);
				}

				path3 path1 = (0.8, 0.2, 0) .. (0.5, 0.5, 0) .. (0.3, 0.7, 0);
				path3 path2 = f(0.8, 0.2, 0) .. f(0.5, 0.5, 0) .. f(0.3, 0.7, 0);
				path3 d = (0.2, 0.3, 0) .. (0.3, 0.4, 0) .. (0.2, 0.7, 0) .. (0.8, 0.9, 0) .. (0.6, 0.2, 0) .. cycle;

				draw(path1, red, Arrow3(TeXHead2));
				draw(path2, red, Arrow3(TeXHead2, position=0.8));

				dot((0.5, 0.5, 0));
				dot(f(0.5, 0.5, 0));
				draw((0.5, 0.5, 0) -- f(0.5, 0.5, 0), dashed);
				draw(d);

				label("$w$", (0.3, 0.7, 0), red, align=SE);
				label("$U$", (0.8, 0.9, 0), align=SE);
			\end{asy}
		\end{wrapfigure}

		Soit $f: U \to \R$ où $U$ est un ouvert. Soit $(a,b) \in U$. Soit $w = (w_1, w_2) \in \R^2$.

		Si 
		\[
			\lim_{t\to 0} \frac{f(a + tw_1, b + tw_2) - f(a,b)}{t}
		\] existe et est réelle, alors on dit que $f$ a une dérivée dans la direction de $w$ et la limite est notée \[
			\mathrm{d}f(w)\,(a,b) = D_w(f)\,(a,b).
		\]
	\end{minipage}
\end{defn}

\begin{exm}
	\begin{align*}
		f: \left( \R_*^+ \right)^2 &\longrightarrow \R \\
		(x,y) &\longmapsto xy+\frac{1}{x}+\frac{1}{y}.
	\end{align*}

	On pose $(a,b) = (1,2)$, $w = (w_1, w_2) = (1,1)$.
	\begin{align*}
		\frac{f(1+t, 2+t) - f(1,2)}{t} &= \frac{1}{t} \left( (1+t)(2+t) + \frac{1}{1+t} + \frac{1}{2+t} - 3 - \frac{1}{2} \right) \\
		&= \frac{1}{t}\left(\cancel 2 + 3t + \po(t) + \cancel 1 - t + \po(t) + \frac{1}{2}\left( \cancel 1 - \frac{t}{2} + \po(t) \right) - \cancel3 - \cancel{\frac{1}{2}} \right) \\
		&= \frac{1}{t} \left( \frac{7}{4} t + \po(t) \right)  \\
		&= \frac{7}{4} + \po(1) \tendsto{t \to 0} \frac{7}{4}. \\
	\end{align*}

	Donc, \[
		\mathrm{d}f(1,1)\,(1,2) = \frac{7}{4}.
	\]
\end{exm}

\begin{rmk}~\\
	\begin{figure}[H]
		\centering
		\begin{asy}
			import solids;
			import graph;
			size(5cm);

			settings.render = 0;
			settings.prc = false;

			path3 par = graph(
				new real(real x) { return x; },
				new real(real x) { return 0; },
				new real(real x) { return x^2; },
				0,3);
			revolution r = revolution(par, axis=Z);

			path3 par2 = graph(
				new real(real x) { return x; },
				new real(real x) { return 0; },
				new real(real x) { return x^2; },
				-3,3);

			draw(r,1,longitudinalpen=nullpen);
			draw(r.silhouette());

			draw((-4, 0, -1) -- (-4, 0, 10) -- (4, 0, 10) -- (4, 0, -1) -- cycle, red);
			draw(par2, deepred);

			draw((4,4.5) -- (7, 4.5), black+0.5mm, Arrow(TeXHead));

			path par2d = graph(new real(real x) { return x^2; }, -3, 3);
			draw(shift((11, 0)) * par2d, deepred);

			dot(O);
			dot((11, 0));
		\end{asy}
	\end{figure}
\end{rmk}


%todo ajouter théorème-définition
\begin{thm}
	Soit $f : U \to \R$, $(a,b) \in U$. On suppose que $\frac{\partial f}{\partial x}$ et $\frac{\partial f}{\partial y}$ existent en $(a,b)$ et sont {\bfseries continues} en $(a,b)$. Alors,
	\begin{align*}
		&\forall (h, k) \in \R^2 \text{ tel que } (a +h, b + k) \in U,\\
		&f(a+ h, b + k) = f(a,b) + h \frac{\partial f}{\partial x}(a,b) + k \frac{\partial f}{\partial y}(a,b) + \po_{(h,k)\to (0,0)}\big(\|(h,k)\|\big).
	\end{align*}

	On dit que $f$ est de classe $\mathcal{C}^1$ si $\frac{\partial f}{\partial x}$ et $\frac{\partial f}{\partial y}$ existent et sont continues.

	\qed
\end{thm}

\begin{rmk}
	En physique, cette formule correspond à : \[
		\mathrm{d}f = \frac{\partial f}{\partial x}\mathrm{d}x + \frac{\partial f}{\partial y} \mathrm{d}y.
	\] En effet :
	\begin{align*}
		\mathrm{d}f &= f(x+ \mathrm{d}x, y + \mathrm{d}y) - f(x,y) \\
		&= \frac{\partial f}{\partial x} \mathrm{d}x + \frac{\partial f}{\partial y} \mathrm{d}y.
	\end{align*}
\end{rmk}

\begin{prop}
	Soit $f: U \to \R$ de classe $\mathcal{C}^1$ en $(a,b) \in U$. Alors, \[
		\forall w = (w_1, w_2) \in \R^2, \mathrm{d}f(w)\,(a,b) = w_1 \frac{\partial f}{\partial x}(a,b) + w_2 \frac{\partial f}{\partial y}(a,b).
	\]
\end{prop}

\begin{prv}
	Soit $w = (w_1, w_2) \in \R^2$. Soit $t \in \R^*$.
	\begin{align*}
		\frac{1}{t}\big(f(a + tw_1, b + tw_2) - f(a,b)\big)
		&= \frac{1}{t} \left( tw_1 \frac{\partial f}{\partial x}(a,b) + tw_2 \frac{\partial f}{\partial y}(a,b) + \po_{t \to 0}\big(\|tw\|\big) \right) \\
		&= w_1 \frac{\partial f}{\partial x}(a,b) + w_2 \frac{\partial f}{\partial y}(a,b) + \po_{t\to 0}(1) \\
		&\tendsto{t\to 0} w_1 \frac{\partial f}{\partial x}(a,b) + w_2\frac{\partial f}{\partial y}(a,b).
	\end{align*}
\end{prv}


\begin{defn}
	Avec les hypothèses précédentes, en posant \[
		\nabla f(a,b) = \left( \frac{\partial f}{\partial x}(a,b), \frac{\partial f}{\partial y}(a,b) \right) 
	\]on obtient \[
		\mathrm{d}f(w)\,(a,b) = \left<w  \mid \nabla f(a,b) \right>
	\] où $\left<\cdot|\cdot \right>$ est le produit scalaire.

	Le vecteur $\nabla f(a,b)$ est appelé \underline{gradient de $f$ en $(a,b)$}.

	Le développement limité à l'ordre 1 de $f$ devient \[
		f\big((a,b)+w\big) = f(a,b) + \left<w \mid \nabla f(a,b) \right> + \po_{w\to 0}(\|w\|)
	\]
\end{defn}

\begin{prop}
	Soit $f : U \to \R$ de classe $\mathcal{C}^1$.

	\begin{figure}[H]
    \centering
    \incfig{gradient}
	\end{figure}

	$\nabla f$ est orthogonal au lignes de niveaux de $f$, son orientation va dans le sens d'une augmentation de $f$.
\end{prop}

\begin{prv}
	Soit $\gamma : I \to U$ une courbe de niveau : \[
		\forall t \in I, f\big(\gamma(t)\big) = \text{cste}.
	\] D'après le lemme suivant : \[
		\forall t \in I, 0 = (f \circ \gamma)'(t) = \mathrm{d}f\big(\gamma'(t)\big)\big(\gamma(t)\big) = \left<\gamma'(t)  \mid \nabla f\big(\gamma(t)\big) \right>
	\] Donc $\nabla f\big(\gamma(t)\big)$ est orthogonal à $\gamma'(t)$.

	Pour tout $t \in I$, on pose $w(t) = t\, \nabla f\big(\gamma(t)\big)$. Donc \[
		f\big(\gamma(t) + w(t)\big) = f\big(\gamma(t)\big) + t \|\nabla f(\gamma(t))\|^2 + \po_{t \to 0}(t)
	\] Pour $t$ assez petit, $f\big(\gamma(t) + w(t)\big) - f\big(\gamma(t)\big)$ est du même signe que $t$.
\end{prv}

\begin{rmk}
	\begin{align*}
		V: \R^3 &\longrightarrow \R \\
		(x,y,z) &\longmapsto -mgz
	\end{align*}
	l'énerge potentielle de pesenteur

	On a donc \[
		\nabla V(x,y,z) = \left( \frac{\partial V}{\partial x}, \frac{\partial V}{\partial y}, \frac{\partial V}{\partial z} \right) = (0, 0, -mg) = \vec{P}.
	\]
\end{rmk}

\begin{lem}
	Soit $f : U \to \R$ de classe $\mathcal{C}^1$, $\gamma : \begin{array}{rcl}
		I &\longrightarrow& U \\
		t &\longmapsto& \big(x(t), y(t)\big)
	\end{array}$ où $x$ et $y$ sont dérivables.

	On pose \[
		\forall t \in I, \gamma'(t) = \big(x'(t), y'(t)\big).
	\] Alors $f \circ \gamma : I \to \R$ est dérivable et
	\begin{align*}
		\forall t \in I, (f \circ \gamma)'(t) &= \mathrm{d}f\big(\gamma'(t)\big) \big(\gamma(t)\big)\\
		&= \left<\gamma'(t)  \mid \nabla f\big(\gamma(t)\big)  \right> \\
		&= x'(t) \frac{\partial f}{\partial x}\big(x(t), y(t)\big) + y'(t) \frac{\partial f}{\partial y}\big(x(t),y(t)\big). \\
	\end{align*}
\end{lem}

\begin{prv}
	On fixe $t \in I$.

	\begin{align*}
		\forall h \neq 0, \frac{f \circ \gamma(t + h) - f \circ \gamma(t)}{h}
		&= \frac{1}{h}\big(f(\gamma(t)) + h\gamma'(t) + \po_{h\to 0}(h) - f(\gamma(t))\big) \\
		&= \frac{1}{h}\bigg(\cancel{f(\gamma(t))} + \left<h\gamma'(t) \mid \nabla f(\gamma(t)) \right> + \po_{h\to 0}(\|h\gamma'(t)\|) - \cancel{f(\gamma(t))}\bigg)\\
		&= \left<\gamma'(t) \mid \nabla f(\gamma(t)) \right> + \po_{h\to 0}(1) \\
		&\tendsto{h\to 0} \left<\gamma'(t)  \mid \nabla f(\gamma(t)) \right>
	\end{align*}
\end{prv}

\begin{defn}
	Soit $f : U \to \R$ de classe $\mathcal{C}^1$ et $(a,b) \in U$. On dit que $(a,b)$ est un \underline{point critique} de $f$ si $\nabla f(a,b) = 0$ i.e. $\frac{\partial f}{\partial x}(a,b) = \frac{\partial f}{\partial y}(a,b) = 0$.

	Dans ce cas, $f(a,b)$ est appelé \underline{valeur critique} de $f$.
\end{defn}

\begin{prop}~\\
	\begin{minipage}{\linewidth}
		\begin{wrapfigure}{r}{3cm}
			\centering
			\vspace{-1cm}
			\begin{asy}
				import solids;
				import graph;
				size(3cm);

				settings.render = 0;
				settings.prc = false;

				path3 par = graph(
					new real(real x) { return x; },
					new real(real x) { return 0; },
					new real(real x) { return -x^2; },
					0,3);
				revolution r = revolution(par, axis=Z);

				draw(r,1,longitudinalpen=nullpen);
				draw(r.silhouette());

				dot("$(a,b)$", O, red, align=N);
				real s = sqrt(2.5);
				path3 g=(s,0,-2.5)..(0,s,-2.5)..(-s,0,-2.5)..(0,-s,-2.5)..cycle;
				draw(g, deepcyan);
			\end{asy}
		\end{wrapfigure}
		Soit $f: U \to \R$ de classe $\mathcal{C}^1$ et $(a,b) \in U$ tel que \[
			\exists r > 0, \forall (x,y) \in B_{(a,b)}(r), f(x,y) \le f(a,b)
		\] Alors $\nabla f(a,b) = (0,0)$.
	\end{minipage}
\end{prop}

\begin{prv}
	Soit $g: x \mapsto f(x,b)$. $g(a)$ est un maximum local de $g$ donc $g'(a) = 0$.

	Or, $g'(a) = \frac{\partial f}{\partial x}(a,b)$

	donc $\frac{\partial f}{\partial x}(a,b) = 0$.

	Soit $h : y \mapsto f(a,y)$. On a de même $h'(b) = 0$.

	Or, $h'(b) = \frac{\partial f}{\partial y}(a,b)$.

	Donc, $\nabla f(a,b) = (0,0)$.
\end{prv}

\begin{rmk}
	Un minimum local est aussi une valeur critique.
\end{rmk}

\begin{figure}[H]
	\centering
	\begin{subfigure}{3cm}
		\centering
		\begin{asy}
			import solids;
			import graph;
			size(3cm);

			settings.render = 0;
			settings.prc = false;

			path3 par = graph(
				new real(real x) { return x; },
				new real(real x) { return 0; },
				new real(real x) { return -x^2; },
				0,3);
			revolution r = revolution(par, axis=Z);

			draw(r,1,longitudinalpen=nullpen);
			draw(r.silhouette());

			dot(O, red);
		\end{asy}
		\caption{Maximum local}
	\end{subfigure}
	\begin{subfigure}{3cm}
		\centering
		\begin{asy}
			import solids;
			import graph;
			size(3cm);

			settings.render = 0;
			settings.prc = false;

			path3 par = graph(
				new real(real x) { return x; },
				new real(real x) { return 0; },
				new real(real x) { return x^2; },
				0,3);
			revolution r = revolution(par, axis=Z);

			draw(r,1,longitudinalpen=nullpen);
			draw(r.silhouette());

			dot(O, red);
		\end{asy}
		\caption{Minimum local}
	\end{subfigure}
	\begin{subfigure}{3cm}
		\centering
		\begin{asy}
			import solids;
			import graph;
			size(3cm);

			settings.render = 0;
			settings.prc = false;
			currentprojection = obliqueZ;

			draw(graph(
				new real(real x) { return x; },
				new real(real x) { return -x^2 / 3; },
				new real(real x) { return 3; },
				-3, 3
			));

			draw(graph(
				new real(real x) { return x; },
				new real(real x) { return -x^2 / 3; },
				new real(real x) { return -3; },
				-3, 3
			));

			draw(graph(
				new real(real x) { return x; },
				new real(real x) { return -x^2 / 3 - 1; },
				new real(real x) { return 0; },
				-3, 3
			));

			draw(graph(
				new real(real x) { return 0; },
				new real(real x) { return x^2 / 9 - 1; },
				new real(real x) { return x; },
				-3, 3
			));

			draw(graph(
				new real(real x) { return -3; },
				new real(real x) { return x^2 / 9 - 4; },
				new real(real x) { return x; },
				-3, 3
			));

			draw(graph(
				new real(real x) { return 3; },
				new real(real x) { return x^2 / 9 - 4; },
				new real(real x) { return x; },
				-3, 3
			));

			dot((0,-1,0), red);
		\end{asy}
		\caption{Point de selle / Point col}
	\end{subfigure}
\end{figure}

\begin{exm}
	On revient à l'exemple donné en introduction : 
	\begin{align*}
		f: \left( \R^*_+ \right)^2 &\longrightarrow \R \\
		(x,y) &\longmapsto 2\left( xy + \frac{1}{x} + \frac{1}{y} \right).
	\end{align*}

	$\left( \R^+_* \right)^2$ est un ouvert de $\R^2$. Soit $(x,y) \in \left( \R^+_* \right)^2$.
	
	On a \[
		\begin{cases}
			\frac{\partial f}{\partial x}(x,y) = 2\left( y - \frac{1}{x^2} \right),\\
			\frac{\partial f}{\partial y}(x,y) = 2\left( x - \frac{1}{y^2} \right).
		\end{cases}
	\]

	\begin{align*}
		&\frac{\partial f}{\partial x}(x,y) = \frac{\partial f}{\partial y}(x,y) = 0\\
		\iff& \begin{cases}
			y = \frac{1}{x^2}\\
			x = \frac{1}{y^2}
		\end{cases}\\
		\iff& \begin{cases}
			y = \frac{1}{x^2}\\
			x = x^4
		\end{cases}\\
		\iff& \begin{cases}
			x = 1\\
			y = 1
		\end{cases}
	\end{align*}

	On vérivie que $f$ présente en effet un minium local en $(1,1)$. \[
		f(1,1) = 6
	\] On fixe $y \in \R^+_*$ et \[
		g : x \mapsto 2\left( xy + \frac{1}{x} + \frac{1}{y} \right).
	\] Donc \[
		\forall x \in \R^+_*, g'(x) = 2\left( y - \frac{1}{x^2} \right).
	\]
	\begin{center}
		\begin{tikzpicture}
			\tkzTabInit{$x$/1,$g'(x)$/1,$g$/2.3}{$0$, $\frac{1}{\sqrt{y}}$, $+\infty$}
			\tkzTabLine{,-,z,+,}
			\tkzTabVar{+/{}, -/$2\left( 2\sqrt{y} +\frac{1}{y} \right)$, +/{}}
		\end{tikzpicture}
	\end{center}
	
	Ainsi, \[
		\forall x \in \R^+_*, \forall y \in \R^+_*, f(x,y) \ge 2\left( 2\sqrt{y} + \frac{1}{y} \right)
	\] Soit $h : y \mapsto 2\sqrt{y} + \frac{1}{y}$. On a \[
		\forall y > 0, h'(y) = \frac{1}{\sqrt{y}} - \frac{1}{y^2} = \frac{y\sqrt{y} - 1}{y^2} = \frac{y^{\frac{3}{2}} - 1}{y^2}
	\]

	\begin{center}
		\begin{tikzpicture}
			\tkzTabInit{$y$/0.7,$h'(y)$/0.7,$h$/1.4}{$0$, $1$, $+\infty$}
			\tkzTabLine{,-,z,+,}
			\tkzTabVar{+/{}, -/$3$, +/{}}
		\end{tikzpicture}
	\end{center}

	Donc, \[
		\forall x,y > 0, f(x,y) \ge 2\times 3 = 6 = f(1,1).
	\]
\end{exm}

\begin{prop}
	[règle de la chaîne]

	Soit $f : \begin{array}{rcl}
		U &\longrightarrow& \R^2 \\
		(x,y) &\longmapsto& f(x,y)
	\end{array}$ de classe $\mathcal{C}^1$ et $U, V$ deux ouverts de $\R^2$.

	Soit $\varphi : \begin{array}{rcl}
		V &\longrightarrow& U \\
		(u,v) &\longmapsto& \varphi(u,v) = \big(x(u,v), y(u,v)\big)
	\end{array}$.

	On suppose que $x$ et $y$ sont de classe $\mathcal{C}^1$ sur $V$.

	Alors,  $f \circ \varphi : \begin{array}{rcl}
		V &\longrightarrow& \R \\
		(u,v) &\longmapsto& f\big(\varphi(u,v)\big)
	\end{array}$ est de classe $\mathcal{C}^1$ et
	\begin{align*}
		\forall (u_0, v_0) \in V, \frac{\partial (f \circ \varphi)}{\partial u}(u_0, v_0)
		&= \frac{\partial f}{\partial x}\big(\varphi(u_0, v_0)\big) \times \frac{\partial x}{\partial u}(u_0, v_0)\\
		&+ \frac{\partial f}{\partial y}\big(\varphi(u_0,v_0)\big) \frac{\partial y}{\partial u}(u_0,v_0)
	\end{align*}
	\begin{align*}
		\forall (u_0, v_0) \in V, \frac{\partial (f \circ \varphi)}{\partial v}(u_0, v_0)
		&= \frac{\partial f}{\partial x}\big(\varphi(u_0, v_0)\big) \times \frac{\partial x}{\partial v}(u_0, v_0)\\
		&+ \frac{\partial f}{\partial y}\big(\varphi(u_0,v_0)\big) \frac{\partial y}{\partial v}(u_0,v_0)
	\end{align*}
\end{prop}

\begin{exm}
	[changement de coordonnées polaires]
	On pose \begin{align*}
		\varphi: \R^+_* \times ]0,2\pi[ &\longrightarrow \R^2\setminus \left( R^+_* \times \{0\} \right) \\
		(r, \theta) &\longmapsto (r \cos \theta, r \sin\theta),
	\end{align*}
	\begin{align*}
		f: \R^2\setminus \left( R^+_* \times \{0\} \right) &\longrightarrow \R \\
		(x,y) &\longmapsto f(x,y),
	\end{align*}
	\begin{align*}
		g: \overbrace{\R^+_* \times ]0, 2\pi[}^{=V} &\longrightarrow \R \\
		(r, \theta) &\longmapsto f(r\cos\theta, r\sin\theta).
	\end{align*}

	\begin{align*}
		\forall (r_0,\theta_0) \in V,&\\[5mm]
		\frac{\partial g}{\partial r}(r_0, \theta_0) &= \frac{\partial f}{\partial x}(r_0\cos\theta_0, r_0\sin\theta_0)\cos\theta_0\\
		&+ \frac{\partial f}{\partial y}(r_0 \cos\theta_0, r_0\sin\theta_0)\sin\theta_0\\
		&= 2r_0\cos^2\theta_0 + 2r_0\sin^2(\theta_0) \\
		&= 2r_0 \\[5mm]
		\frac{\partial g}{\partial \theta}(r_0, \theta_0) &= \frac{\partial f}{\partial x}(r_0\cos\theta_0, r_0\sin\theta_0)r_0\sin\theta_0\\
		&+ \frac{\partial f}{\partial y}(r_0 \cos\theta_0, r_0\sin\theta_0)r_0\cos\theta_0\\
		&= -2{r_0}^2\cos(\theta_0)\sin(\theta_0) + 2{r_0}^2 \sin(\theta_0)\cos(\theta_0)\\
		&= 0 \\
	\end{align*}

	Donc, \[
		g(r, \theta) = r^2.
	\]
\end{exm}

\begin{exm}
	Résoudre \[
		\begin{cases}
			\frac{\partial f}{\partial x} = \frac{x}{x^2+y^2},\\
			\frac{\partial f}{\partial y} = \frac{y}{x^2+y^2}.\\
		\end{cases}
	\]

	On pose $g: (r, \theta) \mapsto f(r \cos\theta, r \sin\theta)$.

	\begin{align*}
		&\frac{\partial g}{\partial r} = \frac{1}{r}\cos^2\theta + \frac{1}{r}\sin^2\theta = \frac{1}{r},\\
		&\frac{\partial g}{\partial \theta} = -\cos(\theta) \sin(\theta) + \sin(\theta)\cos(\theta) = 0.
	\end{align*}

	Donc, \[
		\exists C \in \R, g: (r, \theta) \mapsto \ln r + C
	\] d'où,
	\begin{align*}
		\forall (x,y) \in \R^2 \setminus \{(0,0)\}, f(x,y) &= \ln\left(\sqrt{x^2 + y^2} \right)  + C\\
		&= \frac{1}{2}\ln(x^2 + y^2) + C. \\
	\end{align*}
\end{exm}

\begin{rmk}
	Soit $\mathcal{B} = (e_1, e_2)$ la base canonique de $\R^2$, $f: U \to \R$ de classe $\mathcal{C}^1$ avec $U$ un ouvert de $\R^2$.

	Soit $(x,y) \in U$.

	\begin{align*}
		\Mat_{\mathcal{B}}\big(\nabla f(x,y)\big) = \begin{pmatrix}
			\frac{\partial f}{\partial x}(x,y)\\[2mm]
			\frac{\partial f}{\partial y}(x,y)
		\end{pmatrix}
	\end{align*}

	Soit  \begin{align*}
		\varphi: V &\longrightarrow U \\
		(u,v) &\longmapsto \big(x(u,v), y(u,v)\big) 
	\end{align*} avec $x,y$ de classe $\mathcal{C}^1$. Soit $g = f \circ \varphi$.
	\begin{align*}
		\Mat_{\mathcal{B}}\big(\nabla g(u,v)\big)
		&= \begin{pmatrix}
			\frac{\partial g}{\partial u}(u,v) \\[2mm]
			\frac{\partial g}{\partial v}(u,v)
		\end{pmatrix} \\
		&= \begin{pmatrix}
			\frac{\partial x}{\partial u}(u,v) \frac{\partial f}{\partial x}(x,y)
			+ \frac{\partial y}{\partial u}(u,v)\frac{\partial f}{\partial y}(x,y)\\[3mm]
			\frac{\partial x}{\partial v}(u,v) \frac{\partial f}{\partial x}(x,y)
			+ \frac{\partial y}{\partial v}(u,v) \frac{\partial f}{\partial y}(x,y)
		\end{pmatrix}  \\
		&= \underbrace{\begin{pmatrix}
				\frac{\partial x}{\partial u}(u,v)& \frac{\partial y}{\partial u}(u,v)\\[3mm]
				\frac{\partial x}{\partial v}(u,v)& \frac{\partial y}{\partial v}(u,v)
		\end{pmatrix}}_{J(u,v)} \begin{pmatrix}
			\frac{\partial f}{\partial x}(x,y)\\[3mm]
			\frac{\partial f}{\partial y}(x,y)
		\end{pmatrix} \\
		&= J(u,v) \Mat_{\mathcal{B}}\big(\nabla f(x,y)\big) \\
	\end{align*}
	où $J(u,v) = 
	\begin{pNiceArray}{c:c}
		\Mat_{\mathcal{B}}\big(\nabla x(u,v)\big) & \Mat_{\mathcal{B}}\big(\nabla y(u,v)\big)
	\end{pNiceArray}$.

	On dit que $J(u,v)$ est \underline{la jacobienne} de $\varphi$ en $(u,v)$.
	L'application linéaire canoniquement associée à $J(u,v)$ est la \underline{différentielle de $\varphi$} en $(u,v)$ noté $\mathrm{d}\varphi(u,v)$.

	On a $\mathrm{d}\varphi(u,v) \in \mathcal{L}(R^2)$ et $\Mat_{\mathcal{B}}\big(\mathrm{d}\varphi(u,v)\big) = J(u,v)$.

	Par exemple, la jacobienne du changement de coordonnées polaires est \[
		J = \begin{pmatrix}
			\frac{\partial x}{\partial r} & \frac{\partial y}{\partial r}\\[3mm]
			\frac{\partial x}{\partial \theta} & \frac{\partial y}{\partial \theta}
		\end{pmatrix}
		= \begin{pmatrix}
			\cos\theta&\sin\theta\\
			-r\sin\theta&r\cos\theta
		\end{pmatrix}.
	\]
	$\underbrace{\det(J)}_{\text{le jacobien}} = r\cos^2\theta + r\sin^2\theta = r$

	Dans une intégrale double, si $(x,y) = \varphi(u,v)$, alors $\mathrm{d}x\mathrm{d}y = \det(J)\mathrm{d}u\mathrm{d}v$.

	Ici, \[
		\mathrm{d}x\ \mathrm{d}y = r\ \mathrm{d}r\ \mathrm{d}\theta.
	\]
\end{rmk}

\begin{prv}
	On pose $(x_0, y_0) = \varphi(u_0, v_0)$. Pour tout $(h,k) \in \R^2$ tels que $(u_0 + h, v_0 + k) \in V$, en posant $g = f  \circ \varphi$.

	\begin{align*}
		g(u_0 + h, v_0 + h) &= f\big(x(u_0 + h, v_0 + k), y(u_0 + h, v_0 + k)\big) \\
		&= f\left(
			x(u_0,v_0) + h \frac{\partial x}{\partial u}(u_0,v_0) + k \frac{\partial x}{\partial v}(u_0, v_0) + \po\big(\|(h,k)\|\big), \right.\\
		&\phantom{ = f\bigg(\bigg.}\left. y(u_0, v_0) + h \frac{\partial y}{\partial u}(u_0, v_0) + k \frac{\partial y}{\partial v}(u_0, v_0) + \po\big(\|(h,k)\|\big)
		\right)  \\
		&= f(x_0,y_0) \\
		&~+ \left( h \frac{\partial x}{\partial u}(u_0,v_0) + k \frac{\partial x}{\partial v}(u_0, v_0) + \po(\|(h,k)\|) \right) \frac{\partial f}{\partial x}(x_0,y_0)\\
		&~+ \left( h \frac{\partial y}{\partial u}(u_0, v_0) + k\frac{\partial y}{\partial v}(u_0, v_0) + \po(\|(h,k)\|) \right) \frac{\partial f}{\partial y}(x_0, y_0)\\
		&~+ \po(\|(h,k)\|)\\
		&= f(x_0, y_0) \\
		&~+ h \left( \frac{\partial x}{\partial u}(u_0, v_0) \frac{\partial f}{\partial x}(x_0, y_0) + \frac{\partial y}{\partial u}(u_0, v_0) \frac{\partial f}{\partial y}(x_0, y_0) \right)  \\
		&~+ k\left( \frac{\partial x}{\partial v}(u_0, v_0) \frac{\partial f}{\partial x}(x_0, y_0) + \frac{\partial y}{\partial v}(u_0, v_0) \frac{\partial f}{\partial y}(x_0, y_0) \right) 
		&~+ \po(\|(h,k)\|)\\
		&= g(u_0, v_0) + h \frac{\partial g}{\partial u}(u_0, v_0) + k \frac{\partial g}{\partial v}(u_0, v_0) + \po(\|(h,k)\|) \\
	\end{align*}

	Par identification,
	\[
		\frac{\partial g}{\partial u}(u_0, v_0) = \frac{\partial x}{\partial u}(u_0, v_0) \frac{\partial f}{\partial x}(x_0, y_0) + \frac{\partial y}{\partial u}(u_0, v_0) \frac{\partial f}{\partial y}(x_0,y_0)
	\] et \[
		\frac{\partial g}{\partial v}(u_0, v_0) = \frac{\partial x}{\partial v}(u_0,v_0) \frac{\partial f}{\partial x}(x_0, y_0) + \frac{\partial y}{\partial v}(u_0, v_0) \frac{\partial f}{\partial y}(x_0, y_0).
	\] 
\end{prv}

\begin{exm}
	[Régression linéaire]~\\
	\begin{figure}[H]
		\centering
		\begin{asy}
			import graph;
			axes(EndArrow);
			size(5cm);

			real f(real x) { return x + 0.5; }

			real k = 35 / (7 - 0.5);

			for(int i = 0; i < 35; ++i) {
				real mag = exp(sin(100 * pi/exp(1) * i)) * 0.8 + exp(cos(i*40)/3);
				real eps = mag * cos(10 * exp(1)/pi * i) / 3;
				dot((i/k,f(i/k) + eps));
			}

			draw(graph(f, -1, 7), orange);
		\end{asy}
	\end{figure}
	\[
		y = a x + b
	\] 
	On fixe $(a,b) \in \R^2$. \[
		\varepsilon(a,b) = \sum_{i=1}^n\big( y_i - (ax_i + b) \big)^2
	\] l'erreur totale.

	On veut minimiser $\varepsilon(a,b)$. On a 
	\[
		\forall (a,b) \in \R^2,
		\begin{cases}
			\frac{\partial \varepsilon}{\partial a}(a,b) = -2\sum_{i=1}^{n}(y_i - ax_i - b)x_i,\\
			\frac{\partial \varepsilon}{\partial b}(a,b) = -2\sum_{i=1}^{n}(y_i - ax_i - b).
		\end{cases}
	\]

	Donc,
	\begin{align*}
		(a,b) \text{ point critique de } \varepsilon \iff& \begin{cases}
			a \sum_{i=1}^n {x_i}^2 + b\sum_{i=1}^{n}x_i = \sum_{i=1}^{n} y_ix_i\\
			a\sum_{i=1}^{n}x_i + nb = \sum_{i=1}^ny_i
		\end{cases}\\
		\iff& \begin{cases}
			a \left( \frac{1}{n}\sum_{i=1}^n {x_i}^2 - \overline{x}^2\right) = \overline{y} - \overline{x} \overline{y}\\
			b = \frac{1}{n}\sum_{i=1}^ny_i - \frac{a}{n}\sum_{i=1}^nx_i = \frac{1}{n}\sum_{i=1}^n x_i y_i - \overline{x} \overline{y}
		\end{cases}\\
		&\text{ où } \overline{x} = \frac{1}{n} \sum_{i=1}^n x_i,~\overline{y} = \frac{1}{n}\sum_{i=1}^n y_i\\
		\iff& \begin{cases}
			a = \frac{\Cov(x,y)}{V(x)}\\
			b = \overline{y} - a\overline{x}
		\end{cases}
	\end{align*}

	Coefficient de corrélation: $\frac{\Cov(x,y)}{\sigma_x \sigma_y} \in [-1, 1]$
\end{exm}












		\part{Corps}

\begin{exm}[Problème]
	\begin{itemize}
		\item 
			avec $A = \Z / 9 \Z$, résoudre $\overline{x}^2 = \overline{0}$ \\
			\begin{center}
				\begin{tabular}{|c|c|c|c|c|c|c|c|c|c|c|}
					\hline
					$\overline{x}$&$\overline{0}$& $\overline{1}$ &$\overline{2}$&$\overline{3}$ &$\overline{4}$ &$\overline{5}$ &$\overline{6}$ &$\overline{7}$ &$\overline{8}$& $\overline{9}$ \\
					\hline
					$\overline{x}^2$&$\overline{0}$ &$\overline{1}$ &$\overline{4}$ &$\overline{0}$ &$\overline{7}$ &$7$ &$\overline{0}$ &$\overline{4}$ &$\overline{1}$&$\overline{0}$\\
					\hline
				\end{tabular}
			\end{center}
			On a trouvé 3 solutions: $\overline{0}$, $\overline{3}$, $\overline{6}$.
		\item $\Z / 8\Z$
			\begin{center}
				\begin{tabular}{|c|c|c|c|c|c|c|c|c|}
					\hline
					$\overline{x}$& $\overline{0}$& $\overline{1}$& $\overline{2}$& $\overline{3}$& $\overline{4}$& $\overline{5}$& $\overline{6}$& $\overline{7}$\\
					\hline
					$\overline{x^2}$& $\overline{0}$& $\overline{1}$& $\overline{4}$& $\overline{1}$& $\overline{0}$& $\overline{1}$& $\overline{4}$& $\overline{1}$\\
					\hline
				\end{tabular}
			\end{center}
			$\overline{x}^2=7$ a 4 solutions: $\overline{1}, \overline{7}, \overline{3},\text{ et } \overline{5}$
		\item $A = \mathbbm{H} = \{a + bi + cj + dk  \mid  (a,b,c,d) \in \R^4\}$ \\
			$i^2 = j^2 = k^2 = -1$ 
			\begin{align*}
				\begin{array}{c c c}
					ij = k & jk = i & ji = j\\
					ji = -k & kj = -i & ik = -j
				\end{array}
			\end{align*}
			Dans cet anneau, $-1$ a 6 racines!
	\end{itemize}
\end{exm}

\begin{defn}
	Soit $(\mathbbm{K}, +, \times)$ un ensemble muni de deux lois de composition internes. On dit que c'est un \underline{corps} si
	 \begin{enumerate}
		\item $(\mathbbm{K}, \times)$ est un groupe abélien
		\item $(\mathbbm{K}, \times)$ est un monoïde commutatif
		\item $\forall x \in \mathbbm{K}\setminus \{0_\mathbbm{K}\}, \exists y \in \mathbbm{K}, xy = 1_\mathbbm{K}$
		\item $0_\mathbbm{K} \neq  1_\mathbbm{K}$
	\end{enumerate}
	\index{corps}
\end{defn}

\begin{exm}
	\begin{itemize}
		\item $(\C, +, \times)$ est un corps
		\item $(\R, +, \times)$ est un corps
		\item $(\Q, +, \times)$ est un corps
		\item $(\Z, +, \times)$ n'est pas un corps
	\end{itemize}
\end{exm}

\begin{prop}
	$(\Z / n\Z, +, \times)$ est un corps si et seulement si $n$ est premier.
\end{prop}

\begin{prv}
	\[
		\left( \Z / n\Z \right)^\times = \left\{ \overline{k}  \mid k \wedge n = 1 \right\}
	\] 
\end{prv}


\begin{prop}
	Tout corps est un anneau intègre.
\end{prop}

\begin{prv}
	Soit $(\mathbbm{K}, +, \times)$ un corps. Soient $(a,b) \in \mathbbm{K}^2$ tel que $a \times b = 0_\mathbbm{K}$.\\
	On suppose $a \neq  0_\mathbbm{K}$. Alors, $a$ est inversible et donc \[
		b = a^{-1} \times a \times b = a^{-1} \times 0_\mathbbm{K} = 0_\mathbbm{K}
	\] 
\end{prv}

\begin{exm}
	Soit $(\mathbbm{K},+,\times)$ un corps.\\
	Résoudre \[
		\begin{cases}
			x^2 = 1_\mathbbm{K}\\
			x \in \mathbbm{K}
		\end{cases}
	\]

	\begin{align*}
		x^2 = 1_\mathbbm{K} &\iff x^2 - 1_\mathbbm{K} = 0_\mathbbm{K}\\
		&\iff (x - 1_\mathbbm{K})(x+1_\mathbbm{K}) = 0_\mathbbm{K}\\
		&\iff x - 1_\mathbbm{K} = 0_\mathbbm{K} \text{ ou } x + 1_\mathbbm{K} = 0_\mathbbm{K}\\
		&\iff x = 1_\mathbbm{K} \text{ ou } x = -1_\mathbbm{K}
	\end{align*}

	Il y a au plus 2 solutions.
\end{exm}

\begin{prop}
	Soit $(\mathbbm{K},+,\times )$ un corps et $P$ un polynôme à coefficients dans $\mathbbm{K}$ de degré $n$. Alors, l'équation $P(x) = 0_{\mathbbm{K}}$ a au plus $n$ solutions dans $\mathbbm{K}$ 
	\qed
\end{prop}

\begin{crlr}[(Théorème de Wilson)]
	voir exercice 16 du TD 12
\end{crlr}


\begin{defn}
	Soit $(\mathbbm{K}, +, \times)$ un corps et $L\subset \mathbbm{K}$.\\
	On dit que $L$ est un \underline{sous corps} de $\mathbbm{K}$ si
	\begin{enumerate}
		\item $L$ est un anneau de $(\mathbbm{K}, +, \times)$ non nul
		\item $\forall x \in L\setminus \{0_\mathbbm{K}\}, x^{-1} \in L$ 
	\end{enumerate}
	\vspace{2mm}
	en d'autres termes si
	\begin{enumerate}
		\item $\forall (x,y) \in L^2, x - y \in L$
		\item $\forall (x,y) \in L^2, x \times y^{-1} \in L$
	\end{enumerate}
	\vspace{5mm}
	On dit aussi que $\mathbbm{K}$ est une \underline{extension} de $L$.
	\index{sous corps}
	\index{extension}
\end{defn}

\begin{prop}
	Tout sous corps est un corps. \qed
\end{prop}

\begin{defn}
	Soient $(\mathbbm{K}_1,+,\times )$ et $(\mathbbm{K}_2,+, \times)$ deux corps et $f: \mathbbm{K}_1 \to \mathbbm{K}_2$.\\
	On dit que $f$ est un \underline{morphisme de corps} si $f$ est un morphisme d'anneaux.\\
	i.e. si
	\[
		\begin{cases}
			\forall (x,y) \in {\mathbbm{K}_1}^2,& f(x+y) = f(x) + f(y)\\
			\forall (x,y) \in {\mathbbm{K}_1}^2,& f(x \times y) = f(x) \times f(y)\\
		\end{cases}
	\] 
	\index{homomorphisme (de corps)}
	\index{morphisme (de corps)}
\end{defn}

\begin{prop}
	Tout morphisme de corps est injectif.
\end{prop}

\begin{prv}
	Soit $f: \mathbbm{K}_1 \to \mathbbm{K}_2$ un morphisme de corps.\\
	\begin{itemize}
		\item $\Ker(f)$ est un sous groupe de $(\mathbbm{K}_1, +)$ 
		\item Soit $x \in \Ker(f)$ et $y \in \mathbbm{K}_1$ \[
				f(x \times y) = f(x) \times f(y) = 0_{\mathbbm{K}_2} \times f(y) = 0_{\mathbbm{K}_2}
			\]
		\item Soit $x \in \Ker(f) \setminus \{0_{\mathbbm{K}_1}\}$.\\
			Alors, $x$ est inversible.\\
			\begin{align*}
				\begin{rcases*}
					x \in \Ker(f)\\
					x^{-1} \in \mathbbm{K}_1
				\end{rcases*}& \text{ donc } x \times x ^{-1} \in \Ker(f)\\
				&\text{ donc } 1_{\mathbbm{K}_1} \in \Ker(f)\\
				&\text{ donc } f(1_{\mathbbm{K}_1}) = 0_{\mathbbm{K}_2}
			\end{align*}
			Or, $f(1_{\mathbbm{K}_1}) = 1_{\mathbbm{K}_2} \neq 0_{\mathbbm{K}_2}$
	\end{itemize}
	Donc, $\Ker(f) = \{0_{\mathbbm{K}_1}\}$ donc $f$ est injective.
\end{prv}

\begin{exm}
	$\begin{array}{cc}
		\C &\longrightarrow \C\\
		z &\longmapsto \overline{z}\\
	\end{array}$ est un morphisme de corps
\end{exm}



		\addrecap
	}

	{
		\chap[25]{Séries numériques}
		\renewcommand{\cwd}{../chap25}
		\begin{defn}
	Soit $E$ un $\mathbbm{K}$-espace vectoriel. On dit que $E$ est de \underline{dimension finie} si $E$ a au moins une famille génératrice finie. On dit que $E$ est de \underline{dimension infinie} sinon.
	\index{dimension finie (espace vectoriel)}
	\index{dimension infinie (espace vectoriel)}
\end{defn}

\begin{thm}
	[Théorème de la base extraite]
	Soit $E$ un $\mathbbm{K}$-espace vectoriel non nul de dimension finie. Soit $\mathcal{G}$ une famille génératrice finie de $E$. Alors, il existe une base $\mathcal{B}$ de $\mathcal{E}$ telle que $\mathcal{B} \subset \mathcal{G}$.
\end{thm}

\begin{prv}
	[par récurrence sur $\#G = \Card(G)$]
	\begin{itemize}
		\item Soit $E$ un $\mathbbm{K}$-espace vectoriel non nul engendré par $\mathcal{G} = (u)$.\\
			Si $u = 0_E$, alors $E = \{0_E\}$: une contradiction $\lightning$ \\
			Donc $u \neq 0_E$ donc $(u)$ est libre. En effet, \[
				\forall \lambda \in \mathbbm{K}, \lambda u = 0_E \implies \lambda = 0_\mathbbm{K}
			\] Donc $\mathcal{G}$ est une base de $E$.\\
		\item Soit $n \in \N_*$. Soit $E$ un $\mathbbm{K}$-espace vectoriel. On suppose que si $E$ a une famille génératrice constituée de $n$ vecteurs, alors on peut extraire de cette famille une base de $E$.\\
			Soit $\mathcal{G}$ une famille génératrice de $E$ avec $n+1$ vecteurs.\\
			Si $\mathcal{G}$ est libre, alors $\mathcal{G}$ est une base de $E$. \\
			Si $\mathcal{G}$ n'est pas libre, alors il existe $u \in \mathcal{G}$ tel que $u \in \Vect(\mathcal{G}\setminus \{u\})$ \\
			Donc $\mathcal{G}\setminus \{u\}$ engendre $E$. Or, $\mathcal{G}\setminus \{u\}$ possède $n$ vecteurs. D'après l'hypothèse de récurrence, il existe une base $\mathcal{B}$ de $E$ telle que \[
				\mathcal{B} \subset \mathcal{G} \setminus \{u\} \subset \mathcal{G}
			\] 
	\end{itemize}
\end{prv}

\begin{crlr}
	Tout espace de dimension finie a une base.
	\qed
\end{crlr}

\begin{thm}
	[Théorème de la base incomplète]
	Soit $E$ un $\mathbbm{K}$-espace vectoriel de dimension finie, $\mathcal{G}$ une famille génératrice finie de $E$. $\mathcal{L}$ une famille libre de $E$. Alors, il existe une base $\mathcal{B}$ de $E$ telle que \[
		\mathcal{L} \subset \mathcal{B} \text{ et } \mathcal{B}\setminus \mathcal{L} \subset \mathcal{G}
	\] 
\end{thm}

\begin{prv}
	[par récurrence sur $\#(\mathcal{G}\setminus\mathcal{L})$]
	\begin{itemize}
		\item Avec les notations précédentes, on suppose que $\mathcal{G}\setminus\mathcal{L} \neq \O$ \[
				\forall u \in \mathcal{G}, u \in \mathcal{L}
			\] Donc $\mathcal{G} \subset \mathcal{L}$ donc $\mathcal{L}$ est génératrice donc $\mathcal{L}$ est une base de $E$. On pose $\mathcal{B} = \mathcal{L}$ et alors \[
				\mathcal{L} \subset  \mathcal{B} \text{ et } \mathcal{B}\setminus\mathcal{L} = \O \subset  \mathcal{G}
			\] 
		\item Soit $n \in \N$. On suppose que si $\mathcal{G}$ est génératrice et $\mathcal{L}$ libre avec $\#(\mathcal{G}\setminus\mathcal{L}) = n$ alors il existe une base $\mathcal{B}$ de $E$ telle que \[
			\mathcal{L}\subset \mathcal{B} \text{ et } \mathcal{B}\setminus\mathcal{L}\subset \mathcal{G}
		\] Soient à présent $\mathcal{G}$ une famille génératrice de $E$ et $\mathcal{L}$ une famille libre de $E$ telles que $\#(\mathcal{G}\setminus\mathcal{L}) = n+1 > 0$\\
		Si $\mathcal{L}$ engendre $E$, alors $\mathcal{L}$ est une base de $E$. On pose $\mathcal{B} = \mathcal{L}$ et on a bien \[
			\mathcal{L} \subset  \mathcal{B} \text{ et } \mathcal{B} \setminus \mathcal{L} = \O \subset  \mathcal{G}
		\] On suppose que $\mathcal{L}$ n'engendre pas $E$. Il existe $u \in \mathcal{G}$ tel que $u \not\in \Vec(\mathcal{L})$ (car sinon, $\mathcal{G} \subset \Vect(\mathcal{L})$ et donc $\underbrace{\Vect(\mathcal{G})}_{= E} \subset  \underbrace{\Vect(\mathcal{L})}_{ \subset E}$\\
		Donc $\mathcal{L} \cup \{u\} $ est libre. On pose $\mathcal{L}' = \mathcal{L} \cup \{u\} $ \[
			\mathcal{G}\setminus \mathcal{L}' = \mathcal{G}\setminus (\mathcal{L} \cup \{u\}) = (\mathcal{G}\setminus\mathcal{L})\setminus \{u\} 
		\] donc $\#(\mathcal{G}\setminus\mathcal{L}') = n+1 -1 = n$\\
		D'après l'hypothèse de récurrence, il existe $\mathcal{B}$ une base de $E$ telle que \[
			\mathcal{L} \subset  \mathcal{L}' \subset \mathcal{B} \text{ et } \mathcal{B}\setminus \mathcal{L}' \subset \mathcal{G}
		\] \[
			\mathcal{B} \setminus \mathcal{L} = \underbrace{\mathcal{B}\setminus\mathcal{L}'}_{\subset \mathcal{G}} \cup \underbrace{\{u\}}_{\subset \mathcal{G} \text{ car } u \in \mathcal{G}}
		\] On a $\mathcal{B}\setminus\mathcal{L}\subset \mathcal{G}$
	\end{itemize}
\end{prv}

\begin{thm}
	Soit $E$ un $\mathbbm{K}$-espace vectoriel de dimension finie. Toutes les bases de $E$ ont le même cardinal.
\end{thm}

\begin{prv}
	Soit $\mathcal{G}$ une famille génératrice finie de $E$ et $\mathcal{B} \subset  \mathcal{G}$ une base de $E$. On note $n = \#\mathcal{B}$ \\
	Soit $\mathcal{B}'$ une base de $E$. On pose $p = n - \#(\mathcal{B} \cap  \mathcal{B}')$. Montrons par récurrence sur  $p$ que $\#\mathcal{B} = \#\mathcal{B}'$ 
	\begin{itemize}
		\item On suppose que $p = 0$. Alors, $\#(\mathcal{B} \cap \mathcal{B}') = n$ \\
			Or, $\mathcal{B}' \cap \mathcal{B} \subset \mathcal{B}$ donc $\mathcal{B} \cap \mathcal{B}' = \mathcal{B}$ donc $\mathcal{B} \subset  \mathcal{B}'$ et donc $\mathcal{B} = \mathcal{B}'$ 
		\item Soit $p \in \N$. On suppose que si $\mathcal{B}'$ est une base de $E$ telle que $n - \#(\mathcal{B} \cap \mathcal{B}') = p$, alors $\#\mathcal{B}' = n$ \\
			Aoit $\mathcal{B}'$ une base de $E$ telle que $n - \#(\mathcal{B}\cap \mathcal{B}') = p+1 > 0$ \\
			Donc $\mathcal{B} \cap \mathcal{B}' \neq \mathcal{B}$. Soit $u \in \mathcal{B}' \setminus \mathcal{B}$. D'après le lemme d'échange, il existe $v \in \mathcal{B}\setminus \mathcal{B}'$ tel que $\mathcal{B}' \setminus \{u\} \cup \{v\}$ est une base de $E$. On pose $\mathcal{B}'' = \mathcal{B}' \setminus \{u\} \cup \{v\}$ 
			\begin{align*}
				\mathcal{B}'' \cap \mathcal{B} &= \left( (\mathcal{B}' \setminus \{u\})  \cap \mathcal{B} \right) \cup \{v\} \\
				&= (\mathcal{B}' \cap \mathcal{B}) \cup \{v\} \\
			\end{align*}
			donc,
			\begin{align*}
				n - \#(\mathcal{B}'' \cap \mathcal{B}) &= n - (\#(\mathcal{B}' \cap \mathcal{B}) + 1) \\
				&= p+1- 1 \\
				&= p \\
			\end{align*}
			D'après l'hypothèse de récurrence, \[
				\#\mathcal{B}'' = n
			\] Or, $\#\mathcal{B}'' = \#\mathcal{B}'$
	\end{itemize}
\end{prv}

\begin{lem}
	Soient $\mathcal{B}$ et $\mathcal{B}'$ deux bases de $E$ telles que $\mathcal{B}\subset \mathcal{B}'$. Alors, $\mathcal{B} = \mathcal{B}'$.
\end{lem}

\begin{prv}
	On suppose $\mathcal{B}' \neq \mathcal{B}$. Soit $u \in \mathcal{B}' \setminus \mathcal{B}$
	$u \in E = \Vect(\mathcal{B})$ donc $\mathcal{B} \cup \{u\}$ n'est pas libre.
	Donc $\mathcal{B}\cup \{u\} \subset \mathcal{B}'$ et $\mathcal{B}'$ est libre donc $\mathcal{B}\cup \{u\}$ est libre: une contradiction $\lightning$
\end{prv}

\begin{lem}
	[Lemme d'échange] Soient $\mathcal{B}_1$ et $\mathcal{B}_2$ deux bases de $E$ et $u \in \mathcal{B}_1 \setminus \mathcal{B}_2$. Alors, il existe $v \in \mathcal{B}_2$ tel que $(\mathcal{B}_1 \setminus \{u\}) \cup \{v\}$ soit une base de $E$.
\end{lem}

\begin{prv}
	[1${}^\text{nde}$ méthode]
	On suppose que pout tout $v \in \mathcal{B}_2$, $(\mathcal{B}_1\setminus \{u\}) \cup \{v\}$ n'est pas une base de $E$
	Soit $v \in \mathcal{B}_2$.
	\begin{itemize}
		\item Supposons $(\mathcal{B}_1\setminus \{u\})\cup \{v\}$ non libre. $\mathcal{B}_1 \setminus \{u\}$ est libre. Donc $v \in \Vect(\mathcal{B}_1 \setminus \{u\})$
		\item Supposons $(\mathcal{B}_1\setminus \{u\}) \cup \{v\}$ non génératrice.
			Comme $\mathcal{B}_1$ engendre $E$, $u \not\in \Vect(\mathcal{B}_1\setminus \{v\})$.
			On suppose que $\mathcal{B}_1 \neq \mathcal{B}_2$.
			$\forall v \in \mathcal{B}_2 \setminus \mathcal{B}_1, \Vect(\mathcal{B}_1 \setminus \{v\}) = \Vect(\mathcal{B}_1) = E \ni u$ 
			donc, $(\mathcal{B}_1\setminus \{u\}) \cup \{v\}$ engendre $E$ et donc \[
				v \in \Vect(\mathcal{B}_1 \setminus \{u\})
			\] On a aussi \[
				\forall v \in \mathcal{B}_1 \setminus \{u\}, v \in \Vect(\mathcal{B}_1\setminus \{u\})
			\] Comme $u \not\in \mathcal{B}_2$, on a \[
				\forall v \in \mathcal{B}_2, v \in \Vect(\mathcal{B}_1\setminus \{u\})
			\] docn \[
				E = \Vect(\mathcal{B}_2) \subset \Vect(\mathcal{B}_1\setminus \{u\})
			\] donc $\mathcal{B}_1\setminus \{u\}$ engendre $E$ donc $\mathcal{B}_1\setminus \{u\}$ est une base de $E$. Or, $\mathcal{B}_1 \setminus \{u\}  \subset  \mathcal{B}_1$, donc $\mathcal{B}_1\setminus \{u\} = \mathcal{B}_1$
	\end{itemize}
\end{prv}

\begin{prv}
	[2${}^\text{nde}$ méthode]
	On suppose que pout tout $v \in \mathcal{B}_2$, $(\mathcal{B}_1\setminus \{u\}) \cup \{v\}$ n'est pas une base de $E$
	\begin{itemize}
		\item Comme $u \in \mathcal{B}_1 \setminus \mathcal{B}_2$, nécéssairement $\mathcal{B}_1 \neq \mathcal{B}_2$ donc $\mathcal{B}_2 \not\subset \mathcal{B}_1$, donc $\mathcal{B}_2\setminus\mathcal{B}_1 \neq \O$ 
		\item Soit $v \in \mathcal{B}_2\setminus\mathcal{B}_1$. Il existe $(\lambda_w)_{w\in\mathcal{B}_1}$ une famille de scalaires presque nulle telle que \[
				v = \sum_{w \in \mathcal{B}_1} \lambda_w w - \lambda_u u + + \sum_{w \in \mathcal{B}_1\setminus \{u\}}\lambda_w w
			\]
			Si $\lambda_u \neq 0_E$, alors
			\begin{align*}
				u &= \lambda_u^{-1}\left( v - \sum_{w \in \mathcal{B}_1 \setminus \{u\}} \lambda_w w \right)\\
					&\in \Vect(\mathcal{B}_1\setminus \{u\} \cup v)
			\end{align*}
			 donc $\mathcal{B}_1 \subset \Vect(\mathcal{B}_1\setminus \{u\} \cup \{v\})$\\
			 et donc $E \subset  \Vect(\mathcal{B}_1 \setminus \{u\} \cup \{v\})$ \\
			 et donc $\mathcal{B}_1 \setminus \{u\} \cup \{v\}$ engendre $E$ \\
			 donc $\mathcal{B}_1 \setminus \{u\} \cup \{v\}$ n'est pas libre\\
			 donc $v \in \Vect(\mathcal{B}_1\setminus \{u\})$ (car $\mathcal{B}_1 \setminus \{u\}$ est libre\\
			 donc $\lambda_u = 0_\mathbbm{K}$ $\lightning$\\`

			 Donc, $\lambda_u = 0_\mathbbm{K}$, docn $v \in \Vect(\mathcal{B}_1\setminus \{u\})$ \\
			 On vient de prouver que
			 \begin{align*}
			 	\mathcal{B}_2 \setminus \mathcal{B}_1 \subset \Vect(\mathcal{B}_1 \setminus \{u\})\\
			 	\mathcal{B}_1 \setminus \{u\} \subset \Vect(\mathcal{B}_1 \setminus \{u\})\\
			 \end{align*}
			 Comme $u \not\in \mathcal{B}_2$, \[
			 	\mathcal{B}_2 \subset \Vect(\mathcal{B}_1 \setminus \{u\})
			 \] donc \[
			 	E = \Vect(\mathcal{B}_2) \subset  \Vect(\mathcal{B}_1 \setminus \{u\})
			 \] donc $\mathcal{B}_1 \setminus \{u\}$ engendre $E$. Donc,  $\mathcal{B}_1 \setminus \{u\}$ est une base de $E$.\\
			 Or, $\mathcal{B}_1 \setminus \{u\} \subset  \mathcal{B}_1$, donc $\mathcal{B}_1 \setminus \{u\} = \mathcal{B}_1$
	\end{itemize}
\end{prv}

\begin{defn}
	Soit $E$ un $\mathbbm{K}$-espace vectoriel de dimension finie. Le cardinal commun à toutes les bases de $E$ est appelé \underline{dimension} de $E$ est notée $\dim(E)$ ou $\dim_\mathbbm{K}(E)$\\
	C'est donc aussi le nombre de coordonnées de n'importe quel vecteur dans n'importe quelle base.
	\index{dimension (espace vectoriel)}
\end{defn}

\begin{exm}
	\begin{enumerate}
		\item $\dim_\R(\C) = 2$ et $\dim_\C(\C) = 1$ 
		\item $\dim_\mathbbm{K}(\mathbbm{K}^{n}) = n$ 
		\item $\dim_{\mathbbm{K}}(\mathcal{M}_{n,p}(\mathbbm{K})) = np$
	\end{enumerate}
\end{exm}

\begin{crlr}
	Soit $E$ un $\mathbbm{K}$-espace vectoriel de dimension finie, $\mathcal{L}$ une famille libre de $E$, $\mathcal{G}$ une famille génératrice de $E$. On note $n = \dim(E)$
	\begin{enumerate}
		\item $\#\mathcal{G} \ge n$ et $(\#\mathcal{G} = n \implies \mathcal{G} \text{ est une base de } E$)
		\item $\#\mathcal{L} \le n$ et $(\#\mathcal{L} = n \implies \mathcal{L} \text{ est une base de } E$)
	\end{enumerate}
\end{crlr}

\begin{crlr}
	$\R^{\R}$ est de dimension infinie.
	$\forall i \in \N, e_i: x \mapsto x^i$\\
	$(e_i)_{i\in\N}$ est libre dans $\R^\R$
\end{crlr}

\begin{prop}
	Soient $E$ et $F$ deux $\mathbbm{K}$-espaces vectoriels de dimension finie. Alors $E\times F$ est de dimension finie et $\dim(E\times F) = \dim(E) + \dim(F)$
\end{prop}

\begin{prv}
	Soit $(e_1,\ldots, e_n)$ une base de $E$, $(f_1, \ldots, f_p)$ une base de $F$.
	On pose \[
		\left\{\begin{array}
			{r c l}
			u_1 &=& (e_1,0_F)\\
			u_2 &=& (e_2,0_F)\\
					&\vdots&\\
			u_n &=& (e_n,0_F)\\
			u_{n+1} &=& (0_E, f_1)\\
			u_{n+2} &=& (0_E, f_2)\\
					&\vdots&\\
			u_{n+p} &=& (0_E,f_p)\\
		\end{array}\right.
	\]
	Soit $(x,y) \in E\times F$. \[
		\begin{cases}
			\exists (x_1,\ldots,x_n)\in \mathbbm{K}^n, x = \sum_{i=1}^{n} x_ie_i
			\exists (y_1,\ldots,y_n)\in \mathbbm{K}^n, x = \sum_{j=1}^{p} y_jf_j
		\end{cases}
	\] 
	\begin{align*}
		(x,y) &= \left( \sum_{i=1}^{n} x_ie_i, \sum_{i=1}^{p} y_jf_j \right)  \\
		&= \sum_{i=1}^{n} x_i (e_i + 0_F) + \sum_{j=1}^{p} y_j (0_E, f_j) \\
		&= \sum_{i=1}^{n} x_i u_i + \sum_{j=1}^{p} y_j u_{n+j} \\
	\end{align*}
	Donc, $E\times F = \Vect(u_1, \ldots, u_{n+p})$ donc $E\times F$ est de dimension finie.\\
	Soit $(\lambda_1, \ldots, \lambda_{n+p}) \in \mathbbm{K}^{n+p}$ tel que \[
		(*): \quad \sum_{k=1}^{n+p} \lambda_ku_k = 0_{E\times F} = (0_E, 0_F)
	\]
	\begin{align*}
		(*) &\iff \sum_{k=1}^{n} \lambda_k (e_k, 0_F) + \sum_{k=n+1}^{p} \lambda_k(0_E, f_{k-n}) = (0_E, 0_F)\\
				&\iff \begin{cases}
					\sum_{k=1}^{n} \lambda_k e_k = 0_E\\
					\sum_{k=n+1}^{p} \lambda_k f_{k-n} = 0_F
				\end{cases}\\
				&\iff \begin{cases}
					\forall k \in \left\llbracket 1,n \right\rrbracket, \lambda_k = 0_\mathbbm{K} \qquad&(\text{car $(e_1,\ldots,e_n)$ est libre})\\
					\forall k \in \left\llbracket n+1,n+p \right\rrbracket, \lambda_k = 0_\mathbbm{K} \qquad&(\text{car $(f_1,\ldots,f_n)$ est libre})\\
				\end{cases}
	\end{align*}
	Donc $(u_1, \ldots, u_{n+p})$ est une base de $E\times F$. Donc, $\dim(E\times F) = n + p = \dim(E) + \dim(F)$
\end{prv}

\begin{rmk}
	[Convention]
	\[\dim\big(\{0_E\}\big) = 0\]
\end{rmk}

\begin{thm}
	Soit $E$ un $\mathbbm{K}$-espace vectoriel de dimension finie, $F$ un sous-espace vectoriel de $E$. Alors, $F$ est de dimension finie et  $\dim(F) \le \dim(E)$\\
	Si $\dim(F) = \dim(E)$, alors $F = E$
\end{thm}

\begin{prv}
	On considère \[
		A = \{k \in \N \mid \text{il existe une famille libre de $F$ à $k$ éléments}\} 
	\]
	On suppose $F \neq \{0_E\}$.
	\begin{itemize}
		\item Soit $u \in F\setminus \{0_E\}$. $(u)$ est libre donc $1 \in A$ et donc $A \neq \O$
		\item Soit $\mathcal{L}$ une famille libre de $F$. Alors, $\mathcal{L}$ est une famille libre de $E$ \\
			donc $\#\mathcal{L} \le \dim(E)$\\
			Donc $A$ est majorée par $\dim(E)$ \\
			On en déduit que $A$ a un plus grand élément $p$.
		\item Soit $\mathcal{L}$ une famille libre de $F$ avec $p$ éléments.\\
			Si $\mathcal{L}$ n'engendre pas $F$, alors il existe $u\in F$ tel que $u\not\in \Vect(\mathcal{L})$ et donc $\mathcal{L} \cup \{u\}$ est une famille libre de $F$, donc $p+1 \in A$ en contradiction avec la maximalité de $p$.\\
			Donc $\mathcal{L}$ est une base de $F$ donc $F$ est de dimension finie et $\dim(F) = p \le \dim(E)$\\
	\end{itemize}

	Soit $\mathcal{B}$ une base de $F$. Alors, $\mathcal{B}$ est aussi une famille de libre de de $E$. Donc $\#\mathcal{B} \le \dim(E)$ donc $\dim(F) = \dim(E)$ \\
	Si $\dim(F) = \dim(E)$, alors $\mathcal{B}$ est une base de $E$, et donc $F = \Vect(\mathcal{B}) = E$
\end{prv}

\begin{prop}
	[Formule de Grassmann]
	Soit $E$ un $\mathbbm{K}$-espace vectoriel de dimension finie, $F$ et $G$ deux sous-espace vectoriels de $E$. Alors, \[
		\dim(F+G) = \dim(F) + \dim(G) - \dim(F\cap G)
	\] 
\end{prop}

\begin{prv}
	Soit $(e_1, \ldots, e_p)$ une base de $F\cap G$. $(e_1,\ldots,e_p)$ est une famille libre de $F$.\\
	On complète $(e_1, \ldots, e_p)$ en une base $(e_1, \ldots, e_p, u_1, \ldots, u_q)$ de $F$.\\
	De même, on complète $(e_1, \ldots, e_p)$ en une base $(e_1, \ldots, e_p, v_1, \ldots, v_r)$ de $G$.\\
	On pose  $\mathcal{B} = (e_1, \ldots, e_p, u_1, \ldots, u_q, v_1, \ldots, v_r)$. Montrons que $\mathcal{B}$ est une base de $F+G$
	\begin{itemize}
		\item Soit $u \in F+G$ \\
			On pose $u = v+w$ avec $\begin{cases}
				v\in F\\
				w \in G
			\end{cases}$.\\
			On pose $v = \sum_{i=1}^p \lambda_i e_i + \sum_{i=1}^q \mu_i u_i$ avec $(\lambda_1, \ldots, \lambda_p, \mu_1, \ldots, \lambda_q) \in \mathbbm{K}^{p+q}$\\
			On pose aussi $w = \sum_{i = 1}^p \lambda'_ie_i + \sum_{j=1}^r \nu_j v_j$ avec $(\lambda_1',\ldots,\lambda_p', \nu_1, \ldots, \nu_r) \in \mathbbm{K}^{p+r}$\\
			D'où, \[
				u = \sum_{i=1}^p (\lambda_i + \lambda'_i)e_i + \sum_{j=1}^q \mu_j u_j + \sum_{k=1}^r \nu_k v_k \in \Vect(\mathcal{B})
			\]
		\item Soient $(\lambda_1, \ldots, \lambda_p, \mu_1, \ldots, \mu_q, \nu_1, \ldots, \nu_r) \in \mathbbm{K}^{p+q+r}$.\\
			On suppose \[
				(*)\quad \sum_{i=1}^{p}\lambda_ie_i + \sum_{j=1}^q\mu_ju_j + \sum_{k=1}^r \nu_k v_k = 0_E
			\] 
			D'où, \[
				\underbrace{\sum_{i=1}^p\lambda_i e_i + \sum_{j=1}^q \mu_ju_j}_{\in F} = \underbrace{-\sum_{k=1}^r\nu_jv_k}_{\in G}
			\] 
			Donc, \[
				f = \sum_{i=1}^p \lambda_i e_i + \sum_{j=1}^q \mu_j u_j \in F\cap G
			\] Comme $(e_1, \ldots, e_p)$ est une base de $F\cap G$, $\exists ! (\lambda_1', \ldots, \lambda_p') \in \mathbbm{K}^p$ tel que \[
				f = \sum_{i=1}^p \lambda'_i e_i = \sum_{i=1}^p \lambda'_i e_i + \sum_{j=1}^q 0_\mathbbm{K}u_j
			\] Comme $(e_1, \ldots, e_p, u_1, \ldots, u_q)$ est une base de $F$, \[
				\forall k \in \left\llbracket 1, q \right\rrbracket, \mu_j = 0_\mathbbm{K}
			\] De même, \[
				\forall k \in \left\llbracket 1,r \right\rrbracket , \nu_k = 0_\mathbbm{K}
			\] On remplace dans $(*)$ pour trouver \[
				\sum_{i=1}^p \lambda_ie_i = 0_E
			\] Comme $(e_1, \ldots, e_p)$ est libre, \[
				\forall i \in \left\llbracket 1,p \right\rrbracket, \lambda_i = 0_\mathbbm{K}
			\] Donc $\mathcal{B}$ est libre.\\
			Donc, 
			\begin{align*}
				\dim(F+G) &=  p +q + r \\
				&= (p+q)+ (p+r) - p \\
				&= \dim(F) + \dim(G) - \dim(F\cap G) \\
			\end{align*}
	\end{itemize}
\end{prv}

\begin{crlr}
	Avec les hypothèse précédentes, \[
		E = F \oplus G \iff \begin{cases}
			F \cap  G = \{0_E\} \\
			\dim(E) = \dim(F) + \dim(G)
		\end{cases}
	\] 
\end{crlr}

\begin{prv}
	\begin{itemize}
		\item[``$\implies$''] On suppose $E = F \oplus G$ \\
			Comme la somme est directe, $F \cap G = \{0_E\}$ 
			\begin{align*}
				\dim(E) &= \dim(F)\\
				&= \dim(F) + \dim(G) - \dim(F\cap G)\\
				&= \dim(F) + \dim(G)\\
			\end{align*}
		\item[``$\impliedby$''] On suppose $F\cap G = \{0_E\}$ et $\dim(E) = \dim(F) + \dim(G)$.\\
			On sait déjà que $F+G = F \oplus G$\\
			 \begin{align*}
				\dim(F+G) = \dim(F) + \dim(G) - \dim(F \cap G) = \dim(E)
			\end{align*}
			Donc $F + G = E$
	\end{itemize}
\end{prv}

\begin{prop}
	Soit $F$ un $\mathbbm{K}$-espace vectoriel de dimension finie $n$. Soit $\mathcal{B} = (e_1, \ldots, e_n)$ une base de $F$. L'application
	\begin{align*}
		f: \mathbbm{K}^n &\longrightarrow F \\
		(\lambda_1, \ldots, \lambda_n) &\longmapsto \sum_{i=1}^n \lambda_i e_i
	\end{align*} est bijective.\\
	Si $\mathbbm{K}$ est infini, $\mathbbm{K}^n$ aussi et donc $F$ aussi.\\
	Si $\#\mathbbm{K} = p \in \N_*$,
	\begin{align*}
		\#&\mathbbm{K}^n = p^n\\
		&\vrt=\\
		\#&F
	\end{align*}
\end{prop}


		\part{Dérivation}

\underline{Motivation}:

{
\begin{wrapfigure}{l}{3cm}
	\centering
	\begin{asy}
		import three;

		size(3cm);
		settings.render=0;
		settings.prc=false;
		currentprojection = obliqueZ;

		draw(unitbox);
		draw(shift(1.1Z + 0.05X) * (O -- X), Arrows3(TeXHead2));
		draw(shift(1.1Z + 0.05Y) * (O -- Y), Arrows3(TeXHead2));
		draw(shift(1.1X + 0.05Z) * (O -- Z), Arrows3(TeXHead2));

		label("$x$", (X/2) + (1.1Z + 0.05X), align=S);
		label("$y$", (Y/2) + (1.1Z + 0.05Y), align=W);
		label("$z$", (Z/2) + X, align=SE);
	\end{asy}
\end{wrapfigure}

\begin{align*}
	&S(x,y,z) = 2(xy + xz + yz)\\
	&V(x,y,z) = xyz
\end{align*}

On cherche à minimiser $S$ avec la contrainte $V = 1$.

Soit $f : \begin{array}{rcl}
	\left( \R_*^+ \right)^2 &\longrightarrow& \R \\
	(x,y) &\longmapsto& S\left( x,y,\frac{1}{xy} \right) = 2\left( xy + \frac{1}{y} + \frac{1}{x} \right).
\end{array}$

On cherche $(a,b) \in \left( \R^+_* \right)^2$ tel que \[
	\forall (x,y) \in (\R^+_*), f(x,y) \ge f(a,b).
\]
}

\begin{defn}
	Soit $f: U \to \R$ où $U$ est un ouvert de $\R^2$. Soit $(a,b) \in U$.
	\vspace{2mm}

	Si $\lim_{x \to a} \frac{f(x,b) - f(a,b)}{x - a} \in \R$, alors on dit que $f$ a une dérivée partielle suivant $x$ en $(a,b)$ et cette limite est notée \[
		\partial f_1(a,b) = \frac{\partial f}{\partial x}(a,b).
	\]

	Si $\lim_{y \to b} \frac{f(a,y) - f(a,b)}{y - b} \in \R$, alors on dit que $f$ a une dérivée partielle suivant $y$ et la limite est notée \[
		\partial f_2(a,b) = \frac{\partial f}{\partial y}(a,b).
	\]
\end{defn}

\begin{exm}
	\begin{enumerate}
		\item $f: (x,y) \mapsto xy + x - y$.

			\begin{align*}
				&\frac{\partial f}{\partial x} : (x,y) \mapsto y + 1,\\
				&\frac{\partial f}{\partial y} : (x,y) \mapsto x - 1.
			\end{align*}

		\item $f: (x,y) \mapsto xy + \frac{1}{y}+ \frac{1}{x}$.

			\begin{align*}
				&\frac{\partial f}{\partial x}: (x,y) \mapsto y - \frac{1}{x^2},\\
				&\frac{\partial f}{\partial y}: (x,y) \mapsto x - \frac{1}{y^2}.
			\end{align*}

		\item Trouver $f$ telle que $\begin{cases}
				(1): \qquad \frac{\partial f}{\partial x}=y,\\[2mm]
				(2): \qquad \frac{\partial f}{\partial y} = x.
			\end{cases}$

			D'après $(1)$ : \[
				\forall (x,y), \exists C(y) \in \R, f(x,y) = xy + C(y)
			\] et donc \[
				\frac{\partial f}{\partial y}(x,y) = x + C'(y)
			\] donc $C'(y) = 0$ et donc $C$ est constante.

		\item Trouver $f$ telle que $\begin{cases}
			\frac{\partial f}{\partial x} = -y,\\[2mm]
			\frac{\partial f}{ƒ\partial y} = x.
		\end{cases}$

		Ce n'est pas possible !
	\end{enumerate}
\end{exm}

\begin{defn}~\\
	\begin{minipage}{\linewidth}
		\begin{wrapfigure}{r}{4cm}
			\centering
			\vspace{-5mm}
			\begin{asy}
				import three;
				import graph3;
				size(4cm);

				settings.render = 0;
				settings.prc = false;
				currentprojection = obliqueX;

				draw(O -- X, Arrow3(TeXHead2));
				draw(O -- Y, Arrow3(TeXHead2));
				draw(O -- Z, Arrow3(TeXHead2));

				triple f(real x, real y, real z = 0) { return (x,y,cos(x - 0.5) * cos(y - 0.5)/1.2 + 0.15); }

				real inc = 1 / 5;

				for(real x = 0; x <= 1; x += inc) {
					draw(graph(
						new real(real t) { return x; }, // x
						new real(real y) { return y; }, // y
						new real(real y) { return f(x,y).z; }, // z
						0, 1
					), gray);
				}

				for(real y = 0; y <= 1; y += inc) {
					draw(graph(
						new real(real x) { return x; }, // x
						new real(real t) { return y; }, // y
						new real(real x) { return f(x,y).z; }, // z
						0, 1
					), gray);
				}

				path3 path1 = (0.8, 0.2, 0) .. (0.5, 0.5, 0) .. (0.3, 0.7, 0);
				path3 path2 = f(0.8, 0.2, 0) .. f(0.5, 0.5, 0) .. f(0.3, 0.7, 0);
				path3 d = (0.2, 0.3, 0) .. (0.3, 0.4, 0) .. (0.2, 0.7, 0) .. (0.8, 0.9, 0) .. (0.6, 0.2, 0) .. cycle;

				draw(path1, red, Arrow3(TeXHead2));
				draw(path2, red, Arrow3(TeXHead2, position=0.8));

				dot((0.5, 0.5, 0));
				dot(f(0.5, 0.5, 0));
				draw((0.5, 0.5, 0) -- f(0.5, 0.5, 0), dashed);
				draw(d);

				label("$w$", (0.3, 0.7, 0), red, align=SE);
				label("$U$", (0.8, 0.9, 0), align=SE);
			\end{asy}
		\end{wrapfigure}

		Soit $f: U \to \R$ où $U$ est un ouvert. Soit $(a,b) \in U$. Soit $w = (w_1, w_2) \in \R^2$.

		Si 
		\[
			\lim_{t\to 0} \frac{f(a + tw_1, b + tw_2) - f(a,b)}{t}
		\] existe et est réelle, alors on dit que $f$ a une dérivée dans la direction de $w$ et la limite est notée \[
			\mathrm{d}f(w)\,(a,b) = D_w(f)\,(a,b).
		\]
	\end{minipage}
\end{defn}

\begin{exm}
	\begin{align*}
		f: \left( \R_*^+ \right)^2 &\longrightarrow \R \\
		(x,y) &\longmapsto xy+\frac{1}{x}+\frac{1}{y}.
	\end{align*}

	On pose $(a,b) = (1,2)$, $w = (w_1, w_2) = (1,1)$.
	\begin{align*}
		\frac{f(1+t, 2+t) - f(1,2)}{t} &= \frac{1}{t} \left( (1+t)(2+t) + \frac{1}{1+t} + \frac{1}{2+t} - 3 - \frac{1}{2} \right) \\
		&= \frac{1}{t}\left(\cancel 2 + 3t + \po(t) + \cancel 1 - t + \po(t) + \frac{1}{2}\left( \cancel 1 - \frac{t}{2} + \po(t) \right) - \cancel3 - \cancel{\frac{1}{2}} \right) \\
		&= \frac{1}{t} \left( \frac{7}{4} t + \po(t) \right)  \\
		&= \frac{7}{4} + \po(1) \tendsto{t \to 0} \frac{7}{4}. \\
	\end{align*}

	Donc, \[
		\mathrm{d}f(1,1)\,(1,2) = \frac{7}{4}.
	\]
\end{exm}

\begin{rmk}~\\
	\begin{figure}[H]
		\centering
		\begin{asy}
			import solids;
			import graph;
			size(5cm);

			settings.render = 0;
			settings.prc = false;

			path3 par = graph(
				new real(real x) { return x; },
				new real(real x) { return 0; },
				new real(real x) { return x^2; },
				0,3);
			revolution r = revolution(par, axis=Z);

			path3 par2 = graph(
				new real(real x) { return x; },
				new real(real x) { return 0; },
				new real(real x) { return x^2; },
				-3,3);

			draw(r,1,longitudinalpen=nullpen);
			draw(r.silhouette());

			draw((-4, 0, -1) -- (-4, 0, 10) -- (4, 0, 10) -- (4, 0, -1) -- cycle, red);
			draw(par2, deepred);

			draw((4,4.5) -- (7, 4.5), black+0.5mm, Arrow(TeXHead));

			path par2d = graph(new real(real x) { return x^2; }, -3, 3);
			draw(shift((11, 0)) * par2d, deepred);

			dot(O);
			dot((11, 0));
		\end{asy}
	\end{figure}
\end{rmk}


%todo ajouter théorème-définition
\begin{thm}
	Soit $f : U \to \R$, $(a,b) \in U$. On suppose que $\frac{\partial f}{\partial x}$ et $\frac{\partial f}{\partial y}$ existent en $(a,b)$ et sont {\bfseries continues} en $(a,b)$. Alors,
	\begin{align*}
		&\forall (h, k) \in \R^2 \text{ tel que } (a +h, b + k) \in U,\\
		&f(a+ h, b + k) = f(a,b) + h \frac{\partial f}{\partial x}(a,b) + k \frac{\partial f}{\partial y}(a,b) + \po_{(h,k)\to (0,0)}\big(\|(h,k)\|\big).
	\end{align*}

	On dit que $f$ est de classe $\mathcal{C}^1$ si $\frac{\partial f}{\partial x}$ et $\frac{\partial f}{\partial y}$ existent et sont continues.

	\qed
\end{thm}

\begin{rmk}
	En physique, cette formule correspond à : \[
		\mathrm{d}f = \frac{\partial f}{\partial x}\mathrm{d}x + \frac{\partial f}{\partial y} \mathrm{d}y.
	\] En effet :
	\begin{align*}
		\mathrm{d}f &= f(x+ \mathrm{d}x, y + \mathrm{d}y) - f(x,y) \\
		&= \frac{\partial f}{\partial x} \mathrm{d}x + \frac{\partial f}{\partial y} \mathrm{d}y.
	\end{align*}
\end{rmk}

\begin{prop}
	Soit $f: U \to \R$ de classe $\mathcal{C}^1$ en $(a,b) \in U$. Alors, \[
		\forall w = (w_1, w_2) \in \R^2, \mathrm{d}f(w)\,(a,b) = w_1 \frac{\partial f}{\partial x}(a,b) + w_2 \frac{\partial f}{\partial y}(a,b).
	\]
\end{prop}

\begin{prv}
	Soit $w = (w_1, w_2) \in \R^2$. Soit $t \in \R^*$.
	\begin{align*}
		\frac{1}{t}\big(f(a + tw_1, b + tw_2) - f(a,b)\big)
		&= \frac{1}{t} \left( tw_1 \frac{\partial f}{\partial x}(a,b) + tw_2 \frac{\partial f}{\partial y}(a,b) + \po_{t \to 0}\big(\|tw\|\big) \right) \\
		&= w_1 \frac{\partial f}{\partial x}(a,b) + w_2 \frac{\partial f}{\partial y}(a,b) + \po_{t\to 0}(1) \\
		&\tendsto{t\to 0} w_1 \frac{\partial f}{\partial x}(a,b) + w_2\frac{\partial f}{\partial y}(a,b).
	\end{align*}
\end{prv}


\begin{defn}
	Avec les hypothèses précédentes, en posant \[
		\nabla f(a,b) = \left( \frac{\partial f}{\partial x}(a,b), \frac{\partial f}{\partial y}(a,b) \right) 
	\]on obtient \[
		\mathrm{d}f(w)\,(a,b) = \left<w  \mid \nabla f(a,b) \right>
	\] où $\left<\cdot|\cdot \right>$ est le produit scalaire.

	Le vecteur $\nabla f(a,b)$ est appelé \underline{gradient de $f$ en $(a,b)$}.

	Le développement limité à l'ordre 1 de $f$ devient \[
		f\big((a,b)+w\big) = f(a,b) + \left<w \mid \nabla f(a,b) \right> + \po_{w\to 0}(\|w\|)
	\]
\end{defn}

\begin{prop}
	Soit $f : U \to \R$ de classe $\mathcal{C}^1$.

	\begin{figure}[H]
    \centering
    \incfig{gradient}
	\end{figure}

	$\nabla f$ est orthogonal au lignes de niveaux de $f$, son orientation va dans le sens d'une augmentation de $f$.
\end{prop}

\begin{prv}
	Soit $\gamma : I \to U$ une courbe de niveau : \[
		\forall t \in I, f\big(\gamma(t)\big) = \text{cste}.
	\] D'après le lemme suivant : \[
		\forall t \in I, 0 = (f \circ \gamma)'(t) = \mathrm{d}f\big(\gamma'(t)\big)\big(\gamma(t)\big) = \left<\gamma'(t)  \mid \nabla f\big(\gamma(t)\big) \right>
	\] Donc $\nabla f\big(\gamma(t)\big)$ est orthogonal à $\gamma'(t)$.

	Pour tout $t \in I$, on pose $w(t) = t\, \nabla f\big(\gamma(t)\big)$. Donc \[
		f\big(\gamma(t) + w(t)\big) = f\big(\gamma(t)\big) + t \|\nabla f(\gamma(t))\|^2 + \po_{t \to 0}(t)
	\] Pour $t$ assez petit, $f\big(\gamma(t) + w(t)\big) - f\big(\gamma(t)\big)$ est du même signe que $t$.
\end{prv}

\begin{rmk}
	\begin{align*}
		V: \R^3 &\longrightarrow \R \\
		(x,y,z) &\longmapsto -mgz
	\end{align*}
	l'énerge potentielle de pesenteur

	On a donc \[
		\nabla V(x,y,z) = \left( \frac{\partial V}{\partial x}, \frac{\partial V}{\partial y}, \frac{\partial V}{\partial z} \right) = (0, 0, -mg) = \vec{P}.
	\]
\end{rmk}

\begin{lem}
	Soit $f : U \to \R$ de classe $\mathcal{C}^1$, $\gamma : \begin{array}{rcl}
		I &\longrightarrow& U \\
		t &\longmapsto& \big(x(t), y(t)\big)
	\end{array}$ où $x$ et $y$ sont dérivables.

	On pose \[
		\forall t \in I, \gamma'(t) = \big(x'(t), y'(t)\big).
	\] Alors $f \circ \gamma : I \to \R$ est dérivable et
	\begin{align*}
		\forall t \in I, (f \circ \gamma)'(t) &= \mathrm{d}f\big(\gamma'(t)\big) \big(\gamma(t)\big)\\
		&= \left<\gamma'(t)  \mid \nabla f\big(\gamma(t)\big)  \right> \\
		&= x'(t) \frac{\partial f}{\partial x}\big(x(t), y(t)\big) + y'(t) \frac{\partial f}{\partial y}\big(x(t),y(t)\big). \\
	\end{align*}
\end{lem}

\begin{prv}
	On fixe $t \in I$.

	\begin{align*}
		\forall h \neq 0, \frac{f \circ \gamma(t + h) - f \circ \gamma(t)}{h}
		&= \frac{1}{h}\big(f(\gamma(t)) + h\gamma'(t) + \po_{h\to 0}(h) - f(\gamma(t))\big) \\
		&= \frac{1}{h}\bigg(\cancel{f(\gamma(t))} + \left<h\gamma'(t) \mid \nabla f(\gamma(t)) \right> + \po_{h\to 0}(\|h\gamma'(t)\|) - \cancel{f(\gamma(t))}\bigg)\\
		&= \left<\gamma'(t) \mid \nabla f(\gamma(t)) \right> + \po_{h\to 0}(1) \\
		&\tendsto{h\to 0} \left<\gamma'(t)  \mid \nabla f(\gamma(t)) \right>
	\end{align*}
\end{prv}

\begin{defn}
	Soit $f : U \to \R$ de classe $\mathcal{C}^1$ et $(a,b) \in U$. On dit que $(a,b)$ est un \underline{point critique} de $f$ si $\nabla f(a,b) = 0$ i.e. $\frac{\partial f}{\partial x}(a,b) = \frac{\partial f}{\partial y}(a,b) = 0$.

	Dans ce cas, $f(a,b)$ est appelé \underline{valeur critique} de $f$.
\end{defn}

\begin{prop}~\\
	\begin{minipage}{\linewidth}
		\begin{wrapfigure}{r}{3cm}
			\centering
			\vspace{-1cm}
			\begin{asy}
				import solids;
				import graph;
				size(3cm);

				settings.render = 0;
				settings.prc = false;

				path3 par = graph(
					new real(real x) { return x; },
					new real(real x) { return 0; },
					new real(real x) { return -x^2; },
					0,3);
				revolution r = revolution(par, axis=Z);

				draw(r,1,longitudinalpen=nullpen);
				draw(r.silhouette());

				dot("$(a,b)$", O, red, align=N);
				real s = sqrt(2.5);
				path3 g=(s,0,-2.5)..(0,s,-2.5)..(-s,0,-2.5)..(0,-s,-2.5)..cycle;
				draw(g, deepcyan);
			\end{asy}
		\end{wrapfigure}
		Soit $f: U \to \R$ de classe $\mathcal{C}^1$ et $(a,b) \in U$ tel que \[
			\exists r > 0, \forall (x,y) \in B_{(a,b)}(r), f(x,y) \le f(a,b)
		\] Alors $\nabla f(a,b) = (0,0)$.
	\end{minipage}
\end{prop}

\begin{prv}
	Soit $g: x \mapsto f(x,b)$. $g(a)$ est un maximum local de $g$ donc $g'(a) = 0$.

	Or, $g'(a) = \frac{\partial f}{\partial x}(a,b)$

	donc $\frac{\partial f}{\partial x}(a,b) = 0$.

	Soit $h : y \mapsto f(a,y)$. On a de même $h'(b) = 0$.

	Or, $h'(b) = \frac{\partial f}{\partial y}(a,b)$.

	Donc, $\nabla f(a,b) = (0,0)$.
\end{prv}

\begin{rmk}
	Un minimum local est aussi une valeur critique.
\end{rmk}

\begin{figure}[H]
	\centering
	\begin{subfigure}{3cm}
		\centering
		\begin{asy}
			import solids;
			import graph;
			size(3cm);

			settings.render = 0;
			settings.prc = false;

			path3 par = graph(
				new real(real x) { return x; },
				new real(real x) { return 0; },
				new real(real x) { return -x^2; },
				0,3);
			revolution r = revolution(par, axis=Z);

			draw(r,1,longitudinalpen=nullpen);
			draw(r.silhouette());

			dot(O, red);
		\end{asy}
		\caption{Maximum local}
	\end{subfigure}
	\begin{subfigure}{3cm}
		\centering
		\begin{asy}
			import solids;
			import graph;
			size(3cm);

			settings.render = 0;
			settings.prc = false;

			path3 par = graph(
				new real(real x) { return x; },
				new real(real x) { return 0; },
				new real(real x) { return x^2; },
				0,3);
			revolution r = revolution(par, axis=Z);

			draw(r,1,longitudinalpen=nullpen);
			draw(r.silhouette());

			dot(O, red);
		\end{asy}
		\caption{Minimum local}
	\end{subfigure}
	\begin{subfigure}{3cm}
		\centering
		\begin{asy}
			import solids;
			import graph;
			size(3cm);

			settings.render = 0;
			settings.prc = false;
			currentprojection = obliqueZ;

			draw(graph(
				new real(real x) { return x; },
				new real(real x) { return -x^2 / 3; },
				new real(real x) { return 3; },
				-3, 3
			));

			draw(graph(
				new real(real x) { return x; },
				new real(real x) { return -x^2 / 3; },
				new real(real x) { return -3; },
				-3, 3
			));

			draw(graph(
				new real(real x) { return x; },
				new real(real x) { return -x^2 / 3 - 1; },
				new real(real x) { return 0; },
				-3, 3
			));

			draw(graph(
				new real(real x) { return 0; },
				new real(real x) { return x^2 / 9 - 1; },
				new real(real x) { return x; },
				-3, 3
			));

			draw(graph(
				new real(real x) { return -3; },
				new real(real x) { return x^2 / 9 - 4; },
				new real(real x) { return x; },
				-3, 3
			));

			draw(graph(
				new real(real x) { return 3; },
				new real(real x) { return x^2 / 9 - 4; },
				new real(real x) { return x; },
				-3, 3
			));

			dot((0,-1,0), red);
		\end{asy}
		\caption{Point de selle / Point col}
	\end{subfigure}
\end{figure}

\begin{exm}
	On revient à l'exemple donné en introduction : 
	\begin{align*}
		f: \left( \R^*_+ \right)^2 &\longrightarrow \R \\
		(x,y) &\longmapsto 2\left( xy + \frac{1}{x} + \frac{1}{y} \right).
	\end{align*}

	$\left( \R^+_* \right)^2$ est un ouvert de $\R^2$. Soit $(x,y) \in \left( \R^+_* \right)^2$.
	
	On a \[
		\begin{cases}
			\frac{\partial f}{\partial x}(x,y) = 2\left( y - \frac{1}{x^2} \right),\\
			\frac{\partial f}{\partial y}(x,y) = 2\left( x - \frac{1}{y^2} \right).
		\end{cases}
	\]

	\begin{align*}
		&\frac{\partial f}{\partial x}(x,y) = \frac{\partial f}{\partial y}(x,y) = 0\\
		\iff& \begin{cases}
			y = \frac{1}{x^2}\\
			x = \frac{1}{y^2}
		\end{cases}\\
		\iff& \begin{cases}
			y = \frac{1}{x^2}\\
			x = x^4
		\end{cases}\\
		\iff& \begin{cases}
			x = 1\\
			y = 1
		\end{cases}
	\end{align*}

	On vérivie que $f$ présente en effet un minium local en $(1,1)$. \[
		f(1,1) = 6
	\] On fixe $y \in \R^+_*$ et \[
		g : x \mapsto 2\left( xy + \frac{1}{x} + \frac{1}{y} \right).
	\] Donc \[
		\forall x \in \R^+_*, g'(x) = 2\left( y - \frac{1}{x^2} \right).
	\]
	\begin{center}
		\begin{tikzpicture}
			\tkzTabInit{$x$/1,$g'(x)$/1,$g$/2.3}{$0$, $\frac{1}{\sqrt{y}}$, $+\infty$}
			\tkzTabLine{,-,z,+,}
			\tkzTabVar{+/{}, -/$2\left( 2\sqrt{y} +\frac{1}{y} \right)$, +/{}}
		\end{tikzpicture}
	\end{center}
	
	Ainsi, \[
		\forall x \in \R^+_*, \forall y \in \R^+_*, f(x,y) \ge 2\left( 2\sqrt{y} + \frac{1}{y} \right)
	\] Soit $h : y \mapsto 2\sqrt{y} + \frac{1}{y}$. On a \[
		\forall y > 0, h'(y) = \frac{1}{\sqrt{y}} - \frac{1}{y^2} = \frac{y\sqrt{y} - 1}{y^2} = \frac{y^{\frac{3}{2}} - 1}{y^2}
	\]

	\begin{center}
		\begin{tikzpicture}
			\tkzTabInit{$y$/0.7,$h'(y)$/0.7,$h$/1.4}{$0$, $1$, $+\infty$}
			\tkzTabLine{,-,z,+,}
			\tkzTabVar{+/{}, -/$3$, +/{}}
		\end{tikzpicture}
	\end{center}

	Donc, \[
		\forall x,y > 0, f(x,y) \ge 2\times 3 = 6 = f(1,1).
	\]
\end{exm}

\begin{prop}
	[règle de la chaîne]

	Soit $f : \begin{array}{rcl}
		U &\longrightarrow& \R^2 \\
		(x,y) &\longmapsto& f(x,y)
	\end{array}$ de classe $\mathcal{C}^1$ et $U, V$ deux ouverts de $\R^2$.

	Soit $\varphi : \begin{array}{rcl}
		V &\longrightarrow& U \\
		(u,v) &\longmapsto& \varphi(u,v) = \big(x(u,v), y(u,v)\big)
	\end{array}$.

	On suppose que $x$ et $y$ sont de classe $\mathcal{C}^1$ sur $V$.

	Alors,  $f \circ \varphi : \begin{array}{rcl}
		V &\longrightarrow& \R \\
		(u,v) &\longmapsto& f\big(\varphi(u,v)\big)
	\end{array}$ est de classe $\mathcal{C}^1$ et
	\begin{align*}
		\forall (u_0, v_0) \in V, \frac{\partial (f \circ \varphi)}{\partial u}(u_0, v_0)
		&= \frac{\partial f}{\partial x}\big(\varphi(u_0, v_0)\big) \times \frac{\partial x}{\partial u}(u_0, v_0)\\
		&+ \frac{\partial f}{\partial y}\big(\varphi(u_0,v_0)\big) \frac{\partial y}{\partial u}(u_0,v_0)
	\end{align*}
	\begin{align*}
		\forall (u_0, v_0) \in V, \frac{\partial (f \circ \varphi)}{\partial v}(u_0, v_0)
		&= \frac{\partial f}{\partial x}\big(\varphi(u_0, v_0)\big) \times \frac{\partial x}{\partial v}(u_0, v_0)\\
		&+ \frac{\partial f}{\partial y}\big(\varphi(u_0,v_0)\big) \frac{\partial y}{\partial v}(u_0,v_0)
	\end{align*}
\end{prop}

\begin{exm}
	[changement de coordonnées polaires]
	On pose \begin{align*}
		\varphi: \R^+_* \times ]0,2\pi[ &\longrightarrow \R^2\setminus \left( R^+_* \times \{0\} \right) \\
		(r, \theta) &\longmapsto (r \cos \theta, r \sin\theta),
	\end{align*}
	\begin{align*}
		f: \R^2\setminus \left( R^+_* \times \{0\} \right) &\longrightarrow \R \\
		(x,y) &\longmapsto f(x,y),
	\end{align*}
	\begin{align*}
		g: \overbrace{\R^+_* \times ]0, 2\pi[}^{=V} &\longrightarrow \R \\
		(r, \theta) &\longmapsto f(r\cos\theta, r\sin\theta).
	\end{align*}

	\begin{align*}
		\forall (r_0,\theta_0) \in V,&\\[5mm]
		\frac{\partial g}{\partial r}(r_0, \theta_0) &= \frac{\partial f}{\partial x}(r_0\cos\theta_0, r_0\sin\theta_0)\cos\theta_0\\
		&+ \frac{\partial f}{\partial y}(r_0 \cos\theta_0, r_0\sin\theta_0)\sin\theta_0\\
		&= 2r_0\cos^2\theta_0 + 2r_0\sin^2(\theta_0) \\
		&= 2r_0 \\[5mm]
		\frac{\partial g}{\partial \theta}(r_0, \theta_0) &= \frac{\partial f}{\partial x}(r_0\cos\theta_0, r_0\sin\theta_0)r_0\sin\theta_0\\
		&+ \frac{\partial f}{\partial y}(r_0 \cos\theta_0, r_0\sin\theta_0)r_0\cos\theta_0\\
		&= -2{r_0}^2\cos(\theta_0)\sin(\theta_0) + 2{r_0}^2 \sin(\theta_0)\cos(\theta_0)\\
		&= 0 \\
	\end{align*}

	Donc, \[
		g(r, \theta) = r^2.
	\]
\end{exm}

\begin{exm}
	Résoudre \[
		\begin{cases}
			\frac{\partial f}{\partial x} = \frac{x}{x^2+y^2},\\
			\frac{\partial f}{\partial y} = \frac{y}{x^2+y^2}.\\
		\end{cases}
	\]

	On pose $g: (r, \theta) \mapsto f(r \cos\theta, r \sin\theta)$.

	\begin{align*}
		&\frac{\partial g}{\partial r} = \frac{1}{r}\cos^2\theta + \frac{1}{r}\sin^2\theta = \frac{1}{r},\\
		&\frac{\partial g}{\partial \theta} = -\cos(\theta) \sin(\theta) + \sin(\theta)\cos(\theta) = 0.
	\end{align*}

	Donc, \[
		\exists C \in \R, g: (r, \theta) \mapsto \ln r + C
	\] d'où,
	\begin{align*}
		\forall (x,y) \in \R^2 \setminus \{(0,0)\}, f(x,y) &= \ln\left(\sqrt{x^2 + y^2} \right)  + C\\
		&= \frac{1}{2}\ln(x^2 + y^2) + C. \\
	\end{align*}
\end{exm}

\begin{rmk}
	Soit $\mathcal{B} = (e_1, e_2)$ la base canonique de $\R^2$, $f: U \to \R$ de classe $\mathcal{C}^1$ avec $U$ un ouvert de $\R^2$.

	Soit $(x,y) \in U$.

	\begin{align*}
		\Mat_{\mathcal{B}}\big(\nabla f(x,y)\big) = \begin{pmatrix}
			\frac{\partial f}{\partial x}(x,y)\\[2mm]
			\frac{\partial f}{\partial y}(x,y)
		\end{pmatrix}
	\end{align*}

	Soit  \begin{align*}
		\varphi: V &\longrightarrow U \\
		(u,v) &\longmapsto \big(x(u,v), y(u,v)\big) 
	\end{align*} avec $x,y$ de classe $\mathcal{C}^1$. Soit $g = f \circ \varphi$.
	\begin{align*}
		\Mat_{\mathcal{B}}\big(\nabla g(u,v)\big)
		&= \begin{pmatrix}
			\frac{\partial g}{\partial u}(u,v) \\[2mm]
			\frac{\partial g}{\partial v}(u,v)
		\end{pmatrix} \\
		&= \begin{pmatrix}
			\frac{\partial x}{\partial u}(u,v) \frac{\partial f}{\partial x}(x,y)
			+ \frac{\partial y}{\partial u}(u,v)\frac{\partial f}{\partial y}(x,y)\\[3mm]
			\frac{\partial x}{\partial v}(u,v) \frac{\partial f}{\partial x}(x,y)
			+ \frac{\partial y}{\partial v}(u,v) \frac{\partial f}{\partial y}(x,y)
		\end{pmatrix}  \\
		&= \underbrace{\begin{pmatrix}
				\frac{\partial x}{\partial u}(u,v)& \frac{\partial y}{\partial u}(u,v)\\[3mm]
				\frac{\partial x}{\partial v}(u,v)& \frac{\partial y}{\partial v}(u,v)
		\end{pmatrix}}_{J(u,v)} \begin{pmatrix}
			\frac{\partial f}{\partial x}(x,y)\\[3mm]
			\frac{\partial f}{\partial y}(x,y)
		\end{pmatrix} \\
		&= J(u,v) \Mat_{\mathcal{B}}\big(\nabla f(x,y)\big) \\
	\end{align*}
	où $J(u,v) = 
	\begin{pNiceArray}{c:c}
		\Mat_{\mathcal{B}}\big(\nabla x(u,v)\big) & \Mat_{\mathcal{B}}\big(\nabla y(u,v)\big)
	\end{pNiceArray}$.

	On dit que $J(u,v)$ est \underline{la jacobienne} de $\varphi$ en $(u,v)$.
	L'application linéaire canoniquement associée à $J(u,v)$ est la \underline{différentielle de $\varphi$} en $(u,v)$ noté $\mathrm{d}\varphi(u,v)$.

	On a $\mathrm{d}\varphi(u,v) \in \mathcal{L}(R^2)$ et $\Mat_{\mathcal{B}}\big(\mathrm{d}\varphi(u,v)\big) = J(u,v)$.

	Par exemple, la jacobienne du changement de coordonnées polaires est \[
		J = \begin{pmatrix}
			\frac{\partial x}{\partial r} & \frac{\partial y}{\partial r}\\[3mm]
			\frac{\partial x}{\partial \theta} & \frac{\partial y}{\partial \theta}
		\end{pmatrix}
		= \begin{pmatrix}
			\cos\theta&\sin\theta\\
			-r\sin\theta&r\cos\theta
		\end{pmatrix}.
	\]
	$\underbrace{\det(J)}_{\text{le jacobien}} = r\cos^2\theta + r\sin^2\theta = r$

	Dans une intégrale double, si $(x,y) = \varphi(u,v)$, alors $\mathrm{d}x\mathrm{d}y = \det(J)\mathrm{d}u\mathrm{d}v$.

	Ici, \[
		\mathrm{d}x\ \mathrm{d}y = r\ \mathrm{d}r\ \mathrm{d}\theta.
	\]
\end{rmk}

\begin{prv}
	On pose $(x_0, y_0) = \varphi(u_0, v_0)$. Pour tout $(h,k) \in \R^2$ tels que $(u_0 + h, v_0 + k) \in V$, en posant $g = f  \circ \varphi$.

	\begin{align*}
		g(u_0 + h, v_0 + h) &= f\big(x(u_0 + h, v_0 + k), y(u_0 + h, v_0 + k)\big) \\
		&= f\left(
			x(u_0,v_0) + h \frac{\partial x}{\partial u}(u_0,v_0) + k \frac{\partial x}{\partial v}(u_0, v_0) + \po\big(\|(h,k)\|\big), \right.\\
		&\phantom{ = f\bigg(\bigg.}\left. y(u_0, v_0) + h \frac{\partial y}{\partial u}(u_0, v_0) + k \frac{\partial y}{\partial v}(u_0, v_0) + \po\big(\|(h,k)\|\big)
		\right)  \\
		&= f(x_0,y_0) \\
		&~+ \left( h \frac{\partial x}{\partial u}(u_0,v_0) + k \frac{\partial x}{\partial v}(u_0, v_0) + \po(\|(h,k)\|) \right) \frac{\partial f}{\partial x}(x_0,y_0)\\
		&~+ \left( h \frac{\partial y}{\partial u}(u_0, v_0) + k\frac{\partial y}{\partial v}(u_0, v_0) + \po(\|(h,k)\|) \right) \frac{\partial f}{\partial y}(x_0, y_0)\\
		&~+ \po(\|(h,k)\|)\\
		&= f(x_0, y_0) \\
		&~+ h \left( \frac{\partial x}{\partial u}(u_0, v_0) \frac{\partial f}{\partial x}(x_0, y_0) + \frac{\partial y}{\partial u}(u_0, v_0) \frac{\partial f}{\partial y}(x_0, y_0) \right)  \\
		&~+ k\left( \frac{\partial x}{\partial v}(u_0, v_0) \frac{\partial f}{\partial x}(x_0, y_0) + \frac{\partial y}{\partial v}(u_0, v_0) \frac{\partial f}{\partial y}(x_0, y_0) \right) 
		&~+ \po(\|(h,k)\|)\\
		&= g(u_0, v_0) + h \frac{\partial g}{\partial u}(u_0, v_0) + k \frac{\partial g}{\partial v}(u_0, v_0) + \po(\|(h,k)\|) \\
	\end{align*}

	Par identification,
	\[
		\frac{\partial g}{\partial u}(u_0, v_0) = \frac{\partial x}{\partial u}(u_0, v_0) \frac{\partial f}{\partial x}(x_0, y_0) + \frac{\partial y}{\partial u}(u_0, v_0) \frac{\partial f}{\partial y}(x_0,y_0)
	\] et \[
		\frac{\partial g}{\partial v}(u_0, v_0) = \frac{\partial x}{\partial v}(u_0,v_0) \frac{\partial f}{\partial x}(x_0, y_0) + \frac{\partial y}{\partial v}(u_0, v_0) \frac{\partial f}{\partial y}(x_0, y_0).
	\] 
\end{prv}

\begin{exm}
	[Régression linéaire]~\\
	\begin{figure}[H]
		\centering
		\begin{asy}
			import graph;
			axes(EndArrow);
			size(5cm);

			real f(real x) { return x + 0.5; }

			real k = 35 / (7 - 0.5);

			for(int i = 0; i < 35; ++i) {
				real mag = exp(sin(100 * pi/exp(1) * i)) * 0.8 + exp(cos(i*40)/3);
				real eps = mag * cos(10 * exp(1)/pi * i) / 3;
				dot((i/k,f(i/k) + eps));
			}

			draw(graph(f, -1, 7), orange);
		\end{asy}
	\end{figure}
	\[
		y = a x + b
	\] 
	On fixe $(a,b) \in \R^2$. \[
		\varepsilon(a,b) = \sum_{i=1}^n\big( y_i - (ax_i + b) \big)^2
	\] l'erreur totale.

	On veut minimiser $\varepsilon(a,b)$. On a 
	\[
		\forall (a,b) \in \R^2,
		\begin{cases}
			\frac{\partial \varepsilon}{\partial a}(a,b) = -2\sum_{i=1}^{n}(y_i - ax_i - b)x_i,\\
			\frac{\partial \varepsilon}{\partial b}(a,b) = -2\sum_{i=1}^{n}(y_i - ax_i - b).
		\end{cases}
	\]

	Donc,
	\begin{align*}
		(a,b) \text{ point critique de } \varepsilon \iff& \begin{cases}
			a \sum_{i=1}^n {x_i}^2 + b\sum_{i=1}^{n}x_i = \sum_{i=1}^{n} y_ix_i\\
			a\sum_{i=1}^{n}x_i + nb = \sum_{i=1}^ny_i
		\end{cases}\\
		\iff& \begin{cases}
			a \left( \frac{1}{n}\sum_{i=1}^n {x_i}^2 - \overline{x}^2\right) = \overline{y} - \overline{x} \overline{y}\\
			b = \frac{1}{n}\sum_{i=1}^ny_i - \frac{a}{n}\sum_{i=1}^nx_i = \frac{1}{n}\sum_{i=1}^n x_i y_i - \overline{x} \overline{y}
		\end{cases}\\
		&\text{ où } \overline{x} = \frac{1}{n} \sum_{i=1}^n x_i,~\overline{y} = \frac{1}{n}\sum_{i=1}^n y_i\\
		\iff& \begin{cases}
			a = \frac{\Cov(x,y)}{V(x)}\\
			b = \overline{y} - a\overline{x}
		\end{cases}
	\end{align*}

	Coefficient de corrélation: $\frac{\Cov(x,y)}{\sigma_x \sigma_y} \in [-1, 1]$
\end{exm}












		\part{Corps}

\begin{exm}[Problème]
	\begin{itemize}
		\item 
			avec $A = \Z / 9 \Z$, résoudre $\overline{x}^2 = \overline{0}$ \\
			\begin{center}
				\begin{tabular}{|c|c|c|c|c|c|c|c|c|c|c|}
					\hline
					$\overline{x}$&$\overline{0}$& $\overline{1}$ &$\overline{2}$&$\overline{3}$ &$\overline{4}$ &$\overline{5}$ &$\overline{6}$ &$\overline{7}$ &$\overline{8}$& $\overline{9}$ \\
					\hline
					$\overline{x}^2$&$\overline{0}$ &$\overline{1}$ &$\overline{4}$ &$\overline{0}$ &$\overline{7}$ &$7$ &$\overline{0}$ &$\overline{4}$ &$\overline{1}$&$\overline{0}$\\
					\hline
				\end{tabular}
			\end{center}
			On a trouvé 3 solutions: $\overline{0}$, $\overline{3}$, $\overline{6}$.
		\item $\Z / 8\Z$
			\begin{center}
				\begin{tabular}{|c|c|c|c|c|c|c|c|c|}
					\hline
					$\overline{x}$& $\overline{0}$& $\overline{1}$& $\overline{2}$& $\overline{3}$& $\overline{4}$& $\overline{5}$& $\overline{6}$& $\overline{7}$\\
					\hline
					$\overline{x^2}$& $\overline{0}$& $\overline{1}$& $\overline{4}$& $\overline{1}$& $\overline{0}$& $\overline{1}$& $\overline{4}$& $\overline{1}$\\
					\hline
				\end{tabular}
			\end{center}
			$\overline{x}^2=7$ a 4 solutions: $\overline{1}, \overline{7}, \overline{3},\text{ et } \overline{5}$
		\item $A = \mathbbm{H} = \{a + bi + cj + dk  \mid  (a,b,c,d) \in \R^4\}$ \\
			$i^2 = j^2 = k^2 = -1$ 
			\begin{align*}
				\begin{array}{c c c}
					ij = k & jk = i & ji = j\\
					ji = -k & kj = -i & ik = -j
				\end{array}
			\end{align*}
			Dans cet anneau, $-1$ a 6 racines!
	\end{itemize}
\end{exm}

\begin{defn}
	Soit $(\mathbbm{K}, +, \times)$ un ensemble muni de deux lois de composition internes. On dit que c'est un \underline{corps} si
	 \begin{enumerate}
		\item $(\mathbbm{K}, \times)$ est un groupe abélien
		\item $(\mathbbm{K}, \times)$ est un monoïde commutatif
		\item $\forall x \in \mathbbm{K}\setminus \{0_\mathbbm{K}\}, \exists y \in \mathbbm{K}, xy = 1_\mathbbm{K}$
		\item $0_\mathbbm{K} \neq  1_\mathbbm{K}$
	\end{enumerate}
	\index{corps}
\end{defn}

\begin{exm}
	\begin{itemize}
		\item $(\C, +, \times)$ est un corps
		\item $(\R, +, \times)$ est un corps
		\item $(\Q, +, \times)$ est un corps
		\item $(\Z, +, \times)$ n'est pas un corps
	\end{itemize}
\end{exm}

\begin{prop}
	$(\Z / n\Z, +, \times)$ est un corps si et seulement si $n$ est premier.
\end{prop}

\begin{prv}
	\[
		\left( \Z / n\Z \right)^\times = \left\{ \overline{k}  \mid k \wedge n = 1 \right\}
	\] 
\end{prv}


\begin{prop}
	Tout corps est un anneau intègre.
\end{prop}

\begin{prv}
	Soit $(\mathbbm{K}, +, \times)$ un corps. Soient $(a,b) \in \mathbbm{K}^2$ tel que $a \times b = 0_\mathbbm{K}$.\\
	On suppose $a \neq  0_\mathbbm{K}$. Alors, $a$ est inversible et donc \[
		b = a^{-1} \times a \times b = a^{-1} \times 0_\mathbbm{K} = 0_\mathbbm{K}
	\] 
\end{prv}

\begin{exm}
	Soit $(\mathbbm{K},+,\times)$ un corps.\\
	Résoudre \[
		\begin{cases}
			x^2 = 1_\mathbbm{K}\\
			x \in \mathbbm{K}
		\end{cases}
	\]

	\begin{align*}
		x^2 = 1_\mathbbm{K} &\iff x^2 - 1_\mathbbm{K} = 0_\mathbbm{K}\\
		&\iff (x - 1_\mathbbm{K})(x+1_\mathbbm{K}) = 0_\mathbbm{K}\\
		&\iff x - 1_\mathbbm{K} = 0_\mathbbm{K} \text{ ou } x + 1_\mathbbm{K} = 0_\mathbbm{K}\\
		&\iff x = 1_\mathbbm{K} \text{ ou } x = -1_\mathbbm{K}
	\end{align*}

	Il y a au plus 2 solutions.
\end{exm}

\begin{prop}
	Soit $(\mathbbm{K},+,\times )$ un corps et $P$ un polynôme à coefficients dans $\mathbbm{K}$ de degré $n$. Alors, l'équation $P(x) = 0_{\mathbbm{K}}$ a au plus $n$ solutions dans $\mathbbm{K}$ 
	\qed
\end{prop}

\begin{crlr}[(Théorème de Wilson)]
	voir exercice 16 du TD 12
\end{crlr}


\begin{defn}
	Soit $(\mathbbm{K}, +, \times)$ un corps et $L\subset \mathbbm{K}$.\\
	On dit que $L$ est un \underline{sous corps} de $\mathbbm{K}$ si
	\begin{enumerate}
		\item $L$ est un anneau de $(\mathbbm{K}, +, \times)$ non nul
		\item $\forall x \in L\setminus \{0_\mathbbm{K}\}, x^{-1} \in L$ 
	\end{enumerate}
	\vspace{2mm}
	en d'autres termes si
	\begin{enumerate}
		\item $\forall (x,y) \in L^2, x - y \in L$
		\item $\forall (x,y) \in L^2, x \times y^{-1} \in L$
	\end{enumerate}
	\vspace{5mm}
	On dit aussi que $\mathbbm{K}$ est une \underline{extension} de $L$.
	\index{sous corps}
	\index{extension}
\end{defn}

\begin{prop}
	Tout sous corps est un corps. \qed
\end{prop}

\begin{defn}
	Soient $(\mathbbm{K}_1,+,\times )$ et $(\mathbbm{K}_2,+, \times)$ deux corps et $f: \mathbbm{K}_1 \to \mathbbm{K}_2$.\\
	On dit que $f$ est un \underline{morphisme de corps} si $f$ est un morphisme d'anneaux.\\
	i.e. si
	\[
		\begin{cases}
			\forall (x,y) \in {\mathbbm{K}_1}^2,& f(x+y) = f(x) + f(y)\\
			\forall (x,y) \in {\mathbbm{K}_1}^2,& f(x \times y) = f(x) \times f(y)\\
		\end{cases}
	\] 
	\index{homomorphisme (de corps)}
	\index{morphisme (de corps)}
\end{defn}

\begin{prop}
	Tout morphisme de corps est injectif.
\end{prop}

\begin{prv}
	Soit $f: \mathbbm{K}_1 \to \mathbbm{K}_2$ un morphisme de corps.\\
	\begin{itemize}
		\item $\Ker(f)$ est un sous groupe de $(\mathbbm{K}_1, +)$ 
		\item Soit $x \in \Ker(f)$ et $y \in \mathbbm{K}_1$ \[
				f(x \times y) = f(x) \times f(y) = 0_{\mathbbm{K}_2} \times f(y) = 0_{\mathbbm{K}_2}
			\]
		\item Soit $x \in \Ker(f) \setminus \{0_{\mathbbm{K}_1}\}$.\\
			Alors, $x$ est inversible.\\
			\begin{align*}
				\begin{rcases*}
					x \in \Ker(f)\\
					x^{-1} \in \mathbbm{K}_1
				\end{rcases*}& \text{ donc } x \times x ^{-1} \in \Ker(f)\\
				&\text{ donc } 1_{\mathbbm{K}_1} \in \Ker(f)\\
				&\text{ donc } f(1_{\mathbbm{K}_1}) = 0_{\mathbbm{K}_2}
			\end{align*}
			Or, $f(1_{\mathbbm{K}_1}) = 1_{\mathbbm{K}_2} \neq 0_{\mathbbm{K}_2}$
	\end{itemize}
	Donc, $\Ker(f) = \{0_{\mathbbm{K}_1}\}$ donc $f$ est injective.
\end{prv}

\begin{exm}
	$\begin{array}{cc}
		\C &\longrightarrow \C\\
		z &\longmapsto \overline{z}\\
	\end{array}$ est un morphisme de corps
\end{exm}



		\part{Opérations sur les séries}

\begin{prop}
	L'ensemble $E = \{u \in \C^\N  \mid \Sigma u_n \text{ converge}\}$ est un sous-espace vectoriel de $\C^\N$ et \begin{align*}
		S: E &\longrightarrow \C \\
		u &\longmapsto \sum_{n=0}^{+\infty} u_n
	\end{align*} est une forme linéaire.
	\qed
\end{prop}

\begin{rmk}
	La somme d'une série convergente et d'une série divergente diverge.
	Le produit d'une série divergente par un scalaire non nul diverge.
\end{rmk}

		\part{Comparaison de suites}

\begin{defn}
	Soient $u$ et $v$ deux suites réelles. On dit que $u$ est \underline{dominée} par  $v$ si \[
	\exists M\in \R, \exists N\in \N,\forall n\ge N,\left| u_n \right| \le M \left| v_n \right| 
	\] Dans ce cas, on note $u = O(v)$ ou $u_n = O(v_n)$ et on dit que "$u$ est un grand o de $v$"
\end{defn}

\begin{exm}
	En informatique, on dit qu'un alogirithme a une \underline{complexité linéaire} si son temps d'éxécution est un $O(n)$ 
	Par exemple, on calcule $a^n$ 

	\begin{itemize}
		\item Approche naïve
			\begin{algorithm}
				\begin{algorithmic}[1]
					\State $p \gets 1$
					\For{$i \in \left\llbracket 0,n-1 \right\rrbracket$}
						\State $p \gets p \times a$
					\EndFor
					\State \Return p
				\end{algorithmic}
			\end{algorithm}
			Complexité linéaire $O(n)$
		\item Exponentiation rapide\\
			On écrit $n$ en binaire: \begin{align*}
				n &= \overline{a_k a_{k-1}\ldots a_0}^{(2)}\\
					&= \sum_{i=0}^{k} a_i 2^i
			\end{align*} avec $(a_i) \in \left\{ 0,1 \right\} ^{k+1}$
			\begin{align*}
				a^n &= a^{\sum_{i=0}^{k} a_i 2^i} \\
				&= \prod_{i=0}^{k} a^{a_i 2^i}  \\
			\end{align*}
			
			\begin{algorithm}
				\begin{algorithmic}
					[1]

					\State $s \gets 0$
					\State $p \gets a$
					\For{ $i \in \left\llbracket 0, \log_2(n) \right\rrbracket$}
						\State $p \gets p \times p$
						\If{$a[i] = 1$}
							\State $s \gets s + p$
						\EndIf
					\EndFor
					\State \Return s
				\end{algorithmic}
			\end{algorithm}
			Compléxité logarithmique $O(\log_2(n))$
	\end{itemize}
\end{exm}


\begin{prop}
	$O$ est une relation réfléctive et transitive.
\end{prop}

\begin{prv}
	\begin{itemize}
		\item Soit $u$ une suite. On pose $M = 1$ et \[
			\forall n \in \N, \left| u_n \right| \le M \left| u_n \right|
			\] Donc $u = O(u)$.
		\item Soient $u, v, w$ trois suites telles que  \[
		\begin{cases}
			u = O(v)\\
			v = O(w)
		\end{cases}
		\] Soient $M_1,M_2 \in \R$ et $N_1,N_2\in \N$ tels que \[
		\begin{cases}
			\forall n \ge  N_1, \left| u_n \right| \le M_1 \left| v_n \right| \\
			\forall n \ge  N_2, \left| v_n \right| \le M_2 \left| w_n \right| \\
		\end{cases}
		\] 

		Nécéssairement, $M_1\ge 0$ et $M_2\ge 0$.\\
		Soit $N = \max(N_1,N_2)$. \[
		\forall n \ge  N, \left| u_n \right| \le M_1 \left| v_n \right| \le  M_1M_2 \left| w_n \right| 
		\] Donc $u = O(w)$
	\end{itemize}
\end{prv}

\begin{defn}
	Soient $u$ et $v$ deux suites. On dit que $u$ est \underline{négligeable} devant $v$ si \[
	\forall \varepsilon>0, \exists N\in \N, \forall n\ge N, \left| u_n \right| \le \varepsilon \left| v_n \right| 
	\] Dans ce cas, on note $u = o(v)$ ou $u_n = o(v_n)$ ou on le lit "$u$ est un petit o de $v$"
\end{defn}

\begin{prop}
	$o$ est une relation transitive, non-réfléctive
\end{prop}

\begin{prv}
	\begin{itemize}
		\item Soient $u$, $v$ et $w$ trois suites telles que \[
			\begin{cases}
				u = o(v)\\
				v = o(w)
			\end{cases}
			\] Soit $\varepsilon>0$. Soit $N_1\in \N$ tel que \[
			\forall n \ge N_1, \left| u_n \right| \le \sqrt{\varepsilon}  \left| v_n \right| 
			\] Soit $N_2\in \N$ tel que \[
			\forall n \ge N_2, \left| v_n \right| \le \sqrt{\varepsilon}  \left| w_n \right| 
			\] On pose $N = \max(N_1,N_2)$, alors \[
			\forall n \ge N, \left| u_n \right| \le \sqrt{\varepsilon}  \left| v_n \right| \le \underbrace{\sqrt{\varepsilon} \times \sqrt{\varepsilon}} _\varepsilon \left| w_n \right| 
			\] donc $u = o(w)$
		\item Soit $u$ une suite tel qu'il existe $N \in \N$ tel que \[
		\forall n \ge N, u_n > 0
		\] On suppose que $u = o(u)$, alors \[
		\forall \varepsilon>0,\exists N \in \N, \forall n \ge N, \left| u_n \right| \le \varepsilon \left| u_n \right| 
		\] On pose $\varepsilon = \frac{1}{2}$ alors \[
		\exists N \in \N, \forall n \ge N, \left| u_n \right| \le \frac{1}{2} \left| u_n \right| 
		\] une contradiction
	\end{itemize}
\end{prv}

\begin{prop}
	Soient $u$ et $v$ deux suites.
	\begin{itemize}
		\item $o(u) + o(u) = o(u)$
		\item $v \times o(u) = o(uv)$
		\item $o(u) \times o(v) = o(uv)$
		\item $o(o(u)) = o(u)$
	\end{itemize}
	\qed
\end{prop}

\begin{defn}
	Soient $u$ et $v$ deux suites. On dit que $u$ et $v$ sont \underline{équivalentes} si \[
	u = v + o(v)
	\] i.e. \[
	\forall \varepsilon >0, \exists N \in \N, \forall n \ge N, \left| u_n-v_n \right| \le \varepsilon\left| v_n \right| 
	\] Dans ce cas, on le note $u \sim v$
\end{defn}

\begin{prop}
	$\sim$ est une relation d'équivalence \qed
\end{prop}

\begin{prop}
	Soient $(u,v) \in \R^\N$. On suppose que $v$ ne s'annule pas à partir d'un certain rang
	\begin{enumerate}
		\item $u = o(v) \iff \left( \frac{u_n}{v_n} \right)$ bornée
		\item $u = o(v) \iff \frac{u_n}{v_n} \tendsto{n \to  +\infty} 0$
		\item $u \sim v \iff \frac{u_n}{v_n} \tendsto{n \to  +\infty} 1$
	\end{enumerate}
	\qed
\end{prop}

\begin{prop}
	[Suites de références]
	\begin{enumerate}
		\item $\ln^\alpha(n) = o(n^\beta)$ avec $(\alpha,\beta) \in \left( \R^+_* \right) ^2$ 
		\item $n^\beta = o(a^n)$ avec $\beta > 0$ et $a > 1$ 
		\item $a^n = o(n!)$ avec $a >1$ 
		\item $n! = o(n^n)$
	\end{enumerate}
\end{prop}


\begin{lem}
	[Exercice 10 du TD]
	Soit $u \in \left(\R^+_*\right)^\N$\\
	Si $\frac{u_{n+1}}{u_n} \tendsto{n \to +\infty} \ell < 1$ avec $\ell\in \R$,\\ alors $u_n \tendsto{n \to +\infty} 0$
\end{lem}

\begin{prv} [de la proposition]
	\begin{enumerate}
		\item par croissance comparée
		\item On pose $\forall n \in \N^*, u_n = \frac{n^\beta}{a^n}$. 
			\begin{align*}
				\forall  n \in \N^*, \frac{u_{n+1}}{u_n} &= \left( \frac{n+1}{n} \right) ^\beta \times \frac{1}{a} \\
				&= \frac{1}{a}\left( 1+\frac{1}{n} \right) ^\beta \\
				&\tendsto{n \to +\infty} \frac{1}{a} < 1
			\end{align*}
			Donc, $u_n \tendsto{n \to  +\infty} 0$
		\item On pose $\forall n \in \N, u_n = \frac{a^n}{n!}$ \[
			\forall n \in \N, \frac{u_{n+1}}{u_n} = \frac{a}{n+1} \tendsto{n \to +\infty} 0 < 1
			\] donc $u_n \tendsto{n \to +\infty} 0$
		\item On pose $\forall  n\in \N^*, u_n = \frac{n!}{n^n}$.
			\begin{align*}
				\forall n \in \N^*, \frac{u_{n+1}}{u_n}
				&= (n+1) {\frac{n^n}{(n+1)^{n+1}}} \\
				&= \left( \frac{n}{n+1} \right) ^n \\
				&= e^{n \ln\left( \frac{n}{n+1} \right) } \\
				&= e^{n \ln\left( 1+\frac{1}{n+1} \right)} \\
				&= e^{n(-\frac{1}{n} + o(\frac{1}{n})} \\
				&= e^{-1 + o(1)} \\
				&\tendsto{n \to  +\infty} e^{-1}<1
			\end{align*}
			donc $u_n \tendsto{n\to +\infty} 0$
	\end{enumerate}
\end{prv}

		\part{Matrices par blocs}

\begin{exm}
	Soit $p$ un projecteur de $E$ : \[
		E = \Ker p \oplus \mathrm{Im}\ p
	\] Soit $\mathcal{B} = (e_1, \ldots, e_k, e_{k+1}, \ldots, e_n)$ une base de $E$ avec $\begin{cases}
		\mathrm{Im}(p) = \Vect(e_1, \ldots, e_k)\\
		\Ker(p) = \Vect(e_{k+1}, \ldots, e_n)\\
	\end{cases}$

	Alors, 
	\begin{align*}
		\Mat_\mathcal{B}(p) =
		\left(\begin{NiceArray}{c c c | c c c}
				1&&&0&\Cdots&0\\
				 &\Ddots&&\Vdots&&\Vdots\\
				&&1&0&\Cdots&0\\\hline
				0&\Cdots&0&0&\Cdots&0\\
				\Vdots&&\Vdots&\Vdots&&\Vdots\\
				0&\Cdots&0&0&\Cdots&0\\
		\end{NiceArray}\right)
		= \left( \begin{array}{c|c}
				I_k & 0\\ \hline
				0&0
		\end{array}\right) \\
	\end{align*}

	De même, si $\s$ est une symétrie de $E$, \[
		E = \Ker(\s - \id_E) \oplus \Ker(\s + \id_E)
	.\] Soit $\mathcal{C} = (e_1', \ldots, e_\ell', e_{\ell+1}', \ldots, e'_n)$ avec $\begin{cases}
		\Vect(e'_1, \ldots, e'_\ell) = \Ker(\s - \id_E),\\
		\Vect(e'_{\ell+1}, \ldots, e'_n) = \Ker(\s + \id_E).\\
	\end{cases}$

	Alors
	\[
		\Mat_\mathcal{C}(\s) = \left(\begin{array}{c|c}
				I_\ell &0\\ \hline
				0&-I_{n-\ell}
		\end{array}\right) 
	\]
\end{exm}

\begin{prop}
	Soient $F$ et $G$ supplémentaires dans $E$ : \[
		E = F \oplus G.
	\] Soit $f \in \mathcal{L}(F)$ et $g \in \mathcal{L}(G)$. Alors \[
	\exists !h \in \mathcal{L}(E) h_{|F} = f,\ h_{|G} = g \et h = f \circ p + g \circ q
	\] où $\begin{cases}
		p \text{ est la projection sur $F$ parallèlement à $G$}\\
		q \text{ est la projection sur $G$ parallèlement à $F$}\\
	\end{cases}$.

	On a aussi $q = \id_E - p$.
\end{prop}

\begin{prv}
	\begin{itemize}
		\item[\sc \underline{Analyse}] Soit $h \in \mathcal{L}(E)$ tel que $\begin{cases}
				h_{|F}=f\\
				h_{|G}=g
			\end{cases}$.

			Soit $x \in E$. Alors \[
				x = \underbrace{p(x)}_{\in F} + \underbrace{q(x)}_{\in G}
			\]

			Donc,
			\begin{align*}
				h(x) &= h\big(p(x)\big) + h\big(q(x)\big)\\
				&= f\big(p(x)\big) + g\big(q(x)\big) \\
				&= (f \circ p + g \circ q)(x) \\
			\end{align*}
			Si $h$ existe, alors \[
				h = f \circ p + g \circ q
			\]
		\item[\underline{\sc Synthèse}] On pose $h = f \circ p + g  \circ q$.

			$p$, $q$, $f$ et $g$ sont linéaires donc $h$ aussi.

			Soit $x \in E$.
			\begin{align*}
				h(x) &= f\big(p(x)\big) + g\big(q(x)\big) \\
				&= f(x) + g(0_E) \\
				&= f(x) \\
			\end{align*}
			donc $h_{|F} = f$ et de même $h_{|G}=g$.
	\end{itemize}
\end{prv}

\begin{prop}
	On reprend les notations et hypothèses précédentes. Soit $(e_1, \ldots, e_p)$ une base de $F$, et $(f_1, \ldots, f_q)$ une base de $G$. Alors, $\mathcal{B} = (e_1, \ldots, e_p, f_1, \ldots, f_q)$ est une base de $E$ et \[
		\Mat_\mathcal{B}(h) = \left(
		\begin{array}{c|c}
			A&0\\ \hline
			0&B
		\end{array}\right)
	\] où $\begin{cases}
		A = \Mat_{(e_1, \ldots e_p)}(f)\\
		B = \Mat_{(f_1, \ldots, f_q)}(g)
	\end{cases}$
	\qed
\end{prop}

\begin{prop}
	Soient $(A,A') \in \mathcal{M}_n(\mathbbm{K})^2$ et $(B,B') \in \mathcal{M}_p(\mathbbm{K})^2$.
	\begin{enumerate}
		\item \[
				\left(\begin{array}{c|c}
					A&0\\ \hline
					0&B
				\end{array}\right)
				\left(\begin{array}{c|c}
					A'&0\\ \hline
					0&B'
				\end{array}\right) = 
				\left(\begin{array}{c|c}
					AA'&0\\ \hline
					0&BB'
				\end{array}\right)
			\]
		\item \[
				\left(\begin{array}{c|c}
					A&0\\ \hline
					0&B
				\end{array}\right) \in \mathrm{GL}_{n+p}(\mathbbm{K})	 \iff \begin{cases}
					 A \in \mathrm{GL}_n(\mathbbm{K})\\
					 B \in \mathrm{GL}_p(\mathbbm{K})
				\end{cases}
			\] et dans ce cas, \[
				\left(\begin{array}{c|c}
					A&0\\ \hline
					0&B
				\end{array}\right)^{-1} =
				\left(\begin{array}{c|c}
					A^{-1}&0\\ \hline
					0&B^{-1}
				\end{array}\right)
			\]
		\item \[
				\tr \left(\begin{array}{c|c}
					A&0\\ \hline
					0&B
				\end{array}\right) = \tr A + \tr B
			\]
	\end{enumerate}
\end{prop}

\begin{prv}
	\begin{enumerate}
		\item Soit $\begin{cases}
				f \in \mathcal{L}(F) \text{ tel que } \Mat_\mathcal{B}(f) = A,
				f' \in \mathcal{L}(F) \text{ tel que } \Mat_\mathcal{B}(f') = A',
				g \in \mathcal{L}(G) \text{ tel que } \Mat_\mathcal{C}(g) = B,
				g' \in \mathcal{L}(G) \text{ tel que } \Mat_\mathcal{C}(g') = B'
			\end{cases}$ où $\begin{cases}
				F \oplus G = \mathbbm{K}^{n+p},\\
				\dim(F) = n, \dim(G) = p,\\
				\mathcal{B} \text{ base de } F,\\
				\mathcal{C} \text{ base de } G.\\
			\end{cases}$
			Soit $\begin{cases}
				h \in \mathcal{L}(\mathbbm{K}^{n+p}) \text{ tel que } \begin{cases}
					h_{|F} = f\\
					h_{|G} = g
				\end{cases}\\
				h' \in \mathcal{L}(\mathbbm{K}^{n+p}) \text{ tel que } \begin{cases}
					h'_{|F} = f'\\
					h'_{|G} = g'\\
				\end{cases}
			\end{cases}$
			Soit $\mathcal{D} = \mathcal{B} \cup \mathcal{C}$ une base de $\mathbbm{K}^{n+p}$.
			\begin{align*}
				\left(\begin{array}{c|c}
					A&0\\ \hline
					0&B
				\end{array}\right)
				\left(\begin{array}{c|c}
					A'&0\\ \hline
					0&B'
				\end{array}\right) &= \Mat_{\mathcal{D}}(h) \Mat_{\mathcal{D}}(h')\\
				&= \Mat_{\mathcal{D}}(h \circ h') \\
			\end{align*}
			Or, $(h \circ h')_{|F} = f \circ f'$ et $(h \circ h')_{|G} = g \circ g'$.

			Donc,
			\begin{align*}
				\Mat_\mathcal{D}(h \circ h') &=
					\left(\begin{array}{c|c}
						\Mat_\mathcal{B}(f \circ f')&0\\ \hline
						0&\Mat_\mathcal{C}(g \circ g')
					\end{array}\right)\\
				&=\left(\begin{array}{c|c}
					AA'&0\\ \hline
					0&BB'
				\end{array}\right).
			\end{align*}
	\end{enumerate}
\end{prv}

\begin{prop}
	Soient $A,A' \in \mathcal{M}_n(\mathbbm{K})$, $B,B' \in \mathcal{M}_{n,p}(\mathbbm{K})$, $C,C' \in \mathcal{M}_{p,n}(\mathbbm{K})$ et $D, D' \in \mathcal{M}_p(\mathbbm{K})$.

	\[
		\left(\begin{array}{c|c}
			A&B\\ \hline
			C&D
		\end{array}\right)
		\left(\begin{array}{c|c}
			A'&B'\\ \hline
			C'&D'
		\end{array}\right) = 
		\left(\begin{array}{c|c}
			AA' + BC'& AB' + BD'\\ \hline
			CA' + DC'&CB' + DD'
		\end{array}\right)
	\] Cette formule se généralise à un nombre quelconque de blocs : \[
		\left(\begin{array}{c|c|c|c}
				A_{11}&A_{12}&\cdots&A_{1,n}\\ \hline
				A_{21}&A_{22}&\cdots&A_{2,n}\\ \hline
				\vdots&\vdots&\ddots&\vdots\\ \hline
				A_{p,1}&A_{p,2}&\cdots&A_{p,n}
		\end{array}\right)
		\left(\begin{array}{c|c|c|c}
				A'_{11}&A'_{12}&\cdots&A'_{1,n}\\ \hline
				A'_{21}&A'_{22}&\cdots&A'_{2,n}\\ \hline
				\vdots&\vdots&\ddots&\vdots\\ \hline
				A'_{p,1}&A'_{p,2}&\cdots&A'_{p,n}
		\end{array}\right)
	\] Cette matrice se calcyle comme on s'y attend si les dimensions des blocs autorisent les produits.
\end{prop}

\begin{prop}
	Le rang d'une matrice $A$, c'est la taille de la plus grande matrice carrée inversible que l'on peut extraire de $A$.
	\qed
\end{prop}




		\part{Trigonométrie hyperbolique}

\begin{defn}
	Pour tout $x \in \R$, on pose \[
		\begin{cases}
			\ch x = \frac{e^x + e^{-x}}{2},\\
			\sh x = \frac{e^x - e^{-x}}{2},\\
			\th x = \frac{\sh x}{\ch x}.
		\end{cases}
	\]

	$\ch$ est appelé \underline{cosinus hyperbolique}, $\sh$ est appelé \underline{sinus hyperbolique} et $\th$ est appelé \underline{tangeante hyperbolique}.
	\index{cosinus hyperbolique}
	\index{sinus hyperbolique}
	\index{tangente hyperbolique}
\end{defn}

\begin{rmk}
	Ces formules rappèlent les formules d'Euler : pour tout $x \in \R$,
	\begin{align*}
		\cos x = \frac{e^{ix} + e^{-ix}}{2}\quad\longleftrightarrow\quad\ch x = \frac{e^x + e^{-x}}{2}\\
		\sin x = \frac{e^{ix} - e^{-ix}}{2i}\quad\longleftrightarrow\quad\sh x = \frac{e^x - e^{-x}}{2}\\
	\end{align*}
\end{rmk}

\begin{figure}[H]
	\centering
	\begin{asy}
		import graph;

		size(12cm);

		pair A = (-2, 0);
		pair B = (2, 0);

		real eps = 0.05;

		draw(shift(A) * ((0, -1.3) -- (0, 1.3)), Arrow(TeXHead));
		draw(shift(A) * ((-1.3, 0) -- (1.3, 0)), Arrow(TeXHead));

		draw(circle(A, 1), magenta);
		
		real theta = 38;
		pair M = dir(theta) + A;
		draw(A -- M, red);
		draw(arc(A, 0.35, 0, theta), red, Arrow(TeXHead));
		draw(M -- (A.x-eps, M.y), dashed);
		draw(M -- (M.x, A.y-eps), dashed);
		label("\small$\theta$", 0.5dir(theta/2) + A, red);
		label("\small$\cos\theta$", (M.x, A.y), align=S);
		label("\small$\sin\theta$", (A.x, M.y), align=1.2W);
		dot("\small$M$", M);

		label("\small$x^2 + y^2 = 1$", A + 1.5dir(45+180));

		draw(shift(B) * ((0, -1.3) -- (0, 1.3)), Arrow(TeXHead));
		draw(shift(B) * ((-1.3, 0) -- (1.3, 0)), Arrow(TeXHead));

		real ch(real x) { return (exp(x) + exp(-x)) / 2.; }
		real sh(real x) { return (exp(x) - exp(-x)) / 2.; }
		real nch(real x) { return -ch(x); }

		real k = 1.9; real r = 1.2;
		real t = 1.4;

		draw(shift(B) * scale(0.35) * graph(ch, sh, -k, k), magenta);
		draw(shift(B) * scale(0.35) * graph(nch, sh, -k, k), magenta);

		label("\small$x^2 - y^2 = 1$", B + 1.5dir(45+180) + (0, -0.2));

		M = B + 0.35(ch(t), sh(t));

		draw(M -- (B.x-eps, M.y), dashed);
		draw(M -- (M.x, B.y-eps), dashed);
		dot("\small$M$", M);
		label("\small$\ch x$", (M.x, B.y), align=S);
		label("\small$\sh x$", (B.x, M.y), align=1.2W);

		draw(shift(B) * ((-r, -r)--(r,r)), gray + dashed);
		draw(shift(B) * ((r, -r)--(-r,r)), gray + dashed);
	\end{asy}
\end{figure}


		\part{Applications}
\section{Formule de Stirling}

\begin{prop}
	On a :
	\[
		n! \simi_{n\to +\infty} \sqrt{2\pi n} \left( \frac{n}{e} \right)^n?.
	\]
\end{prop}

\begin{prv}
	\[
		\forall n \in \N^*, \ln(n!) = \sum_{k=1}^n \ln k.
	\]

	$x \mapsto \ln x$ est strictement croissante sur $[1, +\infty[$ donc \[
		\forall k \in \N^*, \forall x \in [k, k+1], \ln x \ge \ln k
	\] donc \[
		\forall k \in \N^*, \int_{k}^{k+1} \ln x~\mathrm{d}x \ge \int_{k}^{k+1} \ln k~\mathrm{d}x = \ln k
	\] et \[
		\forall k \ge 2, \forall x \in [k - 1, k], \ln x \le \ln k
	\] et docn \[
		\forall k \ge 2, \int_{k-1}^{k}  \ln x~\mathrm{d}x \le \int_{k-1}^{k} \ln k~\mathrm{d}x = \ln k
	\] Ainsi \[
		\forall n \ge 2, 
		\int_{1}^{n} \ln x~\mathrm{d}x \ge \sum_{k=2}^n \le \int_{2}^{n+1} \ln x~\mathrm{d}x
	\] Or
	\begin{align*}
		\forall n \ge 2, \int_{1}^{n} \ln x~\mathrm{d}x &= \left[ x \ln x \right]_0^n\\
		&= n \ln(n) - n + 1 \\
		&\simi_{n\to +\infty} n \ln n\\
		\int_{2}^{n+1} \ln x~\mathrm{d}x &= (n+1) \ln(n+1) - (n+1) - 2 \ln(2) + 2 \\
		&\simi_{n\to +\infty} (n+1) \ln(n+1)\\
		&\simi_{n\to +\infty}n \ln n
	\end{align*}
	car
	\begin{align*}
		\ln(n+1) &= \ln\left( n \left( 1+ \frac{1}{n} \right) \right) \\
		&= \ln n + \ln\left( 1+\frac{1}{n} \right) \\
		&= \ln n + \frac{1}{n} + \po\left( \frac{1}{n} \right) \\
		&\sim \ln n \\
	\end{align*}

	Donc \[
		\ln(n!)) \simi_{n\to +\infty} n \ln n
	\]
	Cependant, on a un problème : {\color{orange}
	\begin{align*}
		&\ln(n!) = n \ln n + \po(n \ln n)\\
		\text{donc } & n! = n^n \underbrace{e^{\po(n \ln n)}}_{?}
	\end{align*}}

	On pose \[
		\forall n \in \N^*, u_n = \ln(n!) - n\ln n
	\] $(u_n)$ a même nature que $\Sigma(u_{n+1} - u_n)$ et
	\begin{align*}
		\forall n \in \N^*,
		u_{n+1} - u_n &= \ln\left( \frac{(n+1)!}{n!} \right) - (n+1) \ln(n+1) + n \ln n \\
		&= n\big(\ln n - \ln(n+1)\big) \\
		&= n\ln\left( \frac{n}{n+1} \right) \\
		&= n \ln \left( 1 - \frac{1}{n+1} \right) \\
		&\sim -\frac{n}{n+1} \sim -1 < 0
	\end{align*}

	$\Sigma(-1)$ diverge donc $(u_n)$ diverge.

	{\color{red}
		\underline{Conjecture}
		\[
			u_n = \sum_{k=1}^{n-1}(u_{k+1} - u_k) \underbrace{\sim}_{\mathclap{\substack{~\\\downarrow\\\text{On n'a absolument pas le droit !}}}} \sum_{k=1}^{n-1} (-1) = -(n-1) \sim -n
		\]
	}

	On pose \[
		\forall n \in \N^*, v_n = u_n + n
	\] et donc 
	\begin{align*}
		\forall n \in \N^*, v_{n+1} - v_n &= n \ln\left( 1 - \frac{1}{n+1} \right) + 1 \\
		&= n\left( -\frac{1}{n+1} - \frac{1}{2(n+1)^2} + \po\left( \left( \frac{1}{n+1} \right)^2 \right) \right) + 1 \\
		&= n \left( -\frac{1}{n\left( 1+\frac{1}{n} \right)} - \frac{1}{2n^2\left( 1+\frac{1}{n^2} \right)} + \po\left( \frac{1}{n^2} \right) \right) + 1 \\
		&= -\left( \frac{1}{1+\frac{1}{n}} - \frac{1}{2n} \times \frac{1}{\left( 1+\frac{1}{n} \right)^2} + \po\left( \frac{1}{n} \right) \right) \\
		&= -\left( 1 - \frac{1}{n} + \frac{1}{2n} + \po\left( \frac{1}{n} \right) \right) + 1 \\
		&= \frac{1}{2n} + \po\left( \frac{1}{n} \right) \\
		&\sim \frac{1}{2n} > 0.
	\end{align*}

	{\color{red}
		\[
			v_n \sim \sum_{k=1}^{n-1}(v_{k-1} - v_k) \sim \sum_{k=1}^{n-1} \frac{1}{2k} \sim \frac{1}{2} \ln(n)
		\]
	}

	On pose \[
		\forall n \in \N^*, w_n = v_n - \frac{1}{2} \ln n
	\] et donc
	\begin{align*}
		\forall n \in \N^*,
		w_{n+1}- w_n &= n\ln\left( 1+\frac{1}{n+1} \right) - \frac{1}{2}\ln(n+1) + \frac{1}{2} \ln(n) + 1 \\
		&= n\left( -\frac{1}{n+1} - \frac{1}{2(n+1)^2} - \frac{1}{3(n+1)^3} + \po\left( \frac{1}{(n+1)^3} \right) \right)\\
		&\phantom{=}\,+ 1 + \frac{1}{2} \ln\left( 1 - \frac{1}{n+1} \right) \\
		&= -1 - \frac{1}{2(n+1)} - \frac{1}{3(n+1)^2} + \po\left( \frac{1}{(n+1)^2} \right) \\
		&\phantom{=}\,+ \frac{1}{n+1} + \frac{1}{2(n+1)^2} + 1\\
		&\phantom{=}\,+ \frac{1}{2} \left( -\frac{1}{n+1} - \frac{1}{2(n+1)^2} + \po\left( \frac{1}{(n+1)^2} \right) \right)
		&\sim -\frac{1}{12(n+1)^2}\\
		&\sim -\frac{1}{12n^2} < 0
	\end{align*}
	donc $\Sigma(w_{n+1} - w_n)$ converge et donc $(w_n)$ converge.

	On pose $\ell = \lim_{n\to +\infty} w_n$. Ainsi, \[
		\forall n \in \N^*, w_n = \ell + \po(1)
	\] et donc \[
		\forall n \in \N^*, \ln(n!) = n \ln n - n + \frac{1}{2} \ln(n) + \ell + \po(1)
	\] et alors
	\begin{align*}
		\forall n \in \N^*, n! &= n^n e^{-n} \sqrt{n} e^{\ell} \underbrace{e^{\po(1)}}_{\mathclap{\tendsto{n\to +\infty} 1}} \\
		&\sim \left( \frac{n}{e} \right)^n \sqrt{n} \times K
	\end{align*} avec $K = e^{\ell}$.

	On pose \[
		\forall n \in \N^*, I_n = \int_{0}^{\frac{\pi}{2}} \sin^n x~\mathrm{d}x \sim \sqrt{\frac{\pi}{2n}}
	\]et \hfill (c.f. TD5 / Exercice 8)\[
		I_{2n} = \frac{(2n)!}{\left( 2^n n! \right)^2} \times \frac{\pi}{2}.
	\]

	\begin{align*}
		I_{2n} &\sim \frac{\pi}{2} \cancel{\left( \frac{2n}{2e} \right)^{2n}} \sqrt{2n} K \cancel{\left( \frac{e}{n} \right)^{2n}} \frac{1}{n} \times \frac{1}{K^2}\\
		&\sim \frac{\pi}{K\sqrt{2n}}.
	\end{align*}
	Or \[
		I_{2n} \sim \sqrt{\frac{\pi}{4n}}.
	\] Donc \[
		\frac{\sqrt{\frac{\pi}{4n}}}{\frac{\pi}{K\sqrt{2n}}} \tendsto{n\to +\infty} 1
	\] donc \[
		\frac{K}{\sqrt{2\pi}} \tendsto{n\to +\infty} 1
	\] et donc $K = \sqrt{2\pi}$.
\end{prv}

\section{Développement décimal}

\begin{exm}
	\begin{itemize}
		\item Avec $x = 0,54\mathunderline{54}\ldots$, que vaut $2x$ ?
		\item Avec $x = 0,333\mathunderline{3}\ldots$, que vaut $3x$ ?
			\begin{itemize}
				\item $0.999\mathunderline{9}\ldots$ ?
				\item $3 \times \frac{1}{3} = 1$ ?
			\end{itemize}
	\end{itemize}
\end{exm}

\begin{prop}
	Soit $(a_n)_{n \in \N}$ telle que \[
		\begin{cases}
			a_0 \in \Z,\\
			\forall n \ge 1, a_n \in \left\llbracket 0,9 \right\rrbracket
		\end{cases}
	\]

	La série $\sum \frac{a_n}{10^n}$ converge.
\end{prop}

\begin{prv}
	\[
		\forall n \ge 1, 0 \le \frac{a_n}{10^n} \le \frac{9}{10^n}
	\] $\sum \frac{1}{10^n}$ converge car $\frac{1}{10} \in [0, 1[$.
	Donc $\sum_{n\ge 1} \frac{a_n}{10^n}$ converge donc $\sum_{n\ge 1} \frac{a_n}{10^n}$ converge.
\end{prv}

\begin{defn}
	Soit $x \in \R$. On dit que $x$ admet un \underline{développement décimal} si \[
		\exists a_0 \in \Z, (a_n)_{n\ge 1} \in \left\llbracket 0,9 \right\rrbracket^N,
		x = \sum_{n=0}^{+\infty} \frac{a_n}{10^n}.
	\]
	\index{développement décimal}
\end{defn}

\begin{thm}
	Tou réel $x \in [0, 1[$ admet un développement décimal : \[
		x = \sum_{n=1}^{+\infty} \frac{\left\lfloor 10^n x \right\rfloor - 10 \left\lfloor 10^{n-1} x \right\rfloor}{10^n}
	\]
\end{thm}

\begin{prv}
	\begin{align*}
		\forall n \ge 1,\kern 5mm &\phantom{-}10^n x - 1 < \left\lfloor 10^n x \right\rfloor \le  10^n x\\
		&-10^n x + 10 > -10 \left\lfloor 10^{n-1} x \right\rfloor \ge -10^n x
	\end{align*}
	donc \[
		-1 < \left\lfloor 10^n x \right\rfloor - 10 \left\lfloor 10^{n-1} x \right\rfloor < 10
	\] et donc \[
		\left\lfloor 10^n x \right\rfloor - 10 \left\lfloor 10^{n-1} x \right\rfloor \in \left\llbracket 0,9 \right\rrbracket.
	\]

	De plus,
	\begin{align*}
		\sum_{k=1}^n \frac{\left\lfloor 10^k x \right\rfloor - 10 \left\lfloor 10^{k-1}x \right\rfloor }{10^k} &= \sum_{k=1}^n \left( \frac{\left\lfloor 10^k x \right\rfloor}{10^k} - \frac{\left\lfloor 10^{k-1}x \right\rfloor}{10^{k-1}} \right) \\
		&= \frac{\left\lfloor 10^n x \right\rfloor}{10^n} - \underbrace{\left\lfloor x \right\rfloor}_{=0}\\
		&\tendsto{n\to +\infty} x. \\
	\end{align*}
\end{prv}

\begin{thm}
	Soit $x \in ]0, 1[$.

	\begin{enumerate}
		\item Si $x$ n'est pas décimal (i.e. on ne peut pas l'écrire comme $\sfrac{p}{10^n}$ avec $p \in \Z$ et $n \in \N$), alors $x$ a un unique développement décimal.
		\item Si $x$ est décimal, alors $x$ a exactement 2 développements décimaux :
			\begin{itemize}
				\item il y en a un où, à partir d'un certain rang, tous les chiffres sont nuls,
				\item et un autre où tous les chiffres sont égaux à 9 à parir d'un certain rang.
			\end{itemize}
	\end{enumerate}
\end{thm}

\begin{prv}
	Soit $(a_n)_{n\ge 1} \in \left\llbracket 0,9 \right\rrbracket^{\N^*}$ et $(b_n)_{n\ge 1} \in \left\llbracket 0,9 \right\rrbracket^{\N^*}$ telles que \[
		x = \sum_{n=1}^{+\infty} \frac{a_n}{10^n} = \sum_{n=1}^{+\infty} \frac{b_n}{10^n}
	\] On pose $n_0 = \min \{n \in \N^*  \mid a_n \neq b_n\}$ : \[
		\begin{cases}
			\forall n < n_0, a_n = b_n,\\
			a_{n_0} \neq b_{n_0}.
		\end{cases}
	\] Sans perte de généralité, on suppose $a_{n_0} < b_{n_0}$. On a donc
	\begin{align*}
		0 < \frac{b_{n_0} - a_{n_0}}{10^{n_0}} &= \sum_{n = n_0 + 1}^{+\infty} \frac{a_n - b_n}{10^n} \\
	\end{align*}
	\[
		\forall n \ge n_0, \begin{cases}
			0 \le a_n \le 9\\
			0 \le b_n \le 9
		\end{cases}
	\] donc \[
		\forall n \ge n_0, -9 \le a_n - b_n \le 9
	\] donc \[
		-9 \sum_{n=n_0+1}^{+\infty} \frac{1}{10^n} \le \sum_{n=n_0 + 1}^{+\infty} \frac{a_n - b_n}{10^n} \le 9 \sum_{n=1}^{+\infty} \frac{1}{10^n}.
	\]
	Or,
	\begin{align*}
		\sum_{n=n_0 + 1}^{+\infty} \frac{1}{10^n} &= \frac{1}{10^{n_0+1}} \sum_{n=0}^{+\infty} \frac{1}{10^n} \\
		&= \frac{1}{10^{n_0+1}} \times \frac{1}{1-\frac{1}{10}} \\
		&= \frac{1}{9 \times 10^{n_0}} \\
	\end{align*}
	D'où, \[
		0 < \frac{b_{n_0} - a_{n_0}}{10^{n_0}} \le  \frac{1}{10^{n_0}}
	\] donc \[
		0 < \underbrace{b_{n_0} - a_{n_0}}_{\in \Z} \le 1
	\] donc $b_{n_0} - a_{n_0} = 1$ et donc \[
	\sum_{n = n_0 + 1}^{+\infty} \frac{a_n - b_n}{10^n} = \frac{1}{10^{n_0}}
	\] donc \[
		\forall n > n_0, a_n - b_n = 9
	\] et donc \[
		\forall n > n_0, \begin{cases}
			a_n = 9\\
			b_n = 0
		\end{cases}
	\] Comme \[
		\forall n > n_0, b_n = 0
	\] $x$ est décimal et les deux développements de $x$ sont alors
	\begin{align*}
		x &= 0,a_1\ldots a_{n_0-1}a_{n_0}\mathunderline{9}\ldots\\
		&= 0,a_1\ldots a_{n_0-1}(a_{n_0}+1)\mathunderline{0}\ldots \\
	\end{align*}
\end{prv}

\begin{rmk}
	Avec $x = 0,\!54\mathunderline{54}\ldots$, $100x = 54,\!54\mathunderline{54}\ldots = 54 + x$. On a donc $x = \frac{54}{99}$.

	Avec $x = 0,\!987\,123\,\mathunderline{123}\ldots$, on a
	\begin{align*}
		x &= \frac{987}{1000} + 0,\!000\,\mathunderline{123}\ldots\\
		&= \frac{987}{1000} + \frac{1}{10^3}\underbrace{(0,\!\mathunderline{123}\ldots)}_y \\
	\end{align*}
	On a $1000 y = 123 + y$ et donc $y = \frac{123}{999}$ et donc $x = \frac{987 + \frac{123}{999}}{1000}$.
\end{rmk}





	}

	{
		\chap[26]{Déterminant}
		\renewcommand{\cwd}{../chap26}
		\begin{defn}
	Un \underline{proposition} est un énoncé qui est soit vrai, soit faux.
\end{defn}

\begin{exm}
	\begin{align*}
		\begin{rcases*}
			A: ``B \text{ est vraie }"\\
			B: ``A \text{ est fausse }"\\
		\end{rcases*} \text{ Le système $\{A,B\}$ est une \underline{auto-contradiction}}
	\end{align*}
\end{exm}

\begin{defn}
	\underline{Démontrer} une proposition revient à prouver qu'elle est vraie
\end{defn}

		\begin{defn}
	Soit $E$ un $\mathbbm{K}$-espace vectoriel. On dit que $E$ est de \underline{dimension finie} si $E$ a au moins une famille génératrice finie. On dit que $E$ est de \underline{dimension infinie} sinon.
	\index{dimension finie (espace vectoriel)}
	\index{dimension infinie (espace vectoriel)}
\end{defn}

\begin{thm}
	[Théorème de la base extraite]
	Soit $E$ un $\mathbbm{K}$-espace vectoriel non nul de dimension finie. Soit $\mathcal{G}$ une famille génératrice finie de $E$. Alors, il existe une base $\mathcal{B}$ de $\mathcal{E}$ telle que $\mathcal{B} \subset \mathcal{G}$.
\end{thm}

\begin{prv}
	[par récurrence sur $\#G = \Card(G)$]
	\begin{itemize}
		\item Soit $E$ un $\mathbbm{K}$-espace vectoriel non nul engendré par $\mathcal{G} = (u)$.\\
			Si $u = 0_E$, alors $E = \{0_E\}$: une contradiction $\lightning$ \\
			Donc $u \neq 0_E$ donc $(u)$ est libre. En effet, \[
				\forall \lambda \in \mathbbm{K}, \lambda u = 0_E \implies \lambda = 0_\mathbbm{K}
			\] Donc $\mathcal{G}$ est une base de $E$.\\
		\item Soit $n \in \N_*$. Soit $E$ un $\mathbbm{K}$-espace vectoriel. On suppose que si $E$ a une famille génératrice constituée de $n$ vecteurs, alors on peut extraire de cette famille une base de $E$.\\
			Soit $\mathcal{G}$ une famille génératrice de $E$ avec $n+1$ vecteurs.\\
			Si $\mathcal{G}$ est libre, alors $\mathcal{G}$ est une base de $E$. \\
			Si $\mathcal{G}$ n'est pas libre, alors il existe $u \in \mathcal{G}$ tel que $u \in \Vect(\mathcal{G}\setminus \{u\})$ \\
			Donc $\mathcal{G}\setminus \{u\}$ engendre $E$. Or, $\mathcal{G}\setminus \{u\}$ possède $n$ vecteurs. D'après l'hypothèse de récurrence, il existe une base $\mathcal{B}$ de $E$ telle que \[
				\mathcal{B} \subset \mathcal{G} \setminus \{u\} \subset \mathcal{G}
			\] 
	\end{itemize}
\end{prv}

\begin{crlr}
	Tout espace de dimension finie a une base.
	\qed
\end{crlr}

\begin{thm}
	[Théorème de la base incomplète]
	Soit $E$ un $\mathbbm{K}$-espace vectoriel de dimension finie, $\mathcal{G}$ une famille génératrice finie de $E$. $\mathcal{L}$ une famille libre de $E$. Alors, il existe une base $\mathcal{B}$ de $E$ telle que \[
		\mathcal{L} \subset \mathcal{B} \text{ et } \mathcal{B}\setminus \mathcal{L} \subset \mathcal{G}
	\] 
\end{thm}

\begin{prv}
	[par récurrence sur $\#(\mathcal{G}\setminus\mathcal{L})$]
	\begin{itemize}
		\item Avec les notations précédentes, on suppose que $\mathcal{G}\setminus\mathcal{L} \neq \O$ \[
				\forall u \in \mathcal{G}, u \in \mathcal{L}
			\] Donc $\mathcal{G} \subset \mathcal{L}$ donc $\mathcal{L}$ est génératrice donc $\mathcal{L}$ est une base de $E$. On pose $\mathcal{B} = \mathcal{L}$ et alors \[
				\mathcal{L} \subset  \mathcal{B} \text{ et } \mathcal{B}\setminus\mathcal{L} = \O \subset  \mathcal{G}
			\] 
		\item Soit $n \in \N$. On suppose que si $\mathcal{G}$ est génératrice et $\mathcal{L}$ libre avec $\#(\mathcal{G}\setminus\mathcal{L}) = n$ alors il existe une base $\mathcal{B}$ de $E$ telle que \[
			\mathcal{L}\subset \mathcal{B} \text{ et } \mathcal{B}\setminus\mathcal{L}\subset \mathcal{G}
		\] Soient à présent $\mathcal{G}$ une famille génératrice de $E$ et $\mathcal{L}$ une famille libre de $E$ telles que $\#(\mathcal{G}\setminus\mathcal{L}) = n+1 > 0$\\
		Si $\mathcal{L}$ engendre $E$, alors $\mathcal{L}$ est une base de $E$. On pose $\mathcal{B} = \mathcal{L}$ et on a bien \[
			\mathcal{L} \subset  \mathcal{B} \text{ et } \mathcal{B} \setminus \mathcal{L} = \O \subset  \mathcal{G}
		\] On suppose que $\mathcal{L}$ n'engendre pas $E$. Il existe $u \in \mathcal{G}$ tel que $u \not\in \Vec(\mathcal{L})$ (car sinon, $\mathcal{G} \subset \Vect(\mathcal{L})$ et donc $\underbrace{\Vect(\mathcal{G})}_{= E} \subset  \underbrace{\Vect(\mathcal{L})}_{ \subset E}$\\
		Donc $\mathcal{L} \cup \{u\} $ est libre. On pose $\mathcal{L}' = \mathcal{L} \cup \{u\} $ \[
			\mathcal{G}\setminus \mathcal{L}' = \mathcal{G}\setminus (\mathcal{L} \cup \{u\}) = (\mathcal{G}\setminus\mathcal{L})\setminus \{u\} 
		\] donc $\#(\mathcal{G}\setminus\mathcal{L}') = n+1 -1 = n$\\
		D'après l'hypothèse de récurrence, il existe $\mathcal{B}$ une base de $E$ telle que \[
			\mathcal{L} \subset  \mathcal{L}' \subset \mathcal{B} \text{ et } \mathcal{B}\setminus \mathcal{L}' \subset \mathcal{G}
		\] \[
			\mathcal{B} \setminus \mathcal{L} = \underbrace{\mathcal{B}\setminus\mathcal{L}'}_{\subset \mathcal{G}} \cup \underbrace{\{u\}}_{\subset \mathcal{G} \text{ car } u \in \mathcal{G}}
		\] On a $\mathcal{B}\setminus\mathcal{L}\subset \mathcal{G}$
	\end{itemize}
\end{prv}

\begin{thm}
	Soit $E$ un $\mathbbm{K}$-espace vectoriel de dimension finie. Toutes les bases de $E$ ont le même cardinal.
\end{thm}

\begin{prv}
	Soit $\mathcal{G}$ une famille génératrice finie de $E$ et $\mathcal{B} \subset  \mathcal{G}$ une base de $E$. On note $n = \#\mathcal{B}$ \\
	Soit $\mathcal{B}'$ une base de $E$. On pose $p = n - \#(\mathcal{B} \cap  \mathcal{B}')$. Montrons par récurrence sur  $p$ que $\#\mathcal{B} = \#\mathcal{B}'$ 
	\begin{itemize}
		\item On suppose que $p = 0$. Alors, $\#(\mathcal{B} \cap \mathcal{B}') = n$ \\
			Or, $\mathcal{B}' \cap \mathcal{B} \subset \mathcal{B}$ donc $\mathcal{B} \cap \mathcal{B}' = \mathcal{B}$ donc $\mathcal{B} \subset  \mathcal{B}'$ et donc $\mathcal{B} = \mathcal{B}'$ 
		\item Soit $p \in \N$. On suppose que si $\mathcal{B}'$ est une base de $E$ telle que $n - \#(\mathcal{B} \cap \mathcal{B}') = p$, alors $\#\mathcal{B}' = n$ \\
			Aoit $\mathcal{B}'$ une base de $E$ telle que $n - \#(\mathcal{B}\cap \mathcal{B}') = p+1 > 0$ \\
			Donc $\mathcal{B} \cap \mathcal{B}' \neq \mathcal{B}$. Soit $u \in \mathcal{B}' \setminus \mathcal{B}$. D'après le lemme d'échange, il existe $v \in \mathcal{B}\setminus \mathcal{B}'$ tel que $\mathcal{B}' \setminus \{u\} \cup \{v\}$ est une base de $E$. On pose $\mathcal{B}'' = \mathcal{B}' \setminus \{u\} \cup \{v\}$ 
			\begin{align*}
				\mathcal{B}'' \cap \mathcal{B} &= \left( (\mathcal{B}' \setminus \{u\})  \cap \mathcal{B} \right) \cup \{v\} \\
				&= (\mathcal{B}' \cap \mathcal{B}) \cup \{v\} \\
			\end{align*}
			donc,
			\begin{align*}
				n - \#(\mathcal{B}'' \cap \mathcal{B}) &= n - (\#(\mathcal{B}' \cap \mathcal{B}) + 1) \\
				&= p+1- 1 \\
				&= p \\
			\end{align*}
			D'après l'hypothèse de récurrence, \[
				\#\mathcal{B}'' = n
			\] Or, $\#\mathcal{B}'' = \#\mathcal{B}'$
	\end{itemize}
\end{prv}

\begin{lem}
	Soient $\mathcal{B}$ et $\mathcal{B}'$ deux bases de $E$ telles que $\mathcal{B}\subset \mathcal{B}'$. Alors, $\mathcal{B} = \mathcal{B}'$.
\end{lem}

\begin{prv}
	On suppose $\mathcal{B}' \neq \mathcal{B}$. Soit $u \in \mathcal{B}' \setminus \mathcal{B}$
	$u \in E = \Vect(\mathcal{B})$ donc $\mathcal{B} \cup \{u\}$ n'est pas libre.
	Donc $\mathcal{B}\cup \{u\} \subset \mathcal{B}'$ et $\mathcal{B}'$ est libre donc $\mathcal{B}\cup \{u\}$ est libre: une contradiction $\lightning$
\end{prv}

\begin{lem}
	[Lemme d'échange] Soient $\mathcal{B}_1$ et $\mathcal{B}_2$ deux bases de $E$ et $u \in \mathcal{B}_1 \setminus \mathcal{B}_2$. Alors, il existe $v \in \mathcal{B}_2$ tel que $(\mathcal{B}_1 \setminus \{u\}) \cup \{v\}$ soit une base de $E$.
\end{lem}

\begin{prv}
	[1${}^\text{nde}$ méthode]
	On suppose que pout tout $v \in \mathcal{B}_2$, $(\mathcal{B}_1\setminus \{u\}) \cup \{v\}$ n'est pas une base de $E$
	Soit $v \in \mathcal{B}_2$.
	\begin{itemize}
		\item Supposons $(\mathcal{B}_1\setminus \{u\})\cup \{v\}$ non libre. $\mathcal{B}_1 \setminus \{u\}$ est libre. Donc $v \in \Vect(\mathcal{B}_1 \setminus \{u\})$
		\item Supposons $(\mathcal{B}_1\setminus \{u\}) \cup \{v\}$ non génératrice.
			Comme $\mathcal{B}_1$ engendre $E$, $u \not\in \Vect(\mathcal{B}_1\setminus \{v\})$.
			On suppose que $\mathcal{B}_1 \neq \mathcal{B}_2$.
			$\forall v \in \mathcal{B}_2 \setminus \mathcal{B}_1, \Vect(\mathcal{B}_1 \setminus \{v\}) = \Vect(\mathcal{B}_1) = E \ni u$ 
			donc, $(\mathcal{B}_1\setminus \{u\}) \cup \{v\}$ engendre $E$ et donc \[
				v \in \Vect(\mathcal{B}_1 \setminus \{u\})
			\] On a aussi \[
				\forall v \in \mathcal{B}_1 \setminus \{u\}, v \in \Vect(\mathcal{B}_1\setminus \{u\})
			\] Comme $u \not\in \mathcal{B}_2$, on a \[
				\forall v \in \mathcal{B}_2, v \in \Vect(\mathcal{B}_1\setminus \{u\})
			\] docn \[
				E = \Vect(\mathcal{B}_2) \subset \Vect(\mathcal{B}_1\setminus \{u\})
			\] donc $\mathcal{B}_1\setminus \{u\}$ engendre $E$ donc $\mathcal{B}_1\setminus \{u\}$ est une base de $E$. Or, $\mathcal{B}_1 \setminus \{u\}  \subset  \mathcal{B}_1$, donc $\mathcal{B}_1\setminus \{u\} = \mathcal{B}_1$
	\end{itemize}
\end{prv}

\begin{prv}
	[2${}^\text{nde}$ méthode]
	On suppose que pout tout $v \in \mathcal{B}_2$, $(\mathcal{B}_1\setminus \{u\}) \cup \{v\}$ n'est pas une base de $E$
	\begin{itemize}
		\item Comme $u \in \mathcal{B}_1 \setminus \mathcal{B}_2$, nécéssairement $\mathcal{B}_1 \neq \mathcal{B}_2$ donc $\mathcal{B}_2 \not\subset \mathcal{B}_1$, donc $\mathcal{B}_2\setminus\mathcal{B}_1 \neq \O$ 
		\item Soit $v \in \mathcal{B}_2\setminus\mathcal{B}_1$. Il existe $(\lambda_w)_{w\in\mathcal{B}_1}$ une famille de scalaires presque nulle telle que \[
				v = \sum_{w \in \mathcal{B}_1} \lambda_w w - \lambda_u u + + \sum_{w \in \mathcal{B}_1\setminus \{u\}}\lambda_w w
			\]
			Si $\lambda_u \neq 0_E$, alors
			\begin{align*}
				u &= \lambda_u^{-1}\left( v - \sum_{w \in \mathcal{B}_1 \setminus \{u\}} \lambda_w w \right)\\
					&\in \Vect(\mathcal{B}_1\setminus \{u\} \cup v)
			\end{align*}
			 donc $\mathcal{B}_1 \subset \Vect(\mathcal{B}_1\setminus \{u\} \cup \{v\})$\\
			 et donc $E \subset  \Vect(\mathcal{B}_1 \setminus \{u\} \cup \{v\})$ \\
			 et donc $\mathcal{B}_1 \setminus \{u\} \cup \{v\}$ engendre $E$ \\
			 donc $\mathcal{B}_1 \setminus \{u\} \cup \{v\}$ n'est pas libre\\
			 donc $v \in \Vect(\mathcal{B}_1\setminus \{u\})$ (car $\mathcal{B}_1 \setminus \{u\}$ est libre\\
			 donc $\lambda_u = 0_\mathbbm{K}$ $\lightning$\\`

			 Donc, $\lambda_u = 0_\mathbbm{K}$, docn $v \in \Vect(\mathcal{B}_1\setminus \{u\})$ \\
			 On vient de prouver que
			 \begin{align*}
			 	\mathcal{B}_2 \setminus \mathcal{B}_1 \subset \Vect(\mathcal{B}_1 \setminus \{u\})\\
			 	\mathcal{B}_1 \setminus \{u\} \subset \Vect(\mathcal{B}_1 \setminus \{u\})\\
			 \end{align*}
			 Comme $u \not\in \mathcal{B}_2$, \[
			 	\mathcal{B}_2 \subset \Vect(\mathcal{B}_1 \setminus \{u\})
			 \] donc \[
			 	E = \Vect(\mathcal{B}_2) \subset  \Vect(\mathcal{B}_1 \setminus \{u\})
			 \] donc $\mathcal{B}_1 \setminus \{u\}$ engendre $E$. Donc,  $\mathcal{B}_1 \setminus \{u\}$ est une base de $E$.\\
			 Or, $\mathcal{B}_1 \setminus \{u\} \subset  \mathcal{B}_1$, donc $\mathcal{B}_1 \setminus \{u\} = \mathcal{B}_1$
	\end{itemize}
\end{prv}

\begin{defn}
	Soit $E$ un $\mathbbm{K}$-espace vectoriel de dimension finie. Le cardinal commun à toutes les bases de $E$ est appelé \underline{dimension} de $E$ est notée $\dim(E)$ ou $\dim_\mathbbm{K}(E)$\\
	C'est donc aussi le nombre de coordonnées de n'importe quel vecteur dans n'importe quelle base.
	\index{dimension (espace vectoriel)}
\end{defn}

\begin{exm}
	\begin{enumerate}
		\item $\dim_\R(\C) = 2$ et $\dim_\C(\C) = 1$ 
		\item $\dim_\mathbbm{K}(\mathbbm{K}^{n}) = n$ 
		\item $\dim_{\mathbbm{K}}(\mathcal{M}_{n,p}(\mathbbm{K})) = np$
	\end{enumerate}
\end{exm}

\begin{crlr}
	Soit $E$ un $\mathbbm{K}$-espace vectoriel de dimension finie, $\mathcal{L}$ une famille libre de $E$, $\mathcal{G}$ une famille génératrice de $E$. On note $n = \dim(E)$
	\begin{enumerate}
		\item $\#\mathcal{G} \ge n$ et $(\#\mathcal{G} = n \implies \mathcal{G} \text{ est une base de } E$)
		\item $\#\mathcal{L} \le n$ et $(\#\mathcal{L} = n \implies \mathcal{L} \text{ est une base de } E$)
	\end{enumerate}
\end{crlr}

\begin{crlr}
	$\R^{\R}$ est de dimension infinie.
	$\forall i \in \N, e_i: x \mapsto x^i$\\
	$(e_i)_{i\in\N}$ est libre dans $\R^\R$
\end{crlr}

\begin{prop}
	Soient $E$ et $F$ deux $\mathbbm{K}$-espaces vectoriels de dimension finie. Alors $E\times F$ est de dimension finie et $\dim(E\times F) = \dim(E) + \dim(F)$
\end{prop}

\begin{prv}
	Soit $(e_1,\ldots, e_n)$ une base de $E$, $(f_1, \ldots, f_p)$ une base de $F$.
	On pose \[
		\left\{\begin{array}
			{r c l}
			u_1 &=& (e_1,0_F)\\
			u_2 &=& (e_2,0_F)\\
					&\vdots&\\
			u_n &=& (e_n,0_F)\\
			u_{n+1} &=& (0_E, f_1)\\
			u_{n+2} &=& (0_E, f_2)\\
					&\vdots&\\
			u_{n+p} &=& (0_E,f_p)\\
		\end{array}\right.
	\]
	Soit $(x,y) \in E\times F$. \[
		\begin{cases}
			\exists (x_1,\ldots,x_n)\in \mathbbm{K}^n, x = \sum_{i=1}^{n} x_ie_i
			\exists (y_1,\ldots,y_n)\in \mathbbm{K}^n, x = \sum_{j=1}^{p} y_jf_j
		\end{cases}
	\] 
	\begin{align*}
		(x,y) &= \left( \sum_{i=1}^{n} x_ie_i, \sum_{i=1}^{p} y_jf_j \right)  \\
		&= \sum_{i=1}^{n} x_i (e_i + 0_F) + \sum_{j=1}^{p} y_j (0_E, f_j) \\
		&= \sum_{i=1}^{n} x_i u_i + \sum_{j=1}^{p} y_j u_{n+j} \\
	\end{align*}
	Donc, $E\times F = \Vect(u_1, \ldots, u_{n+p})$ donc $E\times F$ est de dimension finie.\\
	Soit $(\lambda_1, \ldots, \lambda_{n+p}) \in \mathbbm{K}^{n+p}$ tel que \[
		(*): \quad \sum_{k=1}^{n+p} \lambda_ku_k = 0_{E\times F} = (0_E, 0_F)
	\]
	\begin{align*}
		(*) &\iff \sum_{k=1}^{n} \lambda_k (e_k, 0_F) + \sum_{k=n+1}^{p} \lambda_k(0_E, f_{k-n}) = (0_E, 0_F)\\
				&\iff \begin{cases}
					\sum_{k=1}^{n} \lambda_k e_k = 0_E\\
					\sum_{k=n+1}^{p} \lambda_k f_{k-n} = 0_F
				\end{cases}\\
				&\iff \begin{cases}
					\forall k \in \left\llbracket 1,n \right\rrbracket, \lambda_k = 0_\mathbbm{K} \qquad&(\text{car $(e_1,\ldots,e_n)$ est libre})\\
					\forall k \in \left\llbracket n+1,n+p \right\rrbracket, \lambda_k = 0_\mathbbm{K} \qquad&(\text{car $(f_1,\ldots,f_n)$ est libre})\\
				\end{cases}
	\end{align*}
	Donc $(u_1, \ldots, u_{n+p})$ est une base de $E\times F$. Donc, $\dim(E\times F) = n + p = \dim(E) + \dim(F)$
\end{prv}

\begin{rmk}
	[Convention]
	\[\dim\big(\{0_E\}\big) = 0\]
\end{rmk}

\begin{thm}
	Soit $E$ un $\mathbbm{K}$-espace vectoriel de dimension finie, $F$ un sous-espace vectoriel de $E$. Alors, $F$ est de dimension finie et  $\dim(F) \le \dim(E)$\\
	Si $\dim(F) = \dim(E)$, alors $F = E$
\end{thm}

\begin{prv}
	On considère \[
		A = \{k \in \N \mid \text{il existe une famille libre de $F$ à $k$ éléments}\} 
	\]
	On suppose $F \neq \{0_E\}$.
	\begin{itemize}
		\item Soit $u \in F\setminus \{0_E\}$. $(u)$ est libre donc $1 \in A$ et donc $A \neq \O$
		\item Soit $\mathcal{L}$ une famille libre de $F$. Alors, $\mathcal{L}$ est une famille libre de $E$ \\
			donc $\#\mathcal{L} \le \dim(E)$\\
			Donc $A$ est majorée par $\dim(E)$ \\
			On en déduit que $A$ a un plus grand élément $p$.
		\item Soit $\mathcal{L}$ une famille libre de $F$ avec $p$ éléments.\\
			Si $\mathcal{L}$ n'engendre pas $F$, alors il existe $u\in F$ tel que $u\not\in \Vect(\mathcal{L})$ et donc $\mathcal{L} \cup \{u\}$ est une famille libre de $F$, donc $p+1 \in A$ en contradiction avec la maximalité de $p$.\\
			Donc $\mathcal{L}$ est une base de $F$ donc $F$ est de dimension finie et $\dim(F) = p \le \dim(E)$\\
	\end{itemize}

	Soit $\mathcal{B}$ une base de $F$. Alors, $\mathcal{B}$ est aussi une famille de libre de de $E$. Donc $\#\mathcal{B} \le \dim(E)$ donc $\dim(F) = \dim(E)$ \\
	Si $\dim(F) = \dim(E)$, alors $\mathcal{B}$ est une base de $E$, et donc $F = \Vect(\mathcal{B}) = E$
\end{prv}

\begin{prop}
	[Formule de Grassmann]
	Soit $E$ un $\mathbbm{K}$-espace vectoriel de dimension finie, $F$ et $G$ deux sous-espace vectoriels de $E$. Alors, \[
		\dim(F+G) = \dim(F) + \dim(G) - \dim(F\cap G)
	\] 
\end{prop}

\begin{prv}
	Soit $(e_1, \ldots, e_p)$ une base de $F\cap G$. $(e_1,\ldots,e_p)$ est une famille libre de $F$.\\
	On complète $(e_1, \ldots, e_p)$ en une base $(e_1, \ldots, e_p, u_1, \ldots, u_q)$ de $F$.\\
	De même, on complète $(e_1, \ldots, e_p)$ en une base $(e_1, \ldots, e_p, v_1, \ldots, v_r)$ de $G$.\\
	On pose  $\mathcal{B} = (e_1, \ldots, e_p, u_1, \ldots, u_q, v_1, \ldots, v_r)$. Montrons que $\mathcal{B}$ est une base de $F+G$
	\begin{itemize}
		\item Soit $u \in F+G$ \\
			On pose $u = v+w$ avec $\begin{cases}
				v\in F\\
				w \in G
			\end{cases}$.\\
			On pose $v = \sum_{i=1}^p \lambda_i e_i + \sum_{i=1}^q \mu_i u_i$ avec $(\lambda_1, \ldots, \lambda_p, \mu_1, \ldots, \lambda_q) \in \mathbbm{K}^{p+q}$\\
			On pose aussi $w = \sum_{i = 1}^p \lambda'_ie_i + \sum_{j=1}^r \nu_j v_j$ avec $(\lambda_1',\ldots,\lambda_p', \nu_1, \ldots, \nu_r) \in \mathbbm{K}^{p+r}$\\
			D'où, \[
				u = \sum_{i=1}^p (\lambda_i + \lambda'_i)e_i + \sum_{j=1}^q \mu_j u_j + \sum_{k=1}^r \nu_k v_k \in \Vect(\mathcal{B})
			\]
		\item Soient $(\lambda_1, \ldots, \lambda_p, \mu_1, \ldots, \mu_q, \nu_1, \ldots, \nu_r) \in \mathbbm{K}^{p+q+r}$.\\
			On suppose \[
				(*)\quad \sum_{i=1}^{p}\lambda_ie_i + \sum_{j=1}^q\mu_ju_j + \sum_{k=1}^r \nu_k v_k = 0_E
			\] 
			D'où, \[
				\underbrace{\sum_{i=1}^p\lambda_i e_i + \sum_{j=1}^q \mu_ju_j}_{\in F} = \underbrace{-\sum_{k=1}^r\nu_jv_k}_{\in G}
			\] 
			Donc, \[
				f = \sum_{i=1}^p \lambda_i e_i + \sum_{j=1}^q \mu_j u_j \in F\cap G
			\] Comme $(e_1, \ldots, e_p)$ est une base de $F\cap G$, $\exists ! (\lambda_1', \ldots, \lambda_p') \in \mathbbm{K}^p$ tel que \[
				f = \sum_{i=1}^p \lambda'_i e_i = \sum_{i=1}^p \lambda'_i e_i + \sum_{j=1}^q 0_\mathbbm{K}u_j
			\] Comme $(e_1, \ldots, e_p, u_1, \ldots, u_q)$ est une base de $F$, \[
				\forall k \in \left\llbracket 1, q \right\rrbracket, \mu_j = 0_\mathbbm{K}
			\] De même, \[
				\forall k \in \left\llbracket 1,r \right\rrbracket , \nu_k = 0_\mathbbm{K}
			\] On remplace dans $(*)$ pour trouver \[
				\sum_{i=1}^p \lambda_ie_i = 0_E
			\] Comme $(e_1, \ldots, e_p)$ est libre, \[
				\forall i \in \left\llbracket 1,p \right\rrbracket, \lambda_i = 0_\mathbbm{K}
			\] Donc $\mathcal{B}$ est libre.\\
			Donc, 
			\begin{align*}
				\dim(F+G) &=  p +q + r \\
				&= (p+q)+ (p+r) - p \\
				&= \dim(F) + \dim(G) - \dim(F\cap G) \\
			\end{align*}
	\end{itemize}
\end{prv}

\begin{crlr}
	Avec les hypothèse précédentes, \[
		E = F \oplus G \iff \begin{cases}
			F \cap  G = \{0_E\} \\
			\dim(E) = \dim(F) + \dim(G)
		\end{cases}
	\] 
\end{crlr}

\begin{prv}
	\begin{itemize}
		\item[``$\implies$''] On suppose $E = F \oplus G$ \\
			Comme la somme est directe, $F \cap G = \{0_E\}$ 
			\begin{align*}
				\dim(E) &= \dim(F)\\
				&= \dim(F) + \dim(G) - \dim(F\cap G)\\
				&= \dim(F) + \dim(G)\\
			\end{align*}
		\item[``$\impliedby$''] On suppose $F\cap G = \{0_E\}$ et $\dim(E) = \dim(F) + \dim(G)$.\\
			On sait déjà que $F+G = F \oplus G$\\
			 \begin{align*}
				\dim(F+G) = \dim(F) + \dim(G) - \dim(F \cap G) = \dim(E)
			\end{align*}
			Donc $F + G = E$
	\end{itemize}
\end{prv}

\begin{prop}
	Soit $F$ un $\mathbbm{K}$-espace vectoriel de dimension finie $n$. Soit $\mathcal{B} = (e_1, \ldots, e_n)$ une base de $F$. L'application
	\begin{align*}
		f: \mathbbm{K}^n &\longrightarrow F \\
		(\lambda_1, \ldots, \lambda_n) &\longmapsto \sum_{i=1}^n \lambda_i e_i
	\end{align*} est bijective.\\
	Si $\mathbbm{K}$ est infini, $\mathbbm{K}^n$ aussi et donc $F$ aussi.\\
	Si $\#\mathbbm{K} = p \in \N_*$,
	\begin{align*}
		\#&\mathbbm{K}^n = p^n\\
		&\vrt=\\
		\#&F
	\end{align*}
\end{prop}


		\part{Dérivation}

\underline{Motivation}:

{
\begin{wrapfigure}{l}{3cm}
	\centering
	\begin{asy}
		import three;

		size(3cm);
		settings.render=0;
		settings.prc=false;
		currentprojection = obliqueZ;

		draw(unitbox);
		draw(shift(1.1Z + 0.05X) * (O -- X), Arrows3(TeXHead2));
		draw(shift(1.1Z + 0.05Y) * (O -- Y), Arrows3(TeXHead2));
		draw(shift(1.1X + 0.05Z) * (O -- Z), Arrows3(TeXHead2));

		label("$x$", (X/2) + (1.1Z + 0.05X), align=S);
		label("$y$", (Y/2) + (1.1Z + 0.05Y), align=W);
		label("$z$", (Z/2) + X, align=SE);
	\end{asy}
\end{wrapfigure}

\begin{align*}
	&S(x,y,z) = 2(xy + xz + yz)\\
	&V(x,y,z) = xyz
\end{align*}

On cherche à minimiser $S$ avec la contrainte $V = 1$.

Soit $f : \begin{array}{rcl}
	\left( \R_*^+ \right)^2 &\longrightarrow& \R \\
	(x,y) &\longmapsto& S\left( x,y,\frac{1}{xy} \right) = 2\left( xy + \frac{1}{y} + \frac{1}{x} \right).
\end{array}$

On cherche $(a,b) \in \left( \R^+_* \right)^2$ tel que \[
	\forall (x,y) \in (\R^+_*), f(x,y) \ge f(a,b).
\]
}

\begin{defn}
	Soit $f: U \to \R$ où $U$ est un ouvert de $\R^2$. Soit $(a,b) \in U$.
	\vspace{2mm}

	Si $\lim_{x \to a} \frac{f(x,b) - f(a,b)}{x - a} \in \R$, alors on dit que $f$ a une dérivée partielle suivant $x$ en $(a,b)$ et cette limite est notée \[
		\partial f_1(a,b) = \frac{\partial f}{\partial x}(a,b).
	\]

	Si $\lim_{y \to b} \frac{f(a,y) - f(a,b)}{y - b} \in \R$, alors on dit que $f$ a une dérivée partielle suivant $y$ et la limite est notée \[
		\partial f_2(a,b) = \frac{\partial f}{\partial y}(a,b).
	\]
\end{defn}

\begin{exm}
	\begin{enumerate}
		\item $f: (x,y) \mapsto xy + x - y$.

			\begin{align*}
				&\frac{\partial f}{\partial x} : (x,y) \mapsto y + 1,\\
				&\frac{\partial f}{\partial y} : (x,y) \mapsto x - 1.
			\end{align*}

		\item $f: (x,y) \mapsto xy + \frac{1}{y}+ \frac{1}{x}$.

			\begin{align*}
				&\frac{\partial f}{\partial x}: (x,y) \mapsto y - \frac{1}{x^2},\\
				&\frac{\partial f}{\partial y}: (x,y) \mapsto x - \frac{1}{y^2}.
			\end{align*}

		\item Trouver $f$ telle que $\begin{cases}
				(1): \qquad \frac{\partial f}{\partial x}=y,\\[2mm]
				(2): \qquad \frac{\partial f}{\partial y} = x.
			\end{cases}$

			D'après $(1)$ : \[
				\forall (x,y), \exists C(y) \in \R, f(x,y) = xy + C(y)
			\] et donc \[
				\frac{\partial f}{\partial y}(x,y) = x + C'(y)
			\] donc $C'(y) = 0$ et donc $C$ est constante.

		\item Trouver $f$ telle que $\begin{cases}
			\frac{\partial f}{\partial x} = -y,\\[2mm]
			\frac{\partial f}{ƒ\partial y} = x.
		\end{cases}$

		Ce n'est pas possible !
	\end{enumerate}
\end{exm}

\begin{defn}~\\
	\begin{minipage}{\linewidth}
		\begin{wrapfigure}{r}{4cm}
			\centering
			\vspace{-5mm}
			\begin{asy}
				import three;
				import graph3;
				size(4cm);

				settings.render = 0;
				settings.prc = false;
				currentprojection = obliqueX;

				draw(O -- X, Arrow3(TeXHead2));
				draw(O -- Y, Arrow3(TeXHead2));
				draw(O -- Z, Arrow3(TeXHead2));

				triple f(real x, real y, real z = 0) { return (x,y,cos(x - 0.5) * cos(y - 0.5)/1.2 + 0.15); }

				real inc = 1 / 5;

				for(real x = 0; x <= 1; x += inc) {
					draw(graph(
						new real(real t) { return x; }, // x
						new real(real y) { return y; }, // y
						new real(real y) { return f(x,y).z; }, // z
						0, 1
					), gray);
				}

				for(real y = 0; y <= 1; y += inc) {
					draw(graph(
						new real(real x) { return x; }, // x
						new real(real t) { return y; }, // y
						new real(real x) { return f(x,y).z; }, // z
						0, 1
					), gray);
				}

				path3 path1 = (0.8, 0.2, 0) .. (0.5, 0.5, 0) .. (0.3, 0.7, 0);
				path3 path2 = f(0.8, 0.2, 0) .. f(0.5, 0.5, 0) .. f(0.3, 0.7, 0);
				path3 d = (0.2, 0.3, 0) .. (0.3, 0.4, 0) .. (0.2, 0.7, 0) .. (0.8, 0.9, 0) .. (0.6, 0.2, 0) .. cycle;

				draw(path1, red, Arrow3(TeXHead2));
				draw(path2, red, Arrow3(TeXHead2, position=0.8));

				dot((0.5, 0.5, 0));
				dot(f(0.5, 0.5, 0));
				draw((0.5, 0.5, 0) -- f(0.5, 0.5, 0), dashed);
				draw(d);

				label("$w$", (0.3, 0.7, 0), red, align=SE);
				label("$U$", (0.8, 0.9, 0), align=SE);
			\end{asy}
		\end{wrapfigure}

		Soit $f: U \to \R$ où $U$ est un ouvert. Soit $(a,b) \in U$. Soit $w = (w_1, w_2) \in \R^2$.

		Si 
		\[
			\lim_{t\to 0} \frac{f(a + tw_1, b + tw_2) - f(a,b)}{t}
		\] existe et est réelle, alors on dit que $f$ a une dérivée dans la direction de $w$ et la limite est notée \[
			\mathrm{d}f(w)\,(a,b) = D_w(f)\,(a,b).
		\]
	\end{minipage}
\end{defn}

\begin{exm}
	\begin{align*}
		f: \left( \R_*^+ \right)^2 &\longrightarrow \R \\
		(x,y) &\longmapsto xy+\frac{1}{x}+\frac{1}{y}.
	\end{align*}

	On pose $(a,b) = (1,2)$, $w = (w_1, w_2) = (1,1)$.
	\begin{align*}
		\frac{f(1+t, 2+t) - f(1,2)}{t} &= \frac{1}{t} \left( (1+t)(2+t) + \frac{1}{1+t} + \frac{1}{2+t} - 3 - \frac{1}{2} \right) \\
		&= \frac{1}{t}\left(\cancel 2 + 3t + \po(t) + \cancel 1 - t + \po(t) + \frac{1}{2}\left( \cancel 1 - \frac{t}{2} + \po(t) \right) - \cancel3 - \cancel{\frac{1}{2}} \right) \\
		&= \frac{1}{t} \left( \frac{7}{4} t + \po(t) \right)  \\
		&= \frac{7}{4} + \po(1) \tendsto{t \to 0} \frac{7}{4}. \\
	\end{align*}

	Donc, \[
		\mathrm{d}f(1,1)\,(1,2) = \frac{7}{4}.
	\]
\end{exm}

\begin{rmk}~\\
	\begin{figure}[H]
		\centering
		\begin{asy}
			import solids;
			import graph;
			size(5cm);

			settings.render = 0;
			settings.prc = false;

			path3 par = graph(
				new real(real x) { return x; },
				new real(real x) { return 0; },
				new real(real x) { return x^2; },
				0,3);
			revolution r = revolution(par, axis=Z);

			path3 par2 = graph(
				new real(real x) { return x; },
				new real(real x) { return 0; },
				new real(real x) { return x^2; },
				-3,3);

			draw(r,1,longitudinalpen=nullpen);
			draw(r.silhouette());

			draw((-4, 0, -1) -- (-4, 0, 10) -- (4, 0, 10) -- (4, 0, -1) -- cycle, red);
			draw(par2, deepred);

			draw((4,4.5) -- (7, 4.5), black+0.5mm, Arrow(TeXHead));

			path par2d = graph(new real(real x) { return x^2; }, -3, 3);
			draw(shift((11, 0)) * par2d, deepred);

			dot(O);
			dot((11, 0));
		\end{asy}
	\end{figure}
\end{rmk}


%todo ajouter théorème-définition
\begin{thm}
	Soit $f : U \to \R$, $(a,b) \in U$. On suppose que $\frac{\partial f}{\partial x}$ et $\frac{\partial f}{\partial y}$ existent en $(a,b)$ et sont {\bfseries continues} en $(a,b)$. Alors,
	\begin{align*}
		&\forall (h, k) \in \R^2 \text{ tel que } (a +h, b + k) \in U,\\
		&f(a+ h, b + k) = f(a,b) + h \frac{\partial f}{\partial x}(a,b) + k \frac{\partial f}{\partial y}(a,b) + \po_{(h,k)\to (0,0)}\big(\|(h,k)\|\big).
	\end{align*}

	On dit que $f$ est de classe $\mathcal{C}^1$ si $\frac{\partial f}{\partial x}$ et $\frac{\partial f}{\partial y}$ existent et sont continues.

	\qed
\end{thm}

\begin{rmk}
	En physique, cette formule correspond à : \[
		\mathrm{d}f = \frac{\partial f}{\partial x}\mathrm{d}x + \frac{\partial f}{\partial y} \mathrm{d}y.
	\] En effet :
	\begin{align*}
		\mathrm{d}f &= f(x+ \mathrm{d}x, y + \mathrm{d}y) - f(x,y) \\
		&= \frac{\partial f}{\partial x} \mathrm{d}x + \frac{\partial f}{\partial y} \mathrm{d}y.
	\end{align*}
\end{rmk}

\begin{prop}
	Soit $f: U \to \R$ de classe $\mathcal{C}^1$ en $(a,b) \in U$. Alors, \[
		\forall w = (w_1, w_2) \in \R^2, \mathrm{d}f(w)\,(a,b) = w_1 \frac{\partial f}{\partial x}(a,b) + w_2 \frac{\partial f}{\partial y}(a,b).
	\]
\end{prop}

\begin{prv}
	Soit $w = (w_1, w_2) \in \R^2$. Soit $t \in \R^*$.
	\begin{align*}
		\frac{1}{t}\big(f(a + tw_1, b + tw_2) - f(a,b)\big)
		&= \frac{1}{t} \left( tw_1 \frac{\partial f}{\partial x}(a,b) + tw_2 \frac{\partial f}{\partial y}(a,b) + \po_{t \to 0}\big(\|tw\|\big) \right) \\
		&= w_1 \frac{\partial f}{\partial x}(a,b) + w_2 \frac{\partial f}{\partial y}(a,b) + \po_{t\to 0}(1) \\
		&\tendsto{t\to 0} w_1 \frac{\partial f}{\partial x}(a,b) + w_2\frac{\partial f}{\partial y}(a,b).
	\end{align*}
\end{prv}


\begin{defn}
	Avec les hypothèses précédentes, en posant \[
		\nabla f(a,b) = \left( \frac{\partial f}{\partial x}(a,b), \frac{\partial f}{\partial y}(a,b) \right) 
	\]on obtient \[
		\mathrm{d}f(w)\,(a,b) = \left<w  \mid \nabla f(a,b) \right>
	\] où $\left<\cdot|\cdot \right>$ est le produit scalaire.

	Le vecteur $\nabla f(a,b)$ est appelé \underline{gradient de $f$ en $(a,b)$}.

	Le développement limité à l'ordre 1 de $f$ devient \[
		f\big((a,b)+w\big) = f(a,b) + \left<w \mid \nabla f(a,b) \right> + \po_{w\to 0}(\|w\|)
	\]
\end{defn}

\begin{prop}
	Soit $f : U \to \R$ de classe $\mathcal{C}^1$.

	\begin{figure}[H]
    \centering
    \incfig{gradient}
	\end{figure}

	$\nabla f$ est orthogonal au lignes de niveaux de $f$, son orientation va dans le sens d'une augmentation de $f$.
\end{prop}

\begin{prv}
	Soit $\gamma : I \to U$ une courbe de niveau : \[
		\forall t \in I, f\big(\gamma(t)\big) = \text{cste}.
	\] D'après le lemme suivant : \[
		\forall t \in I, 0 = (f \circ \gamma)'(t) = \mathrm{d}f\big(\gamma'(t)\big)\big(\gamma(t)\big) = \left<\gamma'(t)  \mid \nabla f\big(\gamma(t)\big) \right>
	\] Donc $\nabla f\big(\gamma(t)\big)$ est orthogonal à $\gamma'(t)$.

	Pour tout $t \in I$, on pose $w(t) = t\, \nabla f\big(\gamma(t)\big)$. Donc \[
		f\big(\gamma(t) + w(t)\big) = f\big(\gamma(t)\big) + t \|\nabla f(\gamma(t))\|^2 + \po_{t \to 0}(t)
	\] Pour $t$ assez petit, $f\big(\gamma(t) + w(t)\big) - f\big(\gamma(t)\big)$ est du même signe que $t$.
\end{prv}

\begin{rmk}
	\begin{align*}
		V: \R^3 &\longrightarrow \R \\
		(x,y,z) &\longmapsto -mgz
	\end{align*}
	l'énerge potentielle de pesenteur

	On a donc \[
		\nabla V(x,y,z) = \left( \frac{\partial V}{\partial x}, \frac{\partial V}{\partial y}, \frac{\partial V}{\partial z} \right) = (0, 0, -mg) = \vec{P}.
	\]
\end{rmk}

\begin{lem}
	Soit $f : U \to \R$ de classe $\mathcal{C}^1$, $\gamma : \begin{array}{rcl}
		I &\longrightarrow& U \\
		t &\longmapsto& \big(x(t), y(t)\big)
	\end{array}$ où $x$ et $y$ sont dérivables.

	On pose \[
		\forall t \in I, \gamma'(t) = \big(x'(t), y'(t)\big).
	\] Alors $f \circ \gamma : I \to \R$ est dérivable et
	\begin{align*}
		\forall t \in I, (f \circ \gamma)'(t) &= \mathrm{d}f\big(\gamma'(t)\big) \big(\gamma(t)\big)\\
		&= \left<\gamma'(t)  \mid \nabla f\big(\gamma(t)\big)  \right> \\
		&= x'(t) \frac{\partial f}{\partial x}\big(x(t), y(t)\big) + y'(t) \frac{\partial f}{\partial y}\big(x(t),y(t)\big). \\
	\end{align*}
\end{lem}

\begin{prv}
	On fixe $t \in I$.

	\begin{align*}
		\forall h \neq 0, \frac{f \circ \gamma(t + h) - f \circ \gamma(t)}{h}
		&= \frac{1}{h}\big(f(\gamma(t)) + h\gamma'(t) + \po_{h\to 0}(h) - f(\gamma(t))\big) \\
		&= \frac{1}{h}\bigg(\cancel{f(\gamma(t))} + \left<h\gamma'(t) \mid \nabla f(\gamma(t)) \right> + \po_{h\to 0}(\|h\gamma'(t)\|) - \cancel{f(\gamma(t))}\bigg)\\
		&= \left<\gamma'(t) \mid \nabla f(\gamma(t)) \right> + \po_{h\to 0}(1) \\
		&\tendsto{h\to 0} \left<\gamma'(t)  \mid \nabla f(\gamma(t)) \right>
	\end{align*}
\end{prv}

\begin{defn}
	Soit $f : U \to \R$ de classe $\mathcal{C}^1$ et $(a,b) \in U$. On dit que $(a,b)$ est un \underline{point critique} de $f$ si $\nabla f(a,b) = 0$ i.e. $\frac{\partial f}{\partial x}(a,b) = \frac{\partial f}{\partial y}(a,b) = 0$.

	Dans ce cas, $f(a,b)$ est appelé \underline{valeur critique} de $f$.
\end{defn}

\begin{prop}~\\
	\begin{minipage}{\linewidth}
		\begin{wrapfigure}{r}{3cm}
			\centering
			\vspace{-1cm}
			\begin{asy}
				import solids;
				import graph;
				size(3cm);

				settings.render = 0;
				settings.prc = false;

				path3 par = graph(
					new real(real x) { return x; },
					new real(real x) { return 0; },
					new real(real x) { return -x^2; },
					0,3);
				revolution r = revolution(par, axis=Z);

				draw(r,1,longitudinalpen=nullpen);
				draw(r.silhouette());

				dot("$(a,b)$", O, red, align=N);
				real s = sqrt(2.5);
				path3 g=(s,0,-2.5)..(0,s,-2.5)..(-s,0,-2.5)..(0,-s,-2.5)..cycle;
				draw(g, deepcyan);
			\end{asy}
		\end{wrapfigure}
		Soit $f: U \to \R$ de classe $\mathcal{C}^1$ et $(a,b) \in U$ tel que \[
			\exists r > 0, \forall (x,y) \in B_{(a,b)}(r), f(x,y) \le f(a,b)
		\] Alors $\nabla f(a,b) = (0,0)$.
	\end{minipage}
\end{prop}

\begin{prv}
	Soit $g: x \mapsto f(x,b)$. $g(a)$ est un maximum local de $g$ donc $g'(a) = 0$.

	Or, $g'(a) = \frac{\partial f}{\partial x}(a,b)$

	donc $\frac{\partial f}{\partial x}(a,b) = 0$.

	Soit $h : y \mapsto f(a,y)$. On a de même $h'(b) = 0$.

	Or, $h'(b) = \frac{\partial f}{\partial y}(a,b)$.

	Donc, $\nabla f(a,b) = (0,0)$.
\end{prv}

\begin{rmk}
	Un minimum local est aussi une valeur critique.
\end{rmk}

\begin{figure}[H]
	\centering
	\begin{subfigure}{3cm}
		\centering
		\begin{asy}
			import solids;
			import graph;
			size(3cm);

			settings.render = 0;
			settings.prc = false;

			path3 par = graph(
				new real(real x) { return x; },
				new real(real x) { return 0; },
				new real(real x) { return -x^2; },
				0,3);
			revolution r = revolution(par, axis=Z);

			draw(r,1,longitudinalpen=nullpen);
			draw(r.silhouette());

			dot(O, red);
		\end{asy}
		\caption{Maximum local}
	\end{subfigure}
	\begin{subfigure}{3cm}
		\centering
		\begin{asy}
			import solids;
			import graph;
			size(3cm);

			settings.render = 0;
			settings.prc = false;

			path3 par = graph(
				new real(real x) { return x; },
				new real(real x) { return 0; },
				new real(real x) { return x^2; },
				0,3);
			revolution r = revolution(par, axis=Z);

			draw(r,1,longitudinalpen=nullpen);
			draw(r.silhouette());

			dot(O, red);
		\end{asy}
		\caption{Minimum local}
	\end{subfigure}
	\begin{subfigure}{3cm}
		\centering
		\begin{asy}
			import solids;
			import graph;
			size(3cm);

			settings.render = 0;
			settings.prc = false;
			currentprojection = obliqueZ;

			draw(graph(
				new real(real x) { return x; },
				new real(real x) { return -x^2 / 3; },
				new real(real x) { return 3; },
				-3, 3
			));

			draw(graph(
				new real(real x) { return x; },
				new real(real x) { return -x^2 / 3; },
				new real(real x) { return -3; },
				-3, 3
			));

			draw(graph(
				new real(real x) { return x; },
				new real(real x) { return -x^2 / 3 - 1; },
				new real(real x) { return 0; },
				-3, 3
			));

			draw(graph(
				new real(real x) { return 0; },
				new real(real x) { return x^2 / 9 - 1; },
				new real(real x) { return x; },
				-3, 3
			));

			draw(graph(
				new real(real x) { return -3; },
				new real(real x) { return x^2 / 9 - 4; },
				new real(real x) { return x; },
				-3, 3
			));

			draw(graph(
				new real(real x) { return 3; },
				new real(real x) { return x^2 / 9 - 4; },
				new real(real x) { return x; },
				-3, 3
			));

			dot((0,-1,0), red);
		\end{asy}
		\caption{Point de selle / Point col}
	\end{subfigure}
\end{figure}

\begin{exm}
	On revient à l'exemple donné en introduction : 
	\begin{align*}
		f: \left( \R^*_+ \right)^2 &\longrightarrow \R \\
		(x,y) &\longmapsto 2\left( xy + \frac{1}{x} + \frac{1}{y} \right).
	\end{align*}

	$\left( \R^+_* \right)^2$ est un ouvert de $\R^2$. Soit $(x,y) \in \left( \R^+_* \right)^2$.
	
	On a \[
		\begin{cases}
			\frac{\partial f}{\partial x}(x,y) = 2\left( y - \frac{1}{x^2} \right),\\
			\frac{\partial f}{\partial y}(x,y) = 2\left( x - \frac{1}{y^2} \right).
		\end{cases}
	\]

	\begin{align*}
		&\frac{\partial f}{\partial x}(x,y) = \frac{\partial f}{\partial y}(x,y) = 0\\
		\iff& \begin{cases}
			y = \frac{1}{x^2}\\
			x = \frac{1}{y^2}
		\end{cases}\\
		\iff& \begin{cases}
			y = \frac{1}{x^2}\\
			x = x^4
		\end{cases}\\
		\iff& \begin{cases}
			x = 1\\
			y = 1
		\end{cases}
	\end{align*}

	On vérivie que $f$ présente en effet un minium local en $(1,1)$. \[
		f(1,1) = 6
	\] On fixe $y \in \R^+_*$ et \[
		g : x \mapsto 2\left( xy + \frac{1}{x} + \frac{1}{y} \right).
	\] Donc \[
		\forall x \in \R^+_*, g'(x) = 2\left( y - \frac{1}{x^2} \right).
	\]
	\begin{center}
		\begin{tikzpicture}
			\tkzTabInit{$x$/1,$g'(x)$/1,$g$/2.3}{$0$, $\frac{1}{\sqrt{y}}$, $+\infty$}
			\tkzTabLine{,-,z,+,}
			\tkzTabVar{+/{}, -/$2\left( 2\sqrt{y} +\frac{1}{y} \right)$, +/{}}
		\end{tikzpicture}
	\end{center}
	
	Ainsi, \[
		\forall x \in \R^+_*, \forall y \in \R^+_*, f(x,y) \ge 2\left( 2\sqrt{y} + \frac{1}{y} \right)
	\] Soit $h : y \mapsto 2\sqrt{y} + \frac{1}{y}$. On a \[
		\forall y > 0, h'(y) = \frac{1}{\sqrt{y}} - \frac{1}{y^2} = \frac{y\sqrt{y} - 1}{y^2} = \frac{y^{\frac{3}{2}} - 1}{y^2}
	\]

	\begin{center}
		\begin{tikzpicture}
			\tkzTabInit{$y$/0.7,$h'(y)$/0.7,$h$/1.4}{$0$, $1$, $+\infty$}
			\tkzTabLine{,-,z,+,}
			\tkzTabVar{+/{}, -/$3$, +/{}}
		\end{tikzpicture}
	\end{center}

	Donc, \[
		\forall x,y > 0, f(x,y) \ge 2\times 3 = 6 = f(1,1).
	\]
\end{exm}

\begin{prop}
	[règle de la chaîne]

	Soit $f : \begin{array}{rcl}
		U &\longrightarrow& \R^2 \\
		(x,y) &\longmapsto& f(x,y)
	\end{array}$ de classe $\mathcal{C}^1$ et $U, V$ deux ouverts de $\R^2$.

	Soit $\varphi : \begin{array}{rcl}
		V &\longrightarrow& U \\
		(u,v) &\longmapsto& \varphi(u,v) = \big(x(u,v), y(u,v)\big)
	\end{array}$.

	On suppose que $x$ et $y$ sont de classe $\mathcal{C}^1$ sur $V$.

	Alors,  $f \circ \varphi : \begin{array}{rcl}
		V &\longrightarrow& \R \\
		(u,v) &\longmapsto& f\big(\varphi(u,v)\big)
	\end{array}$ est de classe $\mathcal{C}^1$ et
	\begin{align*}
		\forall (u_0, v_0) \in V, \frac{\partial (f \circ \varphi)}{\partial u}(u_0, v_0)
		&= \frac{\partial f}{\partial x}\big(\varphi(u_0, v_0)\big) \times \frac{\partial x}{\partial u}(u_0, v_0)\\
		&+ \frac{\partial f}{\partial y}\big(\varphi(u_0,v_0)\big) \frac{\partial y}{\partial u}(u_0,v_0)
	\end{align*}
	\begin{align*}
		\forall (u_0, v_0) \in V, \frac{\partial (f \circ \varphi)}{\partial v}(u_0, v_0)
		&= \frac{\partial f}{\partial x}\big(\varphi(u_0, v_0)\big) \times \frac{\partial x}{\partial v}(u_0, v_0)\\
		&+ \frac{\partial f}{\partial y}\big(\varphi(u_0,v_0)\big) \frac{\partial y}{\partial v}(u_0,v_0)
	\end{align*}
\end{prop}

\begin{exm}
	[changement de coordonnées polaires]
	On pose \begin{align*}
		\varphi: \R^+_* \times ]0,2\pi[ &\longrightarrow \R^2\setminus \left( R^+_* \times \{0\} \right) \\
		(r, \theta) &\longmapsto (r \cos \theta, r \sin\theta),
	\end{align*}
	\begin{align*}
		f: \R^2\setminus \left( R^+_* \times \{0\} \right) &\longrightarrow \R \\
		(x,y) &\longmapsto f(x,y),
	\end{align*}
	\begin{align*}
		g: \overbrace{\R^+_* \times ]0, 2\pi[}^{=V} &\longrightarrow \R \\
		(r, \theta) &\longmapsto f(r\cos\theta, r\sin\theta).
	\end{align*}

	\begin{align*}
		\forall (r_0,\theta_0) \in V,&\\[5mm]
		\frac{\partial g}{\partial r}(r_0, \theta_0) &= \frac{\partial f}{\partial x}(r_0\cos\theta_0, r_0\sin\theta_0)\cos\theta_0\\
		&+ \frac{\partial f}{\partial y}(r_0 \cos\theta_0, r_0\sin\theta_0)\sin\theta_0\\
		&= 2r_0\cos^2\theta_0 + 2r_0\sin^2(\theta_0) \\
		&= 2r_0 \\[5mm]
		\frac{\partial g}{\partial \theta}(r_0, \theta_0) &= \frac{\partial f}{\partial x}(r_0\cos\theta_0, r_0\sin\theta_0)r_0\sin\theta_0\\
		&+ \frac{\partial f}{\partial y}(r_0 \cos\theta_0, r_0\sin\theta_0)r_0\cos\theta_0\\
		&= -2{r_0}^2\cos(\theta_0)\sin(\theta_0) + 2{r_0}^2 \sin(\theta_0)\cos(\theta_0)\\
		&= 0 \\
	\end{align*}

	Donc, \[
		g(r, \theta) = r^2.
	\]
\end{exm}

\begin{exm}
	Résoudre \[
		\begin{cases}
			\frac{\partial f}{\partial x} = \frac{x}{x^2+y^2},\\
			\frac{\partial f}{\partial y} = \frac{y}{x^2+y^2}.\\
		\end{cases}
	\]

	On pose $g: (r, \theta) \mapsto f(r \cos\theta, r \sin\theta)$.

	\begin{align*}
		&\frac{\partial g}{\partial r} = \frac{1}{r}\cos^2\theta + \frac{1}{r}\sin^2\theta = \frac{1}{r},\\
		&\frac{\partial g}{\partial \theta} = -\cos(\theta) \sin(\theta) + \sin(\theta)\cos(\theta) = 0.
	\end{align*}

	Donc, \[
		\exists C \in \R, g: (r, \theta) \mapsto \ln r + C
	\] d'où,
	\begin{align*}
		\forall (x,y) \in \R^2 \setminus \{(0,0)\}, f(x,y) &= \ln\left(\sqrt{x^2 + y^2} \right)  + C\\
		&= \frac{1}{2}\ln(x^2 + y^2) + C. \\
	\end{align*}
\end{exm}

\begin{rmk}
	Soit $\mathcal{B} = (e_1, e_2)$ la base canonique de $\R^2$, $f: U \to \R$ de classe $\mathcal{C}^1$ avec $U$ un ouvert de $\R^2$.

	Soit $(x,y) \in U$.

	\begin{align*}
		\Mat_{\mathcal{B}}\big(\nabla f(x,y)\big) = \begin{pmatrix}
			\frac{\partial f}{\partial x}(x,y)\\[2mm]
			\frac{\partial f}{\partial y}(x,y)
		\end{pmatrix}
	\end{align*}

	Soit  \begin{align*}
		\varphi: V &\longrightarrow U \\
		(u,v) &\longmapsto \big(x(u,v), y(u,v)\big) 
	\end{align*} avec $x,y$ de classe $\mathcal{C}^1$. Soit $g = f \circ \varphi$.
	\begin{align*}
		\Mat_{\mathcal{B}}\big(\nabla g(u,v)\big)
		&= \begin{pmatrix}
			\frac{\partial g}{\partial u}(u,v) \\[2mm]
			\frac{\partial g}{\partial v}(u,v)
		\end{pmatrix} \\
		&= \begin{pmatrix}
			\frac{\partial x}{\partial u}(u,v) \frac{\partial f}{\partial x}(x,y)
			+ \frac{\partial y}{\partial u}(u,v)\frac{\partial f}{\partial y}(x,y)\\[3mm]
			\frac{\partial x}{\partial v}(u,v) \frac{\partial f}{\partial x}(x,y)
			+ \frac{\partial y}{\partial v}(u,v) \frac{\partial f}{\partial y}(x,y)
		\end{pmatrix}  \\
		&= \underbrace{\begin{pmatrix}
				\frac{\partial x}{\partial u}(u,v)& \frac{\partial y}{\partial u}(u,v)\\[3mm]
				\frac{\partial x}{\partial v}(u,v)& \frac{\partial y}{\partial v}(u,v)
		\end{pmatrix}}_{J(u,v)} \begin{pmatrix}
			\frac{\partial f}{\partial x}(x,y)\\[3mm]
			\frac{\partial f}{\partial y}(x,y)
		\end{pmatrix} \\
		&= J(u,v) \Mat_{\mathcal{B}}\big(\nabla f(x,y)\big) \\
	\end{align*}
	où $J(u,v) = 
	\begin{pNiceArray}{c:c}
		\Mat_{\mathcal{B}}\big(\nabla x(u,v)\big) & \Mat_{\mathcal{B}}\big(\nabla y(u,v)\big)
	\end{pNiceArray}$.

	On dit que $J(u,v)$ est \underline{la jacobienne} de $\varphi$ en $(u,v)$.
	L'application linéaire canoniquement associée à $J(u,v)$ est la \underline{différentielle de $\varphi$} en $(u,v)$ noté $\mathrm{d}\varphi(u,v)$.

	On a $\mathrm{d}\varphi(u,v) \in \mathcal{L}(R^2)$ et $\Mat_{\mathcal{B}}\big(\mathrm{d}\varphi(u,v)\big) = J(u,v)$.

	Par exemple, la jacobienne du changement de coordonnées polaires est \[
		J = \begin{pmatrix}
			\frac{\partial x}{\partial r} & \frac{\partial y}{\partial r}\\[3mm]
			\frac{\partial x}{\partial \theta} & \frac{\partial y}{\partial \theta}
		\end{pmatrix}
		= \begin{pmatrix}
			\cos\theta&\sin\theta\\
			-r\sin\theta&r\cos\theta
		\end{pmatrix}.
	\]
	$\underbrace{\det(J)}_{\text{le jacobien}} = r\cos^2\theta + r\sin^2\theta = r$

	Dans une intégrale double, si $(x,y) = \varphi(u,v)$, alors $\mathrm{d}x\mathrm{d}y = \det(J)\mathrm{d}u\mathrm{d}v$.

	Ici, \[
		\mathrm{d}x\ \mathrm{d}y = r\ \mathrm{d}r\ \mathrm{d}\theta.
	\]
\end{rmk}

\begin{prv}
	On pose $(x_0, y_0) = \varphi(u_0, v_0)$. Pour tout $(h,k) \in \R^2$ tels que $(u_0 + h, v_0 + k) \in V$, en posant $g = f  \circ \varphi$.

	\begin{align*}
		g(u_0 + h, v_0 + h) &= f\big(x(u_0 + h, v_0 + k), y(u_0 + h, v_0 + k)\big) \\
		&= f\left(
			x(u_0,v_0) + h \frac{\partial x}{\partial u}(u_0,v_0) + k \frac{\partial x}{\partial v}(u_0, v_0) + \po\big(\|(h,k)\|\big), \right.\\
		&\phantom{ = f\bigg(\bigg.}\left. y(u_0, v_0) + h \frac{\partial y}{\partial u}(u_0, v_0) + k \frac{\partial y}{\partial v}(u_0, v_0) + \po\big(\|(h,k)\|\big)
		\right)  \\
		&= f(x_0,y_0) \\
		&~+ \left( h \frac{\partial x}{\partial u}(u_0,v_0) + k \frac{\partial x}{\partial v}(u_0, v_0) + \po(\|(h,k)\|) \right) \frac{\partial f}{\partial x}(x_0,y_0)\\
		&~+ \left( h \frac{\partial y}{\partial u}(u_0, v_0) + k\frac{\partial y}{\partial v}(u_0, v_0) + \po(\|(h,k)\|) \right) \frac{\partial f}{\partial y}(x_0, y_0)\\
		&~+ \po(\|(h,k)\|)\\
		&= f(x_0, y_0) \\
		&~+ h \left( \frac{\partial x}{\partial u}(u_0, v_0) \frac{\partial f}{\partial x}(x_0, y_0) + \frac{\partial y}{\partial u}(u_0, v_0) \frac{\partial f}{\partial y}(x_0, y_0) \right)  \\
		&~+ k\left( \frac{\partial x}{\partial v}(u_0, v_0) \frac{\partial f}{\partial x}(x_0, y_0) + \frac{\partial y}{\partial v}(u_0, v_0) \frac{\partial f}{\partial y}(x_0, y_0) \right) 
		&~+ \po(\|(h,k)\|)\\
		&= g(u_0, v_0) + h \frac{\partial g}{\partial u}(u_0, v_0) + k \frac{\partial g}{\partial v}(u_0, v_0) + \po(\|(h,k)\|) \\
	\end{align*}

	Par identification,
	\[
		\frac{\partial g}{\partial u}(u_0, v_0) = \frac{\partial x}{\partial u}(u_0, v_0) \frac{\partial f}{\partial x}(x_0, y_0) + \frac{\partial y}{\partial u}(u_0, v_0) \frac{\partial f}{\partial y}(x_0,y_0)
	\] et \[
		\frac{\partial g}{\partial v}(u_0, v_0) = \frac{\partial x}{\partial v}(u_0,v_0) \frac{\partial f}{\partial x}(x_0, y_0) + \frac{\partial y}{\partial v}(u_0, v_0) \frac{\partial f}{\partial y}(x_0, y_0).
	\] 
\end{prv}

\begin{exm}
	[Régression linéaire]~\\
	\begin{figure}[H]
		\centering
		\begin{asy}
			import graph;
			axes(EndArrow);
			size(5cm);

			real f(real x) { return x + 0.5; }

			real k = 35 / (7 - 0.5);

			for(int i = 0; i < 35; ++i) {
				real mag = exp(sin(100 * pi/exp(1) * i)) * 0.8 + exp(cos(i*40)/3);
				real eps = mag * cos(10 * exp(1)/pi * i) / 3;
				dot((i/k,f(i/k) + eps));
			}

			draw(graph(f, -1, 7), orange);
		\end{asy}
	\end{figure}
	\[
		y = a x + b
	\] 
	On fixe $(a,b) \in \R^2$. \[
		\varepsilon(a,b) = \sum_{i=1}^n\big( y_i - (ax_i + b) \big)^2
	\] l'erreur totale.

	On veut minimiser $\varepsilon(a,b)$. On a 
	\[
		\forall (a,b) \in \R^2,
		\begin{cases}
			\frac{\partial \varepsilon}{\partial a}(a,b) = -2\sum_{i=1}^{n}(y_i - ax_i - b)x_i,\\
			\frac{\partial \varepsilon}{\partial b}(a,b) = -2\sum_{i=1}^{n}(y_i - ax_i - b).
		\end{cases}
	\]

	Donc,
	\begin{align*}
		(a,b) \text{ point critique de } \varepsilon \iff& \begin{cases}
			a \sum_{i=1}^n {x_i}^2 + b\sum_{i=1}^{n}x_i = \sum_{i=1}^{n} y_ix_i\\
			a\sum_{i=1}^{n}x_i + nb = \sum_{i=1}^ny_i
		\end{cases}\\
		\iff& \begin{cases}
			a \left( \frac{1}{n}\sum_{i=1}^n {x_i}^2 - \overline{x}^2\right) = \overline{y} - \overline{x} \overline{y}\\
			b = \frac{1}{n}\sum_{i=1}^ny_i - \frac{a}{n}\sum_{i=1}^nx_i = \frac{1}{n}\sum_{i=1}^n x_i y_i - \overline{x} \overline{y}
		\end{cases}\\
		&\text{ où } \overline{x} = \frac{1}{n} \sum_{i=1}^n x_i,~\overline{y} = \frac{1}{n}\sum_{i=1}^n y_i\\
		\iff& \begin{cases}
			a = \frac{\Cov(x,y)}{V(x)}\\
			b = \overline{y} - a\overline{x}
		\end{cases}
	\end{align*}

	Coefficient de corrélation: $\frac{\Cov(x,y)}{\sigma_x \sigma_y} \in [-1, 1]$
\end{exm}












		\part{Corps}

\begin{exm}[Problème]
	\begin{itemize}
		\item 
			avec $A = \Z / 9 \Z$, résoudre $\overline{x}^2 = \overline{0}$ \\
			\begin{center}
				\begin{tabular}{|c|c|c|c|c|c|c|c|c|c|c|}
					\hline
					$\overline{x}$&$\overline{0}$& $\overline{1}$ &$\overline{2}$&$\overline{3}$ &$\overline{4}$ &$\overline{5}$ &$\overline{6}$ &$\overline{7}$ &$\overline{8}$& $\overline{9}$ \\
					\hline
					$\overline{x}^2$&$\overline{0}$ &$\overline{1}$ &$\overline{4}$ &$\overline{0}$ &$\overline{7}$ &$7$ &$\overline{0}$ &$\overline{4}$ &$\overline{1}$&$\overline{0}$\\
					\hline
				\end{tabular}
			\end{center}
			On a trouvé 3 solutions: $\overline{0}$, $\overline{3}$, $\overline{6}$.
		\item $\Z / 8\Z$
			\begin{center}
				\begin{tabular}{|c|c|c|c|c|c|c|c|c|}
					\hline
					$\overline{x}$& $\overline{0}$& $\overline{1}$& $\overline{2}$& $\overline{3}$& $\overline{4}$& $\overline{5}$& $\overline{6}$& $\overline{7}$\\
					\hline
					$\overline{x^2}$& $\overline{0}$& $\overline{1}$& $\overline{4}$& $\overline{1}$& $\overline{0}$& $\overline{1}$& $\overline{4}$& $\overline{1}$\\
					\hline
				\end{tabular}
			\end{center}
			$\overline{x}^2=7$ a 4 solutions: $\overline{1}, \overline{7}, \overline{3},\text{ et } \overline{5}$
		\item $A = \mathbbm{H} = \{a + bi + cj + dk  \mid  (a,b,c,d) \in \R^4\}$ \\
			$i^2 = j^2 = k^2 = -1$ 
			\begin{align*}
				\begin{array}{c c c}
					ij = k & jk = i & ji = j\\
					ji = -k & kj = -i & ik = -j
				\end{array}
			\end{align*}
			Dans cet anneau, $-1$ a 6 racines!
	\end{itemize}
\end{exm}

\begin{defn}
	Soit $(\mathbbm{K}, +, \times)$ un ensemble muni de deux lois de composition internes. On dit que c'est un \underline{corps} si
	 \begin{enumerate}
		\item $(\mathbbm{K}, \times)$ est un groupe abélien
		\item $(\mathbbm{K}, \times)$ est un monoïde commutatif
		\item $\forall x \in \mathbbm{K}\setminus \{0_\mathbbm{K}\}, \exists y \in \mathbbm{K}, xy = 1_\mathbbm{K}$
		\item $0_\mathbbm{K} \neq  1_\mathbbm{K}$
	\end{enumerate}
	\index{corps}
\end{defn}

\begin{exm}
	\begin{itemize}
		\item $(\C, +, \times)$ est un corps
		\item $(\R, +, \times)$ est un corps
		\item $(\Q, +, \times)$ est un corps
		\item $(\Z, +, \times)$ n'est pas un corps
	\end{itemize}
\end{exm}

\begin{prop}
	$(\Z / n\Z, +, \times)$ est un corps si et seulement si $n$ est premier.
\end{prop}

\begin{prv}
	\[
		\left( \Z / n\Z \right)^\times = \left\{ \overline{k}  \mid k \wedge n = 1 \right\}
	\] 
\end{prv}


\begin{prop}
	Tout corps est un anneau intègre.
\end{prop}

\begin{prv}
	Soit $(\mathbbm{K}, +, \times)$ un corps. Soient $(a,b) \in \mathbbm{K}^2$ tel que $a \times b = 0_\mathbbm{K}$.\\
	On suppose $a \neq  0_\mathbbm{K}$. Alors, $a$ est inversible et donc \[
		b = a^{-1} \times a \times b = a^{-1} \times 0_\mathbbm{K} = 0_\mathbbm{K}
	\] 
\end{prv}

\begin{exm}
	Soit $(\mathbbm{K},+,\times)$ un corps.\\
	Résoudre \[
		\begin{cases}
			x^2 = 1_\mathbbm{K}\\
			x \in \mathbbm{K}
		\end{cases}
	\]

	\begin{align*}
		x^2 = 1_\mathbbm{K} &\iff x^2 - 1_\mathbbm{K} = 0_\mathbbm{K}\\
		&\iff (x - 1_\mathbbm{K})(x+1_\mathbbm{K}) = 0_\mathbbm{K}\\
		&\iff x - 1_\mathbbm{K} = 0_\mathbbm{K} \text{ ou } x + 1_\mathbbm{K} = 0_\mathbbm{K}\\
		&\iff x = 1_\mathbbm{K} \text{ ou } x = -1_\mathbbm{K}
	\end{align*}

	Il y a au plus 2 solutions.
\end{exm}

\begin{prop}
	Soit $(\mathbbm{K},+,\times )$ un corps et $P$ un polynôme à coefficients dans $\mathbbm{K}$ de degré $n$. Alors, l'équation $P(x) = 0_{\mathbbm{K}}$ a au plus $n$ solutions dans $\mathbbm{K}$ 
	\qed
\end{prop}

\begin{crlr}[(Théorème de Wilson)]
	voir exercice 16 du TD 12
\end{crlr}


\begin{defn}
	Soit $(\mathbbm{K}, +, \times)$ un corps et $L\subset \mathbbm{K}$.\\
	On dit que $L$ est un \underline{sous corps} de $\mathbbm{K}$ si
	\begin{enumerate}
		\item $L$ est un anneau de $(\mathbbm{K}, +, \times)$ non nul
		\item $\forall x \in L\setminus \{0_\mathbbm{K}\}, x^{-1} \in L$ 
	\end{enumerate}
	\vspace{2mm}
	en d'autres termes si
	\begin{enumerate}
		\item $\forall (x,y) \in L^2, x - y \in L$
		\item $\forall (x,y) \in L^2, x \times y^{-1} \in L$
	\end{enumerate}
	\vspace{5mm}
	On dit aussi que $\mathbbm{K}$ est une \underline{extension} de $L$.
	\index{sous corps}
	\index{extension}
\end{defn}

\begin{prop}
	Tout sous corps est un corps. \qed
\end{prop}

\begin{defn}
	Soient $(\mathbbm{K}_1,+,\times )$ et $(\mathbbm{K}_2,+, \times)$ deux corps et $f: \mathbbm{K}_1 \to \mathbbm{K}_2$.\\
	On dit que $f$ est un \underline{morphisme de corps} si $f$ est un morphisme d'anneaux.\\
	i.e. si
	\[
		\begin{cases}
			\forall (x,y) \in {\mathbbm{K}_1}^2,& f(x+y) = f(x) + f(y)\\
			\forall (x,y) \in {\mathbbm{K}_1}^2,& f(x \times y) = f(x) \times f(y)\\
		\end{cases}
	\] 
	\index{homomorphisme (de corps)}
	\index{morphisme (de corps)}
\end{defn}

\begin{prop}
	Tout morphisme de corps est injectif.
\end{prop}

\begin{prv}
	Soit $f: \mathbbm{K}_1 \to \mathbbm{K}_2$ un morphisme de corps.\\
	\begin{itemize}
		\item $\Ker(f)$ est un sous groupe de $(\mathbbm{K}_1, +)$ 
		\item Soit $x \in \Ker(f)$ et $y \in \mathbbm{K}_1$ \[
				f(x \times y) = f(x) \times f(y) = 0_{\mathbbm{K}_2} \times f(y) = 0_{\mathbbm{K}_2}
			\]
		\item Soit $x \in \Ker(f) \setminus \{0_{\mathbbm{K}_1}\}$.\\
			Alors, $x$ est inversible.\\
			\begin{align*}
				\begin{rcases*}
					x \in \Ker(f)\\
					x^{-1} \in \mathbbm{K}_1
				\end{rcases*}& \text{ donc } x \times x ^{-1} \in \Ker(f)\\
				&\text{ donc } 1_{\mathbbm{K}_1} \in \Ker(f)\\
				&\text{ donc } f(1_{\mathbbm{K}_1}) = 0_{\mathbbm{K}_2}
			\end{align*}
			Or, $f(1_{\mathbbm{K}_1}) = 1_{\mathbbm{K}_2} \neq 0_{\mathbbm{K}_2}$
	\end{itemize}
	Donc, $\Ker(f) = \{0_{\mathbbm{K}_1}\}$ donc $f$ est injective.
\end{prv}

\begin{exm}
	$\begin{array}{cc}
		\C &\longrightarrow \C\\
		z &\longmapsto \overline{z}\\
	\end{array}$ est un morphisme de corps
\end{exm}



		\part{Opérations sur les séries}

\begin{prop}
	L'ensemble $E = \{u \in \C^\N  \mid \Sigma u_n \text{ converge}\}$ est un sous-espace vectoriel de $\C^\N$ et \begin{align*}
		S: E &\longrightarrow \C \\
		u &\longmapsto \sum_{n=0}^{+\infty} u_n
	\end{align*} est une forme linéaire.
	\qed
\end{prop}

\begin{rmk}
	La somme d'une série convergente et d'une série divergente diverge.
	Le produit d'une série divergente par un scalaire non nul diverge.
\end{rmk}

	}

	{
		\chap[27]{Espace probabilisé fini}
		\renewcommand{\cwd}{../chap27}
		\let\overlin\overline
		\let\overline\bar
		\begin{defn}
	Soit $E$ un $\mathbbm{K}$-espace vectoriel. On dit que $E$ est de \underline{dimension finie} si $E$ a au moins une famille génératrice finie. On dit que $E$ est de \underline{dimension infinie} sinon.
	\index{dimension finie (espace vectoriel)}
	\index{dimension infinie (espace vectoriel)}
\end{defn}

\begin{thm}
	[Théorème de la base extraite]
	Soit $E$ un $\mathbbm{K}$-espace vectoriel non nul de dimension finie. Soit $\mathcal{G}$ une famille génératrice finie de $E$. Alors, il existe une base $\mathcal{B}$ de $\mathcal{E}$ telle que $\mathcal{B} \subset \mathcal{G}$.
\end{thm}

\begin{prv}
	[par récurrence sur $\#G = \Card(G)$]
	\begin{itemize}
		\item Soit $E$ un $\mathbbm{K}$-espace vectoriel non nul engendré par $\mathcal{G} = (u)$.\\
			Si $u = 0_E$, alors $E = \{0_E\}$: une contradiction $\lightning$ \\
			Donc $u \neq 0_E$ donc $(u)$ est libre. En effet, \[
				\forall \lambda \in \mathbbm{K}, \lambda u = 0_E \implies \lambda = 0_\mathbbm{K}
			\] Donc $\mathcal{G}$ est une base de $E$.\\
		\item Soit $n \in \N_*$. Soit $E$ un $\mathbbm{K}$-espace vectoriel. On suppose que si $E$ a une famille génératrice constituée de $n$ vecteurs, alors on peut extraire de cette famille une base de $E$.\\
			Soit $\mathcal{G}$ une famille génératrice de $E$ avec $n+1$ vecteurs.\\
			Si $\mathcal{G}$ est libre, alors $\mathcal{G}$ est une base de $E$. \\
			Si $\mathcal{G}$ n'est pas libre, alors il existe $u \in \mathcal{G}$ tel que $u \in \Vect(\mathcal{G}\setminus \{u\})$ \\
			Donc $\mathcal{G}\setminus \{u\}$ engendre $E$. Or, $\mathcal{G}\setminus \{u\}$ possède $n$ vecteurs. D'après l'hypothèse de récurrence, il existe une base $\mathcal{B}$ de $E$ telle que \[
				\mathcal{B} \subset \mathcal{G} \setminus \{u\} \subset \mathcal{G}
			\] 
	\end{itemize}
\end{prv}

\begin{crlr}
	Tout espace de dimension finie a une base.
	\qed
\end{crlr}

\begin{thm}
	[Théorème de la base incomplète]
	Soit $E$ un $\mathbbm{K}$-espace vectoriel de dimension finie, $\mathcal{G}$ une famille génératrice finie de $E$. $\mathcal{L}$ une famille libre de $E$. Alors, il existe une base $\mathcal{B}$ de $E$ telle que \[
		\mathcal{L} \subset \mathcal{B} \text{ et } \mathcal{B}\setminus \mathcal{L} \subset \mathcal{G}
	\] 
\end{thm}

\begin{prv}
	[par récurrence sur $\#(\mathcal{G}\setminus\mathcal{L})$]
	\begin{itemize}
		\item Avec les notations précédentes, on suppose que $\mathcal{G}\setminus\mathcal{L} \neq \O$ \[
				\forall u \in \mathcal{G}, u \in \mathcal{L}
			\] Donc $\mathcal{G} \subset \mathcal{L}$ donc $\mathcal{L}$ est génératrice donc $\mathcal{L}$ est une base de $E$. On pose $\mathcal{B} = \mathcal{L}$ et alors \[
				\mathcal{L} \subset  \mathcal{B} \text{ et } \mathcal{B}\setminus\mathcal{L} = \O \subset  \mathcal{G}
			\] 
		\item Soit $n \in \N$. On suppose que si $\mathcal{G}$ est génératrice et $\mathcal{L}$ libre avec $\#(\mathcal{G}\setminus\mathcal{L}) = n$ alors il existe une base $\mathcal{B}$ de $E$ telle que \[
			\mathcal{L}\subset \mathcal{B} \text{ et } \mathcal{B}\setminus\mathcal{L}\subset \mathcal{G}
		\] Soient à présent $\mathcal{G}$ une famille génératrice de $E$ et $\mathcal{L}$ une famille libre de $E$ telles que $\#(\mathcal{G}\setminus\mathcal{L}) = n+1 > 0$\\
		Si $\mathcal{L}$ engendre $E$, alors $\mathcal{L}$ est une base de $E$. On pose $\mathcal{B} = \mathcal{L}$ et on a bien \[
			\mathcal{L} \subset  \mathcal{B} \text{ et } \mathcal{B} \setminus \mathcal{L} = \O \subset  \mathcal{G}
		\] On suppose que $\mathcal{L}$ n'engendre pas $E$. Il existe $u \in \mathcal{G}$ tel que $u \not\in \Vec(\mathcal{L})$ (car sinon, $\mathcal{G} \subset \Vect(\mathcal{L})$ et donc $\underbrace{\Vect(\mathcal{G})}_{= E} \subset  \underbrace{\Vect(\mathcal{L})}_{ \subset E}$\\
		Donc $\mathcal{L} \cup \{u\} $ est libre. On pose $\mathcal{L}' = \mathcal{L} \cup \{u\} $ \[
			\mathcal{G}\setminus \mathcal{L}' = \mathcal{G}\setminus (\mathcal{L} \cup \{u\}) = (\mathcal{G}\setminus\mathcal{L})\setminus \{u\} 
		\] donc $\#(\mathcal{G}\setminus\mathcal{L}') = n+1 -1 = n$\\
		D'après l'hypothèse de récurrence, il existe $\mathcal{B}$ une base de $E$ telle que \[
			\mathcal{L} \subset  \mathcal{L}' \subset \mathcal{B} \text{ et } \mathcal{B}\setminus \mathcal{L}' \subset \mathcal{G}
		\] \[
			\mathcal{B} \setminus \mathcal{L} = \underbrace{\mathcal{B}\setminus\mathcal{L}'}_{\subset \mathcal{G}} \cup \underbrace{\{u\}}_{\subset \mathcal{G} \text{ car } u \in \mathcal{G}}
		\] On a $\mathcal{B}\setminus\mathcal{L}\subset \mathcal{G}$
	\end{itemize}
\end{prv}

\begin{thm}
	Soit $E$ un $\mathbbm{K}$-espace vectoriel de dimension finie. Toutes les bases de $E$ ont le même cardinal.
\end{thm}

\begin{prv}
	Soit $\mathcal{G}$ une famille génératrice finie de $E$ et $\mathcal{B} \subset  \mathcal{G}$ une base de $E$. On note $n = \#\mathcal{B}$ \\
	Soit $\mathcal{B}'$ une base de $E$. On pose $p = n - \#(\mathcal{B} \cap  \mathcal{B}')$. Montrons par récurrence sur  $p$ que $\#\mathcal{B} = \#\mathcal{B}'$ 
	\begin{itemize}
		\item On suppose que $p = 0$. Alors, $\#(\mathcal{B} \cap \mathcal{B}') = n$ \\
			Or, $\mathcal{B}' \cap \mathcal{B} \subset \mathcal{B}$ donc $\mathcal{B} \cap \mathcal{B}' = \mathcal{B}$ donc $\mathcal{B} \subset  \mathcal{B}'$ et donc $\mathcal{B} = \mathcal{B}'$ 
		\item Soit $p \in \N$. On suppose que si $\mathcal{B}'$ est une base de $E$ telle que $n - \#(\mathcal{B} \cap \mathcal{B}') = p$, alors $\#\mathcal{B}' = n$ \\
			Aoit $\mathcal{B}'$ une base de $E$ telle que $n - \#(\mathcal{B}\cap \mathcal{B}') = p+1 > 0$ \\
			Donc $\mathcal{B} \cap \mathcal{B}' \neq \mathcal{B}$. Soit $u \in \mathcal{B}' \setminus \mathcal{B}$. D'après le lemme d'échange, il existe $v \in \mathcal{B}\setminus \mathcal{B}'$ tel que $\mathcal{B}' \setminus \{u\} \cup \{v\}$ est une base de $E$. On pose $\mathcal{B}'' = \mathcal{B}' \setminus \{u\} \cup \{v\}$ 
			\begin{align*}
				\mathcal{B}'' \cap \mathcal{B} &= \left( (\mathcal{B}' \setminus \{u\})  \cap \mathcal{B} \right) \cup \{v\} \\
				&= (\mathcal{B}' \cap \mathcal{B}) \cup \{v\} \\
			\end{align*}
			donc,
			\begin{align*}
				n - \#(\mathcal{B}'' \cap \mathcal{B}) &= n - (\#(\mathcal{B}' \cap \mathcal{B}) + 1) \\
				&= p+1- 1 \\
				&= p \\
			\end{align*}
			D'après l'hypothèse de récurrence, \[
				\#\mathcal{B}'' = n
			\] Or, $\#\mathcal{B}'' = \#\mathcal{B}'$
	\end{itemize}
\end{prv}

\begin{lem}
	Soient $\mathcal{B}$ et $\mathcal{B}'$ deux bases de $E$ telles que $\mathcal{B}\subset \mathcal{B}'$. Alors, $\mathcal{B} = \mathcal{B}'$.
\end{lem}

\begin{prv}
	On suppose $\mathcal{B}' \neq \mathcal{B}$. Soit $u \in \mathcal{B}' \setminus \mathcal{B}$
	$u \in E = \Vect(\mathcal{B})$ donc $\mathcal{B} \cup \{u\}$ n'est pas libre.
	Donc $\mathcal{B}\cup \{u\} \subset \mathcal{B}'$ et $\mathcal{B}'$ est libre donc $\mathcal{B}\cup \{u\}$ est libre: une contradiction $\lightning$
\end{prv}

\begin{lem}
	[Lemme d'échange] Soient $\mathcal{B}_1$ et $\mathcal{B}_2$ deux bases de $E$ et $u \in \mathcal{B}_1 \setminus \mathcal{B}_2$. Alors, il existe $v \in \mathcal{B}_2$ tel que $(\mathcal{B}_1 \setminus \{u\}) \cup \{v\}$ soit une base de $E$.
\end{lem}

\begin{prv}
	[1${}^\text{nde}$ méthode]
	On suppose que pout tout $v \in \mathcal{B}_2$, $(\mathcal{B}_1\setminus \{u\}) \cup \{v\}$ n'est pas une base de $E$
	Soit $v \in \mathcal{B}_2$.
	\begin{itemize}
		\item Supposons $(\mathcal{B}_1\setminus \{u\})\cup \{v\}$ non libre. $\mathcal{B}_1 \setminus \{u\}$ est libre. Donc $v \in \Vect(\mathcal{B}_1 \setminus \{u\})$
		\item Supposons $(\mathcal{B}_1\setminus \{u\}) \cup \{v\}$ non génératrice.
			Comme $\mathcal{B}_1$ engendre $E$, $u \not\in \Vect(\mathcal{B}_1\setminus \{v\})$.
			On suppose que $\mathcal{B}_1 \neq \mathcal{B}_2$.
			$\forall v \in \mathcal{B}_2 \setminus \mathcal{B}_1, \Vect(\mathcal{B}_1 \setminus \{v\}) = \Vect(\mathcal{B}_1) = E \ni u$ 
			donc, $(\mathcal{B}_1\setminus \{u\}) \cup \{v\}$ engendre $E$ et donc \[
				v \in \Vect(\mathcal{B}_1 \setminus \{u\})
			\] On a aussi \[
				\forall v \in \mathcal{B}_1 \setminus \{u\}, v \in \Vect(\mathcal{B}_1\setminus \{u\})
			\] Comme $u \not\in \mathcal{B}_2$, on a \[
				\forall v \in \mathcal{B}_2, v \in \Vect(\mathcal{B}_1\setminus \{u\})
			\] docn \[
				E = \Vect(\mathcal{B}_2) \subset \Vect(\mathcal{B}_1\setminus \{u\})
			\] donc $\mathcal{B}_1\setminus \{u\}$ engendre $E$ donc $\mathcal{B}_1\setminus \{u\}$ est une base de $E$. Or, $\mathcal{B}_1 \setminus \{u\}  \subset  \mathcal{B}_1$, donc $\mathcal{B}_1\setminus \{u\} = \mathcal{B}_1$
	\end{itemize}
\end{prv}

\begin{prv}
	[2${}^\text{nde}$ méthode]
	On suppose que pout tout $v \in \mathcal{B}_2$, $(\mathcal{B}_1\setminus \{u\}) \cup \{v\}$ n'est pas une base de $E$
	\begin{itemize}
		\item Comme $u \in \mathcal{B}_1 \setminus \mathcal{B}_2$, nécéssairement $\mathcal{B}_1 \neq \mathcal{B}_2$ donc $\mathcal{B}_2 \not\subset \mathcal{B}_1$, donc $\mathcal{B}_2\setminus\mathcal{B}_1 \neq \O$ 
		\item Soit $v \in \mathcal{B}_2\setminus\mathcal{B}_1$. Il existe $(\lambda_w)_{w\in\mathcal{B}_1}$ une famille de scalaires presque nulle telle que \[
				v = \sum_{w \in \mathcal{B}_1} \lambda_w w - \lambda_u u + + \sum_{w \in \mathcal{B}_1\setminus \{u\}}\lambda_w w
			\]
			Si $\lambda_u \neq 0_E$, alors
			\begin{align*}
				u &= \lambda_u^{-1}\left( v - \sum_{w \in \mathcal{B}_1 \setminus \{u\}} \lambda_w w \right)\\
					&\in \Vect(\mathcal{B}_1\setminus \{u\} \cup v)
			\end{align*}
			 donc $\mathcal{B}_1 \subset \Vect(\mathcal{B}_1\setminus \{u\} \cup \{v\})$\\
			 et donc $E \subset  \Vect(\mathcal{B}_1 \setminus \{u\} \cup \{v\})$ \\
			 et donc $\mathcal{B}_1 \setminus \{u\} \cup \{v\}$ engendre $E$ \\
			 donc $\mathcal{B}_1 \setminus \{u\} \cup \{v\}$ n'est pas libre\\
			 donc $v \in \Vect(\mathcal{B}_1\setminus \{u\})$ (car $\mathcal{B}_1 \setminus \{u\}$ est libre\\
			 donc $\lambda_u = 0_\mathbbm{K}$ $\lightning$\\`

			 Donc, $\lambda_u = 0_\mathbbm{K}$, docn $v \in \Vect(\mathcal{B}_1\setminus \{u\})$ \\
			 On vient de prouver que
			 \begin{align*}
			 	\mathcal{B}_2 \setminus \mathcal{B}_1 \subset \Vect(\mathcal{B}_1 \setminus \{u\})\\
			 	\mathcal{B}_1 \setminus \{u\} \subset \Vect(\mathcal{B}_1 \setminus \{u\})\\
			 \end{align*}
			 Comme $u \not\in \mathcal{B}_2$, \[
			 	\mathcal{B}_2 \subset \Vect(\mathcal{B}_1 \setminus \{u\})
			 \] donc \[
			 	E = \Vect(\mathcal{B}_2) \subset  \Vect(\mathcal{B}_1 \setminus \{u\})
			 \] donc $\mathcal{B}_1 \setminus \{u\}$ engendre $E$. Donc,  $\mathcal{B}_1 \setminus \{u\}$ est une base de $E$.\\
			 Or, $\mathcal{B}_1 \setminus \{u\} \subset  \mathcal{B}_1$, donc $\mathcal{B}_1 \setminus \{u\} = \mathcal{B}_1$
	\end{itemize}
\end{prv}

\begin{defn}
	Soit $E$ un $\mathbbm{K}$-espace vectoriel de dimension finie. Le cardinal commun à toutes les bases de $E$ est appelé \underline{dimension} de $E$ est notée $\dim(E)$ ou $\dim_\mathbbm{K}(E)$\\
	C'est donc aussi le nombre de coordonnées de n'importe quel vecteur dans n'importe quelle base.
	\index{dimension (espace vectoriel)}
\end{defn}

\begin{exm}
	\begin{enumerate}
		\item $\dim_\R(\C) = 2$ et $\dim_\C(\C) = 1$ 
		\item $\dim_\mathbbm{K}(\mathbbm{K}^{n}) = n$ 
		\item $\dim_{\mathbbm{K}}(\mathcal{M}_{n,p}(\mathbbm{K})) = np$
	\end{enumerate}
\end{exm}

\begin{crlr}
	Soit $E$ un $\mathbbm{K}$-espace vectoriel de dimension finie, $\mathcal{L}$ une famille libre de $E$, $\mathcal{G}$ une famille génératrice de $E$. On note $n = \dim(E)$
	\begin{enumerate}
		\item $\#\mathcal{G} \ge n$ et $(\#\mathcal{G} = n \implies \mathcal{G} \text{ est une base de } E$)
		\item $\#\mathcal{L} \le n$ et $(\#\mathcal{L} = n \implies \mathcal{L} \text{ est une base de } E$)
	\end{enumerate}
\end{crlr}

\begin{crlr}
	$\R^{\R}$ est de dimension infinie.
	$\forall i \in \N, e_i: x \mapsto x^i$\\
	$(e_i)_{i\in\N}$ est libre dans $\R^\R$
\end{crlr}

\begin{prop}
	Soient $E$ et $F$ deux $\mathbbm{K}$-espaces vectoriels de dimension finie. Alors $E\times F$ est de dimension finie et $\dim(E\times F) = \dim(E) + \dim(F)$
\end{prop}

\begin{prv}
	Soit $(e_1,\ldots, e_n)$ une base de $E$, $(f_1, \ldots, f_p)$ une base de $F$.
	On pose \[
		\left\{\begin{array}
			{r c l}
			u_1 &=& (e_1,0_F)\\
			u_2 &=& (e_2,0_F)\\
					&\vdots&\\
			u_n &=& (e_n,0_F)\\
			u_{n+1} &=& (0_E, f_1)\\
			u_{n+2} &=& (0_E, f_2)\\
					&\vdots&\\
			u_{n+p} &=& (0_E,f_p)\\
		\end{array}\right.
	\]
	Soit $(x,y) \in E\times F$. \[
		\begin{cases}
			\exists (x_1,\ldots,x_n)\in \mathbbm{K}^n, x = \sum_{i=1}^{n} x_ie_i
			\exists (y_1,\ldots,y_n)\in \mathbbm{K}^n, x = \sum_{j=1}^{p} y_jf_j
		\end{cases}
	\] 
	\begin{align*}
		(x,y) &= \left( \sum_{i=1}^{n} x_ie_i, \sum_{i=1}^{p} y_jf_j \right)  \\
		&= \sum_{i=1}^{n} x_i (e_i + 0_F) + \sum_{j=1}^{p} y_j (0_E, f_j) \\
		&= \sum_{i=1}^{n} x_i u_i + \sum_{j=1}^{p} y_j u_{n+j} \\
	\end{align*}
	Donc, $E\times F = \Vect(u_1, \ldots, u_{n+p})$ donc $E\times F$ est de dimension finie.\\
	Soit $(\lambda_1, \ldots, \lambda_{n+p}) \in \mathbbm{K}^{n+p}$ tel que \[
		(*): \quad \sum_{k=1}^{n+p} \lambda_ku_k = 0_{E\times F} = (0_E, 0_F)
	\]
	\begin{align*}
		(*) &\iff \sum_{k=1}^{n} \lambda_k (e_k, 0_F) + \sum_{k=n+1}^{p} \lambda_k(0_E, f_{k-n}) = (0_E, 0_F)\\
				&\iff \begin{cases}
					\sum_{k=1}^{n} \lambda_k e_k = 0_E\\
					\sum_{k=n+1}^{p} \lambda_k f_{k-n} = 0_F
				\end{cases}\\
				&\iff \begin{cases}
					\forall k \in \left\llbracket 1,n \right\rrbracket, \lambda_k = 0_\mathbbm{K} \qquad&(\text{car $(e_1,\ldots,e_n)$ est libre})\\
					\forall k \in \left\llbracket n+1,n+p \right\rrbracket, \lambda_k = 0_\mathbbm{K} \qquad&(\text{car $(f_1,\ldots,f_n)$ est libre})\\
				\end{cases}
	\end{align*}
	Donc $(u_1, \ldots, u_{n+p})$ est une base de $E\times F$. Donc, $\dim(E\times F) = n + p = \dim(E) + \dim(F)$
\end{prv}

\begin{rmk}
	[Convention]
	\[\dim\big(\{0_E\}\big) = 0\]
\end{rmk}

\begin{thm}
	Soit $E$ un $\mathbbm{K}$-espace vectoriel de dimension finie, $F$ un sous-espace vectoriel de $E$. Alors, $F$ est de dimension finie et  $\dim(F) \le \dim(E)$\\
	Si $\dim(F) = \dim(E)$, alors $F = E$
\end{thm}

\begin{prv}
	On considère \[
		A = \{k \in \N \mid \text{il existe une famille libre de $F$ à $k$ éléments}\} 
	\]
	On suppose $F \neq \{0_E\}$.
	\begin{itemize}
		\item Soit $u \in F\setminus \{0_E\}$. $(u)$ est libre donc $1 \in A$ et donc $A \neq \O$
		\item Soit $\mathcal{L}$ une famille libre de $F$. Alors, $\mathcal{L}$ est une famille libre de $E$ \\
			donc $\#\mathcal{L} \le \dim(E)$\\
			Donc $A$ est majorée par $\dim(E)$ \\
			On en déduit que $A$ a un plus grand élément $p$.
		\item Soit $\mathcal{L}$ une famille libre de $F$ avec $p$ éléments.\\
			Si $\mathcal{L}$ n'engendre pas $F$, alors il existe $u\in F$ tel que $u\not\in \Vect(\mathcal{L})$ et donc $\mathcal{L} \cup \{u\}$ est une famille libre de $F$, donc $p+1 \in A$ en contradiction avec la maximalité de $p$.\\
			Donc $\mathcal{L}$ est une base de $F$ donc $F$ est de dimension finie et $\dim(F) = p \le \dim(E)$\\
	\end{itemize}

	Soit $\mathcal{B}$ une base de $F$. Alors, $\mathcal{B}$ est aussi une famille de libre de de $E$. Donc $\#\mathcal{B} \le \dim(E)$ donc $\dim(F) = \dim(E)$ \\
	Si $\dim(F) = \dim(E)$, alors $\mathcal{B}$ est une base de $E$, et donc $F = \Vect(\mathcal{B}) = E$
\end{prv}

\begin{prop}
	[Formule de Grassmann]
	Soit $E$ un $\mathbbm{K}$-espace vectoriel de dimension finie, $F$ et $G$ deux sous-espace vectoriels de $E$. Alors, \[
		\dim(F+G) = \dim(F) + \dim(G) - \dim(F\cap G)
	\] 
\end{prop}

\begin{prv}
	Soit $(e_1, \ldots, e_p)$ une base de $F\cap G$. $(e_1,\ldots,e_p)$ est une famille libre de $F$.\\
	On complète $(e_1, \ldots, e_p)$ en une base $(e_1, \ldots, e_p, u_1, \ldots, u_q)$ de $F$.\\
	De même, on complète $(e_1, \ldots, e_p)$ en une base $(e_1, \ldots, e_p, v_1, \ldots, v_r)$ de $G$.\\
	On pose  $\mathcal{B} = (e_1, \ldots, e_p, u_1, \ldots, u_q, v_1, \ldots, v_r)$. Montrons que $\mathcal{B}$ est une base de $F+G$
	\begin{itemize}
		\item Soit $u \in F+G$ \\
			On pose $u = v+w$ avec $\begin{cases}
				v\in F\\
				w \in G
			\end{cases}$.\\
			On pose $v = \sum_{i=1}^p \lambda_i e_i + \sum_{i=1}^q \mu_i u_i$ avec $(\lambda_1, \ldots, \lambda_p, \mu_1, \ldots, \lambda_q) \in \mathbbm{K}^{p+q}$\\
			On pose aussi $w = \sum_{i = 1}^p \lambda'_ie_i + \sum_{j=1}^r \nu_j v_j$ avec $(\lambda_1',\ldots,\lambda_p', \nu_1, \ldots, \nu_r) \in \mathbbm{K}^{p+r}$\\
			D'où, \[
				u = \sum_{i=1}^p (\lambda_i + \lambda'_i)e_i + \sum_{j=1}^q \mu_j u_j + \sum_{k=1}^r \nu_k v_k \in \Vect(\mathcal{B})
			\]
		\item Soient $(\lambda_1, \ldots, \lambda_p, \mu_1, \ldots, \mu_q, \nu_1, \ldots, \nu_r) \in \mathbbm{K}^{p+q+r}$.\\
			On suppose \[
				(*)\quad \sum_{i=1}^{p}\lambda_ie_i + \sum_{j=1}^q\mu_ju_j + \sum_{k=1}^r \nu_k v_k = 0_E
			\] 
			D'où, \[
				\underbrace{\sum_{i=1}^p\lambda_i e_i + \sum_{j=1}^q \mu_ju_j}_{\in F} = \underbrace{-\sum_{k=1}^r\nu_jv_k}_{\in G}
			\] 
			Donc, \[
				f = \sum_{i=1}^p \lambda_i e_i + \sum_{j=1}^q \mu_j u_j \in F\cap G
			\] Comme $(e_1, \ldots, e_p)$ est une base de $F\cap G$, $\exists ! (\lambda_1', \ldots, \lambda_p') \in \mathbbm{K}^p$ tel que \[
				f = \sum_{i=1}^p \lambda'_i e_i = \sum_{i=1}^p \lambda'_i e_i + \sum_{j=1}^q 0_\mathbbm{K}u_j
			\] Comme $(e_1, \ldots, e_p, u_1, \ldots, u_q)$ est une base de $F$, \[
				\forall k \in \left\llbracket 1, q \right\rrbracket, \mu_j = 0_\mathbbm{K}
			\] De même, \[
				\forall k \in \left\llbracket 1,r \right\rrbracket , \nu_k = 0_\mathbbm{K}
			\] On remplace dans $(*)$ pour trouver \[
				\sum_{i=1}^p \lambda_ie_i = 0_E
			\] Comme $(e_1, \ldots, e_p)$ est libre, \[
				\forall i \in \left\llbracket 1,p \right\rrbracket, \lambda_i = 0_\mathbbm{K}
			\] Donc $\mathcal{B}$ est libre.\\
			Donc, 
			\begin{align*}
				\dim(F+G) &=  p +q + r \\
				&= (p+q)+ (p+r) - p \\
				&= \dim(F) + \dim(G) - \dim(F\cap G) \\
			\end{align*}
	\end{itemize}
\end{prv}

\begin{crlr}
	Avec les hypothèse précédentes, \[
		E = F \oplus G \iff \begin{cases}
			F \cap  G = \{0_E\} \\
			\dim(E) = \dim(F) + \dim(G)
		\end{cases}
	\] 
\end{crlr}

\begin{prv}
	\begin{itemize}
		\item[``$\implies$''] On suppose $E = F \oplus G$ \\
			Comme la somme est directe, $F \cap G = \{0_E\}$ 
			\begin{align*}
				\dim(E) &= \dim(F)\\
				&= \dim(F) + \dim(G) - \dim(F\cap G)\\
				&= \dim(F) + \dim(G)\\
			\end{align*}
		\item[``$\impliedby$''] On suppose $F\cap G = \{0_E\}$ et $\dim(E) = \dim(F) + \dim(G)$.\\
			On sait déjà que $F+G = F \oplus G$\\
			 \begin{align*}
				\dim(F+G) = \dim(F) + \dim(G) - \dim(F \cap G) = \dim(E)
			\end{align*}
			Donc $F + G = E$
	\end{itemize}
\end{prv}

\begin{prop}
	Soit $F$ un $\mathbbm{K}$-espace vectoriel de dimension finie $n$. Soit $\mathcal{B} = (e_1, \ldots, e_n)$ une base de $F$. L'application
	\begin{align*}
		f: \mathbbm{K}^n &\longrightarrow F \\
		(\lambda_1, \ldots, \lambda_n) &\longmapsto \sum_{i=1}^n \lambda_i e_i
	\end{align*} est bijective.\\
	Si $\mathbbm{K}$ est infini, $\mathbbm{K}^n$ aussi et donc $F$ aussi.\\
	Si $\#\mathbbm{K} = p \in \N_*$,
	\begin{align*}
		\#&\mathbbm{K}^n = p^n\\
		&\vrt=\\
		\#&F
	\end{align*}
\end{prop}


		\part{Dérivation}

\underline{Motivation}:

{
\begin{wrapfigure}{l}{3cm}
	\centering
	\begin{asy}
		import three;

		size(3cm);
		settings.render=0;
		settings.prc=false;
		currentprojection = obliqueZ;

		draw(unitbox);
		draw(shift(1.1Z + 0.05X) * (O -- X), Arrows3(TeXHead2));
		draw(shift(1.1Z + 0.05Y) * (O -- Y), Arrows3(TeXHead2));
		draw(shift(1.1X + 0.05Z) * (O -- Z), Arrows3(TeXHead2));

		label("$x$", (X/2) + (1.1Z + 0.05X), align=S);
		label("$y$", (Y/2) + (1.1Z + 0.05Y), align=W);
		label("$z$", (Z/2) + X, align=SE);
	\end{asy}
\end{wrapfigure}

\begin{align*}
	&S(x,y,z) = 2(xy + xz + yz)\\
	&V(x,y,z) = xyz
\end{align*}

On cherche à minimiser $S$ avec la contrainte $V = 1$.

Soit $f : \begin{array}{rcl}
	\left( \R_*^+ \right)^2 &\longrightarrow& \R \\
	(x,y) &\longmapsto& S\left( x,y,\frac{1}{xy} \right) = 2\left( xy + \frac{1}{y} + \frac{1}{x} \right).
\end{array}$

On cherche $(a,b) \in \left( \R^+_* \right)^2$ tel que \[
	\forall (x,y) \in (\R^+_*), f(x,y) \ge f(a,b).
\]
}

\begin{defn}
	Soit $f: U \to \R$ où $U$ est un ouvert de $\R^2$. Soit $(a,b) \in U$.
	\vspace{2mm}

	Si $\lim_{x \to a} \frac{f(x,b) - f(a,b)}{x - a} \in \R$, alors on dit que $f$ a une dérivée partielle suivant $x$ en $(a,b)$ et cette limite est notée \[
		\partial f_1(a,b) = \frac{\partial f}{\partial x}(a,b).
	\]

	Si $\lim_{y \to b} \frac{f(a,y) - f(a,b)}{y - b} \in \R$, alors on dit que $f$ a une dérivée partielle suivant $y$ et la limite est notée \[
		\partial f_2(a,b) = \frac{\partial f}{\partial y}(a,b).
	\]
\end{defn}

\begin{exm}
	\begin{enumerate}
		\item $f: (x,y) \mapsto xy + x - y$.

			\begin{align*}
				&\frac{\partial f}{\partial x} : (x,y) \mapsto y + 1,\\
				&\frac{\partial f}{\partial y} : (x,y) \mapsto x - 1.
			\end{align*}

		\item $f: (x,y) \mapsto xy + \frac{1}{y}+ \frac{1}{x}$.

			\begin{align*}
				&\frac{\partial f}{\partial x}: (x,y) \mapsto y - \frac{1}{x^2},\\
				&\frac{\partial f}{\partial y}: (x,y) \mapsto x - \frac{1}{y^2}.
			\end{align*}

		\item Trouver $f$ telle que $\begin{cases}
				(1): \qquad \frac{\partial f}{\partial x}=y,\\[2mm]
				(2): \qquad \frac{\partial f}{\partial y} = x.
			\end{cases}$

			D'après $(1)$ : \[
				\forall (x,y), \exists C(y) \in \R, f(x,y) = xy + C(y)
			\] et donc \[
				\frac{\partial f}{\partial y}(x,y) = x + C'(y)
			\] donc $C'(y) = 0$ et donc $C$ est constante.

		\item Trouver $f$ telle que $\begin{cases}
			\frac{\partial f}{\partial x} = -y,\\[2mm]
			\frac{\partial f}{ƒ\partial y} = x.
		\end{cases}$

		Ce n'est pas possible !
	\end{enumerate}
\end{exm}

\begin{defn}~\\
	\begin{minipage}{\linewidth}
		\begin{wrapfigure}{r}{4cm}
			\centering
			\vspace{-5mm}
			\begin{asy}
				import three;
				import graph3;
				size(4cm);

				settings.render = 0;
				settings.prc = false;
				currentprojection = obliqueX;

				draw(O -- X, Arrow3(TeXHead2));
				draw(O -- Y, Arrow3(TeXHead2));
				draw(O -- Z, Arrow3(TeXHead2));

				triple f(real x, real y, real z = 0) { return (x,y,cos(x - 0.5) * cos(y - 0.5)/1.2 + 0.15); }

				real inc = 1 / 5;

				for(real x = 0; x <= 1; x += inc) {
					draw(graph(
						new real(real t) { return x; }, // x
						new real(real y) { return y; }, // y
						new real(real y) { return f(x,y).z; }, // z
						0, 1
					), gray);
				}

				for(real y = 0; y <= 1; y += inc) {
					draw(graph(
						new real(real x) { return x; }, // x
						new real(real t) { return y; }, // y
						new real(real x) { return f(x,y).z; }, // z
						0, 1
					), gray);
				}

				path3 path1 = (0.8, 0.2, 0) .. (0.5, 0.5, 0) .. (0.3, 0.7, 0);
				path3 path2 = f(0.8, 0.2, 0) .. f(0.5, 0.5, 0) .. f(0.3, 0.7, 0);
				path3 d = (0.2, 0.3, 0) .. (0.3, 0.4, 0) .. (0.2, 0.7, 0) .. (0.8, 0.9, 0) .. (0.6, 0.2, 0) .. cycle;

				draw(path1, red, Arrow3(TeXHead2));
				draw(path2, red, Arrow3(TeXHead2, position=0.8));

				dot((0.5, 0.5, 0));
				dot(f(0.5, 0.5, 0));
				draw((0.5, 0.5, 0) -- f(0.5, 0.5, 0), dashed);
				draw(d);

				label("$w$", (0.3, 0.7, 0), red, align=SE);
				label("$U$", (0.8, 0.9, 0), align=SE);
			\end{asy}
		\end{wrapfigure}

		Soit $f: U \to \R$ où $U$ est un ouvert. Soit $(a,b) \in U$. Soit $w = (w_1, w_2) \in \R^2$.

		Si 
		\[
			\lim_{t\to 0} \frac{f(a + tw_1, b + tw_2) - f(a,b)}{t}
		\] existe et est réelle, alors on dit que $f$ a une dérivée dans la direction de $w$ et la limite est notée \[
			\mathrm{d}f(w)\,(a,b) = D_w(f)\,(a,b).
		\]
	\end{minipage}
\end{defn}

\begin{exm}
	\begin{align*}
		f: \left( \R_*^+ \right)^2 &\longrightarrow \R \\
		(x,y) &\longmapsto xy+\frac{1}{x}+\frac{1}{y}.
	\end{align*}

	On pose $(a,b) = (1,2)$, $w = (w_1, w_2) = (1,1)$.
	\begin{align*}
		\frac{f(1+t, 2+t) - f(1,2)}{t} &= \frac{1}{t} \left( (1+t)(2+t) + \frac{1}{1+t} + \frac{1}{2+t} - 3 - \frac{1}{2} \right) \\
		&= \frac{1}{t}\left(\cancel 2 + 3t + \po(t) + \cancel 1 - t + \po(t) + \frac{1}{2}\left( \cancel 1 - \frac{t}{2} + \po(t) \right) - \cancel3 - \cancel{\frac{1}{2}} \right) \\
		&= \frac{1}{t} \left( \frac{7}{4} t + \po(t) \right)  \\
		&= \frac{7}{4} + \po(1) \tendsto{t \to 0} \frac{7}{4}. \\
	\end{align*}

	Donc, \[
		\mathrm{d}f(1,1)\,(1,2) = \frac{7}{4}.
	\]
\end{exm}

\begin{rmk}~\\
	\begin{figure}[H]
		\centering
		\begin{asy}
			import solids;
			import graph;
			size(5cm);

			settings.render = 0;
			settings.prc = false;

			path3 par = graph(
				new real(real x) { return x; },
				new real(real x) { return 0; },
				new real(real x) { return x^2; },
				0,3);
			revolution r = revolution(par, axis=Z);

			path3 par2 = graph(
				new real(real x) { return x; },
				new real(real x) { return 0; },
				new real(real x) { return x^2; },
				-3,3);

			draw(r,1,longitudinalpen=nullpen);
			draw(r.silhouette());

			draw((-4, 0, -1) -- (-4, 0, 10) -- (4, 0, 10) -- (4, 0, -1) -- cycle, red);
			draw(par2, deepred);

			draw((4,4.5) -- (7, 4.5), black+0.5mm, Arrow(TeXHead));

			path par2d = graph(new real(real x) { return x^2; }, -3, 3);
			draw(shift((11, 0)) * par2d, deepred);

			dot(O);
			dot((11, 0));
		\end{asy}
	\end{figure}
\end{rmk}


%todo ajouter théorème-définition
\begin{thm}
	Soit $f : U \to \R$, $(a,b) \in U$. On suppose que $\frac{\partial f}{\partial x}$ et $\frac{\partial f}{\partial y}$ existent en $(a,b)$ et sont {\bfseries continues} en $(a,b)$. Alors,
	\begin{align*}
		&\forall (h, k) \in \R^2 \text{ tel que } (a +h, b + k) \in U,\\
		&f(a+ h, b + k) = f(a,b) + h \frac{\partial f}{\partial x}(a,b) + k \frac{\partial f}{\partial y}(a,b) + \po_{(h,k)\to (0,0)}\big(\|(h,k)\|\big).
	\end{align*}

	On dit que $f$ est de classe $\mathcal{C}^1$ si $\frac{\partial f}{\partial x}$ et $\frac{\partial f}{\partial y}$ existent et sont continues.

	\qed
\end{thm}

\begin{rmk}
	En physique, cette formule correspond à : \[
		\mathrm{d}f = \frac{\partial f}{\partial x}\mathrm{d}x + \frac{\partial f}{\partial y} \mathrm{d}y.
	\] En effet :
	\begin{align*}
		\mathrm{d}f &= f(x+ \mathrm{d}x, y + \mathrm{d}y) - f(x,y) \\
		&= \frac{\partial f}{\partial x} \mathrm{d}x + \frac{\partial f}{\partial y} \mathrm{d}y.
	\end{align*}
\end{rmk}

\begin{prop}
	Soit $f: U \to \R$ de classe $\mathcal{C}^1$ en $(a,b) \in U$. Alors, \[
		\forall w = (w_1, w_2) \in \R^2, \mathrm{d}f(w)\,(a,b) = w_1 \frac{\partial f}{\partial x}(a,b) + w_2 \frac{\partial f}{\partial y}(a,b).
	\]
\end{prop}

\begin{prv}
	Soit $w = (w_1, w_2) \in \R^2$. Soit $t \in \R^*$.
	\begin{align*}
		\frac{1}{t}\big(f(a + tw_1, b + tw_2) - f(a,b)\big)
		&= \frac{1}{t} \left( tw_1 \frac{\partial f}{\partial x}(a,b) + tw_2 \frac{\partial f}{\partial y}(a,b) + \po_{t \to 0}\big(\|tw\|\big) \right) \\
		&= w_1 \frac{\partial f}{\partial x}(a,b) + w_2 \frac{\partial f}{\partial y}(a,b) + \po_{t\to 0}(1) \\
		&\tendsto{t\to 0} w_1 \frac{\partial f}{\partial x}(a,b) + w_2\frac{\partial f}{\partial y}(a,b).
	\end{align*}
\end{prv}


\begin{defn}
	Avec les hypothèses précédentes, en posant \[
		\nabla f(a,b) = \left( \frac{\partial f}{\partial x}(a,b), \frac{\partial f}{\partial y}(a,b) \right) 
	\]on obtient \[
		\mathrm{d}f(w)\,(a,b) = \left<w  \mid \nabla f(a,b) \right>
	\] où $\left<\cdot|\cdot \right>$ est le produit scalaire.

	Le vecteur $\nabla f(a,b)$ est appelé \underline{gradient de $f$ en $(a,b)$}.

	Le développement limité à l'ordre 1 de $f$ devient \[
		f\big((a,b)+w\big) = f(a,b) + \left<w \mid \nabla f(a,b) \right> + \po_{w\to 0}(\|w\|)
	\]
\end{defn}

\begin{prop}
	Soit $f : U \to \R$ de classe $\mathcal{C}^1$.

	\begin{figure}[H]
    \centering
    \incfig{gradient}
	\end{figure}

	$\nabla f$ est orthogonal au lignes de niveaux de $f$, son orientation va dans le sens d'une augmentation de $f$.
\end{prop}

\begin{prv}
	Soit $\gamma : I \to U$ une courbe de niveau : \[
		\forall t \in I, f\big(\gamma(t)\big) = \text{cste}.
	\] D'après le lemme suivant : \[
		\forall t \in I, 0 = (f \circ \gamma)'(t) = \mathrm{d}f\big(\gamma'(t)\big)\big(\gamma(t)\big) = \left<\gamma'(t)  \mid \nabla f\big(\gamma(t)\big) \right>
	\] Donc $\nabla f\big(\gamma(t)\big)$ est orthogonal à $\gamma'(t)$.

	Pour tout $t \in I$, on pose $w(t) = t\, \nabla f\big(\gamma(t)\big)$. Donc \[
		f\big(\gamma(t) + w(t)\big) = f\big(\gamma(t)\big) + t \|\nabla f(\gamma(t))\|^2 + \po_{t \to 0}(t)
	\] Pour $t$ assez petit, $f\big(\gamma(t) + w(t)\big) - f\big(\gamma(t)\big)$ est du même signe que $t$.
\end{prv}

\begin{rmk}
	\begin{align*}
		V: \R^3 &\longrightarrow \R \\
		(x,y,z) &\longmapsto -mgz
	\end{align*}
	l'énerge potentielle de pesenteur

	On a donc \[
		\nabla V(x,y,z) = \left( \frac{\partial V}{\partial x}, \frac{\partial V}{\partial y}, \frac{\partial V}{\partial z} \right) = (0, 0, -mg) = \vec{P}.
	\]
\end{rmk}

\begin{lem}
	Soit $f : U \to \R$ de classe $\mathcal{C}^1$, $\gamma : \begin{array}{rcl}
		I &\longrightarrow& U \\
		t &\longmapsto& \big(x(t), y(t)\big)
	\end{array}$ où $x$ et $y$ sont dérivables.

	On pose \[
		\forall t \in I, \gamma'(t) = \big(x'(t), y'(t)\big).
	\] Alors $f \circ \gamma : I \to \R$ est dérivable et
	\begin{align*}
		\forall t \in I, (f \circ \gamma)'(t) &= \mathrm{d}f\big(\gamma'(t)\big) \big(\gamma(t)\big)\\
		&= \left<\gamma'(t)  \mid \nabla f\big(\gamma(t)\big)  \right> \\
		&= x'(t) \frac{\partial f}{\partial x}\big(x(t), y(t)\big) + y'(t) \frac{\partial f}{\partial y}\big(x(t),y(t)\big). \\
	\end{align*}
\end{lem}

\begin{prv}
	On fixe $t \in I$.

	\begin{align*}
		\forall h \neq 0, \frac{f \circ \gamma(t + h) - f \circ \gamma(t)}{h}
		&= \frac{1}{h}\big(f(\gamma(t)) + h\gamma'(t) + \po_{h\to 0}(h) - f(\gamma(t))\big) \\
		&= \frac{1}{h}\bigg(\cancel{f(\gamma(t))} + \left<h\gamma'(t) \mid \nabla f(\gamma(t)) \right> + \po_{h\to 0}(\|h\gamma'(t)\|) - \cancel{f(\gamma(t))}\bigg)\\
		&= \left<\gamma'(t) \mid \nabla f(\gamma(t)) \right> + \po_{h\to 0}(1) \\
		&\tendsto{h\to 0} \left<\gamma'(t)  \mid \nabla f(\gamma(t)) \right>
	\end{align*}
\end{prv}

\begin{defn}
	Soit $f : U \to \R$ de classe $\mathcal{C}^1$ et $(a,b) \in U$. On dit que $(a,b)$ est un \underline{point critique} de $f$ si $\nabla f(a,b) = 0$ i.e. $\frac{\partial f}{\partial x}(a,b) = \frac{\partial f}{\partial y}(a,b) = 0$.

	Dans ce cas, $f(a,b)$ est appelé \underline{valeur critique} de $f$.
\end{defn}

\begin{prop}~\\
	\begin{minipage}{\linewidth}
		\begin{wrapfigure}{r}{3cm}
			\centering
			\vspace{-1cm}
			\begin{asy}
				import solids;
				import graph;
				size(3cm);

				settings.render = 0;
				settings.prc = false;

				path3 par = graph(
					new real(real x) { return x; },
					new real(real x) { return 0; },
					new real(real x) { return -x^2; },
					0,3);
				revolution r = revolution(par, axis=Z);

				draw(r,1,longitudinalpen=nullpen);
				draw(r.silhouette());

				dot("$(a,b)$", O, red, align=N);
				real s = sqrt(2.5);
				path3 g=(s,0,-2.5)..(0,s,-2.5)..(-s,0,-2.5)..(0,-s,-2.5)..cycle;
				draw(g, deepcyan);
			\end{asy}
		\end{wrapfigure}
		Soit $f: U \to \R$ de classe $\mathcal{C}^1$ et $(a,b) \in U$ tel que \[
			\exists r > 0, \forall (x,y) \in B_{(a,b)}(r), f(x,y) \le f(a,b)
		\] Alors $\nabla f(a,b) = (0,0)$.
	\end{minipage}
\end{prop}

\begin{prv}
	Soit $g: x \mapsto f(x,b)$. $g(a)$ est un maximum local de $g$ donc $g'(a) = 0$.

	Or, $g'(a) = \frac{\partial f}{\partial x}(a,b)$

	donc $\frac{\partial f}{\partial x}(a,b) = 0$.

	Soit $h : y \mapsto f(a,y)$. On a de même $h'(b) = 0$.

	Or, $h'(b) = \frac{\partial f}{\partial y}(a,b)$.

	Donc, $\nabla f(a,b) = (0,0)$.
\end{prv}

\begin{rmk}
	Un minimum local est aussi une valeur critique.
\end{rmk}

\begin{figure}[H]
	\centering
	\begin{subfigure}{3cm}
		\centering
		\begin{asy}
			import solids;
			import graph;
			size(3cm);

			settings.render = 0;
			settings.prc = false;

			path3 par = graph(
				new real(real x) { return x; },
				new real(real x) { return 0; },
				new real(real x) { return -x^2; },
				0,3);
			revolution r = revolution(par, axis=Z);

			draw(r,1,longitudinalpen=nullpen);
			draw(r.silhouette());

			dot(O, red);
		\end{asy}
		\caption{Maximum local}
	\end{subfigure}
	\begin{subfigure}{3cm}
		\centering
		\begin{asy}
			import solids;
			import graph;
			size(3cm);

			settings.render = 0;
			settings.prc = false;

			path3 par = graph(
				new real(real x) { return x; },
				new real(real x) { return 0; },
				new real(real x) { return x^2; },
				0,3);
			revolution r = revolution(par, axis=Z);

			draw(r,1,longitudinalpen=nullpen);
			draw(r.silhouette());

			dot(O, red);
		\end{asy}
		\caption{Minimum local}
	\end{subfigure}
	\begin{subfigure}{3cm}
		\centering
		\begin{asy}
			import solids;
			import graph;
			size(3cm);

			settings.render = 0;
			settings.prc = false;
			currentprojection = obliqueZ;

			draw(graph(
				new real(real x) { return x; },
				new real(real x) { return -x^2 / 3; },
				new real(real x) { return 3; },
				-3, 3
			));

			draw(graph(
				new real(real x) { return x; },
				new real(real x) { return -x^2 / 3; },
				new real(real x) { return -3; },
				-3, 3
			));

			draw(graph(
				new real(real x) { return x; },
				new real(real x) { return -x^2 / 3 - 1; },
				new real(real x) { return 0; },
				-3, 3
			));

			draw(graph(
				new real(real x) { return 0; },
				new real(real x) { return x^2 / 9 - 1; },
				new real(real x) { return x; },
				-3, 3
			));

			draw(graph(
				new real(real x) { return -3; },
				new real(real x) { return x^2 / 9 - 4; },
				new real(real x) { return x; },
				-3, 3
			));

			draw(graph(
				new real(real x) { return 3; },
				new real(real x) { return x^2 / 9 - 4; },
				new real(real x) { return x; },
				-3, 3
			));

			dot((0,-1,0), red);
		\end{asy}
		\caption{Point de selle / Point col}
	\end{subfigure}
\end{figure}

\begin{exm}
	On revient à l'exemple donné en introduction : 
	\begin{align*}
		f: \left( \R^*_+ \right)^2 &\longrightarrow \R \\
		(x,y) &\longmapsto 2\left( xy + \frac{1}{x} + \frac{1}{y} \right).
	\end{align*}

	$\left( \R^+_* \right)^2$ est un ouvert de $\R^2$. Soit $(x,y) \in \left( \R^+_* \right)^2$.
	
	On a \[
		\begin{cases}
			\frac{\partial f}{\partial x}(x,y) = 2\left( y - \frac{1}{x^2} \right),\\
			\frac{\partial f}{\partial y}(x,y) = 2\left( x - \frac{1}{y^2} \right).
		\end{cases}
	\]

	\begin{align*}
		&\frac{\partial f}{\partial x}(x,y) = \frac{\partial f}{\partial y}(x,y) = 0\\
		\iff& \begin{cases}
			y = \frac{1}{x^2}\\
			x = \frac{1}{y^2}
		\end{cases}\\
		\iff& \begin{cases}
			y = \frac{1}{x^2}\\
			x = x^4
		\end{cases}\\
		\iff& \begin{cases}
			x = 1\\
			y = 1
		\end{cases}
	\end{align*}

	On vérivie que $f$ présente en effet un minium local en $(1,1)$. \[
		f(1,1) = 6
	\] On fixe $y \in \R^+_*$ et \[
		g : x \mapsto 2\left( xy + \frac{1}{x} + \frac{1}{y} \right).
	\] Donc \[
		\forall x \in \R^+_*, g'(x) = 2\left( y - \frac{1}{x^2} \right).
	\]
	\begin{center}
		\begin{tikzpicture}
			\tkzTabInit{$x$/1,$g'(x)$/1,$g$/2.3}{$0$, $\frac{1}{\sqrt{y}}$, $+\infty$}
			\tkzTabLine{,-,z,+,}
			\tkzTabVar{+/{}, -/$2\left( 2\sqrt{y} +\frac{1}{y} \right)$, +/{}}
		\end{tikzpicture}
	\end{center}
	
	Ainsi, \[
		\forall x \in \R^+_*, \forall y \in \R^+_*, f(x,y) \ge 2\left( 2\sqrt{y} + \frac{1}{y} \right)
	\] Soit $h : y \mapsto 2\sqrt{y} + \frac{1}{y}$. On a \[
		\forall y > 0, h'(y) = \frac{1}{\sqrt{y}} - \frac{1}{y^2} = \frac{y\sqrt{y} - 1}{y^2} = \frac{y^{\frac{3}{2}} - 1}{y^2}
	\]

	\begin{center}
		\begin{tikzpicture}
			\tkzTabInit{$y$/0.7,$h'(y)$/0.7,$h$/1.4}{$0$, $1$, $+\infty$}
			\tkzTabLine{,-,z,+,}
			\tkzTabVar{+/{}, -/$3$, +/{}}
		\end{tikzpicture}
	\end{center}

	Donc, \[
		\forall x,y > 0, f(x,y) \ge 2\times 3 = 6 = f(1,1).
	\]
\end{exm}

\begin{prop}
	[règle de la chaîne]

	Soit $f : \begin{array}{rcl}
		U &\longrightarrow& \R^2 \\
		(x,y) &\longmapsto& f(x,y)
	\end{array}$ de classe $\mathcal{C}^1$ et $U, V$ deux ouverts de $\R^2$.

	Soit $\varphi : \begin{array}{rcl}
		V &\longrightarrow& U \\
		(u,v) &\longmapsto& \varphi(u,v) = \big(x(u,v), y(u,v)\big)
	\end{array}$.

	On suppose que $x$ et $y$ sont de classe $\mathcal{C}^1$ sur $V$.

	Alors,  $f \circ \varphi : \begin{array}{rcl}
		V &\longrightarrow& \R \\
		(u,v) &\longmapsto& f\big(\varphi(u,v)\big)
	\end{array}$ est de classe $\mathcal{C}^1$ et
	\begin{align*}
		\forall (u_0, v_0) \in V, \frac{\partial (f \circ \varphi)}{\partial u}(u_0, v_0)
		&= \frac{\partial f}{\partial x}\big(\varphi(u_0, v_0)\big) \times \frac{\partial x}{\partial u}(u_0, v_0)\\
		&+ \frac{\partial f}{\partial y}\big(\varphi(u_0,v_0)\big) \frac{\partial y}{\partial u}(u_0,v_0)
	\end{align*}
	\begin{align*}
		\forall (u_0, v_0) \in V, \frac{\partial (f \circ \varphi)}{\partial v}(u_0, v_0)
		&= \frac{\partial f}{\partial x}\big(\varphi(u_0, v_0)\big) \times \frac{\partial x}{\partial v}(u_0, v_0)\\
		&+ \frac{\partial f}{\partial y}\big(\varphi(u_0,v_0)\big) \frac{\partial y}{\partial v}(u_0,v_0)
	\end{align*}
\end{prop}

\begin{exm}
	[changement de coordonnées polaires]
	On pose \begin{align*}
		\varphi: \R^+_* \times ]0,2\pi[ &\longrightarrow \R^2\setminus \left( R^+_* \times \{0\} \right) \\
		(r, \theta) &\longmapsto (r \cos \theta, r \sin\theta),
	\end{align*}
	\begin{align*}
		f: \R^2\setminus \left( R^+_* \times \{0\} \right) &\longrightarrow \R \\
		(x,y) &\longmapsto f(x,y),
	\end{align*}
	\begin{align*}
		g: \overbrace{\R^+_* \times ]0, 2\pi[}^{=V} &\longrightarrow \R \\
		(r, \theta) &\longmapsto f(r\cos\theta, r\sin\theta).
	\end{align*}

	\begin{align*}
		\forall (r_0,\theta_0) \in V,&\\[5mm]
		\frac{\partial g}{\partial r}(r_0, \theta_0) &= \frac{\partial f}{\partial x}(r_0\cos\theta_0, r_0\sin\theta_0)\cos\theta_0\\
		&+ \frac{\partial f}{\partial y}(r_0 \cos\theta_0, r_0\sin\theta_0)\sin\theta_0\\
		&= 2r_0\cos^2\theta_0 + 2r_0\sin^2(\theta_0) \\
		&= 2r_0 \\[5mm]
		\frac{\partial g}{\partial \theta}(r_0, \theta_0) &= \frac{\partial f}{\partial x}(r_0\cos\theta_0, r_0\sin\theta_0)r_0\sin\theta_0\\
		&+ \frac{\partial f}{\partial y}(r_0 \cos\theta_0, r_0\sin\theta_0)r_0\cos\theta_0\\
		&= -2{r_0}^2\cos(\theta_0)\sin(\theta_0) + 2{r_0}^2 \sin(\theta_0)\cos(\theta_0)\\
		&= 0 \\
	\end{align*}

	Donc, \[
		g(r, \theta) = r^2.
	\]
\end{exm}

\begin{exm}
	Résoudre \[
		\begin{cases}
			\frac{\partial f}{\partial x} = \frac{x}{x^2+y^2},\\
			\frac{\partial f}{\partial y} = \frac{y}{x^2+y^2}.\\
		\end{cases}
	\]

	On pose $g: (r, \theta) \mapsto f(r \cos\theta, r \sin\theta)$.

	\begin{align*}
		&\frac{\partial g}{\partial r} = \frac{1}{r}\cos^2\theta + \frac{1}{r}\sin^2\theta = \frac{1}{r},\\
		&\frac{\partial g}{\partial \theta} = -\cos(\theta) \sin(\theta) + \sin(\theta)\cos(\theta) = 0.
	\end{align*}

	Donc, \[
		\exists C \in \R, g: (r, \theta) \mapsto \ln r + C
	\] d'où,
	\begin{align*}
		\forall (x,y) \in \R^2 \setminus \{(0,0)\}, f(x,y) &= \ln\left(\sqrt{x^2 + y^2} \right)  + C\\
		&= \frac{1}{2}\ln(x^2 + y^2) + C. \\
	\end{align*}
\end{exm}

\begin{rmk}
	Soit $\mathcal{B} = (e_1, e_2)$ la base canonique de $\R^2$, $f: U \to \R$ de classe $\mathcal{C}^1$ avec $U$ un ouvert de $\R^2$.

	Soit $(x,y) \in U$.

	\begin{align*}
		\Mat_{\mathcal{B}}\big(\nabla f(x,y)\big) = \begin{pmatrix}
			\frac{\partial f}{\partial x}(x,y)\\[2mm]
			\frac{\partial f}{\partial y}(x,y)
		\end{pmatrix}
	\end{align*}

	Soit  \begin{align*}
		\varphi: V &\longrightarrow U \\
		(u,v) &\longmapsto \big(x(u,v), y(u,v)\big) 
	\end{align*} avec $x,y$ de classe $\mathcal{C}^1$. Soit $g = f \circ \varphi$.
	\begin{align*}
		\Mat_{\mathcal{B}}\big(\nabla g(u,v)\big)
		&= \begin{pmatrix}
			\frac{\partial g}{\partial u}(u,v) \\[2mm]
			\frac{\partial g}{\partial v}(u,v)
		\end{pmatrix} \\
		&= \begin{pmatrix}
			\frac{\partial x}{\partial u}(u,v) \frac{\partial f}{\partial x}(x,y)
			+ \frac{\partial y}{\partial u}(u,v)\frac{\partial f}{\partial y}(x,y)\\[3mm]
			\frac{\partial x}{\partial v}(u,v) \frac{\partial f}{\partial x}(x,y)
			+ \frac{\partial y}{\partial v}(u,v) \frac{\partial f}{\partial y}(x,y)
		\end{pmatrix}  \\
		&= \underbrace{\begin{pmatrix}
				\frac{\partial x}{\partial u}(u,v)& \frac{\partial y}{\partial u}(u,v)\\[3mm]
				\frac{\partial x}{\partial v}(u,v)& \frac{\partial y}{\partial v}(u,v)
		\end{pmatrix}}_{J(u,v)} \begin{pmatrix}
			\frac{\partial f}{\partial x}(x,y)\\[3mm]
			\frac{\partial f}{\partial y}(x,y)
		\end{pmatrix} \\
		&= J(u,v) \Mat_{\mathcal{B}}\big(\nabla f(x,y)\big) \\
	\end{align*}
	où $J(u,v) = 
	\begin{pNiceArray}{c:c}
		\Mat_{\mathcal{B}}\big(\nabla x(u,v)\big) & \Mat_{\mathcal{B}}\big(\nabla y(u,v)\big)
	\end{pNiceArray}$.

	On dit que $J(u,v)$ est \underline{la jacobienne} de $\varphi$ en $(u,v)$.
	L'application linéaire canoniquement associée à $J(u,v)$ est la \underline{différentielle de $\varphi$} en $(u,v)$ noté $\mathrm{d}\varphi(u,v)$.

	On a $\mathrm{d}\varphi(u,v) \in \mathcal{L}(R^2)$ et $\Mat_{\mathcal{B}}\big(\mathrm{d}\varphi(u,v)\big) = J(u,v)$.

	Par exemple, la jacobienne du changement de coordonnées polaires est \[
		J = \begin{pmatrix}
			\frac{\partial x}{\partial r} & \frac{\partial y}{\partial r}\\[3mm]
			\frac{\partial x}{\partial \theta} & \frac{\partial y}{\partial \theta}
		\end{pmatrix}
		= \begin{pmatrix}
			\cos\theta&\sin\theta\\
			-r\sin\theta&r\cos\theta
		\end{pmatrix}.
	\]
	$\underbrace{\det(J)}_{\text{le jacobien}} = r\cos^2\theta + r\sin^2\theta = r$

	Dans une intégrale double, si $(x,y) = \varphi(u,v)$, alors $\mathrm{d}x\mathrm{d}y = \det(J)\mathrm{d}u\mathrm{d}v$.

	Ici, \[
		\mathrm{d}x\ \mathrm{d}y = r\ \mathrm{d}r\ \mathrm{d}\theta.
	\]
\end{rmk}

\begin{prv}
	On pose $(x_0, y_0) = \varphi(u_0, v_0)$. Pour tout $(h,k) \in \R^2$ tels que $(u_0 + h, v_0 + k) \in V$, en posant $g = f  \circ \varphi$.

	\begin{align*}
		g(u_0 + h, v_0 + h) &= f\big(x(u_0 + h, v_0 + k), y(u_0 + h, v_0 + k)\big) \\
		&= f\left(
			x(u_0,v_0) + h \frac{\partial x}{\partial u}(u_0,v_0) + k \frac{\partial x}{\partial v}(u_0, v_0) + \po\big(\|(h,k)\|\big), \right.\\
		&\phantom{ = f\bigg(\bigg.}\left. y(u_0, v_0) + h \frac{\partial y}{\partial u}(u_0, v_0) + k \frac{\partial y}{\partial v}(u_0, v_0) + \po\big(\|(h,k)\|\big)
		\right)  \\
		&= f(x_0,y_0) \\
		&~+ \left( h \frac{\partial x}{\partial u}(u_0,v_0) + k \frac{\partial x}{\partial v}(u_0, v_0) + \po(\|(h,k)\|) \right) \frac{\partial f}{\partial x}(x_0,y_0)\\
		&~+ \left( h \frac{\partial y}{\partial u}(u_0, v_0) + k\frac{\partial y}{\partial v}(u_0, v_0) + \po(\|(h,k)\|) \right) \frac{\partial f}{\partial y}(x_0, y_0)\\
		&~+ \po(\|(h,k)\|)\\
		&= f(x_0, y_0) \\
		&~+ h \left( \frac{\partial x}{\partial u}(u_0, v_0) \frac{\partial f}{\partial x}(x_0, y_0) + \frac{\partial y}{\partial u}(u_0, v_0) \frac{\partial f}{\partial y}(x_0, y_0) \right)  \\
		&~+ k\left( \frac{\partial x}{\partial v}(u_0, v_0) \frac{\partial f}{\partial x}(x_0, y_0) + \frac{\partial y}{\partial v}(u_0, v_0) \frac{\partial f}{\partial y}(x_0, y_0) \right) 
		&~+ \po(\|(h,k)\|)\\
		&= g(u_0, v_0) + h \frac{\partial g}{\partial u}(u_0, v_0) + k \frac{\partial g}{\partial v}(u_0, v_0) + \po(\|(h,k)\|) \\
	\end{align*}

	Par identification,
	\[
		\frac{\partial g}{\partial u}(u_0, v_0) = \frac{\partial x}{\partial u}(u_0, v_0) \frac{\partial f}{\partial x}(x_0, y_0) + \frac{\partial y}{\partial u}(u_0, v_0) \frac{\partial f}{\partial y}(x_0,y_0)
	\] et \[
		\frac{\partial g}{\partial v}(u_0, v_0) = \frac{\partial x}{\partial v}(u_0,v_0) \frac{\partial f}{\partial x}(x_0, y_0) + \frac{\partial y}{\partial v}(u_0, v_0) \frac{\partial f}{\partial y}(x_0, y_0).
	\] 
\end{prv}

\begin{exm}
	[Régression linéaire]~\\
	\begin{figure}[H]
		\centering
		\begin{asy}
			import graph;
			axes(EndArrow);
			size(5cm);

			real f(real x) { return x + 0.5; }

			real k = 35 / (7 - 0.5);

			for(int i = 0; i < 35; ++i) {
				real mag = exp(sin(100 * pi/exp(1) * i)) * 0.8 + exp(cos(i*40)/3);
				real eps = mag * cos(10 * exp(1)/pi * i) / 3;
				dot((i/k,f(i/k) + eps));
			}

			draw(graph(f, -1, 7), orange);
		\end{asy}
	\end{figure}
	\[
		y = a x + b
	\] 
	On fixe $(a,b) \in \R^2$. \[
		\varepsilon(a,b) = \sum_{i=1}^n\big( y_i - (ax_i + b) \big)^2
	\] l'erreur totale.

	On veut minimiser $\varepsilon(a,b)$. On a 
	\[
		\forall (a,b) \in \R^2,
		\begin{cases}
			\frac{\partial \varepsilon}{\partial a}(a,b) = -2\sum_{i=1}^{n}(y_i - ax_i - b)x_i,\\
			\frac{\partial \varepsilon}{\partial b}(a,b) = -2\sum_{i=1}^{n}(y_i - ax_i - b).
		\end{cases}
	\]

	Donc,
	\begin{align*}
		(a,b) \text{ point critique de } \varepsilon \iff& \begin{cases}
			a \sum_{i=1}^n {x_i}^2 + b\sum_{i=1}^{n}x_i = \sum_{i=1}^{n} y_ix_i\\
			a\sum_{i=1}^{n}x_i + nb = \sum_{i=1}^ny_i
		\end{cases}\\
		\iff& \begin{cases}
			a \left( \frac{1}{n}\sum_{i=1}^n {x_i}^2 - \overline{x}^2\right) = \overline{y} - \overline{x} \overline{y}\\
			b = \frac{1}{n}\sum_{i=1}^ny_i - \frac{a}{n}\sum_{i=1}^nx_i = \frac{1}{n}\sum_{i=1}^n x_i y_i - \overline{x} \overline{y}
		\end{cases}\\
		&\text{ où } \overline{x} = \frac{1}{n} \sum_{i=1}^n x_i,~\overline{y} = \frac{1}{n}\sum_{i=1}^n y_i\\
		\iff& \begin{cases}
			a = \frac{\Cov(x,y)}{V(x)}\\
			b = \overline{y} - a\overline{x}
		\end{cases}
	\end{align*}

	Coefficient de corrélation: $\frac{\Cov(x,y)}{\sigma_x \sigma_y} \in [-1, 1]$
\end{exm}












		\part{Corps}

\begin{exm}[Problème]
	\begin{itemize}
		\item 
			avec $A = \Z / 9 \Z$, résoudre $\overline{x}^2 = \overline{0}$ \\
			\begin{center}
				\begin{tabular}{|c|c|c|c|c|c|c|c|c|c|c|}
					\hline
					$\overline{x}$&$\overline{0}$& $\overline{1}$ &$\overline{2}$&$\overline{3}$ &$\overline{4}$ &$\overline{5}$ &$\overline{6}$ &$\overline{7}$ &$\overline{8}$& $\overline{9}$ \\
					\hline
					$\overline{x}^2$&$\overline{0}$ &$\overline{1}$ &$\overline{4}$ &$\overline{0}$ &$\overline{7}$ &$7$ &$\overline{0}$ &$\overline{4}$ &$\overline{1}$&$\overline{0}$\\
					\hline
				\end{tabular}
			\end{center}
			On a trouvé 3 solutions: $\overline{0}$, $\overline{3}$, $\overline{6}$.
		\item $\Z / 8\Z$
			\begin{center}
				\begin{tabular}{|c|c|c|c|c|c|c|c|c|}
					\hline
					$\overline{x}$& $\overline{0}$& $\overline{1}$& $\overline{2}$& $\overline{3}$& $\overline{4}$& $\overline{5}$& $\overline{6}$& $\overline{7}$\\
					\hline
					$\overline{x^2}$& $\overline{0}$& $\overline{1}$& $\overline{4}$& $\overline{1}$& $\overline{0}$& $\overline{1}$& $\overline{4}$& $\overline{1}$\\
					\hline
				\end{tabular}
			\end{center}
			$\overline{x}^2=7$ a 4 solutions: $\overline{1}, \overline{7}, \overline{3},\text{ et } \overline{5}$
		\item $A = \mathbbm{H} = \{a + bi + cj + dk  \mid  (a,b,c,d) \in \R^4\}$ \\
			$i^2 = j^2 = k^2 = -1$ 
			\begin{align*}
				\begin{array}{c c c}
					ij = k & jk = i & ji = j\\
					ji = -k & kj = -i & ik = -j
				\end{array}
			\end{align*}
			Dans cet anneau, $-1$ a 6 racines!
	\end{itemize}
\end{exm}

\begin{defn}
	Soit $(\mathbbm{K}, +, \times)$ un ensemble muni de deux lois de composition internes. On dit que c'est un \underline{corps} si
	 \begin{enumerate}
		\item $(\mathbbm{K}, \times)$ est un groupe abélien
		\item $(\mathbbm{K}, \times)$ est un monoïde commutatif
		\item $\forall x \in \mathbbm{K}\setminus \{0_\mathbbm{K}\}, \exists y \in \mathbbm{K}, xy = 1_\mathbbm{K}$
		\item $0_\mathbbm{K} \neq  1_\mathbbm{K}$
	\end{enumerate}
	\index{corps}
\end{defn}

\begin{exm}
	\begin{itemize}
		\item $(\C, +, \times)$ est un corps
		\item $(\R, +, \times)$ est un corps
		\item $(\Q, +, \times)$ est un corps
		\item $(\Z, +, \times)$ n'est pas un corps
	\end{itemize}
\end{exm}

\begin{prop}
	$(\Z / n\Z, +, \times)$ est un corps si et seulement si $n$ est premier.
\end{prop}

\begin{prv}
	\[
		\left( \Z / n\Z \right)^\times = \left\{ \overline{k}  \mid k \wedge n = 1 \right\}
	\] 
\end{prv}


\begin{prop}
	Tout corps est un anneau intègre.
\end{prop}

\begin{prv}
	Soit $(\mathbbm{K}, +, \times)$ un corps. Soient $(a,b) \in \mathbbm{K}^2$ tel que $a \times b = 0_\mathbbm{K}$.\\
	On suppose $a \neq  0_\mathbbm{K}$. Alors, $a$ est inversible et donc \[
		b = a^{-1} \times a \times b = a^{-1} \times 0_\mathbbm{K} = 0_\mathbbm{K}
	\] 
\end{prv}

\begin{exm}
	Soit $(\mathbbm{K},+,\times)$ un corps.\\
	Résoudre \[
		\begin{cases}
			x^2 = 1_\mathbbm{K}\\
			x \in \mathbbm{K}
		\end{cases}
	\]

	\begin{align*}
		x^2 = 1_\mathbbm{K} &\iff x^2 - 1_\mathbbm{K} = 0_\mathbbm{K}\\
		&\iff (x - 1_\mathbbm{K})(x+1_\mathbbm{K}) = 0_\mathbbm{K}\\
		&\iff x - 1_\mathbbm{K} = 0_\mathbbm{K} \text{ ou } x + 1_\mathbbm{K} = 0_\mathbbm{K}\\
		&\iff x = 1_\mathbbm{K} \text{ ou } x = -1_\mathbbm{K}
	\end{align*}

	Il y a au plus 2 solutions.
\end{exm}

\begin{prop}
	Soit $(\mathbbm{K},+,\times )$ un corps et $P$ un polynôme à coefficients dans $\mathbbm{K}$ de degré $n$. Alors, l'équation $P(x) = 0_{\mathbbm{K}}$ a au plus $n$ solutions dans $\mathbbm{K}$ 
	\qed
\end{prop}

\begin{crlr}[(Théorème de Wilson)]
	voir exercice 16 du TD 12
\end{crlr}


\begin{defn}
	Soit $(\mathbbm{K}, +, \times)$ un corps et $L\subset \mathbbm{K}$.\\
	On dit que $L$ est un \underline{sous corps} de $\mathbbm{K}$ si
	\begin{enumerate}
		\item $L$ est un anneau de $(\mathbbm{K}, +, \times)$ non nul
		\item $\forall x \in L\setminus \{0_\mathbbm{K}\}, x^{-1} \in L$ 
	\end{enumerate}
	\vspace{2mm}
	en d'autres termes si
	\begin{enumerate}
		\item $\forall (x,y) \in L^2, x - y \in L$
		\item $\forall (x,y) \in L^2, x \times y^{-1} \in L$
	\end{enumerate}
	\vspace{5mm}
	On dit aussi que $\mathbbm{K}$ est une \underline{extension} de $L$.
	\index{sous corps}
	\index{extension}
\end{defn}

\begin{prop}
	Tout sous corps est un corps. \qed
\end{prop}

\begin{defn}
	Soient $(\mathbbm{K}_1,+,\times )$ et $(\mathbbm{K}_2,+, \times)$ deux corps et $f: \mathbbm{K}_1 \to \mathbbm{K}_2$.\\
	On dit que $f$ est un \underline{morphisme de corps} si $f$ est un morphisme d'anneaux.\\
	i.e. si
	\[
		\begin{cases}
			\forall (x,y) \in {\mathbbm{K}_1}^2,& f(x+y) = f(x) + f(y)\\
			\forall (x,y) \in {\mathbbm{K}_1}^2,& f(x \times y) = f(x) \times f(y)\\
		\end{cases}
	\] 
	\index{homomorphisme (de corps)}
	\index{morphisme (de corps)}
\end{defn}

\begin{prop}
	Tout morphisme de corps est injectif.
\end{prop}

\begin{prv}
	Soit $f: \mathbbm{K}_1 \to \mathbbm{K}_2$ un morphisme de corps.\\
	\begin{itemize}
		\item $\Ker(f)$ est un sous groupe de $(\mathbbm{K}_1, +)$ 
		\item Soit $x \in \Ker(f)$ et $y \in \mathbbm{K}_1$ \[
				f(x \times y) = f(x) \times f(y) = 0_{\mathbbm{K}_2} \times f(y) = 0_{\mathbbm{K}_2}
			\]
		\item Soit $x \in \Ker(f) \setminus \{0_{\mathbbm{K}_1}\}$.\\
			Alors, $x$ est inversible.\\
			\begin{align*}
				\begin{rcases*}
					x \in \Ker(f)\\
					x^{-1} \in \mathbbm{K}_1
				\end{rcases*}& \text{ donc } x \times x ^{-1} \in \Ker(f)\\
				&\text{ donc } 1_{\mathbbm{K}_1} \in \Ker(f)\\
				&\text{ donc } f(1_{\mathbbm{K}_1}) = 0_{\mathbbm{K}_2}
			\end{align*}
			Or, $f(1_{\mathbbm{K}_1}) = 1_{\mathbbm{K}_2} \neq 0_{\mathbbm{K}_2}$
	\end{itemize}
	Donc, $\Ker(f) = \{0_{\mathbbm{K}_1}\}$ donc $f$ est injective.
\end{prv}

\begin{exm}
	$\begin{array}{cc}
		\C &\longrightarrow \C\\
		z &\longmapsto \overline{z}\\
	\end{array}$ est un morphisme de corps
\end{exm}



		\addrecap
	}

	{
		\chap[28]{Sous-espaces affines d'un espace vectoriel}
		\renewcommand{\cwd}{../chap28}
		\begin{defn}
	Soit $E$ un $\mathbbm{K}$-espace vectoriel. On dit que $E$ est de \underline{dimension finie} si $E$ a au moins une famille génératrice finie. On dit que $E$ est de \underline{dimension infinie} sinon.
	\index{dimension finie (espace vectoriel)}
	\index{dimension infinie (espace vectoriel)}
\end{defn}

\begin{thm}
	[Théorème de la base extraite]
	Soit $E$ un $\mathbbm{K}$-espace vectoriel non nul de dimension finie. Soit $\mathcal{G}$ une famille génératrice finie de $E$. Alors, il existe une base $\mathcal{B}$ de $\mathcal{E}$ telle que $\mathcal{B} \subset \mathcal{G}$.
\end{thm}

\begin{prv}
	[par récurrence sur $\#G = \Card(G)$]
	\begin{itemize}
		\item Soit $E$ un $\mathbbm{K}$-espace vectoriel non nul engendré par $\mathcal{G} = (u)$.\\
			Si $u = 0_E$, alors $E = \{0_E\}$: une contradiction $\lightning$ \\
			Donc $u \neq 0_E$ donc $(u)$ est libre. En effet, \[
				\forall \lambda \in \mathbbm{K}, \lambda u = 0_E \implies \lambda = 0_\mathbbm{K}
			\] Donc $\mathcal{G}$ est une base de $E$.\\
		\item Soit $n \in \N_*$. Soit $E$ un $\mathbbm{K}$-espace vectoriel. On suppose que si $E$ a une famille génératrice constituée de $n$ vecteurs, alors on peut extraire de cette famille une base de $E$.\\
			Soit $\mathcal{G}$ une famille génératrice de $E$ avec $n+1$ vecteurs.\\
			Si $\mathcal{G}$ est libre, alors $\mathcal{G}$ est une base de $E$. \\
			Si $\mathcal{G}$ n'est pas libre, alors il existe $u \in \mathcal{G}$ tel que $u \in \Vect(\mathcal{G}\setminus \{u\})$ \\
			Donc $\mathcal{G}\setminus \{u\}$ engendre $E$. Or, $\mathcal{G}\setminus \{u\}$ possède $n$ vecteurs. D'après l'hypothèse de récurrence, il existe une base $\mathcal{B}$ de $E$ telle que \[
				\mathcal{B} \subset \mathcal{G} \setminus \{u\} \subset \mathcal{G}
			\] 
	\end{itemize}
\end{prv}

\begin{crlr}
	Tout espace de dimension finie a une base.
	\qed
\end{crlr}

\begin{thm}
	[Théorème de la base incomplète]
	Soit $E$ un $\mathbbm{K}$-espace vectoriel de dimension finie, $\mathcal{G}$ une famille génératrice finie de $E$. $\mathcal{L}$ une famille libre de $E$. Alors, il existe une base $\mathcal{B}$ de $E$ telle que \[
		\mathcal{L} \subset \mathcal{B} \text{ et } \mathcal{B}\setminus \mathcal{L} \subset \mathcal{G}
	\] 
\end{thm}

\begin{prv}
	[par récurrence sur $\#(\mathcal{G}\setminus\mathcal{L})$]
	\begin{itemize}
		\item Avec les notations précédentes, on suppose que $\mathcal{G}\setminus\mathcal{L} \neq \O$ \[
				\forall u \in \mathcal{G}, u \in \mathcal{L}
			\] Donc $\mathcal{G} \subset \mathcal{L}$ donc $\mathcal{L}$ est génératrice donc $\mathcal{L}$ est une base de $E$. On pose $\mathcal{B} = \mathcal{L}$ et alors \[
				\mathcal{L} \subset  \mathcal{B} \text{ et } \mathcal{B}\setminus\mathcal{L} = \O \subset  \mathcal{G}
			\] 
		\item Soit $n \in \N$. On suppose que si $\mathcal{G}$ est génératrice et $\mathcal{L}$ libre avec $\#(\mathcal{G}\setminus\mathcal{L}) = n$ alors il existe une base $\mathcal{B}$ de $E$ telle que \[
			\mathcal{L}\subset \mathcal{B} \text{ et } \mathcal{B}\setminus\mathcal{L}\subset \mathcal{G}
		\] Soient à présent $\mathcal{G}$ une famille génératrice de $E$ et $\mathcal{L}$ une famille libre de $E$ telles que $\#(\mathcal{G}\setminus\mathcal{L}) = n+1 > 0$\\
		Si $\mathcal{L}$ engendre $E$, alors $\mathcal{L}$ est une base de $E$. On pose $\mathcal{B} = \mathcal{L}$ et on a bien \[
			\mathcal{L} \subset  \mathcal{B} \text{ et } \mathcal{B} \setminus \mathcal{L} = \O \subset  \mathcal{G}
		\] On suppose que $\mathcal{L}$ n'engendre pas $E$. Il existe $u \in \mathcal{G}$ tel que $u \not\in \Vec(\mathcal{L})$ (car sinon, $\mathcal{G} \subset \Vect(\mathcal{L})$ et donc $\underbrace{\Vect(\mathcal{G})}_{= E} \subset  \underbrace{\Vect(\mathcal{L})}_{ \subset E}$\\
		Donc $\mathcal{L} \cup \{u\} $ est libre. On pose $\mathcal{L}' = \mathcal{L} \cup \{u\} $ \[
			\mathcal{G}\setminus \mathcal{L}' = \mathcal{G}\setminus (\mathcal{L} \cup \{u\}) = (\mathcal{G}\setminus\mathcal{L})\setminus \{u\} 
		\] donc $\#(\mathcal{G}\setminus\mathcal{L}') = n+1 -1 = n$\\
		D'après l'hypothèse de récurrence, il existe $\mathcal{B}$ une base de $E$ telle que \[
			\mathcal{L} \subset  \mathcal{L}' \subset \mathcal{B} \text{ et } \mathcal{B}\setminus \mathcal{L}' \subset \mathcal{G}
		\] \[
			\mathcal{B} \setminus \mathcal{L} = \underbrace{\mathcal{B}\setminus\mathcal{L}'}_{\subset \mathcal{G}} \cup \underbrace{\{u\}}_{\subset \mathcal{G} \text{ car } u \in \mathcal{G}}
		\] On a $\mathcal{B}\setminus\mathcal{L}\subset \mathcal{G}$
	\end{itemize}
\end{prv}

\begin{thm}
	Soit $E$ un $\mathbbm{K}$-espace vectoriel de dimension finie. Toutes les bases de $E$ ont le même cardinal.
\end{thm}

\begin{prv}
	Soit $\mathcal{G}$ une famille génératrice finie de $E$ et $\mathcal{B} \subset  \mathcal{G}$ une base de $E$. On note $n = \#\mathcal{B}$ \\
	Soit $\mathcal{B}'$ une base de $E$. On pose $p = n - \#(\mathcal{B} \cap  \mathcal{B}')$. Montrons par récurrence sur  $p$ que $\#\mathcal{B} = \#\mathcal{B}'$ 
	\begin{itemize}
		\item On suppose que $p = 0$. Alors, $\#(\mathcal{B} \cap \mathcal{B}') = n$ \\
			Or, $\mathcal{B}' \cap \mathcal{B} \subset \mathcal{B}$ donc $\mathcal{B} \cap \mathcal{B}' = \mathcal{B}$ donc $\mathcal{B} \subset  \mathcal{B}'$ et donc $\mathcal{B} = \mathcal{B}'$ 
		\item Soit $p \in \N$. On suppose que si $\mathcal{B}'$ est une base de $E$ telle que $n - \#(\mathcal{B} \cap \mathcal{B}') = p$, alors $\#\mathcal{B}' = n$ \\
			Aoit $\mathcal{B}'$ une base de $E$ telle que $n - \#(\mathcal{B}\cap \mathcal{B}') = p+1 > 0$ \\
			Donc $\mathcal{B} \cap \mathcal{B}' \neq \mathcal{B}$. Soit $u \in \mathcal{B}' \setminus \mathcal{B}$. D'après le lemme d'échange, il existe $v \in \mathcal{B}\setminus \mathcal{B}'$ tel que $\mathcal{B}' \setminus \{u\} \cup \{v\}$ est une base de $E$. On pose $\mathcal{B}'' = \mathcal{B}' \setminus \{u\} \cup \{v\}$ 
			\begin{align*}
				\mathcal{B}'' \cap \mathcal{B} &= \left( (\mathcal{B}' \setminus \{u\})  \cap \mathcal{B} \right) \cup \{v\} \\
				&= (\mathcal{B}' \cap \mathcal{B}) \cup \{v\} \\
			\end{align*}
			donc,
			\begin{align*}
				n - \#(\mathcal{B}'' \cap \mathcal{B}) &= n - (\#(\mathcal{B}' \cap \mathcal{B}) + 1) \\
				&= p+1- 1 \\
				&= p \\
			\end{align*}
			D'après l'hypothèse de récurrence, \[
				\#\mathcal{B}'' = n
			\] Or, $\#\mathcal{B}'' = \#\mathcal{B}'$
	\end{itemize}
\end{prv}

\begin{lem}
	Soient $\mathcal{B}$ et $\mathcal{B}'$ deux bases de $E$ telles que $\mathcal{B}\subset \mathcal{B}'$. Alors, $\mathcal{B} = \mathcal{B}'$.
\end{lem}

\begin{prv}
	On suppose $\mathcal{B}' \neq \mathcal{B}$. Soit $u \in \mathcal{B}' \setminus \mathcal{B}$
	$u \in E = \Vect(\mathcal{B})$ donc $\mathcal{B} \cup \{u\}$ n'est pas libre.
	Donc $\mathcal{B}\cup \{u\} \subset \mathcal{B}'$ et $\mathcal{B}'$ est libre donc $\mathcal{B}\cup \{u\}$ est libre: une contradiction $\lightning$
\end{prv}

\begin{lem}
	[Lemme d'échange] Soient $\mathcal{B}_1$ et $\mathcal{B}_2$ deux bases de $E$ et $u \in \mathcal{B}_1 \setminus \mathcal{B}_2$. Alors, il existe $v \in \mathcal{B}_2$ tel que $(\mathcal{B}_1 \setminus \{u\}) \cup \{v\}$ soit une base de $E$.
\end{lem}

\begin{prv}
	[1${}^\text{nde}$ méthode]
	On suppose que pout tout $v \in \mathcal{B}_2$, $(\mathcal{B}_1\setminus \{u\}) \cup \{v\}$ n'est pas une base de $E$
	Soit $v \in \mathcal{B}_2$.
	\begin{itemize}
		\item Supposons $(\mathcal{B}_1\setminus \{u\})\cup \{v\}$ non libre. $\mathcal{B}_1 \setminus \{u\}$ est libre. Donc $v \in \Vect(\mathcal{B}_1 \setminus \{u\})$
		\item Supposons $(\mathcal{B}_1\setminus \{u\}) \cup \{v\}$ non génératrice.
			Comme $\mathcal{B}_1$ engendre $E$, $u \not\in \Vect(\mathcal{B}_1\setminus \{v\})$.
			On suppose que $\mathcal{B}_1 \neq \mathcal{B}_2$.
			$\forall v \in \mathcal{B}_2 \setminus \mathcal{B}_1, \Vect(\mathcal{B}_1 \setminus \{v\}) = \Vect(\mathcal{B}_1) = E \ni u$ 
			donc, $(\mathcal{B}_1\setminus \{u\}) \cup \{v\}$ engendre $E$ et donc \[
				v \in \Vect(\mathcal{B}_1 \setminus \{u\})
			\] On a aussi \[
				\forall v \in \mathcal{B}_1 \setminus \{u\}, v \in \Vect(\mathcal{B}_1\setminus \{u\})
			\] Comme $u \not\in \mathcal{B}_2$, on a \[
				\forall v \in \mathcal{B}_2, v \in \Vect(\mathcal{B}_1\setminus \{u\})
			\] docn \[
				E = \Vect(\mathcal{B}_2) \subset \Vect(\mathcal{B}_1\setminus \{u\})
			\] donc $\mathcal{B}_1\setminus \{u\}$ engendre $E$ donc $\mathcal{B}_1\setminus \{u\}$ est une base de $E$. Or, $\mathcal{B}_1 \setminus \{u\}  \subset  \mathcal{B}_1$, donc $\mathcal{B}_1\setminus \{u\} = \mathcal{B}_1$
	\end{itemize}
\end{prv}

\begin{prv}
	[2${}^\text{nde}$ méthode]
	On suppose que pout tout $v \in \mathcal{B}_2$, $(\mathcal{B}_1\setminus \{u\}) \cup \{v\}$ n'est pas une base de $E$
	\begin{itemize}
		\item Comme $u \in \mathcal{B}_1 \setminus \mathcal{B}_2$, nécéssairement $\mathcal{B}_1 \neq \mathcal{B}_2$ donc $\mathcal{B}_2 \not\subset \mathcal{B}_1$, donc $\mathcal{B}_2\setminus\mathcal{B}_1 \neq \O$ 
		\item Soit $v \in \mathcal{B}_2\setminus\mathcal{B}_1$. Il existe $(\lambda_w)_{w\in\mathcal{B}_1}$ une famille de scalaires presque nulle telle que \[
				v = \sum_{w \in \mathcal{B}_1} \lambda_w w - \lambda_u u + + \sum_{w \in \mathcal{B}_1\setminus \{u\}}\lambda_w w
			\]
			Si $\lambda_u \neq 0_E$, alors
			\begin{align*}
				u &= \lambda_u^{-1}\left( v - \sum_{w \in \mathcal{B}_1 \setminus \{u\}} \lambda_w w \right)\\
					&\in \Vect(\mathcal{B}_1\setminus \{u\} \cup v)
			\end{align*}
			 donc $\mathcal{B}_1 \subset \Vect(\mathcal{B}_1\setminus \{u\} \cup \{v\})$\\
			 et donc $E \subset  \Vect(\mathcal{B}_1 \setminus \{u\} \cup \{v\})$ \\
			 et donc $\mathcal{B}_1 \setminus \{u\} \cup \{v\}$ engendre $E$ \\
			 donc $\mathcal{B}_1 \setminus \{u\} \cup \{v\}$ n'est pas libre\\
			 donc $v \in \Vect(\mathcal{B}_1\setminus \{u\})$ (car $\mathcal{B}_1 \setminus \{u\}$ est libre\\
			 donc $\lambda_u = 0_\mathbbm{K}$ $\lightning$\\`

			 Donc, $\lambda_u = 0_\mathbbm{K}$, docn $v \in \Vect(\mathcal{B}_1\setminus \{u\})$ \\
			 On vient de prouver que
			 \begin{align*}
			 	\mathcal{B}_2 \setminus \mathcal{B}_1 \subset \Vect(\mathcal{B}_1 \setminus \{u\})\\
			 	\mathcal{B}_1 \setminus \{u\} \subset \Vect(\mathcal{B}_1 \setminus \{u\})\\
			 \end{align*}
			 Comme $u \not\in \mathcal{B}_2$, \[
			 	\mathcal{B}_2 \subset \Vect(\mathcal{B}_1 \setminus \{u\})
			 \] donc \[
			 	E = \Vect(\mathcal{B}_2) \subset  \Vect(\mathcal{B}_1 \setminus \{u\})
			 \] donc $\mathcal{B}_1 \setminus \{u\}$ engendre $E$. Donc,  $\mathcal{B}_1 \setminus \{u\}$ est une base de $E$.\\
			 Or, $\mathcal{B}_1 \setminus \{u\} \subset  \mathcal{B}_1$, donc $\mathcal{B}_1 \setminus \{u\} = \mathcal{B}_1$
	\end{itemize}
\end{prv}

\begin{defn}
	Soit $E$ un $\mathbbm{K}$-espace vectoriel de dimension finie. Le cardinal commun à toutes les bases de $E$ est appelé \underline{dimension} de $E$ est notée $\dim(E)$ ou $\dim_\mathbbm{K}(E)$\\
	C'est donc aussi le nombre de coordonnées de n'importe quel vecteur dans n'importe quelle base.
	\index{dimension (espace vectoriel)}
\end{defn}

\begin{exm}
	\begin{enumerate}
		\item $\dim_\R(\C) = 2$ et $\dim_\C(\C) = 1$ 
		\item $\dim_\mathbbm{K}(\mathbbm{K}^{n}) = n$ 
		\item $\dim_{\mathbbm{K}}(\mathcal{M}_{n,p}(\mathbbm{K})) = np$
	\end{enumerate}
\end{exm}

\begin{crlr}
	Soit $E$ un $\mathbbm{K}$-espace vectoriel de dimension finie, $\mathcal{L}$ une famille libre de $E$, $\mathcal{G}$ une famille génératrice de $E$. On note $n = \dim(E)$
	\begin{enumerate}
		\item $\#\mathcal{G} \ge n$ et $(\#\mathcal{G} = n \implies \mathcal{G} \text{ est une base de } E$)
		\item $\#\mathcal{L} \le n$ et $(\#\mathcal{L} = n \implies \mathcal{L} \text{ est une base de } E$)
	\end{enumerate}
\end{crlr}

\begin{crlr}
	$\R^{\R}$ est de dimension infinie.
	$\forall i \in \N, e_i: x \mapsto x^i$\\
	$(e_i)_{i\in\N}$ est libre dans $\R^\R$
\end{crlr}

\begin{prop}
	Soient $E$ et $F$ deux $\mathbbm{K}$-espaces vectoriels de dimension finie. Alors $E\times F$ est de dimension finie et $\dim(E\times F) = \dim(E) + \dim(F)$
\end{prop}

\begin{prv}
	Soit $(e_1,\ldots, e_n)$ une base de $E$, $(f_1, \ldots, f_p)$ une base de $F$.
	On pose \[
		\left\{\begin{array}
			{r c l}
			u_1 &=& (e_1,0_F)\\
			u_2 &=& (e_2,0_F)\\
					&\vdots&\\
			u_n &=& (e_n,0_F)\\
			u_{n+1} &=& (0_E, f_1)\\
			u_{n+2} &=& (0_E, f_2)\\
					&\vdots&\\
			u_{n+p} &=& (0_E,f_p)\\
		\end{array}\right.
	\]
	Soit $(x,y) \in E\times F$. \[
		\begin{cases}
			\exists (x_1,\ldots,x_n)\in \mathbbm{K}^n, x = \sum_{i=1}^{n} x_ie_i
			\exists (y_1,\ldots,y_n)\in \mathbbm{K}^n, x = \sum_{j=1}^{p} y_jf_j
		\end{cases}
	\] 
	\begin{align*}
		(x,y) &= \left( \sum_{i=1}^{n} x_ie_i, \sum_{i=1}^{p} y_jf_j \right)  \\
		&= \sum_{i=1}^{n} x_i (e_i + 0_F) + \sum_{j=1}^{p} y_j (0_E, f_j) \\
		&= \sum_{i=1}^{n} x_i u_i + \sum_{j=1}^{p} y_j u_{n+j} \\
	\end{align*}
	Donc, $E\times F = \Vect(u_1, \ldots, u_{n+p})$ donc $E\times F$ est de dimension finie.\\
	Soit $(\lambda_1, \ldots, \lambda_{n+p}) \in \mathbbm{K}^{n+p}$ tel que \[
		(*): \quad \sum_{k=1}^{n+p} \lambda_ku_k = 0_{E\times F} = (0_E, 0_F)
	\]
	\begin{align*}
		(*) &\iff \sum_{k=1}^{n} \lambda_k (e_k, 0_F) + \sum_{k=n+1}^{p} \lambda_k(0_E, f_{k-n}) = (0_E, 0_F)\\
				&\iff \begin{cases}
					\sum_{k=1}^{n} \lambda_k e_k = 0_E\\
					\sum_{k=n+1}^{p} \lambda_k f_{k-n} = 0_F
				\end{cases}\\
				&\iff \begin{cases}
					\forall k \in \left\llbracket 1,n \right\rrbracket, \lambda_k = 0_\mathbbm{K} \qquad&(\text{car $(e_1,\ldots,e_n)$ est libre})\\
					\forall k \in \left\llbracket n+1,n+p \right\rrbracket, \lambda_k = 0_\mathbbm{K} \qquad&(\text{car $(f_1,\ldots,f_n)$ est libre})\\
				\end{cases}
	\end{align*}
	Donc $(u_1, \ldots, u_{n+p})$ est une base de $E\times F$. Donc, $\dim(E\times F) = n + p = \dim(E) + \dim(F)$
\end{prv}

\begin{rmk}
	[Convention]
	\[\dim\big(\{0_E\}\big) = 0\]
\end{rmk}

\begin{thm}
	Soit $E$ un $\mathbbm{K}$-espace vectoriel de dimension finie, $F$ un sous-espace vectoriel de $E$. Alors, $F$ est de dimension finie et  $\dim(F) \le \dim(E)$\\
	Si $\dim(F) = \dim(E)$, alors $F = E$
\end{thm}

\begin{prv}
	On considère \[
		A = \{k \in \N \mid \text{il existe une famille libre de $F$ à $k$ éléments}\} 
	\]
	On suppose $F \neq \{0_E\}$.
	\begin{itemize}
		\item Soit $u \in F\setminus \{0_E\}$. $(u)$ est libre donc $1 \in A$ et donc $A \neq \O$
		\item Soit $\mathcal{L}$ une famille libre de $F$. Alors, $\mathcal{L}$ est une famille libre de $E$ \\
			donc $\#\mathcal{L} \le \dim(E)$\\
			Donc $A$ est majorée par $\dim(E)$ \\
			On en déduit que $A$ a un plus grand élément $p$.
		\item Soit $\mathcal{L}$ une famille libre de $F$ avec $p$ éléments.\\
			Si $\mathcal{L}$ n'engendre pas $F$, alors il existe $u\in F$ tel que $u\not\in \Vect(\mathcal{L})$ et donc $\mathcal{L} \cup \{u\}$ est une famille libre de $F$, donc $p+1 \in A$ en contradiction avec la maximalité de $p$.\\
			Donc $\mathcal{L}$ est une base de $F$ donc $F$ est de dimension finie et $\dim(F) = p \le \dim(E)$\\
	\end{itemize}

	Soit $\mathcal{B}$ une base de $F$. Alors, $\mathcal{B}$ est aussi une famille de libre de de $E$. Donc $\#\mathcal{B} \le \dim(E)$ donc $\dim(F) = \dim(E)$ \\
	Si $\dim(F) = \dim(E)$, alors $\mathcal{B}$ est une base de $E$, et donc $F = \Vect(\mathcal{B}) = E$
\end{prv}

\begin{prop}
	[Formule de Grassmann]
	Soit $E$ un $\mathbbm{K}$-espace vectoriel de dimension finie, $F$ et $G$ deux sous-espace vectoriels de $E$. Alors, \[
		\dim(F+G) = \dim(F) + \dim(G) - \dim(F\cap G)
	\] 
\end{prop}

\begin{prv}
	Soit $(e_1, \ldots, e_p)$ une base de $F\cap G$. $(e_1,\ldots,e_p)$ est une famille libre de $F$.\\
	On complète $(e_1, \ldots, e_p)$ en une base $(e_1, \ldots, e_p, u_1, \ldots, u_q)$ de $F$.\\
	De même, on complète $(e_1, \ldots, e_p)$ en une base $(e_1, \ldots, e_p, v_1, \ldots, v_r)$ de $G$.\\
	On pose  $\mathcal{B} = (e_1, \ldots, e_p, u_1, \ldots, u_q, v_1, \ldots, v_r)$. Montrons que $\mathcal{B}$ est une base de $F+G$
	\begin{itemize}
		\item Soit $u \in F+G$ \\
			On pose $u = v+w$ avec $\begin{cases}
				v\in F\\
				w \in G
			\end{cases}$.\\
			On pose $v = \sum_{i=1}^p \lambda_i e_i + \sum_{i=1}^q \mu_i u_i$ avec $(\lambda_1, \ldots, \lambda_p, \mu_1, \ldots, \lambda_q) \in \mathbbm{K}^{p+q}$\\
			On pose aussi $w = \sum_{i = 1}^p \lambda'_ie_i + \sum_{j=1}^r \nu_j v_j$ avec $(\lambda_1',\ldots,\lambda_p', \nu_1, \ldots, \nu_r) \in \mathbbm{K}^{p+r}$\\
			D'où, \[
				u = \sum_{i=1}^p (\lambda_i + \lambda'_i)e_i + \sum_{j=1}^q \mu_j u_j + \sum_{k=1}^r \nu_k v_k \in \Vect(\mathcal{B})
			\]
		\item Soient $(\lambda_1, \ldots, \lambda_p, \mu_1, \ldots, \mu_q, \nu_1, \ldots, \nu_r) \in \mathbbm{K}^{p+q+r}$.\\
			On suppose \[
				(*)\quad \sum_{i=1}^{p}\lambda_ie_i + \sum_{j=1}^q\mu_ju_j + \sum_{k=1}^r \nu_k v_k = 0_E
			\] 
			D'où, \[
				\underbrace{\sum_{i=1}^p\lambda_i e_i + \sum_{j=1}^q \mu_ju_j}_{\in F} = \underbrace{-\sum_{k=1}^r\nu_jv_k}_{\in G}
			\] 
			Donc, \[
				f = \sum_{i=1}^p \lambda_i e_i + \sum_{j=1}^q \mu_j u_j \in F\cap G
			\] Comme $(e_1, \ldots, e_p)$ est une base de $F\cap G$, $\exists ! (\lambda_1', \ldots, \lambda_p') \in \mathbbm{K}^p$ tel que \[
				f = \sum_{i=1}^p \lambda'_i e_i = \sum_{i=1}^p \lambda'_i e_i + \sum_{j=1}^q 0_\mathbbm{K}u_j
			\] Comme $(e_1, \ldots, e_p, u_1, \ldots, u_q)$ est une base de $F$, \[
				\forall k \in \left\llbracket 1, q \right\rrbracket, \mu_j = 0_\mathbbm{K}
			\] De même, \[
				\forall k \in \left\llbracket 1,r \right\rrbracket , \nu_k = 0_\mathbbm{K}
			\] On remplace dans $(*)$ pour trouver \[
				\sum_{i=1}^p \lambda_ie_i = 0_E
			\] Comme $(e_1, \ldots, e_p)$ est libre, \[
				\forall i \in \left\llbracket 1,p \right\rrbracket, \lambda_i = 0_\mathbbm{K}
			\] Donc $\mathcal{B}$ est libre.\\
			Donc, 
			\begin{align*}
				\dim(F+G) &=  p +q + r \\
				&= (p+q)+ (p+r) - p \\
				&= \dim(F) + \dim(G) - \dim(F\cap G) \\
			\end{align*}
	\end{itemize}
\end{prv}

\begin{crlr}
	Avec les hypothèse précédentes, \[
		E = F \oplus G \iff \begin{cases}
			F \cap  G = \{0_E\} \\
			\dim(E) = \dim(F) + \dim(G)
		\end{cases}
	\] 
\end{crlr}

\begin{prv}
	\begin{itemize}
		\item[``$\implies$''] On suppose $E = F \oplus G$ \\
			Comme la somme est directe, $F \cap G = \{0_E\}$ 
			\begin{align*}
				\dim(E) &= \dim(F)\\
				&= \dim(F) + \dim(G) - \dim(F\cap G)\\
				&= \dim(F) + \dim(G)\\
			\end{align*}
		\item[``$\impliedby$''] On suppose $F\cap G = \{0_E\}$ et $\dim(E) = \dim(F) + \dim(G)$.\\
			On sait déjà que $F+G = F \oplus G$\\
			 \begin{align*}
				\dim(F+G) = \dim(F) + \dim(G) - \dim(F \cap G) = \dim(E)
			\end{align*}
			Donc $F + G = E$
	\end{itemize}
\end{prv}

\begin{prop}
	Soit $F$ un $\mathbbm{K}$-espace vectoriel de dimension finie $n$. Soit $\mathcal{B} = (e_1, \ldots, e_n)$ une base de $F$. L'application
	\begin{align*}
		f: \mathbbm{K}^n &\longrightarrow F \\
		(\lambda_1, \ldots, \lambda_n) &\longmapsto \sum_{i=1}^n \lambda_i e_i
	\end{align*} est bijective.\\
	Si $\mathbbm{K}$ est infini, $\mathbbm{K}^n$ aussi et donc $F$ aussi.\\
	Si $\#\mathbbm{K} = p \in \N_*$,
	\begin{align*}
		\#&\mathbbm{K}^n = p^n\\
		&\vrt=\\
		\#&F
	\end{align*}
\end{prop}


		\part{Dérivation}

\underline{Motivation}:

{
\begin{wrapfigure}{l}{3cm}
	\centering
	\begin{asy}
		import three;

		size(3cm);
		settings.render=0;
		settings.prc=false;
		currentprojection = obliqueZ;

		draw(unitbox);
		draw(shift(1.1Z + 0.05X) * (O -- X), Arrows3(TeXHead2));
		draw(shift(1.1Z + 0.05Y) * (O -- Y), Arrows3(TeXHead2));
		draw(shift(1.1X + 0.05Z) * (O -- Z), Arrows3(TeXHead2));

		label("$x$", (X/2) + (1.1Z + 0.05X), align=S);
		label("$y$", (Y/2) + (1.1Z + 0.05Y), align=W);
		label("$z$", (Z/2) + X, align=SE);
	\end{asy}
\end{wrapfigure}

\begin{align*}
	&S(x,y,z) = 2(xy + xz + yz)\\
	&V(x,y,z) = xyz
\end{align*}

On cherche à minimiser $S$ avec la contrainte $V = 1$.

Soit $f : \begin{array}{rcl}
	\left( \R_*^+ \right)^2 &\longrightarrow& \R \\
	(x,y) &\longmapsto& S\left( x,y,\frac{1}{xy} \right) = 2\left( xy + \frac{1}{y} + \frac{1}{x} \right).
\end{array}$

On cherche $(a,b) \in \left( \R^+_* \right)^2$ tel que \[
	\forall (x,y) \in (\R^+_*), f(x,y) \ge f(a,b).
\]
}

\begin{defn}
	Soit $f: U \to \R$ où $U$ est un ouvert de $\R^2$. Soit $(a,b) \in U$.
	\vspace{2mm}

	Si $\lim_{x \to a} \frac{f(x,b) - f(a,b)}{x - a} \in \R$, alors on dit que $f$ a une dérivée partielle suivant $x$ en $(a,b)$ et cette limite est notée \[
		\partial f_1(a,b) = \frac{\partial f}{\partial x}(a,b).
	\]

	Si $\lim_{y \to b} \frac{f(a,y) - f(a,b)}{y - b} \in \R$, alors on dit que $f$ a une dérivée partielle suivant $y$ et la limite est notée \[
		\partial f_2(a,b) = \frac{\partial f}{\partial y}(a,b).
	\]
\end{defn}

\begin{exm}
	\begin{enumerate}
		\item $f: (x,y) \mapsto xy + x - y$.

			\begin{align*}
				&\frac{\partial f}{\partial x} : (x,y) \mapsto y + 1,\\
				&\frac{\partial f}{\partial y} : (x,y) \mapsto x - 1.
			\end{align*}

		\item $f: (x,y) \mapsto xy + \frac{1}{y}+ \frac{1}{x}$.

			\begin{align*}
				&\frac{\partial f}{\partial x}: (x,y) \mapsto y - \frac{1}{x^2},\\
				&\frac{\partial f}{\partial y}: (x,y) \mapsto x - \frac{1}{y^2}.
			\end{align*}

		\item Trouver $f$ telle que $\begin{cases}
				(1): \qquad \frac{\partial f}{\partial x}=y,\\[2mm]
				(2): \qquad \frac{\partial f}{\partial y} = x.
			\end{cases}$

			D'après $(1)$ : \[
				\forall (x,y), \exists C(y) \in \R, f(x,y) = xy + C(y)
			\] et donc \[
				\frac{\partial f}{\partial y}(x,y) = x + C'(y)
			\] donc $C'(y) = 0$ et donc $C$ est constante.

		\item Trouver $f$ telle que $\begin{cases}
			\frac{\partial f}{\partial x} = -y,\\[2mm]
			\frac{\partial f}{ƒ\partial y} = x.
		\end{cases}$

		Ce n'est pas possible !
	\end{enumerate}
\end{exm}

\begin{defn}~\\
	\begin{minipage}{\linewidth}
		\begin{wrapfigure}{r}{4cm}
			\centering
			\vspace{-5mm}
			\begin{asy}
				import three;
				import graph3;
				size(4cm);

				settings.render = 0;
				settings.prc = false;
				currentprojection = obliqueX;

				draw(O -- X, Arrow3(TeXHead2));
				draw(O -- Y, Arrow3(TeXHead2));
				draw(O -- Z, Arrow3(TeXHead2));

				triple f(real x, real y, real z = 0) { return (x,y,cos(x - 0.5) * cos(y - 0.5)/1.2 + 0.15); }

				real inc = 1 / 5;

				for(real x = 0; x <= 1; x += inc) {
					draw(graph(
						new real(real t) { return x; }, // x
						new real(real y) { return y; }, // y
						new real(real y) { return f(x,y).z; }, // z
						0, 1
					), gray);
				}

				for(real y = 0; y <= 1; y += inc) {
					draw(graph(
						new real(real x) { return x; }, // x
						new real(real t) { return y; }, // y
						new real(real x) { return f(x,y).z; }, // z
						0, 1
					), gray);
				}

				path3 path1 = (0.8, 0.2, 0) .. (0.5, 0.5, 0) .. (0.3, 0.7, 0);
				path3 path2 = f(0.8, 0.2, 0) .. f(0.5, 0.5, 0) .. f(0.3, 0.7, 0);
				path3 d = (0.2, 0.3, 0) .. (0.3, 0.4, 0) .. (0.2, 0.7, 0) .. (0.8, 0.9, 0) .. (0.6, 0.2, 0) .. cycle;

				draw(path1, red, Arrow3(TeXHead2));
				draw(path2, red, Arrow3(TeXHead2, position=0.8));

				dot((0.5, 0.5, 0));
				dot(f(0.5, 0.5, 0));
				draw((0.5, 0.5, 0) -- f(0.5, 0.5, 0), dashed);
				draw(d);

				label("$w$", (0.3, 0.7, 0), red, align=SE);
				label("$U$", (0.8, 0.9, 0), align=SE);
			\end{asy}
		\end{wrapfigure}

		Soit $f: U \to \R$ où $U$ est un ouvert. Soit $(a,b) \in U$. Soit $w = (w_1, w_2) \in \R^2$.

		Si 
		\[
			\lim_{t\to 0} \frac{f(a + tw_1, b + tw_2) - f(a,b)}{t}
		\] existe et est réelle, alors on dit que $f$ a une dérivée dans la direction de $w$ et la limite est notée \[
			\mathrm{d}f(w)\,(a,b) = D_w(f)\,(a,b).
		\]
	\end{minipage}
\end{defn}

\begin{exm}
	\begin{align*}
		f: \left( \R_*^+ \right)^2 &\longrightarrow \R \\
		(x,y) &\longmapsto xy+\frac{1}{x}+\frac{1}{y}.
	\end{align*}

	On pose $(a,b) = (1,2)$, $w = (w_1, w_2) = (1,1)$.
	\begin{align*}
		\frac{f(1+t, 2+t) - f(1,2)}{t} &= \frac{1}{t} \left( (1+t)(2+t) + \frac{1}{1+t} + \frac{1}{2+t} - 3 - \frac{1}{2} \right) \\
		&= \frac{1}{t}\left(\cancel 2 + 3t + \po(t) + \cancel 1 - t + \po(t) + \frac{1}{2}\left( \cancel 1 - \frac{t}{2} + \po(t) \right) - \cancel3 - \cancel{\frac{1}{2}} \right) \\
		&= \frac{1}{t} \left( \frac{7}{4} t + \po(t) \right)  \\
		&= \frac{7}{4} + \po(1) \tendsto{t \to 0} \frac{7}{4}. \\
	\end{align*}

	Donc, \[
		\mathrm{d}f(1,1)\,(1,2) = \frac{7}{4}.
	\]
\end{exm}

\begin{rmk}~\\
	\begin{figure}[H]
		\centering
		\begin{asy}
			import solids;
			import graph;
			size(5cm);

			settings.render = 0;
			settings.prc = false;

			path3 par = graph(
				new real(real x) { return x; },
				new real(real x) { return 0; },
				new real(real x) { return x^2; },
				0,3);
			revolution r = revolution(par, axis=Z);

			path3 par2 = graph(
				new real(real x) { return x; },
				new real(real x) { return 0; },
				new real(real x) { return x^2; },
				-3,3);

			draw(r,1,longitudinalpen=nullpen);
			draw(r.silhouette());

			draw((-4, 0, -1) -- (-4, 0, 10) -- (4, 0, 10) -- (4, 0, -1) -- cycle, red);
			draw(par2, deepred);

			draw((4,4.5) -- (7, 4.5), black+0.5mm, Arrow(TeXHead));

			path par2d = graph(new real(real x) { return x^2; }, -3, 3);
			draw(shift((11, 0)) * par2d, deepred);

			dot(O);
			dot((11, 0));
		\end{asy}
	\end{figure}
\end{rmk}


%todo ajouter théorème-définition
\begin{thm}
	Soit $f : U \to \R$, $(a,b) \in U$. On suppose que $\frac{\partial f}{\partial x}$ et $\frac{\partial f}{\partial y}$ existent en $(a,b)$ et sont {\bfseries continues} en $(a,b)$. Alors,
	\begin{align*}
		&\forall (h, k) \in \R^2 \text{ tel que } (a +h, b + k) \in U,\\
		&f(a+ h, b + k) = f(a,b) + h \frac{\partial f}{\partial x}(a,b) + k \frac{\partial f}{\partial y}(a,b) + \po_{(h,k)\to (0,0)}\big(\|(h,k)\|\big).
	\end{align*}

	On dit que $f$ est de classe $\mathcal{C}^1$ si $\frac{\partial f}{\partial x}$ et $\frac{\partial f}{\partial y}$ existent et sont continues.

	\qed
\end{thm}

\begin{rmk}
	En physique, cette formule correspond à : \[
		\mathrm{d}f = \frac{\partial f}{\partial x}\mathrm{d}x + \frac{\partial f}{\partial y} \mathrm{d}y.
	\] En effet :
	\begin{align*}
		\mathrm{d}f &= f(x+ \mathrm{d}x, y + \mathrm{d}y) - f(x,y) \\
		&= \frac{\partial f}{\partial x} \mathrm{d}x + \frac{\partial f}{\partial y} \mathrm{d}y.
	\end{align*}
\end{rmk}

\begin{prop}
	Soit $f: U \to \R$ de classe $\mathcal{C}^1$ en $(a,b) \in U$. Alors, \[
		\forall w = (w_1, w_2) \in \R^2, \mathrm{d}f(w)\,(a,b) = w_1 \frac{\partial f}{\partial x}(a,b) + w_2 \frac{\partial f}{\partial y}(a,b).
	\]
\end{prop}

\begin{prv}
	Soit $w = (w_1, w_2) \in \R^2$. Soit $t \in \R^*$.
	\begin{align*}
		\frac{1}{t}\big(f(a + tw_1, b + tw_2) - f(a,b)\big)
		&= \frac{1}{t} \left( tw_1 \frac{\partial f}{\partial x}(a,b) + tw_2 \frac{\partial f}{\partial y}(a,b) + \po_{t \to 0}\big(\|tw\|\big) \right) \\
		&= w_1 \frac{\partial f}{\partial x}(a,b) + w_2 \frac{\partial f}{\partial y}(a,b) + \po_{t\to 0}(1) \\
		&\tendsto{t\to 0} w_1 \frac{\partial f}{\partial x}(a,b) + w_2\frac{\partial f}{\partial y}(a,b).
	\end{align*}
\end{prv}


\begin{defn}
	Avec les hypothèses précédentes, en posant \[
		\nabla f(a,b) = \left( \frac{\partial f}{\partial x}(a,b), \frac{\partial f}{\partial y}(a,b) \right) 
	\]on obtient \[
		\mathrm{d}f(w)\,(a,b) = \left<w  \mid \nabla f(a,b) \right>
	\] où $\left<\cdot|\cdot \right>$ est le produit scalaire.

	Le vecteur $\nabla f(a,b)$ est appelé \underline{gradient de $f$ en $(a,b)$}.

	Le développement limité à l'ordre 1 de $f$ devient \[
		f\big((a,b)+w\big) = f(a,b) + \left<w \mid \nabla f(a,b) \right> + \po_{w\to 0}(\|w\|)
	\]
\end{defn}

\begin{prop}
	Soit $f : U \to \R$ de classe $\mathcal{C}^1$.

	\begin{figure}[H]
    \centering
    \incfig{gradient}
	\end{figure}

	$\nabla f$ est orthogonal au lignes de niveaux de $f$, son orientation va dans le sens d'une augmentation de $f$.
\end{prop}

\begin{prv}
	Soit $\gamma : I \to U$ une courbe de niveau : \[
		\forall t \in I, f\big(\gamma(t)\big) = \text{cste}.
	\] D'après le lemme suivant : \[
		\forall t \in I, 0 = (f \circ \gamma)'(t) = \mathrm{d}f\big(\gamma'(t)\big)\big(\gamma(t)\big) = \left<\gamma'(t)  \mid \nabla f\big(\gamma(t)\big) \right>
	\] Donc $\nabla f\big(\gamma(t)\big)$ est orthogonal à $\gamma'(t)$.

	Pour tout $t \in I$, on pose $w(t) = t\, \nabla f\big(\gamma(t)\big)$. Donc \[
		f\big(\gamma(t) + w(t)\big) = f\big(\gamma(t)\big) + t \|\nabla f(\gamma(t))\|^2 + \po_{t \to 0}(t)
	\] Pour $t$ assez petit, $f\big(\gamma(t) + w(t)\big) - f\big(\gamma(t)\big)$ est du même signe que $t$.
\end{prv}

\begin{rmk}
	\begin{align*}
		V: \R^3 &\longrightarrow \R \\
		(x,y,z) &\longmapsto -mgz
	\end{align*}
	l'énerge potentielle de pesenteur

	On a donc \[
		\nabla V(x,y,z) = \left( \frac{\partial V}{\partial x}, \frac{\partial V}{\partial y}, \frac{\partial V}{\partial z} \right) = (0, 0, -mg) = \vec{P}.
	\]
\end{rmk}

\begin{lem}
	Soit $f : U \to \R$ de classe $\mathcal{C}^1$, $\gamma : \begin{array}{rcl}
		I &\longrightarrow& U \\
		t &\longmapsto& \big(x(t), y(t)\big)
	\end{array}$ où $x$ et $y$ sont dérivables.

	On pose \[
		\forall t \in I, \gamma'(t) = \big(x'(t), y'(t)\big).
	\] Alors $f \circ \gamma : I \to \R$ est dérivable et
	\begin{align*}
		\forall t \in I, (f \circ \gamma)'(t) &= \mathrm{d}f\big(\gamma'(t)\big) \big(\gamma(t)\big)\\
		&= \left<\gamma'(t)  \mid \nabla f\big(\gamma(t)\big)  \right> \\
		&= x'(t) \frac{\partial f}{\partial x}\big(x(t), y(t)\big) + y'(t) \frac{\partial f}{\partial y}\big(x(t),y(t)\big). \\
	\end{align*}
\end{lem}

\begin{prv}
	On fixe $t \in I$.

	\begin{align*}
		\forall h \neq 0, \frac{f \circ \gamma(t + h) - f \circ \gamma(t)}{h}
		&= \frac{1}{h}\big(f(\gamma(t)) + h\gamma'(t) + \po_{h\to 0}(h) - f(\gamma(t))\big) \\
		&= \frac{1}{h}\bigg(\cancel{f(\gamma(t))} + \left<h\gamma'(t) \mid \nabla f(\gamma(t)) \right> + \po_{h\to 0}(\|h\gamma'(t)\|) - \cancel{f(\gamma(t))}\bigg)\\
		&= \left<\gamma'(t) \mid \nabla f(\gamma(t)) \right> + \po_{h\to 0}(1) \\
		&\tendsto{h\to 0} \left<\gamma'(t)  \mid \nabla f(\gamma(t)) \right>
	\end{align*}
\end{prv}

\begin{defn}
	Soit $f : U \to \R$ de classe $\mathcal{C}^1$ et $(a,b) \in U$. On dit que $(a,b)$ est un \underline{point critique} de $f$ si $\nabla f(a,b) = 0$ i.e. $\frac{\partial f}{\partial x}(a,b) = \frac{\partial f}{\partial y}(a,b) = 0$.

	Dans ce cas, $f(a,b)$ est appelé \underline{valeur critique} de $f$.
\end{defn}

\begin{prop}~\\
	\begin{minipage}{\linewidth}
		\begin{wrapfigure}{r}{3cm}
			\centering
			\vspace{-1cm}
			\begin{asy}
				import solids;
				import graph;
				size(3cm);

				settings.render = 0;
				settings.prc = false;

				path3 par = graph(
					new real(real x) { return x; },
					new real(real x) { return 0; },
					new real(real x) { return -x^2; },
					0,3);
				revolution r = revolution(par, axis=Z);

				draw(r,1,longitudinalpen=nullpen);
				draw(r.silhouette());

				dot("$(a,b)$", O, red, align=N);
				real s = sqrt(2.5);
				path3 g=(s,0,-2.5)..(0,s,-2.5)..(-s,0,-2.5)..(0,-s,-2.5)..cycle;
				draw(g, deepcyan);
			\end{asy}
		\end{wrapfigure}
		Soit $f: U \to \R$ de classe $\mathcal{C}^1$ et $(a,b) \in U$ tel que \[
			\exists r > 0, \forall (x,y) \in B_{(a,b)}(r), f(x,y) \le f(a,b)
		\] Alors $\nabla f(a,b) = (0,0)$.
	\end{minipage}
\end{prop}

\begin{prv}
	Soit $g: x \mapsto f(x,b)$. $g(a)$ est un maximum local de $g$ donc $g'(a) = 0$.

	Or, $g'(a) = \frac{\partial f}{\partial x}(a,b)$

	donc $\frac{\partial f}{\partial x}(a,b) = 0$.

	Soit $h : y \mapsto f(a,y)$. On a de même $h'(b) = 0$.

	Or, $h'(b) = \frac{\partial f}{\partial y}(a,b)$.

	Donc, $\nabla f(a,b) = (0,0)$.
\end{prv}

\begin{rmk}
	Un minimum local est aussi une valeur critique.
\end{rmk}

\begin{figure}[H]
	\centering
	\begin{subfigure}{3cm}
		\centering
		\begin{asy}
			import solids;
			import graph;
			size(3cm);

			settings.render = 0;
			settings.prc = false;

			path3 par = graph(
				new real(real x) { return x; },
				new real(real x) { return 0; },
				new real(real x) { return -x^2; },
				0,3);
			revolution r = revolution(par, axis=Z);

			draw(r,1,longitudinalpen=nullpen);
			draw(r.silhouette());

			dot(O, red);
		\end{asy}
		\caption{Maximum local}
	\end{subfigure}
	\begin{subfigure}{3cm}
		\centering
		\begin{asy}
			import solids;
			import graph;
			size(3cm);

			settings.render = 0;
			settings.prc = false;

			path3 par = graph(
				new real(real x) { return x; },
				new real(real x) { return 0; },
				new real(real x) { return x^2; },
				0,3);
			revolution r = revolution(par, axis=Z);

			draw(r,1,longitudinalpen=nullpen);
			draw(r.silhouette());

			dot(O, red);
		\end{asy}
		\caption{Minimum local}
	\end{subfigure}
	\begin{subfigure}{3cm}
		\centering
		\begin{asy}
			import solids;
			import graph;
			size(3cm);

			settings.render = 0;
			settings.prc = false;
			currentprojection = obliqueZ;

			draw(graph(
				new real(real x) { return x; },
				new real(real x) { return -x^2 / 3; },
				new real(real x) { return 3; },
				-3, 3
			));

			draw(graph(
				new real(real x) { return x; },
				new real(real x) { return -x^2 / 3; },
				new real(real x) { return -3; },
				-3, 3
			));

			draw(graph(
				new real(real x) { return x; },
				new real(real x) { return -x^2 / 3 - 1; },
				new real(real x) { return 0; },
				-3, 3
			));

			draw(graph(
				new real(real x) { return 0; },
				new real(real x) { return x^2 / 9 - 1; },
				new real(real x) { return x; },
				-3, 3
			));

			draw(graph(
				new real(real x) { return -3; },
				new real(real x) { return x^2 / 9 - 4; },
				new real(real x) { return x; },
				-3, 3
			));

			draw(graph(
				new real(real x) { return 3; },
				new real(real x) { return x^2 / 9 - 4; },
				new real(real x) { return x; },
				-3, 3
			));

			dot((0,-1,0), red);
		\end{asy}
		\caption{Point de selle / Point col}
	\end{subfigure}
\end{figure}

\begin{exm}
	On revient à l'exemple donné en introduction : 
	\begin{align*}
		f: \left( \R^*_+ \right)^2 &\longrightarrow \R \\
		(x,y) &\longmapsto 2\left( xy + \frac{1}{x} + \frac{1}{y} \right).
	\end{align*}

	$\left( \R^+_* \right)^2$ est un ouvert de $\R^2$. Soit $(x,y) \in \left( \R^+_* \right)^2$.
	
	On a \[
		\begin{cases}
			\frac{\partial f}{\partial x}(x,y) = 2\left( y - \frac{1}{x^2} \right),\\
			\frac{\partial f}{\partial y}(x,y) = 2\left( x - \frac{1}{y^2} \right).
		\end{cases}
	\]

	\begin{align*}
		&\frac{\partial f}{\partial x}(x,y) = \frac{\partial f}{\partial y}(x,y) = 0\\
		\iff& \begin{cases}
			y = \frac{1}{x^2}\\
			x = \frac{1}{y^2}
		\end{cases}\\
		\iff& \begin{cases}
			y = \frac{1}{x^2}\\
			x = x^4
		\end{cases}\\
		\iff& \begin{cases}
			x = 1\\
			y = 1
		\end{cases}
	\end{align*}

	On vérivie que $f$ présente en effet un minium local en $(1,1)$. \[
		f(1,1) = 6
	\] On fixe $y \in \R^+_*$ et \[
		g : x \mapsto 2\left( xy + \frac{1}{x} + \frac{1}{y} \right).
	\] Donc \[
		\forall x \in \R^+_*, g'(x) = 2\left( y - \frac{1}{x^2} \right).
	\]
	\begin{center}
		\begin{tikzpicture}
			\tkzTabInit{$x$/1,$g'(x)$/1,$g$/2.3}{$0$, $\frac{1}{\sqrt{y}}$, $+\infty$}
			\tkzTabLine{,-,z,+,}
			\tkzTabVar{+/{}, -/$2\left( 2\sqrt{y} +\frac{1}{y} \right)$, +/{}}
		\end{tikzpicture}
	\end{center}
	
	Ainsi, \[
		\forall x \in \R^+_*, \forall y \in \R^+_*, f(x,y) \ge 2\left( 2\sqrt{y} + \frac{1}{y} \right)
	\] Soit $h : y \mapsto 2\sqrt{y} + \frac{1}{y}$. On a \[
		\forall y > 0, h'(y) = \frac{1}{\sqrt{y}} - \frac{1}{y^2} = \frac{y\sqrt{y} - 1}{y^2} = \frac{y^{\frac{3}{2}} - 1}{y^2}
	\]

	\begin{center}
		\begin{tikzpicture}
			\tkzTabInit{$y$/0.7,$h'(y)$/0.7,$h$/1.4}{$0$, $1$, $+\infty$}
			\tkzTabLine{,-,z,+,}
			\tkzTabVar{+/{}, -/$3$, +/{}}
		\end{tikzpicture}
	\end{center}

	Donc, \[
		\forall x,y > 0, f(x,y) \ge 2\times 3 = 6 = f(1,1).
	\]
\end{exm}

\begin{prop}
	[règle de la chaîne]

	Soit $f : \begin{array}{rcl}
		U &\longrightarrow& \R^2 \\
		(x,y) &\longmapsto& f(x,y)
	\end{array}$ de classe $\mathcal{C}^1$ et $U, V$ deux ouverts de $\R^2$.

	Soit $\varphi : \begin{array}{rcl}
		V &\longrightarrow& U \\
		(u,v) &\longmapsto& \varphi(u,v) = \big(x(u,v), y(u,v)\big)
	\end{array}$.

	On suppose que $x$ et $y$ sont de classe $\mathcal{C}^1$ sur $V$.

	Alors,  $f \circ \varphi : \begin{array}{rcl}
		V &\longrightarrow& \R \\
		(u,v) &\longmapsto& f\big(\varphi(u,v)\big)
	\end{array}$ est de classe $\mathcal{C}^1$ et
	\begin{align*}
		\forall (u_0, v_0) \in V, \frac{\partial (f \circ \varphi)}{\partial u}(u_0, v_0)
		&= \frac{\partial f}{\partial x}\big(\varphi(u_0, v_0)\big) \times \frac{\partial x}{\partial u}(u_0, v_0)\\
		&+ \frac{\partial f}{\partial y}\big(\varphi(u_0,v_0)\big) \frac{\partial y}{\partial u}(u_0,v_0)
	\end{align*}
	\begin{align*}
		\forall (u_0, v_0) \in V, \frac{\partial (f \circ \varphi)}{\partial v}(u_0, v_0)
		&= \frac{\partial f}{\partial x}\big(\varphi(u_0, v_0)\big) \times \frac{\partial x}{\partial v}(u_0, v_0)\\
		&+ \frac{\partial f}{\partial y}\big(\varphi(u_0,v_0)\big) \frac{\partial y}{\partial v}(u_0,v_0)
	\end{align*}
\end{prop}

\begin{exm}
	[changement de coordonnées polaires]
	On pose \begin{align*}
		\varphi: \R^+_* \times ]0,2\pi[ &\longrightarrow \R^2\setminus \left( R^+_* \times \{0\} \right) \\
		(r, \theta) &\longmapsto (r \cos \theta, r \sin\theta),
	\end{align*}
	\begin{align*}
		f: \R^2\setminus \left( R^+_* \times \{0\} \right) &\longrightarrow \R \\
		(x,y) &\longmapsto f(x,y),
	\end{align*}
	\begin{align*}
		g: \overbrace{\R^+_* \times ]0, 2\pi[}^{=V} &\longrightarrow \R \\
		(r, \theta) &\longmapsto f(r\cos\theta, r\sin\theta).
	\end{align*}

	\begin{align*}
		\forall (r_0,\theta_0) \in V,&\\[5mm]
		\frac{\partial g}{\partial r}(r_0, \theta_0) &= \frac{\partial f}{\partial x}(r_0\cos\theta_0, r_0\sin\theta_0)\cos\theta_0\\
		&+ \frac{\partial f}{\partial y}(r_0 \cos\theta_0, r_0\sin\theta_0)\sin\theta_0\\
		&= 2r_0\cos^2\theta_0 + 2r_0\sin^2(\theta_0) \\
		&= 2r_0 \\[5mm]
		\frac{\partial g}{\partial \theta}(r_0, \theta_0) &= \frac{\partial f}{\partial x}(r_0\cos\theta_0, r_0\sin\theta_0)r_0\sin\theta_0\\
		&+ \frac{\partial f}{\partial y}(r_0 \cos\theta_0, r_0\sin\theta_0)r_0\cos\theta_0\\
		&= -2{r_0}^2\cos(\theta_0)\sin(\theta_0) + 2{r_0}^2 \sin(\theta_0)\cos(\theta_0)\\
		&= 0 \\
	\end{align*}

	Donc, \[
		g(r, \theta) = r^2.
	\]
\end{exm}

\begin{exm}
	Résoudre \[
		\begin{cases}
			\frac{\partial f}{\partial x} = \frac{x}{x^2+y^2},\\
			\frac{\partial f}{\partial y} = \frac{y}{x^2+y^2}.\\
		\end{cases}
	\]

	On pose $g: (r, \theta) \mapsto f(r \cos\theta, r \sin\theta)$.

	\begin{align*}
		&\frac{\partial g}{\partial r} = \frac{1}{r}\cos^2\theta + \frac{1}{r}\sin^2\theta = \frac{1}{r},\\
		&\frac{\partial g}{\partial \theta} = -\cos(\theta) \sin(\theta) + \sin(\theta)\cos(\theta) = 0.
	\end{align*}

	Donc, \[
		\exists C \in \R, g: (r, \theta) \mapsto \ln r + C
	\] d'où,
	\begin{align*}
		\forall (x,y) \in \R^2 \setminus \{(0,0)\}, f(x,y) &= \ln\left(\sqrt{x^2 + y^2} \right)  + C\\
		&= \frac{1}{2}\ln(x^2 + y^2) + C. \\
	\end{align*}
\end{exm}

\begin{rmk}
	Soit $\mathcal{B} = (e_1, e_2)$ la base canonique de $\R^2$, $f: U \to \R$ de classe $\mathcal{C}^1$ avec $U$ un ouvert de $\R^2$.

	Soit $(x,y) \in U$.

	\begin{align*}
		\Mat_{\mathcal{B}}\big(\nabla f(x,y)\big) = \begin{pmatrix}
			\frac{\partial f}{\partial x}(x,y)\\[2mm]
			\frac{\partial f}{\partial y}(x,y)
		\end{pmatrix}
	\end{align*}

	Soit  \begin{align*}
		\varphi: V &\longrightarrow U \\
		(u,v) &\longmapsto \big(x(u,v), y(u,v)\big) 
	\end{align*} avec $x,y$ de classe $\mathcal{C}^1$. Soit $g = f \circ \varphi$.
	\begin{align*}
		\Mat_{\mathcal{B}}\big(\nabla g(u,v)\big)
		&= \begin{pmatrix}
			\frac{\partial g}{\partial u}(u,v) \\[2mm]
			\frac{\partial g}{\partial v}(u,v)
		\end{pmatrix} \\
		&= \begin{pmatrix}
			\frac{\partial x}{\partial u}(u,v) \frac{\partial f}{\partial x}(x,y)
			+ \frac{\partial y}{\partial u}(u,v)\frac{\partial f}{\partial y}(x,y)\\[3mm]
			\frac{\partial x}{\partial v}(u,v) \frac{\partial f}{\partial x}(x,y)
			+ \frac{\partial y}{\partial v}(u,v) \frac{\partial f}{\partial y}(x,y)
		\end{pmatrix}  \\
		&= \underbrace{\begin{pmatrix}
				\frac{\partial x}{\partial u}(u,v)& \frac{\partial y}{\partial u}(u,v)\\[3mm]
				\frac{\partial x}{\partial v}(u,v)& \frac{\partial y}{\partial v}(u,v)
		\end{pmatrix}}_{J(u,v)} \begin{pmatrix}
			\frac{\partial f}{\partial x}(x,y)\\[3mm]
			\frac{\partial f}{\partial y}(x,y)
		\end{pmatrix} \\
		&= J(u,v) \Mat_{\mathcal{B}}\big(\nabla f(x,y)\big) \\
	\end{align*}
	où $J(u,v) = 
	\begin{pNiceArray}{c:c}
		\Mat_{\mathcal{B}}\big(\nabla x(u,v)\big) & \Mat_{\mathcal{B}}\big(\nabla y(u,v)\big)
	\end{pNiceArray}$.

	On dit que $J(u,v)$ est \underline{la jacobienne} de $\varphi$ en $(u,v)$.
	L'application linéaire canoniquement associée à $J(u,v)$ est la \underline{différentielle de $\varphi$} en $(u,v)$ noté $\mathrm{d}\varphi(u,v)$.

	On a $\mathrm{d}\varphi(u,v) \in \mathcal{L}(R^2)$ et $\Mat_{\mathcal{B}}\big(\mathrm{d}\varphi(u,v)\big) = J(u,v)$.

	Par exemple, la jacobienne du changement de coordonnées polaires est \[
		J = \begin{pmatrix}
			\frac{\partial x}{\partial r} & \frac{\partial y}{\partial r}\\[3mm]
			\frac{\partial x}{\partial \theta} & \frac{\partial y}{\partial \theta}
		\end{pmatrix}
		= \begin{pmatrix}
			\cos\theta&\sin\theta\\
			-r\sin\theta&r\cos\theta
		\end{pmatrix}.
	\]
	$\underbrace{\det(J)}_{\text{le jacobien}} = r\cos^2\theta + r\sin^2\theta = r$

	Dans une intégrale double, si $(x,y) = \varphi(u,v)$, alors $\mathrm{d}x\mathrm{d}y = \det(J)\mathrm{d}u\mathrm{d}v$.

	Ici, \[
		\mathrm{d}x\ \mathrm{d}y = r\ \mathrm{d}r\ \mathrm{d}\theta.
	\]
\end{rmk}

\begin{prv}
	On pose $(x_0, y_0) = \varphi(u_0, v_0)$. Pour tout $(h,k) \in \R^2$ tels que $(u_0 + h, v_0 + k) \in V$, en posant $g = f  \circ \varphi$.

	\begin{align*}
		g(u_0 + h, v_0 + h) &= f\big(x(u_0 + h, v_0 + k), y(u_0 + h, v_0 + k)\big) \\
		&= f\left(
			x(u_0,v_0) + h \frac{\partial x}{\partial u}(u_0,v_0) + k \frac{\partial x}{\partial v}(u_0, v_0) + \po\big(\|(h,k)\|\big), \right.\\
		&\phantom{ = f\bigg(\bigg.}\left. y(u_0, v_0) + h \frac{\partial y}{\partial u}(u_0, v_0) + k \frac{\partial y}{\partial v}(u_0, v_0) + \po\big(\|(h,k)\|\big)
		\right)  \\
		&= f(x_0,y_0) \\
		&~+ \left( h \frac{\partial x}{\partial u}(u_0,v_0) + k \frac{\partial x}{\partial v}(u_0, v_0) + \po(\|(h,k)\|) \right) \frac{\partial f}{\partial x}(x_0,y_0)\\
		&~+ \left( h \frac{\partial y}{\partial u}(u_0, v_0) + k\frac{\partial y}{\partial v}(u_0, v_0) + \po(\|(h,k)\|) \right) \frac{\partial f}{\partial y}(x_0, y_0)\\
		&~+ \po(\|(h,k)\|)\\
		&= f(x_0, y_0) \\
		&~+ h \left( \frac{\partial x}{\partial u}(u_0, v_0) \frac{\partial f}{\partial x}(x_0, y_0) + \frac{\partial y}{\partial u}(u_0, v_0) \frac{\partial f}{\partial y}(x_0, y_0) \right)  \\
		&~+ k\left( \frac{\partial x}{\partial v}(u_0, v_0) \frac{\partial f}{\partial x}(x_0, y_0) + \frac{\partial y}{\partial v}(u_0, v_0) \frac{\partial f}{\partial y}(x_0, y_0) \right) 
		&~+ \po(\|(h,k)\|)\\
		&= g(u_0, v_0) + h \frac{\partial g}{\partial u}(u_0, v_0) + k \frac{\partial g}{\partial v}(u_0, v_0) + \po(\|(h,k)\|) \\
	\end{align*}

	Par identification,
	\[
		\frac{\partial g}{\partial u}(u_0, v_0) = \frac{\partial x}{\partial u}(u_0, v_0) \frac{\partial f}{\partial x}(x_0, y_0) + \frac{\partial y}{\partial u}(u_0, v_0) \frac{\partial f}{\partial y}(x_0,y_0)
	\] et \[
		\frac{\partial g}{\partial v}(u_0, v_0) = \frac{\partial x}{\partial v}(u_0,v_0) \frac{\partial f}{\partial x}(x_0, y_0) + \frac{\partial y}{\partial v}(u_0, v_0) \frac{\partial f}{\partial y}(x_0, y_0).
	\] 
\end{prv}

\begin{exm}
	[Régression linéaire]~\\
	\begin{figure}[H]
		\centering
		\begin{asy}
			import graph;
			axes(EndArrow);
			size(5cm);

			real f(real x) { return x + 0.5; }

			real k = 35 / (7 - 0.5);

			for(int i = 0; i < 35; ++i) {
				real mag = exp(sin(100 * pi/exp(1) * i)) * 0.8 + exp(cos(i*40)/3);
				real eps = mag * cos(10 * exp(1)/pi * i) / 3;
				dot((i/k,f(i/k) + eps));
			}

			draw(graph(f, -1, 7), orange);
		\end{asy}
	\end{figure}
	\[
		y = a x + b
	\] 
	On fixe $(a,b) \in \R^2$. \[
		\varepsilon(a,b) = \sum_{i=1}^n\big( y_i - (ax_i + b) \big)^2
	\] l'erreur totale.

	On veut minimiser $\varepsilon(a,b)$. On a 
	\[
		\forall (a,b) \in \R^2,
		\begin{cases}
			\frac{\partial \varepsilon}{\partial a}(a,b) = -2\sum_{i=1}^{n}(y_i - ax_i - b)x_i,\\
			\frac{\partial \varepsilon}{\partial b}(a,b) = -2\sum_{i=1}^{n}(y_i - ax_i - b).
		\end{cases}
	\]

	Donc,
	\begin{align*}
		(a,b) \text{ point critique de } \varepsilon \iff& \begin{cases}
			a \sum_{i=1}^n {x_i}^2 + b\sum_{i=1}^{n}x_i = \sum_{i=1}^{n} y_ix_i\\
			a\sum_{i=1}^{n}x_i + nb = \sum_{i=1}^ny_i
		\end{cases}\\
		\iff& \begin{cases}
			a \left( \frac{1}{n}\sum_{i=1}^n {x_i}^2 - \overline{x}^2\right) = \overline{y} - \overline{x} \overline{y}\\
			b = \frac{1}{n}\sum_{i=1}^ny_i - \frac{a}{n}\sum_{i=1}^nx_i = \frac{1}{n}\sum_{i=1}^n x_i y_i - \overline{x} \overline{y}
		\end{cases}\\
		&\text{ où } \overline{x} = \frac{1}{n} \sum_{i=1}^n x_i,~\overline{y} = \frac{1}{n}\sum_{i=1}^n y_i\\
		\iff& \begin{cases}
			a = \frac{\Cov(x,y)}{V(x)}\\
			b = \overline{y} - a\overline{x}
		\end{cases}
	\end{align*}

	Coefficient de corrélation: $\frac{\Cov(x,y)}{\sigma_x \sigma_y} \in [-1, 1]$
\end{exm}












		\part{Corps}

\begin{exm}[Problème]
	\begin{itemize}
		\item 
			avec $A = \Z / 9 \Z$, résoudre $\overline{x}^2 = \overline{0}$ \\
			\begin{center}
				\begin{tabular}{|c|c|c|c|c|c|c|c|c|c|c|}
					\hline
					$\overline{x}$&$\overline{0}$& $\overline{1}$ &$\overline{2}$&$\overline{3}$ &$\overline{4}$ &$\overline{5}$ &$\overline{6}$ &$\overline{7}$ &$\overline{8}$& $\overline{9}$ \\
					\hline
					$\overline{x}^2$&$\overline{0}$ &$\overline{1}$ &$\overline{4}$ &$\overline{0}$ &$\overline{7}$ &$7$ &$\overline{0}$ &$\overline{4}$ &$\overline{1}$&$\overline{0}$\\
					\hline
				\end{tabular}
			\end{center}
			On a trouvé 3 solutions: $\overline{0}$, $\overline{3}$, $\overline{6}$.
		\item $\Z / 8\Z$
			\begin{center}
				\begin{tabular}{|c|c|c|c|c|c|c|c|c|}
					\hline
					$\overline{x}$& $\overline{0}$& $\overline{1}$& $\overline{2}$& $\overline{3}$& $\overline{4}$& $\overline{5}$& $\overline{6}$& $\overline{7}$\\
					\hline
					$\overline{x^2}$& $\overline{0}$& $\overline{1}$& $\overline{4}$& $\overline{1}$& $\overline{0}$& $\overline{1}$& $\overline{4}$& $\overline{1}$\\
					\hline
				\end{tabular}
			\end{center}
			$\overline{x}^2=7$ a 4 solutions: $\overline{1}, \overline{7}, \overline{3},\text{ et } \overline{5}$
		\item $A = \mathbbm{H} = \{a + bi + cj + dk  \mid  (a,b,c,d) \in \R^4\}$ \\
			$i^2 = j^2 = k^2 = -1$ 
			\begin{align*}
				\begin{array}{c c c}
					ij = k & jk = i & ji = j\\
					ji = -k & kj = -i & ik = -j
				\end{array}
			\end{align*}
			Dans cet anneau, $-1$ a 6 racines!
	\end{itemize}
\end{exm}

\begin{defn}
	Soit $(\mathbbm{K}, +, \times)$ un ensemble muni de deux lois de composition internes. On dit que c'est un \underline{corps} si
	 \begin{enumerate}
		\item $(\mathbbm{K}, \times)$ est un groupe abélien
		\item $(\mathbbm{K}, \times)$ est un monoïde commutatif
		\item $\forall x \in \mathbbm{K}\setminus \{0_\mathbbm{K}\}, \exists y \in \mathbbm{K}, xy = 1_\mathbbm{K}$
		\item $0_\mathbbm{K} \neq  1_\mathbbm{K}$
	\end{enumerate}
	\index{corps}
\end{defn}

\begin{exm}
	\begin{itemize}
		\item $(\C, +, \times)$ est un corps
		\item $(\R, +, \times)$ est un corps
		\item $(\Q, +, \times)$ est un corps
		\item $(\Z, +, \times)$ n'est pas un corps
	\end{itemize}
\end{exm}

\begin{prop}
	$(\Z / n\Z, +, \times)$ est un corps si et seulement si $n$ est premier.
\end{prop}

\begin{prv}
	\[
		\left( \Z / n\Z \right)^\times = \left\{ \overline{k}  \mid k \wedge n = 1 \right\}
	\] 
\end{prv}


\begin{prop}
	Tout corps est un anneau intègre.
\end{prop}

\begin{prv}
	Soit $(\mathbbm{K}, +, \times)$ un corps. Soient $(a,b) \in \mathbbm{K}^2$ tel que $a \times b = 0_\mathbbm{K}$.\\
	On suppose $a \neq  0_\mathbbm{K}$. Alors, $a$ est inversible et donc \[
		b = a^{-1} \times a \times b = a^{-1} \times 0_\mathbbm{K} = 0_\mathbbm{K}
	\] 
\end{prv}

\begin{exm}
	Soit $(\mathbbm{K},+,\times)$ un corps.\\
	Résoudre \[
		\begin{cases}
			x^2 = 1_\mathbbm{K}\\
			x \in \mathbbm{K}
		\end{cases}
	\]

	\begin{align*}
		x^2 = 1_\mathbbm{K} &\iff x^2 - 1_\mathbbm{K} = 0_\mathbbm{K}\\
		&\iff (x - 1_\mathbbm{K})(x+1_\mathbbm{K}) = 0_\mathbbm{K}\\
		&\iff x - 1_\mathbbm{K} = 0_\mathbbm{K} \text{ ou } x + 1_\mathbbm{K} = 0_\mathbbm{K}\\
		&\iff x = 1_\mathbbm{K} \text{ ou } x = -1_\mathbbm{K}
	\end{align*}

	Il y a au plus 2 solutions.
\end{exm}

\begin{prop}
	Soit $(\mathbbm{K},+,\times )$ un corps et $P$ un polynôme à coefficients dans $\mathbbm{K}$ de degré $n$. Alors, l'équation $P(x) = 0_{\mathbbm{K}}$ a au plus $n$ solutions dans $\mathbbm{K}$ 
	\qed
\end{prop}

\begin{crlr}[(Théorème de Wilson)]
	voir exercice 16 du TD 12
\end{crlr}


\begin{defn}
	Soit $(\mathbbm{K}, +, \times)$ un corps et $L\subset \mathbbm{K}$.\\
	On dit que $L$ est un \underline{sous corps} de $\mathbbm{K}$ si
	\begin{enumerate}
		\item $L$ est un anneau de $(\mathbbm{K}, +, \times)$ non nul
		\item $\forall x \in L\setminus \{0_\mathbbm{K}\}, x^{-1} \in L$ 
	\end{enumerate}
	\vspace{2mm}
	en d'autres termes si
	\begin{enumerate}
		\item $\forall (x,y) \in L^2, x - y \in L$
		\item $\forall (x,y) \in L^2, x \times y^{-1} \in L$
	\end{enumerate}
	\vspace{5mm}
	On dit aussi que $\mathbbm{K}$ est une \underline{extension} de $L$.
	\index{sous corps}
	\index{extension}
\end{defn}

\begin{prop}
	Tout sous corps est un corps. \qed
\end{prop}

\begin{defn}
	Soient $(\mathbbm{K}_1,+,\times )$ et $(\mathbbm{K}_2,+, \times)$ deux corps et $f: \mathbbm{K}_1 \to \mathbbm{K}_2$.\\
	On dit que $f$ est un \underline{morphisme de corps} si $f$ est un morphisme d'anneaux.\\
	i.e. si
	\[
		\begin{cases}
			\forall (x,y) \in {\mathbbm{K}_1}^2,& f(x+y) = f(x) + f(y)\\
			\forall (x,y) \in {\mathbbm{K}_1}^2,& f(x \times y) = f(x) \times f(y)\\
		\end{cases}
	\] 
	\index{homomorphisme (de corps)}
	\index{morphisme (de corps)}
\end{defn}

\begin{prop}
	Tout morphisme de corps est injectif.
\end{prop}

\begin{prv}
	Soit $f: \mathbbm{K}_1 \to \mathbbm{K}_2$ un morphisme de corps.\\
	\begin{itemize}
		\item $\Ker(f)$ est un sous groupe de $(\mathbbm{K}_1, +)$ 
		\item Soit $x \in \Ker(f)$ et $y \in \mathbbm{K}_1$ \[
				f(x \times y) = f(x) \times f(y) = 0_{\mathbbm{K}_2} \times f(y) = 0_{\mathbbm{K}_2}
			\]
		\item Soit $x \in \Ker(f) \setminus \{0_{\mathbbm{K}_1}\}$.\\
			Alors, $x$ est inversible.\\
			\begin{align*}
				\begin{rcases*}
					x \in \Ker(f)\\
					x^{-1} \in \mathbbm{K}_1
				\end{rcases*}& \text{ donc } x \times x ^{-1} \in \Ker(f)\\
				&\text{ donc } 1_{\mathbbm{K}_1} \in \Ker(f)\\
				&\text{ donc } f(1_{\mathbbm{K}_1}) = 0_{\mathbbm{K}_2}
			\end{align*}
			Or, $f(1_{\mathbbm{K}_1}) = 1_{\mathbbm{K}_2} \neq 0_{\mathbbm{K}_2}$
	\end{itemize}
	Donc, $\Ker(f) = \{0_{\mathbbm{K}_1}\}$ donc $f$ est injective.
\end{prv}

\begin{exm}
	$\begin{array}{cc}
		\C &\longrightarrow \C\\
		z &\longmapsto \overline{z}\\
	\end{array}$ est un morphisme de corps
\end{exm}



		\part{Opérations sur les séries}

\begin{prop}
	L'ensemble $E = \{u \in \C^\N  \mid \Sigma u_n \text{ converge}\}$ est un sous-espace vectoriel de $\C^\N$ et \begin{align*}
		S: E &\longrightarrow \C \\
		u &\longmapsto \sum_{n=0}^{+\infty} u_n
	\end{align*} est une forme linéaire.
	\qed
\end{prop}

\begin{rmk}
	La somme d'une série convergente et d'une série divergente diverge.
	Le produit d'une série divergente par un scalaire non nul diverge.
\end{rmk}

	}

	{
		\chap[29]{Produit scalaire}
		\renewcommand{\cwd}{../chap29}
		Dans ce chapitre, $E$ désigne un {\large\color{red}$\R$}-espace vectoriel.
		\par On sait déjà calculer le produit scalaire en dimension 2 et 3 mais l'objectif de ce chapitre est de le généraliser en dimension potentiellement infinie.
		\begin{defn}
	Soit $E$ un $\mathbbm{K}$-espace vectoriel. On dit que $E$ est de \underline{dimension finie} si $E$ a au moins une famille génératrice finie. On dit que $E$ est de \underline{dimension infinie} sinon.
	\index{dimension finie (espace vectoriel)}
	\index{dimension infinie (espace vectoriel)}
\end{defn}

\begin{thm}
	[Théorème de la base extraite]
	Soit $E$ un $\mathbbm{K}$-espace vectoriel non nul de dimension finie. Soit $\mathcal{G}$ une famille génératrice finie de $E$. Alors, il existe une base $\mathcal{B}$ de $\mathcal{E}$ telle que $\mathcal{B} \subset \mathcal{G}$.
\end{thm}

\begin{prv}
	[par récurrence sur $\#G = \Card(G)$]
	\begin{itemize}
		\item Soit $E$ un $\mathbbm{K}$-espace vectoriel non nul engendré par $\mathcal{G} = (u)$.\\
			Si $u = 0_E$, alors $E = \{0_E\}$: une contradiction $\lightning$ \\
			Donc $u \neq 0_E$ donc $(u)$ est libre. En effet, \[
				\forall \lambda \in \mathbbm{K}, \lambda u = 0_E \implies \lambda = 0_\mathbbm{K}
			\] Donc $\mathcal{G}$ est une base de $E$.\\
		\item Soit $n \in \N_*$. Soit $E$ un $\mathbbm{K}$-espace vectoriel. On suppose que si $E$ a une famille génératrice constituée de $n$ vecteurs, alors on peut extraire de cette famille une base de $E$.\\
			Soit $\mathcal{G}$ une famille génératrice de $E$ avec $n+1$ vecteurs.\\
			Si $\mathcal{G}$ est libre, alors $\mathcal{G}$ est une base de $E$. \\
			Si $\mathcal{G}$ n'est pas libre, alors il existe $u \in \mathcal{G}$ tel que $u \in \Vect(\mathcal{G}\setminus \{u\})$ \\
			Donc $\mathcal{G}\setminus \{u\}$ engendre $E$. Or, $\mathcal{G}\setminus \{u\}$ possède $n$ vecteurs. D'après l'hypothèse de récurrence, il existe une base $\mathcal{B}$ de $E$ telle que \[
				\mathcal{B} \subset \mathcal{G} \setminus \{u\} \subset \mathcal{G}
			\] 
	\end{itemize}
\end{prv}

\begin{crlr}
	Tout espace de dimension finie a une base.
	\qed
\end{crlr}

\begin{thm}
	[Théorème de la base incomplète]
	Soit $E$ un $\mathbbm{K}$-espace vectoriel de dimension finie, $\mathcal{G}$ une famille génératrice finie de $E$. $\mathcal{L}$ une famille libre de $E$. Alors, il existe une base $\mathcal{B}$ de $E$ telle que \[
		\mathcal{L} \subset \mathcal{B} \text{ et } \mathcal{B}\setminus \mathcal{L} \subset \mathcal{G}
	\] 
\end{thm}

\begin{prv}
	[par récurrence sur $\#(\mathcal{G}\setminus\mathcal{L})$]
	\begin{itemize}
		\item Avec les notations précédentes, on suppose que $\mathcal{G}\setminus\mathcal{L} \neq \O$ \[
				\forall u \in \mathcal{G}, u \in \mathcal{L}
			\] Donc $\mathcal{G} \subset \mathcal{L}$ donc $\mathcal{L}$ est génératrice donc $\mathcal{L}$ est une base de $E$. On pose $\mathcal{B} = \mathcal{L}$ et alors \[
				\mathcal{L} \subset  \mathcal{B} \text{ et } \mathcal{B}\setminus\mathcal{L} = \O \subset  \mathcal{G}
			\] 
		\item Soit $n \in \N$. On suppose que si $\mathcal{G}$ est génératrice et $\mathcal{L}$ libre avec $\#(\mathcal{G}\setminus\mathcal{L}) = n$ alors il existe une base $\mathcal{B}$ de $E$ telle que \[
			\mathcal{L}\subset \mathcal{B} \text{ et } \mathcal{B}\setminus\mathcal{L}\subset \mathcal{G}
		\] Soient à présent $\mathcal{G}$ une famille génératrice de $E$ et $\mathcal{L}$ une famille libre de $E$ telles que $\#(\mathcal{G}\setminus\mathcal{L}) = n+1 > 0$\\
		Si $\mathcal{L}$ engendre $E$, alors $\mathcal{L}$ est une base de $E$. On pose $\mathcal{B} = \mathcal{L}$ et on a bien \[
			\mathcal{L} \subset  \mathcal{B} \text{ et } \mathcal{B} \setminus \mathcal{L} = \O \subset  \mathcal{G}
		\] On suppose que $\mathcal{L}$ n'engendre pas $E$. Il existe $u \in \mathcal{G}$ tel que $u \not\in \Vec(\mathcal{L})$ (car sinon, $\mathcal{G} \subset \Vect(\mathcal{L})$ et donc $\underbrace{\Vect(\mathcal{G})}_{= E} \subset  \underbrace{\Vect(\mathcal{L})}_{ \subset E}$\\
		Donc $\mathcal{L} \cup \{u\} $ est libre. On pose $\mathcal{L}' = \mathcal{L} \cup \{u\} $ \[
			\mathcal{G}\setminus \mathcal{L}' = \mathcal{G}\setminus (\mathcal{L} \cup \{u\}) = (\mathcal{G}\setminus\mathcal{L})\setminus \{u\} 
		\] donc $\#(\mathcal{G}\setminus\mathcal{L}') = n+1 -1 = n$\\
		D'après l'hypothèse de récurrence, il existe $\mathcal{B}$ une base de $E$ telle que \[
			\mathcal{L} \subset  \mathcal{L}' \subset \mathcal{B} \text{ et } \mathcal{B}\setminus \mathcal{L}' \subset \mathcal{G}
		\] \[
			\mathcal{B} \setminus \mathcal{L} = \underbrace{\mathcal{B}\setminus\mathcal{L}'}_{\subset \mathcal{G}} \cup \underbrace{\{u\}}_{\subset \mathcal{G} \text{ car } u \in \mathcal{G}}
		\] On a $\mathcal{B}\setminus\mathcal{L}\subset \mathcal{G}$
	\end{itemize}
\end{prv}

\begin{thm}
	Soit $E$ un $\mathbbm{K}$-espace vectoriel de dimension finie. Toutes les bases de $E$ ont le même cardinal.
\end{thm}

\begin{prv}
	Soit $\mathcal{G}$ une famille génératrice finie de $E$ et $\mathcal{B} \subset  \mathcal{G}$ une base de $E$. On note $n = \#\mathcal{B}$ \\
	Soit $\mathcal{B}'$ une base de $E$. On pose $p = n - \#(\mathcal{B} \cap  \mathcal{B}')$. Montrons par récurrence sur  $p$ que $\#\mathcal{B} = \#\mathcal{B}'$ 
	\begin{itemize}
		\item On suppose que $p = 0$. Alors, $\#(\mathcal{B} \cap \mathcal{B}') = n$ \\
			Or, $\mathcal{B}' \cap \mathcal{B} \subset \mathcal{B}$ donc $\mathcal{B} \cap \mathcal{B}' = \mathcal{B}$ donc $\mathcal{B} \subset  \mathcal{B}'$ et donc $\mathcal{B} = \mathcal{B}'$ 
		\item Soit $p \in \N$. On suppose que si $\mathcal{B}'$ est une base de $E$ telle que $n - \#(\mathcal{B} \cap \mathcal{B}') = p$, alors $\#\mathcal{B}' = n$ \\
			Aoit $\mathcal{B}'$ une base de $E$ telle que $n - \#(\mathcal{B}\cap \mathcal{B}') = p+1 > 0$ \\
			Donc $\mathcal{B} \cap \mathcal{B}' \neq \mathcal{B}$. Soit $u \in \mathcal{B}' \setminus \mathcal{B}$. D'après le lemme d'échange, il existe $v \in \mathcal{B}\setminus \mathcal{B}'$ tel que $\mathcal{B}' \setminus \{u\} \cup \{v\}$ est une base de $E$. On pose $\mathcal{B}'' = \mathcal{B}' \setminus \{u\} \cup \{v\}$ 
			\begin{align*}
				\mathcal{B}'' \cap \mathcal{B} &= \left( (\mathcal{B}' \setminus \{u\})  \cap \mathcal{B} \right) \cup \{v\} \\
				&= (\mathcal{B}' \cap \mathcal{B}) \cup \{v\} \\
			\end{align*}
			donc,
			\begin{align*}
				n - \#(\mathcal{B}'' \cap \mathcal{B}) &= n - (\#(\mathcal{B}' \cap \mathcal{B}) + 1) \\
				&= p+1- 1 \\
				&= p \\
			\end{align*}
			D'après l'hypothèse de récurrence, \[
				\#\mathcal{B}'' = n
			\] Or, $\#\mathcal{B}'' = \#\mathcal{B}'$
	\end{itemize}
\end{prv}

\begin{lem}
	Soient $\mathcal{B}$ et $\mathcal{B}'$ deux bases de $E$ telles que $\mathcal{B}\subset \mathcal{B}'$. Alors, $\mathcal{B} = \mathcal{B}'$.
\end{lem}

\begin{prv}
	On suppose $\mathcal{B}' \neq \mathcal{B}$. Soit $u \in \mathcal{B}' \setminus \mathcal{B}$
	$u \in E = \Vect(\mathcal{B})$ donc $\mathcal{B} \cup \{u\}$ n'est pas libre.
	Donc $\mathcal{B}\cup \{u\} \subset \mathcal{B}'$ et $\mathcal{B}'$ est libre donc $\mathcal{B}\cup \{u\}$ est libre: une contradiction $\lightning$
\end{prv}

\begin{lem}
	[Lemme d'échange] Soient $\mathcal{B}_1$ et $\mathcal{B}_2$ deux bases de $E$ et $u \in \mathcal{B}_1 \setminus \mathcal{B}_2$. Alors, il existe $v \in \mathcal{B}_2$ tel que $(\mathcal{B}_1 \setminus \{u\}) \cup \{v\}$ soit une base de $E$.
\end{lem}

\begin{prv}
	[1${}^\text{nde}$ méthode]
	On suppose que pout tout $v \in \mathcal{B}_2$, $(\mathcal{B}_1\setminus \{u\}) \cup \{v\}$ n'est pas une base de $E$
	Soit $v \in \mathcal{B}_2$.
	\begin{itemize}
		\item Supposons $(\mathcal{B}_1\setminus \{u\})\cup \{v\}$ non libre. $\mathcal{B}_1 \setminus \{u\}$ est libre. Donc $v \in \Vect(\mathcal{B}_1 \setminus \{u\})$
		\item Supposons $(\mathcal{B}_1\setminus \{u\}) \cup \{v\}$ non génératrice.
			Comme $\mathcal{B}_1$ engendre $E$, $u \not\in \Vect(\mathcal{B}_1\setminus \{v\})$.
			On suppose que $\mathcal{B}_1 \neq \mathcal{B}_2$.
			$\forall v \in \mathcal{B}_2 \setminus \mathcal{B}_1, \Vect(\mathcal{B}_1 \setminus \{v\}) = \Vect(\mathcal{B}_1) = E \ni u$ 
			donc, $(\mathcal{B}_1\setminus \{u\}) \cup \{v\}$ engendre $E$ et donc \[
				v \in \Vect(\mathcal{B}_1 \setminus \{u\})
			\] On a aussi \[
				\forall v \in \mathcal{B}_1 \setminus \{u\}, v \in \Vect(\mathcal{B}_1\setminus \{u\})
			\] Comme $u \not\in \mathcal{B}_2$, on a \[
				\forall v \in \mathcal{B}_2, v \in \Vect(\mathcal{B}_1\setminus \{u\})
			\] docn \[
				E = \Vect(\mathcal{B}_2) \subset \Vect(\mathcal{B}_1\setminus \{u\})
			\] donc $\mathcal{B}_1\setminus \{u\}$ engendre $E$ donc $\mathcal{B}_1\setminus \{u\}$ est une base de $E$. Or, $\mathcal{B}_1 \setminus \{u\}  \subset  \mathcal{B}_1$, donc $\mathcal{B}_1\setminus \{u\} = \mathcal{B}_1$
	\end{itemize}
\end{prv}

\begin{prv}
	[2${}^\text{nde}$ méthode]
	On suppose que pout tout $v \in \mathcal{B}_2$, $(\mathcal{B}_1\setminus \{u\}) \cup \{v\}$ n'est pas une base de $E$
	\begin{itemize}
		\item Comme $u \in \mathcal{B}_1 \setminus \mathcal{B}_2$, nécéssairement $\mathcal{B}_1 \neq \mathcal{B}_2$ donc $\mathcal{B}_2 \not\subset \mathcal{B}_1$, donc $\mathcal{B}_2\setminus\mathcal{B}_1 \neq \O$ 
		\item Soit $v \in \mathcal{B}_2\setminus\mathcal{B}_1$. Il existe $(\lambda_w)_{w\in\mathcal{B}_1}$ une famille de scalaires presque nulle telle que \[
				v = \sum_{w \in \mathcal{B}_1} \lambda_w w - \lambda_u u + + \sum_{w \in \mathcal{B}_1\setminus \{u\}}\lambda_w w
			\]
			Si $\lambda_u \neq 0_E$, alors
			\begin{align*}
				u &= \lambda_u^{-1}\left( v - \sum_{w \in \mathcal{B}_1 \setminus \{u\}} \lambda_w w \right)\\
					&\in \Vect(\mathcal{B}_1\setminus \{u\} \cup v)
			\end{align*}
			 donc $\mathcal{B}_1 \subset \Vect(\mathcal{B}_1\setminus \{u\} \cup \{v\})$\\
			 et donc $E \subset  \Vect(\mathcal{B}_1 \setminus \{u\} \cup \{v\})$ \\
			 et donc $\mathcal{B}_1 \setminus \{u\} \cup \{v\}$ engendre $E$ \\
			 donc $\mathcal{B}_1 \setminus \{u\} \cup \{v\}$ n'est pas libre\\
			 donc $v \in \Vect(\mathcal{B}_1\setminus \{u\})$ (car $\mathcal{B}_1 \setminus \{u\}$ est libre\\
			 donc $\lambda_u = 0_\mathbbm{K}$ $\lightning$\\`

			 Donc, $\lambda_u = 0_\mathbbm{K}$, docn $v \in \Vect(\mathcal{B}_1\setminus \{u\})$ \\
			 On vient de prouver que
			 \begin{align*}
			 	\mathcal{B}_2 \setminus \mathcal{B}_1 \subset \Vect(\mathcal{B}_1 \setminus \{u\})\\
			 	\mathcal{B}_1 \setminus \{u\} \subset \Vect(\mathcal{B}_1 \setminus \{u\})\\
			 \end{align*}
			 Comme $u \not\in \mathcal{B}_2$, \[
			 	\mathcal{B}_2 \subset \Vect(\mathcal{B}_1 \setminus \{u\})
			 \] donc \[
			 	E = \Vect(\mathcal{B}_2) \subset  \Vect(\mathcal{B}_1 \setminus \{u\})
			 \] donc $\mathcal{B}_1 \setminus \{u\}$ engendre $E$. Donc,  $\mathcal{B}_1 \setminus \{u\}$ est une base de $E$.\\
			 Or, $\mathcal{B}_1 \setminus \{u\} \subset  \mathcal{B}_1$, donc $\mathcal{B}_1 \setminus \{u\} = \mathcal{B}_1$
	\end{itemize}
\end{prv}

\begin{defn}
	Soit $E$ un $\mathbbm{K}$-espace vectoriel de dimension finie. Le cardinal commun à toutes les bases de $E$ est appelé \underline{dimension} de $E$ est notée $\dim(E)$ ou $\dim_\mathbbm{K}(E)$\\
	C'est donc aussi le nombre de coordonnées de n'importe quel vecteur dans n'importe quelle base.
	\index{dimension (espace vectoriel)}
\end{defn}

\begin{exm}
	\begin{enumerate}
		\item $\dim_\R(\C) = 2$ et $\dim_\C(\C) = 1$ 
		\item $\dim_\mathbbm{K}(\mathbbm{K}^{n}) = n$ 
		\item $\dim_{\mathbbm{K}}(\mathcal{M}_{n,p}(\mathbbm{K})) = np$
	\end{enumerate}
\end{exm}

\begin{crlr}
	Soit $E$ un $\mathbbm{K}$-espace vectoriel de dimension finie, $\mathcal{L}$ une famille libre de $E$, $\mathcal{G}$ une famille génératrice de $E$. On note $n = \dim(E)$
	\begin{enumerate}
		\item $\#\mathcal{G} \ge n$ et $(\#\mathcal{G} = n \implies \mathcal{G} \text{ est une base de } E$)
		\item $\#\mathcal{L} \le n$ et $(\#\mathcal{L} = n \implies \mathcal{L} \text{ est une base de } E$)
	\end{enumerate}
\end{crlr}

\begin{crlr}
	$\R^{\R}$ est de dimension infinie.
	$\forall i \in \N, e_i: x \mapsto x^i$\\
	$(e_i)_{i\in\N}$ est libre dans $\R^\R$
\end{crlr}

\begin{prop}
	Soient $E$ et $F$ deux $\mathbbm{K}$-espaces vectoriels de dimension finie. Alors $E\times F$ est de dimension finie et $\dim(E\times F) = \dim(E) + \dim(F)$
\end{prop}

\begin{prv}
	Soit $(e_1,\ldots, e_n)$ une base de $E$, $(f_1, \ldots, f_p)$ une base de $F$.
	On pose \[
		\left\{\begin{array}
			{r c l}
			u_1 &=& (e_1,0_F)\\
			u_2 &=& (e_2,0_F)\\
					&\vdots&\\
			u_n &=& (e_n,0_F)\\
			u_{n+1} &=& (0_E, f_1)\\
			u_{n+2} &=& (0_E, f_2)\\
					&\vdots&\\
			u_{n+p} &=& (0_E,f_p)\\
		\end{array}\right.
	\]
	Soit $(x,y) \in E\times F$. \[
		\begin{cases}
			\exists (x_1,\ldots,x_n)\in \mathbbm{K}^n, x = \sum_{i=1}^{n} x_ie_i
			\exists (y_1,\ldots,y_n)\in \mathbbm{K}^n, x = \sum_{j=1}^{p} y_jf_j
		\end{cases}
	\] 
	\begin{align*}
		(x,y) &= \left( \sum_{i=1}^{n} x_ie_i, \sum_{i=1}^{p} y_jf_j \right)  \\
		&= \sum_{i=1}^{n} x_i (e_i + 0_F) + \sum_{j=1}^{p} y_j (0_E, f_j) \\
		&= \sum_{i=1}^{n} x_i u_i + \sum_{j=1}^{p} y_j u_{n+j} \\
	\end{align*}
	Donc, $E\times F = \Vect(u_1, \ldots, u_{n+p})$ donc $E\times F$ est de dimension finie.\\
	Soit $(\lambda_1, \ldots, \lambda_{n+p}) \in \mathbbm{K}^{n+p}$ tel que \[
		(*): \quad \sum_{k=1}^{n+p} \lambda_ku_k = 0_{E\times F} = (0_E, 0_F)
	\]
	\begin{align*}
		(*) &\iff \sum_{k=1}^{n} \lambda_k (e_k, 0_F) + \sum_{k=n+1}^{p} \lambda_k(0_E, f_{k-n}) = (0_E, 0_F)\\
				&\iff \begin{cases}
					\sum_{k=1}^{n} \lambda_k e_k = 0_E\\
					\sum_{k=n+1}^{p} \lambda_k f_{k-n} = 0_F
				\end{cases}\\
				&\iff \begin{cases}
					\forall k \in \left\llbracket 1,n \right\rrbracket, \lambda_k = 0_\mathbbm{K} \qquad&(\text{car $(e_1,\ldots,e_n)$ est libre})\\
					\forall k \in \left\llbracket n+1,n+p \right\rrbracket, \lambda_k = 0_\mathbbm{K} \qquad&(\text{car $(f_1,\ldots,f_n)$ est libre})\\
				\end{cases}
	\end{align*}
	Donc $(u_1, \ldots, u_{n+p})$ est une base de $E\times F$. Donc, $\dim(E\times F) = n + p = \dim(E) + \dim(F)$
\end{prv}

\begin{rmk}
	[Convention]
	\[\dim\big(\{0_E\}\big) = 0\]
\end{rmk}

\begin{thm}
	Soit $E$ un $\mathbbm{K}$-espace vectoriel de dimension finie, $F$ un sous-espace vectoriel de $E$. Alors, $F$ est de dimension finie et  $\dim(F) \le \dim(E)$\\
	Si $\dim(F) = \dim(E)$, alors $F = E$
\end{thm}

\begin{prv}
	On considère \[
		A = \{k \in \N \mid \text{il existe une famille libre de $F$ à $k$ éléments}\} 
	\]
	On suppose $F \neq \{0_E\}$.
	\begin{itemize}
		\item Soit $u \in F\setminus \{0_E\}$. $(u)$ est libre donc $1 \in A$ et donc $A \neq \O$
		\item Soit $\mathcal{L}$ une famille libre de $F$. Alors, $\mathcal{L}$ est une famille libre de $E$ \\
			donc $\#\mathcal{L} \le \dim(E)$\\
			Donc $A$ est majorée par $\dim(E)$ \\
			On en déduit que $A$ a un plus grand élément $p$.
		\item Soit $\mathcal{L}$ une famille libre de $F$ avec $p$ éléments.\\
			Si $\mathcal{L}$ n'engendre pas $F$, alors il existe $u\in F$ tel que $u\not\in \Vect(\mathcal{L})$ et donc $\mathcal{L} \cup \{u\}$ est une famille libre de $F$, donc $p+1 \in A$ en contradiction avec la maximalité de $p$.\\
			Donc $\mathcal{L}$ est une base de $F$ donc $F$ est de dimension finie et $\dim(F) = p \le \dim(E)$\\
	\end{itemize}

	Soit $\mathcal{B}$ une base de $F$. Alors, $\mathcal{B}$ est aussi une famille de libre de de $E$. Donc $\#\mathcal{B} \le \dim(E)$ donc $\dim(F) = \dim(E)$ \\
	Si $\dim(F) = \dim(E)$, alors $\mathcal{B}$ est une base de $E$, et donc $F = \Vect(\mathcal{B}) = E$
\end{prv}

\begin{prop}
	[Formule de Grassmann]
	Soit $E$ un $\mathbbm{K}$-espace vectoriel de dimension finie, $F$ et $G$ deux sous-espace vectoriels de $E$. Alors, \[
		\dim(F+G) = \dim(F) + \dim(G) - \dim(F\cap G)
	\] 
\end{prop}

\begin{prv}
	Soit $(e_1, \ldots, e_p)$ une base de $F\cap G$. $(e_1,\ldots,e_p)$ est une famille libre de $F$.\\
	On complète $(e_1, \ldots, e_p)$ en une base $(e_1, \ldots, e_p, u_1, \ldots, u_q)$ de $F$.\\
	De même, on complète $(e_1, \ldots, e_p)$ en une base $(e_1, \ldots, e_p, v_1, \ldots, v_r)$ de $G$.\\
	On pose  $\mathcal{B} = (e_1, \ldots, e_p, u_1, \ldots, u_q, v_1, \ldots, v_r)$. Montrons que $\mathcal{B}$ est une base de $F+G$
	\begin{itemize}
		\item Soit $u \in F+G$ \\
			On pose $u = v+w$ avec $\begin{cases}
				v\in F\\
				w \in G
			\end{cases}$.\\
			On pose $v = \sum_{i=1}^p \lambda_i e_i + \sum_{i=1}^q \mu_i u_i$ avec $(\lambda_1, \ldots, \lambda_p, \mu_1, \ldots, \lambda_q) \in \mathbbm{K}^{p+q}$\\
			On pose aussi $w = \sum_{i = 1}^p \lambda'_ie_i + \sum_{j=1}^r \nu_j v_j$ avec $(\lambda_1',\ldots,\lambda_p', \nu_1, \ldots, \nu_r) \in \mathbbm{K}^{p+r}$\\
			D'où, \[
				u = \sum_{i=1}^p (\lambda_i + \lambda'_i)e_i + \sum_{j=1}^q \mu_j u_j + \sum_{k=1}^r \nu_k v_k \in \Vect(\mathcal{B})
			\]
		\item Soient $(\lambda_1, \ldots, \lambda_p, \mu_1, \ldots, \mu_q, \nu_1, \ldots, \nu_r) \in \mathbbm{K}^{p+q+r}$.\\
			On suppose \[
				(*)\quad \sum_{i=1}^{p}\lambda_ie_i + \sum_{j=1}^q\mu_ju_j + \sum_{k=1}^r \nu_k v_k = 0_E
			\] 
			D'où, \[
				\underbrace{\sum_{i=1}^p\lambda_i e_i + \sum_{j=1}^q \mu_ju_j}_{\in F} = \underbrace{-\sum_{k=1}^r\nu_jv_k}_{\in G}
			\] 
			Donc, \[
				f = \sum_{i=1}^p \lambda_i e_i + \sum_{j=1}^q \mu_j u_j \in F\cap G
			\] Comme $(e_1, \ldots, e_p)$ est une base de $F\cap G$, $\exists ! (\lambda_1', \ldots, \lambda_p') \in \mathbbm{K}^p$ tel que \[
				f = \sum_{i=1}^p \lambda'_i e_i = \sum_{i=1}^p \lambda'_i e_i + \sum_{j=1}^q 0_\mathbbm{K}u_j
			\] Comme $(e_1, \ldots, e_p, u_1, \ldots, u_q)$ est une base de $F$, \[
				\forall k \in \left\llbracket 1, q \right\rrbracket, \mu_j = 0_\mathbbm{K}
			\] De même, \[
				\forall k \in \left\llbracket 1,r \right\rrbracket , \nu_k = 0_\mathbbm{K}
			\] On remplace dans $(*)$ pour trouver \[
				\sum_{i=1}^p \lambda_ie_i = 0_E
			\] Comme $(e_1, \ldots, e_p)$ est libre, \[
				\forall i \in \left\llbracket 1,p \right\rrbracket, \lambda_i = 0_\mathbbm{K}
			\] Donc $\mathcal{B}$ est libre.\\
			Donc, 
			\begin{align*}
				\dim(F+G) &=  p +q + r \\
				&= (p+q)+ (p+r) - p \\
				&= \dim(F) + \dim(G) - \dim(F\cap G) \\
			\end{align*}
	\end{itemize}
\end{prv}

\begin{crlr}
	Avec les hypothèse précédentes, \[
		E = F \oplus G \iff \begin{cases}
			F \cap  G = \{0_E\} \\
			\dim(E) = \dim(F) + \dim(G)
		\end{cases}
	\] 
\end{crlr}

\begin{prv}
	\begin{itemize}
		\item[``$\implies$''] On suppose $E = F \oplus G$ \\
			Comme la somme est directe, $F \cap G = \{0_E\}$ 
			\begin{align*}
				\dim(E) &= \dim(F)\\
				&= \dim(F) + \dim(G) - \dim(F\cap G)\\
				&= \dim(F) + \dim(G)\\
			\end{align*}
		\item[``$\impliedby$''] On suppose $F\cap G = \{0_E\}$ et $\dim(E) = \dim(F) + \dim(G)$.\\
			On sait déjà que $F+G = F \oplus G$\\
			 \begin{align*}
				\dim(F+G) = \dim(F) + \dim(G) - \dim(F \cap G) = \dim(E)
			\end{align*}
			Donc $F + G = E$
	\end{itemize}
\end{prv}

\begin{prop}
	Soit $F$ un $\mathbbm{K}$-espace vectoriel de dimension finie $n$. Soit $\mathcal{B} = (e_1, \ldots, e_n)$ une base de $F$. L'application
	\begin{align*}
		f: \mathbbm{K}^n &\longrightarrow F \\
		(\lambda_1, \ldots, \lambda_n) &\longmapsto \sum_{i=1}^n \lambda_i e_i
	\end{align*} est bijective.\\
	Si $\mathbbm{K}$ est infini, $\mathbbm{K}^n$ aussi et donc $F$ aussi.\\
	Si $\#\mathbbm{K} = p \in \N_*$,
	\begin{align*}
		\#&\mathbbm{K}^n = p^n\\
		&\vrt=\\
		\#&F
	\end{align*}
\end{prop}


		\part{Dérivation}

\underline{Motivation}:

{
\begin{wrapfigure}{l}{3cm}
	\centering
	\begin{asy}
		import three;

		size(3cm);
		settings.render=0;
		settings.prc=false;
		currentprojection = obliqueZ;

		draw(unitbox);
		draw(shift(1.1Z + 0.05X) * (O -- X), Arrows3(TeXHead2));
		draw(shift(1.1Z + 0.05Y) * (O -- Y), Arrows3(TeXHead2));
		draw(shift(1.1X + 0.05Z) * (O -- Z), Arrows3(TeXHead2));

		label("$x$", (X/2) + (1.1Z + 0.05X), align=S);
		label("$y$", (Y/2) + (1.1Z + 0.05Y), align=W);
		label("$z$", (Z/2) + X, align=SE);
	\end{asy}
\end{wrapfigure}

\begin{align*}
	&S(x,y,z) = 2(xy + xz + yz)\\
	&V(x,y,z) = xyz
\end{align*}

On cherche à minimiser $S$ avec la contrainte $V = 1$.

Soit $f : \begin{array}{rcl}
	\left( \R_*^+ \right)^2 &\longrightarrow& \R \\
	(x,y) &\longmapsto& S\left( x,y,\frac{1}{xy} \right) = 2\left( xy + \frac{1}{y} + \frac{1}{x} \right).
\end{array}$

On cherche $(a,b) \in \left( \R^+_* \right)^2$ tel que \[
	\forall (x,y) \in (\R^+_*), f(x,y) \ge f(a,b).
\]
}

\begin{defn}
	Soit $f: U \to \R$ où $U$ est un ouvert de $\R^2$. Soit $(a,b) \in U$.
	\vspace{2mm}

	Si $\lim_{x \to a} \frac{f(x,b) - f(a,b)}{x - a} \in \R$, alors on dit que $f$ a une dérivée partielle suivant $x$ en $(a,b)$ et cette limite est notée \[
		\partial f_1(a,b) = \frac{\partial f}{\partial x}(a,b).
	\]

	Si $\lim_{y \to b} \frac{f(a,y) - f(a,b)}{y - b} \in \R$, alors on dit que $f$ a une dérivée partielle suivant $y$ et la limite est notée \[
		\partial f_2(a,b) = \frac{\partial f}{\partial y}(a,b).
	\]
\end{defn}

\begin{exm}
	\begin{enumerate}
		\item $f: (x,y) \mapsto xy + x - y$.

			\begin{align*}
				&\frac{\partial f}{\partial x} : (x,y) \mapsto y + 1,\\
				&\frac{\partial f}{\partial y} : (x,y) \mapsto x - 1.
			\end{align*}

		\item $f: (x,y) \mapsto xy + \frac{1}{y}+ \frac{1}{x}$.

			\begin{align*}
				&\frac{\partial f}{\partial x}: (x,y) \mapsto y - \frac{1}{x^2},\\
				&\frac{\partial f}{\partial y}: (x,y) \mapsto x - \frac{1}{y^2}.
			\end{align*}

		\item Trouver $f$ telle que $\begin{cases}
				(1): \qquad \frac{\partial f}{\partial x}=y,\\[2mm]
				(2): \qquad \frac{\partial f}{\partial y} = x.
			\end{cases}$

			D'après $(1)$ : \[
				\forall (x,y), \exists C(y) \in \R, f(x,y) = xy + C(y)
			\] et donc \[
				\frac{\partial f}{\partial y}(x,y) = x + C'(y)
			\] donc $C'(y) = 0$ et donc $C$ est constante.

		\item Trouver $f$ telle que $\begin{cases}
			\frac{\partial f}{\partial x} = -y,\\[2mm]
			\frac{\partial f}{ƒ\partial y} = x.
		\end{cases}$

		Ce n'est pas possible !
	\end{enumerate}
\end{exm}

\begin{defn}~\\
	\begin{minipage}{\linewidth}
		\begin{wrapfigure}{r}{4cm}
			\centering
			\vspace{-5mm}
			\begin{asy}
				import three;
				import graph3;
				size(4cm);

				settings.render = 0;
				settings.prc = false;
				currentprojection = obliqueX;

				draw(O -- X, Arrow3(TeXHead2));
				draw(O -- Y, Arrow3(TeXHead2));
				draw(O -- Z, Arrow3(TeXHead2));

				triple f(real x, real y, real z = 0) { return (x,y,cos(x - 0.5) * cos(y - 0.5)/1.2 + 0.15); }

				real inc = 1 / 5;

				for(real x = 0; x <= 1; x += inc) {
					draw(graph(
						new real(real t) { return x; }, // x
						new real(real y) { return y; }, // y
						new real(real y) { return f(x,y).z; }, // z
						0, 1
					), gray);
				}

				for(real y = 0; y <= 1; y += inc) {
					draw(graph(
						new real(real x) { return x; }, // x
						new real(real t) { return y; }, // y
						new real(real x) { return f(x,y).z; }, // z
						0, 1
					), gray);
				}

				path3 path1 = (0.8, 0.2, 0) .. (0.5, 0.5, 0) .. (0.3, 0.7, 0);
				path3 path2 = f(0.8, 0.2, 0) .. f(0.5, 0.5, 0) .. f(0.3, 0.7, 0);
				path3 d = (0.2, 0.3, 0) .. (0.3, 0.4, 0) .. (0.2, 0.7, 0) .. (0.8, 0.9, 0) .. (0.6, 0.2, 0) .. cycle;

				draw(path1, red, Arrow3(TeXHead2));
				draw(path2, red, Arrow3(TeXHead2, position=0.8));

				dot((0.5, 0.5, 0));
				dot(f(0.5, 0.5, 0));
				draw((0.5, 0.5, 0) -- f(0.5, 0.5, 0), dashed);
				draw(d);

				label("$w$", (0.3, 0.7, 0), red, align=SE);
				label("$U$", (0.8, 0.9, 0), align=SE);
			\end{asy}
		\end{wrapfigure}

		Soit $f: U \to \R$ où $U$ est un ouvert. Soit $(a,b) \in U$. Soit $w = (w_1, w_2) \in \R^2$.

		Si 
		\[
			\lim_{t\to 0} \frac{f(a + tw_1, b + tw_2) - f(a,b)}{t}
		\] existe et est réelle, alors on dit que $f$ a une dérivée dans la direction de $w$ et la limite est notée \[
			\mathrm{d}f(w)\,(a,b) = D_w(f)\,(a,b).
		\]
	\end{minipage}
\end{defn}

\begin{exm}
	\begin{align*}
		f: \left( \R_*^+ \right)^2 &\longrightarrow \R \\
		(x,y) &\longmapsto xy+\frac{1}{x}+\frac{1}{y}.
	\end{align*}

	On pose $(a,b) = (1,2)$, $w = (w_1, w_2) = (1,1)$.
	\begin{align*}
		\frac{f(1+t, 2+t) - f(1,2)}{t} &= \frac{1}{t} \left( (1+t)(2+t) + \frac{1}{1+t} + \frac{1}{2+t} - 3 - \frac{1}{2} \right) \\
		&= \frac{1}{t}\left(\cancel 2 + 3t + \po(t) + \cancel 1 - t + \po(t) + \frac{1}{2}\left( \cancel 1 - \frac{t}{2} + \po(t) \right) - \cancel3 - \cancel{\frac{1}{2}} \right) \\
		&= \frac{1}{t} \left( \frac{7}{4} t + \po(t) \right)  \\
		&= \frac{7}{4} + \po(1) \tendsto{t \to 0} \frac{7}{4}. \\
	\end{align*}

	Donc, \[
		\mathrm{d}f(1,1)\,(1,2) = \frac{7}{4}.
	\]
\end{exm}

\begin{rmk}~\\
	\begin{figure}[H]
		\centering
		\begin{asy}
			import solids;
			import graph;
			size(5cm);

			settings.render = 0;
			settings.prc = false;

			path3 par = graph(
				new real(real x) { return x; },
				new real(real x) { return 0; },
				new real(real x) { return x^2; },
				0,3);
			revolution r = revolution(par, axis=Z);

			path3 par2 = graph(
				new real(real x) { return x; },
				new real(real x) { return 0; },
				new real(real x) { return x^2; },
				-3,3);

			draw(r,1,longitudinalpen=nullpen);
			draw(r.silhouette());

			draw((-4, 0, -1) -- (-4, 0, 10) -- (4, 0, 10) -- (4, 0, -1) -- cycle, red);
			draw(par2, deepred);

			draw((4,4.5) -- (7, 4.5), black+0.5mm, Arrow(TeXHead));

			path par2d = graph(new real(real x) { return x^2; }, -3, 3);
			draw(shift((11, 0)) * par2d, deepred);

			dot(O);
			dot((11, 0));
		\end{asy}
	\end{figure}
\end{rmk}


%todo ajouter théorème-définition
\begin{thm}
	Soit $f : U \to \R$, $(a,b) \in U$. On suppose que $\frac{\partial f}{\partial x}$ et $\frac{\partial f}{\partial y}$ existent en $(a,b)$ et sont {\bfseries continues} en $(a,b)$. Alors,
	\begin{align*}
		&\forall (h, k) \in \R^2 \text{ tel que } (a +h, b + k) \in U,\\
		&f(a+ h, b + k) = f(a,b) + h \frac{\partial f}{\partial x}(a,b) + k \frac{\partial f}{\partial y}(a,b) + \po_{(h,k)\to (0,0)}\big(\|(h,k)\|\big).
	\end{align*}

	On dit que $f$ est de classe $\mathcal{C}^1$ si $\frac{\partial f}{\partial x}$ et $\frac{\partial f}{\partial y}$ existent et sont continues.

	\qed
\end{thm}

\begin{rmk}
	En physique, cette formule correspond à : \[
		\mathrm{d}f = \frac{\partial f}{\partial x}\mathrm{d}x + \frac{\partial f}{\partial y} \mathrm{d}y.
	\] En effet :
	\begin{align*}
		\mathrm{d}f &= f(x+ \mathrm{d}x, y + \mathrm{d}y) - f(x,y) \\
		&= \frac{\partial f}{\partial x} \mathrm{d}x + \frac{\partial f}{\partial y} \mathrm{d}y.
	\end{align*}
\end{rmk}

\begin{prop}
	Soit $f: U \to \R$ de classe $\mathcal{C}^1$ en $(a,b) \in U$. Alors, \[
		\forall w = (w_1, w_2) \in \R^2, \mathrm{d}f(w)\,(a,b) = w_1 \frac{\partial f}{\partial x}(a,b) + w_2 \frac{\partial f}{\partial y}(a,b).
	\]
\end{prop}

\begin{prv}
	Soit $w = (w_1, w_2) \in \R^2$. Soit $t \in \R^*$.
	\begin{align*}
		\frac{1}{t}\big(f(a + tw_1, b + tw_2) - f(a,b)\big)
		&= \frac{1}{t} \left( tw_1 \frac{\partial f}{\partial x}(a,b) + tw_2 \frac{\partial f}{\partial y}(a,b) + \po_{t \to 0}\big(\|tw\|\big) \right) \\
		&= w_1 \frac{\partial f}{\partial x}(a,b) + w_2 \frac{\partial f}{\partial y}(a,b) + \po_{t\to 0}(1) \\
		&\tendsto{t\to 0} w_1 \frac{\partial f}{\partial x}(a,b) + w_2\frac{\partial f}{\partial y}(a,b).
	\end{align*}
\end{prv}


\begin{defn}
	Avec les hypothèses précédentes, en posant \[
		\nabla f(a,b) = \left( \frac{\partial f}{\partial x}(a,b), \frac{\partial f}{\partial y}(a,b) \right) 
	\]on obtient \[
		\mathrm{d}f(w)\,(a,b) = \left<w  \mid \nabla f(a,b) \right>
	\] où $\left<\cdot|\cdot \right>$ est le produit scalaire.

	Le vecteur $\nabla f(a,b)$ est appelé \underline{gradient de $f$ en $(a,b)$}.

	Le développement limité à l'ordre 1 de $f$ devient \[
		f\big((a,b)+w\big) = f(a,b) + \left<w \mid \nabla f(a,b) \right> + \po_{w\to 0}(\|w\|)
	\]
\end{defn}

\begin{prop}
	Soit $f : U \to \R$ de classe $\mathcal{C}^1$.

	\begin{figure}[H]
    \centering
    \incfig{gradient}
	\end{figure}

	$\nabla f$ est orthogonal au lignes de niveaux de $f$, son orientation va dans le sens d'une augmentation de $f$.
\end{prop}

\begin{prv}
	Soit $\gamma : I \to U$ une courbe de niveau : \[
		\forall t \in I, f\big(\gamma(t)\big) = \text{cste}.
	\] D'après le lemme suivant : \[
		\forall t \in I, 0 = (f \circ \gamma)'(t) = \mathrm{d}f\big(\gamma'(t)\big)\big(\gamma(t)\big) = \left<\gamma'(t)  \mid \nabla f\big(\gamma(t)\big) \right>
	\] Donc $\nabla f\big(\gamma(t)\big)$ est orthogonal à $\gamma'(t)$.

	Pour tout $t \in I$, on pose $w(t) = t\, \nabla f\big(\gamma(t)\big)$. Donc \[
		f\big(\gamma(t) + w(t)\big) = f\big(\gamma(t)\big) + t \|\nabla f(\gamma(t))\|^2 + \po_{t \to 0}(t)
	\] Pour $t$ assez petit, $f\big(\gamma(t) + w(t)\big) - f\big(\gamma(t)\big)$ est du même signe que $t$.
\end{prv}

\begin{rmk}
	\begin{align*}
		V: \R^3 &\longrightarrow \R \\
		(x,y,z) &\longmapsto -mgz
	\end{align*}
	l'énerge potentielle de pesenteur

	On a donc \[
		\nabla V(x,y,z) = \left( \frac{\partial V}{\partial x}, \frac{\partial V}{\partial y}, \frac{\partial V}{\partial z} \right) = (0, 0, -mg) = \vec{P}.
	\]
\end{rmk}

\begin{lem}
	Soit $f : U \to \R$ de classe $\mathcal{C}^1$, $\gamma : \begin{array}{rcl}
		I &\longrightarrow& U \\
		t &\longmapsto& \big(x(t), y(t)\big)
	\end{array}$ où $x$ et $y$ sont dérivables.

	On pose \[
		\forall t \in I, \gamma'(t) = \big(x'(t), y'(t)\big).
	\] Alors $f \circ \gamma : I \to \R$ est dérivable et
	\begin{align*}
		\forall t \in I, (f \circ \gamma)'(t) &= \mathrm{d}f\big(\gamma'(t)\big) \big(\gamma(t)\big)\\
		&= \left<\gamma'(t)  \mid \nabla f\big(\gamma(t)\big)  \right> \\
		&= x'(t) \frac{\partial f}{\partial x}\big(x(t), y(t)\big) + y'(t) \frac{\partial f}{\partial y}\big(x(t),y(t)\big). \\
	\end{align*}
\end{lem}

\begin{prv}
	On fixe $t \in I$.

	\begin{align*}
		\forall h \neq 0, \frac{f \circ \gamma(t + h) - f \circ \gamma(t)}{h}
		&= \frac{1}{h}\big(f(\gamma(t)) + h\gamma'(t) + \po_{h\to 0}(h) - f(\gamma(t))\big) \\
		&= \frac{1}{h}\bigg(\cancel{f(\gamma(t))} + \left<h\gamma'(t) \mid \nabla f(\gamma(t)) \right> + \po_{h\to 0}(\|h\gamma'(t)\|) - \cancel{f(\gamma(t))}\bigg)\\
		&= \left<\gamma'(t) \mid \nabla f(\gamma(t)) \right> + \po_{h\to 0}(1) \\
		&\tendsto{h\to 0} \left<\gamma'(t)  \mid \nabla f(\gamma(t)) \right>
	\end{align*}
\end{prv}

\begin{defn}
	Soit $f : U \to \R$ de classe $\mathcal{C}^1$ et $(a,b) \in U$. On dit que $(a,b)$ est un \underline{point critique} de $f$ si $\nabla f(a,b) = 0$ i.e. $\frac{\partial f}{\partial x}(a,b) = \frac{\partial f}{\partial y}(a,b) = 0$.

	Dans ce cas, $f(a,b)$ est appelé \underline{valeur critique} de $f$.
\end{defn}

\begin{prop}~\\
	\begin{minipage}{\linewidth}
		\begin{wrapfigure}{r}{3cm}
			\centering
			\vspace{-1cm}
			\begin{asy}
				import solids;
				import graph;
				size(3cm);

				settings.render = 0;
				settings.prc = false;

				path3 par = graph(
					new real(real x) { return x; },
					new real(real x) { return 0; },
					new real(real x) { return -x^2; },
					0,3);
				revolution r = revolution(par, axis=Z);

				draw(r,1,longitudinalpen=nullpen);
				draw(r.silhouette());

				dot("$(a,b)$", O, red, align=N);
				real s = sqrt(2.5);
				path3 g=(s,0,-2.5)..(0,s,-2.5)..(-s,0,-2.5)..(0,-s,-2.5)..cycle;
				draw(g, deepcyan);
			\end{asy}
		\end{wrapfigure}
		Soit $f: U \to \R$ de classe $\mathcal{C}^1$ et $(a,b) \in U$ tel que \[
			\exists r > 0, \forall (x,y) \in B_{(a,b)}(r), f(x,y) \le f(a,b)
		\] Alors $\nabla f(a,b) = (0,0)$.
	\end{minipage}
\end{prop}

\begin{prv}
	Soit $g: x \mapsto f(x,b)$. $g(a)$ est un maximum local de $g$ donc $g'(a) = 0$.

	Or, $g'(a) = \frac{\partial f}{\partial x}(a,b)$

	donc $\frac{\partial f}{\partial x}(a,b) = 0$.

	Soit $h : y \mapsto f(a,y)$. On a de même $h'(b) = 0$.

	Or, $h'(b) = \frac{\partial f}{\partial y}(a,b)$.

	Donc, $\nabla f(a,b) = (0,0)$.
\end{prv}

\begin{rmk}
	Un minimum local est aussi une valeur critique.
\end{rmk}

\begin{figure}[H]
	\centering
	\begin{subfigure}{3cm}
		\centering
		\begin{asy}
			import solids;
			import graph;
			size(3cm);

			settings.render = 0;
			settings.prc = false;

			path3 par = graph(
				new real(real x) { return x; },
				new real(real x) { return 0; },
				new real(real x) { return -x^2; },
				0,3);
			revolution r = revolution(par, axis=Z);

			draw(r,1,longitudinalpen=nullpen);
			draw(r.silhouette());

			dot(O, red);
		\end{asy}
		\caption{Maximum local}
	\end{subfigure}
	\begin{subfigure}{3cm}
		\centering
		\begin{asy}
			import solids;
			import graph;
			size(3cm);

			settings.render = 0;
			settings.prc = false;

			path3 par = graph(
				new real(real x) { return x; },
				new real(real x) { return 0; },
				new real(real x) { return x^2; },
				0,3);
			revolution r = revolution(par, axis=Z);

			draw(r,1,longitudinalpen=nullpen);
			draw(r.silhouette());

			dot(O, red);
		\end{asy}
		\caption{Minimum local}
	\end{subfigure}
	\begin{subfigure}{3cm}
		\centering
		\begin{asy}
			import solids;
			import graph;
			size(3cm);

			settings.render = 0;
			settings.prc = false;
			currentprojection = obliqueZ;

			draw(graph(
				new real(real x) { return x; },
				new real(real x) { return -x^2 / 3; },
				new real(real x) { return 3; },
				-3, 3
			));

			draw(graph(
				new real(real x) { return x; },
				new real(real x) { return -x^2 / 3; },
				new real(real x) { return -3; },
				-3, 3
			));

			draw(graph(
				new real(real x) { return x; },
				new real(real x) { return -x^2 / 3 - 1; },
				new real(real x) { return 0; },
				-3, 3
			));

			draw(graph(
				new real(real x) { return 0; },
				new real(real x) { return x^2 / 9 - 1; },
				new real(real x) { return x; },
				-3, 3
			));

			draw(graph(
				new real(real x) { return -3; },
				new real(real x) { return x^2 / 9 - 4; },
				new real(real x) { return x; },
				-3, 3
			));

			draw(graph(
				new real(real x) { return 3; },
				new real(real x) { return x^2 / 9 - 4; },
				new real(real x) { return x; },
				-3, 3
			));

			dot((0,-1,0), red);
		\end{asy}
		\caption{Point de selle / Point col}
	\end{subfigure}
\end{figure}

\begin{exm}
	On revient à l'exemple donné en introduction : 
	\begin{align*}
		f: \left( \R^*_+ \right)^2 &\longrightarrow \R \\
		(x,y) &\longmapsto 2\left( xy + \frac{1}{x} + \frac{1}{y} \right).
	\end{align*}

	$\left( \R^+_* \right)^2$ est un ouvert de $\R^2$. Soit $(x,y) \in \left( \R^+_* \right)^2$.
	
	On a \[
		\begin{cases}
			\frac{\partial f}{\partial x}(x,y) = 2\left( y - \frac{1}{x^2} \right),\\
			\frac{\partial f}{\partial y}(x,y) = 2\left( x - \frac{1}{y^2} \right).
		\end{cases}
	\]

	\begin{align*}
		&\frac{\partial f}{\partial x}(x,y) = \frac{\partial f}{\partial y}(x,y) = 0\\
		\iff& \begin{cases}
			y = \frac{1}{x^2}\\
			x = \frac{1}{y^2}
		\end{cases}\\
		\iff& \begin{cases}
			y = \frac{1}{x^2}\\
			x = x^4
		\end{cases}\\
		\iff& \begin{cases}
			x = 1\\
			y = 1
		\end{cases}
	\end{align*}

	On vérivie que $f$ présente en effet un minium local en $(1,1)$. \[
		f(1,1) = 6
	\] On fixe $y \in \R^+_*$ et \[
		g : x \mapsto 2\left( xy + \frac{1}{x} + \frac{1}{y} \right).
	\] Donc \[
		\forall x \in \R^+_*, g'(x) = 2\left( y - \frac{1}{x^2} \right).
	\]
	\begin{center}
		\begin{tikzpicture}
			\tkzTabInit{$x$/1,$g'(x)$/1,$g$/2.3}{$0$, $\frac{1}{\sqrt{y}}$, $+\infty$}
			\tkzTabLine{,-,z,+,}
			\tkzTabVar{+/{}, -/$2\left( 2\sqrt{y} +\frac{1}{y} \right)$, +/{}}
		\end{tikzpicture}
	\end{center}
	
	Ainsi, \[
		\forall x \in \R^+_*, \forall y \in \R^+_*, f(x,y) \ge 2\left( 2\sqrt{y} + \frac{1}{y} \right)
	\] Soit $h : y \mapsto 2\sqrt{y} + \frac{1}{y}$. On a \[
		\forall y > 0, h'(y) = \frac{1}{\sqrt{y}} - \frac{1}{y^2} = \frac{y\sqrt{y} - 1}{y^2} = \frac{y^{\frac{3}{2}} - 1}{y^2}
	\]

	\begin{center}
		\begin{tikzpicture}
			\tkzTabInit{$y$/0.7,$h'(y)$/0.7,$h$/1.4}{$0$, $1$, $+\infty$}
			\tkzTabLine{,-,z,+,}
			\tkzTabVar{+/{}, -/$3$, +/{}}
		\end{tikzpicture}
	\end{center}

	Donc, \[
		\forall x,y > 0, f(x,y) \ge 2\times 3 = 6 = f(1,1).
	\]
\end{exm}

\begin{prop}
	[règle de la chaîne]

	Soit $f : \begin{array}{rcl}
		U &\longrightarrow& \R^2 \\
		(x,y) &\longmapsto& f(x,y)
	\end{array}$ de classe $\mathcal{C}^1$ et $U, V$ deux ouverts de $\R^2$.

	Soit $\varphi : \begin{array}{rcl}
		V &\longrightarrow& U \\
		(u,v) &\longmapsto& \varphi(u,v) = \big(x(u,v), y(u,v)\big)
	\end{array}$.

	On suppose que $x$ et $y$ sont de classe $\mathcal{C}^1$ sur $V$.

	Alors,  $f \circ \varphi : \begin{array}{rcl}
		V &\longrightarrow& \R \\
		(u,v) &\longmapsto& f\big(\varphi(u,v)\big)
	\end{array}$ est de classe $\mathcal{C}^1$ et
	\begin{align*}
		\forall (u_0, v_0) \in V, \frac{\partial (f \circ \varphi)}{\partial u}(u_0, v_0)
		&= \frac{\partial f}{\partial x}\big(\varphi(u_0, v_0)\big) \times \frac{\partial x}{\partial u}(u_0, v_0)\\
		&+ \frac{\partial f}{\partial y}\big(\varphi(u_0,v_0)\big) \frac{\partial y}{\partial u}(u_0,v_0)
	\end{align*}
	\begin{align*}
		\forall (u_0, v_0) \in V, \frac{\partial (f \circ \varphi)}{\partial v}(u_0, v_0)
		&= \frac{\partial f}{\partial x}\big(\varphi(u_0, v_0)\big) \times \frac{\partial x}{\partial v}(u_0, v_0)\\
		&+ \frac{\partial f}{\partial y}\big(\varphi(u_0,v_0)\big) \frac{\partial y}{\partial v}(u_0,v_0)
	\end{align*}
\end{prop}

\begin{exm}
	[changement de coordonnées polaires]
	On pose \begin{align*}
		\varphi: \R^+_* \times ]0,2\pi[ &\longrightarrow \R^2\setminus \left( R^+_* \times \{0\} \right) \\
		(r, \theta) &\longmapsto (r \cos \theta, r \sin\theta),
	\end{align*}
	\begin{align*}
		f: \R^2\setminus \left( R^+_* \times \{0\} \right) &\longrightarrow \R \\
		(x,y) &\longmapsto f(x,y),
	\end{align*}
	\begin{align*}
		g: \overbrace{\R^+_* \times ]0, 2\pi[}^{=V} &\longrightarrow \R \\
		(r, \theta) &\longmapsto f(r\cos\theta, r\sin\theta).
	\end{align*}

	\begin{align*}
		\forall (r_0,\theta_0) \in V,&\\[5mm]
		\frac{\partial g}{\partial r}(r_0, \theta_0) &= \frac{\partial f}{\partial x}(r_0\cos\theta_0, r_0\sin\theta_0)\cos\theta_0\\
		&+ \frac{\partial f}{\partial y}(r_0 \cos\theta_0, r_0\sin\theta_0)\sin\theta_0\\
		&= 2r_0\cos^2\theta_0 + 2r_0\sin^2(\theta_0) \\
		&= 2r_0 \\[5mm]
		\frac{\partial g}{\partial \theta}(r_0, \theta_0) &= \frac{\partial f}{\partial x}(r_0\cos\theta_0, r_0\sin\theta_0)r_0\sin\theta_0\\
		&+ \frac{\partial f}{\partial y}(r_0 \cos\theta_0, r_0\sin\theta_0)r_0\cos\theta_0\\
		&= -2{r_0}^2\cos(\theta_0)\sin(\theta_0) + 2{r_0}^2 \sin(\theta_0)\cos(\theta_0)\\
		&= 0 \\
	\end{align*}

	Donc, \[
		g(r, \theta) = r^2.
	\]
\end{exm}

\begin{exm}
	Résoudre \[
		\begin{cases}
			\frac{\partial f}{\partial x} = \frac{x}{x^2+y^2},\\
			\frac{\partial f}{\partial y} = \frac{y}{x^2+y^2}.\\
		\end{cases}
	\]

	On pose $g: (r, \theta) \mapsto f(r \cos\theta, r \sin\theta)$.

	\begin{align*}
		&\frac{\partial g}{\partial r} = \frac{1}{r}\cos^2\theta + \frac{1}{r}\sin^2\theta = \frac{1}{r},\\
		&\frac{\partial g}{\partial \theta} = -\cos(\theta) \sin(\theta) + \sin(\theta)\cos(\theta) = 0.
	\end{align*}

	Donc, \[
		\exists C \in \R, g: (r, \theta) \mapsto \ln r + C
	\] d'où,
	\begin{align*}
		\forall (x,y) \in \R^2 \setminus \{(0,0)\}, f(x,y) &= \ln\left(\sqrt{x^2 + y^2} \right)  + C\\
		&= \frac{1}{2}\ln(x^2 + y^2) + C. \\
	\end{align*}
\end{exm}

\begin{rmk}
	Soit $\mathcal{B} = (e_1, e_2)$ la base canonique de $\R^2$, $f: U \to \R$ de classe $\mathcal{C}^1$ avec $U$ un ouvert de $\R^2$.

	Soit $(x,y) \in U$.

	\begin{align*}
		\Mat_{\mathcal{B}}\big(\nabla f(x,y)\big) = \begin{pmatrix}
			\frac{\partial f}{\partial x}(x,y)\\[2mm]
			\frac{\partial f}{\partial y}(x,y)
		\end{pmatrix}
	\end{align*}

	Soit  \begin{align*}
		\varphi: V &\longrightarrow U \\
		(u,v) &\longmapsto \big(x(u,v), y(u,v)\big) 
	\end{align*} avec $x,y$ de classe $\mathcal{C}^1$. Soit $g = f \circ \varphi$.
	\begin{align*}
		\Mat_{\mathcal{B}}\big(\nabla g(u,v)\big)
		&= \begin{pmatrix}
			\frac{\partial g}{\partial u}(u,v) \\[2mm]
			\frac{\partial g}{\partial v}(u,v)
		\end{pmatrix} \\
		&= \begin{pmatrix}
			\frac{\partial x}{\partial u}(u,v) \frac{\partial f}{\partial x}(x,y)
			+ \frac{\partial y}{\partial u}(u,v)\frac{\partial f}{\partial y}(x,y)\\[3mm]
			\frac{\partial x}{\partial v}(u,v) \frac{\partial f}{\partial x}(x,y)
			+ \frac{\partial y}{\partial v}(u,v) \frac{\partial f}{\partial y}(x,y)
		\end{pmatrix}  \\
		&= \underbrace{\begin{pmatrix}
				\frac{\partial x}{\partial u}(u,v)& \frac{\partial y}{\partial u}(u,v)\\[3mm]
				\frac{\partial x}{\partial v}(u,v)& \frac{\partial y}{\partial v}(u,v)
		\end{pmatrix}}_{J(u,v)} \begin{pmatrix}
			\frac{\partial f}{\partial x}(x,y)\\[3mm]
			\frac{\partial f}{\partial y}(x,y)
		\end{pmatrix} \\
		&= J(u,v) \Mat_{\mathcal{B}}\big(\nabla f(x,y)\big) \\
	\end{align*}
	où $J(u,v) = 
	\begin{pNiceArray}{c:c}
		\Mat_{\mathcal{B}}\big(\nabla x(u,v)\big) & \Mat_{\mathcal{B}}\big(\nabla y(u,v)\big)
	\end{pNiceArray}$.

	On dit que $J(u,v)$ est \underline{la jacobienne} de $\varphi$ en $(u,v)$.
	L'application linéaire canoniquement associée à $J(u,v)$ est la \underline{différentielle de $\varphi$} en $(u,v)$ noté $\mathrm{d}\varphi(u,v)$.

	On a $\mathrm{d}\varphi(u,v) \in \mathcal{L}(R^2)$ et $\Mat_{\mathcal{B}}\big(\mathrm{d}\varphi(u,v)\big) = J(u,v)$.

	Par exemple, la jacobienne du changement de coordonnées polaires est \[
		J = \begin{pmatrix}
			\frac{\partial x}{\partial r} & \frac{\partial y}{\partial r}\\[3mm]
			\frac{\partial x}{\partial \theta} & \frac{\partial y}{\partial \theta}
		\end{pmatrix}
		= \begin{pmatrix}
			\cos\theta&\sin\theta\\
			-r\sin\theta&r\cos\theta
		\end{pmatrix}.
	\]
	$\underbrace{\det(J)}_{\text{le jacobien}} = r\cos^2\theta + r\sin^2\theta = r$

	Dans une intégrale double, si $(x,y) = \varphi(u,v)$, alors $\mathrm{d}x\mathrm{d}y = \det(J)\mathrm{d}u\mathrm{d}v$.

	Ici, \[
		\mathrm{d}x\ \mathrm{d}y = r\ \mathrm{d}r\ \mathrm{d}\theta.
	\]
\end{rmk}

\begin{prv}
	On pose $(x_0, y_0) = \varphi(u_0, v_0)$. Pour tout $(h,k) \in \R^2$ tels que $(u_0 + h, v_0 + k) \in V$, en posant $g = f  \circ \varphi$.

	\begin{align*}
		g(u_0 + h, v_0 + h) &= f\big(x(u_0 + h, v_0 + k), y(u_0 + h, v_0 + k)\big) \\
		&= f\left(
			x(u_0,v_0) + h \frac{\partial x}{\partial u}(u_0,v_0) + k \frac{\partial x}{\partial v}(u_0, v_0) + \po\big(\|(h,k)\|\big), \right.\\
		&\phantom{ = f\bigg(\bigg.}\left. y(u_0, v_0) + h \frac{\partial y}{\partial u}(u_0, v_0) + k \frac{\partial y}{\partial v}(u_0, v_0) + \po\big(\|(h,k)\|\big)
		\right)  \\
		&= f(x_0,y_0) \\
		&~+ \left( h \frac{\partial x}{\partial u}(u_0,v_0) + k \frac{\partial x}{\partial v}(u_0, v_0) + \po(\|(h,k)\|) \right) \frac{\partial f}{\partial x}(x_0,y_0)\\
		&~+ \left( h \frac{\partial y}{\partial u}(u_0, v_0) + k\frac{\partial y}{\partial v}(u_0, v_0) + \po(\|(h,k)\|) \right) \frac{\partial f}{\partial y}(x_0, y_0)\\
		&~+ \po(\|(h,k)\|)\\
		&= f(x_0, y_0) \\
		&~+ h \left( \frac{\partial x}{\partial u}(u_0, v_0) \frac{\partial f}{\partial x}(x_0, y_0) + \frac{\partial y}{\partial u}(u_0, v_0) \frac{\partial f}{\partial y}(x_0, y_0) \right)  \\
		&~+ k\left( \frac{\partial x}{\partial v}(u_0, v_0) \frac{\partial f}{\partial x}(x_0, y_0) + \frac{\partial y}{\partial v}(u_0, v_0) \frac{\partial f}{\partial y}(x_0, y_0) \right) 
		&~+ \po(\|(h,k)\|)\\
		&= g(u_0, v_0) + h \frac{\partial g}{\partial u}(u_0, v_0) + k \frac{\partial g}{\partial v}(u_0, v_0) + \po(\|(h,k)\|) \\
	\end{align*}

	Par identification,
	\[
		\frac{\partial g}{\partial u}(u_0, v_0) = \frac{\partial x}{\partial u}(u_0, v_0) \frac{\partial f}{\partial x}(x_0, y_0) + \frac{\partial y}{\partial u}(u_0, v_0) \frac{\partial f}{\partial y}(x_0,y_0)
	\] et \[
		\frac{\partial g}{\partial v}(u_0, v_0) = \frac{\partial x}{\partial v}(u_0,v_0) \frac{\partial f}{\partial x}(x_0, y_0) + \frac{\partial y}{\partial v}(u_0, v_0) \frac{\partial f}{\partial y}(x_0, y_0).
	\] 
\end{prv}

\begin{exm}
	[Régression linéaire]~\\
	\begin{figure}[H]
		\centering
		\begin{asy}
			import graph;
			axes(EndArrow);
			size(5cm);

			real f(real x) { return x + 0.5; }

			real k = 35 / (7 - 0.5);

			for(int i = 0; i < 35; ++i) {
				real mag = exp(sin(100 * pi/exp(1) * i)) * 0.8 + exp(cos(i*40)/3);
				real eps = mag * cos(10 * exp(1)/pi * i) / 3;
				dot((i/k,f(i/k) + eps));
			}

			draw(graph(f, -1, 7), orange);
		\end{asy}
	\end{figure}
	\[
		y = a x + b
	\] 
	On fixe $(a,b) \in \R^2$. \[
		\varepsilon(a,b) = \sum_{i=1}^n\big( y_i - (ax_i + b) \big)^2
	\] l'erreur totale.

	On veut minimiser $\varepsilon(a,b)$. On a 
	\[
		\forall (a,b) \in \R^2,
		\begin{cases}
			\frac{\partial \varepsilon}{\partial a}(a,b) = -2\sum_{i=1}^{n}(y_i - ax_i - b)x_i,\\
			\frac{\partial \varepsilon}{\partial b}(a,b) = -2\sum_{i=1}^{n}(y_i - ax_i - b).
		\end{cases}
	\]

	Donc,
	\begin{align*}
		(a,b) \text{ point critique de } \varepsilon \iff& \begin{cases}
			a \sum_{i=1}^n {x_i}^2 + b\sum_{i=1}^{n}x_i = \sum_{i=1}^{n} y_ix_i\\
			a\sum_{i=1}^{n}x_i + nb = \sum_{i=1}^ny_i
		\end{cases}\\
		\iff& \begin{cases}
			a \left( \frac{1}{n}\sum_{i=1}^n {x_i}^2 - \overline{x}^2\right) = \overline{y} - \overline{x} \overline{y}\\
			b = \frac{1}{n}\sum_{i=1}^ny_i - \frac{a}{n}\sum_{i=1}^nx_i = \frac{1}{n}\sum_{i=1}^n x_i y_i - \overline{x} \overline{y}
		\end{cases}\\
		&\text{ où } \overline{x} = \frac{1}{n} \sum_{i=1}^n x_i,~\overline{y} = \frac{1}{n}\sum_{i=1}^n y_i\\
		\iff& \begin{cases}
			a = \frac{\Cov(x,y)}{V(x)}\\
			b = \overline{y} - a\overline{x}
		\end{cases}
	\end{align*}

	Coefficient de corrélation: $\frac{\Cov(x,y)}{\sigma_x \sigma_y} \in [-1, 1]$
\end{exm}












		\part{Corps}

\begin{exm}[Problème]
	\begin{itemize}
		\item 
			avec $A = \Z / 9 \Z$, résoudre $\overline{x}^2 = \overline{0}$ \\
			\begin{center}
				\begin{tabular}{|c|c|c|c|c|c|c|c|c|c|c|}
					\hline
					$\overline{x}$&$\overline{0}$& $\overline{1}$ &$\overline{2}$&$\overline{3}$ &$\overline{4}$ &$\overline{5}$ &$\overline{6}$ &$\overline{7}$ &$\overline{8}$& $\overline{9}$ \\
					\hline
					$\overline{x}^2$&$\overline{0}$ &$\overline{1}$ &$\overline{4}$ &$\overline{0}$ &$\overline{7}$ &$7$ &$\overline{0}$ &$\overline{4}$ &$\overline{1}$&$\overline{0}$\\
					\hline
				\end{tabular}
			\end{center}
			On a trouvé 3 solutions: $\overline{0}$, $\overline{3}$, $\overline{6}$.
		\item $\Z / 8\Z$
			\begin{center}
				\begin{tabular}{|c|c|c|c|c|c|c|c|c|}
					\hline
					$\overline{x}$& $\overline{0}$& $\overline{1}$& $\overline{2}$& $\overline{3}$& $\overline{4}$& $\overline{5}$& $\overline{6}$& $\overline{7}$\\
					\hline
					$\overline{x^2}$& $\overline{0}$& $\overline{1}$& $\overline{4}$& $\overline{1}$& $\overline{0}$& $\overline{1}$& $\overline{4}$& $\overline{1}$\\
					\hline
				\end{tabular}
			\end{center}
			$\overline{x}^2=7$ a 4 solutions: $\overline{1}, \overline{7}, \overline{3},\text{ et } \overline{5}$
		\item $A = \mathbbm{H} = \{a + bi + cj + dk  \mid  (a,b,c,d) \in \R^4\}$ \\
			$i^2 = j^2 = k^2 = -1$ 
			\begin{align*}
				\begin{array}{c c c}
					ij = k & jk = i & ji = j\\
					ji = -k & kj = -i & ik = -j
				\end{array}
			\end{align*}
			Dans cet anneau, $-1$ a 6 racines!
	\end{itemize}
\end{exm}

\begin{defn}
	Soit $(\mathbbm{K}, +, \times)$ un ensemble muni de deux lois de composition internes. On dit que c'est un \underline{corps} si
	 \begin{enumerate}
		\item $(\mathbbm{K}, \times)$ est un groupe abélien
		\item $(\mathbbm{K}, \times)$ est un monoïde commutatif
		\item $\forall x \in \mathbbm{K}\setminus \{0_\mathbbm{K}\}, \exists y \in \mathbbm{K}, xy = 1_\mathbbm{K}$
		\item $0_\mathbbm{K} \neq  1_\mathbbm{K}$
	\end{enumerate}
	\index{corps}
\end{defn}

\begin{exm}
	\begin{itemize}
		\item $(\C, +, \times)$ est un corps
		\item $(\R, +, \times)$ est un corps
		\item $(\Q, +, \times)$ est un corps
		\item $(\Z, +, \times)$ n'est pas un corps
	\end{itemize}
\end{exm}

\begin{prop}
	$(\Z / n\Z, +, \times)$ est un corps si et seulement si $n$ est premier.
\end{prop}

\begin{prv}
	\[
		\left( \Z / n\Z \right)^\times = \left\{ \overline{k}  \mid k \wedge n = 1 \right\}
	\] 
\end{prv}


\begin{prop}
	Tout corps est un anneau intègre.
\end{prop}

\begin{prv}
	Soit $(\mathbbm{K}, +, \times)$ un corps. Soient $(a,b) \in \mathbbm{K}^2$ tel que $a \times b = 0_\mathbbm{K}$.\\
	On suppose $a \neq  0_\mathbbm{K}$. Alors, $a$ est inversible et donc \[
		b = a^{-1} \times a \times b = a^{-1} \times 0_\mathbbm{K} = 0_\mathbbm{K}
	\] 
\end{prv}

\begin{exm}
	Soit $(\mathbbm{K},+,\times)$ un corps.\\
	Résoudre \[
		\begin{cases}
			x^2 = 1_\mathbbm{K}\\
			x \in \mathbbm{K}
		\end{cases}
	\]

	\begin{align*}
		x^2 = 1_\mathbbm{K} &\iff x^2 - 1_\mathbbm{K} = 0_\mathbbm{K}\\
		&\iff (x - 1_\mathbbm{K})(x+1_\mathbbm{K}) = 0_\mathbbm{K}\\
		&\iff x - 1_\mathbbm{K} = 0_\mathbbm{K} \text{ ou } x + 1_\mathbbm{K} = 0_\mathbbm{K}\\
		&\iff x = 1_\mathbbm{K} \text{ ou } x = -1_\mathbbm{K}
	\end{align*}

	Il y a au plus 2 solutions.
\end{exm}

\begin{prop}
	Soit $(\mathbbm{K},+,\times )$ un corps et $P$ un polynôme à coefficients dans $\mathbbm{K}$ de degré $n$. Alors, l'équation $P(x) = 0_{\mathbbm{K}}$ a au plus $n$ solutions dans $\mathbbm{K}$ 
	\qed
\end{prop}

\begin{crlr}[(Théorème de Wilson)]
	voir exercice 16 du TD 12
\end{crlr}


\begin{defn}
	Soit $(\mathbbm{K}, +, \times)$ un corps et $L\subset \mathbbm{K}$.\\
	On dit que $L$ est un \underline{sous corps} de $\mathbbm{K}$ si
	\begin{enumerate}
		\item $L$ est un anneau de $(\mathbbm{K}, +, \times)$ non nul
		\item $\forall x \in L\setminus \{0_\mathbbm{K}\}, x^{-1} \in L$ 
	\end{enumerate}
	\vspace{2mm}
	en d'autres termes si
	\begin{enumerate}
		\item $\forall (x,y) \in L^2, x - y \in L$
		\item $\forall (x,y) \in L^2, x \times y^{-1} \in L$
	\end{enumerate}
	\vspace{5mm}
	On dit aussi que $\mathbbm{K}$ est une \underline{extension} de $L$.
	\index{sous corps}
	\index{extension}
\end{defn}

\begin{prop}
	Tout sous corps est un corps. \qed
\end{prop}

\begin{defn}
	Soient $(\mathbbm{K}_1,+,\times )$ et $(\mathbbm{K}_2,+, \times)$ deux corps et $f: \mathbbm{K}_1 \to \mathbbm{K}_2$.\\
	On dit que $f$ est un \underline{morphisme de corps} si $f$ est un morphisme d'anneaux.\\
	i.e. si
	\[
		\begin{cases}
			\forall (x,y) \in {\mathbbm{K}_1}^2,& f(x+y) = f(x) + f(y)\\
			\forall (x,y) \in {\mathbbm{K}_1}^2,& f(x \times y) = f(x) \times f(y)\\
		\end{cases}
	\] 
	\index{homomorphisme (de corps)}
	\index{morphisme (de corps)}
\end{defn}

\begin{prop}
	Tout morphisme de corps est injectif.
\end{prop}

\begin{prv}
	Soit $f: \mathbbm{K}_1 \to \mathbbm{K}_2$ un morphisme de corps.\\
	\begin{itemize}
		\item $\Ker(f)$ est un sous groupe de $(\mathbbm{K}_1, +)$ 
		\item Soit $x \in \Ker(f)$ et $y \in \mathbbm{K}_1$ \[
				f(x \times y) = f(x) \times f(y) = 0_{\mathbbm{K}_2} \times f(y) = 0_{\mathbbm{K}_2}
			\]
		\item Soit $x \in \Ker(f) \setminus \{0_{\mathbbm{K}_1}\}$.\\
			Alors, $x$ est inversible.\\
			\begin{align*}
				\begin{rcases*}
					x \in \Ker(f)\\
					x^{-1} \in \mathbbm{K}_1
				\end{rcases*}& \text{ donc } x \times x ^{-1} \in \Ker(f)\\
				&\text{ donc } 1_{\mathbbm{K}_1} \in \Ker(f)\\
				&\text{ donc } f(1_{\mathbbm{K}_1}) = 0_{\mathbbm{K}_2}
			\end{align*}
			Or, $f(1_{\mathbbm{K}_1}) = 1_{\mathbbm{K}_2} \neq 0_{\mathbbm{K}_2}$
	\end{itemize}
	Donc, $\Ker(f) = \{0_{\mathbbm{K}_1}\}$ donc $f$ est injective.
\end{prv}

\begin{exm}
	$\begin{array}{cc}
		\C &\longrightarrow \C\\
		z &\longmapsto \overline{z}\\
	\end{array}$ est un morphisme de corps
\end{exm}



		\part{Opérations sur les séries}

\begin{prop}
	L'ensemble $E = \{u \in \C^\N  \mid \Sigma u_n \text{ converge}\}$ est un sous-espace vectoriel de $\C^\N$ et \begin{align*}
		S: E &\longrightarrow \C \\
		u &\longmapsto \sum_{n=0}^{+\infty} u_n
	\end{align*} est une forme linéaire.
	\qed
\end{prop}

\begin{rmk}
	La somme d'une série convergente et d'une série divergente diverge.
	Le produit d'une série divergente par un scalaire non nul diverge.
\end{rmk}

		\part{Comparaison de suites}

\begin{defn}
	Soient $u$ et $v$ deux suites réelles. On dit que $u$ est \underline{dominée} par  $v$ si \[
	\exists M\in \R, \exists N\in \N,\forall n\ge N,\left| u_n \right| \le M \left| v_n \right| 
	\] Dans ce cas, on note $u = O(v)$ ou $u_n = O(v_n)$ et on dit que "$u$ est un grand o de $v$"
\end{defn}

\begin{exm}
	En informatique, on dit qu'un alogirithme a une \underline{complexité linéaire} si son temps d'éxécution est un $O(n)$ 
	Par exemple, on calcule $a^n$ 

	\begin{itemize}
		\item Approche naïve
			\begin{algorithm}
				\begin{algorithmic}[1]
					\State $p \gets 1$
					\For{$i \in \left\llbracket 0,n-1 \right\rrbracket$}
						\State $p \gets p \times a$
					\EndFor
					\State \Return p
				\end{algorithmic}
			\end{algorithm}
			Complexité linéaire $O(n)$
		\item Exponentiation rapide\\
			On écrit $n$ en binaire: \begin{align*}
				n &= \overline{a_k a_{k-1}\ldots a_0}^{(2)}\\
					&= \sum_{i=0}^{k} a_i 2^i
			\end{align*} avec $(a_i) \in \left\{ 0,1 \right\} ^{k+1}$
			\begin{align*}
				a^n &= a^{\sum_{i=0}^{k} a_i 2^i} \\
				&= \prod_{i=0}^{k} a^{a_i 2^i}  \\
			\end{align*}
			
			\begin{algorithm}
				\begin{algorithmic}
					[1]

					\State $s \gets 0$
					\State $p \gets a$
					\For{ $i \in \left\llbracket 0, \log_2(n) \right\rrbracket$}
						\State $p \gets p \times p$
						\If{$a[i] = 1$}
							\State $s \gets s + p$
						\EndIf
					\EndFor
					\State \Return s
				\end{algorithmic}
			\end{algorithm}
			Compléxité logarithmique $O(\log_2(n))$
	\end{itemize}
\end{exm}


\begin{prop}
	$O$ est une relation réfléctive et transitive.
\end{prop}

\begin{prv}
	\begin{itemize}
		\item Soit $u$ une suite. On pose $M = 1$ et \[
			\forall n \in \N, \left| u_n \right| \le M \left| u_n \right|
			\] Donc $u = O(u)$.
		\item Soient $u, v, w$ trois suites telles que  \[
		\begin{cases}
			u = O(v)\\
			v = O(w)
		\end{cases}
		\] Soient $M_1,M_2 \in \R$ et $N_1,N_2\in \N$ tels que \[
		\begin{cases}
			\forall n \ge  N_1, \left| u_n \right| \le M_1 \left| v_n \right| \\
			\forall n \ge  N_2, \left| v_n \right| \le M_2 \left| w_n \right| \\
		\end{cases}
		\] 

		Nécéssairement, $M_1\ge 0$ et $M_2\ge 0$.\\
		Soit $N = \max(N_1,N_2)$. \[
		\forall n \ge  N, \left| u_n \right| \le M_1 \left| v_n \right| \le  M_1M_2 \left| w_n \right| 
		\] Donc $u = O(w)$
	\end{itemize}
\end{prv}

\begin{defn}
	Soient $u$ et $v$ deux suites. On dit que $u$ est \underline{négligeable} devant $v$ si \[
	\forall \varepsilon>0, \exists N\in \N, \forall n\ge N, \left| u_n \right| \le \varepsilon \left| v_n \right| 
	\] Dans ce cas, on note $u = o(v)$ ou $u_n = o(v_n)$ ou on le lit "$u$ est un petit o de $v$"
\end{defn}

\begin{prop}
	$o$ est une relation transitive, non-réfléctive
\end{prop}

\begin{prv}
	\begin{itemize}
		\item Soient $u$, $v$ et $w$ trois suites telles que \[
			\begin{cases}
				u = o(v)\\
				v = o(w)
			\end{cases}
			\] Soit $\varepsilon>0$. Soit $N_1\in \N$ tel que \[
			\forall n \ge N_1, \left| u_n \right| \le \sqrt{\varepsilon}  \left| v_n \right| 
			\] Soit $N_2\in \N$ tel que \[
			\forall n \ge N_2, \left| v_n \right| \le \sqrt{\varepsilon}  \left| w_n \right| 
			\] On pose $N = \max(N_1,N_2)$, alors \[
			\forall n \ge N, \left| u_n \right| \le \sqrt{\varepsilon}  \left| v_n \right| \le \underbrace{\sqrt{\varepsilon} \times \sqrt{\varepsilon}} _\varepsilon \left| w_n \right| 
			\] donc $u = o(w)$
		\item Soit $u$ une suite tel qu'il existe $N \in \N$ tel que \[
		\forall n \ge N, u_n > 0
		\] On suppose que $u = o(u)$, alors \[
		\forall \varepsilon>0,\exists N \in \N, \forall n \ge N, \left| u_n \right| \le \varepsilon \left| u_n \right| 
		\] On pose $\varepsilon = \frac{1}{2}$ alors \[
		\exists N \in \N, \forall n \ge N, \left| u_n \right| \le \frac{1}{2} \left| u_n \right| 
		\] une contradiction
	\end{itemize}
\end{prv}

\begin{prop}
	Soient $u$ et $v$ deux suites.
	\begin{itemize}
		\item $o(u) + o(u) = o(u)$
		\item $v \times o(u) = o(uv)$
		\item $o(u) \times o(v) = o(uv)$
		\item $o(o(u)) = o(u)$
	\end{itemize}
	\qed
\end{prop}

\begin{defn}
	Soient $u$ et $v$ deux suites. On dit que $u$ et $v$ sont \underline{équivalentes} si \[
	u = v + o(v)
	\] i.e. \[
	\forall \varepsilon >0, \exists N \in \N, \forall n \ge N, \left| u_n-v_n \right| \le \varepsilon\left| v_n \right| 
	\] Dans ce cas, on le note $u \sim v$
\end{defn}

\begin{prop}
	$\sim$ est une relation d'équivalence \qed
\end{prop}

\begin{prop}
	Soient $(u,v) \in \R^\N$. On suppose que $v$ ne s'annule pas à partir d'un certain rang
	\begin{enumerate}
		\item $u = o(v) \iff \left( \frac{u_n}{v_n} \right)$ bornée
		\item $u = o(v) \iff \frac{u_n}{v_n} \tendsto{n \to  +\infty} 0$
		\item $u \sim v \iff \frac{u_n}{v_n} \tendsto{n \to  +\infty} 1$
	\end{enumerate}
	\qed
\end{prop}

\begin{prop}
	[Suites de références]
	\begin{enumerate}
		\item $\ln^\alpha(n) = o(n^\beta)$ avec $(\alpha,\beta) \in \left( \R^+_* \right) ^2$ 
		\item $n^\beta = o(a^n)$ avec $\beta > 0$ et $a > 1$ 
		\item $a^n = o(n!)$ avec $a >1$ 
		\item $n! = o(n^n)$
	\end{enumerate}
\end{prop}


\begin{lem}
	[Exercice 10 du TD]
	Soit $u \in \left(\R^+_*\right)^\N$\\
	Si $\frac{u_{n+1}}{u_n} \tendsto{n \to +\infty} \ell < 1$ avec $\ell\in \R$,\\ alors $u_n \tendsto{n \to +\infty} 0$
\end{lem}

\begin{prv} [de la proposition]
	\begin{enumerate}
		\item par croissance comparée
		\item On pose $\forall n \in \N^*, u_n = \frac{n^\beta}{a^n}$. 
			\begin{align*}
				\forall  n \in \N^*, \frac{u_{n+1}}{u_n} &= \left( \frac{n+1}{n} \right) ^\beta \times \frac{1}{a} \\
				&= \frac{1}{a}\left( 1+\frac{1}{n} \right) ^\beta \\
				&\tendsto{n \to +\infty} \frac{1}{a} < 1
			\end{align*}
			Donc, $u_n \tendsto{n \to  +\infty} 0$
		\item On pose $\forall n \in \N, u_n = \frac{a^n}{n!}$ \[
			\forall n \in \N, \frac{u_{n+1}}{u_n} = \frac{a}{n+1} \tendsto{n \to +\infty} 0 < 1
			\] donc $u_n \tendsto{n \to +\infty} 0$
		\item On pose $\forall  n\in \N^*, u_n = \frac{n!}{n^n}$.
			\begin{align*}
				\forall n \in \N^*, \frac{u_{n+1}}{u_n}
				&= (n+1) {\frac{n^n}{(n+1)^{n+1}}} \\
				&= \left( \frac{n}{n+1} \right) ^n \\
				&= e^{n \ln\left( \frac{n}{n+1} \right) } \\
				&= e^{n \ln\left( 1+\frac{1}{n+1} \right)} \\
				&= e^{n(-\frac{1}{n} + o(\frac{1}{n})} \\
				&= e^{-1 + o(1)} \\
				&\tendsto{n \to  +\infty} e^{-1}<1
			\end{align*}
			donc $u_n \tendsto{n\to +\infty} 0$
	\end{enumerate}
\end{prv}

	}

	{
		\chap[30]{Intégrale de Riemann}
		\renewcommand{\cwd}{../chap30}
		\begin{defn}
	Soit $E$ un $\mathbbm{K}$-espace vectoriel. On dit que $E$ est de \underline{dimension finie} si $E$ a au moins une famille génératrice finie. On dit que $E$ est de \underline{dimension infinie} sinon.
	\index{dimension finie (espace vectoriel)}
	\index{dimension infinie (espace vectoriel)}
\end{defn}

\begin{thm}
	[Théorème de la base extraite]
	Soit $E$ un $\mathbbm{K}$-espace vectoriel non nul de dimension finie. Soit $\mathcal{G}$ une famille génératrice finie de $E$. Alors, il existe une base $\mathcal{B}$ de $\mathcal{E}$ telle que $\mathcal{B} \subset \mathcal{G}$.
\end{thm}

\begin{prv}
	[par récurrence sur $\#G = \Card(G)$]
	\begin{itemize}
		\item Soit $E$ un $\mathbbm{K}$-espace vectoriel non nul engendré par $\mathcal{G} = (u)$.\\
			Si $u = 0_E$, alors $E = \{0_E\}$: une contradiction $\lightning$ \\
			Donc $u \neq 0_E$ donc $(u)$ est libre. En effet, \[
				\forall \lambda \in \mathbbm{K}, \lambda u = 0_E \implies \lambda = 0_\mathbbm{K}
			\] Donc $\mathcal{G}$ est une base de $E$.\\
		\item Soit $n \in \N_*$. Soit $E$ un $\mathbbm{K}$-espace vectoriel. On suppose que si $E$ a une famille génératrice constituée de $n$ vecteurs, alors on peut extraire de cette famille une base de $E$.\\
			Soit $\mathcal{G}$ une famille génératrice de $E$ avec $n+1$ vecteurs.\\
			Si $\mathcal{G}$ est libre, alors $\mathcal{G}$ est une base de $E$. \\
			Si $\mathcal{G}$ n'est pas libre, alors il existe $u \in \mathcal{G}$ tel que $u \in \Vect(\mathcal{G}\setminus \{u\})$ \\
			Donc $\mathcal{G}\setminus \{u\}$ engendre $E$. Or, $\mathcal{G}\setminus \{u\}$ possède $n$ vecteurs. D'après l'hypothèse de récurrence, il existe une base $\mathcal{B}$ de $E$ telle que \[
				\mathcal{B} \subset \mathcal{G} \setminus \{u\} \subset \mathcal{G}
			\] 
	\end{itemize}
\end{prv}

\begin{crlr}
	Tout espace de dimension finie a une base.
	\qed
\end{crlr}

\begin{thm}
	[Théorème de la base incomplète]
	Soit $E$ un $\mathbbm{K}$-espace vectoriel de dimension finie, $\mathcal{G}$ une famille génératrice finie de $E$. $\mathcal{L}$ une famille libre de $E$. Alors, il existe une base $\mathcal{B}$ de $E$ telle que \[
		\mathcal{L} \subset \mathcal{B} \text{ et } \mathcal{B}\setminus \mathcal{L} \subset \mathcal{G}
	\] 
\end{thm}

\begin{prv}
	[par récurrence sur $\#(\mathcal{G}\setminus\mathcal{L})$]
	\begin{itemize}
		\item Avec les notations précédentes, on suppose que $\mathcal{G}\setminus\mathcal{L} \neq \O$ \[
				\forall u \in \mathcal{G}, u \in \mathcal{L}
			\] Donc $\mathcal{G} \subset \mathcal{L}$ donc $\mathcal{L}$ est génératrice donc $\mathcal{L}$ est une base de $E$. On pose $\mathcal{B} = \mathcal{L}$ et alors \[
				\mathcal{L} \subset  \mathcal{B} \text{ et } \mathcal{B}\setminus\mathcal{L} = \O \subset  \mathcal{G}
			\] 
		\item Soit $n \in \N$. On suppose que si $\mathcal{G}$ est génératrice et $\mathcal{L}$ libre avec $\#(\mathcal{G}\setminus\mathcal{L}) = n$ alors il existe une base $\mathcal{B}$ de $E$ telle que \[
			\mathcal{L}\subset \mathcal{B} \text{ et } \mathcal{B}\setminus\mathcal{L}\subset \mathcal{G}
		\] Soient à présent $\mathcal{G}$ une famille génératrice de $E$ et $\mathcal{L}$ une famille libre de $E$ telles que $\#(\mathcal{G}\setminus\mathcal{L}) = n+1 > 0$\\
		Si $\mathcal{L}$ engendre $E$, alors $\mathcal{L}$ est une base de $E$. On pose $\mathcal{B} = \mathcal{L}$ et on a bien \[
			\mathcal{L} \subset  \mathcal{B} \text{ et } \mathcal{B} \setminus \mathcal{L} = \O \subset  \mathcal{G}
		\] On suppose que $\mathcal{L}$ n'engendre pas $E$. Il existe $u \in \mathcal{G}$ tel que $u \not\in \Vec(\mathcal{L})$ (car sinon, $\mathcal{G} \subset \Vect(\mathcal{L})$ et donc $\underbrace{\Vect(\mathcal{G})}_{= E} \subset  \underbrace{\Vect(\mathcal{L})}_{ \subset E}$\\
		Donc $\mathcal{L} \cup \{u\} $ est libre. On pose $\mathcal{L}' = \mathcal{L} \cup \{u\} $ \[
			\mathcal{G}\setminus \mathcal{L}' = \mathcal{G}\setminus (\mathcal{L} \cup \{u\}) = (\mathcal{G}\setminus\mathcal{L})\setminus \{u\} 
		\] donc $\#(\mathcal{G}\setminus\mathcal{L}') = n+1 -1 = n$\\
		D'après l'hypothèse de récurrence, il existe $\mathcal{B}$ une base de $E$ telle que \[
			\mathcal{L} \subset  \mathcal{L}' \subset \mathcal{B} \text{ et } \mathcal{B}\setminus \mathcal{L}' \subset \mathcal{G}
		\] \[
			\mathcal{B} \setminus \mathcal{L} = \underbrace{\mathcal{B}\setminus\mathcal{L}'}_{\subset \mathcal{G}} \cup \underbrace{\{u\}}_{\subset \mathcal{G} \text{ car } u \in \mathcal{G}}
		\] On a $\mathcal{B}\setminus\mathcal{L}\subset \mathcal{G}$
	\end{itemize}
\end{prv}

\begin{thm}
	Soit $E$ un $\mathbbm{K}$-espace vectoriel de dimension finie. Toutes les bases de $E$ ont le même cardinal.
\end{thm}

\begin{prv}
	Soit $\mathcal{G}$ une famille génératrice finie de $E$ et $\mathcal{B} \subset  \mathcal{G}$ une base de $E$. On note $n = \#\mathcal{B}$ \\
	Soit $\mathcal{B}'$ une base de $E$. On pose $p = n - \#(\mathcal{B} \cap  \mathcal{B}')$. Montrons par récurrence sur  $p$ que $\#\mathcal{B} = \#\mathcal{B}'$ 
	\begin{itemize}
		\item On suppose que $p = 0$. Alors, $\#(\mathcal{B} \cap \mathcal{B}') = n$ \\
			Or, $\mathcal{B}' \cap \mathcal{B} \subset \mathcal{B}$ donc $\mathcal{B} \cap \mathcal{B}' = \mathcal{B}$ donc $\mathcal{B} \subset  \mathcal{B}'$ et donc $\mathcal{B} = \mathcal{B}'$ 
		\item Soit $p \in \N$. On suppose que si $\mathcal{B}'$ est une base de $E$ telle que $n - \#(\mathcal{B} \cap \mathcal{B}') = p$, alors $\#\mathcal{B}' = n$ \\
			Aoit $\mathcal{B}'$ une base de $E$ telle que $n - \#(\mathcal{B}\cap \mathcal{B}') = p+1 > 0$ \\
			Donc $\mathcal{B} \cap \mathcal{B}' \neq \mathcal{B}$. Soit $u \in \mathcal{B}' \setminus \mathcal{B}$. D'après le lemme d'échange, il existe $v \in \mathcal{B}\setminus \mathcal{B}'$ tel que $\mathcal{B}' \setminus \{u\} \cup \{v\}$ est une base de $E$. On pose $\mathcal{B}'' = \mathcal{B}' \setminus \{u\} \cup \{v\}$ 
			\begin{align*}
				\mathcal{B}'' \cap \mathcal{B} &= \left( (\mathcal{B}' \setminus \{u\})  \cap \mathcal{B} \right) \cup \{v\} \\
				&= (\mathcal{B}' \cap \mathcal{B}) \cup \{v\} \\
			\end{align*}
			donc,
			\begin{align*}
				n - \#(\mathcal{B}'' \cap \mathcal{B}) &= n - (\#(\mathcal{B}' \cap \mathcal{B}) + 1) \\
				&= p+1- 1 \\
				&= p \\
			\end{align*}
			D'après l'hypothèse de récurrence, \[
				\#\mathcal{B}'' = n
			\] Or, $\#\mathcal{B}'' = \#\mathcal{B}'$
	\end{itemize}
\end{prv}

\begin{lem}
	Soient $\mathcal{B}$ et $\mathcal{B}'$ deux bases de $E$ telles que $\mathcal{B}\subset \mathcal{B}'$. Alors, $\mathcal{B} = \mathcal{B}'$.
\end{lem}

\begin{prv}
	On suppose $\mathcal{B}' \neq \mathcal{B}$. Soit $u \in \mathcal{B}' \setminus \mathcal{B}$
	$u \in E = \Vect(\mathcal{B})$ donc $\mathcal{B} \cup \{u\}$ n'est pas libre.
	Donc $\mathcal{B}\cup \{u\} \subset \mathcal{B}'$ et $\mathcal{B}'$ est libre donc $\mathcal{B}\cup \{u\}$ est libre: une contradiction $\lightning$
\end{prv}

\begin{lem}
	[Lemme d'échange] Soient $\mathcal{B}_1$ et $\mathcal{B}_2$ deux bases de $E$ et $u \in \mathcal{B}_1 \setminus \mathcal{B}_2$. Alors, il existe $v \in \mathcal{B}_2$ tel que $(\mathcal{B}_1 \setminus \{u\}) \cup \{v\}$ soit une base de $E$.
\end{lem}

\begin{prv}
	[1${}^\text{nde}$ méthode]
	On suppose que pout tout $v \in \mathcal{B}_2$, $(\mathcal{B}_1\setminus \{u\}) \cup \{v\}$ n'est pas une base de $E$
	Soit $v \in \mathcal{B}_2$.
	\begin{itemize}
		\item Supposons $(\mathcal{B}_1\setminus \{u\})\cup \{v\}$ non libre. $\mathcal{B}_1 \setminus \{u\}$ est libre. Donc $v \in \Vect(\mathcal{B}_1 \setminus \{u\})$
		\item Supposons $(\mathcal{B}_1\setminus \{u\}) \cup \{v\}$ non génératrice.
			Comme $\mathcal{B}_1$ engendre $E$, $u \not\in \Vect(\mathcal{B}_1\setminus \{v\})$.
			On suppose que $\mathcal{B}_1 \neq \mathcal{B}_2$.
			$\forall v \in \mathcal{B}_2 \setminus \mathcal{B}_1, \Vect(\mathcal{B}_1 \setminus \{v\}) = \Vect(\mathcal{B}_1) = E \ni u$ 
			donc, $(\mathcal{B}_1\setminus \{u\}) \cup \{v\}$ engendre $E$ et donc \[
				v \in \Vect(\mathcal{B}_1 \setminus \{u\})
			\] On a aussi \[
				\forall v \in \mathcal{B}_1 \setminus \{u\}, v \in \Vect(\mathcal{B}_1\setminus \{u\})
			\] Comme $u \not\in \mathcal{B}_2$, on a \[
				\forall v \in \mathcal{B}_2, v \in \Vect(\mathcal{B}_1\setminus \{u\})
			\] docn \[
				E = \Vect(\mathcal{B}_2) \subset \Vect(\mathcal{B}_1\setminus \{u\})
			\] donc $\mathcal{B}_1\setminus \{u\}$ engendre $E$ donc $\mathcal{B}_1\setminus \{u\}$ est une base de $E$. Or, $\mathcal{B}_1 \setminus \{u\}  \subset  \mathcal{B}_1$, donc $\mathcal{B}_1\setminus \{u\} = \mathcal{B}_1$
	\end{itemize}
\end{prv}

\begin{prv}
	[2${}^\text{nde}$ méthode]
	On suppose que pout tout $v \in \mathcal{B}_2$, $(\mathcal{B}_1\setminus \{u\}) \cup \{v\}$ n'est pas une base de $E$
	\begin{itemize}
		\item Comme $u \in \mathcal{B}_1 \setminus \mathcal{B}_2$, nécéssairement $\mathcal{B}_1 \neq \mathcal{B}_2$ donc $\mathcal{B}_2 \not\subset \mathcal{B}_1$, donc $\mathcal{B}_2\setminus\mathcal{B}_1 \neq \O$ 
		\item Soit $v \in \mathcal{B}_2\setminus\mathcal{B}_1$. Il existe $(\lambda_w)_{w\in\mathcal{B}_1}$ une famille de scalaires presque nulle telle que \[
				v = \sum_{w \in \mathcal{B}_1} \lambda_w w - \lambda_u u + + \sum_{w \in \mathcal{B}_1\setminus \{u\}}\lambda_w w
			\]
			Si $\lambda_u \neq 0_E$, alors
			\begin{align*}
				u &= \lambda_u^{-1}\left( v - \sum_{w \in \mathcal{B}_1 \setminus \{u\}} \lambda_w w \right)\\
					&\in \Vect(\mathcal{B}_1\setminus \{u\} \cup v)
			\end{align*}
			 donc $\mathcal{B}_1 \subset \Vect(\mathcal{B}_1\setminus \{u\} \cup \{v\})$\\
			 et donc $E \subset  \Vect(\mathcal{B}_1 \setminus \{u\} \cup \{v\})$ \\
			 et donc $\mathcal{B}_1 \setminus \{u\} \cup \{v\}$ engendre $E$ \\
			 donc $\mathcal{B}_1 \setminus \{u\} \cup \{v\}$ n'est pas libre\\
			 donc $v \in \Vect(\mathcal{B}_1\setminus \{u\})$ (car $\mathcal{B}_1 \setminus \{u\}$ est libre\\
			 donc $\lambda_u = 0_\mathbbm{K}$ $\lightning$\\`

			 Donc, $\lambda_u = 0_\mathbbm{K}$, docn $v \in \Vect(\mathcal{B}_1\setminus \{u\})$ \\
			 On vient de prouver que
			 \begin{align*}
			 	\mathcal{B}_2 \setminus \mathcal{B}_1 \subset \Vect(\mathcal{B}_1 \setminus \{u\})\\
			 	\mathcal{B}_1 \setminus \{u\} \subset \Vect(\mathcal{B}_1 \setminus \{u\})\\
			 \end{align*}
			 Comme $u \not\in \mathcal{B}_2$, \[
			 	\mathcal{B}_2 \subset \Vect(\mathcal{B}_1 \setminus \{u\})
			 \] donc \[
			 	E = \Vect(\mathcal{B}_2) \subset  \Vect(\mathcal{B}_1 \setminus \{u\})
			 \] donc $\mathcal{B}_1 \setminus \{u\}$ engendre $E$. Donc,  $\mathcal{B}_1 \setminus \{u\}$ est une base de $E$.\\
			 Or, $\mathcal{B}_1 \setminus \{u\} \subset  \mathcal{B}_1$, donc $\mathcal{B}_1 \setminus \{u\} = \mathcal{B}_1$
	\end{itemize}
\end{prv}

\begin{defn}
	Soit $E$ un $\mathbbm{K}$-espace vectoriel de dimension finie. Le cardinal commun à toutes les bases de $E$ est appelé \underline{dimension} de $E$ est notée $\dim(E)$ ou $\dim_\mathbbm{K}(E)$\\
	C'est donc aussi le nombre de coordonnées de n'importe quel vecteur dans n'importe quelle base.
	\index{dimension (espace vectoriel)}
\end{defn}

\begin{exm}
	\begin{enumerate}
		\item $\dim_\R(\C) = 2$ et $\dim_\C(\C) = 1$ 
		\item $\dim_\mathbbm{K}(\mathbbm{K}^{n}) = n$ 
		\item $\dim_{\mathbbm{K}}(\mathcal{M}_{n,p}(\mathbbm{K})) = np$
	\end{enumerate}
\end{exm}

\begin{crlr}
	Soit $E$ un $\mathbbm{K}$-espace vectoriel de dimension finie, $\mathcal{L}$ une famille libre de $E$, $\mathcal{G}$ une famille génératrice de $E$. On note $n = \dim(E)$
	\begin{enumerate}
		\item $\#\mathcal{G} \ge n$ et $(\#\mathcal{G} = n \implies \mathcal{G} \text{ est une base de } E$)
		\item $\#\mathcal{L} \le n$ et $(\#\mathcal{L} = n \implies \mathcal{L} \text{ est une base de } E$)
	\end{enumerate}
\end{crlr}

\begin{crlr}
	$\R^{\R}$ est de dimension infinie.
	$\forall i \in \N, e_i: x \mapsto x^i$\\
	$(e_i)_{i\in\N}$ est libre dans $\R^\R$
\end{crlr}

\begin{prop}
	Soient $E$ et $F$ deux $\mathbbm{K}$-espaces vectoriels de dimension finie. Alors $E\times F$ est de dimension finie et $\dim(E\times F) = \dim(E) + \dim(F)$
\end{prop}

\begin{prv}
	Soit $(e_1,\ldots, e_n)$ une base de $E$, $(f_1, \ldots, f_p)$ une base de $F$.
	On pose \[
		\left\{\begin{array}
			{r c l}
			u_1 &=& (e_1,0_F)\\
			u_2 &=& (e_2,0_F)\\
					&\vdots&\\
			u_n &=& (e_n,0_F)\\
			u_{n+1} &=& (0_E, f_1)\\
			u_{n+2} &=& (0_E, f_2)\\
					&\vdots&\\
			u_{n+p} &=& (0_E,f_p)\\
		\end{array}\right.
	\]
	Soit $(x,y) \in E\times F$. \[
		\begin{cases}
			\exists (x_1,\ldots,x_n)\in \mathbbm{K}^n, x = \sum_{i=1}^{n} x_ie_i
			\exists (y_1,\ldots,y_n)\in \mathbbm{K}^n, x = \sum_{j=1}^{p} y_jf_j
		\end{cases}
	\] 
	\begin{align*}
		(x,y) &= \left( \sum_{i=1}^{n} x_ie_i, \sum_{i=1}^{p} y_jf_j \right)  \\
		&= \sum_{i=1}^{n} x_i (e_i + 0_F) + \sum_{j=1}^{p} y_j (0_E, f_j) \\
		&= \sum_{i=1}^{n} x_i u_i + \sum_{j=1}^{p} y_j u_{n+j} \\
	\end{align*}
	Donc, $E\times F = \Vect(u_1, \ldots, u_{n+p})$ donc $E\times F$ est de dimension finie.\\
	Soit $(\lambda_1, \ldots, \lambda_{n+p}) \in \mathbbm{K}^{n+p}$ tel que \[
		(*): \quad \sum_{k=1}^{n+p} \lambda_ku_k = 0_{E\times F} = (0_E, 0_F)
	\]
	\begin{align*}
		(*) &\iff \sum_{k=1}^{n} \lambda_k (e_k, 0_F) + \sum_{k=n+1}^{p} \lambda_k(0_E, f_{k-n}) = (0_E, 0_F)\\
				&\iff \begin{cases}
					\sum_{k=1}^{n} \lambda_k e_k = 0_E\\
					\sum_{k=n+1}^{p} \lambda_k f_{k-n} = 0_F
				\end{cases}\\
				&\iff \begin{cases}
					\forall k \in \left\llbracket 1,n \right\rrbracket, \lambda_k = 0_\mathbbm{K} \qquad&(\text{car $(e_1,\ldots,e_n)$ est libre})\\
					\forall k \in \left\llbracket n+1,n+p \right\rrbracket, \lambda_k = 0_\mathbbm{K} \qquad&(\text{car $(f_1,\ldots,f_n)$ est libre})\\
				\end{cases}
	\end{align*}
	Donc $(u_1, \ldots, u_{n+p})$ est une base de $E\times F$. Donc, $\dim(E\times F) = n + p = \dim(E) + \dim(F)$
\end{prv}

\begin{rmk}
	[Convention]
	\[\dim\big(\{0_E\}\big) = 0\]
\end{rmk}

\begin{thm}
	Soit $E$ un $\mathbbm{K}$-espace vectoriel de dimension finie, $F$ un sous-espace vectoriel de $E$. Alors, $F$ est de dimension finie et  $\dim(F) \le \dim(E)$\\
	Si $\dim(F) = \dim(E)$, alors $F = E$
\end{thm}

\begin{prv}
	On considère \[
		A = \{k \in \N \mid \text{il existe une famille libre de $F$ à $k$ éléments}\} 
	\]
	On suppose $F \neq \{0_E\}$.
	\begin{itemize}
		\item Soit $u \in F\setminus \{0_E\}$. $(u)$ est libre donc $1 \in A$ et donc $A \neq \O$
		\item Soit $\mathcal{L}$ une famille libre de $F$. Alors, $\mathcal{L}$ est une famille libre de $E$ \\
			donc $\#\mathcal{L} \le \dim(E)$\\
			Donc $A$ est majorée par $\dim(E)$ \\
			On en déduit que $A$ a un plus grand élément $p$.
		\item Soit $\mathcal{L}$ une famille libre de $F$ avec $p$ éléments.\\
			Si $\mathcal{L}$ n'engendre pas $F$, alors il existe $u\in F$ tel que $u\not\in \Vect(\mathcal{L})$ et donc $\mathcal{L} \cup \{u\}$ est une famille libre de $F$, donc $p+1 \in A$ en contradiction avec la maximalité de $p$.\\
			Donc $\mathcal{L}$ est une base de $F$ donc $F$ est de dimension finie et $\dim(F) = p \le \dim(E)$\\
	\end{itemize}

	Soit $\mathcal{B}$ une base de $F$. Alors, $\mathcal{B}$ est aussi une famille de libre de de $E$. Donc $\#\mathcal{B} \le \dim(E)$ donc $\dim(F) = \dim(E)$ \\
	Si $\dim(F) = \dim(E)$, alors $\mathcal{B}$ est une base de $E$, et donc $F = \Vect(\mathcal{B}) = E$
\end{prv}

\begin{prop}
	[Formule de Grassmann]
	Soit $E$ un $\mathbbm{K}$-espace vectoriel de dimension finie, $F$ et $G$ deux sous-espace vectoriels de $E$. Alors, \[
		\dim(F+G) = \dim(F) + \dim(G) - \dim(F\cap G)
	\] 
\end{prop}

\begin{prv}
	Soit $(e_1, \ldots, e_p)$ une base de $F\cap G$. $(e_1,\ldots,e_p)$ est une famille libre de $F$.\\
	On complète $(e_1, \ldots, e_p)$ en une base $(e_1, \ldots, e_p, u_1, \ldots, u_q)$ de $F$.\\
	De même, on complète $(e_1, \ldots, e_p)$ en une base $(e_1, \ldots, e_p, v_1, \ldots, v_r)$ de $G$.\\
	On pose  $\mathcal{B} = (e_1, \ldots, e_p, u_1, \ldots, u_q, v_1, \ldots, v_r)$. Montrons que $\mathcal{B}$ est une base de $F+G$
	\begin{itemize}
		\item Soit $u \in F+G$ \\
			On pose $u = v+w$ avec $\begin{cases}
				v\in F\\
				w \in G
			\end{cases}$.\\
			On pose $v = \sum_{i=1}^p \lambda_i e_i + \sum_{i=1}^q \mu_i u_i$ avec $(\lambda_1, \ldots, \lambda_p, \mu_1, \ldots, \lambda_q) \in \mathbbm{K}^{p+q}$\\
			On pose aussi $w = \sum_{i = 1}^p \lambda'_ie_i + \sum_{j=1}^r \nu_j v_j$ avec $(\lambda_1',\ldots,\lambda_p', \nu_1, \ldots, \nu_r) \in \mathbbm{K}^{p+r}$\\
			D'où, \[
				u = \sum_{i=1}^p (\lambda_i + \lambda'_i)e_i + \sum_{j=1}^q \mu_j u_j + \sum_{k=1}^r \nu_k v_k \in \Vect(\mathcal{B})
			\]
		\item Soient $(\lambda_1, \ldots, \lambda_p, \mu_1, \ldots, \mu_q, \nu_1, \ldots, \nu_r) \in \mathbbm{K}^{p+q+r}$.\\
			On suppose \[
				(*)\quad \sum_{i=1}^{p}\lambda_ie_i + \sum_{j=1}^q\mu_ju_j + \sum_{k=1}^r \nu_k v_k = 0_E
			\] 
			D'où, \[
				\underbrace{\sum_{i=1}^p\lambda_i e_i + \sum_{j=1}^q \mu_ju_j}_{\in F} = \underbrace{-\sum_{k=1}^r\nu_jv_k}_{\in G}
			\] 
			Donc, \[
				f = \sum_{i=1}^p \lambda_i e_i + \sum_{j=1}^q \mu_j u_j \in F\cap G
			\] Comme $(e_1, \ldots, e_p)$ est une base de $F\cap G$, $\exists ! (\lambda_1', \ldots, \lambda_p') \in \mathbbm{K}^p$ tel que \[
				f = \sum_{i=1}^p \lambda'_i e_i = \sum_{i=1}^p \lambda'_i e_i + \sum_{j=1}^q 0_\mathbbm{K}u_j
			\] Comme $(e_1, \ldots, e_p, u_1, \ldots, u_q)$ est une base de $F$, \[
				\forall k \in \left\llbracket 1, q \right\rrbracket, \mu_j = 0_\mathbbm{K}
			\] De même, \[
				\forall k \in \left\llbracket 1,r \right\rrbracket , \nu_k = 0_\mathbbm{K}
			\] On remplace dans $(*)$ pour trouver \[
				\sum_{i=1}^p \lambda_ie_i = 0_E
			\] Comme $(e_1, \ldots, e_p)$ est libre, \[
				\forall i \in \left\llbracket 1,p \right\rrbracket, \lambda_i = 0_\mathbbm{K}
			\] Donc $\mathcal{B}$ est libre.\\
			Donc, 
			\begin{align*}
				\dim(F+G) &=  p +q + r \\
				&= (p+q)+ (p+r) - p \\
				&= \dim(F) + \dim(G) - \dim(F\cap G) \\
			\end{align*}
	\end{itemize}
\end{prv}

\begin{crlr}
	Avec les hypothèse précédentes, \[
		E = F \oplus G \iff \begin{cases}
			F \cap  G = \{0_E\} \\
			\dim(E) = \dim(F) + \dim(G)
		\end{cases}
	\] 
\end{crlr}

\begin{prv}
	\begin{itemize}
		\item[``$\implies$''] On suppose $E = F \oplus G$ \\
			Comme la somme est directe, $F \cap G = \{0_E\}$ 
			\begin{align*}
				\dim(E) &= \dim(F)\\
				&= \dim(F) + \dim(G) - \dim(F\cap G)\\
				&= \dim(F) + \dim(G)\\
			\end{align*}
		\item[``$\impliedby$''] On suppose $F\cap G = \{0_E\}$ et $\dim(E) = \dim(F) + \dim(G)$.\\
			On sait déjà que $F+G = F \oplus G$\\
			 \begin{align*}
				\dim(F+G) = \dim(F) + \dim(G) - \dim(F \cap G) = \dim(E)
			\end{align*}
			Donc $F + G = E$
	\end{itemize}
\end{prv}

\begin{prop}
	Soit $F$ un $\mathbbm{K}$-espace vectoriel de dimension finie $n$. Soit $\mathcal{B} = (e_1, \ldots, e_n)$ une base de $F$. L'application
	\begin{align*}
		f: \mathbbm{K}^n &\longrightarrow F \\
		(\lambda_1, \ldots, \lambda_n) &\longmapsto \sum_{i=1}^n \lambda_i e_i
	\end{align*} est bijective.\\
	Si $\mathbbm{K}$ est infini, $\mathbbm{K}^n$ aussi et donc $F$ aussi.\\
	Si $\#\mathbbm{K} = p \in \N_*$,
	\begin{align*}
		\#&\mathbbm{K}^n = p^n\\
		&\vrt=\\
		\#&F
	\end{align*}
\end{prop}


		\part{Dérivation}

\underline{Motivation}:

{
\begin{wrapfigure}{l}{3cm}
	\centering
	\begin{asy}
		import three;

		size(3cm);
		settings.render=0;
		settings.prc=false;
		currentprojection = obliqueZ;

		draw(unitbox);
		draw(shift(1.1Z + 0.05X) * (O -- X), Arrows3(TeXHead2));
		draw(shift(1.1Z + 0.05Y) * (O -- Y), Arrows3(TeXHead2));
		draw(shift(1.1X + 0.05Z) * (O -- Z), Arrows3(TeXHead2));

		label("$x$", (X/2) + (1.1Z + 0.05X), align=S);
		label("$y$", (Y/2) + (1.1Z + 0.05Y), align=W);
		label("$z$", (Z/2) + X, align=SE);
	\end{asy}
\end{wrapfigure}

\begin{align*}
	&S(x,y,z) = 2(xy + xz + yz)\\
	&V(x,y,z) = xyz
\end{align*}

On cherche à minimiser $S$ avec la contrainte $V = 1$.

Soit $f : \begin{array}{rcl}
	\left( \R_*^+ \right)^2 &\longrightarrow& \R \\
	(x,y) &\longmapsto& S\left( x,y,\frac{1}{xy} \right) = 2\left( xy + \frac{1}{y} + \frac{1}{x} \right).
\end{array}$

On cherche $(a,b) \in \left( \R^+_* \right)^2$ tel que \[
	\forall (x,y) \in (\R^+_*), f(x,y) \ge f(a,b).
\]
}

\begin{defn}
	Soit $f: U \to \R$ où $U$ est un ouvert de $\R^2$. Soit $(a,b) \in U$.
	\vspace{2mm}

	Si $\lim_{x \to a} \frac{f(x,b) - f(a,b)}{x - a} \in \R$, alors on dit que $f$ a une dérivée partielle suivant $x$ en $(a,b)$ et cette limite est notée \[
		\partial f_1(a,b) = \frac{\partial f}{\partial x}(a,b).
	\]

	Si $\lim_{y \to b} \frac{f(a,y) - f(a,b)}{y - b} \in \R$, alors on dit que $f$ a une dérivée partielle suivant $y$ et la limite est notée \[
		\partial f_2(a,b) = \frac{\partial f}{\partial y}(a,b).
	\]
\end{defn}

\begin{exm}
	\begin{enumerate}
		\item $f: (x,y) \mapsto xy + x - y$.

			\begin{align*}
				&\frac{\partial f}{\partial x} : (x,y) \mapsto y + 1,\\
				&\frac{\partial f}{\partial y} : (x,y) \mapsto x - 1.
			\end{align*}

		\item $f: (x,y) \mapsto xy + \frac{1}{y}+ \frac{1}{x}$.

			\begin{align*}
				&\frac{\partial f}{\partial x}: (x,y) \mapsto y - \frac{1}{x^2},\\
				&\frac{\partial f}{\partial y}: (x,y) \mapsto x - \frac{1}{y^2}.
			\end{align*}

		\item Trouver $f$ telle que $\begin{cases}
				(1): \qquad \frac{\partial f}{\partial x}=y,\\[2mm]
				(2): \qquad \frac{\partial f}{\partial y} = x.
			\end{cases}$

			D'après $(1)$ : \[
				\forall (x,y), \exists C(y) \in \R, f(x,y) = xy + C(y)
			\] et donc \[
				\frac{\partial f}{\partial y}(x,y) = x + C'(y)
			\] donc $C'(y) = 0$ et donc $C$ est constante.

		\item Trouver $f$ telle que $\begin{cases}
			\frac{\partial f}{\partial x} = -y,\\[2mm]
			\frac{\partial f}{ƒ\partial y} = x.
		\end{cases}$

		Ce n'est pas possible !
	\end{enumerate}
\end{exm}

\begin{defn}~\\
	\begin{minipage}{\linewidth}
		\begin{wrapfigure}{r}{4cm}
			\centering
			\vspace{-5mm}
			\begin{asy}
				import three;
				import graph3;
				size(4cm);

				settings.render = 0;
				settings.prc = false;
				currentprojection = obliqueX;

				draw(O -- X, Arrow3(TeXHead2));
				draw(O -- Y, Arrow3(TeXHead2));
				draw(O -- Z, Arrow3(TeXHead2));

				triple f(real x, real y, real z = 0) { return (x,y,cos(x - 0.5) * cos(y - 0.5)/1.2 + 0.15); }

				real inc = 1 / 5;

				for(real x = 0; x <= 1; x += inc) {
					draw(graph(
						new real(real t) { return x; }, // x
						new real(real y) { return y; }, // y
						new real(real y) { return f(x,y).z; }, // z
						0, 1
					), gray);
				}

				for(real y = 0; y <= 1; y += inc) {
					draw(graph(
						new real(real x) { return x; }, // x
						new real(real t) { return y; }, // y
						new real(real x) { return f(x,y).z; }, // z
						0, 1
					), gray);
				}

				path3 path1 = (0.8, 0.2, 0) .. (0.5, 0.5, 0) .. (0.3, 0.7, 0);
				path3 path2 = f(0.8, 0.2, 0) .. f(0.5, 0.5, 0) .. f(0.3, 0.7, 0);
				path3 d = (0.2, 0.3, 0) .. (0.3, 0.4, 0) .. (0.2, 0.7, 0) .. (0.8, 0.9, 0) .. (0.6, 0.2, 0) .. cycle;

				draw(path1, red, Arrow3(TeXHead2));
				draw(path2, red, Arrow3(TeXHead2, position=0.8));

				dot((0.5, 0.5, 0));
				dot(f(0.5, 0.5, 0));
				draw((0.5, 0.5, 0) -- f(0.5, 0.5, 0), dashed);
				draw(d);

				label("$w$", (0.3, 0.7, 0), red, align=SE);
				label("$U$", (0.8, 0.9, 0), align=SE);
			\end{asy}
		\end{wrapfigure}

		Soit $f: U \to \R$ où $U$ est un ouvert. Soit $(a,b) \in U$. Soit $w = (w_1, w_2) \in \R^2$.

		Si 
		\[
			\lim_{t\to 0} \frac{f(a + tw_1, b + tw_2) - f(a,b)}{t}
		\] existe et est réelle, alors on dit que $f$ a une dérivée dans la direction de $w$ et la limite est notée \[
			\mathrm{d}f(w)\,(a,b) = D_w(f)\,(a,b).
		\]
	\end{minipage}
\end{defn}

\begin{exm}
	\begin{align*}
		f: \left( \R_*^+ \right)^2 &\longrightarrow \R \\
		(x,y) &\longmapsto xy+\frac{1}{x}+\frac{1}{y}.
	\end{align*}

	On pose $(a,b) = (1,2)$, $w = (w_1, w_2) = (1,1)$.
	\begin{align*}
		\frac{f(1+t, 2+t) - f(1,2)}{t} &= \frac{1}{t} \left( (1+t)(2+t) + \frac{1}{1+t} + \frac{1}{2+t} - 3 - \frac{1}{2} \right) \\
		&= \frac{1}{t}\left(\cancel 2 + 3t + \po(t) + \cancel 1 - t + \po(t) + \frac{1}{2}\left( \cancel 1 - \frac{t}{2} + \po(t) \right) - \cancel3 - \cancel{\frac{1}{2}} \right) \\
		&= \frac{1}{t} \left( \frac{7}{4} t + \po(t) \right)  \\
		&= \frac{7}{4} + \po(1) \tendsto{t \to 0} \frac{7}{4}. \\
	\end{align*}

	Donc, \[
		\mathrm{d}f(1,1)\,(1,2) = \frac{7}{4}.
	\]
\end{exm}

\begin{rmk}~\\
	\begin{figure}[H]
		\centering
		\begin{asy}
			import solids;
			import graph;
			size(5cm);

			settings.render = 0;
			settings.prc = false;

			path3 par = graph(
				new real(real x) { return x; },
				new real(real x) { return 0; },
				new real(real x) { return x^2; },
				0,3);
			revolution r = revolution(par, axis=Z);

			path3 par2 = graph(
				new real(real x) { return x; },
				new real(real x) { return 0; },
				new real(real x) { return x^2; },
				-3,3);

			draw(r,1,longitudinalpen=nullpen);
			draw(r.silhouette());

			draw((-4, 0, -1) -- (-4, 0, 10) -- (4, 0, 10) -- (4, 0, -1) -- cycle, red);
			draw(par2, deepred);

			draw((4,4.5) -- (7, 4.5), black+0.5mm, Arrow(TeXHead));

			path par2d = graph(new real(real x) { return x^2; }, -3, 3);
			draw(shift((11, 0)) * par2d, deepred);

			dot(O);
			dot((11, 0));
		\end{asy}
	\end{figure}
\end{rmk}


%todo ajouter théorème-définition
\begin{thm}
	Soit $f : U \to \R$, $(a,b) \in U$. On suppose que $\frac{\partial f}{\partial x}$ et $\frac{\partial f}{\partial y}$ existent en $(a,b)$ et sont {\bfseries continues} en $(a,b)$. Alors,
	\begin{align*}
		&\forall (h, k) \in \R^2 \text{ tel que } (a +h, b + k) \in U,\\
		&f(a+ h, b + k) = f(a,b) + h \frac{\partial f}{\partial x}(a,b) + k \frac{\partial f}{\partial y}(a,b) + \po_{(h,k)\to (0,0)}\big(\|(h,k)\|\big).
	\end{align*}

	On dit que $f$ est de classe $\mathcal{C}^1$ si $\frac{\partial f}{\partial x}$ et $\frac{\partial f}{\partial y}$ existent et sont continues.

	\qed
\end{thm}

\begin{rmk}
	En physique, cette formule correspond à : \[
		\mathrm{d}f = \frac{\partial f}{\partial x}\mathrm{d}x + \frac{\partial f}{\partial y} \mathrm{d}y.
	\] En effet :
	\begin{align*}
		\mathrm{d}f &= f(x+ \mathrm{d}x, y + \mathrm{d}y) - f(x,y) \\
		&= \frac{\partial f}{\partial x} \mathrm{d}x + \frac{\partial f}{\partial y} \mathrm{d}y.
	\end{align*}
\end{rmk}

\begin{prop}
	Soit $f: U \to \R$ de classe $\mathcal{C}^1$ en $(a,b) \in U$. Alors, \[
		\forall w = (w_1, w_2) \in \R^2, \mathrm{d}f(w)\,(a,b) = w_1 \frac{\partial f}{\partial x}(a,b) + w_2 \frac{\partial f}{\partial y}(a,b).
	\]
\end{prop}

\begin{prv}
	Soit $w = (w_1, w_2) \in \R^2$. Soit $t \in \R^*$.
	\begin{align*}
		\frac{1}{t}\big(f(a + tw_1, b + tw_2) - f(a,b)\big)
		&= \frac{1}{t} \left( tw_1 \frac{\partial f}{\partial x}(a,b) + tw_2 \frac{\partial f}{\partial y}(a,b) + \po_{t \to 0}\big(\|tw\|\big) \right) \\
		&= w_1 \frac{\partial f}{\partial x}(a,b) + w_2 \frac{\partial f}{\partial y}(a,b) + \po_{t\to 0}(1) \\
		&\tendsto{t\to 0} w_1 \frac{\partial f}{\partial x}(a,b) + w_2\frac{\partial f}{\partial y}(a,b).
	\end{align*}
\end{prv}


\begin{defn}
	Avec les hypothèses précédentes, en posant \[
		\nabla f(a,b) = \left( \frac{\partial f}{\partial x}(a,b), \frac{\partial f}{\partial y}(a,b) \right) 
	\]on obtient \[
		\mathrm{d}f(w)\,(a,b) = \left<w  \mid \nabla f(a,b) \right>
	\] où $\left<\cdot|\cdot \right>$ est le produit scalaire.

	Le vecteur $\nabla f(a,b)$ est appelé \underline{gradient de $f$ en $(a,b)$}.

	Le développement limité à l'ordre 1 de $f$ devient \[
		f\big((a,b)+w\big) = f(a,b) + \left<w \mid \nabla f(a,b) \right> + \po_{w\to 0}(\|w\|)
	\]
\end{defn}

\begin{prop}
	Soit $f : U \to \R$ de classe $\mathcal{C}^1$.

	\begin{figure}[H]
    \centering
    \incfig{gradient}
	\end{figure}

	$\nabla f$ est orthogonal au lignes de niveaux de $f$, son orientation va dans le sens d'une augmentation de $f$.
\end{prop}

\begin{prv}
	Soit $\gamma : I \to U$ une courbe de niveau : \[
		\forall t \in I, f\big(\gamma(t)\big) = \text{cste}.
	\] D'après le lemme suivant : \[
		\forall t \in I, 0 = (f \circ \gamma)'(t) = \mathrm{d}f\big(\gamma'(t)\big)\big(\gamma(t)\big) = \left<\gamma'(t)  \mid \nabla f\big(\gamma(t)\big) \right>
	\] Donc $\nabla f\big(\gamma(t)\big)$ est orthogonal à $\gamma'(t)$.

	Pour tout $t \in I$, on pose $w(t) = t\, \nabla f\big(\gamma(t)\big)$. Donc \[
		f\big(\gamma(t) + w(t)\big) = f\big(\gamma(t)\big) + t \|\nabla f(\gamma(t))\|^2 + \po_{t \to 0}(t)
	\] Pour $t$ assez petit, $f\big(\gamma(t) + w(t)\big) - f\big(\gamma(t)\big)$ est du même signe que $t$.
\end{prv}

\begin{rmk}
	\begin{align*}
		V: \R^3 &\longrightarrow \R \\
		(x,y,z) &\longmapsto -mgz
	\end{align*}
	l'énerge potentielle de pesenteur

	On a donc \[
		\nabla V(x,y,z) = \left( \frac{\partial V}{\partial x}, \frac{\partial V}{\partial y}, \frac{\partial V}{\partial z} \right) = (0, 0, -mg) = \vec{P}.
	\]
\end{rmk}

\begin{lem}
	Soit $f : U \to \R$ de classe $\mathcal{C}^1$, $\gamma : \begin{array}{rcl}
		I &\longrightarrow& U \\
		t &\longmapsto& \big(x(t), y(t)\big)
	\end{array}$ où $x$ et $y$ sont dérivables.

	On pose \[
		\forall t \in I, \gamma'(t) = \big(x'(t), y'(t)\big).
	\] Alors $f \circ \gamma : I \to \R$ est dérivable et
	\begin{align*}
		\forall t \in I, (f \circ \gamma)'(t) &= \mathrm{d}f\big(\gamma'(t)\big) \big(\gamma(t)\big)\\
		&= \left<\gamma'(t)  \mid \nabla f\big(\gamma(t)\big)  \right> \\
		&= x'(t) \frac{\partial f}{\partial x}\big(x(t), y(t)\big) + y'(t) \frac{\partial f}{\partial y}\big(x(t),y(t)\big). \\
	\end{align*}
\end{lem}

\begin{prv}
	On fixe $t \in I$.

	\begin{align*}
		\forall h \neq 0, \frac{f \circ \gamma(t + h) - f \circ \gamma(t)}{h}
		&= \frac{1}{h}\big(f(\gamma(t)) + h\gamma'(t) + \po_{h\to 0}(h) - f(\gamma(t))\big) \\
		&= \frac{1}{h}\bigg(\cancel{f(\gamma(t))} + \left<h\gamma'(t) \mid \nabla f(\gamma(t)) \right> + \po_{h\to 0}(\|h\gamma'(t)\|) - \cancel{f(\gamma(t))}\bigg)\\
		&= \left<\gamma'(t) \mid \nabla f(\gamma(t)) \right> + \po_{h\to 0}(1) \\
		&\tendsto{h\to 0} \left<\gamma'(t)  \mid \nabla f(\gamma(t)) \right>
	\end{align*}
\end{prv}

\begin{defn}
	Soit $f : U \to \R$ de classe $\mathcal{C}^1$ et $(a,b) \in U$. On dit que $(a,b)$ est un \underline{point critique} de $f$ si $\nabla f(a,b) = 0$ i.e. $\frac{\partial f}{\partial x}(a,b) = \frac{\partial f}{\partial y}(a,b) = 0$.

	Dans ce cas, $f(a,b)$ est appelé \underline{valeur critique} de $f$.
\end{defn}

\begin{prop}~\\
	\begin{minipage}{\linewidth}
		\begin{wrapfigure}{r}{3cm}
			\centering
			\vspace{-1cm}
			\begin{asy}
				import solids;
				import graph;
				size(3cm);

				settings.render = 0;
				settings.prc = false;

				path3 par = graph(
					new real(real x) { return x; },
					new real(real x) { return 0; },
					new real(real x) { return -x^2; },
					0,3);
				revolution r = revolution(par, axis=Z);

				draw(r,1,longitudinalpen=nullpen);
				draw(r.silhouette());

				dot("$(a,b)$", O, red, align=N);
				real s = sqrt(2.5);
				path3 g=(s,0,-2.5)..(0,s,-2.5)..(-s,0,-2.5)..(0,-s,-2.5)..cycle;
				draw(g, deepcyan);
			\end{asy}
		\end{wrapfigure}
		Soit $f: U \to \R$ de classe $\mathcal{C}^1$ et $(a,b) \in U$ tel que \[
			\exists r > 0, \forall (x,y) \in B_{(a,b)}(r), f(x,y) \le f(a,b)
		\] Alors $\nabla f(a,b) = (0,0)$.
	\end{minipage}
\end{prop}

\begin{prv}
	Soit $g: x \mapsto f(x,b)$. $g(a)$ est un maximum local de $g$ donc $g'(a) = 0$.

	Or, $g'(a) = \frac{\partial f}{\partial x}(a,b)$

	donc $\frac{\partial f}{\partial x}(a,b) = 0$.

	Soit $h : y \mapsto f(a,y)$. On a de même $h'(b) = 0$.

	Or, $h'(b) = \frac{\partial f}{\partial y}(a,b)$.

	Donc, $\nabla f(a,b) = (0,0)$.
\end{prv}

\begin{rmk}
	Un minimum local est aussi une valeur critique.
\end{rmk}

\begin{figure}[H]
	\centering
	\begin{subfigure}{3cm}
		\centering
		\begin{asy}
			import solids;
			import graph;
			size(3cm);

			settings.render = 0;
			settings.prc = false;

			path3 par = graph(
				new real(real x) { return x; },
				new real(real x) { return 0; },
				new real(real x) { return -x^2; },
				0,3);
			revolution r = revolution(par, axis=Z);

			draw(r,1,longitudinalpen=nullpen);
			draw(r.silhouette());

			dot(O, red);
		\end{asy}
		\caption{Maximum local}
	\end{subfigure}
	\begin{subfigure}{3cm}
		\centering
		\begin{asy}
			import solids;
			import graph;
			size(3cm);

			settings.render = 0;
			settings.prc = false;

			path3 par = graph(
				new real(real x) { return x; },
				new real(real x) { return 0; },
				new real(real x) { return x^2; },
				0,3);
			revolution r = revolution(par, axis=Z);

			draw(r,1,longitudinalpen=nullpen);
			draw(r.silhouette());

			dot(O, red);
		\end{asy}
		\caption{Minimum local}
	\end{subfigure}
	\begin{subfigure}{3cm}
		\centering
		\begin{asy}
			import solids;
			import graph;
			size(3cm);

			settings.render = 0;
			settings.prc = false;
			currentprojection = obliqueZ;

			draw(graph(
				new real(real x) { return x; },
				new real(real x) { return -x^2 / 3; },
				new real(real x) { return 3; },
				-3, 3
			));

			draw(graph(
				new real(real x) { return x; },
				new real(real x) { return -x^2 / 3; },
				new real(real x) { return -3; },
				-3, 3
			));

			draw(graph(
				new real(real x) { return x; },
				new real(real x) { return -x^2 / 3 - 1; },
				new real(real x) { return 0; },
				-3, 3
			));

			draw(graph(
				new real(real x) { return 0; },
				new real(real x) { return x^2 / 9 - 1; },
				new real(real x) { return x; },
				-3, 3
			));

			draw(graph(
				new real(real x) { return -3; },
				new real(real x) { return x^2 / 9 - 4; },
				new real(real x) { return x; },
				-3, 3
			));

			draw(graph(
				new real(real x) { return 3; },
				new real(real x) { return x^2 / 9 - 4; },
				new real(real x) { return x; },
				-3, 3
			));

			dot((0,-1,0), red);
		\end{asy}
		\caption{Point de selle / Point col}
	\end{subfigure}
\end{figure}

\begin{exm}
	On revient à l'exemple donné en introduction : 
	\begin{align*}
		f: \left( \R^*_+ \right)^2 &\longrightarrow \R \\
		(x,y) &\longmapsto 2\left( xy + \frac{1}{x} + \frac{1}{y} \right).
	\end{align*}

	$\left( \R^+_* \right)^2$ est un ouvert de $\R^2$. Soit $(x,y) \in \left( \R^+_* \right)^2$.
	
	On a \[
		\begin{cases}
			\frac{\partial f}{\partial x}(x,y) = 2\left( y - \frac{1}{x^2} \right),\\
			\frac{\partial f}{\partial y}(x,y) = 2\left( x - \frac{1}{y^2} \right).
		\end{cases}
	\]

	\begin{align*}
		&\frac{\partial f}{\partial x}(x,y) = \frac{\partial f}{\partial y}(x,y) = 0\\
		\iff& \begin{cases}
			y = \frac{1}{x^2}\\
			x = \frac{1}{y^2}
		\end{cases}\\
		\iff& \begin{cases}
			y = \frac{1}{x^2}\\
			x = x^4
		\end{cases}\\
		\iff& \begin{cases}
			x = 1\\
			y = 1
		\end{cases}
	\end{align*}

	On vérivie que $f$ présente en effet un minium local en $(1,1)$. \[
		f(1,1) = 6
	\] On fixe $y \in \R^+_*$ et \[
		g : x \mapsto 2\left( xy + \frac{1}{x} + \frac{1}{y} \right).
	\] Donc \[
		\forall x \in \R^+_*, g'(x) = 2\left( y - \frac{1}{x^2} \right).
	\]
	\begin{center}
		\begin{tikzpicture}
			\tkzTabInit{$x$/1,$g'(x)$/1,$g$/2.3}{$0$, $\frac{1}{\sqrt{y}}$, $+\infty$}
			\tkzTabLine{,-,z,+,}
			\tkzTabVar{+/{}, -/$2\left( 2\sqrt{y} +\frac{1}{y} \right)$, +/{}}
		\end{tikzpicture}
	\end{center}
	
	Ainsi, \[
		\forall x \in \R^+_*, \forall y \in \R^+_*, f(x,y) \ge 2\left( 2\sqrt{y} + \frac{1}{y} \right)
	\] Soit $h : y \mapsto 2\sqrt{y} + \frac{1}{y}$. On a \[
		\forall y > 0, h'(y) = \frac{1}{\sqrt{y}} - \frac{1}{y^2} = \frac{y\sqrt{y} - 1}{y^2} = \frac{y^{\frac{3}{2}} - 1}{y^2}
	\]

	\begin{center}
		\begin{tikzpicture}
			\tkzTabInit{$y$/0.7,$h'(y)$/0.7,$h$/1.4}{$0$, $1$, $+\infty$}
			\tkzTabLine{,-,z,+,}
			\tkzTabVar{+/{}, -/$3$, +/{}}
		\end{tikzpicture}
	\end{center}

	Donc, \[
		\forall x,y > 0, f(x,y) \ge 2\times 3 = 6 = f(1,1).
	\]
\end{exm}

\begin{prop}
	[règle de la chaîne]

	Soit $f : \begin{array}{rcl}
		U &\longrightarrow& \R^2 \\
		(x,y) &\longmapsto& f(x,y)
	\end{array}$ de classe $\mathcal{C}^1$ et $U, V$ deux ouverts de $\R^2$.

	Soit $\varphi : \begin{array}{rcl}
		V &\longrightarrow& U \\
		(u,v) &\longmapsto& \varphi(u,v) = \big(x(u,v), y(u,v)\big)
	\end{array}$.

	On suppose que $x$ et $y$ sont de classe $\mathcal{C}^1$ sur $V$.

	Alors,  $f \circ \varphi : \begin{array}{rcl}
		V &\longrightarrow& \R \\
		(u,v) &\longmapsto& f\big(\varphi(u,v)\big)
	\end{array}$ est de classe $\mathcal{C}^1$ et
	\begin{align*}
		\forall (u_0, v_0) \in V, \frac{\partial (f \circ \varphi)}{\partial u}(u_0, v_0)
		&= \frac{\partial f}{\partial x}\big(\varphi(u_0, v_0)\big) \times \frac{\partial x}{\partial u}(u_0, v_0)\\
		&+ \frac{\partial f}{\partial y}\big(\varphi(u_0,v_0)\big) \frac{\partial y}{\partial u}(u_0,v_0)
	\end{align*}
	\begin{align*}
		\forall (u_0, v_0) \in V, \frac{\partial (f \circ \varphi)}{\partial v}(u_0, v_0)
		&= \frac{\partial f}{\partial x}\big(\varphi(u_0, v_0)\big) \times \frac{\partial x}{\partial v}(u_0, v_0)\\
		&+ \frac{\partial f}{\partial y}\big(\varphi(u_0,v_0)\big) \frac{\partial y}{\partial v}(u_0,v_0)
	\end{align*}
\end{prop}

\begin{exm}
	[changement de coordonnées polaires]
	On pose \begin{align*}
		\varphi: \R^+_* \times ]0,2\pi[ &\longrightarrow \R^2\setminus \left( R^+_* \times \{0\} \right) \\
		(r, \theta) &\longmapsto (r \cos \theta, r \sin\theta),
	\end{align*}
	\begin{align*}
		f: \R^2\setminus \left( R^+_* \times \{0\} \right) &\longrightarrow \R \\
		(x,y) &\longmapsto f(x,y),
	\end{align*}
	\begin{align*}
		g: \overbrace{\R^+_* \times ]0, 2\pi[}^{=V} &\longrightarrow \R \\
		(r, \theta) &\longmapsto f(r\cos\theta, r\sin\theta).
	\end{align*}

	\begin{align*}
		\forall (r_0,\theta_0) \in V,&\\[5mm]
		\frac{\partial g}{\partial r}(r_0, \theta_0) &= \frac{\partial f}{\partial x}(r_0\cos\theta_0, r_0\sin\theta_0)\cos\theta_0\\
		&+ \frac{\partial f}{\partial y}(r_0 \cos\theta_0, r_0\sin\theta_0)\sin\theta_0\\
		&= 2r_0\cos^2\theta_0 + 2r_0\sin^2(\theta_0) \\
		&= 2r_0 \\[5mm]
		\frac{\partial g}{\partial \theta}(r_0, \theta_0) &= \frac{\partial f}{\partial x}(r_0\cos\theta_0, r_0\sin\theta_0)r_0\sin\theta_0\\
		&+ \frac{\partial f}{\partial y}(r_0 \cos\theta_0, r_0\sin\theta_0)r_0\cos\theta_0\\
		&= -2{r_0}^2\cos(\theta_0)\sin(\theta_0) + 2{r_0}^2 \sin(\theta_0)\cos(\theta_0)\\
		&= 0 \\
	\end{align*}

	Donc, \[
		g(r, \theta) = r^2.
	\]
\end{exm}

\begin{exm}
	Résoudre \[
		\begin{cases}
			\frac{\partial f}{\partial x} = \frac{x}{x^2+y^2},\\
			\frac{\partial f}{\partial y} = \frac{y}{x^2+y^2}.\\
		\end{cases}
	\]

	On pose $g: (r, \theta) \mapsto f(r \cos\theta, r \sin\theta)$.

	\begin{align*}
		&\frac{\partial g}{\partial r} = \frac{1}{r}\cos^2\theta + \frac{1}{r}\sin^2\theta = \frac{1}{r},\\
		&\frac{\partial g}{\partial \theta} = -\cos(\theta) \sin(\theta) + \sin(\theta)\cos(\theta) = 0.
	\end{align*}

	Donc, \[
		\exists C \in \R, g: (r, \theta) \mapsto \ln r + C
	\] d'où,
	\begin{align*}
		\forall (x,y) \in \R^2 \setminus \{(0,0)\}, f(x,y) &= \ln\left(\sqrt{x^2 + y^2} \right)  + C\\
		&= \frac{1}{2}\ln(x^2 + y^2) + C. \\
	\end{align*}
\end{exm}

\begin{rmk}
	Soit $\mathcal{B} = (e_1, e_2)$ la base canonique de $\R^2$, $f: U \to \R$ de classe $\mathcal{C}^1$ avec $U$ un ouvert de $\R^2$.

	Soit $(x,y) \in U$.

	\begin{align*}
		\Mat_{\mathcal{B}}\big(\nabla f(x,y)\big) = \begin{pmatrix}
			\frac{\partial f}{\partial x}(x,y)\\[2mm]
			\frac{\partial f}{\partial y}(x,y)
		\end{pmatrix}
	\end{align*}

	Soit  \begin{align*}
		\varphi: V &\longrightarrow U \\
		(u,v) &\longmapsto \big(x(u,v), y(u,v)\big) 
	\end{align*} avec $x,y$ de classe $\mathcal{C}^1$. Soit $g = f \circ \varphi$.
	\begin{align*}
		\Mat_{\mathcal{B}}\big(\nabla g(u,v)\big)
		&= \begin{pmatrix}
			\frac{\partial g}{\partial u}(u,v) \\[2mm]
			\frac{\partial g}{\partial v}(u,v)
		\end{pmatrix} \\
		&= \begin{pmatrix}
			\frac{\partial x}{\partial u}(u,v) \frac{\partial f}{\partial x}(x,y)
			+ \frac{\partial y}{\partial u}(u,v)\frac{\partial f}{\partial y}(x,y)\\[3mm]
			\frac{\partial x}{\partial v}(u,v) \frac{\partial f}{\partial x}(x,y)
			+ \frac{\partial y}{\partial v}(u,v) \frac{\partial f}{\partial y}(x,y)
		\end{pmatrix}  \\
		&= \underbrace{\begin{pmatrix}
				\frac{\partial x}{\partial u}(u,v)& \frac{\partial y}{\partial u}(u,v)\\[3mm]
				\frac{\partial x}{\partial v}(u,v)& \frac{\partial y}{\partial v}(u,v)
		\end{pmatrix}}_{J(u,v)} \begin{pmatrix}
			\frac{\partial f}{\partial x}(x,y)\\[3mm]
			\frac{\partial f}{\partial y}(x,y)
		\end{pmatrix} \\
		&= J(u,v) \Mat_{\mathcal{B}}\big(\nabla f(x,y)\big) \\
	\end{align*}
	où $J(u,v) = 
	\begin{pNiceArray}{c:c}
		\Mat_{\mathcal{B}}\big(\nabla x(u,v)\big) & \Mat_{\mathcal{B}}\big(\nabla y(u,v)\big)
	\end{pNiceArray}$.

	On dit que $J(u,v)$ est \underline{la jacobienne} de $\varphi$ en $(u,v)$.
	L'application linéaire canoniquement associée à $J(u,v)$ est la \underline{différentielle de $\varphi$} en $(u,v)$ noté $\mathrm{d}\varphi(u,v)$.

	On a $\mathrm{d}\varphi(u,v) \in \mathcal{L}(R^2)$ et $\Mat_{\mathcal{B}}\big(\mathrm{d}\varphi(u,v)\big) = J(u,v)$.

	Par exemple, la jacobienne du changement de coordonnées polaires est \[
		J = \begin{pmatrix}
			\frac{\partial x}{\partial r} & \frac{\partial y}{\partial r}\\[3mm]
			\frac{\partial x}{\partial \theta} & \frac{\partial y}{\partial \theta}
		\end{pmatrix}
		= \begin{pmatrix}
			\cos\theta&\sin\theta\\
			-r\sin\theta&r\cos\theta
		\end{pmatrix}.
	\]
	$\underbrace{\det(J)}_{\text{le jacobien}} = r\cos^2\theta + r\sin^2\theta = r$

	Dans une intégrale double, si $(x,y) = \varphi(u,v)$, alors $\mathrm{d}x\mathrm{d}y = \det(J)\mathrm{d}u\mathrm{d}v$.

	Ici, \[
		\mathrm{d}x\ \mathrm{d}y = r\ \mathrm{d}r\ \mathrm{d}\theta.
	\]
\end{rmk}

\begin{prv}
	On pose $(x_0, y_0) = \varphi(u_0, v_0)$. Pour tout $(h,k) \in \R^2$ tels que $(u_0 + h, v_0 + k) \in V$, en posant $g = f  \circ \varphi$.

	\begin{align*}
		g(u_0 + h, v_0 + h) &= f\big(x(u_0 + h, v_0 + k), y(u_0 + h, v_0 + k)\big) \\
		&= f\left(
			x(u_0,v_0) + h \frac{\partial x}{\partial u}(u_0,v_0) + k \frac{\partial x}{\partial v}(u_0, v_0) + \po\big(\|(h,k)\|\big), \right.\\
		&\phantom{ = f\bigg(\bigg.}\left. y(u_0, v_0) + h \frac{\partial y}{\partial u}(u_0, v_0) + k \frac{\partial y}{\partial v}(u_0, v_0) + \po\big(\|(h,k)\|\big)
		\right)  \\
		&= f(x_0,y_0) \\
		&~+ \left( h \frac{\partial x}{\partial u}(u_0,v_0) + k \frac{\partial x}{\partial v}(u_0, v_0) + \po(\|(h,k)\|) \right) \frac{\partial f}{\partial x}(x_0,y_0)\\
		&~+ \left( h \frac{\partial y}{\partial u}(u_0, v_0) + k\frac{\partial y}{\partial v}(u_0, v_0) + \po(\|(h,k)\|) \right) \frac{\partial f}{\partial y}(x_0, y_0)\\
		&~+ \po(\|(h,k)\|)\\
		&= f(x_0, y_0) \\
		&~+ h \left( \frac{\partial x}{\partial u}(u_0, v_0) \frac{\partial f}{\partial x}(x_0, y_0) + \frac{\partial y}{\partial u}(u_0, v_0) \frac{\partial f}{\partial y}(x_0, y_0) \right)  \\
		&~+ k\left( \frac{\partial x}{\partial v}(u_0, v_0) \frac{\partial f}{\partial x}(x_0, y_0) + \frac{\partial y}{\partial v}(u_0, v_0) \frac{\partial f}{\partial y}(x_0, y_0) \right) 
		&~+ \po(\|(h,k)\|)\\
		&= g(u_0, v_0) + h \frac{\partial g}{\partial u}(u_0, v_0) + k \frac{\partial g}{\partial v}(u_0, v_0) + \po(\|(h,k)\|) \\
	\end{align*}

	Par identification,
	\[
		\frac{\partial g}{\partial u}(u_0, v_0) = \frac{\partial x}{\partial u}(u_0, v_0) \frac{\partial f}{\partial x}(x_0, y_0) + \frac{\partial y}{\partial u}(u_0, v_0) \frac{\partial f}{\partial y}(x_0,y_0)
	\] et \[
		\frac{\partial g}{\partial v}(u_0, v_0) = \frac{\partial x}{\partial v}(u_0,v_0) \frac{\partial f}{\partial x}(x_0, y_0) + \frac{\partial y}{\partial v}(u_0, v_0) \frac{\partial f}{\partial y}(x_0, y_0).
	\] 
\end{prv}

\begin{exm}
	[Régression linéaire]~\\
	\begin{figure}[H]
		\centering
		\begin{asy}
			import graph;
			axes(EndArrow);
			size(5cm);

			real f(real x) { return x + 0.5; }

			real k = 35 / (7 - 0.5);

			for(int i = 0; i < 35; ++i) {
				real mag = exp(sin(100 * pi/exp(1) * i)) * 0.8 + exp(cos(i*40)/3);
				real eps = mag * cos(10 * exp(1)/pi * i) / 3;
				dot((i/k,f(i/k) + eps));
			}

			draw(graph(f, -1, 7), orange);
		\end{asy}
	\end{figure}
	\[
		y = a x + b
	\] 
	On fixe $(a,b) \in \R^2$. \[
		\varepsilon(a,b) = \sum_{i=1}^n\big( y_i - (ax_i + b) \big)^2
	\] l'erreur totale.

	On veut minimiser $\varepsilon(a,b)$. On a 
	\[
		\forall (a,b) \in \R^2,
		\begin{cases}
			\frac{\partial \varepsilon}{\partial a}(a,b) = -2\sum_{i=1}^{n}(y_i - ax_i - b)x_i,\\
			\frac{\partial \varepsilon}{\partial b}(a,b) = -2\sum_{i=1}^{n}(y_i - ax_i - b).
		\end{cases}
	\]

	Donc,
	\begin{align*}
		(a,b) \text{ point critique de } \varepsilon \iff& \begin{cases}
			a \sum_{i=1}^n {x_i}^2 + b\sum_{i=1}^{n}x_i = \sum_{i=1}^{n} y_ix_i\\
			a\sum_{i=1}^{n}x_i + nb = \sum_{i=1}^ny_i
		\end{cases}\\
		\iff& \begin{cases}
			a \left( \frac{1}{n}\sum_{i=1}^n {x_i}^2 - \overline{x}^2\right) = \overline{y} - \overline{x} \overline{y}\\
			b = \frac{1}{n}\sum_{i=1}^ny_i - \frac{a}{n}\sum_{i=1}^nx_i = \frac{1}{n}\sum_{i=1}^n x_i y_i - \overline{x} \overline{y}
		\end{cases}\\
		&\text{ où } \overline{x} = \frac{1}{n} \sum_{i=1}^n x_i,~\overline{y} = \frac{1}{n}\sum_{i=1}^n y_i\\
		\iff& \begin{cases}
			a = \frac{\Cov(x,y)}{V(x)}\\
			b = \overline{y} - a\overline{x}
		\end{cases}
	\end{align*}

	Coefficient de corrélation: $\frac{\Cov(x,y)}{\sigma_x \sigma_y} \in [-1, 1]$
\end{exm}












		\part{Corps}

\begin{exm}[Problème]
	\begin{itemize}
		\item 
			avec $A = \Z / 9 \Z$, résoudre $\overline{x}^2 = \overline{0}$ \\
			\begin{center}
				\begin{tabular}{|c|c|c|c|c|c|c|c|c|c|c|}
					\hline
					$\overline{x}$&$\overline{0}$& $\overline{1}$ &$\overline{2}$&$\overline{3}$ &$\overline{4}$ &$\overline{5}$ &$\overline{6}$ &$\overline{7}$ &$\overline{8}$& $\overline{9}$ \\
					\hline
					$\overline{x}^2$&$\overline{0}$ &$\overline{1}$ &$\overline{4}$ &$\overline{0}$ &$\overline{7}$ &$7$ &$\overline{0}$ &$\overline{4}$ &$\overline{1}$&$\overline{0}$\\
					\hline
				\end{tabular}
			\end{center}
			On a trouvé 3 solutions: $\overline{0}$, $\overline{3}$, $\overline{6}$.
		\item $\Z / 8\Z$
			\begin{center}
				\begin{tabular}{|c|c|c|c|c|c|c|c|c|}
					\hline
					$\overline{x}$& $\overline{0}$& $\overline{1}$& $\overline{2}$& $\overline{3}$& $\overline{4}$& $\overline{5}$& $\overline{6}$& $\overline{7}$\\
					\hline
					$\overline{x^2}$& $\overline{0}$& $\overline{1}$& $\overline{4}$& $\overline{1}$& $\overline{0}$& $\overline{1}$& $\overline{4}$& $\overline{1}$\\
					\hline
				\end{tabular}
			\end{center}
			$\overline{x}^2=7$ a 4 solutions: $\overline{1}, \overline{7}, \overline{3},\text{ et } \overline{5}$
		\item $A = \mathbbm{H} = \{a + bi + cj + dk  \mid  (a,b,c,d) \in \R^4\}$ \\
			$i^2 = j^2 = k^2 = -1$ 
			\begin{align*}
				\begin{array}{c c c}
					ij = k & jk = i & ji = j\\
					ji = -k & kj = -i & ik = -j
				\end{array}
			\end{align*}
			Dans cet anneau, $-1$ a 6 racines!
	\end{itemize}
\end{exm}

\begin{defn}
	Soit $(\mathbbm{K}, +, \times)$ un ensemble muni de deux lois de composition internes. On dit que c'est un \underline{corps} si
	 \begin{enumerate}
		\item $(\mathbbm{K}, \times)$ est un groupe abélien
		\item $(\mathbbm{K}, \times)$ est un monoïde commutatif
		\item $\forall x \in \mathbbm{K}\setminus \{0_\mathbbm{K}\}, \exists y \in \mathbbm{K}, xy = 1_\mathbbm{K}$
		\item $0_\mathbbm{K} \neq  1_\mathbbm{K}$
	\end{enumerate}
	\index{corps}
\end{defn}

\begin{exm}
	\begin{itemize}
		\item $(\C, +, \times)$ est un corps
		\item $(\R, +, \times)$ est un corps
		\item $(\Q, +, \times)$ est un corps
		\item $(\Z, +, \times)$ n'est pas un corps
	\end{itemize}
\end{exm}

\begin{prop}
	$(\Z / n\Z, +, \times)$ est un corps si et seulement si $n$ est premier.
\end{prop}

\begin{prv}
	\[
		\left( \Z / n\Z \right)^\times = \left\{ \overline{k}  \mid k \wedge n = 1 \right\}
	\] 
\end{prv}


\begin{prop}
	Tout corps est un anneau intègre.
\end{prop}

\begin{prv}
	Soit $(\mathbbm{K}, +, \times)$ un corps. Soient $(a,b) \in \mathbbm{K}^2$ tel que $a \times b = 0_\mathbbm{K}$.\\
	On suppose $a \neq  0_\mathbbm{K}$. Alors, $a$ est inversible et donc \[
		b = a^{-1} \times a \times b = a^{-1} \times 0_\mathbbm{K} = 0_\mathbbm{K}
	\] 
\end{prv}

\begin{exm}
	Soit $(\mathbbm{K},+,\times)$ un corps.\\
	Résoudre \[
		\begin{cases}
			x^2 = 1_\mathbbm{K}\\
			x \in \mathbbm{K}
		\end{cases}
	\]

	\begin{align*}
		x^2 = 1_\mathbbm{K} &\iff x^2 - 1_\mathbbm{K} = 0_\mathbbm{K}\\
		&\iff (x - 1_\mathbbm{K})(x+1_\mathbbm{K}) = 0_\mathbbm{K}\\
		&\iff x - 1_\mathbbm{K} = 0_\mathbbm{K} \text{ ou } x + 1_\mathbbm{K} = 0_\mathbbm{K}\\
		&\iff x = 1_\mathbbm{K} \text{ ou } x = -1_\mathbbm{K}
	\end{align*}

	Il y a au plus 2 solutions.
\end{exm}

\begin{prop}
	Soit $(\mathbbm{K},+,\times )$ un corps et $P$ un polynôme à coefficients dans $\mathbbm{K}$ de degré $n$. Alors, l'équation $P(x) = 0_{\mathbbm{K}}$ a au plus $n$ solutions dans $\mathbbm{K}$ 
	\qed
\end{prop}

\begin{crlr}[(Théorème de Wilson)]
	voir exercice 16 du TD 12
\end{crlr}


\begin{defn}
	Soit $(\mathbbm{K}, +, \times)$ un corps et $L\subset \mathbbm{K}$.\\
	On dit que $L$ est un \underline{sous corps} de $\mathbbm{K}$ si
	\begin{enumerate}
		\item $L$ est un anneau de $(\mathbbm{K}, +, \times)$ non nul
		\item $\forall x \in L\setminus \{0_\mathbbm{K}\}, x^{-1} \in L$ 
	\end{enumerate}
	\vspace{2mm}
	en d'autres termes si
	\begin{enumerate}
		\item $\forall (x,y) \in L^2, x - y \in L$
		\item $\forall (x,y) \in L^2, x \times y^{-1} \in L$
	\end{enumerate}
	\vspace{5mm}
	On dit aussi que $\mathbbm{K}$ est une \underline{extension} de $L$.
	\index{sous corps}
	\index{extension}
\end{defn}

\begin{prop}
	Tout sous corps est un corps. \qed
\end{prop}

\begin{defn}
	Soient $(\mathbbm{K}_1,+,\times )$ et $(\mathbbm{K}_2,+, \times)$ deux corps et $f: \mathbbm{K}_1 \to \mathbbm{K}_2$.\\
	On dit que $f$ est un \underline{morphisme de corps} si $f$ est un morphisme d'anneaux.\\
	i.e. si
	\[
		\begin{cases}
			\forall (x,y) \in {\mathbbm{K}_1}^2,& f(x+y) = f(x) + f(y)\\
			\forall (x,y) \in {\mathbbm{K}_1}^2,& f(x \times y) = f(x) \times f(y)\\
		\end{cases}
	\] 
	\index{homomorphisme (de corps)}
	\index{morphisme (de corps)}
\end{defn}

\begin{prop}
	Tout morphisme de corps est injectif.
\end{prop}

\begin{prv}
	Soit $f: \mathbbm{K}_1 \to \mathbbm{K}_2$ un morphisme de corps.\\
	\begin{itemize}
		\item $\Ker(f)$ est un sous groupe de $(\mathbbm{K}_1, +)$ 
		\item Soit $x \in \Ker(f)$ et $y \in \mathbbm{K}_1$ \[
				f(x \times y) = f(x) \times f(y) = 0_{\mathbbm{K}_2} \times f(y) = 0_{\mathbbm{K}_2}
			\]
		\item Soit $x \in \Ker(f) \setminus \{0_{\mathbbm{K}_1}\}$.\\
			Alors, $x$ est inversible.\\
			\begin{align*}
				\begin{rcases*}
					x \in \Ker(f)\\
					x^{-1} \in \mathbbm{K}_1
				\end{rcases*}& \text{ donc } x \times x ^{-1} \in \Ker(f)\\
				&\text{ donc } 1_{\mathbbm{K}_1} \in \Ker(f)\\
				&\text{ donc } f(1_{\mathbbm{K}_1}) = 0_{\mathbbm{K}_2}
			\end{align*}
			Or, $f(1_{\mathbbm{K}_1}) = 1_{\mathbbm{K}_2} \neq 0_{\mathbbm{K}_2}$
	\end{itemize}
	Donc, $\Ker(f) = \{0_{\mathbbm{K}_1}\}$ donc $f$ est injective.
\end{prv}

\begin{exm}
	$\begin{array}{cc}
		\C &\longrightarrow \C\\
		z &\longmapsto \overline{z}\\
	\end{array}$ est un morphisme de corps
\end{exm}



		\part{Opérations sur les séries}

\begin{prop}
	L'ensemble $E = \{u \in \C^\N  \mid \Sigma u_n \text{ converge}\}$ est un sous-espace vectoriel de $\C^\N$ et \begin{align*}
		S: E &\longrightarrow \C \\
		u &\longmapsto \sum_{n=0}^{+\infty} u_n
	\end{align*} est une forme linéaire.
	\qed
\end{prop}

\begin{rmk}
	La somme d'une série convergente et d'une série divergente diverge.
	Le produit d'une série divergente par un scalaire non nul diverge.
\end{rmk}

		\part{Comparaison de suites}

\begin{defn}
	Soient $u$ et $v$ deux suites réelles. On dit que $u$ est \underline{dominée} par  $v$ si \[
	\exists M\in \R, \exists N\in \N,\forall n\ge N,\left| u_n \right| \le M \left| v_n \right| 
	\] Dans ce cas, on note $u = O(v)$ ou $u_n = O(v_n)$ et on dit que "$u$ est un grand o de $v$"
\end{defn}

\begin{exm}
	En informatique, on dit qu'un alogirithme a une \underline{complexité linéaire} si son temps d'éxécution est un $O(n)$ 
	Par exemple, on calcule $a^n$ 

	\begin{itemize}
		\item Approche naïve
			\begin{algorithm}
				\begin{algorithmic}[1]
					\State $p \gets 1$
					\For{$i \in \left\llbracket 0,n-1 \right\rrbracket$}
						\State $p \gets p \times a$
					\EndFor
					\State \Return p
				\end{algorithmic}
			\end{algorithm}
			Complexité linéaire $O(n)$
		\item Exponentiation rapide\\
			On écrit $n$ en binaire: \begin{align*}
				n &= \overline{a_k a_{k-1}\ldots a_0}^{(2)}\\
					&= \sum_{i=0}^{k} a_i 2^i
			\end{align*} avec $(a_i) \in \left\{ 0,1 \right\} ^{k+1}$
			\begin{align*}
				a^n &= a^{\sum_{i=0}^{k} a_i 2^i} \\
				&= \prod_{i=0}^{k} a^{a_i 2^i}  \\
			\end{align*}
			
			\begin{algorithm}
				\begin{algorithmic}
					[1]

					\State $s \gets 0$
					\State $p \gets a$
					\For{ $i \in \left\llbracket 0, \log_2(n) \right\rrbracket$}
						\State $p \gets p \times p$
						\If{$a[i] = 1$}
							\State $s \gets s + p$
						\EndIf
					\EndFor
					\State \Return s
				\end{algorithmic}
			\end{algorithm}
			Compléxité logarithmique $O(\log_2(n))$
	\end{itemize}
\end{exm}


\begin{prop}
	$O$ est une relation réfléctive et transitive.
\end{prop}

\begin{prv}
	\begin{itemize}
		\item Soit $u$ une suite. On pose $M = 1$ et \[
			\forall n \in \N, \left| u_n \right| \le M \left| u_n \right|
			\] Donc $u = O(u)$.
		\item Soient $u, v, w$ trois suites telles que  \[
		\begin{cases}
			u = O(v)\\
			v = O(w)
		\end{cases}
		\] Soient $M_1,M_2 \in \R$ et $N_1,N_2\in \N$ tels que \[
		\begin{cases}
			\forall n \ge  N_1, \left| u_n \right| \le M_1 \left| v_n \right| \\
			\forall n \ge  N_2, \left| v_n \right| \le M_2 \left| w_n \right| \\
		\end{cases}
		\] 

		Nécéssairement, $M_1\ge 0$ et $M_2\ge 0$.\\
		Soit $N = \max(N_1,N_2)$. \[
		\forall n \ge  N, \left| u_n \right| \le M_1 \left| v_n \right| \le  M_1M_2 \left| w_n \right| 
		\] Donc $u = O(w)$
	\end{itemize}
\end{prv}

\begin{defn}
	Soient $u$ et $v$ deux suites. On dit que $u$ est \underline{négligeable} devant $v$ si \[
	\forall \varepsilon>0, \exists N\in \N, \forall n\ge N, \left| u_n \right| \le \varepsilon \left| v_n \right| 
	\] Dans ce cas, on note $u = o(v)$ ou $u_n = o(v_n)$ ou on le lit "$u$ est un petit o de $v$"
\end{defn}

\begin{prop}
	$o$ est une relation transitive, non-réfléctive
\end{prop}

\begin{prv}
	\begin{itemize}
		\item Soient $u$, $v$ et $w$ trois suites telles que \[
			\begin{cases}
				u = o(v)\\
				v = o(w)
			\end{cases}
			\] Soit $\varepsilon>0$. Soit $N_1\in \N$ tel que \[
			\forall n \ge N_1, \left| u_n \right| \le \sqrt{\varepsilon}  \left| v_n \right| 
			\] Soit $N_2\in \N$ tel que \[
			\forall n \ge N_2, \left| v_n \right| \le \sqrt{\varepsilon}  \left| w_n \right| 
			\] On pose $N = \max(N_1,N_2)$, alors \[
			\forall n \ge N, \left| u_n \right| \le \sqrt{\varepsilon}  \left| v_n \right| \le \underbrace{\sqrt{\varepsilon} \times \sqrt{\varepsilon}} _\varepsilon \left| w_n \right| 
			\] donc $u = o(w)$
		\item Soit $u$ une suite tel qu'il existe $N \in \N$ tel que \[
		\forall n \ge N, u_n > 0
		\] On suppose que $u = o(u)$, alors \[
		\forall \varepsilon>0,\exists N \in \N, \forall n \ge N, \left| u_n \right| \le \varepsilon \left| u_n \right| 
		\] On pose $\varepsilon = \frac{1}{2}$ alors \[
		\exists N \in \N, \forall n \ge N, \left| u_n \right| \le \frac{1}{2} \left| u_n \right| 
		\] une contradiction
	\end{itemize}
\end{prv}

\begin{prop}
	Soient $u$ et $v$ deux suites.
	\begin{itemize}
		\item $o(u) + o(u) = o(u)$
		\item $v \times o(u) = o(uv)$
		\item $o(u) \times o(v) = o(uv)$
		\item $o(o(u)) = o(u)$
	\end{itemize}
	\qed
\end{prop}

\begin{defn}
	Soient $u$ et $v$ deux suites. On dit que $u$ et $v$ sont \underline{équivalentes} si \[
	u = v + o(v)
	\] i.e. \[
	\forall \varepsilon >0, \exists N \in \N, \forall n \ge N, \left| u_n-v_n \right| \le \varepsilon\left| v_n \right| 
	\] Dans ce cas, on le note $u \sim v$
\end{defn}

\begin{prop}
	$\sim$ est une relation d'équivalence \qed
\end{prop}

\begin{prop}
	Soient $(u,v) \in \R^\N$. On suppose que $v$ ne s'annule pas à partir d'un certain rang
	\begin{enumerate}
		\item $u = o(v) \iff \left( \frac{u_n}{v_n} \right)$ bornée
		\item $u = o(v) \iff \frac{u_n}{v_n} \tendsto{n \to  +\infty} 0$
		\item $u \sim v \iff \frac{u_n}{v_n} \tendsto{n \to  +\infty} 1$
	\end{enumerate}
	\qed
\end{prop}

\begin{prop}
	[Suites de références]
	\begin{enumerate}
		\item $\ln^\alpha(n) = o(n^\beta)$ avec $(\alpha,\beta) \in \left( \R^+_* \right) ^2$ 
		\item $n^\beta = o(a^n)$ avec $\beta > 0$ et $a > 1$ 
		\item $a^n = o(n!)$ avec $a >1$ 
		\item $n! = o(n^n)$
	\end{enumerate}
\end{prop}


\begin{lem}
	[Exercice 10 du TD]
	Soit $u \in \left(\R^+_*\right)^\N$\\
	Si $\frac{u_{n+1}}{u_n} \tendsto{n \to +\infty} \ell < 1$ avec $\ell\in \R$,\\ alors $u_n \tendsto{n \to +\infty} 0$
\end{lem}

\begin{prv} [de la proposition]
	\begin{enumerate}
		\item par croissance comparée
		\item On pose $\forall n \in \N^*, u_n = \frac{n^\beta}{a^n}$. 
			\begin{align*}
				\forall  n \in \N^*, \frac{u_{n+1}}{u_n} &= \left( \frac{n+1}{n} \right) ^\beta \times \frac{1}{a} \\
				&= \frac{1}{a}\left( 1+\frac{1}{n} \right) ^\beta \\
				&\tendsto{n \to +\infty} \frac{1}{a} < 1
			\end{align*}
			Donc, $u_n \tendsto{n \to  +\infty} 0$
		\item On pose $\forall n \in \N, u_n = \frac{a^n}{n!}$ \[
			\forall n \in \N, \frac{u_{n+1}}{u_n} = \frac{a}{n+1} \tendsto{n \to +\infty} 0 < 1
			\] donc $u_n \tendsto{n \to +\infty} 0$
		\item On pose $\forall  n\in \N^*, u_n = \frac{n!}{n^n}$.
			\begin{align*}
				\forall n \in \N^*, \frac{u_{n+1}}{u_n}
				&= (n+1) {\frac{n^n}{(n+1)^{n+1}}} \\
				&= \left( \frac{n}{n+1} \right) ^n \\
				&= e^{n \ln\left( \frac{n}{n+1} \right) } \\
				&= e^{n \ln\left( 1+\frac{1}{n+1} \right)} \\
				&= e^{n(-\frac{1}{n} + o(\frac{1}{n})} \\
				&= e^{-1 + o(1)} \\
				&\tendsto{n \to  +\infty} e^{-1}<1
			\end{align*}
			donc $u_n \tendsto{n\to +\infty} 0$
	\end{enumerate}
\end{prv}

		\part{Matrices par blocs}

\begin{exm}
	Soit $p$ un projecteur de $E$ : \[
		E = \Ker p \oplus \mathrm{Im}\ p
	\] Soit $\mathcal{B} = (e_1, \ldots, e_k, e_{k+1}, \ldots, e_n)$ une base de $E$ avec $\begin{cases}
		\mathrm{Im}(p) = \Vect(e_1, \ldots, e_k)\\
		\Ker(p) = \Vect(e_{k+1}, \ldots, e_n)\\
	\end{cases}$

	Alors, 
	\begin{align*}
		\Mat_\mathcal{B}(p) =
		\left(\begin{NiceArray}{c c c | c c c}
				1&&&0&\Cdots&0\\
				 &\Ddots&&\Vdots&&\Vdots\\
				&&1&0&\Cdots&0\\\hline
				0&\Cdots&0&0&\Cdots&0\\
				\Vdots&&\Vdots&\Vdots&&\Vdots\\
				0&\Cdots&0&0&\Cdots&0\\
		\end{NiceArray}\right)
		= \left( \begin{array}{c|c}
				I_k & 0\\ \hline
				0&0
		\end{array}\right) \\
	\end{align*}

	De même, si $\s$ est une symétrie de $E$, \[
		E = \Ker(\s - \id_E) \oplus \Ker(\s + \id_E)
	.\] Soit $\mathcal{C} = (e_1', \ldots, e_\ell', e_{\ell+1}', \ldots, e'_n)$ avec $\begin{cases}
		\Vect(e'_1, \ldots, e'_\ell) = \Ker(\s - \id_E),\\
		\Vect(e'_{\ell+1}, \ldots, e'_n) = \Ker(\s + \id_E).\\
	\end{cases}$

	Alors
	\[
		\Mat_\mathcal{C}(\s) = \left(\begin{array}{c|c}
				I_\ell &0\\ \hline
				0&-I_{n-\ell}
		\end{array}\right) 
	\]
\end{exm}

\begin{prop}
	Soient $F$ et $G$ supplémentaires dans $E$ : \[
		E = F \oplus G.
	\] Soit $f \in \mathcal{L}(F)$ et $g \in \mathcal{L}(G)$. Alors \[
	\exists !h \in \mathcal{L}(E) h_{|F} = f,\ h_{|G} = g \et h = f \circ p + g \circ q
	\] où $\begin{cases}
		p \text{ est la projection sur $F$ parallèlement à $G$}\\
		q \text{ est la projection sur $G$ parallèlement à $F$}\\
	\end{cases}$.

	On a aussi $q = \id_E - p$.
\end{prop}

\begin{prv}
	\begin{itemize}
		\item[\sc \underline{Analyse}] Soit $h \in \mathcal{L}(E)$ tel que $\begin{cases}
				h_{|F}=f\\
				h_{|G}=g
			\end{cases}$.

			Soit $x \in E$. Alors \[
				x = \underbrace{p(x)}_{\in F} + \underbrace{q(x)}_{\in G}
			\]

			Donc,
			\begin{align*}
				h(x) &= h\big(p(x)\big) + h\big(q(x)\big)\\
				&= f\big(p(x)\big) + g\big(q(x)\big) \\
				&= (f \circ p + g \circ q)(x) \\
			\end{align*}
			Si $h$ existe, alors \[
				h = f \circ p + g \circ q
			\]
		\item[\underline{\sc Synthèse}] On pose $h = f \circ p + g  \circ q$.

			$p$, $q$, $f$ et $g$ sont linéaires donc $h$ aussi.

			Soit $x \in E$.
			\begin{align*}
				h(x) &= f\big(p(x)\big) + g\big(q(x)\big) \\
				&= f(x) + g(0_E) \\
				&= f(x) \\
			\end{align*}
			donc $h_{|F} = f$ et de même $h_{|G}=g$.
	\end{itemize}
\end{prv}

\begin{prop}
	On reprend les notations et hypothèses précédentes. Soit $(e_1, \ldots, e_p)$ une base de $F$, et $(f_1, \ldots, f_q)$ une base de $G$. Alors, $\mathcal{B} = (e_1, \ldots, e_p, f_1, \ldots, f_q)$ est une base de $E$ et \[
		\Mat_\mathcal{B}(h) = \left(
		\begin{array}{c|c}
			A&0\\ \hline
			0&B
		\end{array}\right)
	\] où $\begin{cases}
		A = \Mat_{(e_1, \ldots e_p)}(f)\\
		B = \Mat_{(f_1, \ldots, f_q)}(g)
	\end{cases}$
	\qed
\end{prop}

\begin{prop}
	Soient $(A,A') \in \mathcal{M}_n(\mathbbm{K})^2$ et $(B,B') \in \mathcal{M}_p(\mathbbm{K})^2$.
	\begin{enumerate}
		\item \[
				\left(\begin{array}{c|c}
					A&0\\ \hline
					0&B
				\end{array}\right)
				\left(\begin{array}{c|c}
					A'&0\\ \hline
					0&B'
				\end{array}\right) = 
				\left(\begin{array}{c|c}
					AA'&0\\ \hline
					0&BB'
				\end{array}\right)
			\]
		\item \[
				\left(\begin{array}{c|c}
					A&0\\ \hline
					0&B
				\end{array}\right) \in \mathrm{GL}_{n+p}(\mathbbm{K})	 \iff \begin{cases}
					 A \in \mathrm{GL}_n(\mathbbm{K})\\
					 B \in \mathrm{GL}_p(\mathbbm{K})
				\end{cases}
			\] et dans ce cas, \[
				\left(\begin{array}{c|c}
					A&0\\ \hline
					0&B
				\end{array}\right)^{-1} =
				\left(\begin{array}{c|c}
					A^{-1}&0\\ \hline
					0&B^{-1}
				\end{array}\right)
			\]
		\item \[
				\tr \left(\begin{array}{c|c}
					A&0\\ \hline
					0&B
				\end{array}\right) = \tr A + \tr B
			\]
	\end{enumerate}
\end{prop}

\begin{prv}
	\begin{enumerate}
		\item Soit $\begin{cases}
				f \in \mathcal{L}(F) \text{ tel que } \Mat_\mathcal{B}(f) = A,
				f' \in \mathcal{L}(F) \text{ tel que } \Mat_\mathcal{B}(f') = A',
				g \in \mathcal{L}(G) \text{ tel que } \Mat_\mathcal{C}(g) = B,
				g' \in \mathcal{L}(G) \text{ tel que } \Mat_\mathcal{C}(g') = B'
			\end{cases}$ où $\begin{cases}
				F \oplus G = \mathbbm{K}^{n+p},\\
				\dim(F) = n, \dim(G) = p,\\
				\mathcal{B} \text{ base de } F,\\
				\mathcal{C} \text{ base de } G.\\
			\end{cases}$
			Soit $\begin{cases}
				h \in \mathcal{L}(\mathbbm{K}^{n+p}) \text{ tel que } \begin{cases}
					h_{|F} = f\\
					h_{|G} = g
				\end{cases}\\
				h' \in \mathcal{L}(\mathbbm{K}^{n+p}) \text{ tel que } \begin{cases}
					h'_{|F} = f'\\
					h'_{|G} = g'\\
				\end{cases}
			\end{cases}$
			Soit $\mathcal{D} = \mathcal{B} \cup \mathcal{C}$ une base de $\mathbbm{K}^{n+p}$.
			\begin{align*}
				\left(\begin{array}{c|c}
					A&0\\ \hline
					0&B
				\end{array}\right)
				\left(\begin{array}{c|c}
					A'&0\\ \hline
					0&B'
				\end{array}\right) &= \Mat_{\mathcal{D}}(h) \Mat_{\mathcal{D}}(h')\\
				&= \Mat_{\mathcal{D}}(h \circ h') \\
			\end{align*}
			Or, $(h \circ h')_{|F} = f \circ f'$ et $(h \circ h')_{|G} = g \circ g'$.

			Donc,
			\begin{align*}
				\Mat_\mathcal{D}(h \circ h') &=
					\left(\begin{array}{c|c}
						\Mat_\mathcal{B}(f \circ f')&0\\ \hline
						0&\Mat_\mathcal{C}(g \circ g')
					\end{array}\right)\\
				&=\left(\begin{array}{c|c}
					AA'&0\\ \hline
					0&BB'
				\end{array}\right).
			\end{align*}
	\end{enumerate}
\end{prv}

\begin{prop}
	Soient $A,A' \in \mathcal{M}_n(\mathbbm{K})$, $B,B' \in \mathcal{M}_{n,p}(\mathbbm{K})$, $C,C' \in \mathcal{M}_{p,n}(\mathbbm{K})$ et $D, D' \in \mathcal{M}_p(\mathbbm{K})$.

	\[
		\left(\begin{array}{c|c}
			A&B\\ \hline
			C&D
		\end{array}\right)
		\left(\begin{array}{c|c}
			A'&B'\\ \hline
			C'&D'
		\end{array}\right) = 
		\left(\begin{array}{c|c}
			AA' + BC'& AB' + BD'\\ \hline
			CA' + DC'&CB' + DD'
		\end{array}\right)
	\] Cette formule se généralise à un nombre quelconque de blocs : \[
		\left(\begin{array}{c|c|c|c}
				A_{11}&A_{12}&\cdots&A_{1,n}\\ \hline
				A_{21}&A_{22}&\cdots&A_{2,n}\\ \hline
				\vdots&\vdots&\ddots&\vdots\\ \hline
				A_{p,1}&A_{p,2}&\cdots&A_{p,n}
		\end{array}\right)
		\left(\begin{array}{c|c|c|c}
				A'_{11}&A'_{12}&\cdots&A'_{1,n}\\ \hline
				A'_{21}&A'_{22}&\cdots&A'_{2,n}\\ \hline
				\vdots&\vdots&\ddots&\vdots\\ \hline
				A'_{p,1}&A'_{p,2}&\cdots&A'_{p,n}
		\end{array}\right)
	\] Cette matrice se calcyle comme on s'y attend si les dimensions des blocs autorisent les produits.
\end{prop}

\begin{prop}
	Le rang d'une matrice $A$, c'est la taille de la plus grande matrice carrée inversible que l'on peut extraire de $A$.
	\qed
\end{prop}




		\part{Trigonométrie hyperbolique}

\begin{defn}
	Pour tout $x \in \R$, on pose \[
		\begin{cases}
			\ch x = \frac{e^x + e^{-x}}{2},\\
			\sh x = \frac{e^x - e^{-x}}{2},\\
			\th x = \frac{\sh x}{\ch x}.
		\end{cases}
	\]

	$\ch$ est appelé \underline{cosinus hyperbolique}, $\sh$ est appelé \underline{sinus hyperbolique} et $\th$ est appelé \underline{tangeante hyperbolique}.
	\index{cosinus hyperbolique}
	\index{sinus hyperbolique}
	\index{tangente hyperbolique}
\end{defn}

\begin{rmk}
	Ces formules rappèlent les formules d'Euler : pour tout $x \in \R$,
	\begin{align*}
		\cos x = \frac{e^{ix} + e^{-ix}}{2}\quad\longleftrightarrow\quad\ch x = \frac{e^x + e^{-x}}{2}\\
		\sin x = \frac{e^{ix} - e^{-ix}}{2i}\quad\longleftrightarrow\quad\sh x = \frac{e^x - e^{-x}}{2}\\
	\end{align*}
\end{rmk}

\begin{figure}[H]
	\centering
	\begin{asy}
		import graph;

		size(12cm);

		pair A = (-2, 0);
		pair B = (2, 0);

		real eps = 0.05;

		draw(shift(A) * ((0, -1.3) -- (0, 1.3)), Arrow(TeXHead));
		draw(shift(A) * ((-1.3, 0) -- (1.3, 0)), Arrow(TeXHead));

		draw(circle(A, 1), magenta);
		
		real theta = 38;
		pair M = dir(theta) + A;
		draw(A -- M, red);
		draw(arc(A, 0.35, 0, theta), red, Arrow(TeXHead));
		draw(M -- (A.x-eps, M.y), dashed);
		draw(M -- (M.x, A.y-eps), dashed);
		label("\small$\theta$", 0.5dir(theta/2) + A, red);
		label("\small$\cos\theta$", (M.x, A.y), align=S);
		label("\small$\sin\theta$", (A.x, M.y), align=1.2W);
		dot("\small$M$", M);

		label("\small$x^2 + y^2 = 1$", A + 1.5dir(45+180));

		draw(shift(B) * ((0, -1.3) -- (0, 1.3)), Arrow(TeXHead));
		draw(shift(B) * ((-1.3, 0) -- (1.3, 0)), Arrow(TeXHead));

		real ch(real x) { return (exp(x) + exp(-x)) / 2.; }
		real sh(real x) { return (exp(x) - exp(-x)) / 2.; }
		real nch(real x) { return -ch(x); }

		real k = 1.9; real r = 1.2;
		real t = 1.4;

		draw(shift(B) * scale(0.35) * graph(ch, sh, -k, k), magenta);
		draw(shift(B) * scale(0.35) * graph(nch, sh, -k, k), magenta);

		label("\small$x^2 - y^2 = 1$", B + 1.5dir(45+180) + (0, -0.2));

		M = B + 0.35(ch(t), sh(t));

		draw(M -- (B.x-eps, M.y), dashed);
		draw(M -- (M.x, B.y-eps), dashed);
		dot("\small$M$", M);
		label("\small$\ch x$", (M.x, B.y), align=S);
		label("\small$\sh x$", (B.x, M.y), align=1.2W);

		draw(shift(B) * ((-r, -r)--(r,r)), gray + dashed);
		draw(shift(B) * ((r, -r)--(-r,r)), gray + dashed);
	\end{asy}
\end{figure}


		\part{Applications}
\section{Formule de Stirling}

\begin{prop}
	On a :
	\[
		n! \simi_{n\to +\infty} \sqrt{2\pi n} \left( \frac{n}{e} \right)^n?.
	\]
\end{prop}

\begin{prv}
	\[
		\forall n \in \N^*, \ln(n!) = \sum_{k=1}^n \ln k.
	\]

	$x \mapsto \ln x$ est strictement croissante sur $[1, +\infty[$ donc \[
		\forall k \in \N^*, \forall x \in [k, k+1], \ln x \ge \ln k
	\] donc \[
		\forall k \in \N^*, \int_{k}^{k+1} \ln x~\mathrm{d}x \ge \int_{k}^{k+1} \ln k~\mathrm{d}x = \ln k
	\] et \[
		\forall k \ge 2, \forall x \in [k - 1, k], \ln x \le \ln k
	\] et docn \[
		\forall k \ge 2, \int_{k-1}^{k}  \ln x~\mathrm{d}x \le \int_{k-1}^{k} \ln k~\mathrm{d}x = \ln k
	\] Ainsi \[
		\forall n \ge 2, 
		\int_{1}^{n} \ln x~\mathrm{d}x \ge \sum_{k=2}^n \le \int_{2}^{n+1} \ln x~\mathrm{d}x
	\] Or
	\begin{align*}
		\forall n \ge 2, \int_{1}^{n} \ln x~\mathrm{d}x &= \left[ x \ln x \right]_0^n\\
		&= n \ln(n) - n + 1 \\
		&\simi_{n\to +\infty} n \ln n\\
		\int_{2}^{n+1} \ln x~\mathrm{d}x &= (n+1) \ln(n+1) - (n+1) - 2 \ln(2) + 2 \\
		&\simi_{n\to +\infty} (n+1) \ln(n+1)\\
		&\simi_{n\to +\infty}n \ln n
	\end{align*}
	car
	\begin{align*}
		\ln(n+1) &= \ln\left( n \left( 1+ \frac{1}{n} \right) \right) \\
		&= \ln n + \ln\left( 1+\frac{1}{n} \right) \\
		&= \ln n + \frac{1}{n} + \po\left( \frac{1}{n} \right) \\
		&\sim \ln n \\
	\end{align*}

	Donc \[
		\ln(n!)) \simi_{n\to +\infty} n \ln n
	\]
	Cependant, on a un problème : {\color{orange}
	\begin{align*}
		&\ln(n!) = n \ln n + \po(n \ln n)\\
		\text{donc } & n! = n^n \underbrace{e^{\po(n \ln n)}}_{?}
	\end{align*}}

	On pose \[
		\forall n \in \N^*, u_n = \ln(n!) - n\ln n
	\] $(u_n)$ a même nature que $\Sigma(u_{n+1} - u_n)$ et
	\begin{align*}
		\forall n \in \N^*,
		u_{n+1} - u_n &= \ln\left( \frac{(n+1)!}{n!} \right) - (n+1) \ln(n+1) + n \ln n \\
		&= n\big(\ln n - \ln(n+1)\big) \\
		&= n\ln\left( \frac{n}{n+1} \right) \\
		&= n \ln \left( 1 - \frac{1}{n+1} \right) \\
		&\sim -\frac{n}{n+1} \sim -1 < 0
	\end{align*}

	$\Sigma(-1)$ diverge donc $(u_n)$ diverge.

	{\color{red}
		\underline{Conjecture}
		\[
			u_n = \sum_{k=1}^{n-1}(u_{k+1} - u_k) \underbrace{\sim}_{\mathclap{\substack{~\\\downarrow\\\text{On n'a absolument pas le droit !}}}} \sum_{k=1}^{n-1} (-1) = -(n-1) \sim -n
		\]
	}

	On pose \[
		\forall n \in \N^*, v_n = u_n + n
	\] et donc 
	\begin{align*}
		\forall n \in \N^*, v_{n+1} - v_n &= n \ln\left( 1 - \frac{1}{n+1} \right) + 1 \\
		&= n\left( -\frac{1}{n+1} - \frac{1}{2(n+1)^2} + \po\left( \left( \frac{1}{n+1} \right)^2 \right) \right) + 1 \\
		&= n \left( -\frac{1}{n\left( 1+\frac{1}{n} \right)} - \frac{1}{2n^2\left( 1+\frac{1}{n^2} \right)} + \po\left( \frac{1}{n^2} \right) \right) + 1 \\
		&= -\left( \frac{1}{1+\frac{1}{n}} - \frac{1}{2n} \times \frac{1}{\left( 1+\frac{1}{n} \right)^2} + \po\left( \frac{1}{n} \right) \right) \\
		&= -\left( 1 - \frac{1}{n} + \frac{1}{2n} + \po\left( \frac{1}{n} \right) \right) + 1 \\
		&= \frac{1}{2n} + \po\left( \frac{1}{n} \right) \\
		&\sim \frac{1}{2n} > 0.
	\end{align*}

	{\color{red}
		\[
			v_n \sim \sum_{k=1}^{n-1}(v_{k-1} - v_k) \sim \sum_{k=1}^{n-1} \frac{1}{2k} \sim \frac{1}{2} \ln(n)
		\]
	}

	On pose \[
		\forall n \in \N^*, w_n = v_n - \frac{1}{2} \ln n
	\] et donc
	\begin{align*}
		\forall n \in \N^*,
		w_{n+1}- w_n &= n\ln\left( 1+\frac{1}{n+1} \right) - \frac{1}{2}\ln(n+1) + \frac{1}{2} \ln(n) + 1 \\
		&= n\left( -\frac{1}{n+1} - \frac{1}{2(n+1)^2} - \frac{1}{3(n+1)^3} + \po\left( \frac{1}{(n+1)^3} \right) \right)\\
		&\phantom{=}\,+ 1 + \frac{1}{2} \ln\left( 1 - \frac{1}{n+1} \right) \\
		&= -1 - \frac{1}{2(n+1)} - \frac{1}{3(n+1)^2} + \po\left( \frac{1}{(n+1)^2} \right) \\
		&\phantom{=}\,+ \frac{1}{n+1} + \frac{1}{2(n+1)^2} + 1\\
		&\phantom{=}\,+ \frac{1}{2} \left( -\frac{1}{n+1} - \frac{1}{2(n+1)^2} + \po\left( \frac{1}{(n+1)^2} \right) \right)
		&\sim -\frac{1}{12(n+1)^2}\\
		&\sim -\frac{1}{12n^2} < 0
	\end{align*}
	donc $\Sigma(w_{n+1} - w_n)$ converge et donc $(w_n)$ converge.

	On pose $\ell = \lim_{n\to +\infty} w_n$. Ainsi, \[
		\forall n \in \N^*, w_n = \ell + \po(1)
	\] et donc \[
		\forall n \in \N^*, \ln(n!) = n \ln n - n + \frac{1}{2} \ln(n) + \ell + \po(1)
	\] et alors
	\begin{align*}
		\forall n \in \N^*, n! &= n^n e^{-n} \sqrt{n} e^{\ell} \underbrace{e^{\po(1)}}_{\mathclap{\tendsto{n\to +\infty} 1}} \\
		&\sim \left( \frac{n}{e} \right)^n \sqrt{n} \times K
	\end{align*} avec $K = e^{\ell}$.

	On pose \[
		\forall n \in \N^*, I_n = \int_{0}^{\frac{\pi}{2}} \sin^n x~\mathrm{d}x \sim \sqrt{\frac{\pi}{2n}}
	\]et \hfill (c.f. TD5 / Exercice 8)\[
		I_{2n} = \frac{(2n)!}{\left( 2^n n! \right)^2} \times \frac{\pi}{2}.
	\]

	\begin{align*}
		I_{2n} &\sim \frac{\pi}{2} \cancel{\left( \frac{2n}{2e} \right)^{2n}} \sqrt{2n} K \cancel{\left( \frac{e}{n} \right)^{2n}} \frac{1}{n} \times \frac{1}{K^2}\\
		&\sim \frac{\pi}{K\sqrt{2n}}.
	\end{align*}
	Or \[
		I_{2n} \sim \sqrt{\frac{\pi}{4n}}.
	\] Donc \[
		\frac{\sqrt{\frac{\pi}{4n}}}{\frac{\pi}{K\sqrt{2n}}} \tendsto{n\to +\infty} 1
	\] donc \[
		\frac{K}{\sqrt{2\pi}} \tendsto{n\to +\infty} 1
	\] et donc $K = \sqrt{2\pi}$.
\end{prv}

\section{Développement décimal}

\begin{exm}
	\begin{itemize}
		\item Avec $x = 0,54\mathunderline{54}\ldots$, que vaut $2x$ ?
		\item Avec $x = 0,333\mathunderline{3}\ldots$, que vaut $3x$ ?
			\begin{itemize}
				\item $0.999\mathunderline{9}\ldots$ ?
				\item $3 \times \frac{1}{3} = 1$ ?
			\end{itemize}
	\end{itemize}
\end{exm}

\begin{prop}
	Soit $(a_n)_{n \in \N}$ telle que \[
		\begin{cases}
			a_0 \in \Z,\\
			\forall n \ge 1, a_n \in \left\llbracket 0,9 \right\rrbracket
		\end{cases}
	\]

	La série $\sum \frac{a_n}{10^n}$ converge.
\end{prop}

\begin{prv}
	\[
		\forall n \ge 1, 0 \le \frac{a_n}{10^n} \le \frac{9}{10^n}
	\] $\sum \frac{1}{10^n}$ converge car $\frac{1}{10} \in [0, 1[$.
	Donc $\sum_{n\ge 1} \frac{a_n}{10^n}$ converge donc $\sum_{n\ge 1} \frac{a_n}{10^n}$ converge.
\end{prv}

\begin{defn}
	Soit $x \in \R$. On dit que $x$ admet un \underline{développement décimal} si \[
		\exists a_0 \in \Z, (a_n)_{n\ge 1} \in \left\llbracket 0,9 \right\rrbracket^N,
		x = \sum_{n=0}^{+\infty} \frac{a_n}{10^n}.
	\]
	\index{développement décimal}
\end{defn}

\begin{thm}
	Tou réel $x \in [0, 1[$ admet un développement décimal : \[
		x = \sum_{n=1}^{+\infty} \frac{\left\lfloor 10^n x \right\rfloor - 10 \left\lfloor 10^{n-1} x \right\rfloor}{10^n}
	\]
\end{thm}

\begin{prv}
	\begin{align*}
		\forall n \ge 1,\kern 5mm &\phantom{-}10^n x - 1 < \left\lfloor 10^n x \right\rfloor \le  10^n x\\
		&-10^n x + 10 > -10 \left\lfloor 10^{n-1} x \right\rfloor \ge -10^n x
	\end{align*}
	donc \[
		-1 < \left\lfloor 10^n x \right\rfloor - 10 \left\lfloor 10^{n-1} x \right\rfloor < 10
	\] et donc \[
		\left\lfloor 10^n x \right\rfloor - 10 \left\lfloor 10^{n-1} x \right\rfloor \in \left\llbracket 0,9 \right\rrbracket.
	\]

	De plus,
	\begin{align*}
		\sum_{k=1}^n \frac{\left\lfloor 10^k x \right\rfloor - 10 \left\lfloor 10^{k-1}x \right\rfloor }{10^k} &= \sum_{k=1}^n \left( \frac{\left\lfloor 10^k x \right\rfloor}{10^k} - \frac{\left\lfloor 10^{k-1}x \right\rfloor}{10^{k-1}} \right) \\
		&= \frac{\left\lfloor 10^n x \right\rfloor}{10^n} - \underbrace{\left\lfloor x \right\rfloor}_{=0}\\
		&\tendsto{n\to +\infty} x. \\
	\end{align*}
\end{prv}

\begin{thm}
	Soit $x \in ]0, 1[$.

	\begin{enumerate}
		\item Si $x$ n'est pas décimal (i.e. on ne peut pas l'écrire comme $\sfrac{p}{10^n}$ avec $p \in \Z$ et $n \in \N$), alors $x$ a un unique développement décimal.
		\item Si $x$ est décimal, alors $x$ a exactement 2 développements décimaux :
			\begin{itemize}
				\item il y en a un où, à partir d'un certain rang, tous les chiffres sont nuls,
				\item et un autre où tous les chiffres sont égaux à 9 à parir d'un certain rang.
			\end{itemize}
	\end{enumerate}
\end{thm}

\begin{prv}
	Soit $(a_n)_{n\ge 1} \in \left\llbracket 0,9 \right\rrbracket^{\N^*}$ et $(b_n)_{n\ge 1} \in \left\llbracket 0,9 \right\rrbracket^{\N^*}$ telles que \[
		x = \sum_{n=1}^{+\infty} \frac{a_n}{10^n} = \sum_{n=1}^{+\infty} \frac{b_n}{10^n}
	\] On pose $n_0 = \min \{n \in \N^*  \mid a_n \neq b_n\}$ : \[
		\begin{cases}
			\forall n < n_0, a_n = b_n,\\
			a_{n_0} \neq b_{n_0}.
		\end{cases}
	\] Sans perte de généralité, on suppose $a_{n_0} < b_{n_0}$. On a donc
	\begin{align*}
		0 < \frac{b_{n_0} - a_{n_0}}{10^{n_0}} &= \sum_{n = n_0 + 1}^{+\infty} \frac{a_n - b_n}{10^n} \\
	\end{align*}
	\[
		\forall n \ge n_0, \begin{cases}
			0 \le a_n \le 9\\
			0 \le b_n \le 9
		\end{cases}
	\] donc \[
		\forall n \ge n_0, -9 \le a_n - b_n \le 9
	\] donc \[
		-9 \sum_{n=n_0+1}^{+\infty} \frac{1}{10^n} \le \sum_{n=n_0 + 1}^{+\infty} \frac{a_n - b_n}{10^n} \le 9 \sum_{n=1}^{+\infty} \frac{1}{10^n}.
	\]
	Or,
	\begin{align*}
		\sum_{n=n_0 + 1}^{+\infty} \frac{1}{10^n} &= \frac{1}{10^{n_0+1}} \sum_{n=0}^{+\infty} \frac{1}{10^n} \\
		&= \frac{1}{10^{n_0+1}} \times \frac{1}{1-\frac{1}{10}} \\
		&= \frac{1}{9 \times 10^{n_0}} \\
	\end{align*}
	D'où, \[
		0 < \frac{b_{n_0} - a_{n_0}}{10^{n_0}} \le  \frac{1}{10^{n_0}}
	\] donc \[
		0 < \underbrace{b_{n_0} - a_{n_0}}_{\in \Z} \le 1
	\] donc $b_{n_0} - a_{n_0} = 1$ et donc \[
	\sum_{n = n_0 + 1}^{+\infty} \frac{a_n - b_n}{10^n} = \frac{1}{10^{n_0}}
	\] donc \[
		\forall n > n_0, a_n - b_n = 9
	\] et donc \[
		\forall n > n_0, \begin{cases}
			a_n = 9\\
			b_n = 0
		\end{cases}
	\] Comme \[
		\forall n > n_0, b_n = 0
	\] $x$ est décimal et les deux développements de $x$ sont alors
	\begin{align*}
		x &= 0,a_1\ldots a_{n_0-1}a_{n_0}\mathunderline{9}\ldots\\
		&= 0,a_1\ldots a_{n_0-1}(a_{n_0}+1)\mathunderline{0}\ldots \\
	\end{align*}
\end{prv}

\begin{rmk}
	Avec $x = 0,\!54\mathunderline{54}\ldots$, $100x = 54,\!54\mathunderline{54}\ldots = 54 + x$. On a donc $x = \frac{54}{99}$.

	Avec $x = 0,\!987\,123\,\mathunderline{123}\ldots$, on a
	\begin{align*}
		x &= \frac{987}{1000} + 0,\!000\,\mathunderline{123}\ldots\\
		&= \frac{987}{1000} + \frac{1}{10^3}\underbrace{(0,\!\mathunderline{123}\ldots)}_y \\
	\end{align*}
	On a $1000 y = 123 + y$ et donc $y = \frac{123}{999}$ et donc $x = \frac{987 + \frac{123}{999}}{1000}$.
\end{rmk}





	}

	{
		\chap[31]{Variables aléatoires}
		\renewcommand{\cwd}{../chap31}
		\begin{defn}
	Soit $E$ un $\mathbbm{K}$-espace vectoriel. On dit que $E$ est de \underline{dimension finie} si $E$ a au moins une famille génératrice finie. On dit que $E$ est de \underline{dimension infinie} sinon.
	\index{dimension finie (espace vectoriel)}
	\index{dimension infinie (espace vectoriel)}
\end{defn}

\begin{thm}
	[Théorème de la base extraite]
	Soit $E$ un $\mathbbm{K}$-espace vectoriel non nul de dimension finie. Soit $\mathcal{G}$ une famille génératrice finie de $E$. Alors, il existe une base $\mathcal{B}$ de $\mathcal{E}$ telle que $\mathcal{B} \subset \mathcal{G}$.
\end{thm}

\begin{prv}
	[par récurrence sur $\#G = \Card(G)$]
	\begin{itemize}
		\item Soit $E$ un $\mathbbm{K}$-espace vectoriel non nul engendré par $\mathcal{G} = (u)$.\\
			Si $u = 0_E$, alors $E = \{0_E\}$: une contradiction $\lightning$ \\
			Donc $u \neq 0_E$ donc $(u)$ est libre. En effet, \[
				\forall \lambda \in \mathbbm{K}, \lambda u = 0_E \implies \lambda = 0_\mathbbm{K}
			\] Donc $\mathcal{G}$ est une base de $E$.\\
		\item Soit $n \in \N_*$. Soit $E$ un $\mathbbm{K}$-espace vectoriel. On suppose que si $E$ a une famille génératrice constituée de $n$ vecteurs, alors on peut extraire de cette famille une base de $E$.\\
			Soit $\mathcal{G}$ une famille génératrice de $E$ avec $n+1$ vecteurs.\\
			Si $\mathcal{G}$ est libre, alors $\mathcal{G}$ est une base de $E$. \\
			Si $\mathcal{G}$ n'est pas libre, alors il existe $u \in \mathcal{G}$ tel que $u \in \Vect(\mathcal{G}\setminus \{u\})$ \\
			Donc $\mathcal{G}\setminus \{u\}$ engendre $E$. Or, $\mathcal{G}\setminus \{u\}$ possède $n$ vecteurs. D'après l'hypothèse de récurrence, il existe une base $\mathcal{B}$ de $E$ telle que \[
				\mathcal{B} \subset \mathcal{G} \setminus \{u\} \subset \mathcal{G}
			\] 
	\end{itemize}
\end{prv}

\begin{crlr}
	Tout espace de dimension finie a une base.
	\qed
\end{crlr}

\begin{thm}
	[Théorème de la base incomplète]
	Soit $E$ un $\mathbbm{K}$-espace vectoriel de dimension finie, $\mathcal{G}$ une famille génératrice finie de $E$. $\mathcal{L}$ une famille libre de $E$. Alors, il existe une base $\mathcal{B}$ de $E$ telle que \[
		\mathcal{L} \subset \mathcal{B} \text{ et } \mathcal{B}\setminus \mathcal{L} \subset \mathcal{G}
	\] 
\end{thm}

\begin{prv}
	[par récurrence sur $\#(\mathcal{G}\setminus\mathcal{L})$]
	\begin{itemize}
		\item Avec les notations précédentes, on suppose que $\mathcal{G}\setminus\mathcal{L} \neq \O$ \[
				\forall u \in \mathcal{G}, u \in \mathcal{L}
			\] Donc $\mathcal{G} \subset \mathcal{L}$ donc $\mathcal{L}$ est génératrice donc $\mathcal{L}$ est une base de $E$. On pose $\mathcal{B} = \mathcal{L}$ et alors \[
				\mathcal{L} \subset  \mathcal{B} \text{ et } \mathcal{B}\setminus\mathcal{L} = \O \subset  \mathcal{G}
			\] 
		\item Soit $n \in \N$. On suppose que si $\mathcal{G}$ est génératrice et $\mathcal{L}$ libre avec $\#(\mathcal{G}\setminus\mathcal{L}) = n$ alors il existe une base $\mathcal{B}$ de $E$ telle que \[
			\mathcal{L}\subset \mathcal{B} \text{ et } \mathcal{B}\setminus\mathcal{L}\subset \mathcal{G}
		\] Soient à présent $\mathcal{G}$ une famille génératrice de $E$ et $\mathcal{L}$ une famille libre de $E$ telles que $\#(\mathcal{G}\setminus\mathcal{L}) = n+1 > 0$\\
		Si $\mathcal{L}$ engendre $E$, alors $\mathcal{L}$ est une base de $E$. On pose $\mathcal{B} = \mathcal{L}$ et on a bien \[
			\mathcal{L} \subset  \mathcal{B} \text{ et } \mathcal{B} \setminus \mathcal{L} = \O \subset  \mathcal{G}
		\] On suppose que $\mathcal{L}$ n'engendre pas $E$. Il existe $u \in \mathcal{G}$ tel que $u \not\in \Vec(\mathcal{L})$ (car sinon, $\mathcal{G} \subset \Vect(\mathcal{L})$ et donc $\underbrace{\Vect(\mathcal{G})}_{= E} \subset  \underbrace{\Vect(\mathcal{L})}_{ \subset E}$\\
		Donc $\mathcal{L} \cup \{u\} $ est libre. On pose $\mathcal{L}' = \mathcal{L} \cup \{u\} $ \[
			\mathcal{G}\setminus \mathcal{L}' = \mathcal{G}\setminus (\mathcal{L} \cup \{u\}) = (\mathcal{G}\setminus\mathcal{L})\setminus \{u\} 
		\] donc $\#(\mathcal{G}\setminus\mathcal{L}') = n+1 -1 = n$\\
		D'après l'hypothèse de récurrence, il existe $\mathcal{B}$ une base de $E$ telle que \[
			\mathcal{L} \subset  \mathcal{L}' \subset \mathcal{B} \text{ et } \mathcal{B}\setminus \mathcal{L}' \subset \mathcal{G}
		\] \[
			\mathcal{B} \setminus \mathcal{L} = \underbrace{\mathcal{B}\setminus\mathcal{L}'}_{\subset \mathcal{G}} \cup \underbrace{\{u\}}_{\subset \mathcal{G} \text{ car } u \in \mathcal{G}}
		\] On a $\mathcal{B}\setminus\mathcal{L}\subset \mathcal{G}$
	\end{itemize}
\end{prv}

\begin{thm}
	Soit $E$ un $\mathbbm{K}$-espace vectoriel de dimension finie. Toutes les bases de $E$ ont le même cardinal.
\end{thm}

\begin{prv}
	Soit $\mathcal{G}$ une famille génératrice finie de $E$ et $\mathcal{B} \subset  \mathcal{G}$ une base de $E$. On note $n = \#\mathcal{B}$ \\
	Soit $\mathcal{B}'$ une base de $E$. On pose $p = n - \#(\mathcal{B} \cap  \mathcal{B}')$. Montrons par récurrence sur  $p$ que $\#\mathcal{B} = \#\mathcal{B}'$ 
	\begin{itemize}
		\item On suppose que $p = 0$. Alors, $\#(\mathcal{B} \cap \mathcal{B}') = n$ \\
			Or, $\mathcal{B}' \cap \mathcal{B} \subset \mathcal{B}$ donc $\mathcal{B} \cap \mathcal{B}' = \mathcal{B}$ donc $\mathcal{B} \subset  \mathcal{B}'$ et donc $\mathcal{B} = \mathcal{B}'$ 
		\item Soit $p \in \N$. On suppose que si $\mathcal{B}'$ est une base de $E$ telle que $n - \#(\mathcal{B} \cap \mathcal{B}') = p$, alors $\#\mathcal{B}' = n$ \\
			Aoit $\mathcal{B}'$ une base de $E$ telle que $n - \#(\mathcal{B}\cap \mathcal{B}') = p+1 > 0$ \\
			Donc $\mathcal{B} \cap \mathcal{B}' \neq \mathcal{B}$. Soit $u \in \mathcal{B}' \setminus \mathcal{B}$. D'après le lemme d'échange, il existe $v \in \mathcal{B}\setminus \mathcal{B}'$ tel que $\mathcal{B}' \setminus \{u\} \cup \{v\}$ est une base de $E$. On pose $\mathcal{B}'' = \mathcal{B}' \setminus \{u\} \cup \{v\}$ 
			\begin{align*}
				\mathcal{B}'' \cap \mathcal{B} &= \left( (\mathcal{B}' \setminus \{u\})  \cap \mathcal{B} \right) \cup \{v\} \\
				&= (\mathcal{B}' \cap \mathcal{B}) \cup \{v\} \\
			\end{align*}
			donc,
			\begin{align*}
				n - \#(\mathcal{B}'' \cap \mathcal{B}) &= n - (\#(\mathcal{B}' \cap \mathcal{B}) + 1) \\
				&= p+1- 1 \\
				&= p \\
			\end{align*}
			D'après l'hypothèse de récurrence, \[
				\#\mathcal{B}'' = n
			\] Or, $\#\mathcal{B}'' = \#\mathcal{B}'$
	\end{itemize}
\end{prv}

\begin{lem}
	Soient $\mathcal{B}$ et $\mathcal{B}'$ deux bases de $E$ telles que $\mathcal{B}\subset \mathcal{B}'$. Alors, $\mathcal{B} = \mathcal{B}'$.
\end{lem}

\begin{prv}
	On suppose $\mathcal{B}' \neq \mathcal{B}$. Soit $u \in \mathcal{B}' \setminus \mathcal{B}$
	$u \in E = \Vect(\mathcal{B})$ donc $\mathcal{B} \cup \{u\}$ n'est pas libre.
	Donc $\mathcal{B}\cup \{u\} \subset \mathcal{B}'$ et $\mathcal{B}'$ est libre donc $\mathcal{B}\cup \{u\}$ est libre: une contradiction $\lightning$
\end{prv}

\begin{lem}
	[Lemme d'échange] Soient $\mathcal{B}_1$ et $\mathcal{B}_2$ deux bases de $E$ et $u \in \mathcal{B}_1 \setminus \mathcal{B}_2$. Alors, il existe $v \in \mathcal{B}_2$ tel que $(\mathcal{B}_1 \setminus \{u\}) \cup \{v\}$ soit une base de $E$.
\end{lem}

\begin{prv}
	[1${}^\text{nde}$ méthode]
	On suppose que pout tout $v \in \mathcal{B}_2$, $(\mathcal{B}_1\setminus \{u\}) \cup \{v\}$ n'est pas une base de $E$
	Soit $v \in \mathcal{B}_2$.
	\begin{itemize}
		\item Supposons $(\mathcal{B}_1\setminus \{u\})\cup \{v\}$ non libre. $\mathcal{B}_1 \setminus \{u\}$ est libre. Donc $v \in \Vect(\mathcal{B}_1 \setminus \{u\})$
		\item Supposons $(\mathcal{B}_1\setminus \{u\}) \cup \{v\}$ non génératrice.
			Comme $\mathcal{B}_1$ engendre $E$, $u \not\in \Vect(\mathcal{B}_1\setminus \{v\})$.
			On suppose que $\mathcal{B}_1 \neq \mathcal{B}_2$.
			$\forall v \in \mathcal{B}_2 \setminus \mathcal{B}_1, \Vect(\mathcal{B}_1 \setminus \{v\}) = \Vect(\mathcal{B}_1) = E \ni u$ 
			donc, $(\mathcal{B}_1\setminus \{u\}) \cup \{v\}$ engendre $E$ et donc \[
				v \in \Vect(\mathcal{B}_1 \setminus \{u\})
			\] On a aussi \[
				\forall v \in \mathcal{B}_1 \setminus \{u\}, v \in \Vect(\mathcal{B}_1\setminus \{u\})
			\] Comme $u \not\in \mathcal{B}_2$, on a \[
				\forall v \in \mathcal{B}_2, v \in \Vect(\mathcal{B}_1\setminus \{u\})
			\] docn \[
				E = \Vect(\mathcal{B}_2) \subset \Vect(\mathcal{B}_1\setminus \{u\})
			\] donc $\mathcal{B}_1\setminus \{u\}$ engendre $E$ donc $\mathcal{B}_1\setminus \{u\}$ est une base de $E$. Or, $\mathcal{B}_1 \setminus \{u\}  \subset  \mathcal{B}_1$, donc $\mathcal{B}_1\setminus \{u\} = \mathcal{B}_1$
	\end{itemize}
\end{prv}

\begin{prv}
	[2${}^\text{nde}$ méthode]
	On suppose que pout tout $v \in \mathcal{B}_2$, $(\mathcal{B}_1\setminus \{u\}) \cup \{v\}$ n'est pas une base de $E$
	\begin{itemize}
		\item Comme $u \in \mathcal{B}_1 \setminus \mathcal{B}_2$, nécéssairement $\mathcal{B}_1 \neq \mathcal{B}_2$ donc $\mathcal{B}_2 \not\subset \mathcal{B}_1$, donc $\mathcal{B}_2\setminus\mathcal{B}_1 \neq \O$ 
		\item Soit $v \in \mathcal{B}_2\setminus\mathcal{B}_1$. Il existe $(\lambda_w)_{w\in\mathcal{B}_1}$ une famille de scalaires presque nulle telle que \[
				v = \sum_{w \in \mathcal{B}_1} \lambda_w w - \lambda_u u + + \sum_{w \in \mathcal{B}_1\setminus \{u\}}\lambda_w w
			\]
			Si $\lambda_u \neq 0_E$, alors
			\begin{align*}
				u &= \lambda_u^{-1}\left( v - \sum_{w \in \mathcal{B}_1 \setminus \{u\}} \lambda_w w \right)\\
					&\in \Vect(\mathcal{B}_1\setminus \{u\} \cup v)
			\end{align*}
			 donc $\mathcal{B}_1 \subset \Vect(\mathcal{B}_1\setminus \{u\} \cup \{v\})$\\
			 et donc $E \subset  \Vect(\mathcal{B}_1 \setminus \{u\} \cup \{v\})$ \\
			 et donc $\mathcal{B}_1 \setminus \{u\} \cup \{v\}$ engendre $E$ \\
			 donc $\mathcal{B}_1 \setminus \{u\} \cup \{v\}$ n'est pas libre\\
			 donc $v \in \Vect(\mathcal{B}_1\setminus \{u\})$ (car $\mathcal{B}_1 \setminus \{u\}$ est libre\\
			 donc $\lambda_u = 0_\mathbbm{K}$ $\lightning$\\`

			 Donc, $\lambda_u = 0_\mathbbm{K}$, docn $v \in \Vect(\mathcal{B}_1\setminus \{u\})$ \\
			 On vient de prouver que
			 \begin{align*}
			 	\mathcal{B}_2 \setminus \mathcal{B}_1 \subset \Vect(\mathcal{B}_1 \setminus \{u\})\\
			 	\mathcal{B}_1 \setminus \{u\} \subset \Vect(\mathcal{B}_1 \setminus \{u\})\\
			 \end{align*}
			 Comme $u \not\in \mathcal{B}_2$, \[
			 	\mathcal{B}_2 \subset \Vect(\mathcal{B}_1 \setminus \{u\})
			 \] donc \[
			 	E = \Vect(\mathcal{B}_2) \subset  \Vect(\mathcal{B}_1 \setminus \{u\})
			 \] donc $\mathcal{B}_1 \setminus \{u\}$ engendre $E$. Donc,  $\mathcal{B}_1 \setminus \{u\}$ est une base de $E$.\\
			 Or, $\mathcal{B}_1 \setminus \{u\} \subset  \mathcal{B}_1$, donc $\mathcal{B}_1 \setminus \{u\} = \mathcal{B}_1$
	\end{itemize}
\end{prv}

\begin{defn}
	Soit $E$ un $\mathbbm{K}$-espace vectoriel de dimension finie. Le cardinal commun à toutes les bases de $E$ est appelé \underline{dimension} de $E$ est notée $\dim(E)$ ou $\dim_\mathbbm{K}(E)$\\
	C'est donc aussi le nombre de coordonnées de n'importe quel vecteur dans n'importe quelle base.
	\index{dimension (espace vectoriel)}
\end{defn}

\begin{exm}
	\begin{enumerate}
		\item $\dim_\R(\C) = 2$ et $\dim_\C(\C) = 1$ 
		\item $\dim_\mathbbm{K}(\mathbbm{K}^{n}) = n$ 
		\item $\dim_{\mathbbm{K}}(\mathcal{M}_{n,p}(\mathbbm{K})) = np$
	\end{enumerate}
\end{exm}

\begin{crlr}
	Soit $E$ un $\mathbbm{K}$-espace vectoriel de dimension finie, $\mathcal{L}$ une famille libre de $E$, $\mathcal{G}$ une famille génératrice de $E$. On note $n = \dim(E)$
	\begin{enumerate}
		\item $\#\mathcal{G} \ge n$ et $(\#\mathcal{G} = n \implies \mathcal{G} \text{ est une base de } E$)
		\item $\#\mathcal{L} \le n$ et $(\#\mathcal{L} = n \implies \mathcal{L} \text{ est une base de } E$)
	\end{enumerate}
\end{crlr}

\begin{crlr}
	$\R^{\R}$ est de dimension infinie.
	$\forall i \in \N, e_i: x \mapsto x^i$\\
	$(e_i)_{i\in\N}$ est libre dans $\R^\R$
\end{crlr}

\begin{prop}
	Soient $E$ et $F$ deux $\mathbbm{K}$-espaces vectoriels de dimension finie. Alors $E\times F$ est de dimension finie et $\dim(E\times F) = \dim(E) + \dim(F)$
\end{prop}

\begin{prv}
	Soit $(e_1,\ldots, e_n)$ une base de $E$, $(f_1, \ldots, f_p)$ une base de $F$.
	On pose \[
		\left\{\begin{array}
			{r c l}
			u_1 &=& (e_1,0_F)\\
			u_2 &=& (e_2,0_F)\\
					&\vdots&\\
			u_n &=& (e_n,0_F)\\
			u_{n+1} &=& (0_E, f_1)\\
			u_{n+2} &=& (0_E, f_2)\\
					&\vdots&\\
			u_{n+p} &=& (0_E,f_p)\\
		\end{array}\right.
	\]
	Soit $(x,y) \in E\times F$. \[
		\begin{cases}
			\exists (x_1,\ldots,x_n)\in \mathbbm{K}^n, x = \sum_{i=1}^{n} x_ie_i
			\exists (y_1,\ldots,y_n)\in \mathbbm{K}^n, x = \sum_{j=1}^{p} y_jf_j
		\end{cases}
	\] 
	\begin{align*}
		(x,y) &= \left( \sum_{i=1}^{n} x_ie_i, \sum_{i=1}^{p} y_jf_j \right)  \\
		&= \sum_{i=1}^{n} x_i (e_i + 0_F) + \sum_{j=1}^{p} y_j (0_E, f_j) \\
		&= \sum_{i=1}^{n} x_i u_i + \sum_{j=1}^{p} y_j u_{n+j} \\
	\end{align*}
	Donc, $E\times F = \Vect(u_1, \ldots, u_{n+p})$ donc $E\times F$ est de dimension finie.\\
	Soit $(\lambda_1, \ldots, \lambda_{n+p}) \in \mathbbm{K}^{n+p}$ tel que \[
		(*): \quad \sum_{k=1}^{n+p} \lambda_ku_k = 0_{E\times F} = (0_E, 0_F)
	\]
	\begin{align*}
		(*) &\iff \sum_{k=1}^{n} \lambda_k (e_k, 0_F) + \sum_{k=n+1}^{p} \lambda_k(0_E, f_{k-n}) = (0_E, 0_F)\\
				&\iff \begin{cases}
					\sum_{k=1}^{n} \lambda_k e_k = 0_E\\
					\sum_{k=n+1}^{p} \lambda_k f_{k-n} = 0_F
				\end{cases}\\
				&\iff \begin{cases}
					\forall k \in \left\llbracket 1,n \right\rrbracket, \lambda_k = 0_\mathbbm{K} \qquad&(\text{car $(e_1,\ldots,e_n)$ est libre})\\
					\forall k \in \left\llbracket n+1,n+p \right\rrbracket, \lambda_k = 0_\mathbbm{K} \qquad&(\text{car $(f_1,\ldots,f_n)$ est libre})\\
				\end{cases}
	\end{align*}
	Donc $(u_1, \ldots, u_{n+p})$ est une base de $E\times F$. Donc, $\dim(E\times F) = n + p = \dim(E) + \dim(F)$
\end{prv}

\begin{rmk}
	[Convention]
	\[\dim\big(\{0_E\}\big) = 0\]
\end{rmk}

\begin{thm}
	Soit $E$ un $\mathbbm{K}$-espace vectoriel de dimension finie, $F$ un sous-espace vectoriel de $E$. Alors, $F$ est de dimension finie et  $\dim(F) \le \dim(E)$\\
	Si $\dim(F) = \dim(E)$, alors $F = E$
\end{thm}

\begin{prv}
	On considère \[
		A = \{k \in \N \mid \text{il existe une famille libre de $F$ à $k$ éléments}\} 
	\]
	On suppose $F \neq \{0_E\}$.
	\begin{itemize}
		\item Soit $u \in F\setminus \{0_E\}$. $(u)$ est libre donc $1 \in A$ et donc $A \neq \O$
		\item Soit $\mathcal{L}$ une famille libre de $F$. Alors, $\mathcal{L}$ est une famille libre de $E$ \\
			donc $\#\mathcal{L} \le \dim(E)$\\
			Donc $A$ est majorée par $\dim(E)$ \\
			On en déduit que $A$ a un plus grand élément $p$.
		\item Soit $\mathcal{L}$ une famille libre de $F$ avec $p$ éléments.\\
			Si $\mathcal{L}$ n'engendre pas $F$, alors il existe $u\in F$ tel que $u\not\in \Vect(\mathcal{L})$ et donc $\mathcal{L} \cup \{u\}$ est une famille libre de $F$, donc $p+1 \in A$ en contradiction avec la maximalité de $p$.\\
			Donc $\mathcal{L}$ est une base de $F$ donc $F$ est de dimension finie et $\dim(F) = p \le \dim(E)$\\
	\end{itemize}

	Soit $\mathcal{B}$ une base de $F$. Alors, $\mathcal{B}$ est aussi une famille de libre de de $E$. Donc $\#\mathcal{B} \le \dim(E)$ donc $\dim(F) = \dim(E)$ \\
	Si $\dim(F) = \dim(E)$, alors $\mathcal{B}$ est une base de $E$, et donc $F = \Vect(\mathcal{B}) = E$
\end{prv}

\begin{prop}
	[Formule de Grassmann]
	Soit $E$ un $\mathbbm{K}$-espace vectoriel de dimension finie, $F$ et $G$ deux sous-espace vectoriels de $E$. Alors, \[
		\dim(F+G) = \dim(F) + \dim(G) - \dim(F\cap G)
	\] 
\end{prop}

\begin{prv}
	Soit $(e_1, \ldots, e_p)$ une base de $F\cap G$. $(e_1,\ldots,e_p)$ est une famille libre de $F$.\\
	On complète $(e_1, \ldots, e_p)$ en une base $(e_1, \ldots, e_p, u_1, \ldots, u_q)$ de $F$.\\
	De même, on complète $(e_1, \ldots, e_p)$ en une base $(e_1, \ldots, e_p, v_1, \ldots, v_r)$ de $G$.\\
	On pose  $\mathcal{B} = (e_1, \ldots, e_p, u_1, \ldots, u_q, v_1, \ldots, v_r)$. Montrons que $\mathcal{B}$ est une base de $F+G$
	\begin{itemize}
		\item Soit $u \in F+G$ \\
			On pose $u = v+w$ avec $\begin{cases}
				v\in F\\
				w \in G
			\end{cases}$.\\
			On pose $v = \sum_{i=1}^p \lambda_i e_i + \sum_{i=1}^q \mu_i u_i$ avec $(\lambda_1, \ldots, \lambda_p, \mu_1, \ldots, \lambda_q) \in \mathbbm{K}^{p+q}$\\
			On pose aussi $w = \sum_{i = 1}^p \lambda'_ie_i + \sum_{j=1}^r \nu_j v_j$ avec $(\lambda_1',\ldots,\lambda_p', \nu_1, \ldots, \nu_r) \in \mathbbm{K}^{p+r}$\\
			D'où, \[
				u = \sum_{i=1}^p (\lambda_i + \lambda'_i)e_i + \sum_{j=1}^q \mu_j u_j + \sum_{k=1}^r \nu_k v_k \in \Vect(\mathcal{B})
			\]
		\item Soient $(\lambda_1, \ldots, \lambda_p, \mu_1, \ldots, \mu_q, \nu_1, \ldots, \nu_r) \in \mathbbm{K}^{p+q+r}$.\\
			On suppose \[
				(*)\quad \sum_{i=1}^{p}\lambda_ie_i + \sum_{j=1}^q\mu_ju_j + \sum_{k=1}^r \nu_k v_k = 0_E
			\] 
			D'où, \[
				\underbrace{\sum_{i=1}^p\lambda_i e_i + \sum_{j=1}^q \mu_ju_j}_{\in F} = \underbrace{-\sum_{k=1}^r\nu_jv_k}_{\in G}
			\] 
			Donc, \[
				f = \sum_{i=1}^p \lambda_i e_i + \sum_{j=1}^q \mu_j u_j \in F\cap G
			\] Comme $(e_1, \ldots, e_p)$ est une base de $F\cap G$, $\exists ! (\lambda_1', \ldots, \lambda_p') \in \mathbbm{K}^p$ tel que \[
				f = \sum_{i=1}^p \lambda'_i e_i = \sum_{i=1}^p \lambda'_i e_i + \sum_{j=1}^q 0_\mathbbm{K}u_j
			\] Comme $(e_1, \ldots, e_p, u_1, \ldots, u_q)$ est une base de $F$, \[
				\forall k \in \left\llbracket 1, q \right\rrbracket, \mu_j = 0_\mathbbm{K}
			\] De même, \[
				\forall k \in \left\llbracket 1,r \right\rrbracket , \nu_k = 0_\mathbbm{K}
			\] On remplace dans $(*)$ pour trouver \[
				\sum_{i=1}^p \lambda_ie_i = 0_E
			\] Comme $(e_1, \ldots, e_p)$ est libre, \[
				\forall i \in \left\llbracket 1,p \right\rrbracket, \lambda_i = 0_\mathbbm{K}
			\] Donc $\mathcal{B}$ est libre.\\
			Donc, 
			\begin{align*}
				\dim(F+G) &=  p +q + r \\
				&= (p+q)+ (p+r) - p \\
				&= \dim(F) + \dim(G) - \dim(F\cap G) \\
			\end{align*}
	\end{itemize}
\end{prv}

\begin{crlr}
	Avec les hypothèse précédentes, \[
		E = F \oplus G \iff \begin{cases}
			F \cap  G = \{0_E\} \\
			\dim(E) = \dim(F) + \dim(G)
		\end{cases}
	\] 
\end{crlr}

\begin{prv}
	\begin{itemize}
		\item[``$\implies$''] On suppose $E = F \oplus G$ \\
			Comme la somme est directe, $F \cap G = \{0_E\}$ 
			\begin{align*}
				\dim(E) &= \dim(F)\\
				&= \dim(F) + \dim(G) - \dim(F\cap G)\\
				&= \dim(F) + \dim(G)\\
			\end{align*}
		\item[``$\impliedby$''] On suppose $F\cap G = \{0_E\}$ et $\dim(E) = \dim(F) + \dim(G)$.\\
			On sait déjà que $F+G = F \oplus G$\\
			 \begin{align*}
				\dim(F+G) = \dim(F) + \dim(G) - \dim(F \cap G) = \dim(E)
			\end{align*}
			Donc $F + G = E$
	\end{itemize}
\end{prv}

\begin{prop}
	Soit $F$ un $\mathbbm{K}$-espace vectoriel de dimension finie $n$. Soit $\mathcal{B} = (e_1, \ldots, e_n)$ une base de $F$. L'application
	\begin{align*}
		f: \mathbbm{K}^n &\longrightarrow F \\
		(\lambda_1, \ldots, \lambda_n) &\longmapsto \sum_{i=1}^n \lambda_i e_i
	\end{align*} est bijective.\\
	Si $\mathbbm{K}$ est infini, $\mathbbm{K}^n$ aussi et donc $F$ aussi.\\
	Si $\#\mathbbm{K} = p \in \N_*$,
	\begin{align*}
		\#&\mathbbm{K}^n = p^n\\
		&\vrt=\\
		\#&F
	\end{align*}
\end{prop}


		\part{Dérivation}

\underline{Motivation}:

{
\begin{wrapfigure}{l}{3cm}
	\centering
	\begin{asy}
		import three;

		size(3cm);
		settings.render=0;
		settings.prc=false;
		currentprojection = obliqueZ;

		draw(unitbox);
		draw(shift(1.1Z + 0.05X) * (O -- X), Arrows3(TeXHead2));
		draw(shift(1.1Z + 0.05Y) * (O -- Y), Arrows3(TeXHead2));
		draw(shift(1.1X + 0.05Z) * (O -- Z), Arrows3(TeXHead2));

		label("$x$", (X/2) + (1.1Z + 0.05X), align=S);
		label("$y$", (Y/2) + (1.1Z + 0.05Y), align=W);
		label("$z$", (Z/2) + X, align=SE);
	\end{asy}
\end{wrapfigure}

\begin{align*}
	&S(x,y,z) = 2(xy + xz + yz)\\
	&V(x,y,z) = xyz
\end{align*}

On cherche à minimiser $S$ avec la contrainte $V = 1$.

Soit $f : \begin{array}{rcl}
	\left( \R_*^+ \right)^2 &\longrightarrow& \R \\
	(x,y) &\longmapsto& S\left( x,y,\frac{1}{xy} \right) = 2\left( xy + \frac{1}{y} + \frac{1}{x} \right).
\end{array}$

On cherche $(a,b) \in \left( \R^+_* \right)^2$ tel que \[
	\forall (x,y) \in (\R^+_*), f(x,y) \ge f(a,b).
\]
}

\begin{defn}
	Soit $f: U \to \R$ où $U$ est un ouvert de $\R^2$. Soit $(a,b) \in U$.
	\vspace{2mm}

	Si $\lim_{x \to a} \frac{f(x,b) - f(a,b)}{x - a} \in \R$, alors on dit que $f$ a une dérivée partielle suivant $x$ en $(a,b)$ et cette limite est notée \[
		\partial f_1(a,b) = \frac{\partial f}{\partial x}(a,b).
	\]

	Si $\lim_{y \to b} \frac{f(a,y) - f(a,b)}{y - b} \in \R$, alors on dit que $f$ a une dérivée partielle suivant $y$ et la limite est notée \[
		\partial f_2(a,b) = \frac{\partial f}{\partial y}(a,b).
	\]
\end{defn}

\begin{exm}
	\begin{enumerate}
		\item $f: (x,y) \mapsto xy + x - y$.

			\begin{align*}
				&\frac{\partial f}{\partial x} : (x,y) \mapsto y + 1,\\
				&\frac{\partial f}{\partial y} : (x,y) \mapsto x - 1.
			\end{align*}

		\item $f: (x,y) \mapsto xy + \frac{1}{y}+ \frac{1}{x}$.

			\begin{align*}
				&\frac{\partial f}{\partial x}: (x,y) \mapsto y - \frac{1}{x^2},\\
				&\frac{\partial f}{\partial y}: (x,y) \mapsto x - \frac{1}{y^2}.
			\end{align*}

		\item Trouver $f$ telle que $\begin{cases}
				(1): \qquad \frac{\partial f}{\partial x}=y,\\[2mm]
				(2): \qquad \frac{\partial f}{\partial y} = x.
			\end{cases}$

			D'après $(1)$ : \[
				\forall (x,y), \exists C(y) \in \R, f(x,y) = xy + C(y)
			\] et donc \[
				\frac{\partial f}{\partial y}(x,y) = x + C'(y)
			\] donc $C'(y) = 0$ et donc $C$ est constante.

		\item Trouver $f$ telle que $\begin{cases}
			\frac{\partial f}{\partial x} = -y,\\[2mm]
			\frac{\partial f}{ƒ\partial y} = x.
		\end{cases}$

		Ce n'est pas possible !
	\end{enumerate}
\end{exm}

\begin{defn}~\\
	\begin{minipage}{\linewidth}
		\begin{wrapfigure}{r}{4cm}
			\centering
			\vspace{-5mm}
			\begin{asy}
				import three;
				import graph3;
				size(4cm);

				settings.render = 0;
				settings.prc = false;
				currentprojection = obliqueX;

				draw(O -- X, Arrow3(TeXHead2));
				draw(O -- Y, Arrow3(TeXHead2));
				draw(O -- Z, Arrow3(TeXHead2));

				triple f(real x, real y, real z = 0) { return (x,y,cos(x - 0.5) * cos(y - 0.5)/1.2 + 0.15); }

				real inc = 1 / 5;

				for(real x = 0; x <= 1; x += inc) {
					draw(graph(
						new real(real t) { return x; }, // x
						new real(real y) { return y; }, // y
						new real(real y) { return f(x,y).z; }, // z
						0, 1
					), gray);
				}

				for(real y = 0; y <= 1; y += inc) {
					draw(graph(
						new real(real x) { return x; }, // x
						new real(real t) { return y; }, // y
						new real(real x) { return f(x,y).z; }, // z
						0, 1
					), gray);
				}

				path3 path1 = (0.8, 0.2, 0) .. (0.5, 0.5, 0) .. (0.3, 0.7, 0);
				path3 path2 = f(0.8, 0.2, 0) .. f(0.5, 0.5, 0) .. f(0.3, 0.7, 0);
				path3 d = (0.2, 0.3, 0) .. (0.3, 0.4, 0) .. (0.2, 0.7, 0) .. (0.8, 0.9, 0) .. (0.6, 0.2, 0) .. cycle;

				draw(path1, red, Arrow3(TeXHead2));
				draw(path2, red, Arrow3(TeXHead2, position=0.8));

				dot((0.5, 0.5, 0));
				dot(f(0.5, 0.5, 0));
				draw((0.5, 0.5, 0) -- f(0.5, 0.5, 0), dashed);
				draw(d);

				label("$w$", (0.3, 0.7, 0), red, align=SE);
				label("$U$", (0.8, 0.9, 0), align=SE);
			\end{asy}
		\end{wrapfigure}

		Soit $f: U \to \R$ où $U$ est un ouvert. Soit $(a,b) \in U$. Soit $w = (w_1, w_2) \in \R^2$.

		Si 
		\[
			\lim_{t\to 0} \frac{f(a + tw_1, b + tw_2) - f(a,b)}{t}
		\] existe et est réelle, alors on dit que $f$ a une dérivée dans la direction de $w$ et la limite est notée \[
			\mathrm{d}f(w)\,(a,b) = D_w(f)\,(a,b).
		\]
	\end{minipage}
\end{defn}

\begin{exm}
	\begin{align*}
		f: \left( \R_*^+ \right)^2 &\longrightarrow \R \\
		(x,y) &\longmapsto xy+\frac{1}{x}+\frac{1}{y}.
	\end{align*}

	On pose $(a,b) = (1,2)$, $w = (w_1, w_2) = (1,1)$.
	\begin{align*}
		\frac{f(1+t, 2+t) - f(1,2)}{t} &= \frac{1}{t} \left( (1+t)(2+t) + \frac{1}{1+t} + \frac{1}{2+t} - 3 - \frac{1}{2} \right) \\
		&= \frac{1}{t}\left(\cancel 2 + 3t + \po(t) + \cancel 1 - t + \po(t) + \frac{1}{2}\left( \cancel 1 - \frac{t}{2} + \po(t) \right) - \cancel3 - \cancel{\frac{1}{2}} \right) \\
		&= \frac{1}{t} \left( \frac{7}{4} t + \po(t) \right)  \\
		&= \frac{7}{4} + \po(1) \tendsto{t \to 0} \frac{7}{4}. \\
	\end{align*}

	Donc, \[
		\mathrm{d}f(1,1)\,(1,2) = \frac{7}{4}.
	\]
\end{exm}

\begin{rmk}~\\
	\begin{figure}[H]
		\centering
		\begin{asy}
			import solids;
			import graph;
			size(5cm);

			settings.render = 0;
			settings.prc = false;

			path3 par = graph(
				new real(real x) { return x; },
				new real(real x) { return 0; },
				new real(real x) { return x^2; },
				0,3);
			revolution r = revolution(par, axis=Z);

			path3 par2 = graph(
				new real(real x) { return x; },
				new real(real x) { return 0; },
				new real(real x) { return x^2; },
				-3,3);

			draw(r,1,longitudinalpen=nullpen);
			draw(r.silhouette());

			draw((-4, 0, -1) -- (-4, 0, 10) -- (4, 0, 10) -- (4, 0, -1) -- cycle, red);
			draw(par2, deepred);

			draw((4,4.5) -- (7, 4.5), black+0.5mm, Arrow(TeXHead));

			path par2d = graph(new real(real x) { return x^2; }, -3, 3);
			draw(shift((11, 0)) * par2d, deepred);

			dot(O);
			dot((11, 0));
		\end{asy}
	\end{figure}
\end{rmk}


%todo ajouter théorème-définition
\begin{thm}
	Soit $f : U \to \R$, $(a,b) \in U$. On suppose que $\frac{\partial f}{\partial x}$ et $\frac{\partial f}{\partial y}$ existent en $(a,b)$ et sont {\bfseries continues} en $(a,b)$. Alors,
	\begin{align*}
		&\forall (h, k) \in \R^2 \text{ tel que } (a +h, b + k) \in U,\\
		&f(a+ h, b + k) = f(a,b) + h \frac{\partial f}{\partial x}(a,b) + k \frac{\partial f}{\partial y}(a,b) + \po_{(h,k)\to (0,0)}\big(\|(h,k)\|\big).
	\end{align*}

	On dit que $f$ est de classe $\mathcal{C}^1$ si $\frac{\partial f}{\partial x}$ et $\frac{\partial f}{\partial y}$ existent et sont continues.

	\qed
\end{thm}

\begin{rmk}
	En physique, cette formule correspond à : \[
		\mathrm{d}f = \frac{\partial f}{\partial x}\mathrm{d}x + \frac{\partial f}{\partial y} \mathrm{d}y.
	\] En effet :
	\begin{align*}
		\mathrm{d}f &= f(x+ \mathrm{d}x, y + \mathrm{d}y) - f(x,y) \\
		&= \frac{\partial f}{\partial x} \mathrm{d}x + \frac{\partial f}{\partial y} \mathrm{d}y.
	\end{align*}
\end{rmk}

\begin{prop}
	Soit $f: U \to \R$ de classe $\mathcal{C}^1$ en $(a,b) \in U$. Alors, \[
		\forall w = (w_1, w_2) \in \R^2, \mathrm{d}f(w)\,(a,b) = w_1 \frac{\partial f}{\partial x}(a,b) + w_2 \frac{\partial f}{\partial y}(a,b).
	\]
\end{prop}

\begin{prv}
	Soit $w = (w_1, w_2) \in \R^2$. Soit $t \in \R^*$.
	\begin{align*}
		\frac{1}{t}\big(f(a + tw_1, b + tw_2) - f(a,b)\big)
		&= \frac{1}{t} \left( tw_1 \frac{\partial f}{\partial x}(a,b) + tw_2 \frac{\partial f}{\partial y}(a,b) + \po_{t \to 0}\big(\|tw\|\big) \right) \\
		&= w_1 \frac{\partial f}{\partial x}(a,b) + w_2 \frac{\partial f}{\partial y}(a,b) + \po_{t\to 0}(1) \\
		&\tendsto{t\to 0} w_1 \frac{\partial f}{\partial x}(a,b) + w_2\frac{\partial f}{\partial y}(a,b).
	\end{align*}
\end{prv}


\begin{defn}
	Avec les hypothèses précédentes, en posant \[
		\nabla f(a,b) = \left( \frac{\partial f}{\partial x}(a,b), \frac{\partial f}{\partial y}(a,b) \right) 
	\]on obtient \[
		\mathrm{d}f(w)\,(a,b) = \left<w  \mid \nabla f(a,b) \right>
	\] où $\left<\cdot|\cdot \right>$ est le produit scalaire.

	Le vecteur $\nabla f(a,b)$ est appelé \underline{gradient de $f$ en $(a,b)$}.

	Le développement limité à l'ordre 1 de $f$ devient \[
		f\big((a,b)+w\big) = f(a,b) + \left<w \mid \nabla f(a,b) \right> + \po_{w\to 0}(\|w\|)
	\]
\end{defn}

\begin{prop}
	Soit $f : U \to \R$ de classe $\mathcal{C}^1$.

	\begin{figure}[H]
    \centering
    \incfig{gradient}
	\end{figure}

	$\nabla f$ est orthogonal au lignes de niveaux de $f$, son orientation va dans le sens d'une augmentation de $f$.
\end{prop}

\begin{prv}
	Soit $\gamma : I \to U$ une courbe de niveau : \[
		\forall t \in I, f\big(\gamma(t)\big) = \text{cste}.
	\] D'après le lemme suivant : \[
		\forall t \in I, 0 = (f \circ \gamma)'(t) = \mathrm{d}f\big(\gamma'(t)\big)\big(\gamma(t)\big) = \left<\gamma'(t)  \mid \nabla f\big(\gamma(t)\big) \right>
	\] Donc $\nabla f\big(\gamma(t)\big)$ est orthogonal à $\gamma'(t)$.

	Pour tout $t \in I$, on pose $w(t) = t\, \nabla f\big(\gamma(t)\big)$. Donc \[
		f\big(\gamma(t) + w(t)\big) = f\big(\gamma(t)\big) + t \|\nabla f(\gamma(t))\|^2 + \po_{t \to 0}(t)
	\] Pour $t$ assez petit, $f\big(\gamma(t) + w(t)\big) - f\big(\gamma(t)\big)$ est du même signe que $t$.
\end{prv}

\begin{rmk}
	\begin{align*}
		V: \R^3 &\longrightarrow \R \\
		(x,y,z) &\longmapsto -mgz
	\end{align*}
	l'énerge potentielle de pesenteur

	On a donc \[
		\nabla V(x,y,z) = \left( \frac{\partial V}{\partial x}, \frac{\partial V}{\partial y}, \frac{\partial V}{\partial z} \right) = (0, 0, -mg) = \vec{P}.
	\]
\end{rmk}

\begin{lem}
	Soit $f : U \to \R$ de classe $\mathcal{C}^1$, $\gamma : \begin{array}{rcl}
		I &\longrightarrow& U \\
		t &\longmapsto& \big(x(t), y(t)\big)
	\end{array}$ où $x$ et $y$ sont dérivables.

	On pose \[
		\forall t \in I, \gamma'(t) = \big(x'(t), y'(t)\big).
	\] Alors $f \circ \gamma : I \to \R$ est dérivable et
	\begin{align*}
		\forall t \in I, (f \circ \gamma)'(t) &= \mathrm{d}f\big(\gamma'(t)\big) \big(\gamma(t)\big)\\
		&= \left<\gamma'(t)  \mid \nabla f\big(\gamma(t)\big)  \right> \\
		&= x'(t) \frac{\partial f}{\partial x}\big(x(t), y(t)\big) + y'(t) \frac{\partial f}{\partial y}\big(x(t),y(t)\big). \\
	\end{align*}
\end{lem}

\begin{prv}
	On fixe $t \in I$.

	\begin{align*}
		\forall h \neq 0, \frac{f \circ \gamma(t + h) - f \circ \gamma(t)}{h}
		&= \frac{1}{h}\big(f(\gamma(t)) + h\gamma'(t) + \po_{h\to 0}(h) - f(\gamma(t))\big) \\
		&= \frac{1}{h}\bigg(\cancel{f(\gamma(t))} + \left<h\gamma'(t) \mid \nabla f(\gamma(t)) \right> + \po_{h\to 0}(\|h\gamma'(t)\|) - \cancel{f(\gamma(t))}\bigg)\\
		&= \left<\gamma'(t) \mid \nabla f(\gamma(t)) \right> + \po_{h\to 0}(1) \\
		&\tendsto{h\to 0} \left<\gamma'(t)  \mid \nabla f(\gamma(t)) \right>
	\end{align*}
\end{prv}

\begin{defn}
	Soit $f : U \to \R$ de classe $\mathcal{C}^1$ et $(a,b) \in U$. On dit que $(a,b)$ est un \underline{point critique} de $f$ si $\nabla f(a,b) = 0$ i.e. $\frac{\partial f}{\partial x}(a,b) = \frac{\partial f}{\partial y}(a,b) = 0$.

	Dans ce cas, $f(a,b)$ est appelé \underline{valeur critique} de $f$.
\end{defn}

\begin{prop}~\\
	\begin{minipage}{\linewidth}
		\begin{wrapfigure}{r}{3cm}
			\centering
			\vspace{-1cm}
			\begin{asy}
				import solids;
				import graph;
				size(3cm);

				settings.render = 0;
				settings.prc = false;

				path3 par = graph(
					new real(real x) { return x; },
					new real(real x) { return 0; },
					new real(real x) { return -x^2; },
					0,3);
				revolution r = revolution(par, axis=Z);

				draw(r,1,longitudinalpen=nullpen);
				draw(r.silhouette());

				dot("$(a,b)$", O, red, align=N);
				real s = sqrt(2.5);
				path3 g=(s,0,-2.5)..(0,s,-2.5)..(-s,0,-2.5)..(0,-s,-2.5)..cycle;
				draw(g, deepcyan);
			\end{asy}
		\end{wrapfigure}
		Soit $f: U \to \R$ de classe $\mathcal{C}^1$ et $(a,b) \in U$ tel que \[
			\exists r > 0, \forall (x,y) \in B_{(a,b)}(r), f(x,y) \le f(a,b)
		\] Alors $\nabla f(a,b) = (0,0)$.
	\end{minipage}
\end{prop}

\begin{prv}
	Soit $g: x \mapsto f(x,b)$. $g(a)$ est un maximum local de $g$ donc $g'(a) = 0$.

	Or, $g'(a) = \frac{\partial f}{\partial x}(a,b)$

	donc $\frac{\partial f}{\partial x}(a,b) = 0$.

	Soit $h : y \mapsto f(a,y)$. On a de même $h'(b) = 0$.

	Or, $h'(b) = \frac{\partial f}{\partial y}(a,b)$.

	Donc, $\nabla f(a,b) = (0,0)$.
\end{prv}

\begin{rmk}
	Un minimum local est aussi une valeur critique.
\end{rmk}

\begin{figure}[H]
	\centering
	\begin{subfigure}{3cm}
		\centering
		\begin{asy}
			import solids;
			import graph;
			size(3cm);

			settings.render = 0;
			settings.prc = false;

			path3 par = graph(
				new real(real x) { return x; },
				new real(real x) { return 0; },
				new real(real x) { return -x^2; },
				0,3);
			revolution r = revolution(par, axis=Z);

			draw(r,1,longitudinalpen=nullpen);
			draw(r.silhouette());

			dot(O, red);
		\end{asy}
		\caption{Maximum local}
	\end{subfigure}
	\begin{subfigure}{3cm}
		\centering
		\begin{asy}
			import solids;
			import graph;
			size(3cm);

			settings.render = 0;
			settings.prc = false;

			path3 par = graph(
				new real(real x) { return x; },
				new real(real x) { return 0; },
				new real(real x) { return x^2; },
				0,3);
			revolution r = revolution(par, axis=Z);

			draw(r,1,longitudinalpen=nullpen);
			draw(r.silhouette());

			dot(O, red);
		\end{asy}
		\caption{Minimum local}
	\end{subfigure}
	\begin{subfigure}{3cm}
		\centering
		\begin{asy}
			import solids;
			import graph;
			size(3cm);

			settings.render = 0;
			settings.prc = false;
			currentprojection = obliqueZ;

			draw(graph(
				new real(real x) { return x; },
				new real(real x) { return -x^2 / 3; },
				new real(real x) { return 3; },
				-3, 3
			));

			draw(graph(
				new real(real x) { return x; },
				new real(real x) { return -x^2 / 3; },
				new real(real x) { return -3; },
				-3, 3
			));

			draw(graph(
				new real(real x) { return x; },
				new real(real x) { return -x^2 / 3 - 1; },
				new real(real x) { return 0; },
				-3, 3
			));

			draw(graph(
				new real(real x) { return 0; },
				new real(real x) { return x^2 / 9 - 1; },
				new real(real x) { return x; },
				-3, 3
			));

			draw(graph(
				new real(real x) { return -3; },
				new real(real x) { return x^2 / 9 - 4; },
				new real(real x) { return x; },
				-3, 3
			));

			draw(graph(
				new real(real x) { return 3; },
				new real(real x) { return x^2 / 9 - 4; },
				new real(real x) { return x; },
				-3, 3
			));

			dot((0,-1,0), red);
		\end{asy}
		\caption{Point de selle / Point col}
	\end{subfigure}
\end{figure}

\begin{exm}
	On revient à l'exemple donné en introduction : 
	\begin{align*}
		f: \left( \R^*_+ \right)^2 &\longrightarrow \R \\
		(x,y) &\longmapsto 2\left( xy + \frac{1}{x} + \frac{1}{y} \right).
	\end{align*}

	$\left( \R^+_* \right)^2$ est un ouvert de $\R^2$. Soit $(x,y) \in \left( \R^+_* \right)^2$.
	
	On a \[
		\begin{cases}
			\frac{\partial f}{\partial x}(x,y) = 2\left( y - \frac{1}{x^2} \right),\\
			\frac{\partial f}{\partial y}(x,y) = 2\left( x - \frac{1}{y^2} \right).
		\end{cases}
	\]

	\begin{align*}
		&\frac{\partial f}{\partial x}(x,y) = \frac{\partial f}{\partial y}(x,y) = 0\\
		\iff& \begin{cases}
			y = \frac{1}{x^2}\\
			x = \frac{1}{y^2}
		\end{cases}\\
		\iff& \begin{cases}
			y = \frac{1}{x^2}\\
			x = x^4
		\end{cases}\\
		\iff& \begin{cases}
			x = 1\\
			y = 1
		\end{cases}
	\end{align*}

	On vérivie que $f$ présente en effet un minium local en $(1,1)$. \[
		f(1,1) = 6
	\] On fixe $y \in \R^+_*$ et \[
		g : x \mapsto 2\left( xy + \frac{1}{x} + \frac{1}{y} \right).
	\] Donc \[
		\forall x \in \R^+_*, g'(x) = 2\left( y - \frac{1}{x^2} \right).
	\]
	\begin{center}
		\begin{tikzpicture}
			\tkzTabInit{$x$/1,$g'(x)$/1,$g$/2.3}{$0$, $\frac{1}{\sqrt{y}}$, $+\infty$}
			\tkzTabLine{,-,z,+,}
			\tkzTabVar{+/{}, -/$2\left( 2\sqrt{y} +\frac{1}{y} \right)$, +/{}}
		\end{tikzpicture}
	\end{center}
	
	Ainsi, \[
		\forall x \in \R^+_*, \forall y \in \R^+_*, f(x,y) \ge 2\left( 2\sqrt{y} + \frac{1}{y} \right)
	\] Soit $h : y \mapsto 2\sqrt{y} + \frac{1}{y}$. On a \[
		\forall y > 0, h'(y) = \frac{1}{\sqrt{y}} - \frac{1}{y^2} = \frac{y\sqrt{y} - 1}{y^2} = \frac{y^{\frac{3}{2}} - 1}{y^2}
	\]

	\begin{center}
		\begin{tikzpicture}
			\tkzTabInit{$y$/0.7,$h'(y)$/0.7,$h$/1.4}{$0$, $1$, $+\infty$}
			\tkzTabLine{,-,z,+,}
			\tkzTabVar{+/{}, -/$3$, +/{}}
		\end{tikzpicture}
	\end{center}

	Donc, \[
		\forall x,y > 0, f(x,y) \ge 2\times 3 = 6 = f(1,1).
	\]
\end{exm}

\begin{prop}
	[règle de la chaîne]

	Soit $f : \begin{array}{rcl}
		U &\longrightarrow& \R^2 \\
		(x,y) &\longmapsto& f(x,y)
	\end{array}$ de classe $\mathcal{C}^1$ et $U, V$ deux ouverts de $\R^2$.

	Soit $\varphi : \begin{array}{rcl}
		V &\longrightarrow& U \\
		(u,v) &\longmapsto& \varphi(u,v) = \big(x(u,v), y(u,v)\big)
	\end{array}$.

	On suppose que $x$ et $y$ sont de classe $\mathcal{C}^1$ sur $V$.

	Alors,  $f \circ \varphi : \begin{array}{rcl}
		V &\longrightarrow& \R \\
		(u,v) &\longmapsto& f\big(\varphi(u,v)\big)
	\end{array}$ est de classe $\mathcal{C}^1$ et
	\begin{align*}
		\forall (u_0, v_0) \in V, \frac{\partial (f \circ \varphi)}{\partial u}(u_0, v_0)
		&= \frac{\partial f}{\partial x}\big(\varphi(u_0, v_0)\big) \times \frac{\partial x}{\partial u}(u_0, v_0)\\
		&+ \frac{\partial f}{\partial y}\big(\varphi(u_0,v_0)\big) \frac{\partial y}{\partial u}(u_0,v_0)
	\end{align*}
	\begin{align*}
		\forall (u_0, v_0) \in V, \frac{\partial (f \circ \varphi)}{\partial v}(u_0, v_0)
		&= \frac{\partial f}{\partial x}\big(\varphi(u_0, v_0)\big) \times \frac{\partial x}{\partial v}(u_0, v_0)\\
		&+ \frac{\partial f}{\partial y}\big(\varphi(u_0,v_0)\big) \frac{\partial y}{\partial v}(u_0,v_0)
	\end{align*}
\end{prop}

\begin{exm}
	[changement de coordonnées polaires]
	On pose \begin{align*}
		\varphi: \R^+_* \times ]0,2\pi[ &\longrightarrow \R^2\setminus \left( R^+_* \times \{0\} \right) \\
		(r, \theta) &\longmapsto (r \cos \theta, r \sin\theta),
	\end{align*}
	\begin{align*}
		f: \R^2\setminus \left( R^+_* \times \{0\} \right) &\longrightarrow \R \\
		(x,y) &\longmapsto f(x,y),
	\end{align*}
	\begin{align*}
		g: \overbrace{\R^+_* \times ]0, 2\pi[}^{=V} &\longrightarrow \R \\
		(r, \theta) &\longmapsto f(r\cos\theta, r\sin\theta).
	\end{align*}

	\begin{align*}
		\forall (r_0,\theta_0) \in V,&\\[5mm]
		\frac{\partial g}{\partial r}(r_0, \theta_0) &= \frac{\partial f}{\partial x}(r_0\cos\theta_0, r_0\sin\theta_0)\cos\theta_0\\
		&+ \frac{\partial f}{\partial y}(r_0 \cos\theta_0, r_0\sin\theta_0)\sin\theta_0\\
		&= 2r_0\cos^2\theta_0 + 2r_0\sin^2(\theta_0) \\
		&= 2r_0 \\[5mm]
		\frac{\partial g}{\partial \theta}(r_0, \theta_0) &= \frac{\partial f}{\partial x}(r_0\cos\theta_0, r_0\sin\theta_0)r_0\sin\theta_0\\
		&+ \frac{\partial f}{\partial y}(r_0 \cos\theta_0, r_0\sin\theta_0)r_0\cos\theta_0\\
		&= -2{r_0}^2\cos(\theta_0)\sin(\theta_0) + 2{r_0}^2 \sin(\theta_0)\cos(\theta_0)\\
		&= 0 \\
	\end{align*}

	Donc, \[
		g(r, \theta) = r^2.
	\]
\end{exm}

\begin{exm}
	Résoudre \[
		\begin{cases}
			\frac{\partial f}{\partial x} = \frac{x}{x^2+y^2},\\
			\frac{\partial f}{\partial y} = \frac{y}{x^2+y^2}.\\
		\end{cases}
	\]

	On pose $g: (r, \theta) \mapsto f(r \cos\theta, r \sin\theta)$.

	\begin{align*}
		&\frac{\partial g}{\partial r} = \frac{1}{r}\cos^2\theta + \frac{1}{r}\sin^2\theta = \frac{1}{r},\\
		&\frac{\partial g}{\partial \theta} = -\cos(\theta) \sin(\theta) + \sin(\theta)\cos(\theta) = 0.
	\end{align*}

	Donc, \[
		\exists C \in \R, g: (r, \theta) \mapsto \ln r + C
	\] d'où,
	\begin{align*}
		\forall (x,y) \in \R^2 \setminus \{(0,0)\}, f(x,y) &= \ln\left(\sqrt{x^2 + y^2} \right)  + C\\
		&= \frac{1}{2}\ln(x^2 + y^2) + C. \\
	\end{align*}
\end{exm}

\begin{rmk}
	Soit $\mathcal{B} = (e_1, e_2)$ la base canonique de $\R^2$, $f: U \to \R$ de classe $\mathcal{C}^1$ avec $U$ un ouvert de $\R^2$.

	Soit $(x,y) \in U$.

	\begin{align*}
		\Mat_{\mathcal{B}}\big(\nabla f(x,y)\big) = \begin{pmatrix}
			\frac{\partial f}{\partial x}(x,y)\\[2mm]
			\frac{\partial f}{\partial y}(x,y)
		\end{pmatrix}
	\end{align*}

	Soit  \begin{align*}
		\varphi: V &\longrightarrow U \\
		(u,v) &\longmapsto \big(x(u,v), y(u,v)\big) 
	\end{align*} avec $x,y$ de classe $\mathcal{C}^1$. Soit $g = f \circ \varphi$.
	\begin{align*}
		\Mat_{\mathcal{B}}\big(\nabla g(u,v)\big)
		&= \begin{pmatrix}
			\frac{\partial g}{\partial u}(u,v) \\[2mm]
			\frac{\partial g}{\partial v}(u,v)
		\end{pmatrix} \\
		&= \begin{pmatrix}
			\frac{\partial x}{\partial u}(u,v) \frac{\partial f}{\partial x}(x,y)
			+ \frac{\partial y}{\partial u}(u,v)\frac{\partial f}{\partial y}(x,y)\\[3mm]
			\frac{\partial x}{\partial v}(u,v) \frac{\partial f}{\partial x}(x,y)
			+ \frac{\partial y}{\partial v}(u,v) \frac{\partial f}{\partial y}(x,y)
		\end{pmatrix}  \\
		&= \underbrace{\begin{pmatrix}
				\frac{\partial x}{\partial u}(u,v)& \frac{\partial y}{\partial u}(u,v)\\[3mm]
				\frac{\partial x}{\partial v}(u,v)& \frac{\partial y}{\partial v}(u,v)
		\end{pmatrix}}_{J(u,v)} \begin{pmatrix}
			\frac{\partial f}{\partial x}(x,y)\\[3mm]
			\frac{\partial f}{\partial y}(x,y)
		\end{pmatrix} \\
		&= J(u,v) \Mat_{\mathcal{B}}\big(\nabla f(x,y)\big) \\
	\end{align*}
	où $J(u,v) = 
	\begin{pNiceArray}{c:c}
		\Mat_{\mathcal{B}}\big(\nabla x(u,v)\big) & \Mat_{\mathcal{B}}\big(\nabla y(u,v)\big)
	\end{pNiceArray}$.

	On dit que $J(u,v)$ est \underline{la jacobienne} de $\varphi$ en $(u,v)$.
	L'application linéaire canoniquement associée à $J(u,v)$ est la \underline{différentielle de $\varphi$} en $(u,v)$ noté $\mathrm{d}\varphi(u,v)$.

	On a $\mathrm{d}\varphi(u,v) \in \mathcal{L}(R^2)$ et $\Mat_{\mathcal{B}}\big(\mathrm{d}\varphi(u,v)\big) = J(u,v)$.

	Par exemple, la jacobienne du changement de coordonnées polaires est \[
		J = \begin{pmatrix}
			\frac{\partial x}{\partial r} & \frac{\partial y}{\partial r}\\[3mm]
			\frac{\partial x}{\partial \theta} & \frac{\partial y}{\partial \theta}
		\end{pmatrix}
		= \begin{pmatrix}
			\cos\theta&\sin\theta\\
			-r\sin\theta&r\cos\theta
		\end{pmatrix}.
	\]
	$\underbrace{\det(J)}_{\text{le jacobien}} = r\cos^2\theta + r\sin^2\theta = r$

	Dans une intégrale double, si $(x,y) = \varphi(u,v)$, alors $\mathrm{d}x\mathrm{d}y = \det(J)\mathrm{d}u\mathrm{d}v$.

	Ici, \[
		\mathrm{d}x\ \mathrm{d}y = r\ \mathrm{d}r\ \mathrm{d}\theta.
	\]
\end{rmk}

\begin{prv}
	On pose $(x_0, y_0) = \varphi(u_0, v_0)$. Pour tout $(h,k) \in \R^2$ tels que $(u_0 + h, v_0 + k) \in V$, en posant $g = f  \circ \varphi$.

	\begin{align*}
		g(u_0 + h, v_0 + h) &= f\big(x(u_0 + h, v_0 + k), y(u_0 + h, v_0 + k)\big) \\
		&= f\left(
			x(u_0,v_0) + h \frac{\partial x}{\partial u}(u_0,v_0) + k \frac{\partial x}{\partial v}(u_0, v_0) + \po\big(\|(h,k)\|\big), \right.\\
		&\phantom{ = f\bigg(\bigg.}\left. y(u_0, v_0) + h \frac{\partial y}{\partial u}(u_0, v_0) + k \frac{\partial y}{\partial v}(u_0, v_0) + \po\big(\|(h,k)\|\big)
		\right)  \\
		&= f(x_0,y_0) \\
		&~+ \left( h \frac{\partial x}{\partial u}(u_0,v_0) + k \frac{\partial x}{\partial v}(u_0, v_0) + \po(\|(h,k)\|) \right) \frac{\partial f}{\partial x}(x_0,y_0)\\
		&~+ \left( h \frac{\partial y}{\partial u}(u_0, v_0) + k\frac{\partial y}{\partial v}(u_0, v_0) + \po(\|(h,k)\|) \right) \frac{\partial f}{\partial y}(x_0, y_0)\\
		&~+ \po(\|(h,k)\|)\\
		&= f(x_0, y_0) \\
		&~+ h \left( \frac{\partial x}{\partial u}(u_0, v_0) \frac{\partial f}{\partial x}(x_0, y_0) + \frac{\partial y}{\partial u}(u_0, v_0) \frac{\partial f}{\partial y}(x_0, y_0) \right)  \\
		&~+ k\left( \frac{\partial x}{\partial v}(u_0, v_0) \frac{\partial f}{\partial x}(x_0, y_0) + \frac{\partial y}{\partial v}(u_0, v_0) \frac{\partial f}{\partial y}(x_0, y_0) \right) 
		&~+ \po(\|(h,k)\|)\\
		&= g(u_0, v_0) + h \frac{\partial g}{\partial u}(u_0, v_0) + k \frac{\partial g}{\partial v}(u_0, v_0) + \po(\|(h,k)\|) \\
	\end{align*}

	Par identification,
	\[
		\frac{\partial g}{\partial u}(u_0, v_0) = \frac{\partial x}{\partial u}(u_0, v_0) \frac{\partial f}{\partial x}(x_0, y_0) + \frac{\partial y}{\partial u}(u_0, v_0) \frac{\partial f}{\partial y}(x_0,y_0)
	\] et \[
		\frac{\partial g}{\partial v}(u_0, v_0) = \frac{\partial x}{\partial v}(u_0,v_0) \frac{\partial f}{\partial x}(x_0, y_0) + \frac{\partial y}{\partial v}(u_0, v_0) \frac{\partial f}{\partial y}(x_0, y_0).
	\] 
\end{prv}

\begin{exm}
	[Régression linéaire]~\\
	\begin{figure}[H]
		\centering
		\begin{asy}
			import graph;
			axes(EndArrow);
			size(5cm);

			real f(real x) { return x + 0.5; }

			real k = 35 / (7 - 0.5);

			for(int i = 0; i < 35; ++i) {
				real mag = exp(sin(100 * pi/exp(1) * i)) * 0.8 + exp(cos(i*40)/3);
				real eps = mag * cos(10 * exp(1)/pi * i) / 3;
				dot((i/k,f(i/k) + eps));
			}

			draw(graph(f, -1, 7), orange);
		\end{asy}
	\end{figure}
	\[
		y = a x + b
	\] 
	On fixe $(a,b) \in \R^2$. \[
		\varepsilon(a,b) = \sum_{i=1}^n\big( y_i - (ax_i + b) \big)^2
	\] l'erreur totale.

	On veut minimiser $\varepsilon(a,b)$. On a 
	\[
		\forall (a,b) \in \R^2,
		\begin{cases}
			\frac{\partial \varepsilon}{\partial a}(a,b) = -2\sum_{i=1}^{n}(y_i - ax_i - b)x_i,\\
			\frac{\partial \varepsilon}{\partial b}(a,b) = -2\sum_{i=1}^{n}(y_i - ax_i - b).
		\end{cases}
	\]

	Donc,
	\begin{align*}
		(a,b) \text{ point critique de } \varepsilon \iff& \begin{cases}
			a \sum_{i=1}^n {x_i}^2 + b\sum_{i=1}^{n}x_i = \sum_{i=1}^{n} y_ix_i\\
			a\sum_{i=1}^{n}x_i + nb = \sum_{i=1}^ny_i
		\end{cases}\\
		\iff& \begin{cases}
			a \left( \frac{1}{n}\sum_{i=1}^n {x_i}^2 - \overline{x}^2\right) = \overline{y} - \overline{x} \overline{y}\\
			b = \frac{1}{n}\sum_{i=1}^ny_i - \frac{a}{n}\sum_{i=1}^nx_i = \frac{1}{n}\sum_{i=1}^n x_i y_i - \overline{x} \overline{y}
		\end{cases}\\
		&\text{ où } \overline{x} = \frac{1}{n} \sum_{i=1}^n x_i,~\overline{y} = \frac{1}{n}\sum_{i=1}^n y_i\\
		\iff& \begin{cases}
			a = \frac{\Cov(x,y)}{V(x)}\\
			b = \overline{y} - a\overline{x}
		\end{cases}
	\end{align*}

	Coefficient de corrélation: $\frac{\Cov(x,y)}{\sigma_x \sigma_y} \in [-1, 1]$
\end{exm}












		\part{Corps}

\begin{exm}[Problème]
	\begin{itemize}
		\item 
			avec $A = \Z / 9 \Z$, résoudre $\overline{x}^2 = \overline{0}$ \\
			\begin{center}
				\begin{tabular}{|c|c|c|c|c|c|c|c|c|c|c|}
					\hline
					$\overline{x}$&$\overline{0}$& $\overline{1}$ &$\overline{2}$&$\overline{3}$ &$\overline{4}$ &$\overline{5}$ &$\overline{6}$ &$\overline{7}$ &$\overline{8}$& $\overline{9}$ \\
					\hline
					$\overline{x}^2$&$\overline{0}$ &$\overline{1}$ &$\overline{4}$ &$\overline{0}$ &$\overline{7}$ &$7$ &$\overline{0}$ &$\overline{4}$ &$\overline{1}$&$\overline{0}$\\
					\hline
				\end{tabular}
			\end{center}
			On a trouvé 3 solutions: $\overline{0}$, $\overline{3}$, $\overline{6}$.
		\item $\Z / 8\Z$
			\begin{center}
				\begin{tabular}{|c|c|c|c|c|c|c|c|c|}
					\hline
					$\overline{x}$& $\overline{0}$& $\overline{1}$& $\overline{2}$& $\overline{3}$& $\overline{4}$& $\overline{5}$& $\overline{6}$& $\overline{7}$\\
					\hline
					$\overline{x^2}$& $\overline{0}$& $\overline{1}$& $\overline{4}$& $\overline{1}$& $\overline{0}$& $\overline{1}$& $\overline{4}$& $\overline{1}$\\
					\hline
				\end{tabular}
			\end{center}
			$\overline{x}^2=7$ a 4 solutions: $\overline{1}, \overline{7}, \overline{3},\text{ et } \overline{5}$
		\item $A = \mathbbm{H} = \{a + bi + cj + dk  \mid  (a,b,c,d) \in \R^4\}$ \\
			$i^2 = j^2 = k^2 = -1$ 
			\begin{align*}
				\begin{array}{c c c}
					ij = k & jk = i & ji = j\\
					ji = -k & kj = -i & ik = -j
				\end{array}
			\end{align*}
			Dans cet anneau, $-1$ a 6 racines!
	\end{itemize}
\end{exm}

\begin{defn}
	Soit $(\mathbbm{K}, +, \times)$ un ensemble muni de deux lois de composition internes. On dit que c'est un \underline{corps} si
	 \begin{enumerate}
		\item $(\mathbbm{K}, \times)$ est un groupe abélien
		\item $(\mathbbm{K}, \times)$ est un monoïde commutatif
		\item $\forall x \in \mathbbm{K}\setminus \{0_\mathbbm{K}\}, \exists y \in \mathbbm{K}, xy = 1_\mathbbm{K}$
		\item $0_\mathbbm{K} \neq  1_\mathbbm{K}$
	\end{enumerate}
	\index{corps}
\end{defn}

\begin{exm}
	\begin{itemize}
		\item $(\C, +, \times)$ est un corps
		\item $(\R, +, \times)$ est un corps
		\item $(\Q, +, \times)$ est un corps
		\item $(\Z, +, \times)$ n'est pas un corps
	\end{itemize}
\end{exm}

\begin{prop}
	$(\Z / n\Z, +, \times)$ est un corps si et seulement si $n$ est premier.
\end{prop}

\begin{prv}
	\[
		\left( \Z / n\Z \right)^\times = \left\{ \overline{k}  \mid k \wedge n = 1 \right\}
	\] 
\end{prv}


\begin{prop}
	Tout corps est un anneau intègre.
\end{prop}

\begin{prv}
	Soit $(\mathbbm{K}, +, \times)$ un corps. Soient $(a,b) \in \mathbbm{K}^2$ tel que $a \times b = 0_\mathbbm{K}$.\\
	On suppose $a \neq  0_\mathbbm{K}$. Alors, $a$ est inversible et donc \[
		b = a^{-1} \times a \times b = a^{-1} \times 0_\mathbbm{K} = 0_\mathbbm{K}
	\] 
\end{prv}

\begin{exm}
	Soit $(\mathbbm{K},+,\times)$ un corps.\\
	Résoudre \[
		\begin{cases}
			x^2 = 1_\mathbbm{K}\\
			x \in \mathbbm{K}
		\end{cases}
	\]

	\begin{align*}
		x^2 = 1_\mathbbm{K} &\iff x^2 - 1_\mathbbm{K} = 0_\mathbbm{K}\\
		&\iff (x - 1_\mathbbm{K})(x+1_\mathbbm{K}) = 0_\mathbbm{K}\\
		&\iff x - 1_\mathbbm{K} = 0_\mathbbm{K} \text{ ou } x + 1_\mathbbm{K} = 0_\mathbbm{K}\\
		&\iff x = 1_\mathbbm{K} \text{ ou } x = -1_\mathbbm{K}
	\end{align*}

	Il y a au plus 2 solutions.
\end{exm}

\begin{prop}
	Soit $(\mathbbm{K},+,\times )$ un corps et $P$ un polynôme à coefficients dans $\mathbbm{K}$ de degré $n$. Alors, l'équation $P(x) = 0_{\mathbbm{K}}$ a au plus $n$ solutions dans $\mathbbm{K}$ 
	\qed
\end{prop}

\begin{crlr}[(Théorème de Wilson)]
	voir exercice 16 du TD 12
\end{crlr}


\begin{defn}
	Soit $(\mathbbm{K}, +, \times)$ un corps et $L\subset \mathbbm{K}$.\\
	On dit que $L$ est un \underline{sous corps} de $\mathbbm{K}$ si
	\begin{enumerate}
		\item $L$ est un anneau de $(\mathbbm{K}, +, \times)$ non nul
		\item $\forall x \in L\setminus \{0_\mathbbm{K}\}, x^{-1} \in L$ 
	\end{enumerate}
	\vspace{2mm}
	en d'autres termes si
	\begin{enumerate}
		\item $\forall (x,y) \in L^2, x - y \in L$
		\item $\forall (x,y) \in L^2, x \times y^{-1} \in L$
	\end{enumerate}
	\vspace{5mm}
	On dit aussi que $\mathbbm{K}$ est une \underline{extension} de $L$.
	\index{sous corps}
	\index{extension}
\end{defn}

\begin{prop}
	Tout sous corps est un corps. \qed
\end{prop}

\begin{defn}
	Soient $(\mathbbm{K}_1,+,\times )$ et $(\mathbbm{K}_2,+, \times)$ deux corps et $f: \mathbbm{K}_1 \to \mathbbm{K}_2$.\\
	On dit que $f$ est un \underline{morphisme de corps} si $f$ est un morphisme d'anneaux.\\
	i.e. si
	\[
		\begin{cases}
			\forall (x,y) \in {\mathbbm{K}_1}^2,& f(x+y) = f(x) + f(y)\\
			\forall (x,y) \in {\mathbbm{K}_1}^2,& f(x \times y) = f(x) \times f(y)\\
		\end{cases}
	\] 
	\index{homomorphisme (de corps)}
	\index{morphisme (de corps)}
\end{defn}

\begin{prop}
	Tout morphisme de corps est injectif.
\end{prop}

\begin{prv}
	Soit $f: \mathbbm{K}_1 \to \mathbbm{K}_2$ un morphisme de corps.\\
	\begin{itemize}
		\item $\Ker(f)$ est un sous groupe de $(\mathbbm{K}_1, +)$ 
		\item Soit $x \in \Ker(f)$ et $y \in \mathbbm{K}_1$ \[
				f(x \times y) = f(x) \times f(y) = 0_{\mathbbm{K}_2} \times f(y) = 0_{\mathbbm{K}_2}
			\]
		\item Soit $x \in \Ker(f) \setminus \{0_{\mathbbm{K}_1}\}$.\\
			Alors, $x$ est inversible.\\
			\begin{align*}
				\begin{rcases*}
					x \in \Ker(f)\\
					x^{-1} \in \mathbbm{K}_1
				\end{rcases*}& \text{ donc } x \times x ^{-1} \in \Ker(f)\\
				&\text{ donc } 1_{\mathbbm{K}_1} \in \Ker(f)\\
				&\text{ donc } f(1_{\mathbbm{K}_1}) = 0_{\mathbbm{K}_2}
			\end{align*}
			Or, $f(1_{\mathbbm{K}_1}) = 1_{\mathbbm{K}_2} \neq 0_{\mathbbm{K}_2}$
	\end{itemize}
	Donc, $\Ker(f) = \{0_{\mathbbm{K}_1}\}$ donc $f$ est injective.
\end{prv}

\begin{exm}
	$\begin{array}{cc}
		\C &\longrightarrow \C\\
		z &\longmapsto \overline{z}\\
	\end{array}$ est un morphisme de corps
\end{exm}



		\part{Opérations sur les séries}

\begin{prop}
	L'ensemble $E = \{u \in \C^\N  \mid \Sigma u_n \text{ converge}\}$ est un sous-espace vectoriel de $\C^\N$ et \begin{align*}
		S: E &\longrightarrow \C \\
		u &\longmapsto \sum_{n=0}^{+\infty} u_n
	\end{align*} est une forme linéaire.
	\qed
\end{prop}

\begin{rmk}
	La somme d'une série convergente et d'une série divergente diverge.
	Le produit d'une série divergente par un scalaire non nul diverge.
\end{rmk}

		\part{Comparaison de suites}

\begin{defn}
	Soient $u$ et $v$ deux suites réelles. On dit que $u$ est \underline{dominée} par  $v$ si \[
	\exists M\in \R, \exists N\in \N,\forall n\ge N,\left| u_n \right| \le M \left| v_n \right| 
	\] Dans ce cas, on note $u = O(v)$ ou $u_n = O(v_n)$ et on dit que "$u$ est un grand o de $v$"
\end{defn}

\begin{exm}
	En informatique, on dit qu'un alogirithme a une \underline{complexité linéaire} si son temps d'éxécution est un $O(n)$ 
	Par exemple, on calcule $a^n$ 

	\begin{itemize}
		\item Approche naïve
			\begin{algorithm}
				\begin{algorithmic}[1]
					\State $p \gets 1$
					\For{$i \in \left\llbracket 0,n-1 \right\rrbracket$}
						\State $p \gets p \times a$
					\EndFor
					\State \Return p
				\end{algorithmic}
			\end{algorithm}
			Complexité linéaire $O(n)$
		\item Exponentiation rapide\\
			On écrit $n$ en binaire: \begin{align*}
				n &= \overline{a_k a_{k-1}\ldots a_0}^{(2)}\\
					&= \sum_{i=0}^{k} a_i 2^i
			\end{align*} avec $(a_i) \in \left\{ 0,1 \right\} ^{k+1}$
			\begin{align*}
				a^n &= a^{\sum_{i=0}^{k} a_i 2^i} \\
				&= \prod_{i=0}^{k} a^{a_i 2^i}  \\
			\end{align*}
			
			\begin{algorithm}
				\begin{algorithmic}
					[1]

					\State $s \gets 0$
					\State $p \gets a$
					\For{ $i \in \left\llbracket 0, \log_2(n) \right\rrbracket$}
						\State $p \gets p \times p$
						\If{$a[i] = 1$}
							\State $s \gets s + p$
						\EndIf
					\EndFor
					\State \Return s
				\end{algorithmic}
			\end{algorithm}
			Compléxité logarithmique $O(\log_2(n))$
	\end{itemize}
\end{exm}


\begin{prop}
	$O$ est une relation réfléctive et transitive.
\end{prop}

\begin{prv}
	\begin{itemize}
		\item Soit $u$ une suite. On pose $M = 1$ et \[
			\forall n \in \N, \left| u_n \right| \le M \left| u_n \right|
			\] Donc $u = O(u)$.
		\item Soient $u, v, w$ trois suites telles que  \[
		\begin{cases}
			u = O(v)\\
			v = O(w)
		\end{cases}
		\] Soient $M_1,M_2 \in \R$ et $N_1,N_2\in \N$ tels que \[
		\begin{cases}
			\forall n \ge  N_1, \left| u_n \right| \le M_1 \left| v_n \right| \\
			\forall n \ge  N_2, \left| v_n \right| \le M_2 \left| w_n \right| \\
		\end{cases}
		\] 

		Nécéssairement, $M_1\ge 0$ et $M_2\ge 0$.\\
		Soit $N = \max(N_1,N_2)$. \[
		\forall n \ge  N, \left| u_n \right| \le M_1 \left| v_n \right| \le  M_1M_2 \left| w_n \right| 
		\] Donc $u = O(w)$
	\end{itemize}
\end{prv}

\begin{defn}
	Soient $u$ et $v$ deux suites. On dit que $u$ est \underline{négligeable} devant $v$ si \[
	\forall \varepsilon>0, \exists N\in \N, \forall n\ge N, \left| u_n \right| \le \varepsilon \left| v_n \right| 
	\] Dans ce cas, on note $u = o(v)$ ou $u_n = o(v_n)$ ou on le lit "$u$ est un petit o de $v$"
\end{defn}

\begin{prop}
	$o$ est une relation transitive, non-réfléctive
\end{prop}

\begin{prv}
	\begin{itemize}
		\item Soient $u$, $v$ et $w$ trois suites telles que \[
			\begin{cases}
				u = o(v)\\
				v = o(w)
			\end{cases}
			\] Soit $\varepsilon>0$. Soit $N_1\in \N$ tel que \[
			\forall n \ge N_1, \left| u_n \right| \le \sqrt{\varepsilon}  \left| v_n \right| 
			\] Soit $N_2\in \N$ tel que \[
			\forall n \ge N_2, \left| v_n \right| \le \sqrt{\varepsilon}  \left| w_n \right| 
			\] On pose $N = \max(N_1,N_2)$, alors \[
			\forall n \ge N, \left| u_n \right| \le \sqrt{\varepsilon}  \left| v_n \right| \le \underbrace{\sqrt{\varepsilon} \times \sqrt{\varepsilon}} _\varepsilon \left| w_n \right| 
			\] donc $u = o(w)$
		\item Soit $u$ une suite tel qu'il existe $N \in \N$ tel que \[
		\forall n \ge N, u_n > 0
		\] On suppose que $u = o(u)$, alors \[
		\forall \varepsilon>0,\exists N \in \N, \forall n \ge N, \left| u_n \right| \le \varepsilon \left| u_n \right| 
		\] On pose $\varepsilon = \frac{1}{2}$ alors \[
		\exists N \in \N, \forall n \ge N, \left| u_n \right| \le \frac{1}{2} \left| u_n \right| 
		\] une contradiction
	\end{itemize}
\end{prv}

\begin{prop}
	Soient $u$ et $v$ deux suites.
	\begin{itemize}
		\item $o(u) + o(u) = o(u)$
		\item $v \times o(u) = o(uv)$
		\item $o(u) \times o(v) = o(uv)$
		\item $o(o(u)) = o(u)$
	\end{itemize}
	\qed
\end{prop}

\begin{defn}
	Soient $u$ et $v$ deux suites. On dit que $u$ et $v$ sont \underline{équivalentes} si \[
	u = v + o(v)
	\] i.e. \[
	\forall \varepsilon >0, \exists N \in \N, \forall n \ge N, \left| u_n-v_n \right| \le \varepsilon\left| v_n \right| 
	\] Dans ce cas, on le note $u \sim v$
\end{defn}

\begin{prop}
	$\sim$ est une relation d'équivalence \qed
\end{prop}

\begin{prop}
	Soient $(u,v) \in \R^\N$. On suppose que $v$ ne s'annule pas à partir d'un certain rang
	\begin{enumerate}
		\item $u = o(v) \iff \left( \frac{u_n}{v_n} \right)$ bornée
		\item $u = o(v) \iff \frac{u_n}{v_n} \tendsto{n \to  +\infty} 0$
		\item $u \sim v \iff \frac{u_n}{v_n} \tendsto{n \to  +\infty} 1$
	\end{enumerate}
	\qed
\end{prop}

\begin{prop}
	[Suites de références]
	\begin{enumerate}
		\item $\ln^\alpha(n) = o(n^\beta)$ avec $(\alpha,\beta) \in \left( \R^+_* \right) ^2$ 
		\item $n^\beta = o(a^n)$ avec $\beta > 0$ et $a > 1$ 
		\item $a^n = o(n!)$ avec $a >1$ 
		\item $n! = o(n^n)$
	\end{enumerate}
\end{prop}


\begin{lem}
	[Exercice 10 du TD]
	Soit $u \in \left(\R^+_*\right)^\N$\\
	Si $\frac{u_{n+1}}{u_n} \tendsto{n \to +\infty} \ell < 1$ avec $\ell\in \R$,\\ alors $u_n \tendsto{n \to +\infty} 0$
\end{lem}

\begin{prv} [de la proposition]
	\begin{enumerate}
		\item par croissance comparée
		\item On pose $\forall n \in \N^*, u_n = \frac{n^\beta}{a^n}$. 
			\begin{align*}
				\forall  n \in \N^*, \frac{u_{n+1}}{u_n} &= \left( \frac{n+1}{n} \right) ^\beta \times \frac{1}{a} \\
				&= \frac{1}{a}\left( 1+\frac{1}{n} \right) ^\beta \\
				&\tendsto{n \to +\infty} \frac{1}{a} < 1
			\end{align*}
			Donc, $u_n \tendsto{n \to  +\infty} 0$
		\item On pose $\forall n \in \N, u_n = \frac{a^n}{n!}$ \[
			\forall n \in \N, \frac{u_{n+1}}{u_n} = \frac{a}{n+1} \tendsto{n \to +\infty} 0 < 1
			\] donc $u_n \tendsto{n \to +\infty} 0$
		\item On pose $\forall  n\in \N^*, u_n = \frac{n!}{n^n}$.
			\begin{align*}
				\forall n \in \N^*, \frac{u_{n+1}}{u_n}
				&= (n+1) {\frac{n^n}{(n+1)^{n+1}}} \\
				&= \left( \frac{n}{n+1} \right) ^n \\
				&= e^{n \ln\left( \frac{n}{n+1} \right) } \\
				&= e^{n \ln\left( 1+\frac{1}{n+1} \right)} \\
				&= e^{n(-\frac{1}{n} + o(\frac{1}{n})} \\
				&= e^{-1 + o(1)} \\
				&\tendsto{n \to  +\infty} e^{-1}<1
			\end{align*}
			donc $u_n \tendsto{n\to +\infty} 0$
	\end{enumerate}
\end{prv}

		\part{Matrices par blocs}

\begin{exm}
	Soit $p$ un projecteur de $E$ : \[
		E = \Ker p \oplus \mathrm{Im}\ p
	\] Soit $\mathcal{B} = (e_1, \ldots, e_k, e_{k+1}, \ldots, e_n)$ une base de $E$ avec $\begin{cases}
		\mathrm{Im}(p) = \Vect(e_1, \ldots, e_k)\\
		\Ker(p) = \Vect(e_{k+1}, \ldots, e_n)\\
	\end{cases}$

	Alors, 
	\begin{align*}
		\Mat_\mathcal{B}(p) =
		\left(\begin{NiceArray}{c c c | c c c}
				1&&&0&\Cdots&0\\
				 &\Ddots&&\Vdots&&\Vdots\\
				&&1&0&\Cdots&0\\\hline
				0&\Cdots&0&0&\Cdots&0\\
				\Vdots&&\Vdots&\Vdots&&\Vdots\\
				0&\Cdots&0&0&\Cdots&0\\
		\end{NiceArray}\right)
		= \left( \begin{array}{c|c}
				I_k & 0\\ \hline
				0&0
		\end{array}\right) \\
	\end{align*}

	De même, si $\s$ est une symétrie de $E$, \[
		E = \Ker(\s - \id_E) \oplus \Ker(\s + \id_E)
	.\] Soit $\mathcal{C} = (e_1', \ldots, e_\ell', e_{\ell+1}', \ldots, e'_n)$ avec $\begin{cases}
		\Vect(e'_1, \ldots, e'_\ell) = \Ker(\s - \id_E),\\
		\Vect(e'_{\ell+1}, \ldots, e'_n) = \Ker(\s + \id_E).\\
	\end{cases}$

	Alors
	\[
		\Mat_\mathcal{C}(\s) = \left(\begin{array}{c|c}
				I_\ell &0\\ \hline
				0&-I_{n-\ell}
		\end{array}\right) 
	\]
\end{exm}

\begin{prop}
	Soient $F$ et $G$ supplémentaires dans $E$ : \[
		E = F \oplus G.
	\] Soit $f \in \mathcal{L}(F)$ et $g \in \mathcal{L}(G)$. Alors \[
	\exists !h \in \mathcal{L}(E) h_{|F} = f,\ h_{|G} = g \et h = f \circ p + g \circ q
	\] où $\begin{cases}
		p \text{ est la projection sur $F$ parallèlement à $G$}\\
		q \text{ est la projection sur $G$ parallèlement à $F$}\\
	\end{cases}$.

	On a aussi $q = \id_E - p$.
\end{prop}

\begin{prv}
	\begin{itemize}
		\item[\sc \underline{Analyse}] Soit $h \in \mathcal{L}(E)$ tel que $\begin{cases}
				h_{|F}=f\\
				h_{|G}=g
			\end{cases}$.

			Soit $x \in E$. Alors \[
				x = \underbrace{p(x)}_{\in F} + \underbrace{q(x)}_{\in G}
			\]

			Donc,
			\begin{align*}
				h(x) &= h\big(p(x)\big) + h\big(q(x)\big)\\
				&= f\big(p(x)\big) + g\big(q(x)\big) \\
				&= (f \circ p + g \circ q)(x) \\
			\end{align*}
			Si $h$ existe, alors \[
				h = f \circ p + g \circ q
			\]
		\item[\underline{\sc Synthèse}] On pose $h = f \circ p + g  \circ q$.

			$p$, $q$, $f$ et $g$ sont linéaires donc $h$ aussi.

			Soit $x \in E$.
			\begin{align*}
				h(x) &= f\big(p(x)\big) + g\big(q(x)\big) \\
				&= f(x) + g(0_E) \\
				&= f(x) \\
			\end{align*}
			donc $h_{|F} = f$ et de même $h_{|G}=g$.
	\end{itemize}
\end{prv}

\begin{prop}
	On reprend les notations et hypothèses précédentes. Soit $(e_1, \ldots, e_p)$ une base de $F$, et $(f_1, \ldots, f_q)$ une base de $G$. Alors, $\mathcal{B} = (e_1, \ldots, e_p, f_1, \ldots, f_q)$ est une base de $E$ et \[
		\Mat_\mathcal{B}(h) = \left(
		\begin{array}{c|c}
			A&0\\ \hline
			0&B
		\end{array}\right)
	\] où $\begin{cases}
		A = \Mat_{(e_1, \ldots e_p)}(f)\\
		B = \Mat_{(f_1, \ldots, f_q)}(g)
	\end{cases}$
	\qed
\end{prop}

\begin{prop}
	Soient $(A,A') \in \mathcal{M}_n(\mathbbm{K})^2$ et $(B,B') \in \mathcal{M}_p(\mathbbm{K})^2$.
	\begin{enumerate}
		\item \[
				\left(\begin{array}{c|c}
					A&0\\ \hline
					0&B
				\end{array}\right)
				\left(\begin{array}{c|c}
					A'&0\\ \hline
					0&B'
				\end{array}\right) = 
				\left(\begin{array}{c|c}
					AA'&0\\ \hline
					0&BB'
				\end{array}\right)
			\]
		\item \[
				\left(\begin{array}{c|c}
					A&0\\ \hline
					0&B
				\end{array}\right) \in \mathrm{GL}_{n+p}(\mathbbm{K})	 \iff \begin{cases}
					 A \in \mathrm{GL}_n(\mathbbm{K})\\
					 B \in \mathrm{GL}_p(\mathbbm{K})
				\end{cases}
			\] et dans ce cas, \[
				\left(\begin{array}{c|c}
					A&0\\ \hline
					0&B
				\end{array}\right)^{-1} =
				\left(\begin{array}{c|c}
					A^{-1}&0\\ \hline
					0&B^{-1}
				\end{array}\right)
			\]
		\item \[
				\tr \left(\begin{array}{c|c}
					A&0\\ \hline
					0&B
				\end{array}\right) = \tr A + \tr B
			\]
	\end{enumerate}
\end{prop}

\begin{prv}
	\begin{enumerate}
		\item Soit $\begin{cases}
				f \in \mathcal{L}(F) \text{ tel que } \Mat_\mathcal{B}(f) = A,
				f' \in \mathcal{L}(F) \text{ tel que } \Mat_\mathcal{B}(f') = A',
				g \in \mathcal{L}(G) \text{ tel que } \Mat_\mathcal{C}(g) = B,
				g' \in \mathcal{L}(G) \text{ tel que } \Mat_\mathcal{C}(g') = B'
			\end{cases}$ où $\begin{cases}
				F \oplus G = \mathbbm{K}^{n+p},\\
				\dim(F) = n, \dim(G) = p,\\
				\mathcal{B} \text{ base de } F,\\
				\mathcal{C} \text{ base de } G.\\
			\end{cases}$
			Soit $\begin{cases}
				h \in \mathcal{L}(\mathbbm{K}^{n+p}) \text{ tel que } \begin{cases}
					h_{|F} = f\\
					h_{|G} = g
				\end{cases}\\
				h' \in \mathcal{L}(\mathbbm{K}^{n+p}) \text{ tel que } \begin{cases}
					h'_{|F} = f'\\
					h'_{|G} = g'\\
				\end{cases}
			\end{cases}$
			Soit $\mathcal{D} = \mathcal{B} \cup \mathcal{C}$ une base de $\mathbbm{K}^{n+p}$.
			\begin{align*}
				\left(\begin{array}{c|c}
					A&0\\ \hline
					0&B
				\end{array}\right)
				\left(\begin{array}{c|c}
					A'&0\\ \hline
					0&B'
				\end{array}\right) &= \Mat_{\mathcal{D}}(h) \Mat_{\mathcal{D}}(h')\\
				&= \Mat_{\mathcal{D}}(h \circ h') \\
			\end{align*}
			Or, $(h \circ h')_{|F} = f \circ f'$ et $(h \circ h')_{|G} = g \circ g'$.

			Donc,
			\begin{align*}
				\Mat_\mathcal{D}(h \circ h') &=
					\left(\begin{array}{c|c}
						\Mat_\mathcal{B}(f \circ f')&0\\ \hline
						0&\Mat_\mathcal{C}(g \circ g')
					\end{array}\right)\\
				&=\left(\begin{array}{c|c}
					AA'&0\\ \hline
					0&BB'
				\end{array}\right).
			\end{align*}
	\end{enumerate}
\end{prv}

\begin{prop}
	Soient $A,A' \in \mathcal{M}_n(\mathbbm{K})$, $B,B' \in \mathcal{M}_{n,p}(\mathbbm{K})$, $C,C' \in \mathcal{M}_{p,n}(\mathbbm{K})$ et $D, D' \in \mathcal{M}_p(\mathbbm{K})$.

	\[
		\left(\begin{array}{c|c}
			A&B\\ \hline
			C&D
		\end{array}\right)
		\left(\begin{array}{c|c}
			A'&B'\\ \hline
			C'&D'
		\end{array}\right) = 
		\left(\begin{array}{c|c}
			AA' + BC'& AB' + BD'\\ \hline
			CA' + DC'&CB' + DD'
		\end{array}\right)
	\] Cette formule se généralise à un nombre quelconque de blocs : \[
		\left(\begin{array}{c|c|c|c}
				A_{11}&A_{12}&\cdots&A_{1,n}\\ \hline
				A_{21}&A_{22}&\cdots&A_{2,n}\\ \hline
				\vdots&\vdots&\ddots&\vdots\\ \hline
				A_{p,1}&A_{p,2}&\cdots&A_{p,n}
		\end{array}\right)
		\left(\begin{array}{c|c|c|c}
				A'_{11}&A'_{12}&\cdots&A'_{1,n}\\ \hline
				A'_{21}&A'_{22}&\cdots&A'_{2,n}\\ \hline
				\vdots&\vdots&\ddots&\vdots\\ \hline
				A'_{p,1}&A'_{p,2}&\cdots&A'_{p,n}
		\end{array}\right)
	\] Cette matrice se calcyle comme on s'y attend si les dimensions des blocs autorisent les produits.
\end{prop}

\begin{prop}
	Le rang d'une matrice $A$, c'est la taille de la plus grande matrice carrée inversible que l'on peut extraire de $A$.
	\qed
\end{prop}




		\part{Trigonométrie hyperbolique}

\begin{defn}
	Pour tout $x \in \R$, on pose \[
		\begin{cases}
			\ch x = \frac{e^x + e^{-x}}{2},\\
			\sh x = \frac{e^x - e^{-x}}{2},\\
			\th x = \frac{\sh x}{\ch x}.
		\end{cases}
	\]

	$\ch$ est appelé \underline{cosinus hyperbolique}, $\sh$ est appelé \underline{sinus hyperbolique} et $\th$ est appelé \underline{tangeante hyperbolique}.
	\index{cosinus hyperbolique}
	\index{sinus hyperbolique}
	\index{tangente hyperbolique}
\end{defn}

\begin{rmk}
	Ces formules rappèlent les formules d'Euler : pour tout $x \in \R$,
	\begin{align*}
		\cos x = \frac{e^{ix} + e^{-ix}}{2}\quad\longleftrightarrow\quad\ch x = \frac{e^x + e^{-x}}{2}\\
		\sin x = \frac{e^{ix} - e^{-ix}}{2i}\quad\longleftrightarrow\quad\sh x = \frac{e^x - e^{-x}}{2}\\
	\end{align*}
\end{rmk}

\begin{figure}[H]
	\centering
	\begin{asy}
		import graph;

		size(12cm);

		pair A = (-2, 0);
		pair B = (2, 0);

		real eps = 0.05;

		draw(shift(A) * ((0, -1.3) -- (0, 1.3)), Arrow(TeXHead));
		draw(shift(A) * ((-1.3, 0) -- (1.3, 0)), Arrow(TeXHead));

		draw(circle(A, 1), magenta);
		
		real theta = 38;
		pair M = dir(theta) + A;
		draw(A -- M, red);
		draw(arc(A, 0.35, 0, theta), red, Arrow(TeXHead));
		draw(M -- (A.x-eps, M.y), dashed);
		draw(M -- (M.x, A.y-eps), dashed);
		label("\small$\theta$", 0.5dir(theta/2) + A, red);
		label("\small$\cos\theta$", (M.x, A.y), align=S);
		label("\small$\sin\theta$", (A.x, M.y), align=1.2W);
		dot("\small$M$", M);

		label("\small$x^2 + y^2 = 1$", A + 1.5dir(45+180));

		draw(shift(B) * ((0, -1.3) -- (0, 1.3)), Arrow(TeXHead));
		draw(shift(B) * ((-1.3, 0) -- (1.3, 0)), Arrow(TeXHead));

		real ch(real x) { return (exp(x) + exp(-x)) / 2.; }
		real sh(real x) { return (exp(x) - exp(-x)) / 2.; }
		real nch(real x) { return -ch(x); }

		real k = 1.9; real r = 1.2;
		real t = 1.4;

		draw(shift(B) * scale(0.35) * graph(ch, sh, -k, k), magenta);
		draw(shift(B) * scale(0.35) * graph(nch, sh, -k, k), magenta);

		label("\small$x^2 - y^2 = 1$", B + 1.5dir(45+180) + (0, -0.2));

		M = B + 0.35(ch(t), sh(t));

		draw(M -- (B.x-eps, M.y), dashed);
		draw(M -- (M.x, B.y-eps), dashed);
		dot("\small$M$", M);
		label("\small$\ch x$", (M.x, B.y), align=S);
		label("\small$\sh x$", (B.x, M.y), align=1.2W);

		draw(shift(B) * ((-r, -r)--(r,r)), gray + dashed);
		draw(shift(B) * ((r, -r)--(-r,r)), gray + dashed);
	\end{asy}
\end{figure}


		\part{Applications}
\section{Formule de Stirling}

\begin{prop}
	On a :
	\[
		n! \simi_{n\to +\infty} \sqrt{2\pi n} \left( \frac{n}{e} \right)^n?.
	\]
\end{prop}

\begin{prv}
	\[
		\forall n \in \N^*, \ln(n!) = \sum_{k=1}^n \ln k.
	\]

	$x \mapsto \ln x$ est strictement croissante sur $[1, +\infty[$ donc \[
		\forall k \in \N^*, \forall x \in [k, k+1], \ln x \ge \ln k
	\] donc \[
		\forall k \in \N^*, \int_{k}^{k+1} \ln x~\mathrm{d}x \ge \int_{k}^{k+1} \ln k~\mathrm{d}x = \ln k
	\] et \[
		\forall k \ge 2, \forall x \in [k - 1, k], \ln x \le \ln k
	\] et docn \[
		\forall k \ge 2, \int_{k-1}^{k}  \ln x~\mathrm{d}x \le \int_{k-1}^{k} \ln k~\mathrm{d}x = \ln k
	\] Ainsi \[
		\forall n \ge 2, 
		\int_{1}^{n} \ln x~\mathrm{d}x \ge \sum_{k=2}^n \le \int_{2}^{n+1} \ln x~\mathrm{d}x
	\] Or
	\begin{align*}
		\forall n \ge 2, \int_{1}^{n} \ln x~\mathrm{d}x &= \left[ x \ln x \right]_0^n\\
		&= n \ln(n) - n + 1 \\
		&\simi_{n\to +\infty} n \ln n\\
		\int_{2}^{n+1} \ln x~\mathrm{d}x &= (n+1) \ln(n+1) - (n+1) - 2 \ln(2) + 2 \\
		&\simi_{n\to +\infty} (n+1) \ln(n+1)\\
		&\simi_{n\to +\infty}n \ln n
	\end{align*}
	car
	\begin{align*}
		\ln(n+1) &= \ln\left( n \left( 1+ \frac{1}{n} \right) \right) \\
		&= \ln n + \ln\left( 1+\frac{1}{n} \right) \\
		&= \ln n + \frac{1}{n} + \po\left( \frac{1}{n} \right) \\
		&\sim \ln n \\
	\end{align*}

	Donc \[
		\ln(n!)) \simi_{n\to +\infty} n \ln n
	\]
	Cependant, on a un problème : {\color{orange}
	\begin{align*}
		&\ln(n!) = n \ln n + \po(n \ln n)\\
		\text{donc } & n! = n^n \underbrace{e^{\po(n \ln n)}}_{?}
	\end{align*}}

	On pose \[
		\forall n \in \N^*, u_n = \ln(n!) - n\ln n
	\] $(u_n)$ a même nature que $\Sigma(u_{n+1} - u_n)$ et
	\begin{align*}
		\forall n \in \N^*,
		u_{n+1} - u_n &= \ln\left( \frac{(n+1)!}{n!} \right) - (n+1) \ln(n+1) + n \ln n \\
		&= n\big(\ln n - \ln(n+1)\big) \\
		&= n\ln\left( \frac{n}{n+1} \right) \\
		&= n \ln \left( 1 - \frac{1}{n+1} \right) \\
		&\sim -\frac{n}{n+1} \sim -1 < 0
	\end{align*}

	$\Sigma(-1)$ diverge donc $(u_n)$ diverge.

	{\color{red}
		\underline{Conjecture}
		\[
			u_n = \sum_{k=1}^{n-1}(u_{k+1} - u_k) \underbrace{\sim}_{\mathclap{\substack{~\\\downarrow\\\text{On n'a absolument pas le droit !}}}} \sum_{k=1}^{n-1} (-1) = -(n-1) \sim -n
		\]
	}

	On pose \[
		\forall n \in \N^*, v_n = u_n + n
	\] et donc 
	\begin{align*}
		\forall n \in \N^*, v_{n+1} - v_n &= n \ln\left( 1 - \frac{1}{n+1} \right) + 1 \\
		&= n\left( -\frac{1}{n+1} - \frac{1}{2(n+1)^2} + \po\left( \left( \frac{1}{n+1} \right)^2 \right) \right) + 1 \\
		&= n \left( -\frac{1}{n\left( 1+\frac{1}{n} \right)} - \frac{1}{2n^2\left( 1+\frac{1}{n^2} \right)} + \po\left( \frac{1}{n^2} \right) \right) + 1 \\
		&= -\left( \frac{1}{1+\frac{1}{n}} - \frac{1}{2n} \times \frac{1}{\left( 1+\frac{1}{n} \right)^2} + \po\left( \frac{1}{n} \right) \right) \\
		&= -\left( 1 - \frac{1}{n} + \frac{1}{2n} + \po\left( \frac{1}{n} \right) \right) + 1 \\
		&= \frac{1}{2n} + \po\left( \frac{1}{n} \right) \\
		&\sim \frac{1}{2n} > 0.
	\end{align*}

	{\color{red}
		\[
			v_n \sim \sum_{k=1}^{n-1}(v_{k-1} - v_k) \sim \sum_{k=1}^{n-1} \frac{1}{2k} \sim \frac{1}{2} \ln(n)
		\]
	}

	On pose \[
		\forall n \in \N^*, w_n = v_n - \frac{1}{2} \ln n
	\] et donc
	\begin{align*}
		\forall n \in \N^*,
		w_{n+1}- w_n &= n\ln\left( 1+\frac{1}{n+1} \right) - \frac{1}{2}\ln(n+1) + \frac{1}{2} \ln(n) + 1 \\
		&= n\left( -\frac{1}{n+1} - \frac{1}{2(n+1)^2} - \frac{1}{3(n+1)^3} + \po\left( \frac{1}{(n+1)^3} \right) \right)\\
		&\phantom{=}\,+ 1 + \frac{1}{2} \ln\left( 1 - \frac{1}{n+1} \right) \\
		&= -1 - \frac{1}{2(n+1)} - \frac{1}{3(n+1)^2} + \po\left( \frac{1}{(n+1)^2} \right) \\
		&\phantom{=}\,+ \frac{1}{n+1} + \frac{1}{2(n+1)^2} + 1\\
		&\phantom{=}\,+ \frac{1}{2} \left( -\frac{1}{n+1} - \frac{1}{2(n+1)^2} + \po\left( \frac{1}{(n+1)^2} \right) \right)
		&\sim -\frac{1}{12(n+1)^2}\\
		&\sim -\frac{1}{12n^2} < 0
	\end{align*}
	donc $\Sigma(w_{n+1} - w_n)$ converge et donc $(w_n)$ converge.

	On pose $\ell = \lim_{n\to +\infty} w_n$. Ainsi, \[
		\forall n \in \N^*, w_n = \ell + \po(1)
	\] et donc \[
		\forall n \in \N^*, \ln(n!) = n \ln n - n + \frac{1}{2} \ln(n) + \ell + \po(1)
	\] et alors
	\begin{align*}
		\forall n \in \N^*, n! &= n^n e^{-n} \sqrt{n} e^{\ell} \underbrace{e^{\po(1)}}_{\mathclap{\tendsto{n\to +\infty} 1}} \\
		&\sim \left( \frac{n}{e} \right)^n \sqrt{n} \times K
	\end{align*} avec $K = e^{\ell}$.

	On pose \[
		\forall n \in \N^*, I_n = \int_{0}^{\frac{\pi}{2}} \sin^n x~\mathrm{d}x \sim \sqrt{\frac{\pi}{2n}}
	\]et \hfill (c.f. TD5 / Exercice 8)\[
		I_{2n} = \frac{(2n)!}{\left( 2^n n! \right)^2} \times \frac{\pi}{2}.
	\]

	\begin{align*}
		I_{2n} &\sim \frac{\pi}{2} \cancel{\left( \frac{2n}{2e} \right)^{2n}} \sqrt{2n} K \cancel{\left( \frac{e}{n} \right)^{2n}} \frac{1}{n} \times \frac{1}{K^2}\\
		&\sim \frac{\pi}{K\sqrt{2n}}.
	\end{align*}
	Or \[
		I_{2n} \sim \sqrt{\frac{\pi}{4n}}.
	\] Donc \[
		\frac{\sqrt{\frac{\pi}{4n}}}{\frac{\pi}{K\sqrt{2n}}} \tendsto{n\to +\infty} 1
	\] donc \[
		\frac{K}{\sqrt{2\pi}} \tendsto{n\to +\infty} 1
	\] et donc $K = \sqrt{2\pi}$.
\end{prv}

\section{Développement décimal}

\begin{exm}
	\begin{itemize}
		\item Avec $x = 0,54\mathunderline{54}\ldots$, que vaut $2x$ ?
		\item Avec $x = 0,333\mathunderline{3}\ldots$, que vaut $3x$ ?
			\begin{itemize}
				\item $0.999\mathunderline{9}\ldots$ ?
				\item $3 \times \frac{1}{3} = 1$ ?
			\end{itemize}
	\end{itemize}
\end{exm}

\begin{prop}
	Soit $(a_n)_{n \in \N}$ telle que \[
		\begin{cases}
			a_0 \in \Z,\\
			\forall n \ge 1, a_n \in \left\llbracket 0,9 \right\rrbracket
		\end{cases}
	\]

	La série $\sum \frac{a_n}{10^n}$ converge.
\end{prop}

\begin{prv}
	\[
		\forall n \ge 1, 0 \le \frac{a_n}{10^n} \le \frac{9}{10^n}
	\] $\sum \frac{1}{10^n}$ converge car $\frac{1}{10} \in [0, 1[$.
	Donc $\sum_{n\ge 1} \frac{a_n}{10^n}$ converge donc $\sum_{n\ge 1} \frac{a_n}{10^n}$ converge.
\end{prv}

\begin{defn}
	Soit $x \in \R$. On dit que $x$ admet un \underline{développement décimal} si \[
		\exists a_0 \in \Z, (a_n)_{n\ge 1} \in \left\llbracket 0,9 \right\rrbracket^N,
		x = \sum_{n=0}^{+\infty} \frac{a_n}{10^n}.
	\]
	\index{développement décimal}
\end{defn}

\begin{thm}
	Tou réel $x \in [0, 1[$ admet un développement décimal : \[
		x = \sum_{n=1}^{+\infty} \frac{\left\lfloor 10^n x \right\rfloor - 10 \left\lfloor 10^{n-1} x \right\rfloor}{10^n}
	\]
\end{thm}

\begin{prv}
	\begin{align*}
		\forall n \ge 1,\kern 5mm &\phantom{-}10^n x - 1 < \left\lfloor 10^n x \right\rfloor \le  10^n x\\
		&-10^n x + 10 > -10 \left\lfloor 10^{n-1} x \right\rfloor \ge -10^n x
	\end{align*}
	donc \[
		-1 < \left\lfloor 10^n x \right\rfloor - 10 \left\lfloor 10^{n-1} x \right\rfloor < 10
	\] et donc \[
		\left\lfloor 10^n x \right\rfloor - 10 \left\lfloor 10^{n-1} x \right\rfloor \in \left\llbracket 0,9 \right\rrbracket.
	\]

	De plus,
	\begin{align*}
		\sum_{k=1}^n \frac{\left\lfloor 10^k x \right\rfloor - 10 \left\lfloor 10^{k-1}x \right\rfloor }{10^k} &= \sum_{k=1}^n \left( \frac{\left\lfloor 10^k x \right\rfloor}{10^k} - \frac{\left\lfloor 10^{k-1}x \right\rfloor}{10^{k-1}} \right) \\
		&= \frac{\left\lfloor 10^n x \right\rfloor}{10^n} - \underbrace{\left\lfloor x \right\rfloor}_{=0}\\
		&\tendsto{n\to +\infty} x. \\
	\end{align*}
\end{prv}

\begin{thm}
	Soit $x \in ]0, 1[$.

	\begin{enumerate}
		\item Si $x$ n'est pas décimal (i.e. on ne peut pas l'écrire comme $\sfrac{p}{10^n}$ avec $p \in \Z$ et $n \in \N$), alors $x$ a un unique développement décimal.
		\item Si $x$ est décimal, alors $x$ a exactement 2 développements décimaux :
			\begin{itemize}
				\item il y en a un où, à partir d'un certain rang, tous les chiffres sont nuls,
				\item et un autre où tous les chiffres sont égaux à 9 à parir d'un certain rang.
			\end{itemize}
	\end{enumerate}
\end{thm}

\begin{prv}
	Soit $(a_n)_{n\ge 1} \in \left\llbracket 0,9 \right\rrbracket^{\N^*}$ et $(b_n)_{n\ge 1} \in \left\llbracket 0,9 \right\rrbracket^{\N^*}$ telles que \[
		x = \sum_{n=1}^{+\infty} \frac{a_n}{10^n} = \sum_{n=1}^{+\infty} \frac{b_n}{10^n}
	\] On pose $n_0 = \min \{n \in \N^*  \mid a_n \neq b_n\}$ : \[
		\begin{cases}
			\forall n < n_0, a_n = b_n,\\
			a_{n_0} \neq b_{n_0}.
		\end{cases}
	\] Sans perte de généralité, on suppose $a_{n_0} < b_{n_0}$. On a donc
	\begin{align*}
		0 < \frac{b_{n_0} - a_{n_0}}{10^{n_0}} &= \sum_{n = n_0 + 1}^{+\infty} \frac{a_n - b_n}{10^n} \\
	\end{align*}
	\[
		\forall n \ge n_0, \begin{cases}
			0 \le a_n \le 9\\
			0 \le b_n \le 9
		\end{cases}
	\] donc \[
		\forall n \ge n_0, -9 \le a_n - b_n \le 9
	\] donc \[
		-9 \sum_{n=n_0+1}^{+\infty} \frac{1}{10^n} \le \sum_{n=n_0 + 1}^{+\infty} \frac{a_n - b_n}{10^n} \le 9 \sum_{n=1}^{+\infty} \frac{1}{10^n}.
	\]
	Or,
	\begin{align*}
		\sum_{n=n_0 + 1}^{+\infty} \frac{1}{10^n} &= \frac{1}{10^{n_0+1}} \sum_{n=0}^{+\infty} \frac{1}{10^n} \\
		&= \frac{1}{10^{n_0+1}} \times \frac{1}{1-\frac{1}{10}} \\
		&= \frac{1}{9 \times 10^{n_0}} \\
	\end{align*}
	D'où, \[
		0 < \frac{b_{n_0} - a_{n_0}}{10^{n_0}} \le  \frac{1}{10^{n_0}}
	\] donc \[
		0 < \underbrace{b_{n_0} - a_{n_0}}_{\in \Z} \le 1
	\] donc $b_{n_0} - a_{n_0} = 1$ et donc \[
	\sum_{n = n_0 + 1}^{+\infty} \frac{a_n - b_n}{10^n} = \frac{1}{10^{n_0}}
	\] donc \[
		\forall n > n_0, a_n - b_n = 9
	\] et donc \[
		\forall n > n_0, \begin{cases}
			a_n = 9\\
			b_n = 0
		\end{cases}
	\] Comme \[
		\forall n > n_0, b_n = 0
	\] $x$ est décimal et les deux développements de $x$ sont alors
	\begin{align*}
		x &= 0,a_1\ldots a_{n_0-1}a_{n_0}\mathunderline{9}\ldots\\
		&= 0,a_1\ldots a_{n_0-1}(a_{n_0}+1)\mathunderline{0}\ldots \\
	\end{align*}
\end{prv}

\begin{rmk}
	Avec $x = 0,\!54\mathunderline{54}\ldots$, $100x = 54,\!54\mathunderline{54}\ldots = 54 + x$. On a donc $x = \frac{54}{99}$.

	Avec $x = 0,\!987\,123\,\mathunderline{123}\ldots$, on a
	\begin{align*}
		x &= \frac{987}{1000} + 0,\!000\,\mathunderline{123}\ldots\\
		&= \frac{987}{1000} + \frac{1}{10^3}\underbrace{(0,\!\mathunderline{123}\ldots)}_y \\
	\end{align*}
	On a $1000 y = 123 + y$ et donc $y = \frac{123}{999}$ et donc $x = \frac{987 + \frac{123}{999}}{1000}$.
\end{rmk}





	}

	{
		\chap[32]{Familles sommables}
		\renewcommand{\cwd}{../chap32}
		\begin{defn}
	Un \underline{proposition} est un énoncé qui est soit vrai, soit faux.
\end{defn}

\begin{exm}
	\begin{align*}
		\begin{rcases*}
			A: ``B \text{ est vraie }"\\
			B: ``A \text{ est fausse }"\\
		\end{rcases*} \text{ Le système $\{A,B\}$ est une \underline{auto-contradiction}}
	\end{align*}
\end{exm}

\begin{defn}
	\underline{Démontrer} une proposition revient à prouver qu'elle est vraie
\end{defn}

		\begin{defn}
	Soit $E$ un $\mathbbm{K}$-espace vectoriel. On dit que $E$ est de \underline{dimension finie} si $E$ a au moins une famille génératrice finie. On dit que $E$ est de \underline{dimension infinie} sinon.
	\index{dimension finie (espace vectoriel)}
	\index{dimension infinie (espace vectoriel)}
\end{defn}

\begin{thm}
	[Théorème de la base extraite]
	Soit $E$ un $\mathbbm{K}$-espace vectoriel non nul de dimension finie. Soit $\mathcal{G}$ une famille génératrice finie de $E$. Alors, il existe une base $\mathcal{B}$ de $\mathcal{E}$ telle que $\mathcal{B} \subset \mathcal{G}$.
\end{thm}

\begin{prv}
	[par récurrence sur $\#G = \Card(G)$]
	\begin{itemize}
		\item Soit $E$ un $\mathbbm{K}$-espace vectoriel non nul engendré par $\mathcal{G} = (u)$.\\
			Si $u = 0_E$, alors $E = \{0_E\}$: une contradiction $\lightning$ \\
			Donc $u \neq 0_E$ donc $(u)$ est libre. En effet, \[
				\forall \lambda \in \mathbbm{K}, \lambda u = 0_E \implies \lambda = 0_\mathbbm{K}
			\] Donc $\mathcal{G}$ est une base de $E$.\\
		\item Soit $n \in \N_*$. Soit $E$ un $\mathbbm{K}$-espace vectoriel. On suppose que si $E$ a une famille génératrice constituée de $n$ vecteurs, alors on peut extraire de cette famille une base de $E$.\\
			Soit $\mathcal{G}$ une famille génératrice de $E$ avec $n+1$ vecteurs.\\
			Si $\mathcal{G}$ est libre, alors $\mathcal{G}$ est une base de $E$. \\
			Si $\mathcal{G}$ n'est pas libre, alors il existe $u \in \mathcal{G}$ tel que $u \in \Vect(\mathcal{G}\setminus \{u\})$ \\
			Donc $\mathcal{G}\setminus \{u\}$ engendre $E$. Or, $\mathcal{G}\setminus \{u\}$ possède $n$ vecteurs. D'après l'hypothèse de récurrence, il existe une base $\mathcal{B}$ de $E$ telle que \[
				\mathcal{B} \subset \mathcal{G} \setminus \{u\} \subset \mathcal{G}
			\] 
	\end{itemize}
\end{prv}

\begin{crlr}
	Tout espace de dimension finie a une base.
	\qed
\end{crlr}

\begin{thm}
	[Théorème de la base incomplète]
	Soit $E$ un $\mathbbm{K}$-espace vectoriel de dimension finie, $\mathcal{G}$ une famille génératrice finie de $E$. $\mathcal{L}$ une famille libre de $E$. Alors, il existe une base $\mathcal{B}$ de $E$ telle que \[
		\mathcal{L} \subset \mathcal{B} \text{ et } \mathcal{B}\setminus \mathcal{L} \subset \mathcal{G}
	\] 
\end{thm}

\begin{prv}
	[par récurrence sur $\#(\mathcal{G}\setminus\mathcal{L})$]
	\begin{itemize}
		\item Avec les notations précédentes, on suppose que $\mathcal{G}\setminus\mathcal{L} \neq \O$ \[
				\forall u \in \mathcal{G}, u \in \mathcal{L}
			\] Donc $\mathcal{G} \subset \mathcal{L}$ donc $\mathcal{L}$ est génératrice donc $\mathcal{L}$ est une base de $E$. On pose $\mathcal{B} = \mathcal{L}$ et alors \[
				\mathcal{L} \subset  \mathcal{B} \text{ et } \mathcal{B}\setminus\mathcal{L} = \O \subset  \mathcal{G}
			\] 
		\item Soit $n \in \N$. On suppose que si $\mathcal{G}$ est génératrice et $\mathcal{L}$ libre avec $\#(\mathcal{G}\setminus\mathcal{L}) = n$ alors il existe une base $\mathcal{B}$ de $E$ telle que \[
			\mathcal{L}\subset \mathcal{B} \text{ et } \mathcal{B}\setminus\mathcal{L}\subset \mathcal{G}
		\] Soient à présent $\mathcal{G}$ une famille génératrice de $E$ et $\mathcal{L}$ une famille libre de $E$ telles que $\#(\mathcal{G}\setminus\mathcal{L}) = n+1 > 0$\\
		Si $\mathcal{L}$ engendre $E$, alors $\mathcal{L}$ est une base de $E$. On pose $\mathcal{B} = \mathcal{L}$ et on a bien \[
			\mathcal{L} \subset  \mathcal{B} \text{ et } \mathcal{B} \setminus \mathcal{L} = \O \subset  \mathcal{G}
		\] On suppose que $\mathcal{L}$ n'engendre pas $E$. Il existe $u \in \mathcal{G}$ tel que $u \not\in \Vec(\mathcal{L})$ (car sinon, $\mathcal{G} \subset \Vect(\mathcal{L})$ et donc $\underbrace{\Vect(\mathcal{G})}_{= E} \subset  \underbrace{\Vect(\mathcal{L})}_{ \subset E}$\\
		Donc $\mathcal{L} \cup \{u\} $ est libre. On pose $\mathcal{L}' = \mathcal{L} \cup \{u\} $ \[
			\mathcal{G}\setminus \mathcal{L}' = \mathcal{G}\setminus (\mathcal{L} \cup \{u\}) = (\mathcal{G}\setminus\mathcal{L})\setminus \{u\} 
		\] donc $\#(\mathcal{G}\setminus\mathcal{L}') = n+1 -1 = n$\\
		D'après l'hypothèse de récurrence, il existe $\mathcal{B}$ une base de $E$ telle que \[
			\mathcal{L} \subset  \mathcal{L}' \subset \mathcal{B} \text{ et } \mathcal{B}\setminus \mathcal{L}' \subset \mathcal{G}
		\] \[
			\mathcal{B} \setminus \mathcal{L} = \underbrace{\mathcal{B}\setminus\mathcal{L}'}_{\subset \mathcal{G}} \cup \underbrace{\{u\}}_{\subset \mathcal{G} \text{ car } u \in \mathcal{G}}
		\] On a $\mathcal{B}\setminus\mathcal{L}\subset \mathcal{G}$
	\end{itemize}
\end{prv}

\begin{thm}
	Soit $E$ un $\mathbbm{K}$-espace vectoriel de dimension finie. Toutes les bases de $E$ ont le même cardinal.
\end{thm}

\begin{prv}
	Soit $\mathcal{G}$ une famille génératrice finie de $E$ et $\mathcal{B} \subset  \mathcal{G}$ une base de $E$. On note $n = \#\mathcal{B}$ \\
	Soit $\mathcal{B}'$ une base de $E$. On pose $p = n - \#(\mathcal{B} \cap  \mathcal{B}')$. Montrons par récurrence sur  $p$ que $\#\mathcal{B} = \#\mathcal{B}'$ 
	\begin{itemize}
		\item On suppose que $p = 0$. Alors, $\#(\mathcal{B} \cap \mathcal{B}') = n$ \\
			Or, $\mathcal{B}' \cap \mathcal{B} \subset \mathcal{B}$ donc $\mathcal{B} \cap \mathcal{B}' = \mathcal{B}$ donc $\mathcal{B} \subset  \mathcal{B}'$ et donc $\mathcal{B} = \mathcal{B}'$ 
		\item Soit $p \in \N$. On suppose que si $\mathcal{B}'$ est une base de $E$ telle que $n - \#(\mathcal{B} \cap \mathcal{B}') = p$, alors $\#\mathcal{B}' = n$ \\
			Aoit $\mathcal{B}'$ une base de $E$ telle que $n - \#(\mathcal{B}\cap \mathcal{B}') = p+1 > 0$ \\
			Donc $\mathcal{B} \cap \mathcal{B}' \neq \mathcal{B}$. Soit $u \in \mathcal{B}' \setminus \mathcal{B}$. D'après le lemme d'échange, il existe $v \in \mathcal{B}\setminus \mathcal{B}'$ tel que $\mathcal{B}' \setminus \{u\} \cup \{v\}$ est une base de $E$. On pose $\mathcal{B}'' = \mathcal{B}' \setminus \{u\} \cup \{v\}$ 
			\begin{align*}
				\mathcal{B}'' \cap \mathcal{B} &= \left( (\mathcal{B}' \setminus \{u\})  \cap \mathcal{B} \right) \cup \{v\} \\
				&= (\mathcal{B}' \cap \mathcal{B}) \cup \{v\} \\
			\end{align*}
			donc,
			\begin{align*}
				n - \#(\mathcal{B}'' \cap \mathcal{B}) &= n - (\#(\mathcal{B}' \cap \mathcal{B}) + 1) \\
				&= p+1- 1 \\
				&= p \\
			\end{align*}
			D'après l'hypothèse de récurrence, \[
				\#\mathcal{B}'' = n
			\] Or, $\#\mathcal{B}'' = \#\mathcal{B}'$
	\end{itemize}
\end{prv}

\begin{lem}
	Soient $\mathcal{B}$ et $\mathcal{B}'$ deux bases de $E$ telles que $\mathcal{B}\subset \mathcal{B}'$. Alors, $\mathcal{B} = \mathcal{B}'$.
\end{lem}

\begin{prv}
	On suppose $\mathcal{B}' \neq \mathcal{B}$. Soit $u \in \mathcal{B}' \setminus \mathcal{B}$
	$u \in E = \Vect(\mathcal{B})$ donc $\mathcal{B} \cup \{u\}$ n'est pas libre.
	Donc $\mathcal{B}\cup \{u\} \subset \mathcal{B}'$ et $\mathcal{B}'$ est libre donc $\mathcal{B}\cup \{u\}$ est libre: une contradiction $\lightning$
\end{prv}

\begin{lem}
	[Lemme d'échange] Soient $\mathcal{B}_1$ et $\mathcal{B}_2$ deux bases de $E$ et $u \in \mathcal{B}_1 \setminus \mathcal{B}_2$. Alors, il existe $v \in \mathcal{B}_2$ tel que $(\mathcal{B}_1 \setminus \{u\}) \cup \{v\}$ soit une base de $E$.
\end{lem}

\begin{prv}
	[1${}^\text{nde}$ méthode]
	On suppose que pout tout $v \in \mathcal{B}_2$, $(\mathcal{B}_1\setminus \{u\}) \cup \{v\}$ n'est pas une base de $E$
	Soit $v \in \mathcal{B}_2$.
	\begin{itemize}
		\item Supposons $(\mathcal{B}_1\setminus \{u\})\cup \{v\}$ non libre. $\mathcal{B}_1 \setminus \{u\}$ est libre. Donc $v \in \Vect(\mathcal{B}_1 \setminus \{u\})$
		\item Supposons $(\mathcal{B}_1\setminus \{u\}) \cup \{v\}$ non génératrice.
			Comme $\mathcal{B}_1$ engendre $E$, $u \not\in \Vect(\mathcal{B}_1\setminus \{v\})$.
			On suppose que $\mathcal{B}_1 \neq \mathcal{B}_2$.
			$\forall v \in \mathcal{B}_2 \setminus \mathcal{B}_1, \Vect(\mathcal{B}_1 \setminus \{v\}) = \Vect(\mathcal{B}_1) = E \ni u$ 
			donc, $(\mathcal{B}_1\setminus \{u\}) \cup \{v\}$ engendre $E$ et donc \[
				v \in \Vect(\mathcal{B}_1 \setminus \{u\})
			\] On a aussi \[
				\forall v \in \mathcal{B}_1 \setminus \{u\}, v \in \Vect(\mathcal{B}_1\setminus \{u\})
			\] Comme $u \not\in \mathcal{B}_2$, on a \[
				\forall v \in \mathcal{B}_2, v \in \Vect(\mathcal{B}_1\setminus \{u\})
			\] docn \[
				E = \Vect(\mathcal{B}_2) \subset \Vect(\mathcal{B}_1\setminus \{u\})
			\] donc $\mathcal{B}_1\setminus \{u\}$ engendre $E$ donc $\mathcal{B}_1\setminus \{u\}$ est une base de $E$. Or, $\mathcal{B}_1 \setminus \{u\}  \subset  \mathcal{B}_1$, donc $\mathcal{B}_1\setminus \{u\} = \mathcal{B}_1$
	\end{itemize}
\end{prv}

\begin{prv}
	[2${}^\text{nde}$ méthode]
	On suppose que pout tout $v \in \mathcal{B}_2$, $(\mathcal{B}_1\setminus \{u\}) \cup \{v\}$ n'est pas une base de $E$
	\begin{itemize}
		\item Comme $u \in \mathcal{B}_1 \setminus \mathcal{B}_2$, nécéssairement $\mathcal{B}_1 \neq \mathcal{B}_2$ donc $\mathcal{B}_2 \not\subset \mathcal{B}_1$, donc $\mathcal{B}_2\setminus\mathcal{B}_1 \neq \O$ 
		\item Soit $v \in \mathcal{B}_2\setminus\mathcal{B}_1$. Il existe $(\lambda_w)_{w\in\mathcal{B}_1}$ une famille de scalaires presque nulle telle que \[
				v = \sum_{w \in \mathcal{B}_1} \lambda_w w - \lambda_u u + + \sum_{w \in \mathcal{B}_1\setminus \{u\}}\lambda_w w
			\]
			Si $\lambda_u \neq 0_E$, alors
			\begin{align*}
				u &= \lambda_u^{-1}\left( v - \sum_{w \in \mathcal{B}_1 \setminus \{u\}} \lambda_w w \right)\\
					&\in \Vect(\mathcal{B}_1\setminus \{u\} \cup v)
			\end{align*}
			 donc $\mathcal{B}_1 \subset \Vect(\mathcal{B}_1\setminus \{u\} \cup \{v\})$\\
			 et donc $E \subset  \Vect(\mathcal{B}_1 \setminus \{u\} \cup \{v\})$ \\
			 et donc $\mathcal{B}_1 \setminus \{u\} \cup \{v\}$ engendre $E$ \\
			 donc $\mathcal{B}_1 \setminus \{u\} \cup \{v\}$ n'est pas libre\\
			 donc $v \in \Vect(\mathcal{B}_1\setminus \{u\})$ (car $\mathcal{B}_1 \setminus \{u\}$ est libre\\
			 donc $\lambda_u = 0_\mathbbm{K}$ $\lightning$\\`

			 Donc, $\lambda_u = 0_\mathbbm{K}$, docn $v \in \Vect(\mathcal{B}_1\setminus \{u\})$ \\
			 On vient de prouver que
			 \begin{align*}
			 	\mathcal{B}_2 \setminus \mathcal{B}_1 \subset \Vect(\mathcal{B}_1 \setminus \{u\})\\
			 	\mathcal{B}_1 \setminus \{u\} \subset \Vect(\mathcal{B}_1 \setminus \{u\})\\
			 \end{align*}
			 Comme $u \not\in \mathcal{B}_2$, \[
			 	\mathcal{B}_2 \subset \Vect(\mathcal{B}_1 \setminus \{u\})
			 \] donc \[
			 	E = \Vect(\mathcal{B}_2) \subset  \Vect(\mathcal{B}_1 \setminus \{u\})
			 \] donc $\mathcal{B}_1 \setminus \{u\}$ engendre $E$. Donc,  $\mathcal{B}_1 \setminus \{u\}$ est une base de $E$.\\
			 Or, $\mathcal{B}_1 \setminus \{u\} \subset  \mathcal{B}_1$, donc $\mathcal{B}_1 \setminus \{u\} = \mathcal{B}_1$
	\end{itemize}
\end{prv}

\begin{defn}
	Soit $E$ un $\mathbbm{K}$-espace vectoriel de dimension finie. Le cardinal commun à toutes les bases de $E$ est appelé \underline{dimension} de $E$ est notée $\dim(E)$ ou $\dim_\mathbbm{K}(E)$\\
	C'est donc aussi le nombre de coordonnées de n'importe quel vecteur dans n'importe quelle base.
	\index{dimension (espace vectoriel)}
\end{defn}

\begin{exm}
	\begin{enumerate}
		\item $\dim_\R(\C) = 2$ et $\dim_\C(\C) = 1$ 
		\item $\dim_\mathbbm{K}(\mathbbm{K}^{n}) = n$ 
		\item $\dim_{\mathbbm{K}}(\mathcal{M}_{n,p}(\mathbbm{K})) = np$
	\end{enumerate}
\end{exm}

\begin{crlr}
	Soit $E$ un $\mathbbm{K}$-espace vectoriel de dimension finie, $\mathcal{L}$ une famille libre de $E$, $\mathcal{G}$ une famille génératrice de $E$. On note $n = \dim(E)$
	\begin{enumerate}
		\item $\#\mathcal{G} \ge n$ et $(\#\mathcal{G} = n \implies \mathcal{G} \text{ est une base de } E$)
		\item $\#\mathcal{L} \le n$ et $(\#\mathcal{L} = n \implies \mathcal{L} \text{ est une base de } E$)
	\end{enumerate}
\end{crlr}

\begin{crlr}
	$\R^{\R}$ est de dimension infinie.
	$\forall i \in \N, e_i: x \mapsto x^i$\\
	$(e_i)_{i\in\N}$ est libre dans $\R^\R$
\end{crlr}

\begin{prop}
	Soient $E$ et $F$ deux $\mathbbm{K}$-espaces vectoriels de dimension finie. Alors $E\times F$ est de dimension finie et $\dim(E\times F) = \dim(E) + \dim(F)$
\end{prop}

\begin{prv}
	Soit $(e_1,\ldots, e_n)$ une base de $E$, $(f_1, \ldots, f_p)$ une base de $F$.
	On pose \[
		\left\{\begin{array}
			{r c l}
			u_1 &=& (e_1,0_F)\\
			u_2 &=& (e_2,0_F)\\
					&\vdots&\\
			u_n &=& (e_n,0_F)\\
			u_{n+1} &=& (0_E, f_1)\\
			u_{n+2} &=& (0_E, f_2)\\
					&\vdots&\\
			u_{n+p} &=& (0_E,f_p)\\
		\end{array}\right.
	\]
	Soit $(x,y) \in E\times F$. \[
		\begin{cases}
			\exists (x_1,\ldots,x_n)\in \mathbbm{K}^n, x = \sum_{i=1}^{n} x_ie_i
			\exists (y_1,\ldots,y_n)\in \mathbbm{K}^n, x = \sum_{j=1}^{p} y_jf_j
		\end{cases}
	\] 
	\begin{align*}
		(x,y) &= \left( \sum_{i=1}^{n} x_ie_i, \sum_{i=1}^{p} y_jf_j \right)  \\
		&= \sum_{i=1}^{n} x_i (e_i + 0_F) + \sum_{j=1}^{p} y_j (0_E, f_j) \\
		&= \sum_{i=1}^{n} x_i u_i + \sum_{j=1}^{p} y_j u_{n+j} \\
	\end{align*}
	Donc, $E\times F = \Vect(u_1, \ldots, u_{n+p})$ donc $E\times F$ est de dimension finie.\\
	Soit $(\lambda_1, \ldots, \lambda_{n+p}) \in \mathbbm{K}^{n+p}$ tel que \[
		(*): \quad \sum_{k=1}^{n+p} \lambda_ku_k = 0_{E\times F} = (0_E, 0_F)
	\]
	\begin{align*}
		(*) &\iff \sum_{k=1}^{n} \lambda_k (e_k, 0_F) + \sum_{k=n+1}^{p} \lambda_k(0_E, f_{k-n}) = (0_E, 0_F)\\
				&\iff \begin{cases}
					\sum_{k=1}^{n} \lambda_k e_k = 0_E\\
					\sum_{k=n+1}^{p} \lambda_k f_{k-n} = 0_F
				\end{cases}\\
				&\iff \begin{cases}
					\forall k \in \left\llbracket 1,n \right\rrbracket, \lambda_k = 0_\mathbbm{K} \qquad&(\text{car $(e_1,\ldots,e_n)$ est libre})\\
					\forall k \in \left\llbracket n+1,n+p \right\rrbracket, \lambda_k = 0_\mathbbm{K} \qquad&(\text{car $(f_1,\ldots,f_n)$ est libre})\\
				\end{cases}
	\end{align*}
	Donc $(u_1, \ldots, u_{n+p})$ est une base de $E\times F$. Donc, $\dim(E\times F) = n + p = \dim(E) + \dim(F)$
\end{prv}

\begin{rmk}
	[Convention]
	\[\dim\big(\{0_E\}\big) = 0\]
\end{rmk}

\begin{thm}
	Soit $E$ un $\mathbbm{K}$-espace vectoriel de dimension finie, $F$ un sous-espace vectoriel de $E$. Alors, $F$ est de dimension finie et  $\dim(F) \le \dim(E)$\\
	Si $\dim(F) = \dim(E)$, alors $F = E$
\end{thm}

\begin{prv}
	On considère \[
		A = \{k \in \N \mid \text{il existe une famille libre de $F$ à $k$ éléments}\} 
	\]
	On suppose $F \neq \{0_E\}$.
	\begin{itemize}
		\item Soit $u \in F\setminus \{0_E\}$. $(u)$ est libre donc $1 \in A$ et donc $A \neq \O$
		\item Soit $\mathcal{L}$ une famille libre de $F$. Alors, $\mathcal{L}$ est une famille libre de $E$ \\
			donc $\#\mathcal{L} \le \dim(E)$\\
			Donc $A$ est majorée par $\dim(E)$ \\
			On en déduit que $A$ a un plus grand élément $p$.
		\item Soit $\mathcal{L}$ une famille libre de $F$ avec $p$ éléments.\\
			Si $\mathcal{L}$ n'engendre pas $F$, alors il existe $u\in F$ tel que $u\not\in \Vect(\mathcal{L})$ et donc $\mathcal{L} \cup \{u\}$ est une famille libre de $F$, donc $p+1 \in A$ en contradiction avec la maximalité de $p$.\\
			Donc $\mathcal{L}$ est une base de $F$ donc $F$ est de dimension finie et $\dim(F) = p \le \dim(E)$\\
	\end{itemize}

	Soit $\mathcal{B}$ une base de $F$. Alors, $\mathcal{B}$ est aussi une famille de libre de de $E$. Donc $\#\mathcal{B} \le \dim(E)$ donc $\dim(F) = \dim(E)$ \\
	Si $\dim(F) = \dim(E)$, alors $\mathcal{B}$ est une base de $E$, et donc $F = \Vect(\mathcal{B}) = E$
\end{prv}

\begin{prop}
	[Formule de Grassmann]
	Soit $E$ un $\mathbbm{K}$-espace vectoriel de dimension finie, $F$ et $G$ deux sous-espace vectoriels de $E$. Alors, \[
		\dim(F+G) = \dim(F) + \dim(G) - \dim(F\cap G)
	\] 
\end{prop}

\begin{prv}
	Soit $(e_1, \ldots, e_p)$ une base de $F\cap G$. $(e_1,\ldots,e_p)$ est une famille libre de $F$.\\
	On complète $(e_1, \ldots, e_p)$ en une base $(e_1, \ldots, e_p, u_1, \ldots, u_q)$ de $F$.\\
	De même, on complète $(e_1, \ldots, e_p)$ en une base $(e_1, \ldots, e_p, v_1, \ldots, v_r)$ de $G$.\\
	On pose  $\mathcal{B} = (e_1, \ldots, e_p, u_1, \ldots, u_q, v_1, \ldots, v_r)$. Montrons que $\mathcal{B}$ est une base de $F+G$
	\begin{itemize}
		\item Soit $u \in F+G$ \\
			On pose $u = v+w$ avec $\begin{cases}
				v\in F\\
				w \in G
			\end{cases}$.\\
			On pose $v = \sum_{i=1}^p \lambda_i e_i + \sum_{i=1}^q \mu_i u_i$ avec $(\lambda_1, \ldots, \lambda_p, \mu_1, \ldots, \lambda_q) \in \mathbbm{K}^{p+q}$\\
			On pose aussi $w = \sum_{i = 1}^p \lambda'_ie_i + \sum_{j=1}^r \nu_j v_j$ avec $(\lambda_1',\ldots,\lambda_p', \nu_1, \ldots, \nu_r) \in \mathbbm{K}^{p+r}$\\
			D'où, \[
				u = \sum_{i=1}^p (\lambda_i + \lambda'_i)e_i + \sum_{j=1}^q \mu_j u_j + \sum_{k=1}^r \nu_k v_k \in \Vect(\mathcal{B})
			\]
		\item Soient $(\lambda_1, \ldots, \lambda_p, \mu_1, \ldots, \mu_q, \nu_1, \ldots, \nu_r) \in \mathbbm{K}^{p+q+r}$.\\
			On suppose \[
				(*)\quad \sum_{i=1}^{p}\lambda_ie_i + \sum_{j=1}^q\mu_ju_j + \sum_{k=1}^r \nu_k v_k = 0_E
			\] 
			D'où, \[
				\underbrace{\sum_{i=1}^p\lambda_i e_i + \sum_{j=1}^q \mu_ju_j}_{\in F} = \underbrace{-\sum_{k=1}^r\nu_jv_k}_{\in G}
			\] 
			Donc, \[
				f = \sum_{i=1}^p \lambda_i e_i + \sum_{j=1}^q \mu_j u_j \in F\cap G
			\] Comme $(e_1, \ldots, e_p)$ est une base de $F\cap G$, $\exists ! (\lambda_1', \ldots, \lambda_p') \in \mathbbm{K}^p$ tel que \[
				f = \sum_{i=1}^p \lambda'_i e_i = \sum_{i=1}^p \lambda'_i e_i + \sum_{j=1}^q 0_\mathbbm{K}u_j
			\] Comme $(e_1, \ldots, e_p, u_1, \ldots, u_q)$ est une base de $F$, \[
				\forall k \in \left\llbracket 1, q \right\rrbracket, \mu_j = 0_\mathbbm{K}
			\] De même, \[
				\forall k \in \left\llbracket 1,r \right\rrbracket , \nu_k = 0_\mathbbm{K}
			\] On remplace dans $(*)$ pour trouver \[
				\sum_{i=1}^p \lambda_ie_i = 0_E
			\] Comme $(e_1, \ldots, e_p)$ est libre, \[
				\forall i \in \left\llbracket 1,p \right\rrbracket, \lambda_i = 0_\mathbbm{K}
			\] Donc $\mathcal{B}$ est libre.\\
			Donc, 
			\begin{align*}
				\dim(F+G) &=  p +q + r \\
				&= (p+q)+ (p+r) - p \\
				&= \dim(F) + \dim(G) - \dim(F\cap G) \\
			\end{align*}
	\end{itemize}
\end{prv}

\begin{crlr}
	Avec les hypothèse précédentes, \[
		E = F \oplus G \iff \begin{cases}
			F \cap  G = \{0_E\} \\
			\dim(E) = \dim(F) + \dim(G)
		\end{cases}
	\] 
\end{crlr}

\begin{prv}
	\begin{itemize}
		\item[``$\implies$''] On suppose $E = F \oplus G$ \\
			Comme la somme est directe, $F \cap G = \{0_E\}$ 
			\begin{align*}
				\dim(E) &= \dim(F)\\
				&= \dim(F) + \dim(G) - \dim(F\cap G)\\
				&= \dim(F) + \dim(G)\\
			\end{align*}
		\item[``$\impliedby$''] On suppose $F\cap G = \{0_E\}$ et $\dim(E) = \dim(F) + \dim(G)$.\\
			On sait déjà que $F+G = F \oplus G$\\
			 \begin{align*}
				\dim(F+G) = \dim(F) + \dim(G) - \dim(F \cap G) = \dim(E)
			\end{align*}
			Donc $F + G = E$
	\end{itemize}
\end{prv}

\begin{prop}
	Soit $F$ un $\mathbbm{K}$-espace vectoriel de dimension finie $n$. Soit $\mathcal{B} = (e_1, \ldots, e_n)$ une base de $F$. L'application
	\begin{align*}
		f: \mathbbm{K}^n &\longrightarrow F \\
		(\lambda_1, \ldots, \lambda_n) &\longmapsto \sum_{i=1}^n \lambda_i e_i
	\end{align*} est bijective.\\
	Si $\mathbbm{K}$ est infini, $\mathbbm{K}^n$ aussi et donc $F$ aussi.\\
	Si $\#\mathbbm{K} = p \in \N_*$,
	\begin{align*}
		\#&\mathbbm{K}^n = p^n\\
		&\vrt=\\
		\#&F
	\end{align*}
\end{prop}


		\part{Dérivation}

\underline{Motivation}:

{
\begin{wrapfigure}{l}{3cm}
	\centering
	\begin{asy}
		import three;

		size(3cm);
		settings.render=0;
		settings.prc=false;
		currentprojection = obliqueZ;

		draw(unitbox);
		draw(shift(1.1Z + 0.05X) * (O -- X), Arrows3(TeXHead2));
		draw(shift(1.1Z + 0.05Y) * (O -- Y), Arrows3(TeXHead2));
		draw(shift(1.1X + 0.05Z) * (O -- Z), Arrows3(TeXHead2));

		label("$x$", (X/2) + (1.1Z + 0.05X), align=S);
		label("$y$", (Y/2) + (1.1Z + 0.05Y), align=W);
		label("$z$", (Z/2) + X, align=SE);
	\end{asy}
\end{wrapfigure}

\begin{align*}
	&S(x,y,z) = 2(xy + xz + yz)\\
	&V(x,y,z) = xyz
\end{align*}

On cherche à minimiser $S$ avec la contrainte $V = 1$.

Soit $f : \begin{array}{rcl}
	\left( \R_*^+ \right)^2 &\longrightarrow& \R \\
	(x,y) &\longmapsto& S\left( x,y,\frac{1}{xy} \right) = 2\left( xy + \frac{1}{y} + \frac{1}{x} \right).
\end{array}$

On cherche $(a,b) \in \left( \R^+_* \right)^2$ tel que \[
	\forall (x,y) \in (\R^+_*), f(x,y) \ge f(a,b).
\]
}

\begin{defn}
	Soit $f: U \to \R$ où $U$ est un ouvert de $\R^2$. Soit $(a,b) \in U$.
	\vspace{2mm}

	Si $\lim_{x \to a} \frac{f(x,b) - f(a,b)}{x - a} \in \R$, alors on dit que $f$ a une dérivée partielle suivant $x$ en $(a,b)$ et cette limite est notée \[
		\partial f_1(a,b) = \frac{\partial f}{\partial x}(a,b).
	\]

	Si $\lim_{y \to b} \frac{f(a,y) - f(a,b)}{y - b} \in \R$, alors on dit que $f$ a une dérivée partielle suivant $y$ et la limite est notée \[
		\partial f_2(a,b) = \frac{\partial f}{\partial y}(a,b).
	\]
\end{defn}

\begin{exm}
	\begin{enumerate}
		\item $f: (x,y) \mapsto xy + x - y$.

			\begin{align*}
				&\frac{\partial f}{\partial x} : (x,y) \mapsto y + 1,\\
				&\frac{\partial f}{\partial y} : (x,y) \mapsto x - 1.
			\end{align*}

		\item $f: (x,y) \mapsto xy + \frac{1}{y}+ \frac{1}{x}$.

			\begin{align*}
				&\frac{\partial f}{\partial x}: (x,y) \mapsto y - \frac{1}{x^2},\\
				&\frac{\partial f}{\partial y}: (x,y) \mapsto x - \frac{1}{y^2}.
			\end{align*}

		\item Trouver $f$ telle que $\begin{cases}
				(1): \qquad \frac{\partial f}{\partial x}=y,\\[2mm]
				(2): \qquad \frac{\partial f}{\partial y} = x.
			\end{cases}$

			D'après $(1)$ : \[
				\forall (x,y), \exists C(y) \in \R, f(x,y) = xy + C(y)
			\] et donc \[
				\frac{\partial f}{\partial y}(x,y) = x + C'(y)
			\] donc $C'(y) = 0$ et donc $C$ est constante.

		\item Trouver $f$ telle que $\begin{cases}
			\frac{\partial f}{\partial x} = -y,\\[2mm]
			\frac{\partial f}{ƒ\partial y} = x.
		\end{cases}$

		Ce n'est pas possible !
	\end{enumerate}
\end{exm}

\begin{defn}~\\
	\begin{minipage}{\linewidth}
		\begin{wrapfigure}{r}{4cm}
			\centering
			\vspace{-5mm}
			\begin{asy}
				import three;
				import graph3;
				size(4cm);

				settings.render = 0;
				settings.prc = false;
				currentprojection = obliqueX;

				draw(O -- X, Arrow3(TeXHead2));
				draw(O -- Y, Arrow3(TeXHead2));
				draw(O -- Z, Arrow3(TeXHead2));

				triple f(real x, real y, real z = 0) { return (x,y,cos(x - 0.5) * cos(y - 0.5)/1.2 + 0.15); }

				real inc = 1 / 5;

				for(real x = 0; x <= 1; x += inc) {
					draw(graph(
						new real(real t) { return x; }, // x
						new real(real y) { return y; }, // y
						new real(real y) { return f(x,y).z; }, // z
						0, 1
					), gray);
				}

				for(real y = 0; y <= 1; y += inc) {
					draw(graph(
						new real(real x) { return x; }, // x
						new real(real t) { return y; }, // y
						new real(real x) { return f(x,y).z; }, // z
						0, 1
					), gray);
				}

				path3 path1 = (0.8, 0.2, 0) .. (0.5, 0.5, 0) .. (0.3, 0.7, 0);
				path3 path2 = f(0.8, 0.2, 0) .. f(0.5, 0.5, 0) .. f(0.3, 0.7, 0);
				path3 d = (0.2, 0.3, 0) .. (0.3, 0.4, 0) .. (0.2, 0.7, 0) .. (0.8, 0.9, 0) .. (0.6, 0.2, 0) .. cycle;

				draw(path1, red, Arrow3(TeXHead2));
				draw(path2, red, Arrow3(TeXHead2, position=0.8));

				dot((0.5, 0.5, 0));
				dot(f(0.5, 0.5, 0));
				draw((0.5, 0.5, 0) -- f(0.5, 0.5, 0), dashed);
				draw(d);

				label("$w$", (0.3, 0.7, 0), red, align=SE);
				label("$U$", (0.8, 0.9, 0), align=SE);
			\end{asy}
		\end{wrapfigure}

		Soit $f: U \to \R$ où $U$ est un ouvert. Soit $(a,b) \in U$. Soit $w = (w_1, w_2) \in \R^2$.

		Si 
		\[
			\lim_{t\to 0} \frac{f(a + tw_1, b + tw_2) - f(a,b)}{t}
		\] existe et est réelle, alors on dit que $f$ a une dérivée dans la direction de $w$ et la limite est notée \[
			\mathrm{d}f(w)\,(a,b) = D_w(f)\,(a,b).
		\]
	\end{minipage}
\end{defn}

\begin{exm}
	\begin{align*}
		f: \left( \R_*^+ \right)^2 &\longrightarrow \R \\
		(x,y) &\longmapsto xy+\frac{1}{x}+\frac{1}{y}.
	\end{align*}

	On pose $(a,b) = (1,2)$, $w = (w_1, w_2) = (1,1)$.
	\begin{align*}
		\frac{f(1+t, 2+t) - f(1,2)}{t} &= \frac{1}{t} \left( (1+t)(2+t) + \frac{1}{1+t} + \frac{1}{2+t} - 3 - \frac{1}{2} \right) \\
		&= \frac{1}{t}\left(\cancel 2 + 3t + \po(t) + \cancel 1 - t + \po(t) + \frac{1}{2}\left( \cancel 1 - \frac{t}{2} + \po(t) \right) - \cancel3 - \cancel{\frac{1}{2}} \right) \\
		&= \frac{1}{t} \left( \frac{7}{4} t + \po(t) \right)  \\
		&= \frac{7}{4} + \po(1) \tendsto{t \to 0} \frac{7}{4}. \\
	\end{align*}

	Donc, \[
		\mathrm{d}f(1,1)\,(1,2) = \frac{7}{4}.
	\]
\end{exm}

\begin{rmk}~\\
	\begin{figure}[H]
		\centering
		\begin{asy}
			import solids;
			import graph;
			size(5cm);

			settings.render = 0;
			settings.prc = false;

			path3 par = graph(
				new real(real x) { return x; },
				new real(real x) { return 0; },
				new real(real x) { return x^2; },
				0,3);
			revolution r = revolution(par, axis=Z);

			path3 par2 = graph(
				new real(real x) { return x; },
				new real(real x) { return 0; },
				new real(real x) { return x^2; },
				-3,3);

			draw(r,1,longitudinalpen=nullpen);
			draw(r.silhouette());

			draw((-4, 0, -1) -- (-4, 0, 10) -- (4, 0, 10) -- (4, 0, -1) -- cycle, red);
			draw(par2, deepred);

			draw((4,4.5) -- (7, 4.5), black+0.5mm, Arrow(TeXHead));

			path par2d = graph(new real(real x) { return x^2; }, -3, 3);
			draw(shift((11, 0)) * par2d, deepred);

			dot(O);
			dot((11, 0));
		\end{asy}
	\end{figure}
\end{rmk}


%todo ajouter théorème-définition
\begin{thm}
	Soit $f : U \to \R$, $(a,b) \in U$. On suppose que $\frac{\partial f}{\partial x}$ et $\frac{\partial f}{\partial y}$ existent en $(a,b)$ et sont {\bfseries continues} en $(a,b)$. Alors,
	\begin{align*}
		&\forall (h, k) \in \R^2 \text{ tel que } (a +h, b + k) \in U,\\
		&f(a+ h, b + k) = f(a,b) + h \frac{\partial f}{\partial x}(a,b) + k \frac{\partial f}{\partial y}(a,b) + \po_{(h,k)\to (0,0)}\big(\|(h,k)\|\big).
	\end{align*}

	On dit que $f$ est de classe $\mathcal{C}^1$ si $\frac{\partial f}{\partial x}$ et $\frac{\partial f}{\partial y}$ existent et sont continues.

	\qed
\end{thm}

\begin{rmk}
	En physique, cette formule correspond à : \[
		\mathrm{d}f = \frac{\partial f}{\partial x}\mathrm{d}x + \frac{\partial f}{\partial y} \mathrm{d}y.
	\] En effet :
	\begin{align*}
		\mathrm{d}f &= f(x+ \mathrm{d}x, y + \mathrm{d}y) - f(x,y) \\
		&= \frac{\partial f}{\partial x} \mathrm{d}x + \frac{\partial f}{\partial y} \mathrm{d}y.
	\end{align*}
\end{rmk}

\begin{prop}
	Soit $f: U \to \R$ de classe $\mathcal{C}^1$ en $(a,b) \in U$. Alors, \[
		\forall w = (w_1, w_2) \in \R^2, \mathrm{d}f(w)\,(a,b) = w_1 \frac{\partial f}{\partial x}(a,b) + w_2 \frac{\partial f}{\partial y}(a,b).
	\]
\end{prop}

\begin{prv}
	Soit $w = (w_1, w_2) \in \R^2$. Soit $t \in \R^*$.
	\begin{align*}
		\frac{1}{t}\big(f(a + tw_1, b + tw_2) - f(a,b)\big)
		&= \frac{1}{t} \left( tw_1 \frac{\partial f}{\partial x}(a,b) + tw_2 \frac{\partial f}{\partial y}(a,b) + \po_{t \to 0}\big(\|tw\|\big) \right) \\
		&= w_1 \frac{\partial f}{\partial x}(a,b) + w_2 \frac{\partial f}{\partial y}(a,b) + \po_{t\to 0}(1) \\
		&\tendsto{t\to 0} w_1 \frac{\partial f}{\partial x}(a,b) + w_2\frac{\partial f}{\partial y}(a,b).
	\end{align*}
\end{prv}


\begin{defn}
	Avec les hypothèses précédentes, en posant \[
		\nabla f(a,b) = \left( \frac{\partial f}{\partial x}(a,b), \frac{\partial f}{\partial y}(a,b) \right) 
	\]on obtient \[
		\mathrm{d}f(w)\,(a,b) = \left<w  \mid \nabla f(a,b) \right>
	\] où $\left<\cdot|\cdot \right>$ est le produit scalaire.

	Le vecteur $\nabla f(a,b)$ est appelé \underline{gradient de $f$ en $(a,b)$}.

	Le développement limité à l'ordre 1 de $f$ devient \[
		f\big((a,b)+w\big) = f(a,b) + \left<w \mid \nabla f(a,b) \right> + \po_{w\to 0}(\|w\|)
	\]
\end{defn}

\begin{prop}
	Soit $f : U \to \R$ de classe $\mathcal{C}^1$.

	\begin{figure}[H]
    \centering
    \incfig{gradient}
	\end{figure}

	$\nabla f$ est orthogonal au lignes de niveaux de $f$, son orientation va dans le sens d'une augmentation de $f$.
\end{prop}

\begin{prv}
	Soit $\gamma : I \to U$ une courbe de niveau : \[
		\forall t \in I, f\big(\gamma(t)\big) = \text{cste}.
	\] D'après le lemme suivant : \[
		\forall t \in I, 0 = (f \circ \gamma)'(t) = \mathrm{d}f\big(\gamma'(t)\big)\big(\gamma(t)\big) = \left<\gamma'(t)  \mid \nabla f\big(\gamma(t)\big) \right>
	\] Donc $\nabla f\big(\gamma(t)\big)$ est orthogonal à $\gamma'(t)$.

	Pour tout $t \in I$, on pose $w(t) = t\, \nabla f\big(\gamma(t)\big)$. Donc \[
		f\big(\gamma(t) + w(t)\big) = f\big(\gamma(t)\big) + t \|\nabla f(\gamma(t))\|^2 + \po_{t \to 0}(t)
	\] Pour $t$ assez petit, $f\big(\gamma(t) + w(t)\big) - f\big(\gamma(t)\big)$ est du même signe que $t$.
\end{prv}

\begin{rmk}
	\begin{align*}
		V: \R^3 &\longrightarrow \R \\
		(x,y,z) &\longmapsto -mgz
	\end{align*}
	l'énerge potentielle de pesenteur

	On a donc \[
		\nabla V(x,y,z) = \left( \frac{\partial V}{\partial x}, \frac{\partial V}{\partial y}, \frac{\partial V}{\partial z} \right) = (0, 0, -mg) = \vec{P}.
	\]
\end{rmk}

\begin{lem}
	Soit $f : U \to \R$ de classe $\mathcal{C}^1$, $\gamma : \begin{array}{rcl}
		I &\longrightarrow& U \\
		t &\longmapsto& \big(x(t), y(t)\big)
	\end{array}$ où $x$ et $y$ sont dérivables.

	On pose \[
		\forall t \in I, \gamma'(t) = \big(x'(t), y'(t)\big).
	\] Alors $f \circ \gamma : I \to \R$ est dérivable et
	\begin{align*}
		\forall t \in I, (f \circ \gamma)'(t) &= \mathrm{d}f\big(\gamma'(t)\big) \big(\gamma(t)\big)\\
		&= \left<\gamma'(t)  \mid \nabla f\big(\gamma(t)\big)  \right> \\
		&= x'(t) \frac{\partial f}{\partial x}\big(x(t), y(t)\big) + y'(t) \frac{\partial f}{\partial y}\big(x(t),y(t)\big). \\
	\end{align*}
\end{lem}

\begin{prv}
	On fixe $t \in I$.

	\begin{align*}
		\forall h \neq 0, \frac{f \circ \gamma(t + h) - f \circ \gamma(t)}{h}
		&= \frac{1}{h}\big(f(\gamma(t)) + h\gamma'(t) + \po_{h\to 0}(h) - f(\gamma(t))\big) \\
		&= \frac{1}{h}\bigg(\cancel{f(\gamma(t))} + \left<h\gamma'(t) \mid \nabla f(\gamma(t)) \right> + \po_{h\to 0}(\|h\gamma'(t)\|) - \cancel{f(\gamma(t))}\bigg)\\
		&= \left<\gamma'(t) \mid \nabla f(\gamma(t)) \right> + \po_{h\to 0}(1) \\
		&\tendsto{h\to 0} \left<\gamma'(t)  \mid \nabla f(\gamma(t)) \right>
	\end{align*}
\end{prv}

\begin{defn}
	Soit $f : U \to \R$ de classe $\mathcal{C}^1$ et $(a,b) \in U$. On dit que $(a,b)$ est un \underline{point critique} de $f$ si $\nabla f(a,b) = 0$ i.e. $\frac{\partial f}{\partial x}(a,b) = \frac{\partial f}{\partial y}(a,b) = 0$.

	Dans ce cas, $f(a,b)$ est appelé \underline{valeur critique} de $f$.
\end{defn}

\begin{prop}~\\
	\begin{minipage}{\linewidth}
		\begin{wrapfigure}{r}{3cm}
			\centering
			\vspace{-1cm}
			\begin{asy}
				import solids;
				import graph;
				size(3cm);

				settings.render = 0;
				settings.prc = false;

				path3 par = graph(
					new real(real x) { return x; },
					new real(real x) { return 0; },
					new real(real x) { return -x^2; },
					0,3);
				revolution r = revolution(par, axis=Z);

				draw(r,1,longitudinalpen=nullpen);
				draw(r.silhouette());

				dot("$(a,b)$", O, red, align=N);
				real s = sqrt(2.5);
				path3 g=(s,0,-2.5)..(0,s,-2.5)..(-s,0,-2.5)..(0,-s,-2.5)..cycle;
				draw(g, deepcyan);
			\end{asy}
		\end{wrapfigure}
		Soit $f: U \to \R$ de classe $\mathcal{C}^1$ et $(a,b) \in U$ tel que \[
			\exists r > 0, \forall (x,y) \in B_{(a,b)}(r), f(x,y) \le f(a,b)
		\] Alors $\nabla f(a,b) = (0,0)$.
	\end{minipage}
\end{prop}

\begin{prv}
	Soit $g: x \mapsto f(x,b)$. $g(a)$ est un maximum local de $g$ donc $g'(a) = 0$.

	Or, $g'(a) = \frac{\partial f}{\partial x}(a,b)$

	donc $\frac{\partial f}{\partial x}(a,b) = 0$.

	Soit $h : y \mapsto f(a,y)$. On a de même $h'(b) = 0$.

	Or, $h'(b) = \frac{\partial f}{\partial y}(a,b)$.

	Donc, $\nabla f(a,b) = (0,0)$.
\end{prv}

\begin{rmk}
	Un minimum local est aussi une valeur critique.
\end{rmk}

\begin{figure}[H]
	\centering
	\begin{subfigure}{3cm}
		\centering
		\begin{asy}
			import solids;
			import graph;
			size(3cm);

			settings.render = 0;
			settings.prc = false;

			path3 par = graph(
				new real(real x) { return x; },
				new real(real x) { return 0; },
				new real(real x) { return -x^2; },
				0,3);
			revolution r = revolution(par, axis=Z);

			draw(r,1,longitudinalpen=nullpen);
			draw(r.silhouette());

			dot(O, red);
		\end{asy}
		\caption{Maximum local}
	\end{subfigure}
	\begin{subfigure}{3cm}
		\centering
		\begin{asy}
			import solids;
			import graph;
			size(3cm);

			settings.render = 0;
			settings.prc = false;

			path3 par = graph(
				new real(real x) { return x; },
				new real(real x) { return 0; },
				new real(real x) { return x^2; },
				0,3);
			revolution r = revolution(par, axis=Z);

			draw(r,1,longitudinalpen=nullpen);
			draw(r.silhouette());

			dot(O, red);
		\end{asy}
		\caption{Minimum local}
	\end{subfigure}
	\begin{subfigure}{3cm}
		\centering
		\begin{asy}
			import solids;
			import graph;
			size(3cm);

			settings.render = 0;
			settings.prc = false;
			currentprojection = obliqueZ;

			draw(graph(
				new real(real x) { return x; },
				new real(real x) { return -x^2 / 3; },
				new real(real x) { return 3; },
				-3, 3
			));

			draw(graph(
				new real(real x) { return x; },
				new real(real x) { return -x^2 / 3; },
				new real(real x) { return -3; },
				-3, 3
			));

			draw(graph(
				new real(real x) { return x; },
				new real(real x) { return -x^2 / 3 - 1; },
				new real(real x) { return 0; },
				-3, 3
			));

			draw(graph(
				new real(real x) { return 0; },
				new real(real x) { return x^2 / 9 - 1; },
				new real(real x) { return x; },
				-3, 3
			));

			draw(graph(
				new real(real x) { return -3; },
				new real(real x) { return x^2 / 9 - 4; },
				new real(real x) { return x; },
				-3, 3
			));

			draw(graph(
				new real(real x) { return 3; },
				new real(real x) { return x^2 / 9 - 4; },
				new real(real x) { return x; },
				-3, 3
			));

			dot((0,-1,0), red);
		\end{asy}
		\caption{Point de selle / Point col}
	\end{subfigure}
\end{figure}

\begin{exm}
	On revient à l'exemple donné en introduction : 
	\begin{align*}
		f: \left( \R^*_+ \right)^2 &\longrightarrow \R \\
		(x,y) &\longmapsto 2\left( xy + \frac{1}{x} + \frac{1}{y} \right).
	\end{align*}

	$\left( \R^+_* \right)^2$ est un ouvert de $\R^2$. Soit $(x,y) \in \left( \R^+_* \right)^2$.
	
	On a \[
		\begin{cases}
			\frac{\partial f}{\partial x}(x,y) = 2\left( y - \frac{1}{x^2} \right),\\
			\frac{\partial f}{\partial y}(x,y) = 2\left( x - \frac{1}{y^2} \right).
		\end{cases}
	\]

	\begin{align*}
		&\frac{\partial f}{\partial x}(x,y) = \frac{\partial f}{\partial y}(x,y) = 0\\
		\iff& \begin{cases}
			y = \frac{1}{x^2}\\
			x = \frac{1}{y^2}
		\end{cases}\\
		\iff& \begin{cases}
			y = \frac{1}{x^2}\\
			x = x^4
		\end{cases}\\
		\iff& \begin{cases}
			x = 1\\
			y = 1
		\end{cases}
	\end{align*}

	On vérivie que $f$ présente en effet un minium local en $(1,1)$. \[
		f(1,1) = 6
	\] On fixe $y \in \R^+_*$ et \[
		g : x \mapsto 2\left( xy + \frac{1}{x} + \frac{1}{y} \right).
	\] Donc \[
		\forall x \in \R^+_*, g'(x) = 2\left( y - \frac{1}{x^2} \right).
	\]
	\begin{center}
		\begin{tikzpicture}
			\tkzTabInit{$x$/1,$g'(x)$/1,$g$/2.3}{$0$, $\frac{1}{\sqrt{y}}$, $+\infty$}
			\tkzTabLine{,-,z,+,}
			\tkzTabVar{+/{}, -/$2\left( 2\sqrt{y} +\frac{1}{y} \right)$, +/{}}
		\end{tikzpicture}
	\end{center}
	
	Ainsi, \[
		\forall x \in \R^+_*, \forall y \in \R^+_*, f(x,y) \ge 2\left( 2\sqrt{y} + \frac{1}{y} \right)
	\] Soit $h : y \mapsto 2\sqrt{y} + \frac{1}{y}$. On a \[
		\forall y > 0, h'(y) = \frac{1}{\sqrt{y}} - \frac{1}{y^2} = \frac{y\sqrt{y} - 1}{y^2} = \frac{y^{\frac{3}{2}} - 1}{y^2}
	\]

	\begin{center}
		\begin{tikzpicture}
			\tkzTabInit{$y$/0.7,$h'(y)$/0.7,$h$/1.4}{$0$, $1$, $+\infty$}
			\tkzTabLine{,-,z,+,}
			\tkzTabVar{+/{}, -/$3$, +/{}}
		\end{tikzpicture}
	\end{center}

	Donc, \[
		\forall x,y > 0, f(x,y) \ge 2\times 3 = 6 = f(1,1).
	\]
\end{exm}

\begin{prop}
	[règle de la chaîne]

	Soit $f : \begin{array}{rcl}
		U &\longrightarrow& \R^2 \\
		(x,y) &\longmapsto& f(x,y)
	\end{array}$ de classe $\mathcal{C}^1$ et $U, V$ deux ouverts de $\R^2$.

	Soit $\varphi : \begin{array}{rcl}
		V &\longrightarrow& U \\
		(u,v) &\longmapsto& \varphi(u,v) = \big(x(u,v), y(u,v)\big)
	\end{array}$.

	On suppose que $x$ et $y$ sont de classe $\mathcal{C}^1$ sur $V$.

	Alors,  $f \circ \varphi : \begin{array}{rcl}
		V &\longrightarrow& \R \\
		(u,v) &\longmapsto& f\big(\varphi(u,v)\big)
	\end{array}$ est de classe $\mathcal{C}^1$ et
	\begin{align*}
		\forall (u_0, v_0) \in V, \frac{\partial (f \circ \varphi)}{\partial u}(u_0, v_0)
		&= \frac{\partial f}{\partial x}\big(\varphi(u_0, v_0)\big) \times \frac{\partial x}{\partial u}(u_0, v_0)\\
		&+ \frac{\partial f}{\partial y}\big(\varphi(u_0,v_0)\big) \frac{\partial y}{\partial u}(u_0,v_0)
	\end{align*}
	\begin{align*}
		\forall (u_0, v_0) \in V, \frac{\partial (f \circ \varphi)}{\partial v}(u_0, v_0)
		&= \frac{\partial f}{\partial x}\big(\varphi(u_0, v_0)\big) \times \frac{\partial x}{\partial v}(u_0, v_0)\\
		&+ \frac{\partial f}{\partial y}\big(\varphi(u_0,v_0)\big) \frac{\partial y}{\partial v}(u_0,v_0)
	\end{align*}
\end{prop}

\begin{exm}
	[changement de coordonnées polaires]
	On pose \begin{align*}
		\varphi: \R^+_* \times ]0,2\pi[ &\longrightarrow \R^2\setminus \left( R^+_* \times \{0\} \right) \\
		(r, \theta) &\longmapsto (r \cos \theta, r \sin\theta),
	\end{align*}
	\begin{align*}
		f: \R^2\setminus \left( R^+_* \times \{0\} \right) &\longrightarrow \R \\
		(x,y) &\longmapsto f(x,y),
	\end{align*}
	\begin{align*}
		g: \overbrace{\R^+_* \times ]0, 2\pi[}^{=V} &\longrightarrow \R \\
		(r, \theta) &\longmapsto f(r\cos\theta, r\sin\theta).
	\end{align*}

	\begin{align*}
		\forall (r_0,\theta_0) \in V,&\\[5mm]
		\frac{\partial g}{\partial r}(r_0, \theta_0) &= \frac{\partial f}{\partial x}(r_0\cos\theta_0, r_0\sin\theta_0)\cos\theta_0\\
		&+ \frac{\partial f}{\partial y}(r_0 \cos\theta_0, r_0\sin\theta_0)\sin\theta_0\\
		&= 2r_0\cos^2\theta_0 + 2r_0\sin^2(\theta_0) \\
		&= 2r_0 \\[5mm]
		\frac{\partial g}{\partial \theta}(r_0, \theta_0) &= \frac{\partial f}{\partial x}(r_0\cos\theta_0, r_0\sin\theta_0)r_0\sin\theta_0\\
		&+ \frac{\partial f}{\partial y}(r_0 \cos\theta_0, r_0\sin\theta_0)r_0\cos\theta_0\\
		&= -2{r_0}^2\cos(\theta_0)\sin(\theta_0) + 2{r_0}^2 \sin(\theta_0)\cos(\theta_0)\\
		&= 0 \\
	\end{align*}

	Donc, \[
		g(r, \theta) = r^2.
	\]
\end{exm}

\begin{exm}
	Résoudre \[
		\begin{cases}
			\frac{\partial f}{\partial x} = \frac{x}{x^2+y^2},\\
			\frac{\partial f}{\partial y} = \frac{y}{x^2+y^2}.\\
		\end{cases}
	\]

	On pose $g: (r, \theta) \mapsto f(r \cos\theta, r \sin\theta)$.

	\begin{align*}
		&\frac{\partial g}{\partial r} = \frac{1}{r}\cos^2\theta + \frac{1}{r}\sin^2\theta = \frac{1}{r},\\
		&\frac{\partial g}{\partial \theta} = -\cos(\theta) \sin(\theta) + \sin(\theta)\cos(\theta) = 0.
	\end{align*}

	Donc, \[
		\exists C \in \R, g: (r, \theta) \mapsto \ln r + C
	\] d'où,
	\begin{align*}
		\forall (x,y) \in \R^2 \setminus \{(0,0)\}, f(x,y) &= \ln\left(\sqrt{x^2 + y^2} \right)  + C\\
		&= \frac{1}{2}\ln(x^2 + y^2) + C. \\
	\end{align*}
\end{exm}

\begin{rmk}
	Soit $\mathcal{B} = (e_1, e_2)$ la base canonique de $\R^2$, $f: U \to \R$ de classe $\mathcal{C}^1$ avec $U$ un ouvert de $\R^2$.

	Soit $(x,y) \in U$.

	\begin{align*}
		\Mat_{\mathcal{B}}\big(\nabla f(x,y)\big) = \begin{pmatrix}
			\frac{\partial f}{\partial x}(x,y)\\[2mm]
			\frac{\partial f}{\partial y}(x,y)
		\end{pmatrix}
	\end{align*}

	Soit  \begin{align*}
		\varphi: V &\longrightarrow U \\
		(u,v) &\longmapsto \big(x(u,v), y(u,v)\big) 
	\end{align*} avec $x,y$ de classe $\mathcal{C}^1$. Soit $g = f \circ \varphi$.
	\begin{align*}
		\Mat_{\mathcal{B}}\big(\nabla g(u,v)\big)
		&= \begin{pmatrix}
			\frac{\partial g}{\partial u}(u,v) \\[2mm]
			\frac{\partial g}{\partial v}(u,v)
		\end{pmatrix} \\
		&= \begin{pmatrix}
			\frac{\partial x}{\partial u}(u,v) \frac{\partial f}{\partial x}(x,y)
			+ \frac{\partial y}{\partial u}(u,v)\frac{\partial f}{\partial y}(x,y)\\[3mm]
			\frac{\partial x}{\partial v}(u,v) \frac{\partial f}{\partial x}(x,y)
			+ \frac{\partial y}{\partial v}(u,v) \frac{\partial f}{\partial y}(x,y)
		\end{pmatrix}  \\
		&= \underbrace{\begin{pmatrix}
				\frac{\partial x}{\partial u}(u,v)& \frac{\partial y}{\partial u}(u,v)\\[3mm]
				\frac{\partial x}{\partial v}(u,v)& \frac{\partial y}{\partial v}(u,v)
		\end{pmatrix}}_{J(u,v)} \begin{pmatrix}
			\frac{\partial f}{\partial x}(x,y)\\[3mm]
			\frac{\partial f}{\partial y}(x,y)
		\end{pmatrix} \\
		&= J(u,v) \Mat_{\mathcal{B}}\big(\nabla f(x,y)\big) \\
	\end{align*}
	où $J(u,v) = 
	\begin{pNiceArray}{c:c}
		\Mat_{\mathcal{B}}\big(\nabla x(u,v)\big) & \Mat_{\mathcal{B}}\big(\nabla y(u,v)\big)
	\end{pNiceArray}$.

	On dit que $J(u,v)$ est \underline{la jacobienne} de $\varphi$ en $(u,v)$.
	L'application linéaire canoniquement associée à $J(u,v)$ est la \underline{différentielle de $\varphi$} en $(u,v)$ noté $\mathrm{d}\varphi(u,v)$.

	On a $\mathrm{d}\varphi(u,v) \in \mathcal{L}(R^2)$ et $\Mat_{\mathcal{B}}\big(\mathrm{d}\varphi(u,v)\big) = J(u,v)$.

	Par exemple, la jacobienne du changement de coordonnées polaires est \[
		J = \begin{pmatrix}
			\frac{\partial x}{\partial r} & \frac{\partial y}{\partial r}\\[3mm]
			\frac{\partial x}{\partial \theta} & \frac{\partial y}{\partial \theta}
		\end{pmatrix}
		= \begin{pmatrix}
			\cos\theta&\sin\theta\\
			-r\sin\theta&r\cos\theta
		\end{pmatrix}.
	\]
	$\underbrace{\det(J)}_{\text{le jacobien}} = r\cos^2\theta + r\sin^2\theta = r$

	Dans une intégrale double, si $(x,y) = \varphi(u,v)$, alors $\mathrm{d}x\mathrm{d}y = \det(J)\mathrm{d}u\mathrm{d}v$.

	Ici, \[
		\mathrm{d}x\ \mathrm{d}y = r\ \mathrm{d}r\ \mathrm{d}\theta.
	\]
\end{rmk}

\begin{prv}
	On pose $(x_0, y_0) = \varphi(u_0, v_0)$. Pour tout $(h,k) \in \R^2$ tels que $(u_0 + h, v_0 + k) \in V$, en posant $g = f  \circ \varphi$.

	\begin{align*}
		g(u_0 + h, v_0 + h) &= f\big(x(u_0 + h, v_0 + k), y(u_0 + h, v_0 + k)\big) \\
		&= f\left(
			x(u_0,v_0) + h \frac{\partial x}{\partial u}(u_0,v_0) + k \frac{\partial x}{\partial v}(u_0, v_0) + \po\big(\|(h,k)\|\big), \right.\\
		&\phantom{ = f\bigg(\bigg.}\left. y(u_0, v_0) + h \frac{\partial y}{\partial u}(u_0, v_0) + k \frac{\partial y}{\partial v}(u_0, v_0) + \po\big(\|(h,k)\|\big)
		\right)  \\
		&= f(x_0,y_0) \\
		&~+ \left( h \frac{\partial x}{\partial u}(u_0,v_0) + k \frac{\partial x}{\partial v}(u_0, v_0) + \po(\|(h,k)\|) \right) \frac{\partial f}{\partial x}(x_0,y_0)\\
		&~+ \left( h \frac{\partial y}{\partial u}(u_0, v_0) + k\frac{\partial y}{\partial v}(u_0, v_0) + \po(\|(h,k)\|) \right) \frac{\partial f}{\partial y}(x_0, y_0)\\
		&~+ \po(\|(h,k)\|)\\
		&= f(x_0, y_0) \\
		&~+ h \left( \frac{\partial x}{\partial u}(u_0, v_0) \frac{\partial f}{\partial x}(x_0, y_0) + \frac{\partial y}{\partial u}(u_0, v_0) \frac{\partial f}{\partial y}(x_0, y_0) \right)  \\
		&~+ k\left( \frac{\partial x}{\partial v}(u_0, v_0) \frac{\partial f}{\partial x}(x_0, y_0) + \frac{\partial y}{\partial v}(u_0, v_0) \frac{\partial f}{\partial y}(x_0, y_0) \right) 
		&~+ \po(\|(h,k)\|)\\
		&= g(u_0, v_0) + h \frac{\partial g}{\partial u}(u_0, v_0) + k \frac{\partial g}{\partial v}(u_0, v_0) + \po(\|(h,k)\|) \\
	\end{align*}

	Par identification,
	\[
		\frac{\partial g}{\partial u}(u_0, v_0) = \frac{\partial x}{\partial u}(u_0, v_0) \frac{\partial f}{\partial x}(x_0, y_0) + \frac{\partial y}{\partial u}(u_0, v_0) \frac{\partial f}{\partial y}(x_0,y_0)
	\] et \[
		\frac{\partial g}{\partial v}(u_0, v_0) = \frac{\partial x}{\partial v}(u_0,v_0) \frac{\partial f}{\partial x}(x_0, y_0) + \frac{\partial y}{\partial v}(u_0, v_0) \frac{\partial f}{\partial y}(x_0, y_0).
	\] 
\end{prv}

\begin{exm}
	[Régression linéaire]~\\
	\begin{figure}[H]
		\centering
		\begin{asy}
			import graph;
			axes(EndArrow);
			size(5cm);

			real f(real x) { return x + 0.5; }

			real k = 35 / (7 - 0.5);

			for(int i = 0; i < 35; ++i) {
				real mag = exp(sin(100 * pi/exp(1) * i)) * 0.8 + exp(cos(i*40)/3);
				real eps = mag * cos(10 * exp(1)/pi * i) / 3;
				dot((i/k,f(i/k) + eps));
			}

			draw(graph(f, -1, 7), orange);
		\end{asy}
	\end{figure}
	\[
		y = a x + b
	\] 
	On fixe $(a,b) \in \R^2$. \[
		\varepsilon(a,b) = \sum_{i=1}^n\big( y_i - (ax_i + b) \big)^2
	\] l'erreur totale.

	On veut minimiser $\varepsilon(a,b)$. On a 
	\[
		\forall (a,b) \in \R^2,
		\begin{cases}
			\frac{\partial \varepsilon}{\partial a}(a,b) = -2\sum_{i=1}^{n}(y_i - ax_i - b)x_i,\\
			\frac{\partial \varepsilon}{\partial b}(a,b) = -2\sum_{i=1}^{n}(y_i - ax_i - b).
		\end{cases}
	\]

	Donc,
	\begin{align*}
		(a,b) \text{ point critique de } \varepsilon \iff& \begin{cases}
			a \sum_{i=1}^n {x_i}^2 + b\sum_{i=1}^{n}x_i = \sum_{i=1}^{n} y_ix_i\\
			a\sum_{i=1}^{n}x_i + nb = \sum_{i=1}^ny_i
		\end{cases}\\
		\iff& \begin{cases}
			a \left( \frac{1}{n}\sum_{i=1}^n {x_i}^2 - \overline{x}^2\right) = \overline{y} - \overline{x} \overline{y}\\
			b = \frac{1}{n}\sum_{i=1}^ny_i - \frac{a}{n}\sum_{i=1}^nx_i = \frac{1}{n}\sum_{i=1}^n x_i y_i - \overline{x} \overline{y}
		\end{cases}\\
		&\text{ où } \overline{x} = \frac{1}{n} \sum_{i=1}^n x_i,~\overline{y} = \frac{1}{n}\sum_{i=1}^n y_i\\
		\iff& \begin{cases}
			a = \frac{\Cov(x,y)}{V(x)}\\
			b = \overline{y} - a\overline{x}
		\end{cases}
	\end{align*}

	Coefficient de corrélation: $\frac{\Cov(x,y)}{\sigma_x \sigma_y} \in [-1, 1]$
\end{exm}












		\part{Corps}

\begin{exm}[Problème]
	\begin{itemize}
		\item 
			avec $A = \Z / 9 \Z$, résoudre $\overline{x}^2 = \overline{0}$ \\
			\begin{center}
				\begin{tabular}{|c|c|c|c|c|c|c|c|c|c|c|}
					\hline
					$\overline{x}$&$\overline{0}$& $\overline{1}$ &$\overline{2}$&$\overline{3}$ &$\overline{4}$ &$\overline{5}$ &$\overline{6}$ &$\overline{7}$ &$\overline{8}$& $\overline{9}$ \\
					\hline
					$\overline{x}^2$&$\overline{0}$ &$\overline{1}$ &$\overline{4}$ &$\overline{0}$ &$\overline{7}$ &$7$ &$\overline{0}$ &$\overline{4}$ &$\overline{1}$&$\overline{0}$\\
					\hline
				\end{tabular}
			\end{center}
			On a trouvé 3 solutions: $\overline{0}$, $\overline{3}$, $\overline{6}$.
		\item $\Z / 8\Z$
			\begin{center}
				\begin{tabular}{|c|c|c|c|c|c|c|c|c|}
					\hline
					$\overline{x}$& $\overline{0}$& $\overline{1}$& $\overline{2}$& $\overline{3}$& $\overline{4}$& $\overline{5}$& $\overline{6}$& $\overline{7}$\\
					\hline
					$\overline{x^2}$& $\overline{0}$& $\overline{1}$& $\overline{4}$& $\overline{1}$& $\overline{0}$& $\overline{1}$& $\overline{4}$& $\overline{1}$\\
					\hline
				\end{tabular}
			\end{center}
			$\overline{x}^2=7$ a 4 solutions: $\overline{1}, \overline{7}, \overline{3},\text{ et } \overline{5}$
		\item $A = \mathbbm{H} = \{a + bi + cj + dk  \mid  (a,b,c,d) \in \R^4\}$ \\
			$i^2 = j^2 = k^2 = -1$ 
			\begin{align*}
				\begin{array}{c c c}
					ij = k & jk = i & ji = j\\
					ji = -k & kj = -i & ik = -j
				\end{array}
			\end{align*}
			Dans cet anneau, $-1$ a 6 racines!
	\end{itemize}
\end{exm}

\begin{defn}
	Soit $(\mathbbm{K}, +, \times)$ un ensemble muni de deux lois de composition internes. On dit que c'est un \underline{corps} si
	 \begin{enumerate}
		\item $(\mathbbm{K}, \times)$ est un groupe abélien
		\item $(\mathbbm{K}, \times)$ est un monoïde commutatif
		\item $\forall x \in \mathbbm{K}\setminus \{0_\mathbbm{K}\}, \exists y \in \mathbbm{K}, xy = 1_\mathbbm{K}$
		\item $0_\mathbbm{K} \neq  1_\mathbbm{K}$
	\end{enumerate}
	\index{corps}
\end{defn}

\begin{exm}
	\begin{itemize}
		\item $(\C, +, \times)$ est un corps
		\item $(\R, +, \times)$ est un corps
		\item $(\Q, +, \times)$ est un corps
		\item $(\Z, +, \times)$ n'est pas un corps
	\end{itemize}
\end{exm}

\begin{prop}
	$(\Z / n\Z, +, \times)$ est un corps si et seulement si $n$ est premier.
\end{prop}

\begin{prv}
	\[
		\left( \Z / n\Z \right)^\times = \left\{ \overline{k}  \mid k \wedge n = 1 \right\}
	\] 
\end{prv}


\begin{prop}
	Tout corps est un anneau intègre.
\end{prop}

\begin{prv}
	Soit $(\mathbbm{K}, +, \times)$ un corps. Soient $(a,b) \in \mathbbm{K}^2$ tel que $a \times b = 0_\mathbbm{K}$.\\
	On suppose $a \neq  0_\mathbbm{K}$. Alors, $a$ est inversible et donc \[
		b = a^{-1} \times a \times b = a^{-1} \times 0_\mathbbm{K} = 0_\mathbbm{K}
	\] 
\end{prv}

\begin{exm}
	Soit $(\mathbbm{K},+,\times)$ un corps.\\
	Résoudre \[
		\begin{cases}
			x^2 = 1_\mathbbm{K}\\
			x \in \mathbbm{K}
		\end{cases}
	\]

	\begin{align*}
		x^2 = 1_\mathbbm{K} &\iff x^2 - 1_\mathbbm{K} = 0_\mathbbm{K}\\
		&\iff (x - 1_\mathbbm{K})(x+1_\mathbbm{K}) = 0_\mathbbm{K}\\
		&\iff x - 1_\mathbbm{K} = 0_\mathbbm{K} \text{ ou } x + 1_\mathbbm{K} = 0_\mathbbm{K}\\
		&\iff x = 1_\mathbbm{K} \text{ ou } x = -1_\mathbbm{K}
	\end{align*}

	Il y a au plus 2 solutions.
\end{exm}

\begin{prop}
	Soit $(\mathbbm{K},+,\times )$ un corps et $P$ un polynôme à coefficients dans $\mathbbm{K}$ de degré $n$. Alors, l'équation $P(x) = 0_{\mathbbm{K}}$ a au plus $n$ solutions dans $\mathbbm{K}$ 
	\qed
\end{prop}

\begin{crlr}[(Théorème de Wilson)]
	voir exercice 16 du TD 12
\end{crlr}


\begin{defn}
	Soit $(\mathbbm{K}, +, \times)$ un corps et $L\subset \mathbbm{K}$.\\
	On dit que $L$ est un \underline{sous corps} de $\mathbbm{K}$ si
	\begin{enumerate}
		\item $L$ est un anneau de $(\mathbbm{K}, +, \times)$ non nul
		\item $\forall x \in L\setminus \{0_\mathbbm{K}\}, x^{-1} \in L$ 
	\end{enumerate}
	\vspace{2mm}
	en d'autres termes si
	\begin{enumerate}
		\item $\forall (x,y) \in L^2, x - y \in L$
		\item $\forall (x,y) \in L^2, x \times y^{-1} \in L$
	\end{enumerate}
	\vspace{5mm}
	On dit aussi que $\mathbbm{K}$ est une \underline{extension} de $L$.
	\index{sous corps}
	\index{extension}
\end{defn}

\begin{prop}
	Tout sous corps est un corps. \qed
\end{prop}

\begin{defn}
	Soient $(\mathbbm{K}_1,+,\times )$ et $(\mathbbm{K}_2,+, \times)$ deux corps et $f: \mathbbm{K}_1 \to \mathbbm{K}_2$.\\
	On dit que $f$ est un \underline{morphisme de corps} si $f$ est un morphisme d'anneaux.\\
	i.e. si
	\[
		\begin{cases}
			\forall (x,y) \in {\mathbbm{K}_1}^2,& f(x+y) = f(x) + f(y)\\
			\forall (x,y) \in {\mathbbm{K}_1}^2,& f(x \times y) = f(x) \times f(y)\\
		\end{cases}
	\] 
	\index{homomorphisme (de corps)}
	\index{morphisme (de corps)}
\end{defn}

\begin{prop}
	Tout morphisme de corps est injectif.
\end{prop}

\begin{prv}
	Soit $f: \mathbbm{K}_1 \to \mathbbm{K}_2$ un morphisme de corps.\\
	\begin{itemize}
		\item $\Ker(f)$ est un sous groupe de $(\mathbbm{K}_1, +)$ 
		\item Soit $x \in \Ker(f)$ et $y \in \mathbbm{K}_1$ \[
				f(x \times y) = f(x) \times f(y) = 0_{\mathbbm{K}_2} \times f(y) = 0_{\mathbbm{K}_2}
			\]
		\item Soit $x \in \Ker(f) \setminus \{0_{\mathbbm{K}_1}\}$.\\
			Alors, $x$ est inversible.\\
			\begin{align*}
				\begin{rcases*}
					x \in \Ker(f)\\
					x^{-1} \in \mathbbm{K}_1
				\end{rcases*}& \text{ donc } x \times x ^{-1} \in \Ker(f)\\
				&\text{ donc } 1_{\mathbbm{K}_1} \in \Ker(f)\\
				&\text{ donc } f(1_{\mathbbm{K}_1}) = 0_{\mathbbm{K}_2}
			\end{align*}
			Or, $f(1_{\mathbbm{K}_1}) = 1_{\mathbbm{K}_2} \neq 0_{\mathbbm{K}_2}$
	\end{itemize}
	Donc, $\Ker(f) = \{0_{\mathbbm{K}_1}\}$ donc $f$ est injective.
\end{prv}

\begin{exm}
	$\begin{array}{cc}
		\C &\longrightarrow \C\\
		z &\longmapsto \overline{z}\\
	\end{array}$ est un morphisme de corps
\end{exm}



		\part{Opérations sur les séries}

\begin{prop}
	L'ensemble $E = \{u \in \C^\N  \mid \Sigma u_n \text{ converge}\}$ est un sous-espace vectoriel de $\C^\N$ et \begin{align*}
		S: E &\longrightarrow \C \\
		u &\longmapsto \sum_{n=0}^{+\infty} u_n
	\end{align*} est une forme linéaire.
	\qed
\end{prop}

\begin{rmk}
	La somme d'une série convergente et d'une série divergente diverge.
	Le produit d'une série divergente par un scalaire non nul diverge.
\end{rmk}

	}

	{
		\chapter*{Annexe}
		\renewcommand*\parttitle{Annexe}
		\renewcommand{\cwd}{../chap33}
		\begin{defn}
	Soit $E$ un $\mathbbm{K}$-espace vectoriel. On dit que $E$ est de \underline{dimension finie} si $E$ a au moins une famille génératrice finie. On dit que $E$ est de \underline{dimension infinie} sinon.
	\index{dimension finie (espace vectoriel)}
	\index{dimension infinie (espace vectoriel)}
\end{defn}

\begin{thm}
	[Théorème de la base extraite]
	Soit $E$ un $\mathbbm{K}$-espace vectoriel non nul de dimension finie. Soit $\mathcal{G}$ une famille génératrice finie de $E$. Alors, il existe une base $\mathcal{B}$ de $\mathcal{E}$ telle que $\mathcal{B} \subset \mathcal{G}$.
\end{thm}

\begin{prv}
	[par récurrence sur $\#G = \Card(G)$]
	\begin{itemize}
		\item Soit $E$ un $\mathbbm{K}$-espace vectoriel non nul engendré par $\mathcal{G} = (u)$.\\
			Si $u = 0_E$, alors $E = \{0_E\}$: une contradiction $\lightning$ \\
			Donc $u \neq 0_E$ donc $(u)$ est libre. En effet, \[
				\forall \lambda \in \mathbbm{K}, \lambda u = 0_E \implies \lambda = 0_\mathbbm{K}
			\] Donc $\mathcal{G}$ est une base de $E$.\\
		\item Soit $n \in \N_*$. Soit $E$ un $\mathbbm{K}$-espace vectoriel. On suppose que si $E$ a une famille génératrice constituée de $n$ vecteurs, alors on peut extraire de cette famille une base de $E$.\\
			Soit $\mathcal{G}$ une famille génératrice de $E$ avec $n+1$ vecteurs.\\
			Si $\mathcal{G}$ est libre, alors $\mathcal{G}$ est une base de $E$. \\
			Si $\mathcal{G}$ n'est pas libre, alors il existe $u \in \mathcal{G}$ tel que $u \in \Vect(\mathcal{G}\setminus \{u\})$ \\
			Donc $\mathcal{G}\setminus \{u\}$ engendre $E$. Or, $\mathcal{G}\setminus \{u\}$ possède $n$ vecteurs. D'après l'hypothèse de récurrence, il existe une base $\mathcal{B}$ de $E$ telle que \[
				\mathcal{B} \subset \mathcal{G} \setminus \{u\} \subset \mathcal{G}
			\] 
	\end{itemize}
\end{prv}

\begin{crlr}
	Tout espace de dimension finie a une base.
	\qed
\end{crlr}

\begin{thm}
	[Théorème de la base incomplète]
	Soit $E$ un $\mathbbm{K}$-espace vectoriel de dimension finie, $\mathcal{G}$ une famille génératrice finie de $E$. $\mathcal{L}$ une famille libre de $E$. Alors, il existe une base $\mathcal{B}$ de $E$ telle que \[
		\mathcal{L} \subset \mathcal{B} \text{ et } \mathcal{B}\setminus \mathcal{L} \subset \mathcal{G}
	\] 
\end{thm}

\begin{prv}
	[par récurrence sur $\#(\mathcal{G}\setminus\mathcal{L})$]
	\begin{itemize}
		\item Avec les notations précédentes, on suppose que $\mathcal{G}\setminus\mathcal{L} \neq \O$ \[
				\forall u \in \mathcal{G}, u \in \mathcal{L}
			\] Donc $\mathcal{G} \subset \mathcal{L}$ donc $\mathcal{L}$ est génératrice donc $\mathcal{L}$ est une base de $E$. On pose $\mathcal{B} = \mathcal{L}$ et alors \[
				\mathcal{L} \subset  \mathcal{B} \text{ et } \mathcal{B}\setminus\mathcal{L} = \O \subset  \mathcal{G}
			\] 
		\item Soit $n \in \N$. On suppose que si $\mathcal{G}$ est génératrice et $\mathcal{L}$ libre avec $\#(\mathcal{G}\setminus\mathcal{L}) = n$ alors il existe une base $\mathcal{B}$ de $E$ telle que \[
			\mathcal{L}\subset \mathcal{B} \text{ et } \mathcal{B}\setminus\mathcal{L}\subset \mathcal{G}
		\] Soient à présent $\mathcal{G}$ une famille génératrice de $E$ et $\mathcal{L}$ une famille libre de $E$ telles que $\#(\mathcal{G}\setminus\mathcal{L}) = n+1 > 0$\\
		Si $\mathcal{L}$ engendre $E$, alors $\mathcal{L}$ est une base de $E$. On pose $\mathcal{B} = \mathcal{L}$ et on a bien \[
			\mathcal{L} \subset  \mathcal{B} \text{ et } \mathcal{B} \setminus \mathcal{L} = \O \subset  \mathcal{G}
		\] On suppose que $\mathcal{L}$ n'engendre pas $E$. Il existe $u \in \mathcal{G}$ tel que $u \not\in \Vec(\mathcal{L})$ (car sinon, $\mathcal{G} \subset \Vect(\mathcal{L})$ et donc $\underbrace{\Vect(\mathcal{G})}_{= E} \subset  \underbrace{\Vect(\mathcal{L})}_{ \subset E}$\\
		Donc $\mathcal{L} \cup \{u\} $ est libre. On pose $\mathcal{L}' = \mathcal{L} \cup \{u\} $ \[
			\mathcal{G}\setminus \mathcal{L}' = \mathcal{G}\setminus (\mathcal{L} \cup \{u\}) = (\mathcal{G}\setminus\mathcal{L})\setminus \{u\} 
		\] donc $\#(\mathcal{G}\setminus\mathcal{L}') = n+1 -1 = n$\\
		D'après l'hypothèse de récurrence, il existe $\mathcal{B}$ une base de $E$ telle que \[
			\mathcal{L} \subset  \mathcal{L}' \subset \mathcal{B} \text{ et } \mathcal{B}\setminus \mathcal{L}' \subset \mathcal{G}
		\] \[
			\mathcal{B} \setminus \mathcal{L} = \underbrace{\mathcal{B}\setminus\mathcal{L}'}_{\subset \mathcal{G}} \cup \underbrace{\{u\}}_{\subset \mathcal{G} \text{ car } u \in \mathcal{G}}
		\] On a $\mathcal{B}\setminus\mathcal{L}\subset \mathcal{G}$
	\end{itemize}
\end{prv}

\begin{thm}
	Soit $E$ un $\mathbbm{K}$-espace vectoriel de dimension finie. Toutes les bases de $E$ ont le même cardinal.
\end{thm}

\begin{prv}
	Soit $\mathcal{G}$ une famille génératrice finie de $E$ et $\mathcal{B} \subset  \mathcal{G}$ une base de $E$. On note $n = \#\mathcal{B}$ \\
	Soit $\mathcal{B}'$ une base de $E$. On pose $p = n - \#(\mathcal{B} \cap  \mathcal{B}')$. Montrons par récurrence sur  $p$ que $\#\mathcal{B} = \#\mathcal{B}'$ 
	\begin{itemize}
		\item On suppose que $p = 0$. Alors, $\#(\mathcal{B} \cap \mathcal{B}') = n$ \\
			Or, $\mathcal{B}' \cap \mathcal{B} \subset \mathcal{B}$ donc $\mathcal{B} \cap \mathcal{B}' = \mathcal{B}$ donc $\mathcal{B} \subset  \mathcal{B}'$ et donc $\mathcal{B} = \mathcal{B}'$ 
		\item Soit $p \in \N$. On suppose que si $\mathcal{B}'$ est une base de $E$ telle que $n - \#(\mathcal{B} \cap \mathcal{B}') = p$, alors $\#\mathcal{B}' = n$ \\
			Aoit $\mathcal{B}'$ une base de $E$ telle que $n - \#(\mathcal{B}\cap \mathcal{B}') = p+1 > 0$ \\
			Donc $\mathcal{B} \cap \mathcal{B}' \neq \mathcal{B}$. Soit $u \in \mathcal{B}' \setminus \mathcal{B}$. D'après le lemme d'échange, il existe $v \in \mathcal{B}\setminus \mathcal{B}'$ tel que $\mathcal{B}' \setminus \{u\} \cup \{v\}$ est une base de $E$. On pose $\mathcal{B}'' = \mathcal{B}' \setminus \{u\} \cup \{v\}$ 
			\begin{align*}
				\mathcal{B}'' \cap \mathcal{B} &= \left( (\mathcal{B}' \setminus \{u\})  \cap \mathcal{B} \right) \cup \{v\} \\
				&= (\mathcal{B}' \cap \mathcal{B}) \cup \{v\} \\
			\end{align*}
			donc,
			\begin{align*}
				n - \#(\mathcal{B}'' \cap \mathcal{B}) &= n - (\#(\mathcal{B}' \cap \mathcal{B}) + 1) \\
				&= p+1- 1 \\
				&= p \\
			\end{align*}
			D'après l'hypothèse de récurrence, \[
				\#\mathcal{B}'' = n
			\] Or, $\#\mathcal{B}'' = \#\mathcal{B}'$
	\end{itemize}
\end{prv}

\begin{lem}
	Soient $\mathcal{B}$ et $\mathcal{B}'$ deux bases de $E$ telles que $\mathcal{B}\subset \mathcal{B}'$. Alors, $\mathcal{B} = \mathcal{B}'$.
\end{lem}

\begin{prv}
	On suppose $\mathcal{B}' \neq \mathcal{B}$. Soit $u \in \mathcal{B}' \setminus \mathcal{B}$
	$u \in E = \Vect(\mathcal{B})$ donc $\mathcal{B} \cup \{u\}$ n'est pas libre.
	Donc $\mathcal{B}\cup \{u\} \subset \mathcal{B}'$ et $\mathcal{B}'$ est libre donc $\mathcal{B}\cup \{u\}$ est libre: une contradiction $\lightning$
\end{prv}

\begin{lem}
	[Lemme d'échange] Soient $\mathcal{B}_1$ et $\mathcal{B}_2$ deux bases de $E$ et $u \in \mathcal{B}_1 \setminus \mathcal{B}_2$. Alors, il existe $v \in \mathcal{B}_2$ tel que $(\mathcal{B}_1 \setminus \{u\}) \cup \{v\}$ soit une base de $E$.
\end{lem}

\begin{prv}
	[1${}^\text{nde}$ méthode]
	On suppose que pout tout $v \in \mathcal{B}_2$, $(\mathcal{B}_1\setminus \{u\}) \cup \{v\}$ n'est pas une base de $E$
	Soit $v \in \mathcal{B}_2$.
	\begin{itemize}
		\item Supposons $(\mathcal{B}_1\setminus \{u\})\cup \{v\}$ non libre. $\mathcal{B}_1 \setminus \{u\}$ est libre. Donc $v \in \Vect(\mathcal{B}_1 \setminus \{u\})$
		\item Supposons $(\mathcal{B}_1\setminus \{u\}) \cup \{v\}$ non génératrice.
			Comme $\mathcal{B}_1$ engendre $E$, $u \not\in \Vect(\mathcal{B}_1\setminus \{v\})$.
			On suppose que $\mathcal{B}_1 \neq \mathcal{B}_2$.
			$\forall v \in \mathcal{B}_2 \setminus \mathcal{B}_1, \Vect(\mathcal{B}_1 \setminus \{v\}) = \Vect(\mathcal{B}_1) = E \ni u$ 
			donc, $(\mathcal{B}_1\setminus \{u\}) \cup \{v\}$ engendre $E$ et donc \[
				v \in \Vect(\mathcal{B}_1 \setminus \{u\})
			\] On a aussi \[
				\forall v \in \mathcal{B}_1 \setminus \{u\}, v \in \Vect(\mathcal{B}_1\setminus \{u\})
			\] Comme $u \not\in \mathcal{B}_2$, on a \[
				\forall v \in \mathcal{B}_2, v \in \Vect(\mathcal{B}_1\setminus \{u\})
			\] docn \[
				E = \Vect(\mathcal{B}_2) \subset \Vect(\mathcal{B}_1\setminus \{u\})
			\] donc $\mathcal{B}_1\setminus \{u\}$ engendre $E$ donc $\mathcal{B}_1\setminus \{u\}$ est une base de $E$. Or, $\mathcal{B}_1 \setminus \{u\}  \subset  \mathcal{B}_1$, donc $\mathcal{B}_1\setminus \{u\} = \mathcal{B}_1$
	\end{itemize}
\end{prv}

\begin{prv}
	[2${}^\text{nde}$ méthode]
	On suppose que pout tout $v \in \mathcal{B}_2$, $(\mathcal{B}_1\setminus \{u\}) \cup \{v\}$ n'est pas une base de $E$
	\begin{itemize}
		\item Comme $u \in \mathcal{B}_1 \setminus \mathcal{B}_2$, nécéssairement $\mathcal{B}_1 \neq \mathcal{B}_2$ donc $\mathcal{B}_2 \not\subset \mathcal{B}_1$, donc $\mathcal{B}_2\setminus\mathcal{B}_1 \neq \O$ 
		\item Soit $v \in \mathcal{B}_2\setminus\mathcal{B}_1$. Il existe $(\lambda_w)_{w\in\mathcal{B}_1}$ une famille de scalaires presque nulle telle que \[
				v = \sum_{w \in \mathcal{B}_1} \lambda_w w - \lambda_u u + + \sum_{w \in \mathcal{B}_1\setminus \{u\}}\lambda_w w
			\]
			Si $\lambda_u \neq 0_E$, alors
			\begin{align*}
				u &= \lambda_u^{-1}\left( v - \sum_{w \in \mathcal{B}_1 \setminus \{u\}} \lambda_w w \right)\\
					&\in \Vect(\mathcal{B}_1\setminus \{u\} \cup v)
			\end{align*}
			 donc $\mathcal{B}_1 \subset \Vect(\mathcal{B}_1\setminus \{u\} \cup \{v\})$\\
			 et donc $E \subset  \Vect(\mathcal{B}_1 \setminus \{u\} \cup \{v\})$ \\
			 et donc $\mathcal{B}_1 \setminus \{u\} \cup \{v\}$ engendre $E$ \\
			 donc $\mathcal{B}_1 \setminus \{u\} \cup \{v\}$ n'est pas libre\\
			 donc $v \in \Vect(\mathcal{B}_1\setminus \{u\})$ (car $\mathcal{B}_1 \setminus \{u\}$ est libre\\
			 donc $\lambda_u = 0_\mathbbm{K}$ $\lightning$\\`

			 Donc, $\lambda_u = 0_\mathbbm{K}$, docn $v \in \Vect(\mathcal{B}_1\setminus \{u\})$ \\
			 On vient de prouver que
			 \begin{align*}
			 	\mathcal{B}_2 \setminus \mathcal{B}_1 \subset \Vect(\mathcal{B}_1 \setminus \{u\})\\
			 	\mathcal{B}_1 \setminus \{u\} \subset \Vect(\mathcal{B}_1 \setminus \{u\})\\
			 \end{align*}
			 Comme $u \not\in \mathcal{B}_2$, \[
			 	\mathcal{B}_2 \subset \Vect(\mathcal{B}_1 \setminus \{u\})
			 \] donc \[
			 	E = \Vect(\mathcal{B}_2) \subset  \Vect(\mathcal{B}_1 \setminus \{u\})
			 \] donc $\mathcal{B}_1 \setminus \{u\}$ engendre $E$. Donc,  $\mathcal{B}_1 \setminus \{u\}$ est une base de $E$.\\
			 Or, $\mathcal{B}_1 \setminus \{u\} \subset  \mathcal{B}_1$, donc $\mathcal{B}_1 \setminus \{u\} = \mathcal{B}_1$
	\end{itemize}
\end{prv}

\begin{defn}
	Soit $E$ un $\mathbbm{K}$-espace vectoriel de dimension finie. Le cardinal commun à toutes les bases de $E$ est appelé \underline{dimension} de $E$ est notée $\dim(E)$ ou $\dim_\mathbbm{K}(E)$\\
	C'est donc aussi le nombre de coordonnées de n'importe quel vecteur dans n'importe quelle base.
	\index{dimension (espace vectoriel)}
\end{defn}

\begin{exm}
	\begin{enumerate}
		\item $\dim_\R(\C) = 2$ et $\dim_\C(\C) = 1$ 
		\item $\dim_\mathbbm{K}(\mathbbm{K}^{n}) = n$ 
		\item $\dim_{\mathbbm{K}}(\mathcal{M}_{n,p}(\mathbbm{K})) = np$
	\end{enumerate}
\end{exm}

\begin{crlr}
	Soit $E$ un $\mathbbm{K}$-espace vectoriel de dimension finie, $\mathcal{L}$ une famille libre de $E$, $\mathcal{G}$ une famille génératrice de $E$. On note $n = \dim(E)$
	\begin{enumerate}
		\item $\#\mathcal{G} \ge n$ et $(\#\mathcal{G} = n \implies \mathcal{G} \text{ est une base de } E$)
		\item $\#\mathcal{L} \le n$ et $(\#\mathcal{L} = n \implies \mathcal{L} \text{ est une base de } E$)
	\end{enumerate}
\end{crlr}

\begin{crlr}
	$\R^{\R}$ est de dimension infinie.
	$\forall i \in \N, e_i: x \mapsto x^i$\\
	$(e_i)_{i\in\N}$ est libre dans $\R^\R$
\end{crlr}

\begin{prop}
	Soient $E$ et $F$ deux $\mathbbm{K}$-espaces vectoriels de dimension finie. Alors $E\times F$ est de dimension finie et $\dim(E\times F) = \dim(E) + \dim(F)$
\end{prop}

\begin{prv}
	Soit $(e_1,\ldots, e_n)$ une base de $E$, $(f_1, \ldots, f_p)$ une base de $F$.
	On pose \[
		\left\{\begin{array}
			{r c l}
			u_1 &=& (e_1,0_F)\\
			u_2 &=& (e_2,0_F)\\
					&\vdots&\\
			u_n &=& (e_n,0_F)\\
			u_{n+1} &=& (0_E, f_1)\\
			u_{n+2} &=& (0_E, f_2)\\
					&\vdots&\\
			u_{n+p} &=& (0_E,f_p)\\
		\end{array}\right.
	\]
	Soit $(x,y) \in E\times F$. \[
		\begin{cases}
			\exists (x_1,\ldots,x_n)\in \mathbbm{K}^n, x = \sum_{i=1}^{n} x_ie_i
			\exists (y_1,\ldots,y_n)\in \mathbbm{K}^n, x = \sum_{j=1}^{p} y_jf_j
		\end{cases}
	\] 
	\begin{align*}
		(x,y) &= \left( \sum_{i=1}^{n} x_ie_i, \sum_{i=1}^{p} y_jf_j \right)  \\
		&= \sum_{i=1}^{n} x_i (e_i + 0_F) + \sum_{j=1}^{p} y_j (0_E, f_j) \\
		&= \sum_{i=1}^{n} x_i u_i + \sum_{j=1}^{p} y_j u_{n+j} \\
	\end{align*}
	Donc, $E\times F = \Vect(u_1, \ldots, u_{n+p})$ donc $E\times F$ est de dimension finie.\\
	Soit $(\lambda_1, \ldots, \lambda_{n+p}) \in \mathbbm{K}^{n+p}$ tel que \[
		(*): \quad \sum_{k=1}^{n+p} \lambda_ku_k = 0_{E\times F} = (0_E, 0_F)
	\]
	\begin{align*}
		(*) &\iff \sum_{k=1}^{n} \lambda_k (e_k, 0_F) + \sum_{k=n+1}^{p} \lambda_k(0_E, f_{k-n}) = (0_E, 0_F)\\
				&\iff \begin{cases}
					\sum_{k=1}^{n} \lambda_k e_k = 0_E\\
					\sum_{k=n+1}^{p} \lambda_k f_{k-n} = 0_F
				\end{cases}\\
				&\iff \begin{cases}
					\forall k \in \left\llbracket 1,n \right\rrbracket, \lambda_k = 0_\mathbbm{K} \qquad&(\text{car $(e_1,\ldots,e_n)$ est libre})\\
					\forall k \in \left\llbracket n+1,n+p \right\rrbracket, \lambda_k = 0_\mathbbm{K} \qquad&(\text{car $(f_1,\ldots,f_n)$ est libre})\\
				\end{cases}
	\end{align*}
	Donc $(u_1, \ldots, u_{n+p})$ est une base de $E\times F$. Donc, $\dim(E\times F) = n + p = \dim(E) + \dim(F)$
\end{prv}

\begin{rmk}
	[Convention]
	\[\dim\big(\{0_E\}\big) = 0\]
\end{rmk}

\begin{thm}
	Soit $E$ un $\mathbbm{K}$-espace vectoriel de dimension finie, $F$ un sous-espace vectoriel de $E$. Alors, $F$ est de dimension finie et  $\dim(F) \le \dim(E)$\\
	Si $\dim(F) = \dim(E)$, alors $F = E$
\end{thm}

\begin{prv}
	On considère \[
		A = \{k \in \N \mid \text{il existe une famille libre de $F$ à $k$ éléments}\} 
	\]
	On suppose $F \neq \{0_E\}$.
	\begin{itemize}
		\item Soit $u \in F\setminus \{0_E\}$. $(u)$ est libre donc $1 \in A$ et donc $A \neq \O$
		\item Soit $\mathcal{L}$ une famille libre de $F$. Alors, $\mathcal{L}$ est une famille libre de $E$ \\
			donc $\#\mathcal{L} \le \dim(E)$\\
			Donc $A$ est majorée par $\dim(E)$ \\
			On en déduit que $A$ a un plus grand élément $p$.
		\item Soit $\mathcal{L}$ une famille libre de $F$ avec $p$ éléments.\\
			Si $\mathcal{L}$ n'engendre pas $F$, alors il existe $u\in F$ tel que $u\not\in \Vect(\mathcal{L})$ et donc $\mathcal{L} \cup \{u\}$ est une famille libre de $F$, donc $p+1 \in A$ en contradiction avec la maximalité de $p$.\\
			Donc $\mathcal{L}$ est une base de $F$ donc $F$ est de dimension finie et $\dim(F) = p \le \dim(E)$\\
	\end{itemize}

	Soit $\mathcal{B}$ une base de $F$. Alors, $\mathcal{B}$ est aussi une famille de libre de de $E$. Donc $\#\mathcal{B} \le \dim(E)$ donc $\dim(F) = \dim(E)$ \\
	Si $\dim(F) = \dim(E)$, alors $\mathcal{B}$ est une base de $E$, et donc $F = \Vect(\mathcal{B}) = E$
\end{prv}

\begin{prop}
	[Formule de Grassmann]
	Soit $E$ un $\mathbbm{K}$-espace vectoriel de dimension finie, $F$ et $G$ deux sous-espace vectoriels de $E$. Alors, \[
		\dim(F+G) = \dim(F) + \dim(G) - \dim(F\cap G)
	\] 
\end{prop}

\begin{prv}
	Soit $(e_1, \ldots, e_p)$ une base de $F\cap G$. $(e_1,\ldots,e_p)$ est une famille libre de $F$.\\
	On complète $(e_1, \ldots, e_p)$ en une base $(e_1, \ldots, e_p, u_1, \ldots, u_q)$ de $F$.\\
	De même, on complète $(e_1, \ldots, e_p)$ en une base $(e_1, \ldots, e_p, v_1, \ldots, v_r)$ de $G$.\\
	On pose  $\mathcal{B} = (e_1, \ldots, e_p, u_1, \ldots, u_q, v_1, \ldots, v_r)$. Montrons que $\mathcal{B}$ est une base de $F+G$
	\begin{itemize}
		\item Soit $u \in F+G$ \\
			On pose $u = v+w$ avec $\begin{cases}
				v\in F\\
				w \in G
			\end{cases}$.\\
			On pose $v = \sum_{i=1}^p \lambda_i e_i + \sum_{i=1}^q \mu_i u_i$ avec $(\lambda_1, \ldots, \lambda_p, \mu_1, \ldots, \lambda_q) \in \mathbbm{K}^{p+q}$\\
			On pose aussi $w = \sum_{i = 1}^p \lambda'_ie_i + \sum_{j=1}^r \nu_j v_j$ avec $(\lambda_1',\ldots,\lambda_p', \nu_1, \ldots, \nu_r) \in \mathbbm{K}^{p+r}$\\
			D'où, \[
				u = \sum_{i=1}^p (\lambda_i + \lambda'_i)e_i + \sum_{j=1}^q \mu_j u_j + \sum_{k=1}^r \nu_k v_k \in \Vect(\mathcal{B})
			\]
		\item Soient $(\lambda_1, \ldots, \lambda_p, \mu_1, \ldots, \mu_q, \nu_1, \ldots, \nu_r) \in \mathbbm{K}^{p+q+r}$.\\
			On suppose \[
				(*)\quad \sum_{i=1}^{p}\lambda_ie_i + \sum_{j=1}^q\mu_ju_j + \sum_{k=1}^r \nu_k v_k = 0_E
			\] 
			D'où, \[
				\underbrace{\sum_{i=1}^p\lambda_i e_i + \sum_{j=1}^q \mu_ju_j}_{\in F} = \underbrace{-\sum_{k=1}^r\nu_jv_k}_{\in G}
			\] 
			Donc, \[
				f = \sum_{i=1}^p \lambda_i e_i + \sum_{j=1}^q \mu_j u_j \in F\cap G
			\] Comme $(e_1, \ldots, e_p)$ est une base de $F\cap G$, $\exists ! (\lambda_1', \ldots, \lambda_p') \in \mathbbm{K}^p$ tel que \[
				f = \sum_{i=1}^p \lambda'_i e_i = \sum_{i=1}^p \lambda'_i e_i + \sum_{j=1}^q 0_\mathbbm{K}u_j
			\] Comme $(e_1, \ldots, e_p, u_1, \ldots, u_q)$ est une base de $F$, \[
				\forall k \in \left\llbracket 1, q \right\rrbracket, \mu_j = 0_\mathbbm{K}
			\] De même, \[
				\forall k \in \left\llbracket 1,r \right\rrbracket , \nu_k = 0_\mathbbm{K}
			\] On remplace dans $(*)$ pour trouver \[
				\sum_{i=1}^p \lambda_ie_i = 0_E
			\] Comme $(e_1, \ldots, e_p)$ est libre, \[
				\forall i \in \left\llbracket 1,p \right\rrbracket, \lambda_i = 0_\mathbbm{K}
			\] Donc $\mathcal{B}$ est libre.\\
			Donc, 
			\begin{align*}
				\dim(F+G) &=  p +q + r \\
				&= (p+q)+ (p+r) - p \\
				&= \dim(F) + \dim(G) - \dim(F\cap G) \\
			\end{align*}
	\end{itemize}
\end{prv}

\begin{crlr}
	Avec les hypothèse précédentes, \[
		E = F \oplus G \iff \begin{cases}
			F \cap  G = \{0_E\} \\
			\dim(E) = \dim(F) + \dim(G)
		\end{cases}
	\] 
\end{crlr}

\begin{prv}
	\begin{itemize}
		\item[``$\implies$''] On suppose $E = F \oplus G$ \\
			Comme la somme est directe, $F \cap G = \{0_E\}$ 
			\begin{align*}
				\dim(E) &= \dim(F)\\
				&= \dim(F) + \dim(G) - \dim(F\cap G)\\
				&= \dim(F) + \dim(G)\\
			\end{align*}
		\item[``$\impliedby$''] On suppose $F\cap G = \{0_E\}$ et $\dim(E) = \dim(F) + \dim(G)$.\\
			On sait déjà que $F+G = F \oplus G$\\
			 \begin{align*}
				\dim(F+G) = \dim(F) + \dim(G) - \dim(F \cap G) = \dim(E)
			\end{align*}
			Donc $F + G = E$
	\end{itemize}
\end{prv}

\begin{prop}
	Soit $F$ un $\mathbbm{K}$-espace vectoriel de dimension finie $n$. Soit $\mathcal{B} = (e_1, \ldots, e_n)$ une base de $F$. L'application
	\begin{align*}
		f: \mathbbm{K}^n &\longrightarrow F \\
		(\lambda_1, \ldots, \lambda_n) &\longmapsto \sum_{i=1}^n \lambda_i e_i
	\end{align*} est bijective.\\
	Si $\mathbbm{K}$ est infini, $\mathbbm{K}^n$ aussi et donc $F$ aussi.\\
	Si $\#\mathbbm{K} = p \in \N_*$,
	\begin{align*}
		\#&\mathbbm{K}^n = p^n\\
		&\vrt=\\
		\#&F
	\end{align*}
\end{prop}


		\part{Dérivation}

\underline{Motivation}:

{
\begin{wrapfigure}{l}{3cm}
	\centering
	\begin{asy}
		import three;

		size(3cm);
		settings.render=0;
		settings.prc=false;
		currentprojection = obliqueZ;

		draw(unitbox);
		draw(shift(1.1Z + 0.05X) * (O -- X), Arrows3(TeXHead2));
		draw(shift(1.1Z + 0.05Y) * (O -- Y), Arrows3(TeXHead2));
		draw(shift(1.1X + 0.05Z) * (O -- Z), Arrows3(TeXHead2));

		label("$x$", (X/2) + (1.1Z + 0.05X), align=S);
		label("$y$", (Y/2) + (1.1Z + 0.05Y), align=W);
		label("$z$", (Z/2) + X, align=SE);
	\end{asy}
\end{wrapfigure}

\begin{align*}
	&S(x,y,z) = 2(xy + xz + yz)\\
	&V(x,y,z) = xyz
\end{align*}

On cherche à minimiser $S$ avec la contrainte $V = 1$.

Soit $f : \begin{array}{rcl}
	\left( \R_*^+ \right)^2 &\longrightarrow& \R \\
	(x,y) &\longmapsto& S\left( x,y,\frac{1}{xy} \right) = 2\left( xy + \frac{1}{y} + \frac{1}{x} \right).
\end{array}$

On cherche $(a,b) \in \left( \R^+_* \right)^2$ tel que \[
	\forall (x,y) \in (\R^+_*), f(x,y) \ge f(a,b).
\]
}

\begin{defn}
	Soit $f: U \to \R$ où $U$ est un ouvert de $\R^2$. Soit $(a,b) \in U$.
	\vspace{2mm}

	Si $\lim_{x \to a} \frac{f(x,b) - f(a,b)}{x - a} \in \R$, alors on dit que $f$ a une dérivée partielle suivant $x$ en $(a,b)$ et cette limite est notée \[
		\partial f_1(a,b) = \frac{\partial f}{\partial x}(a,b).
	\]

	Si $\lim_{y \to b} \frac{f(a,y) - f(a,b)}{y - b} \in \R$, alors on dit que $f$ a une dérivée partielle suivant $y$ et la limite est notée \[
		\partial f_2(a,b) = \frac{\partial f}{\partial y}(a,b).
	\]
\end{defn}

\begin{exm}
	\begin{enumerate}
		\item $f: (x,y) \mapsto xy + x - y$.

			\begin{align*}
				&\frac{\partial f}{\partial x} : (x,y) \mapsto y + 1,\\
				&\frac{\partial f}{\partial y} : (x,y) \mapsto x - 1.
			\end{align*}

		\item $f: (x,y) \mapsto xy + \frac{1}{y}+ \frac{1}{x}$.

			\begin{align*}
				&\frac{\partial f}{\partial x}: (x,y) \mapsto y - \frac{1}{x^2},\\
				&\frac{\partial f}{\partial y}: (x,y) \mapsto x - \frac{1}{y^2}.
			\end{align*}

		\item Trouver $f$ telle que $\begin{cases}
				(1): \qquad \frac{\partial f}{\partial x}=y,\\[2mm]
				(2): \qquad \frac{\partial f}{\partial y} = x.
			\end{cases}$

			D'après $(1)$ : \[
				\forall (x,y), \exists C(y) \in \R, f(x,y) = xy + C(y)
			\] et donc \[
				\frac{\partial f}{\partial y}(x,y) = x + C'(y)
			\] donc $C'(y) = 0$ et donc $C$ est constante.

		\item Trouver $f$ telle que $\begin{cases}
			\frac{\partial f}{\partial x} = -y,\\[2mm]
			\frac{\partial f}{ƒ\partial y} = x.
		\end{cases}$

		Ce n'est pas possible !
	\end{enumerate}
\end{exm}

\begin{defn}~\\
	\begin{minipage}{\linewidth}
		\begin{wrapfigure}{r}{4cm}
			\centering
			\vspace{-5mm}
			\begin{asy}
				import three;
				import graph3;
				size(4cm);

				settings.render = 0;
				settings.prc = false;
				currentprojection = obliqueX;

				draw(O -- X, Arrow3(TeXHead2));
				draw(O -- Y, Arrow3(TeXHead2));
				draw(O -- Z, Arrow3(TeXHead2));

				triple f(real x, real y, real z = 0) { return (x,y,cos(x - 0.5) * cos(y - 0.5)/1.2 + 0.15); }

				real inc = 1 / 5;

				for(real x = 0; x <= 1; x += inc) {
					draw(graph(
						new real(real t) { return x; }, // x
						new real(real y) { return y; }, // y
						new real(real y) { return f(x,y).z; }, // z
						0, 1
					), gray);
				}

				for(real y = 0; y <= 1; y += inc) {
					draw(graph(
						new real(real x) { return x; }, // x
						new real(real t) { return y; }, // y
						new real(real x) { return f(x,y).z; }, // z
						0, 1
					), gray);
				}

				path3 path1 = (0.8, 0.2, 0) .. (0.5, 0.5, 0) .. (0.3, 0.7, 0);
				path3 path2 = f(0.8, 0.2, 0) .. f(0.5, 0.5, 0) .. f(0.3, 0.7, 0);
				path3 d = (0.2, 0.3, 0) .. (0.3, 0.4, 0) .. (0.2, 0.7, 0) .. (0.8, 0.9, 0) .. (0.6, 0.2, 0) .. cycle;

				draw(path1, red, Arrow3(TeXHead2));
				draw(path2, red, Arrow3(TeXHead2, position=0.8));

				dot((0.5, 0.5, 0));
				dot(f(0.5, 0.5, 0));
				draw((0.5, 0.5, 0) -- f(0.5, 0.5, 0), dashed);
				draw(d);

				label("$w$", (0.3, 0.7, 0), red, align=SE);
				label("$U$", (0.8, 0.9, 0), align=SE);
			\end{asy}
		\end{wrapfigure}

		Soit $f: U \to \R$ où $U$ est un ouvert. Soit $(a,b) \in U$. Soit $w = (w_1, w_2) \in \R^2$.

		Si 
		\[
			\lim_{t\to 0} \frac{f(a + tw_1, b + tw_2) - f(a,b)}{t}
		\] existe et est réelle, alors on dit que $f$ a une dérivée dans la direction de $w$ et la limite est notée \[
			\mathrm{d}f(w)\,(a,b) = D_w(f)\,(a,b).
		\]
	\end{minipage}
\end{defn}

\begin{exm}
	\begin{align*}
		f: \left( \R_*^+ \right)^2 &\longrightarrow \R \\
		(x,y) &\longmapsto xy+\frac{1}{x}+\frac{1}{y}.
	\end{align*}

	On pose $(a,b) = (1,2)$, $w = (w_1, w_2) = (1,1)$.
	\begin{align*}
		\frac{f(1+t, 2+t) - f(1,2)}{t} &= \frac{1}{t} \left( (1+t)(2+t) + \frac{1}{1+t} + \frac{1}{2+t} - 3 - \frac{1}{2} \right) \\
		&= \frac{1}{t}\left(\cancel 2 + 3t + \po(t) + \cancel 1 - t + \po(t) + \frac{1}{2}\left( \cancel 1 - \frac{t}{2} + \po(t) \right) - \cancel3 - \cancel{\frac{1}{2}} \right) \\
		&= \frac{1}{t} \left( \frac{7}{4} t + \po(t) \right)  \\
		&= \frac{7}{4} + \po(1) \tendsto{t \to 0} \frac{7}{4}. \\
	\end{align*}

	Donc, \[
		\mathrm{d}f(1,1)\,(1,2) = \frac{7}{4}.
	\]
\end{exm}

\begin{rmk}~\\
	\begin{figure}[H]
		\centering
		\begin{asy}
			import solids;
			import graph;
			size(5cm);

			settings.render = 0;
			settings.prc = false;

			path3 par = graph(
				new real(real x) { return x; },
				new real(real x) { return 0; },
				new real(real x) { return x^2; },
				0,3);
			revolution r = revolution(par, axis=Z);

			path3 par2 = graph(
				new real(real x) { return x; },
				new real(real x) { return 0; },
				new real(real x) { return x^2; },
				-3,3);

			draw(r,1,longitudinalpen=nullpen);
			draw(r.silhouette());

			draw((-4, 0, -1) -- (-4, 0, 10) -- (4, 0, 10) -- (4, 0, -1) -- cycle, red);
			draw(par2, deepred);

			draw((4,4.5) -- (7, 4.5), black+0.5mm, Arrow(TeXHead));

			path par2d = graph(new real(real x) { return x^2; }, -3, 3);
			draw(shift((11, 0)) * par2d, deepred);

			dot(O);
			dot((11, 0));
		\end{asy}
	\end{figure}
\end{rmk}


%todo ajouter théorème-définition
\begin{thm}
	Soit $f : U \to \R$, $(a,b) \in U$. On suppose que $\frac{\partial f}{\partial x}$ et $\frac{\partial f}{\partial y}$ existent en $(a,b)$ et sont {\bfseries continues} en $(a,b)$. Alors,
	\begin{align*}
		&\forall (h, k) \in \R^2 \text{ tel que } (a +h, b + k) \in U,\\
		&f(a+ h, b + k) = f(a,b) + h \frac{\partial f}{\partial x}(a,b) + k \frac{\partial f}{\partial y}(a,b) + \po_{(h,k)\to (0,0)}\big(\|(h,k)\|\big).
	\end{align*}

	On dit que $f$ est de classe $\mathcal{C}^1$ si $\frac{\partial f}{\partial x}$ et $\frac{\partial f}{\partial y}$ existent et sont continues.

	\qed
\end{thm}

\begin{rmk}
	En physique, cette formule correspond à : \[
		\mathrm{d}f = \frac{\partial f}{\partial x}\mathrm{d}x + \frac{\partial f}{\partial y} \mathrm{d}y.
	\] En effet :
	\begin{align*}
		\mathrm{d}f &= f(x+ \mathrm{d}x, y + \mathrm{d}y) - f(x,y) \\
		&= \frac{\partial f}{\partial x} \mathrm{d}x + \frac{\partial f}{\partial y} \mathrm{d}y.
	\end{align*}
\end{rmk}

\begin{prop}
	Soit $f: U \to \R$ de classe $\mathcal{C}^1$ en $(a,b) \in U$. Alors, \[
		\forall w = (w_1, w_2) \in \R^2, \mathrm{d}f(w)\,(a,b) = w_1 \frac{\partial f}{\partial x}(a,b) + w_2 \frac{\partial f}{\partial y}(a,b).
	\]
\end{prop}

\begin{prv}
	Soit $w = (w_1, w_2) \in \R^2$. Soit $t \in \R^*$.
	\begin{align*}
		\frac{1}{t}\big(f(a + tw_1, b + tw_2) - f(a,b)\big)
		&= \frac{1}{t} \left( tw_1 \frac{\partial f}{\partial x}(a,b) + tw_2 \frac{\partial f}{\partial y}(a,b) + \po_{t \to 0}\big(\|tw\|\big) \right) \\
		&= w_1 \frac{\partial f}{\partial x}(a,b) + w_2 \frac{\partial f}{\partial y}(a,b) + \po_{t\to 0}(1) \\
		&\tendsto{t\to 0} w_1 \frac{\partial f}{\partial x}(a,b) + w_2\frac{\partial f}{\partial y}(a,b).
	\end{align*}
\end{prv}


\begin{defn}
	Avec les hypothèses précédentes, en posant \[
		\nabla f(a,b) = \left( \frac{\partial f}{\partial x}(a,b), \frac{\partial f}{\partial y}(a,b) \right) 
	\]on obtient \[
		\mathrm{d}f(w)\,(a,b) = \left<w  \mid \nabla f(a,b) \right>
	\] où $\left<\cdot|\cdot \right>$ est le produit scalaire.

	Le vecteur $\nabla f(a,b)$ est appelé \underline{gradient de $f$ en $(a,b)$}.

	Le développement limité à l'ordre 1 de $f$ devient \[
		f\big((a,b)+w\big) = f(a,b) + \left<w \mid \nabla f(a,b) \right> + \po_{w\to 0}(\|w\|)
	\]
\end{defn}

\begin{prop}
	Soit $f : U \to \R$ de classe $\mathcal{C}^1$.

	\begin{figure}[H]
    \centering
    \incfig{gradient}
	\end{figure}

	$\nabla f$ est orthogonal au lignes de niveaux de $f$, son orientation va dans le sens d'une augmentation de $f$.
\end{prop}

\begin{prv}
	Soit $\gamma : I \to U$ une courbe de niveau : \[
		\forall t \in I, f\big(\gamma(t)\big) = \text{cste}.
	\] D'après le lemme suivant : \[
		\forall t \in I, 0 = (f \circ \gamma)'(t) = \mathrm{d}f\big(\gamma'(t)\big)\big(\gamma(t)\big) = \left<\gamma'(t)  \mid \nabla f\big(\gamma(t)\big) \right>
	\] Donc $\nabla f\big(\gamma(t)\big)$ est orthogonal à $\gamma'(t)$.

	Pour tout $t \in I$, on pose $w(t) = t\, \nabla f\big(\gamma(t)\big)$. Donc \[
		f\big(\gamma(t) + w(t)\big) = f\big(\gamma(t)\big) + t \|\nabla f(\gamma(t))\|^2 + \po_{t \to 0}(t)
	\] Pour $t$ assez petit, $f\big(\gamma(t) + w(t)\big) - f\big(\gamma(t)\big)$ est du même signe que $t$.
\end{prv}

\begin{rmk}
	\begin{align*}
		V: \R^3 &\longrightarrow \R \\
		(x,y,z) &\longmapsto -mgz
	\end{align*}
	l'énerge potentielle de pesenteur

	On a donc \[
		\nabla V(x,y,z) = \left( \frac{\partial V}{\partial x}, \frac{\partial V}{\partial y}, \frac{\partial V}{\partial z} \right) = (0, 0, -mg) = \vec{P}.
	\]
\end{rmk}

\begin{lem}
	Soit $f : U \to \R$ de classe $\mathcal{C}^1$, $\gamma : \begin{array}{rcl}
		I &\longrightarrow& U \\
		t &\longmapsto& \big(x(t), y(t)\big)
	\end{array}$ où $x$ et $y$ sont dérivables.

	On pose \[
		\forall t \in I, \gamma'(t) = \big(x'(t), y'(t)\big).
	\] Alors $f \circ \gamma : I \to \R$ est dérivable et
	\begin{align*}
		\forall t \in I, (f \circ \gamma)'(t) &= \mathrm{d}f\big(\gamma'(t)\big) \big(\gamma(t)\big)\\
		&= \left<\gamma'(t)  \mid \nabla f\big(\gamma(t)\big)  \right> \\
		&= x'(t) \frac{\partial f}{\partial x}\big(x(t), y(t)\big) + y'(t) \frac{\partial f}{\partial y}\big(x(t),y(t)\big). \\
	\end{align*}
\end{lem}

\begin{prv}
	On fixe $t \in I$.

	\begin{align*}
		\forall h \neq 0, \frac{f \circ \gamma(t + h) - f \circ \gamma(t)}{h}
		&= \frac{1}{h}\big(f(\gamma(t)) + h\gamma'(t) + \po_{h\to 0}(h) - f(\gamma(t))\big) \\
		&= \frac{1}{h}\bigg(\cancel{f(\gamma(t))} + \left<h\gamma'(t) \mid \nabla f(\gamma(t)) \right> + \po_{h\to 0}(\|h\gamma'(t)\|) - \cancel{f(\gamma(t))}\bigg)\\
		&= \left<\gamma'(t) \mid \nabla f(\gamma(t)) \right> + \po_{h\to 0}(1) \\
		&\tendsto{h\to 0} \left<\gamma'(t)  \mid \nabla f(\gamma(t)) \right>
	\end{align*}
\end{prv}

\begin{defn}
	Soit $f : U \to \R$ de classe $\mathcal{C}^1$ et $(a,b) \in U$. On dit que $(a,b)$ est un \underline{point critique} de $f$ si $\nabla f(a,b) = 0$ i.e. $\frac{\partial f}{\partial x}(a,b) = \frac{\partial f}{\partial y}(a,b) = 0$.

	Dans ce cas, $f(a,b)$ est appelé \underline{valeur critique} de $f$.
\end{defn}

\begin{prop}~\\
	\begin{minipage}{\linewidth}
		\begin{wrapfigure}{r}{3cm}
			\centering
			\vspace{-1cm}
			\begin{asy}
				import solids;
				import graph;
				size(3cm);

				settings.render = 0;
				settings.prc = false;

				path3 par = graph(
					new real(real x) { return x; },
					new real(real x) { return 0; },
					new real(real x) { return -x^2; },
					0,3);
				revolution r = revolution(par, axis=Z);

				draw(r,1,longitudinalpen=nullpen);
				draw(r.silhouette());

				dot("$(a,b)$", O, red, align=N);
				real s = sqrt(2.5);
				path3 g=(s,0,-2.5)..(0,s,-2.5)..(-s,0,-2.5)..(0,-s,-2.5)..cycle;
				draw(g, deepcyan);
			\end{asy}
		\end{wrapfigure}
		Soit $f: U \to \R$ de classe $\mathcal{C}^1$ et $(a,b) \in U$ tel que \[
			\exists r > 0, \forall (x,y) \in B_{(a,b)}(r), f(x,y) \le f(a,b)
		\] Alors $\nabla f(a,b) = (0,0)$.
	\end{minipage}
\end{prop}

\begin{prv}
	Soit $g: x \mapsto f(x,b)$. $g(a)$ est un maximum local de $g$ donc $g'(a) = 0$.

	Or, $g'(a) = \frac{\partial f}{\partial x}(a,b)$

	donc $\frac{\partial f}{\partial x}(a,b) = 0$.

	Soit $h : y \mapsto f(a,y)$. On a de même $h'(b) = 0$.

	Or, $h'(b) = \frac{\partial f}{\partial y}(a,b)$.

	Donc, $\nabla f(a,b) = (0,0)$.
\end{prv}

\begin{rmk}
	Un minimum local est aussi une valeur critique.
\end{rmk}

\begin{figure}[H]
	\centering
	\begin{subfigure}{3cm}
		\centering
		\begin{asy}
			import solids;
			import graph;
			size(3cm);

			settings.render = 0;
			settings.prc = false;

			path3 par = graph(
				new real(real x) { return x; },
				new real(real x) { return 0; },
				new real(real x) { return -x^2; },
				0,3);
			revolution r = revolution(par, axis=Z);

			draw(r,1,longitudinalpen=nullpen);
			draw(r.silhouette());

			dot(O, red);
		\end{asy}
		\caption{Maximum local}
	\end{subfigure}
	\begin{subfigure}{3cm}
		\centering
		\begin{asy}
			import solids;
			import graph;
			size(3cm);

			settings.render = 0;
			settings.prc = false;

			path3 par = graph(
				new real(real x) { return x; },
				new real(real x) { return 0; },
				new real(real x) { return x^2; },
				0,3);
			revolution r = revolution(par, axis=Z);

			draw(r,1,longitudinalpen=nullpen);
			draw(r.silhouette());

			dot(O, red);
		\end{asy}
		\caption{Minimum local}
	\end{subfigure}
	\begin{subfigure}{3cm}
		\centering
		\begin{asy}
			import solids;
			import graph;
			size(3cm);

			settings.render = 0;
			settings.prc = false;
			currentprojection = obliqueZ;

			draw(graph(
				new real(real x) { return x; },
				new real(real x) { return -x^2 / 3; },
				new real(real x) { return 3; },
				-3, 3
			));

			draw(graph(
				new real(real x) { return x; },
				new real(real x) { return -x^2 / 3; },
				new real(real x) { return -3; },
				-3, 3
			));

			draw(graph(
				new real(real x) { return x; },
				new real(real x) { return -x^2 / 3 - 1; },
				new real(real x) { return 0; },
				-3, 3
			));

			draw(graph(
				new real(real x) { return 0; },
				new real(real x) { return x^2 / 9 - 1; },
				new real(real x) { return x; },
				-3, 3
			));

			draw(graph(
				new real(real x) { return -3; },
				new real(real x) { return x^2 / 9 - 4; },
				new real(real x) { return x; },
				-3, 3
			));

			draw(graph(
				new real(real x) { return 3; },
				new real(real x) { return x^2 / 9 - 4; },
				new real(real x) { return x; },
				-3, 3
			));

			dot((0,-1,0), red);
		\end{asy}
		\caption{Point de selle / Point col}
	\end{subfigure}
\end{figure}

\begin{exm}
	On revient à l'exemple donné en introduction : 
	\begin{align*}
		f: \left( \R^*_+ \right)^2 &\longrightarrow \R \\
		(x,y) &\longmapsto 2\left( xy + \frac{1}{x} + \frac{1}{y} \right).
	\end{align*}

	$\left( \R^+_* \right)^2$ est un ouvert de $\R^2$. Soit $(x,y) \in \left( \R^+_* \right)^2$.
	
	On a \[
		\begin{cases}
			\frac{\partial f}{\partial x}(x,y) = 2\left( y - \frac{1}{x^2} \right),\\
			\frac{\partial f}{\partial y}(x,y) = 2\left( x - \frac{1}{y^2} \right).
		\end{cases}
	\]

	\begin{align*}
		&\frac{\partial f}{\partial x}(x,y) = \frac{\partial f}{\partial y}(x,y) = 0\\
		\iff& \begin{cases}
			y = \frac{1}{x^2}\\
			x = \frac{1}{y^2}
		\end{cases}\\
		\iff& \begin{cases}
			y = \frac{1}{x^2}\\
			x = x^4
		\end{cases}\\
		\iff& \begin{cases}
			x = 1\\
			y = 1
		\end{cases}
	\end{align*}

	On vérivie que $f$ présente en effet un minium local en $(1,1)$. \[
		f(1,1) = 6
	\] On fixe $y \in \R^+_*$ et \[
		g : x \mapsto 2\left( xy + \frac{1}{x} + \frac{1}{y} \right).
	\] Donc \[
		\forall x \in \R^+_*, g'(x) = 2\left( y - \frac{1}{x^2} \right).
	\]
	\begin{center}
		\begin{tikzpicture}
			\tkzTabInit{$x$/1,$g'(x)$/1,$g$/2.3}{$0$, $\frac{1}{\sqrt{y}}$, $+\infty$}
			\tkzTabLine{,-,z,+,}
			\tkzTabVar{+/{}, -/$2\left( 2\sqrt{y} +\frac{1}{y} \right)$, +/{}}
		\end{tikzpicture}
	\end{center}
	
	Ainsi, \[
		\forall x \in \R^+_*, \forall y \in \R^+_*, f(x,y) \ge 2\left( 2\sqrt{y} + \frac{1}{y} \right)
	\] Soit $h : y \mapsto 2\sqrt{y} + \frac{1}{y}$. On a \[
		\forall y > 0, h'(y) = \frac{1}{\sqrt{y}} - \frac{1}{y^2} = \frac{y\sqrt{y} - 1}{y^2} = \frac{y^{\frac{3}{2}} - 1}{y^2}
	\]

	\begin{center}
		\begin{tikzpicture}
			\tkzTabInit{$y$/0.7,$h'(y)$/0.7,$h$/1.4}{$0$, $1$, $+\infty$}
			\tkzTabLine{,-,z,+,}
			\tkzTabVar{+/{}, -/$3$, +/{}}
		\end{tikzpicture}
	\end{center}

	Donc, \[
		\forall x,y > 0, f(x,y) \ge 2\times 3 = 6 = f(1,1).
	\]
\end{exm}

\begin{prop}
	[règle de la chaîne]

	Soit $f : \begin{array}{rcl}
		U &\longrightarrow& \R^2 \\
		(x,y) &\longmapsto& f(x,y)
	\end{array}$ de classe $\mathcal{C}^1$ et $U, V$ deux ouverts de $\R^2$.

	Soit $\varphi : \begin{array}{rcl}
		V &\longrightarrow& U \\
		(u,v) &\longmapsto& \varphi(u,v) = \big(x(u,v), y(u,v)\big)
	\end{array}$.

	On suppose que $x$ et $y$ sont de classe $\mathcal{C}^1$ sur $V$.

	Alors,  $f \circ \varphi : \begin{array}{rcl}
		V &\longrightarrow& \R \\
		(u,v) &\longmapsto& f\big(\varphi(u,v)\big)
	\end{array}$ est de classe $\mathcal{C}^1$ et
	\begin{align*}
		\forall (u_0, v_0) \in V, \frac{\partial (f \circ \varphi)}{\partial u}(u_0, v_0)
		&= \frac{\partial f}{\partial x}\big(\varphi(u_0, v_0)\big) \times \frac{\partial x}{\partial u}(u_0, v_0)\\
		&+ \frac{\partial f}{\partial y}\big(\varphi(u_0,v_0)\big) \frac{\partial y}{\partial u}(u_0,v_0)
	\end{align*}
	\begin{align*}
		\forall (u_0, v_0) \in V, \frac{\partial (f \circ \varphi)}{\partial v}(u_0, v_0)
		&= \frac{\partial f}{\partial x}\big(\varphi(u_0, v_0)\big) \times \frac{\partial x}{\partial v}(u_0, v_0)\\
		&+ \frac{\partial f}{\partial y}\big(\varphi(u_0,v_0)\big) \frac{\partial y}{\partial v}(u_0,v_0)
	\end{align*}
\end{prop}

\begin{exm}
	[changement de coordonnées polaires]
	On pose \begin{align*}
		\varphi: \R^+_* \times ]0,2\pi[ &\longrightarrow \R^2\setminus \left( R^+_* \times \{0\} \right) \\
		(r, \theta) &\longmapsto (r \cos \theta, r \sin\theta),
	\end{align*}
	\begin{align*}
		f: \R^2\setminus \left( R^+_* \times \{0\} \right) &\longrightarrow \R \\
		(x,y) &\longmapsto f(x,y),
	\end{align*}
	\begin{align*}
		g: \overbrace{\R^+_* \times ]0, 2\pi[}^{=V} &\longrightarrow \R \\
		(r, \theta) &\longmapsto f(r\cos\theta, r\sin\theta).
	\end{align*}

	\begin{align*}
		\forall (r_0,\theta_0) \in V,&\\[5mm]
		\frac{\partial g}{\partial r}(r_0, \theta_0) &= \frac{\partial f}{\partial x}(r_0\cos\theta_0, r_0\sin\theta_0)\cos\theta_0\\
		&+ \frac{\partial f}{\partial y}(r_0 \cos\theta_0, r_0\sin\theta_0)\sin\theta_0\\
		&= 2r_0\cos^2\theta_0 + 2r_0\sin^2(\theta_0) \\
		&= 2r_0 \\[5mm]
		\frac{\partial g}{\partial \theta}(r_0, \theta_0) &= \frac{\partial f}{\partial x}(r_0\cos\theta_0, r_0\sin\theta_0)r_0\sin\theta_0\\
		&+ \frac{\partial f}{\partial y}(r_0 \cos\theta_0, r_0\sin\theta_0)r_0\cos\theta_0\\
		&= -2{r_0}^2\cos(\theta_0)\sin(\theta_0) + 2{r_0}^2 \sin(\theta_0)\cos(\theta_0)\\
		&= 0 \\
	\end{align*}

	Donc, \[
		g(r, \theta) = r^2.
	\]
\end{exm}

\begin{exm}
	Résoudre \[
		\begin{cases}
			\frac{\partial f}{\partial x} = \frac{x}{x^2+y^2},\\
			\frac{\partial f}{\partial y} = \frac{y}{x^2+y^2}.\\
		\end{cases}
	\]

	On pose $g: (r, \theta) \mapsto f(r \cos\theta, r \sin\theta)$.

	\begin{align*}
		&\frac{\partial g}{\partial r} = \frac{1}{r}\cos^2\theta + \frac{1}{r}\sin^2\theta = \frac{1}{r},\\
		&\frac{\partial g}{\partial \theta} = -\cos(\theta) \sin(\theta) + \sin(\theta)\cos(\theta) = 0.
	\end{align*}

	Donc, \[
		\exists C \in \R, g: (r, \theta) \mapsto \ln r + C
	\] d'où,
	\begin{align*}
		\forall (x,y) \in \R^2 \setminus \{(0,0)\}, f(x,y) &= \ln\left(\sqrt{x^2 + y^2} \right)  + C\\
		&= \frac{1}{2}\ln(x^2 + y^2) + C. \\
	\end{align*}
\end{exm}

\begin{rmk}
	Soit $\mathcal{B} = (e_1, e_2)$ la base canonique de $\R^2$, $f: U \to \R$ de classe $\mathcal{C}^1$ avec $U$ un ouvert de $\R^2$.

	Soit $(x,y) \in U$.

	\begin{align*}
		\Mat_{\mathcal{B}}\big(\nabla f(x,y)\big) = \begin{pmatrix}
			\frac{\partial f}{\partial x}(x,y)\\[2mm]
			\frac{\partial f}{\partial y}(x,y)
		\end{pmatrix}
	\end{align*}

	Soit  \begin{align*}
		\varphi: V &\longrightarrow U \\
		(u,v) &\longmapsto \big(x(u,v), y(u,v)\big) 
	\end{align*} avec $x,y$ de classe $\mathcal{C}^1$. Soit $g = f \circ \varphi$.
	\begin{align*}
		\Mat_{\mathcal{B}}\big(\nabla g(u,v)\big)
		&= \begin{pmatrix}
			\frac{\partial g}{\partial u}(u,v) \\[2mm]
			\frac{\partial g}{\partial v}(u,v)
		\end{pmatrix} \\
		&= \begin{pmatrix}
			\frac{\partial x}{\partial u}(u,v) \frac{\partial f}{\partial x}(x,y)
			+ \frac{\partial y}{\partial u}(u,v)\frac{\partial f}{\partial y}(x,y)\\[3mm]
			\frac{\partial x}{\partial v}(u,v) \frac{\partial f}{\partial x}(x,y)
			+ \frac{\partial y}{\partial v}(u,v) \frac{\partial f}{\partial y}(x,y)
		\end{pmatrix}  \\
		&= \underbrace{\begin{pmatrix}
				\frac{\partial x}{\partial u}(u,v)& \frac{\partial y}{\partial u}(u,v)\\[3mm]
				\frac{\partial x}{\partial v}(u,v)& \frac{\partial y}{\partial v}(u,v)
		\end{pmatrix}}_{J(u,v)} \begin{pmatrix}
			\frac{\partial f}{\partial x}(x,y)\\[3mm]
			\frac{\partial f}{\partial y}(x,y)
		\end{pmatrix} \\
		&= J(u,v) \Mat_{\mathcal{B}}\big(\nabla f(x,y)\big) \\
	\end{align*}
	où $J(u,v) = 
	\begin{pNiceArray}{c:c}
		\Mat_{\mathcal{B}}\big(\nabla x(u,v)\big) & \Mat_{\mathcal{B}}\big(\nabla y(u,v)\big)
	\end{pNiceArray}$.

	On dit que $J(u,v)$ est \underline{la jacobienne} de $\varphi$ en $(u,v)$.
	L'application linéaire canoniquement associée à $J(u,v)$ est la \underline{différentielle de $\varphi$} en $(u,v)$ noté $\mathrm{d}\varphi(u,v)$.

	On a $\mathrm{d}\varphi(u,v) \in \mathcal{L}(R^2)$ et $\Mat_{\mathcal{B}}\big(\mathrm{d}\varphi(u,v)\big) = J(u,v)$.

	Par exemple, la jacobienne du changement de coordonnées polaires est \[
		J = \begin{pmatrix}
			\frac{\partial x}{\partial r} & \frac{\partial y}{\partial r}\\[3mm]
			\frac{\partial x}{\partial \theta} & \frac{\partial y}{\partial \theta}
		\end{pmatrix}
		= \begin{pmatrix}
			\cos\theta&\sin\theta\\
			-r\sin\theta&r\cos\theta
		\end{pmatrix}.
	\]
	$\underbrace{\det(J)}_{\text{le jacobien}} = r\cos^2\theta + r\sin^2\theta = r$

	Dans une intégrale double, si $(x,y) = \varphi(u,v)$, alors $\mathrm{d}x\mathrm{d}y = \det(J)\mathrm{d}u\mathrm{d}v$.

	Ici, \[
		\mathrm{d}x\ \mathrm{d}y = r\ \mathrm{d}r\ \mathrm{d}\theta.
	\]
\end{rmk}

\begin{prv}
	On pose $(x_0, y_0) = \varphi(u_0, v_0)$. Pour tout $(h,k) \in \R^2$ tels que $(u_0 + h, v_0 + k) \in V$, en posant $g = f  \circ \varphi$.

	\begin{align*}
		g(u_0 + h, v_0 + h) &= f\big(x(u_0 + h, v_0 + k), y(u_0 + h, v_0 + k)\big) \\
		&= f\left(
			x(u_0,v_0) + h \frac{\partial x}{\partial u}(u_0,v_0) + k \frac{\partial x}{\partial v}(u_0, v_0) + \po\big(\|(h,k)\|\big), \right.\\
		&\phantom{ = f\bigg(\bigg.}\left. y(u_0, v_0) + h \frac{\partial y}{\partial u}(u_0, v_0) + k \frac{\partial y}{\partial v}(u_0, v_0) + \po\big(\|(h,k)\|\big)
		\right)  \\
		&= f(x_0,y_0) \\
		&~+ \left( h \frac{\partial x}{\partial u}(u_0,v_0) + k \frac{\partial x}{\partial v}(u_0, v_0) + \po(\|(h,k)\|) \right) \frac{\partial f}{\partial x}(x_0,y_0)\\
		&~+ \left( h \frac{\partial y}{\partial u}(u_0, v_0) + k\frac{\partial y}{\partial v}(u_0, v_0) + \po(\|(h,k)\|) \right) \frac{\partial f}{\partial y}(x_0, y_0)\\
		&~+ \po(\|(h,k)\|)\\
		&= f(x_0, y_0) \\
		&~+ h \left( \frac{\partial x}{\partial u}(u_0, v_0) \frac{\partial f}{\partial x}(x_0, y_0) + \frac{\partial y}{\partial u}(u_0, v_0) \frac{\partial f}{\partial y}(x_0, y_0) \right)  \\
		&~+ k\left( \frac{\partial x}{\partial v}(u_0, v_0) \frac{\partial f}{\partial x}(x_0, y_0) + \frac{\partial y}{\partial v}(u_0, v_0) \frac{\partial f}{\partial y}(x_0, y_0) \right) 
		&~+ \po(\|(h,k)\|)\\
		&= g(u_0, v_0) + h \frac{\partial g}{\partial u}(u_0, v_0) + k \frac{\partial g}{\partial v}(u_0, v_0) + \po(\|(h,k)\|) \\
	\end{align*}

	Par identification,
	\[
		\frac{\partial g}{\partial u}(u_0, v_0) = \frac{\partial x}{\partial u}(u_0, v_0) \frac{\partial f}{\partial x}(x_0, y_0) + \frac{\partial y}{\partial u}(u_0, v_0) \frac{\partial f}{\partial y}(x_0,y_0)
	\] et \[
		\frac{\partial g}{\partial v}(u_0, v_0) = \frac{\partial x}{\partial v}(u_0,v_0) \frac{\partial f}{\partial x}(x_0, y_0) + \frac{\partial y}{\partial v}(u_0, v_0) \frac{\partial f}{\partial y}(x_0, y_0).
	\] 
\end{prv}

\begin{exm}
	[Régression linéaire]~\\
	\begin{figure}[H]
		\centering
		\begin{asy}
			import graph;
			axes(EndArrow);
			size(5cm);

			real f(real x) { return x + 0.5; }

			real k = 35 / (7 - 0.5);

			for(int i = 0; i < 35; ++i) {
				real mag = exp(sin(100 * pi/exp(1) * i)) * 0.8 + exp(cos(i*40)/3);
				real eps = mag * cos(10 * exp(1)/pi * i) / 3;
				dot((i/k,f(i/k) + eps));
			}

			draw(graph(f, -1, 7), orange);
		\end{asy}
	\end{figure}
	\[
		y = a x + b
	\] 
	On fixe $(a,b) \in \R^2$. \[
		\varepsilon(a,b) = \sum_{i=1}^n\big( y_i - (ax_i + b) \big)^2
	\] l'erreur totale.

	On veut minimiser $\varepsilon(a,b)$. On a 
	\[
		\forall (a,b) \in \R^2,
		\begin{cases}
			\frac{\partial \varepsilon}{\partial a}(a,b) = -2\sum_{i=1}^{n}(y_i - ax_i - b)x_i,\\
			\frac{\partial \varepsilon}{\partial b}(a,b) = -2\sum_{i=1}^{n}(y_i - ax_i - b).
		\end{cases}
	\]

	Donc,
	\begin{align*}
		(a,b) \text{ point critique de } \varepsilon \iff& \begin{cases}
			a \sum_{i=1}^n {x_i}^2 + b\sum_{i=1}^{n}x_i = \sum_{i=1}^{n} y_ix_i\\
			a\sum_{i=1}^{n}x_i + nb = \sum_{i=1}^ny_i
		\end{cases}\\
		\iff& \begin{cases}
			a \left( \frac{1}{n}\sum_{i=1}^n {x_i}^2 - \overline{x}^2\right) = \overline{y} - \overline{x} \overline{y}\\
			b = \frac{1}{n}\sum_{i=1}^ny_i - \frac{a}{n}\sum_{i=1}^nx_i = \frac{1}{n}\sum_{i=1}^n x_i y_i - \overline{x} \overline{y}
		\end{cases}\\
		&\text{ où } \overline{x} = \frac{1}{n} \sum_{i=1}^n x_i,~\overline{y} = \frac{1}{n}\sum_{i=1}^n y_i\\
		\iff& \begin{cases}
			a = \frac{\Cov(x,y)}{V(x)}\\
			b = \overline{y} - a\overline{x}
		\end{cases}
	\end{align*}

	Coefficient de corrélation: $\frac{\Cov(x,y)}{\sigma_x \sigma_y} \in [-1, 1]$
\end{exm}












	}

	\clearpage
	%\lhead{}
	%\renewcommand*\parttitle{Index}
	\printindex
\end{document}
