\part{Exercice 8}

\[
	A = \{(X,Y) \in \mathcal{P}(E)^2  \mid X \subset Y\}.
\]
\begin{itemize}
	\item[\underline{\sc Méthode 1}] $A = \bigcupdot_{y=0}^n A_y$
		où \[
			\forall y \in \left\llbracket 0,n \right\rrbracket, A_y = \{(X,Y) \in \mathcal{P}(E)^2  \mid X\subset Y \et \#Y = y\}.
		\]
		Soit $y \in \left\llbracket 0,n \right\rrbracket$.
		
		On note $C_y$ l'ensemble \[
			C_y = \{Y \in \mathcal{P}(E)  \mid \#Y = y\}.
		\] On a $\#C_y = {n \choose y}$. Soit $Y \in C_y$. On pose \[
			B_Y = \{X \in \mathcal{P}(E)  \mid X \subset Y\}.
		\] Donc, \[
			A_y = \bigcupdot_{Y \in C_y} B_Y.
		\] Soit $f : \begin{array}{rcl}
			B_Y &\longrightarrow& \mathcal{P}(Y) \\
			(X,Y) &\longmapsto& X.
		\end{array}$

		Donc, \[
			\#B_y = \#\mathcal{P}(Y) = 2^y.
		\] D'où,
		\begin{align*}
			\#A &= \sum_{y=0}^n \sum_{Y \in C_y} 2^y\\
			&= \sum_{y=0}^n {n \choose y} 2^y = 3^n. \\
		\end{align*}
	\item[\underline{\sc Méthode 2}] \[
			A = \bigcupdot_{X \in \mathcal{P}(E)} \underbrace{\{(X,Y) \mid Y \supset X\}}_{B_X}.
		\]
		On pose $f : \begin{array}{rcl}
			B_X &\longrightarrow& \{Y \in \mathcal{P}(E)  \mid Y \supset X\} = C_X \\
			(X,Y) &\longmapsto& Y,
		\end{array}$

		$g : \begin{array}{rcl}
			C_X &\longrightarrow& \mathcal{P}(E \setminus X) \\
			Y &\longmapsto& Y\setminus X,
		\end{array}$

		et $h : \begin{array}{rcl}
			\mathcal{P}(E\setminus X) &\longrightarrow& C_X \\
			Z &\longmapsto& Z \cup X.
		\end{array}$

		Les applications $g$ et $h$ sont réciproques donc 
		\begin{align*}
			\#C_X &= \#\mathcal{P}(E\setminus X) \\
			&= 2^{\#(E\setminus X)} \\
			&= 2^{n - \#X}. \\
		\end{align*}
		Donc
		\begin{align*}
			\#A &= \sum_{X \in \mathcal{P}(E)}2^{n - \#X}\\
			&= \sum_{k=0}^n \sum_{\substack{X \in \mathcal{P}(E)\\ \#X = k}} 2^{n-k} \\
			&= \sum_{k=0}^n 2^{n-k} {n \choose k} = 3^n.
		\end{align*}
	\item[\underline{\sc Méthode 3}] % todo graph
		On note $x_1, \ldots, x_n$ les élements de $E$. Soit $(X,Y) \in A$. On pose \[
			\forall i \in \left\llbracket 1,n \right\rrbracket, a_i^{(X,Y)} = \begin{cases}
				0 &\text{ si } x_i \in X,\\
				1 &\text{ si } x_i \in Y \setminus X,\\
				2 &\text{ si } x_i \not\in Y.
			\end{cases}
		\] On dispose donc de \begin{align*}
			f: A &\longrightarrow \left\llbracket 0,2 \right\rrbracket^n \\
			(X,Y) &\longmapsto \big(a_1(X,Y), a_2(X,Y), \ldots, a_n(X,Y)\big).
		\end{align*}
		Soit  \begin{align*}
			g: \left\llbracket 0,2 \right\rrbracket^n &\longrightarrow A  \\
			(a_1, \ldots, a_n) &\longmapsto \big(\{x_i  \mid i \in \left\llbracket 1,n \right\rrbracket \et a_i = 0\}, \{x_i  \mid i \in \left\llbracket 1,n \right\rrbracket \et a_i \neq 2\}  \big) 
		\end{align*}

		Les applications $f$ et $g$ sont réciproques donc \[
			\#A = \#\big(\left\llbracket 0,2 \right\rrbracket\big)^n = 3^n. 
		\]
\end{itemize}
