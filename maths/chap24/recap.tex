\begin{multicols}{2}
	\begin{recap-box}
		\[
			\#S_n = n!
		.\]
	\end{recap-box}
	\begin{recap-box}
		\[
			\sigma = \begin{pmatrix}
				1&2&3&\cdots&n\\
				\sigma(1)&\sigma(2)&\sigma(3)&\cdots&\sigma(n)
			\end{pmatrix}
		.\]
	\end{recap-box}
	\begin{recap-box}
		On dit que $d$ est l'ordre de $\sigma$ si \[
			\begin{cases}
				\sigma^d = \id;\\
				\forall i \in \left\llbracket 1,d-1 \right\rrbracket,\;\sigma^i \neq \id.
			\end{cases}
		\]
	\end{recap-box}
	\begin{recap-box}
		L'orbite de $\sigma$ pour $x \in \left\llbracket 1,n \right\rrbracket$ est \phantom{$\frac{n}{2}$} \[
			\big\{\,x,\sigma(x),\ldots,\sigma^{d-1}(x)\,\big\}
		\] où $d$ est l'ordre de $\sigma$.\phantom{$\frac{n}{2}$}
	\end{recap-box}
	\begin{recap-box}
		On dit que $\sigma$ est un $k$-cycle si
		\begin{itemize}
			\item $\sigma$ a $n - k$ points fixes;
			\item les autres éléments sont dans une même orbite.\\
		\end{itemize}

		Dans ce cas, $\sigma$ est noté \[
			\sigma = \mat{x&\sigma(x)&\sigma^2(x)&\cdots&\sigma^{k-1}(x)}
		\] et l'ordre de $\sigma$ est $k$.
	\end{recap-box}
	\begin{recap-box}
		Le support de $\sigma$ est l'ensemble des éléments qui ``changent'' après l'application de $\sigma$ : \[
			\Supp(\sigma) = \big\{ x \in \left\llbracket 1,n \right\rrbracket  \mid \sigma(x) \neq x \big\}
		.\]
	\end{recap-box}
	\begin{recap-box}
		Toute permutation peut être décomposée, de manière unique, en cycles à supports disjoints.
	\end{recap-box}
	\begin{recap-box}
		Soit $\gamma$ un $k$-cycle : \[
			\gamma = \mat{a_1&\cdots&a_n-k}
		.\] Alors, pour toute permutation $\sigma$, \[
			\sigma \gamma \sigma^{-1} = \mat{\sigma(a_1) & \cdots & \sigma(a_k)}I
		\] et c'est un $k$-cycle.
	\end{recap-box}
	\begin{recap-box}
		Une transposition est un cycle de longueur 2.
	\end{recap-box}
	\begin{recap-box}
		Toute permutation se décompose en produit de transpositions mais cette décomposition n'est pas unique.
	\end{recap-box}
	\begin{recap-box}
		Une inversion est un couple $(i,j)$ avec $i <j$ mais $\sigma(i) > \sigma(j)$.
	\end{recap-box}
	\begin{recap-box}
		La signature d'une permutation $\sigma$ est $\varepsilon(\sigma) = (-1)^k$ où $k$ est le nombre d'inversions.
		C'est un morphisme de groupes donc \[
			\varepsilon\left( \prod_{i=1}^N \sigma_i \right) = \prod_{i=1}^N \varepsilon(\sigma_i)
		.\] La signature d'une transposition est $-1$.
	\end{recap-box}
	\begin{recap-box}
		Une permutation paire est une permutation de signature $1$.

		Une permutation impaire est une permutation de signature $-1$.

		L'ensemble des permutations paires forment un sous-groupe : le groupe alterné.
	\end{recap-box}
\end{multicols}
