\part{Exercice 10}

\underline{\sc Cas particulier} : $p = 2$, $n = 15$ donc $x_1 + x_2 = 15$.

\begin{align*}
	\left.\begin{array}{ccc}
		0 &+& 15\\
		1 &+& 14\\
		~&\vdots&\\
		15&+&0\\
	\end{array}\right\} 16
\end{align*}

Avec $p = 3$ et $n = 15$, on a donc $x_1 + x_2 + x_3 = 15$.

\begin{align*}
	15 &= 1 + 1 + 1 + 1 + 1 \plusbar 1 + 1 + 1 + 1 + 1 + 1 + 1 \plusbar 1 + 1 + 1\\
	&= 5 + 7 + 3 \\
\end{align*}

Avec $(x_1, x_2, \ldots, x_p) \in \left( \N^* \right)^p$, on en a ${n-1 \choose p-1}$.

\begin{align*}
	15 &= 12 + 0 + 3\\
		 &~~~~~~~\vrt\leftarrow\\
	18 &= 13 + 1 + 4 \\
\end{align*}

On a donc ${n + p - 1 \choose p - 1}$.
