\part{Transpositions}

\begin{defn}
	Une \underline{transposition} est un cycle de longueur 2 : $\begin{pmatrix}
		a&b
	\end{pmatrix}$ avec $a \neq b$.
\end{defn}

\begin{exm}
	Avec $n = 5$ et $\gamma = \begin{pmatrix}
		2&4&1
	\end{pmatrix}$.

	\begin{figure}[H]
		\centering

		\begin{asy}
			size(5cm);

			real rho = 0.15; // circles

			void draw_cycle(pair O, real r ...int[] nums) {
				int n = nums.length;
				real eps = (15 / r) * 0.8;

				for(int i = 0; i < n; ++i) {
					real theta_1 = (360/n) * (i+1);
					real theta_2 = (360/n) * i;

					pair C = O + dir(theta_2) * r;

					draw(circle(C, rho));
					label("$" + string(nums[i]) + "$", C);
					draw(arc(O, r, theta_2+eps, theta_1-eps), Arrow(TeXHead));
				}
			}

			draw_cycle((-1,0), 0.8, 1, 2, 4);
			draw_cycle((1,0), 0.3, 3);
			draw_cycle((2,0), 0.3, 5);
		\end{asy}
	\end{figure}

	\[
		\gamma = \begin{pmatrix}
			1&4
		\end{pmatrix} \begin{pmatrix}
			1&2
		\end{pmatrix}
	\]

	Avec $n = 6$ et $\gamma = \begin{pmatrix}
		1&3&5&6&2
	\end{pmatrix} = \begin{pmatrix}
		1&2&3&4&5&6\\
		3&1&5&4&6&2
	\end{pmatrix}$.

	Donc, \[
		\gamma = \begin{pmatrix}
			1&2
		\end{pmatrix} \begin{pmatrix}
			1&6
		\end{pmatrix} \begin{pmatrix}
			1&5
		\end{pmatrix} \begin{pmatrix}
			1&3
		\end{pmatrix}
	\] 
	\[
		\begin{pmatrix}
			1&2&3&4&5&6\\
			3&2&1&4&5&6\\
			3&2&5&4&1&6\\
			3&2&5&4&6&1\\
			3&1&5&4&6&2\\
		\end{pmatrix}
	\]

	Et, \[
		\gamma = \begin{pmatrix}
			1&3
		\end{pmatrix} \begin{pmatrix}
			2&3
		\end{pmatrix} \begin{pmatrix}
			3&5
		\end{pmatrix} \begin{pmatrix}
			5&6
		\end{pmatrix} 
	\]

	\[
		\begin{pmatrix}
			1&2&3&4&5&6\\
			1&2&3&4&6&5\\
			1&2&5&4&6&3\\
			1&3&5&4&6&2\\
			3&1&5&4&6&2\\
		\end{pmatrix} 
	\] 
\end{exm}

\begin{exm}
	\[
		\begin{pmatrix}
			1&4
		\end{pmatrix} = \begin{pmatrix}
			1&2
		\end{pmatrix} \begin{pmatrix}
			2&3
		\end{pmatrix} \begin{pmatrix}
			3&4
		\end{pmatrix} \begin{pmatrix}
			2&3
		\end{pmatrix} \begin{pmatrix}
			1&2
		\end{pmatrix}
	\]
	On n'a pas toujours le même nombre de transpositions mais la parité du nombre reste la même (proposition plus loin).
\end{exm}

\begin{thm}
	Toute permutation se décompose en produit de transpositions.
\end{thm}

\begin{prv}
	Soit $\gamma = \begin{pmatrix}
		a_1&\cdots&a_k
	\end{pmatrix}$ un $k$-cycle.

	On remarque que
	\[
		\gamma = \begin{pmatrix}
			a_1&a_k
		\end{pmatrix} \cdots \begin{pmatrix}
			a_1&a_4
		\end{pmatrix} \begin{pmatrix}
			a_1&a_3
		\end{pmatrix} \begin{pmatrix}
			a_1&a_2
		\end{pmatrix}
	\] C'est un produit de transpositions.
\end{prv}

\begin{exm}
	Avec $n = 10$ et $\sigma = \begin{pmatrix}
		1&2&3&4&5&6&7&8&9&10\\
		9&8&1&7&2&3&4&5&10&6
	\end{pmatrix}$.

	On a
	\begin{align*}
		\sigma &= \begin{pmatrix}
			1&9&10&6&3
		\end{pmatrix} \begin{pmatrix}
			2&8&5
		\end{pmatrix} \begin{pmatrix}
			4&7
		\end{pmatrix}\\
		&= \begin{pmatrix}
			1&3
		\end{pmatrix} \begin{pmatrix}
			1&6
		\end{pmatrix} \begin{pmatrix}
			1&10
		\end{pmatrix} \begin{pmatrix}
			1&9
		\end{pmatrix} \begin{pmatrix}
			2&5
		\end{pmatrix} \begin{pmatrix}
			2&8
		\end{pmatrix} \begin{pmatrix}
			4&7
		\end{pmatrix} \\
	\end{align*}

	Vérification : \[
		\begin{pmatrix}
			1&2&3&4&5&6&7&8&9&10\\
			1&2&3&7&5&6&4&8&9&10\\
			1&8&3&7&5&6&4&2&9&10\\
			1&8&3&7&2&6&4&5&9&10\\
			9&8&3&7&2&6&4&5&1&10\\
			9&8&3&7&2&6&4&5&10&1\\
			9&8&3&7&2&1&4&5&10&6\\
			9&8&1&7&2&3&4&5&10&6\\
		\end{pmatrix} 
	\] 
\end{exm}
