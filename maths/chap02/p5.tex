\part{Fonctions de $\R$ dans $\C$}

\begin{defn}
	Soit $f$ définie sur $D \subset \R$ à valeurs dans $\C$ ($\forall x \in D, f(x) \in \C$)\\
	On pose:
	\begin{align*}
		\Re(f): D &\longrightarrow \R \\
		x &\longmapsto \Re(f(x))
	\end{align*} et \begin{align*}
		\Im(f): D &\longrightarrow \R \\
		x &\longmapsto \Im(f(x))
	\end{align*}
	\index{Partie réelle (application)}
	\index{Partie imaginaire (application)}
\end{defn}

\begin{exm}
	\begin{align*}
		f: [0,2\pi[ &\longrightarrow \C \\
		x &\longmapsto e^{(1+i)x}
	\end{align*}

	On a: \begin{align*}
		\Re(f): [0,2\pi[ &\longrightarrow \R \\
		x &\longmapsto e^x\cos(x)
	\end{align*} et \begin{align*}
		\Im(f): [0,2\pi[ &\longrightarrow \R \\
		x &\longmapsto e^x\sin(s)
	\end{align*}
\end{exm}

\begin{defn}
	Soit $f: D \to \C$.
	On dit que
	\begin{itemize}
		\item $f$ est \underline{continue} si $\Re(f)$ et $\Im(f)$ sont continues
		\item $f$ est \underline{dérivable} si $\Re(f)$ et $\Im(f)$ sont dérivables.\\
			Dans ce cas, la dérivée de $f$ est \begin{align*}
				f': D &\longrightarrow \C \\
				x &\longmapsto \Re(f)'(x) + i\Im(f)'(x)
			\end{align*}
	\end{itemize}
	\index{continuité (application complexe)}
	\index{dérivabilité (application complexe)}
\end{defn}

\begin{exm}
	\begin{align*}
		f: [0,2\pi[ &\longrightarrow \C \\
		x &\longmapsto e^{(1+i)x}
	\end{align*}
	$x\mapsto e^x$ et $x\mapsto \cos(x)$ sont dérivables sur $[0,2\pi[$ donc $\Re(f)$ est dérivable.\\
	$x\mapsto e^x$ et $x\mapsto \sin(x)$ sont dérivables sur $[0,2\pi[$ donc $\Im(f)$ est dérivable.\\
	Donc $f$ est dérivable. \[
		\forall x \in [0,2\pi[,
		\begin{cases}
			\Re(f)'(x) = e^x\cos(x) - e^x\sin(x)\\
			\Im(f)'(x) = e^x\cos(x) + e^x\sin(x)\\
		\end{cases}
	\]  Donc, \[
		\forall x \in [0,2\pi[, f'(x) = e^x (\cos(x) - \sin(x)) +ie^x(\sin(x) + \cos(x))
	\] 
\end{exm}

\begin{rmk}
	On peut représenter $f$ de la fa\c con suivante.\\
	\begin{align*}
		f: [0,2\pi[ &\longrightarrow \C \\
		t &\longmapsto e^{(1+i)t}
	\end{align*}
	\begin{figure}[H]
		\centering
		\begin{asy}
			import graph;

			size(7cm);
			axes("$\Re$", "$\Im$", EndArrow);

			dot((-3, -4), white+0);
			dot(( 8,  3), white+0);

			pair F(real x) {return exp(x * (0.3,1)); }

			draw(graph(F, 0, 2pi), magenta + 1.4);
			xtick(1); label("$1$", (1,0), align=S);
			xtick(F(2pi).x); label("$e^{2\pi}$", F(2pi), align=N);
		\end{asy}
	\end{figure}
\end{rmk}

\begin{prop}
	Soient $u$ et $v$ deux fonctions dérivables sur $D\subset \R$ à valeurs dans $\C$ 
	\begin{enumerate}
		\item $u+v$ dérivable et $(u+v)' = u' + v'$ 
		\item $uv$ dérivable et $(uv)' = u'v + v'u$ 
		\item Si $v \neq 0$, $\frac{u}{v}$ dérivable et $\left( \frac{u}{v} \right) = \frac{u'v-v'u}{v^2}$
	\end{enumerate}
\end{prop}

\begin{prv}
	On pose $\begin{cases}
		a = \Re(u)\\
		b = \Im(u)\\
	\end{cases}$ et $\begin{cases}
		c = \Re(v)\\
		d = \Im(v)\\
	\end{cases}$\\
	\begin{enumerate}
		\item $\begin{cases}
				\Re(u+v) = a + c\\
				\Im(u+v) = b + d\\
			\end{cases}$ donc $\begin{cases}
				\Re(u+v)' = a'+c'\\
				\Im(u+v)' = b'+d'\\
			\end{cases}$ Donc,
			\begin{align*}
				(u+v)' &= a'+c'+i(b'+d')\\
				&= (a'+ib') + (c'+id') \\
				&= u'+v' \\
			\end{align*}
		\item $\begin{cases}
				\Re(uv) = ac-bd\\
				\Im(uv) = ad+bc
			\end{cases}$ donc $\Re(uv)$ et $\Im(uv)$ sont dérivables et \[
				\begin{cases}
					\Re(uv)' = a'c+c'a-b'd-d'b\\
					\Im(uv)' = a'd+d'a+b'c+c'b\\
				\end{cases}
			\] Donc, \[
				(uv)' = a'c+c'a-b'd-d'b + i(a'd+d'a+b'c+c'b)
			\] 
			Or, \[
				\begin{cases}
					u'v = (a'+ib')(c+id) = a'c - b'd + i(b'c + a'd)\\
					v'u = (a+ib)(c'+id') = ac' - bd' + i(bc' + ad')\\
				\end{cases}
			\] Donc, \[
			(uv)' = u'v + v'u
			\]
		\item On suppose que \[
				\forall x \in D, v(x) \neq 0
			\] 
			On a donc \[
				\forall x \in D, \frac{u}{v} = \frac{a+ib}{c+id} = \frac{(a+ib)(c-id)}{c^2+d^2} = \frac{ac + bd}{c^2+d^2} + i \frac{bc - ad}{c^2+d^2}
			\]
			C'est plus simple de voir $\frac{u}{v}$ comme le produit de $u$ et de $\frac{1}{v}$ 
			\[
				\frac{1}{v} = \frac{1}{c+id} = \frac{c-id}{c^2+d^2}
			\]
			$\underbrace{\frac{c}{c^2+d^2}}_{= \Re\left( \frac{1}{v} \right)}$ et $\underbrace{-\frac{d}{c^2+d^2}}_{= \Im\left( \frac{1}{v} \right)}$ sont dérivables donc $\frac{1}{v}$ aussi\\
			\[
				\begin{cases}
					\Re\left( \frac{1}{v} \right)' = \left( \frac{c}{c^2+d^2} \right)' = \frac{c'(c^2+d^2) - c(2cc' + 2dd')}{\left(c^2+d^2\right)^2}\\
					\Im\left( \frac{1}{v} \right)' = \left(-\frac{d}{c^2+d^2} \right)' = \frac{-d'(c^2+d^2) + d(2cc' + 2dd')}{\left(c^2+d^2\right)^2}\\
				\end{cases}
			\] Donc, d'une part,
			\begin{align*}
				\left( \frac{1}{v} \right) ' &= \frac{c'(c^2+d^2)-c(2cc'-2dd')-id'(c^2+d^2) + d(2cc' + 2dd')}{\left( c^2+d^2 \right) ^2}\\
				&= \frac{(c^2+d^2)(c'-id') + (2cc' + 2dd')(-c+id)}{\left( c^2+d^2 \right) ^2} \\
				&= \frac{-c'c^2+c'd^2-2cdd' + i(2cc'd-d'c^2+d^2d')}{\left( c^2+d^2 \right) ^2} \\
			\end{align*}
			D'autre part,
			\begin{align*}
				\frac{-v'}{v^2} &= \frac{-c'-d'i}{(c+di)^2}\\
				&= \frac{-(c'+id')(c-id)^2}{\left( c^2+d^2 \right) ^2} \\
				&= -\frac{(c'+id')(c^2-2icd-d^2)}{\left( c^2+d^2 \right) ^2} \\
				&= \frac{-c'c^2+c'd^2-2cdd' + i(2cc'd - d'c^2+d'd^2)}{\left( c^2+d^2 \right) ^2} \\
				&= \left( \frac{1}{v} \right)' \\
			\end{align*}
			Donc, $\frac{u}{v}$ dérivable et \[
				\left( \frac{u}{v} \right)' = u'\left( \frac{1}{v} \right) +u\left( \frac{1}{v} \right)' = \frac{u'}{v} - \frac{uv'}{v^2} = \frac{u'v-uv'}{v^2}
			\]
	\end{enumerate}
\end{prv}


\begin{prop}
	Soit $v: D \to \R$ et $u: \R\to \C$ deux fonctions dérivables (avec $D \subset \R$).\\
	Alors, $u \circ v$ est dérivable et \[
		(u \circ v)' = (u' \circ v)\times v'
	\]
\end{prop}

\begin{prv}
	On pose $u = a+ib$ avec $\begin{cases}
		a = \Re(u)\\
		b = \Im(u)
	\end{cases}$ donc, \[
		\forall x \in \R, u(x) = a(x) + ib(x)
	\] Donc, \[
		\forall x \in D, (u \circ v)(x) = a(v(x)) + ib(v(x))
	\] Donc,
	\begin{align*}
		\Re(u \circ v) = a \circ v\\
		\Im(u \circ v) = b \circ v\\
	\end{align*} Or,
	\begin{align*}
		\Re(u \circ v)' = (a \circ v)' = (a' \circ v)\times v'\\
		\Im(u \circ v)' = (b \circ v)' = (b' \circ v)\times v'\\
	\end{align*} D'où
	\begin{align*}
		(u \circ v)' &= (a' \circ v)\times v' + i(b' \circ v)\times v'\\
		&= (a' \circ v + ib' \circ v) \times  v' \\
		&= ((a'+ib')  \circ  v)\times v' \\
		&= (u' \circ v) \times  v' \\
	\end{align*}
\end{prv}

\begin{prop}
	Soit $u: D \to \C$ et $f: \begin{array}{rcl}
		D &\longrightarrow &\C \\
		x &\longmapsto &e^{u(x)}
	\end{array}$ \\
	Alors, $f$ est dérivable sur $D$ et  \[
		\forall x \in D, f'(x) = u'(x) e^{u(x)}
	\]
\end{prop}

\begin{prv}
	On pose $\begin{cases}
		 a = \Re(u)\\
		 b = \Im(u)
	\end{cases}$ donc
	\begin{align*}
		\forall x \in D, f(x) &= e^{u(x)} \\
		&= e^{a(x)+ib(x)} \\
		&= e^{a(x)} (\cos(b(x)) + i\sin(b(x))) \\
	\end{align*}
	Donc, $\begin{cases}
		\Re(f): x\mapsto e^{a(x)}\cos(b(x))\\
		\Im(f): x\mapsto e^{a(x)}\sin(b(x))\\
	\end{cases}$ \\
	$a$, $b$, $\cos$, $\sin$, $\exp$ sont dérivables donc $\Re(f)$ et $\Im(f)$ aussi donc $f$ est dérivable.\\
\end{prv}










