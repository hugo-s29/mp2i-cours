\part{Exercice 1}

\begin{itemize}
	\item[\sc Analyse] Soient $P$ une probabilité sur $\left\llbracket 1,n \right\rrbracket$ et $\alpha \in \R$ tels que \[
			\forall k,\,P(\{k\}) = \alpha k
		.\] D'où,
		\begin{align*}
			1 &= P\big(\left\llbracket 1,n \right\rrbracket\big) = P\left( \bigcupdot_{k=1}^n \{k\} \right)\\
			&= \sum_{k=1}^n P(\{k\}) = \alpha \sum_{k=1}^n k \\
			&= \alpha \frac{n(n+1)}{2} \\
		\end{align*}
		et donc \[
			\alpha = \frac{2}{n(n+1)}
		.\]
	\item[\sc Synthèse] On pose $\alpha = \frac{2}{n(n+1)}$ et \[
		\forall k \in \left\llbracket 1,n \right\rrbracket, p_k = \alpha k
	.\] On remarque que \[
		\forall k \in \left\llbracket 1,n \right\rrbracket, p_k \ge 0
	\] et \[
		\sum_{k=1}^n p_k = \frac{2}{n(n+1)} \sum_{k=1}^n k = \frac{2}{n(n+1)} \times \frac{n(n+1)}{2} = 1
	.\] D'après un résultat du cours, il existe une probabilité $P$ sur $ \left\llbracket 1,n \right\rrbracket $ telle que \[
		\forall k \in \left\llbracket 1,n \right\rrbracket,\,P(\{k\}) = p_k = \alpha k
	.\]
\end{itemize}


