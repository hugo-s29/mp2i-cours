\part{Densité}

\begin{defn}
	Soit $A \in \mathcal{P}(\R)$. On dit que $A$ est \underline{dense dans $\R$}\index{densité (partie de $\R$)} si, pour tout intervalle $I$ ouvert non vide de $\R$, $A \cap I \neq \O$.
\end{defn}

\begin{thm}
	$\Q$ est dense dans $\R$.
\end{thm}

\begin{prv}
	Soit $I$ un intervalle ouvert non vide. Soient $a < b$ deux éléments de $I$. On cherche $(p,q) \in \Z\times \N^*$ tel que \[
		a \le \frac{p}{q}\le b.
	\] Comme $\R$ est archimédien, il existe $q \in \N^*$ tel que \[
		q (b-a) \ge 2 > 1.
	\] Donc, l'intervalle $[qa, qb]$ a une longueure supérieure à $1$, il contient donc au moins un entier $p$.

	En effet, sinon on a $\left\lfloor qa \right\rfloor = \left\lfloor qb \right\rfloor$ et alors \[
		\begin{cases}
			\left\lfloor qa \right\rfloor \le qb < \left\lfloor qa \right\rfloor + 1\\
			\left\lfloor qa \right\rfloor \le qa < \left\lfloor qa \right\rfloor + 1
		\end{cases}
	\] et alors \[
		-1 < qb - qa < 1
	\] une contradiction. On a donc $qa \le p \le qb$ et finalement \[
		a \le \frac{p}{q} \le b.
	\]
\end{prv}

\begin{thm}
	$\R\setminus\Q$ est dense dans $\R$.
\end{thm}

\missingpart

