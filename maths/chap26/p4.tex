\part{Développement suivant une ligne ou une colonne}

\begin{exm}
	\[
		\begin{vmatrix}
			+&-&+&-\\
			-&+&-&+\\
			+&-&+&-\\
			-&+&-&+\\
		\end{vmatrix}
	\] 

	\begin{align*}
		\left|
		\begin{array}{c|c|cc}
			\cline{2-2}
			1&2&3&0\\
			4&5&6&-1\\
			7&8&9&-2\\
			3&2&1&0\\\cline{2-2}
		\end{array}
		\right| &= -2 \begin{vmatrix}
			4&6&-1\\
			7&9&-2\\
			3&1&0
		\end{vmatrix}
		+ 5
		\begin{vmatrix}
			1&3&0\\
			7&9&-2\\
			3&1&0
		\end{vmatrix}
		-8
		\begin{vmatrix}
			1&3&0\\
			4&6&-1\\
			3&1&0
		\end{vmatrix} + 2
		\begin{vmatrix}
			1&3&0\\
			4&6&-1\\
			7&9&-2
		\end{vmatrix}\\
		\left|
		\begin{array}{ccc|c|}
			\cline{4-4}
			1&2&3&0\\
			4&5&6&-1\\
			7&8&9&-2\\
			3&2&1&0\\\cline{4-4}
		\end{array}
		\right|
		&= -1
		\begin{vmatrix}
			1&2&3\\
			7&8&9\\
			3&2&1
		\end{vmatrix} + 2
		\begin{vmatrix}
			1&2&3\\
			4&5&6\\
			3&2&1
		\end{vmatrix}\\
		\left|
		\begin{array}{cccc}
			1&2&3&0\\
			4&5&6&-1\\
			7&8&9&-2\\
			\cline{1-4}
			\multicolumn{1}{|c}{3}&2&1&\multicolumn{1}{c|}{0}\\\cline{1-4}
		\end{array}
		\right| &= -3
		\begin{vmatrix}
			2&3&0\\
			5&6&-1\\
			8&9&-1\\
		\end{vmatrix}
		+ 2
		\begin{vmatrix}
			1&3&0\\
			4&6&-1\\
			7&9&-2
		\end{vmatrix} -
		\begin{vmatrix}
			1&2&0\\
			4&5&-1\\
			7&8&9
		\end{vmatrix}\\
	\end{align*}

	$E = \C^4$ et $\mathcal{B} = (e_1, e_2, e_3, e_4)$ base canonique de $E$.

	On pose $\begin{cases}
		u_1 = (1,4,7,3),\\
		u_2 = (2,5,8,2),\\
		u_3 = (3,6,9,1),\\
		u_4 = (0,-1,-2,0).
	\end{cases}$ et $A = \Mat_\mathcal{B}(u_1, u_2, u_3, u_4) = \Mat_\mathcal{B}(u_1, 2e_1 + 5e_2 + 8e_3 + 2e_4, u_3, u_4)$.

	Donc,
	\begin{align*}
		\det A &= \det_\mathcal{B}(u_1, 2e_1 + 5e_2 + 8e_3 + 2e_4, u_3, u_4) \\
		&= \phantom{+}\,2\det_\mathcal{B}(u_1, e_1, u_3, u_4) \\
		&\phantom{=}+ 5\det_\mathcal{B}(u_1, e_2, u_3, u_4)\\
		&\phantom{=}+ 9\det_\mathcal{B}(u_1, e_3, u_3, u_4)\\
		&\phantom{=}+ 2\det_\mathcal{B}(u_1, e_4, u_3, u_4).\\
	\end{align*}

	\begin{align*}
		\det_\mathcal{B}(u_1, e_1, u_3, u_4) &= 
		\begin{vmatrix}
			1&1&3&0\\
			4&0&6&-1\\
			7&0&9&-2\\
			3&0&1&0
		\end{vmatrix}\\
		&= -
		\begin{vmatrix}
			\bx 1 & 1 & 3 & 0\\
			0&4&6&-1\\
			0&7&9&-2\\
			0&3&1&0
		\end{vmatrix}\\
		&= -
		\left|\begin{NiceArray}{c|ccc}
			1&0&0&0\\\hline
			0&4&6&-1\\
			0&7&9&-2\\
			0&3&1&0
		\end{NiceArray}\right|\\
		&= \ldots \\
		&= - \left|
		\begin{NiceArray}{c|ccc}
			1&&0&\\\hline
			 &\Block{3-3}{\diagbox{(0)}{*}}\\
			0\\
			\\
		\end{NiceArray}\right|  \\
	\end{align*}

	Idem pour $e_2, e_3, e_4$.
\end{exm}

\begin{prop}[Laplace]
	Soit $A = (a_{ij}) \in \mathcal{M}_n(\mathbbm{K})$. \[
		\forall j \in \left\llbracket 1,n \right\rrbracket,\, \det(A) = \sum_{i=1}^n (-1)^{i+j} a_{i,j} m_{i,j}
	\] où $m_{i,j}$ est le \underline{mineur} d'indices $(i,j)$, i.e. le déterminant de la matrice obtenue en supprimant la ligne $i$ et la colonne $j$ de la matrice $A$. \[
		\forall i \in \left\llbracket 1,n \right\rrbracket,\,
		\det(A) = \sum_{j=1}^n (-1)^{i+j} a_{i,j} m_{i,j}.
	\]\qed
\end{prop}

\begin{prop}[Vandermonde]
	Soient $(a_1, \ldots, a_n) \in \mathbbm{K}^n$. \[
		\begin{vmatrix}
			1&a_1&a_1^2&\cdots&a_1^{n-1}\\
			1&a_2&a_2^2&\cdots&a_2^{n-1}\\
			\vdots&\vdots&\vdots&\ddots&\vdots\\
			1&a_n&a_n^2&\cdots&a_n^{n-1}
		\end{vmatrix} = \prod_{j<i} (a_j - a_i)
	\]
\end{prop}

\begin{prv}[par récurrence sur $n$]
	\begin{itemize}
		\item Soit $n \in \N$. Soient $a_1, \ldots, a_{n+1} \in \mathbbm{K}$. On pose \[
				\Delta_{n+1} =
				\begin{vmatrix}
					1&a_1&\cdots&a_1^n\\
					\vdots&\vdots&\ddots&\vdots\\
					1&a_{n+1}&\cdots&a_{n+1}^{n}
				\end{vmatrix}
			\] On développe $\Delta_{n+1}$ suivant la dernière ligne et on obtient une expression polynomial en $a_{n+1}$ : \[
					\Delta_{n+1} = P(a_{n+1}) \text{ avec } P \in \mathbbm{K}_n[X]
			\] et le coefficent devant $X^n$ est $\Delta_n$.
			\begin{itemize}
				\item[\sc Cas 1] On suppose $\Delta_n \neq 0$. D'après l'hypothèse de récurrence : \[
						\begin{cases}
							\forall i,j \in \left\llbracket 1,n \right\rrbracket,\, i\neq j \implies a_i \neq a_j\\
							\dom(P_n) = \Delta_n = \prod_{n\ge j> i\ge 1} (a_j - a_i)\\
							\deg(P_n) = n\\
							\forall i \in \left\llbracket 1,n \right\rrbracket, P(a_i) = 0
						\end{cases}
					\] donc \[
						P = \Delta_n (X-a_1) \cdots (X-a_n)
					\] donc
					\begin{align*}
						\Delta_{n+1} &= \prod_{n \ge j > i \ge 1} (a_j - a_i)\;(a_{n+1}-a_1)\cdots(a_{n+1} - a_n)\\
						&= \prod_{n+1 \ge j > i \ge 1} (a_j - a_i) \\
					\end{align*}
				\item[\sc Cas 2] On suppose $\Delta_n = 0$ : \[
						\exists i,j \in \left\llbracket 1,n \right\rrbracket,\, i\neq j \et a_i = a_j.
					\] Alors, \[
						\Delta_{n+1} = 0 = \prod_{n+1\ge j > i \ge 1} (a_j - a_i).
					\]
			\end{itemize}
	\end{itemize}
\end{prv}

\begin{prop-defn}
	Soit $A \in \mathcal{M}_n(\mathbbm{K})$, et $\com(A)$ la \underline{comatrice}\index{comatrice} de $A$ : c'est la matrice \[
		\left( (-1)^{i+j}\; m_{i,j} \right)_{1\le i,j\le n}
	\] où $m_{i,j}$ est le mineur d'indices $(i,j)$ de $A$.

	On a \[
		A\;\t\com(A) = \t\com(A)\;A = \det(A)\;I_n.
	\] Si $\det A \neq 0$, alors \[
		A^{-1} = \frac{1}{\det(A)} \; \t\com(A).
	\]
\end{prop-defn}

\begin{exm}
	Avec $A = \begin{pmatrix}
		a&c\\
		b&d
	\end{pmatrix}$, on a $\det A = ad - bc$.

	\[
		\com A = \begin{pmatrix}
			+d&-b\\
			-c&+a
		\end{pmatrix} \text{ donc } \t \com(A) = \begin{pmatrix}
			d & -c\\
			-b & a
		\end{pmatrix}
	\] Si $\det(A) \neq 0$, \[
		A^{-1} = \frac{1}{ad - bc} \begin{pmatrix}
			d&-c\\
			-b&a
		\end{pmatrix}
	\]
\end{exm}

\begin{exm}
	Avec $A = \begin{pmatrix}
		a_1&b_1&c_1\\
		a_2&b_2&c_2\\
		a_3&b_3&c_3\\
	\end{pmatrix}$, on a \[
		\com A = \begin{pmatrix}
			b_2c_3-b_3c_2 & a_3c_2-a_2c_3 & a_2b_3-a_3b_2\\[2mm]
			c_1b_3-b_1c_3 & a_1c_3-a_3c_1 & a_3b_1-a_1 b_3\\[2mm]
			b_1c_2-b_2c_1 & a_2c_1-a_1c_2 & a_1b_2-a_2b_1
		\end{pmatrix}
	\]
\end{exm}

\begin{prv}
	On note $C = A\; \t\com(A) = (c_{i,j})$.
	\begin{align*}
		\forall i \in \left\llbracket 1,n \right\rrbracket,
		c_{ii} = \sum_{j=1}^n a_{ij} (-1)^{i+j} m_{ij} = \det(A).
	\end{align*}
	Soient $i \in \left\llbracket 1,n \right\rrbracket$, $j \in \left\llbracket 1,n \right\rrbracket$ avec $i \neq j$. \[
		c_{ij} = \sum_{k=1}^n a_{ik}(-1)^{j+k} m_{jk}.
	\] C'est le déterminant de la matrice obtenue en rempla\c cant la $j$-ème ligne de $A$ par la $i$-ème, qui a donc $2$ lignes identiques donc $c_{ij} = 0$.

	L'autre formule ``$\;\t\com(A)\;A = \det(A)I_n$'' se démontre de la même fa\c con.
\end{prv}

\begin{rmk}
	En pratique, on n'utilise jamais cette formule pour trouver $A^{-1}$ (sauf si $n = 2$).
\end{rmk}

\begin{exo}[Formules de Cramer]~
	\centered{\large\color{red}\sc Inutiles et Hors-Programme}

	Soit $S$ le système $AX = B$ avec $A \in \mathrm{GL}_n(\mathbbm{K})$ et $B \in \mathcal{M}_{n,1}(\mathbbm{K})$.

	L'unique solution du système $S$ est donnée par \[
		X = \begin{pmatrix}
			x_1\\\vdots\\x_n
		\end{pmatrix} \text{ où } \forall i,\; x_i = \frac{\det A_i}{\det A}
	\] où $A_i$ est obtenue en repla\c cant la $i$-ème colonne de $A$ par $B$.

	\underline{Solution :} Soit $(x_1, \ldots, x_n)$ l'unique solution de $S$. \[
		\sum_{i=1}^n x_i C_i(A) = B
	\] où $C_i(A)$ est la $i$-ème colonne de $A$.
	\begin{align*}
		\forall j, \det(A_j) &= \det\big(C_1(A), C_2(A), \ldots, C_{j-1}(A), B, C_{j+1}(A),\ldots, C_n(A)\big)\\
		&= \sum_{i=1}^n x_i \det\big(C_1(A), \ldots, C_{j-1}(A), C_i(A), C_{j+1}(A), \ldots, c_n(A)\big) \\
		&= x_j \det(A) \\
	\end{align*}
\end{exo}

\centered{Retour sur la diagonalisation}

Soit $A \in \mathcal{M}_n(\mathbbm{K})$. On cherche $P \in \mathrm{GL}_n(\mathbbm{K})$ telle que \[
	P\;A\;P^{-1} =
	\begin{pNiceMatrix}
		\lambda_1&\Block{2-2}{(0)}&\\
		\Block{2-2}{(0)}&\Ddots\\
		&&\lambda_n
	\end{pNiceMatrix}
\] Soit $f \in \mathcal{L}(\mathbbm{K}^n)$ telle que $\Mat_\mathcal{B}(f) = A$ où $\mathcal{B}$ est la base canonique de $\mathbbm{K}^n$. On cherche $\mathcal{C} = (u_1, \ldots, u_n) \in \mathbbm{K}^n$ telle que \[
	\exists \lambda_i, f(u_i) = \lambda_i u_i
\]

\begin{align*}
	&\exists \lambda \in \mathbbm{K},\,\exists u\neq 0,\,f(u) = \lambda u\\
	\iff & \exists \lambda \in \mathbbm{K},\,\exists u\neq 0,\, u \in \Ker(f - \lambda \id_{\mathbbm{K}^n})\\
	\iff & \exists \lambda \in \mathbbm{K},\,f-\lambda \id_{\mathbbm{K}^n} \text{ n'est pas injective}\\
	\iff & \exists \lambda \in \mathbbm{K},\, f - \lambda \id_{\mathbbm{K}^n} \not\in \mathrm{GL}(\mathbbm{K}^n)\\
	\iff & \exists \lambda \in \mathbbm{K},\, \det(\lambda - \id_{\mathbbm{K}^n}) = 0\\
	\iff & \exists \lambda \in \mathbbm{K},\,\underbrace{\det(A - \lambda I_n)}_{\text{polynôme caractéristique de $A$}} = 0
\end{align*}
