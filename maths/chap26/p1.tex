\part{Définitions}

\begin{defn}
	Soit $E$ un $\mathbbm{K}$-espace vectoriel de dimension $n < +\infty$ et $f : E^n \longrightarrow \mathbbm{K}$. On dit que $f$ est \underline{multilinéaire} si
	\[
		\begin{array}{c}
			\forall i \in \left\llbracket 1,n \right\rrbracket, \forall (u_1, \ldots, u_{i-1}, u_{i+1}, \ldots, u_n) \in E^{n-1},\\[3mm]
			\text{l'application } \begin{array}{rcl}
				E &\longrightarrow& \mathbbm{K} \\
				u &\longmapsto& f(u_1, \ldots, u_{i-1}, u, u_{i+1}, \ldots, u_{n})
			\end{array} \text{ est linéaire.}
		\end{array}
	\] 
	\index{multilinéaire (application)}

	On dit que $f$ est \underline{antisymétrique} si
	\[
		\begin{array}{c}
			\forall i < j, \forall (u_1,\ldots, u_n) \in E^n,\\
			\phantom{=-}f(u_1, \ldots, u_{i-1}, u_{\color{red} j}, u_{i+1}, \ldots, u_{j-1}, u_{\color{red}i}, u_{j+1}, \ldots, u_n)\\
			=-f(u_1, \ldots, u_{i-1}, u_{\color{red} i}, u_{i+1}, \ldots, u_{j-1}, u_{\color{red}j}, u_{j+1}, \ldots, u_n).
		\end{array}
	\] 

	\index{antisymétrique (application)}

	On dit que $f$ est \underline{alternée} si \[
		\forall (u_1, \ldots, u_n) \in E^n, \big(\exists i < j, u_i = u_j \implies f(u_1,\ldots,u_n) = 0\big).
	\]
	\index{alternée (application)}
\end{defn}

\begin{prop}
	Soit $\mathbbm{K}$ un corps tel que $1 + 1 \neq 0$, $E$ un $\mathbbm{K}$-espace vectoriel de dimension $n$ et $f : E^{n}\to \mathbbm{K}$ une forme multilinéaire.

	Alors, \[
		f \text{ alternée} \iff f \text{ antisymétrique}.
	\]
\end{prop}

\begin{prv}
	\begin{itemize}
		\item[``$\implies$''] On suppose $f$ alternée.

			Soit $(u_1, \ldots, u_n) \in E^n$ et $(i,j) \in \left\llbracket 1,n \right\rrbracket^2$ avec $i < j$.

			\begin{align*}
				0 &= f(u_1, \ldots, u_{i-1}, u_i + u_j, u_{i+1}, \ldots, u_{j-1}, u_i + u_j, u_{j+1}, \ldots, u_n)\\
				&=\phantom{+} f(u_1, \ldots, u_{i-1}, u_i, u_{i+1}, \ldots, u_{j-1}, u_i + u_j, u_{j+1},\ldots, u_n) \\
				&\phantom{=}+f(u_1, \ldots, u_{i-1}, u_j, u_{i+1}, \ldots, u_{j-1}, u_i + u_j, u_{j+1}, \ldots, u_n)\\
				&=\,\phantom{+} f(u_1, \ldots, u_{i-1}, u_i, u_{i+1}, \ldots, u_{j-1}, u_i, u_{j+1}, \ldots, u_n) \\
				&\phantom{=}\, + f(u_1, \ldots, u_{i-1}, u_i, u_{i+1}, \ldots, u_{j-1}, u_j, u_{j+1}, \ldots, u_n)\\
				&\phantom{=}\, + f(u_1, \ldots, u_{i-1}, u_j, u_{i+1}, \ldots, u_{j-1}, u_i, u_{j+1}, \ldots, u_n)\\
				&\phantom{=}\, + f(u_1, \ldots, u_{i-1}, u_j, u_{i+1}, \ldots, u_{j-1}, u_j, u_{j+1}, \ldots, u_n)\\
			\end{align*}

			donc
			\begin{align*}
				&f(u_1, \ldots, u_{i-1}, u_j, u_{i+1}, u_{i+1}, \ldots, u_{j-1}, u_i, u_{j+1}, \ldots, u_n)\\
				=-&f(u_1, \ldots, u_{i-1}, u_i, u_{i+1}, u_{i+1}, \ldots, u_{j-1}, u_j, u_{j+1}, \ldots, u_n).\\
			\end{align*}

			Donc, $f$ est antisymétrique.
		\item[``$\impliedby$''] On suppose $f$ antisymétrique.
			Soit $(u_1, \ldots, u_n) \in E^n$ et $(i,j) \in \left\llbracket 1,n \right\rrbracket^2$ tels que $i<j$ et $u_i < u_j$.

			\begin{align*}
				&f(u_1, \ldots, u_{i-1}, u_i, u_{i+1}, \ldots, u_{j-1}, u_j, u_{j+1}, \ldots, u_n)\\
				=&f(u_1, \ldots, u_{i-1}, u_j, u_{i+1}, \ldots, u_{j-1}, u_i, u_{j+1}, \ldots, u_n)\\
				=-&f(u_1, \ldots, u_{i-1}, u_i, u_{i+1}, \ldots, u_{j-1}, u_j, u_{j+1}, \ldots, u_n)\\
			\end{align*}

			D'où, \[
				f(u_1, \ldots, u_n)\underbrace{(1+1)}_{\neq 0} = 0
			\] donc $f(u_1, \ldots, u_n) = 0$.
	\end{itemize}
\end{prv}

Dans le reste du chapitre, $\mathbbm{K}$ est un corps avec $1 + 1 \neq 0$.

\begin{thm}
	Soit $E$ un $\mathbbm{K}$-espace vectoriel de dimension $n$. L'ensemble des formes multilinéaires alternées de $E$ est un sous-espace vectoriel de $\mathbbm{K}^{(E^n)}$ de dimension 1.
\end{thm}

\begin{prv}[pas exigible]
	Soit $\mathcal{B} = (e_1, \ldots, e_n)$ une base de $E$. Soit $f$ une forme multilinéaire alternée. Soit $(u_1, \ldots, u_n) \in E^n$.
	\[
		\forall j \in \left\llbracket 1,n \right\rrbracket,\, \exists (a_{i,j})_{1\le i\le n},\, u_j = \sum_{i=1}^n a_{i,j} e_i
	\]

	\begin{align*}
		f(u_1, \ldots, u_n) &= f\left( \sum_{i=1}^n a_{i,1} e_i, u_2, \ldots, u_n \right) \\
		&= \sum_{i=1}^n a_{i,1} f(e_i, u_2, \ldots, u_n) \\
		&= \sum_{i_1 = 1}^n a_{i_1,1} f\left( e_{i_1}, \sum_{i_2 = 1}^n a_{i_2,2} e_{i_2}, u_3, \ldots, u_n \right)\\
		&= \sum_{i_1 = 1}^n \sum_{i_2=1}^n a_{i_1,1} a_{i_2,2}, f(e_{i_1}, e_{i_2}, u_3, \ldots,u_n) \\
		&\kern1.5mm\vdots \\
		&= \sum_{i_1=1}^n \sum_{i_2=1}^n \cdots \sum_{i_n = 1}^n a_{i_1,1} a_{i_2,2} \cdots a_{i_n,n} f(e_{i_1}, e_{i_2}, \ldots, e_{i_n})\\
		&= \sum_{\sigma \in S_n} a_{\sigma(1),1} a_{\sigma(2),2} \cdots a_{\sigma(n),n} f(e_{\sigma(1)}, e_{\sigma(2)},\ldots, e_{\sigma(n)}) \\
		&= \underbrace{\sum_{\sigma \in S_n}\left( \varepsilon(\sigma) \prod_{j=1}^n a_{\sigma(j), j} \right)}_{A(u_1, \ldots, u_n)}\; \underbrace{f(e_1, \ldots, e_n)}_{\in \mathbbm{K}}\\
	\end{align*}

	D'où, $f = f(e_1, \ldots, e_n)\;A$.

	Donc, l'ensemble des formes multilinéaires alternées est $\Vect(A)$. De plus,
	\begin{align*}
		A(e_1, \ldots, e_n) &= \sum_{\sigma \in S_n} \varepsilon(\sigma) \prod_{j=1}^n  \delta_{\sigma(j), j} \\
		&= \varepsilon(\id) \prod_{j=1}^n 1 = 1 \neq 0. \\
	\end{align*}
\end{prv}

\begin{defn}
	Soit $E$ un $\mathbbm{K}$-espace vectoriel de dimension $n$ et $\mathcal{B} = (e_1, \ldots, e_n)$ une base de $E$.

	Il existe une unique forme $f$ multilinéaire alternée sur $E$ telle que $f(e_1, \ldots, e_n) = 1$. Elle est donnée par la formule \[
		\forall (u_1, \ldots, u_n) \in E^n,
		f(u_1, \ldots, u_n) = \sum_{\sigma \in S_n} \varepsilon(\sigma) \prod_{j=1}^n a_{\sigma(j), j}
	\] où \[
		\forall i,j,\;a_{i,j} \text{ est la $i$-ème coordonnée de $u_j$ dans la base $\mathcal{B}$.}
	\] Cette application est appelée \underlin{déterminant dans la base $\mathcal{B}$} et noté $\det_{\mathcal{B}}$.
	\index{déterminant (dans une base $\mathcal{B}$)}
\end{defn}

\begin{prop}
	Soient $\mathcal{B} = (e_1, \ldots, e_n)$ et $\mathcal{C} = (u_1, \ldots, u_n)$ deux bases de $E$. Alors \[
		\det_\mathcal{C} = \det_\mathcal{C}(e_1, \ldots, e_n)\;\det_\mathcal{B}
	\] i.e. \[
		\forall (v_1, \ldots, v_n) \in E^n,
		\det_\mathcal{C}(v_1, \ldots, v_n) = \det_\mathcal{C}(e_1, \ldots, e_n) \; \det_\mathcal{B}(v_1, \ldots, v_n).
	\]
\end{prop}

\begin{prv}
	On sait que $\det_\mathcal{B}$ et $\det_\mathcal{C}$ sont colinéaires : \[
		\exists \lambda \in \mathbbm{K},\, \forall (v_1, \ldots, v_n) \in E^n,\, \det_\mathcal{C}(v_1, \ldots, v_n) = \lambda \det_\mathcal{B}(v_1, \ldots, v_n).
	\] En particulier, \[
		\det_\mathcal{C}(e_1, \ldots, e_n) = \lambda \det_\mathcal{B}(e_1, \ldots, e_n) = \lambda.
	\]
\end{prv}

\begin{rmk}[notation]
	Avec les notations précédentes, on note $\det_\mathcal{B}(\mathcal{C})$ au lieu de $\det_\mathcal{B}(u_1, \ldots, u_n)$.
\end{rmk}

\begin{crlr}
	Avec les notations précédentes, $\det_\mathcal{B}(\mathcal{C}) \neq 0$ et \[
		\det_\mathcal{B}(\mathcal{C}) = \left( \det_\mathcal{C}(\mathcal{B}) \right)^{-1}
	\]
\end{crlr}

\begin{prv}
	On sait que \[
		\forall (v_1, \ldots, v_n) \in E^n, \det_\mathcal{C}(v_1, \ldots, v_n) = \det_\mathcal{B}(\mathcal{C}) \;\det_\mathcal{B}(v_1, \ldots, v_n).
	\] En particulier, \[
		1 = \det_\mathcal{C}(\mathcal{C}) = \det_\mathcal{C}(\mathcal{B}) \;\det_\mathcal{B}(\mathcal{C}).
	\]
\end{prv}

\begin{thm}
	Soit $\mathcal{B} = (e_1, \ldots, e_n)$ une base de $E$, $\mathcal{C} = (u_1, \ldots, u_n)$ une famille liée (i.e. $\mathcal{C}$ n'est pas libre). Alors $\det_\mathcal{B}(\mathcal{C}) = 0$.
\end{thm}

\begin{prv}
	Sans perte de généralité, on peut supposer que $u_n \in \Vect(u_1, \ldots, u_{n-1})$ : \[
		u_n = \sum_{k=1}^{n-1} \lambda_k u_k \text{ où } \lambda_1, \ldots, \lambda_{n-1} \in \mathbbm{K}.
	\]
	\begin{align*}
		\det_\mathcal{B}(u_1, \ldots, u_{n-1}, u_n) &= \sum_{k=1}^{n-1} \lambda_k \underbrace{\det_{\mathcal{B}}(u_1, \ldots, u_{n-1}, u_k)}_{=0} \\
			&= 0 \\
	\end{align*}
\end{prv}
