\part{Exponentielle complexe}

\begin{defn}
	Pour $z \in \C$, on pose \[
		\exp(z) = e^{\Re(z)}\times (\cos(\Im(z))+i\sin(\Im(z))
	\] Ainsi, si $z = a+ib$ avec $(a,b)\in \R^2$, \[
		\exp(z) = \exp(a+ib) = e^{a}\times (\cos(b)+i(\sin(b)) = e^{a}e^{ib}
	\] 
\end{defn}

\begin{prop}
	Soient $z_1, z_2 \in \C$. \[
		\exp(z_1+z_2) = \exp(z_1)\times \exp(z_2)
	\] 
\end{prop}

\begin{prv}
	On pose $\begin{cases}
		z_1 = a+ib\\
		z_2 = c+id
	\end{cases}$ avec $(a,b,c,d)\in \R^4$ \\
	\begin{align*}
		\exp(z_1)\times \exp(z_2) &= e^{a}\times e^{ib}\times e^{c}\times e^{id} \\
		&= e^{a+c}e^{i(b+d)} \\
		&= \exp(z_1+z_2) \\
	\end{align*}
\end{prv}

\begin{rmk}
	[Notation]
	On écrit $e^{z}$ à la place de $\exp(z)$ pour $z \in \C$.
\end{rmk}

\begin{prop}
	\begin{minipage}
		{\linewidth}
		\begin{wrapfigure}
			{r}{0.5\linewidth}
			\centering
			\begin{asy}[width=8cm]
				import graph;
				size(10cm);
				defaultpen(fontsize(8pt));

				pair A = (-6.5, 0);
				pair B = (+6.5, 0);

				real w = 5;
				pair z = (3,1.4);

				// input axes
				draw((A - (w, 0)) -- (A + (w, 0)), Arrow);
				draw((A - (0, w)) -- (A + (0, w)), Arrow);

				// output axes
				draw((B - (w, 0)) -- (B + (w, 0)), Arrow);
				draw((B - (0, w)) -- (B + (0, w)), Arrow);

				// input lines & point
				dot("$z$", A + z, fontsize(11pt), align=NW);
				draw(A+(-w, z.y)--A+(w, z.y), red); label("$y=\Im(z)$", A+(-w, z.y), red, align=W);
				draw(A+(z.x, -w)--A+(z.x, w), blue); label("$x=\Re(z)$", A+(z.x, -w), blue, align=S);

				// output line/circle & point
				pair z2 = expi(z.y / 2) * length(z) * 0.6;
				draw(circle(B, length(z) * 0.6), blue);
				draw(B--B+4*sqrt(2)*unit(z2), red);
				label("$e^{\Re(z)}$", B - (length(z) * 0.6, 0), blue, align=SE);
				draw(B-(length(z2), 0)--B, blue, Arrows(TeXHead));
				draw(arc(B, length(z2) + 1, 0, degrees(z2)), red, Arrow(TeXHead));
				label("$\Im(z)$", B + unit(z2) * length(z2) + 1, red, align=E);
				dot("$e^z$", B + z2, fontsize(11pt), align=N);

				guide arr = A - (0, -w/2) .. (0, w) .. B - (0, -w/2);
				real l = arctime(arr, arclength(arr));
				draw(subpath(arr, 0.3l, 0.45l));
				draw(subpath(arr, 0.55l, 0.7l), Arrow(TeXHead));
				label("$e^z$", point(arr, 0.5l));
			\end{asy}
		\end{wrapfigure}
		\[
			\forall z\in \C, \begin{cases}
				\left| e^z \right| = e^{\Re(z)}\\
				\arg(e^{z}) \equiv \Im(z) \mod{2\pi}
			\end{cases}
		\] 
	\end{minipage}
\end{prop}

\begin{rmk}
	$\exp: \C \to \C$ n'est pas bijective:
	\begin{itemize}
		\item $\begin{cases}
				\exp(0) = \exp(2i\pi) = 1\\
				0 \neq 2i\pi
			\end{cases}$ \\
		\item 0 n'a pas d'antécédant
	\end{itemize}
	Il n'y a donc pas de logarithme complexe.
\end{rmk}
