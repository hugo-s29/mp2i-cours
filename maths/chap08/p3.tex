\part{Relations binaires}

\begin{defn}
	Soit $E$ un ensemble. Un \underline{relation (binaire)} sur $E$ est un prédicat définit sur $E^2$.
	\index{relation (binaire)}
\end{defn}

\begin{exm}
	\begin{enumerate}
		\item Avec $E = \C$, $=$ est une relation binaire,
		\item Avec $E = \R$, $\le$ est une relation binaire,
		\item Avec $E$ l'humanité et la relation binaire $\wedge$ : \[
				x\wedge y \iff x \text{ et } y \text{ ont la même mère}.
			\]
	\end{enumerate}
\end{exm}

\begin{defn}
	Soit $E$ un ensemble, $\diamond$ une relation sur $E$. On dit que $\diamond$ est un \underline{relation d'équivalence} si
	\begin{enumerate}
		\item $\forall x \in E, x \diamond x$, \hfill (\underline{réflectivité})
		\item $\forall x,y, \in E, x\diamond y \implies y \diamond x$, \hfill (\underline{symétrie})
		\item $\forall x,y,z \in E,\quad\begin{rcases*}
				x \diamond y\\
				y \diamond z
			\end{rcases*} \implies x \diamond z$ \hfill (\underline{transitivité})
	\end{enumerate}
	\index{relation d'équivalence}
	\index{réflectivité (relation)}
	\index{symétrie (relation)}
	\index{transitivité (relation)}
\end{defn}

\begin{exm}
	Avec $E = \Z$ et \[
		x \diamond y \iff x \equiv y\mod 3
	\] ``$\diamond$'' est une relation d'équivalence.
\end{exm}

\begin{rmk}
	Le but d'une relation d'équivalence est d'identifier des objets différents.

	\begin{figure}[H]
		\centering
		\begin{subfigure}{4cm}
			\begin{asy}
				size(4cm);
				real eps = 0.1;
				draw((-2,0)--(2,0), dashed);

				// outline
				draw((-2-eps,-eps)--(-2,eps)--(-2+eps,-eps),white+2);
				draw((2-eps,eps)--(2,-eps)--(2+eps,eps),white+2);
				filldraw(circle((0,0), 1.5eps), white, white+2);

				draw((-2-eps,-eps)--(-2,eps)--(-2+eps,-eps));
				draw((2-eps,eps)--(2,-eps)--(2+eps,eps));
				
				draw(box((-2,-1),(2,1)));
				label("\ding{34}", (0,-0.022));

				guide p1 = (0,-1.5)..(-2,-1.5)..(-2,-0.5)..(0,-0.5)..(2,-0.5)..(2,1.5)..(0,1.5)..(-2,1.5)..(-2,0.5)..(0,0.5)..(2,0.5)..(2,-1.5)..cycle;
				guide p2 = (0,0.5)..(2,0.5)..(2,-1.5)..(0,-1.5)..(-2,-1.5)..(-2,-0.5)..(0,-0.5)..(2,-0.5)..(2,1.5)..(0,1.5)..(-2,1.5)..(-2,0.5)..cycle;
				guide p3 = (0,1.5)..(-2,1.5)..(-2,0.5)..(0,0.5)..(2,0.5)..(2,-1.5)..(0,-1.5)..(-2,-1.5)..(-2,-0.5)..(0,-0.5)..(2,-0.5)..(2,1.5)..cycle;
				guide p4 = (0,-0.5)..(2,-0.5)..(2,1.5)..(0,1.5)..(-2,1.5)..(-2,0.5)..(0,0.5)..(2,0.5)..(2,-1.5)..(0,-1.5)..(-2,-1.5)..(-2,-0.5)..cycle;
				draw(p1, magenta, Arrow(TeXHead));
				draw(p2, magenta, Arrow(TeXHead));
				draw(p3, magenta, Arrow(TeXHead));
				draw(p4, magenta, Arrow(TeXHead));
			\end{asy}
		\end{subfigure}
			$\qquad\qquad\qquad$
		\begin{subfigure}{4cm}
			\begin{asy}
				import patterns;
				add("hatch",hatch(2mm,magenta));

				size(4cm);
				real eps = 0.1;
				real a = 1/3;

				draw((-2,-a)--(2,-a), dashed);
				draw((-2,a)--(2,a), dashed);
				fill(box((-2,-a),(2,a)), pattern("hatch"));

				// outline
				draw((-2-eps,-eps)--(-2,eps)--(-2+eps,-eps),white+2);
				draw((2-eps,eps)--(2,-eps)--(2+eps,eps),white+2);
				filldraw(circle((0,-a), 1.5eps), white, white+2);
				filldraw(circle((0,a), 1.5eps), white, white+2);
				label("\ding{34}", (0,-0.022-a), white+fontsize(14pt));
				label("\ding{34}", (0,-0.022+a), white+fontsize(14pt));

				draw((-2-eps,-eps)--(-2,eps)--(-2+eps,-eps));
				draw((2-eps,eps)--(2,-eps)--(2+eps,eps));
				
				draw(box((-2,-1),(2,1)));
				label("\ding{34}", (0,-0.022-a));
				label("\ding{34}", (0,-0.022+a));
			\end{asy}
		\end{subfigure}
	\end{figure}
\end{rmk}

\begin{defn}
	Soit $E$ un ensemble et $\diamond$ une relation d'équivalence sur $E$. Soit $x \in E$. La \underline{classe de $x$ (modulo $\diamond$)} est \[
		\Cl_{\diamond}(x) = \Cl(x) = \overline{x} = \{y \in E  \mid y \diamond x\}.
	\]
	\index{classe d'équivalence (relation)}
\end{defn}

\begin{exm}
	\begin{enumerate}
		\item Avec $E = \C$ et $\diamond = ``="$, \[
				\forall z \in \C, \overline{z} = \Cl(z) = \{z\}.
			\]
		\item Avec $E = \Z$ et $\diamond = \text{congruence modulo } 5$, on a 
			\begin{align*}
				&\overline{0} = \{5k \mid k \in \Z\}\qquad\qquad\qquad
				&\overline{1} = \{5k + 1 \mid k \in \Z\}\\
				&\overline{2} = \{5k + 2 \mid k \in \Z\}
				&\overline{3} = \{5k + 3 \mid k \in \Z\}\\
				&\overline{4} = \{5k + 4 \mid k \in \Z\}
				&\overline{5} = \overline{0}
			\end{align*}

			On constate que \[
				x \equiv y \mod 5 \iff \overline{x} = \overline{y}.
			\]
	\end{enumerate}
\end{exm}

\begin{prop}
	Soit $E$ un ensemble muni d'une relation d'équivalence $\diamond$. Alors \[
		\forall x,y \in E, x\diamond y \iff \overline{x} = \overline{y}.
	\]
\end{prop}

\begin{prv}
	Soient $x,y \in E$.
	\begin{itemize}
		\item On suppose $x\diamond y$. Soit $z \in \overline{x}$. On sait que $z\diamond x$ et $y\diamond x$. Par transitivité, on en déduit que $z \diamond y$ et donc $z \in \overline{y}$.
		\item Soit $z \in \overline{y}$, donc $y\diamond z$. Or $x \diamond y$.
			Comme $\diamond$ est symétrique, on a $y\diamond x$ et par transitivité, on a donc $z \diamond x$. Donc $z \in \overline{x}$.
		\item On suppose $\overline{x} = \overline{y}$. $\diamond$ réfléctive donc $x\diamond x$ et donc $x \in \overline{x} = \overline{y}$ donc $x \in \overline{y}$ et donc $x\diamond y$.
	\end{itemize}
\end{prv}

\begin{mdframed}
	~\\
	\centered{\underline{\scshape \LARGE Hors-Programme}}\vspace{4mm}

	\begin{defn}
		Soit $E$ un ensemble et $\diamond$ une relation d'équivalence.

		L'ensemble \[
			\{ \overline{x}  \mid x \in E\}  = \sfrac{E}{\diamond}
		\] est appelé \underline{quotient de $E$ modulo $\diamond$}.
		\index{quotient (relation, ensemble)}
	\end{defn}

	\begin{exm}
		\begin{enumerate}
			\item $E = \Z$ et $\diamond=$ congruence modulo 5 : \[
					\sfrac{E}{\diamond} = \left\{ \overline{0}, \overline{1},\overline{2},\overline{3},\overline{4} \right\} = \sfrac{\Z}{5\Z}
				\]
			\item {\itshape Construction de $\Q$}

				On suppose avoir déjà construit $\Z$ mais pas $\Q$ :
				on veut donc donner un définition de $\sfrac{p}{q}$ sans parler de division.

				On pose \[
					E = \Z\times \N^* = \{(p,q) \mid p \in \Z, q \in \N^*\}.
				\] Soit $\sim$ la relation définie par \[
					(p,q) \sim (p', q') \iff pq' = p'q
				\] Montrons que $\sim$ est une relation d'équivalence.
				\begin{itemize}
					\item Soient $(p,q) \in E$. $\sim$ est réfléctive car $(p,q)\sim (p,q) \iff pq = pq$.
					\item Soient $(p,q),(p',q') \in E$. On suppose $(p,q)\sim(p',q')$.
						\begin{align*}
							(p,q)\sim(p',q') \iff& pq' = p'q\\
							\iff& p'q = pq'\\
							\iff& (p',q')\sim(p,q)
						\end{align*}
						Donc $\sim$ est symétrique.
					\item Soient $(p,q), (p',q'), (p'', q'') \in E$. On suppose \[
							\begin{cases}
								(p,q)\sim(p',q')\\
								(p',q')\sim(p'',q'')\\
							\end{cases}
						\]

						On sait que \[
							(p,q)\sim(p'',q'') \iff pq'' = p'' q
						\] Or, \[
							\begin{cases}
								pq' = qp'\\
								p'q'' = p''q'
							\end{cases} \text{ donc }
							pq'p'q'' = p'q'p''q'
						\] Donc \[
							p' q' (p q'' - p'' q) = 0
						\] et donc \[
							p' = 0 \ou pq'' - p'' q = 0
						\]
						Si $p' = 0$, alors $\begin{cases}
							pq' = 0\\
							p'' q' = 0
						\end{cases}$ et donc $\begin{cases}
							p = 0\\
							p'' = 0
						\end{cases}$. On a donc \[
							pq'' = 0 = p'' q
						\]
						Si $p' \neq 0$, on a $pq'' - p''q = 0$ et donc \[
							pq'' = p''q
						\]

						On a donc $(p,q)\sim(p'',q'')$.
				\end{itemize}

				On pose $\Q = \sfrac{E}{\sim}$ et \[
					\forall (p,q) \in E,~\frac{p}{q} = \Cl\big((p,q)\big).
				\] Ainsi,
				\begin{align*}
					\frac{p}{q} = \frac{p'}{q'} \iff& \Cl\big((p,q)\big) = \Cl\big((p',q')\big)\\
					\iff& (p,q)\sim(p',q')\\
					\iff& pq' = p'q
				\end{align*}
			\item {\itshape Construction de $\Z$ à partir de $\N$}

				On pose $E = \N\times \N^*$ et $\sim$ la relation $(p,q)\sim(p',q') \iff p + q' = p' + q$.

				$\sim$ est une relation d'équivalence. On pose donc $\Z = \sfrac{\N}{\sim}$ et pour $n \in \N$, on définit $n$ par $\Cl\big((n,0)\big)$ et $-n$ par $\Cl\big((0,n)\big)$.
			\item {\itshape Constrcution de $\C$ à partir de $\R$}

				On pose $E$ l'ensemble des polynômes à coefficients réels ($E = \R[X]$) et $\diamond$ la relation d'équivalence \[
					P \diamond Q \iff P\equiv Q \mod{x^2 + 1}
				\]
				On pose $\C = \sfrac{E}{\diamond}$.

				\missingpart
		\end{enumerate}
	\end{exm}
\end{mdframed}

\begin{defn}
	Soit $E$ un ensemble et $(A_i)_{i\in I}$ une famille de parties de $E$.

	On dit que $(A_i)_{i\in I}$ est une \underline{partition} de $E$ si \[
		\begin{cases}
			E = \bigcup_{i \in I} A_i\\
			\forall i \neq j, A_i \cap A_j = \O
		\end{cases}
	\]

	On a donc \[
		\forall x \in E, \exists! i \in I, x \in A_i.
	\]
	\index{partition (ensemble)}
\end{defn}

\begin{prop}
	Soit $E$ un ensemble muni d'une relation d'équivalence $\diamond$. 
	Les classes d'équivalences de $E$ modulo $\diamond$ forment une partition de $E$.
\end{prop}

\begin{prv}
	\begin{itemize}
		\item Soit $x \in E$. On sait que $x \diamond x$ donc $\overline{x} \ni x$.
			On a montré $E \subset \bigcup_{y \in E} \overline{y}$.
		\item $\forall y \in E, \overline{y} \subset E$ donc $ E \supset \left( \bigcup_{y \in E} \overline{y} \right)$.
		\item Soit $x,y\in E$ tel que $\overline{x} \neq \overline{y}$. Montrons que $\overline{x} \cap \overline{y} = \O$. Soit $z \in \overline{x} \cap \overline{y}$. $z \in \overline{x}$ donc $z\diamond x$. De même, $z \in \overline{y}$ donc $z \diamond y$. Par transitivité, $x\diamond y$ et donc $\overline{x} = \overline{y}$ : une contradiction.
	\end{itemize}
\end{prv}

\begin{prop}
	Soit $E$ un ensemble et $(A_i)_{i\in I}$ une partition de $E$ telle que \[
		\forall i \in I, A_i \neq \O.
	\] Alors il existe une relation d'équivalence $\diamond$ telle que pour tout $i \in I$, $A_i$ est une classe d'équivalence modulo $\diamond$.
\end{prop}

\begin{prv}
	Soit $\diamond$ la relation définie par \[
		x \diamond y \iff \exists i \in I, \begin{cases}
			x \in A_i\\
			y \in A_i
		\end{cases}
	\]
	\begin{itemize}
		\item Soit $x \in E$. Comme $E = \bigcup_{i \in I} A_i$, il existe $i \in I$ tel que $x \in A_i$ donc $x \diamond x$.
		\item Soient $x,y \in E$. On suppose $x \diamond y$. Soit $i \in I$ tel que $\begin{cases}
				x \in A_i\\
				y \in A_i
			\end{cases}$ donc $\begin{cases}
				y \in A_i\\
				x \in A_i
			\end{cases}$ et donc $y \diamond x$.
		\item Soit $x,y,z \in E$. On suppose $x \diamond y$ et $y \diamond z$.\\
			Soit $i \in I$ tel que $\begin{cases}
				x \in A_i\\y\in A_i.
			\end{cases}$\\
			Soit $j \in I$ tel que $\begin{cases}
				y \in A_j\\
				z \in A_j.
			\end{cases}$

			On a donc $y \in A_i \cap A_j$. Si $i \neq j$, alors $y \in \O$ : une contradiction. Donc $i = j$ et donc $\begin{cases}
				x \in A_i\\
				z \in A_i
			\end{cases}$. On en déduit que $x\diamond z$.
	\end{itemize}

	Ainsi $\diamond$ est une relation d'équivalence.
	\begin{itemize}
		\item Soit $i \in I$ et soit $x \in A_i \neq \O$. \[
			\overline{x} = \{y \in E  \mid y \diamond x\} = \{y \in E \mid y\in A_i\} = A_i.
		\]
	\end{itemize}
\end{prv}

\begin{defn}
	Soit $E$ un ensemble et $\diamond$. On dit que $\diamond$ est une \underline{relation d'ordre} sur $E$ si
	\begin{enumerate}
		\item $\diamond$ est réfléctive ($\forall x \in E, x \diamond x$),
		\item $\diamond$ est \underline{anti-symétrique} : \[
				\forall x,y \in E, \quad \begin{rcases*}
					x \diamond y\\
					y\diamond x
				\end{rcases*} \implies x = y,
			\]
		\item $\diamond$ est transitive $\Big(\forall x,y,z \in E, (x \diamond y \et y \diamond z) \implies x \diamond z\Big)$.
	\end{enumerate}

	En général, la relation $\diamond$ est notée $\le$ ou $\preceq$. On dit aussi que $(E, \diamond)$ est un \underline{ensemble ordonné}.
	\index{relation d'ordre}
	\index{anti symétrie (relation)}
	\index{ensemble ordonné}
\end{defn}

\begin{exm}
	\begin{enumerate}
		\item $(\R, \le)$ est un ensemble ordonné.
		\item $\big(\mathcal{P}(E), \subset\!\big)$ est un ensemble ordonné.
		\item $(\N, \mid)$ est un ensemble ordonné.
		\item $(\text{\itshape MP2I}, \preceq)$ avec \[x \preceq y \iff \text{ note de }x \le \text{note de } y\] n'est un ensemble ordonné car $\preceq$ n'est pas anti symétrique.
		\item $E = \N^2$ et $\preceq$ définie par \[
					(x,y)\preceq(x',y') \iff x < x' \ou \begin{cases}
						x = x'\\
						y \le y'
					\end{cases}
			\] $(E, \preceq)$ est un ensemble ordonné.
	\end{enumerate}
\end{exm}

\begin{defn}
	Soit $(E,\le)$ un ensemble ordonné. Soient $x,y \in E$. On dit que $x$ et $y$ sont \underline{comparables} si \[
		x \le y \ou y \le x.
	\] On dit que $\le$ est un \underline{ordre total} si tous les éléments de $E$ sont comparables 2 à 2.
	\index{ordre total (relation d'ordre)}
\end{defn}

\begin{exm}
	\begin{itemize}
		\item $(\R, \le)$ est totalement ordonné
		\item $\big(\mathcal{P}(E), \subset\!\big)$ n'est pas totalement ordonné en général : 

			Soient $a,b \in E$ avec $a \neq b$. $\{a\}$ et $\{b\}$ ne sont pas comparables.
		\item $(\N, \mid)$ n'est pas totalement ordonné :

			$2\nmid 5$ et $5 \nmid 2$ donc $2$ et $5$ ne sont pas comparables.
	\end{itemize}
\end{exm}

\begin{defn}
	Soit $(E, \le)$ un ensemble ordonné, $A \in \mathcal{P}(E)$ et $M \in E$. On dit que \underline{$A$ est majorée par $M$}, que \underline{$M$ majore $A$} ou que \underline{$M$ est un majorant de $A$} si \[
		\forall a \in A, a \le M.
	\] Soit $m \in E$. On dit que \underline{$A$ est minorée par $m$}, que \underline{$m$ minore $A$} ou que \underline{$m$ est un minorant de $A$} si \[
		\forall a \in A, m \le a.
	\]
	\index{minorant (ensemble)}
	\index{minorer (ensemble)}
	\index{majorant (ensemble)}
	\index{majorer (ensemble)}
\end{defn}

\missingpart

\begin{exm}
	\begin{enumerate}
		\item $E = \R$ muni de $\le$ et $A = [2,5]$.

			On sait que $\sup A = 5$ car \[
				\forall x \in A, x \le 5
			\] et \[
				\forall y \le 5,\quad 5 > \frac{y+5}{2} > y
			\] donc $y$ ne majore pas $A$.
		\item $E = \R$ avec $\le $ et $A = ]2,5[$. $A\not\ni\sup A = 5$ par le même raisonnement.
		\item $E = \N^*$ avec $\mid$ et $A = \{p,q\}$ avec $p \neq q \in E$. $\sup A = \PPCM(p,q) = p \vee q$ (c.f. chapitre 10 arithmétique)
		\item $\mathcal{P}(E)$ avec $\subset$ et $A = \{P,Q\}$ avec $P, Q \in \mathcal{P}(E)$ et $P \neq Q$. $\sup A = P \cup Q$.
		\item $E = \{0,1\} \times \Z$ muni de $\le$ défini par \[
				(x_1,y_1) \le (x_2,y_2) \iff x_1 < x_2 \ou \begin{cases}
					x_1 = x_2\\
					y_1 \le y_2
				\end{cases}
		\] et $A = \{0\} \times \Z$. $(x,y)$ majore $A \iff x = 1$ donc $A$ est majorée mais n'a pas de borne supérieure.
	\end{enumerate}
\end{exm}

\begin{prop}
	Soit $(E, \le)$ un ensemble ordonné et $A \in \mathcal{P}(E)$. Si $A$ a une borne supérieure, alors celle-ci est unique. On la note $\sup A$.
\end{prop}

\begin{prv}
	Soit $M_1$ et $M_2$ deux bornes supérieures de $A$.

	Donc $M_2$ majore $A$. Comme $M_1$ est une borne supérieure de $A$, on a $M_1\le M_2$.

	De même, on en déduit que $M_2 \le M_1$.

	Comme $\le$ est antisymétrique, $M_1 = M_2$.
\end{prv}

