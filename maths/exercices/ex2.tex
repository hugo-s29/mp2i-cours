\documentclass[a4paper]{report}

\usepackage[utf8]{inputenc}
\usepackage[T1]{fontenc}
\usepackage{textcomp}
\usepackage[french]{babel}
\usepackage{amsmath, amssymb}
\usepackage{bbm}
\usepackage{amsthm}
\usepackage{tikz}
\usepackage{pgfplots}
\usepackage{mathtools}
\usepackage{tkz-tab}
\usepackage[inline]{asymptote}
\usepackage{frcursive}
\usepackage{verbatim}
\usepackage{moresize}
\usepackage{algorithm}
\usepackage{algpseudocode}
\usepackage{calligra}
\usepackage{diagbox}
\usepackage{centernot}
\usepackage{multicol}
\usepackage{nicematrix}
\usepackage{stmaryrd}
\usepackage{setspace}
\usepackage{chngpage}
\usepackage{cancel}
\usepackage{esvect}
\usepackage{wrapfig}
\usepackage{floatflt}
\usepackage{calligra}
\usepackage[cuteinductors,european,straightvoltages,europeanresistors]{circuitikz}

\usetikzlibrary{babel}
\usetikzlibrary{tikzmark,calc,fit,arrows}

\newif\ifsimple
\simplefalse
\let\underlin\underline

\usepackage{graphicx}
\newcommand\longvdots[1]{\raisebox{1em}{\rotatebox{-90}{\hbox to #1 {\dotfill}}}}

\pgfplotsset{compat=1.17}
\let\vec\vv

\definecolor{green}{HTML}{60A917}

\everymath{\displaystyle}
\let\textstyle\displaystyle
\let\scriptstyle\displaystyle
\let\scriptscriptstyle\displaystyle

\def\asydir{asy}

% figure support
\usepackage{import}
\usepackage{xifthen}
\pdfminorversion=7
\usepackage{pdfpages}
\usepackage{transparent}
\newcommand{\incfig}[1]{%
	\def\svgwidth{\columnwidth}
	\import{./figures/}{#1.pdf_tex}
}


\usepackage{calrsfs}
\usepackage{mathrsfs}
\usepackage{stmaryrd}
\usepackage{float}

\setlength{\parindent}{0em}
\setlength{\parskip}{0em}

\let\oldemptyset\emptyset
\let\emptyset\varnothing

\let\ge\geqslant
\let\le\leqslant

\newcommand{\C}{\mathbbm{C}}
\newcommand{\R}{\mathbbm{R}}
\newcommand{\Z}{\mathbbm{Z}}
\newcommand{\N}{\mathbbm{N}}
\newcommand{\Q}{\mathbbm{Q}}
\renewcommand{\O}{\emptyset}

\renewcommand\Re{\expandafter\mathfrak{Re}}
\renewcommand\Im{\expandafter\mathfrak{Im}}

\renewcommand{\thepart}{\Roman{part}} 

\DeclareMathOperator{\Arctan}{Arctan}
\DeclareMathOperator{\Card}{Card}
\DeclareMathOperator{\Ker}{Ker}
\DeclareMathOperator{\Aut}{Aut}
\DeclareMathOperator{\id}{id}
\DeclareMathOperator{\rg}{rg}
\DeclareMathOperator{\argmax}{argmax}
\DeclareMathOperator{\argmin}{argmin}
\DeclareMathOperator{\Vect}{Vect}
\DeclareMathOperator{\cotan}{cotan}
\DeclareMathOperator*{\po}{\text{\cursive o}}
\DeclareMathOperator*{\dom}{dom}

\pdfsuppresswarningpagegroup=1
\reversemarginpar

\newcommand{\defblock}[5]{
	\newenvironment{#1}[1][{}]
		{
			\if##1\empty\else
				{#3 ##1 #5}
			\fi
			\marginpar{#2}
			~\\
		}
		{
			#4
			\vspace{8mm}
		}
}

\newcommand{\defoptblock}[6]{
	\AtBeginDocument{
		\ifsimple
			\newenvironment{#1}[1][{}]
				{
					\expandafter\comment
				}
				{
					\expandafter\endcomment
					\vspace{-8mm}
					#6
					\vspace{8mm}
				}
		\else
			\defblock{#1}{#2}{#3}{#4}{#5}
		\fi
	}
}

\defblock{defn}{\bf Definition}{\bf\hfill}{}{\hfill}
\defblock{prop}{\bf Proposition}{\bf\hfill}{}{\hfill}
\defblock{crlr}{\bf Corollaire}{\bf\hfill}{}{\hfill}
\defblock{prop-defn}{\bf Proposition\\Définition}{\bf\hfill}{}{\hfill}
\defblock{thm}{\bf Théorème}{\bf\hfill}{}{\hfill}
\defblock{lem}{\bf Lemme}{\bf\hfill}{}{\hfill}

\defoptblock{exm}{Exemple}{}{}{}{}
\defoptblock{exo}{Exemple}{}{}{}{}

\defblock{rmk}{\it Remarque}{\it}{}{}
\defoptblock{prv}{\it Preuve}{\it}{\qed}{\hfill}{\hfill$\blacksquare$}

% \AtBeginDocument{
	% \newenvironment{cblk}[3][{}][{}]
		% {
			% \if#1\empty\else
				% {#1}
			% \fi
			% \marginpar{#3}
			% ~\\
		% }
		% {
			% #2
			% \vspace{8mm}
		% }
% }

\makeatother
\usepackage{fancyhdr}
\pagestyle{fancy}

\fancyhead[R]{}
\fancyhead[L]{\thepart}
\fancyhead[C]{\parttitle}

\fancyfoot[R]{\thepage}
\fancyfoot[L]{}
\fancyfoot[C]{}

\newcommand*\parttitle{}
\let\origpart\part
\renewcommand*{\part}[2][]{%
   \ifx\\#1\\% optional argument not present?
      \origpart{#2}%
      \renewcommand*\parttitle{#2}%
   \else
      \origpart[#1]{#2}%
      \renewcommand*\parttitle{#1}%
   \fi
}

\makeatletter

\newcommand{\tendsto}[1]{\xrightarrow[#1]{}}
\newcommand{\danger}{{\large\fontencoding{U}\fontfamily{futs}\selectfont\char 66\relax}\;}
\newcommand{\ex}{\fbox{ex}\;}
\renewcommand{\mod}[1]{~\left[ #1 \right]}

\newcommand{\vrt}[1]{\rotatebox{90}{$#1$}}

\DeclareMathOperator{\ou}{\text{ ou }}
\DeclareMathOperator{\et}{\text{ et }}
\DeclareMathOperator{\si}{\text{ si }}
\DeclareMathOperator{\non}{\text{ non }}

\renewcommand{\title}[2]{
	\AtBeginDocument{
		\begin{titlepage}
			\begin{center}
				\vspace{10cm}
				{\Large \sc Chapitre #1}\\
				\vspace{1cm}
				{\HUGE \cursive #2}\\
				\vfill
				Hugo {\sc Salou} MP2I\\
				{\ssmall Dernière mise à jour le \@date }
			\end{center}
		\end{titlepage}
	}
}

\usepackage{pdfpages}

\setlength{\parindent}{0em}
\setlength{\parskip}{0em}
\renewcommand{\thesection}{}
\renewcommand{\thesubsection}{}
\let\cal\mathcal
\let\mathcal\mmathcal
\let\overline\bar

\usetikzlibrary{trees}

\tikzstyle{level 1}=[level distance=3.5cm, sibling distance=3.5cm]
\tikzstyle{level 2}=[level distance=3.5cm, sibling distance=2cm]

\tikzstyle{bag} = [text width=4em, text centered]
\tikzstyle{end} = [minimum width=3pt, inner sep=0pt]

\begin{document}
	\begin{titlepage}
		\begin{center}
			~
			\vfill
			{\HUGE Sujet de mathématiques du~baccalauréat général}\\
			\vspace{1cm}
			{\LARGE Mercredi 11 mai 2022}\\
			\vfill
			\vfill
			\vfill
			Hugo {\sc Salou} MP2I\\
			{\ssmall Dernière mise à jour le \@date }
		\end{center}
	\end{titlepage}
	\part{Sujet}
	\includepdf[pages={2-5}]{ex2-sujet.pdf}
	\part{Réponses}
	\section{Exercice 1 \hfill (7 points)}
	\subsection{Partie A : Étude du premier protocole}
	\begin{enumerate}
		\item
			\begin{enumerate}
				\item On a $f = u \times \exp(v)$ avec $\forall t \in [0,10], \begin{cases}
						u(t) = 3t\\
						v(t) = -\frac{1}{2}t + 1
					\end{cases}$.

					De cela, on en déduit que $\forall t \in [0,10], \begin{cases}
						u'(t) = 3\\
						v'(t) = -\frac{1}{2}
					\end{cases}$.

					Or, $f' = u' \exp(v) + uv' \exp(v)$. D'où, \[
						\forall t \in [0, 10],\,f'(t) = 3e^{-\frac{1}{2}t + 1} - \frac{3}{2}te^{-\frac{1}{2}t+1}
					.\]
				\item On cherche pour quelles valeurs de $t$,  $f'(t)$ s'annule.
					\begin{align*}
						f'(t) = 0 \iff& \left( 3 - \frac{3}{2}t \right)e^{-\frac{1}{2}t + 1} = 0\\
						\iff& 3 - \frac{3}{2}t = 0 \text{ car } \forall u \in \R,\,e^{u} > 0\\
						\iff& t = 2
					\end{align*}
					Donc, $f'$ s'annule en $2$.

					Comme dit précédemment,  $\forall u \in \R, \,e^{u} > 0$. Donc, le signe de $f'$ est celui de $t \mapsto 3 - \frac{3}{2}t$.
					\[
						f'(t) > 0 \iff 3 - \frac{3}{2}t > 0 \iff t > 2
					.\]
					On calcule les valeurs de $f$ pour 0, 2 et 10 : \[
						\begin{cases}
							f(0) = 0\\
							f(2) = 6\\
							f(10) = \frac{30}{e^4}
						\end{cases}
					.\]
					D'où,
					\begin{center}
						\begin{tikzpicture}
							\tkzTabInit[espcl=1.5]{$x$/0.7,$f'(x)$/1,$f$/2}{$0$,$2$,$10$}
							\tkzTabLine{,-,z,+,}
							\tkzTabVar{-/0,+/6,-/$\frac{30}{e^4}$}
						\end{tikzpicture}
					\end{center}
				\item Selon cette modélisation, la quantité de médicament est maximale au bout de 2 jours : le patient présente 5 mg de médicament dans le sang.
			\end{enumerate}
		\item
			\begin{enumerate}
				\item Sur $[0,2]$, la fonction $f$ est strictement croissante, continue et on a $f(0) \le 5 \le f(2)$. Donc, d'après le corolaire du théorème des valeurs intermédiaires, $f(t) = 5$ admet une unique solution $\alpha \in [0,2]$. On a $\alpha \simeq 1,\!02$ jours (?).
				\item On a $\forall t \in [\alpha, \beta],\,f(t) \ge 5 \text{ mg}$. La durée efficace du médicament est donc $\beta - \alpha \simeq 2.44 \text{ jours }$. On en déduit que le médicament est efficace pendant environ $3\,514$ minutes.
			\end{enumerate}
	\end{enumerate}
	\subsection{Partie B : Étude du second protocole}
	\begin{enumerate}
		\item On a $u_1 = \left( 1 - \frac{30}{100} \right) u_0 + 1,\!8 = 0,\!7u_0 + 1,\!8 = 1,\!4 + 1,\!8 = 3,\!2$ mg.
		\item On ne conserve que $70$ \% du médicament d'une heure à l'autre et on nous injecte $1,\!8$ mg. D'où \[
				u_{n+1} = 0,\!7u_n + 1,\!8
			.\]
		\item
			\begin{enumerate}
				\item On pose, pour tout entier $n$, \[
						P(n): ``u_n \le u_{n+1} < 6"
					.\]
					\begin{itemize}
						\item Pour $n = 0$, on a bien $u_0 \le u_1 < 6$ car $u_0 = 2$ mg et $u_1 = 3,\!2$ mg.
						\item Soit $n \in \N$. On suppose $P(n)$ vraie. Montrons que $P(n+1)$ est vraie également.
							Comme $u_{n+1} < 6$, alors, $u_{n+2} = 0,\!7 u_n + 1,\!8 < 0,\!7 \times 6 + 1,\!8 = 6$.
							Comme $u_n \le u_{n+1}$, on a $0,\!7u_n \le 0,\!7u_{n+1}$ et donc \[
								\underbrace{0,\!7u_n + 1,\!8}_{u_{n+1}} \le \underbrace{0,\!7u_{n+1} + 1,\!8}_{u_{n+2}}
							.\] On a bien vérifié $u_{n+1} \le u_{n+2} < 6$.
					\end{itemize}
					On en déduit que \[
						\forall n \in \N,\,, u_n \le u_{n+1} < 6
					.\]
				\item Comme $(u_n)_{n \in \N}$ est croissante et majorée par 6, d'après le théorème de la convergence des suites monotones, $(u_n)$ converge vers $\ell$.
				\item On cherche un point fixe de la fonction $x\mapsto 0,\!7x + 1,\!8$.
					\[
						\frac{7}{10}x + \frac{18}{10} = x \iff \frac{3}{10}x = \frac{18}{10} \iff x = \frac{18}{3} = 6
					.\]
					On en déduit que, comme $(u_n)$ converge, elle converge vers 6. Le quantité de médicament va s'approcher progressivement de 6 (mais ne l'atteindra jamais).
			\end{enumerate}
		\item
			\begin{enumerate}
				\item Soit $n \in \N$. Comme $u_n < 6$, on a $v_n = 6 - v_n > 0$.
					\begin{align*}
						\frac{v_{n+1}}{v_n} &= \frac{6 - u_{n+1}}{6 - u_n}\\
						&= \frac{4,\!2 - 0,\!7u_n}{6 - u_n} \\
						&= 0,\!7 \times \frac{6 - u_n}{6 - u_n} \\
						&= 0,\!7 \\
					\end{align*}
					Donc, $(v_n)_{n\in\N}$ est une suite géométrique de raison $0,\!7$.
				\item On a $v_0 = 6 - u_0 = 4$. D'où \[
						\forall n \in \N,\,v_n = 4 \times (0,\!7)^n
					.\] On en déduit que, comme $\forall n \in \N,\, u_n = 6 - v_n$, \[
						\forall n \in \N,\,u_n = 6 - 4 \times (0,\!7)^n
					.\]
				\item On cherche $n \in \N$ tel que
					\begin{align*}
						u_n \ge 5,\!5 \iff& 6 - 4 \times (0,\!7)^n = 5,\!5\\
						\iff& 4 \times (0,\!7)^n = 0,\!5\\
						\iff& (0,\!7)^n = \frac{1}{8}\\
						\iff& n \ln(0,\!7) = \ln \frac{1}{8}\\
						\iff& n = \frac{-\ln 8}{\ln 0,\!7} \simeq 5.83 \text{ jours}
					\end{align*}
					On peut donc arrêter les injections après 6 jours.
			\end{enumerate}
	\end{enumerate}
	\section{Exercice 2 \hfill (7 points)}
	\begin{enumerate}
		\item
			\begin{enumerate}
				\item On procède par identification : la droite ayant pour vecteur directeur $\begin{pmatrix}
						u_x\\u_y\\u_z
					\end{pmatrix}$ et passant par le point de coordonnées $(M_x, M_y, M_z)$ est donnée par le système \[
						\begin{cases}
							x = M_x + u_x\,t\\
							y = M_y + u_y\,t\\
							z = M_z + u_z\,t
						\end{cases} \text{ avec } t \in \R
					.\] On en déduit donc que $\vec{u}\begin{pmatrix}
						2\\-1\\2
					\end{pmatrix}$.
				\item On a, d'après la première équation du système, $-1 = 1 + 2t \iff t = -1$. En replaçant $t$ par $-1$ dans les autres équations, on obtient bien les coordonnées se de $B$ selon $\vec{\jmath}$ et $\vec{k}$ : \[
						\begin{cases}
							y = 2 - (-1) = 3\\
							z = 2 - 2 = 0
						\end{cases}
					.\] Donc, $B$ appartient bien à la droite $\mathcal{D}$.
				\item On a $\vec{AB}\begin{pmatrix}
						0\\
						2\\
						-3
					\end{pmatrix}$ et donc \[
						\vec{AB} \cdot \vec{u} = 0 \times 2 - 1 \times  2 - 3 \times 2 = -8
					.\]
			\end{enumerate}
		\item
			\begin{enumerate}
				\item Comme $\mathcal{P}$ est orthogonal à $\mathcal{D}$, un vecteur normal de $\mathcal{P}$ est $\vec{u}$. On sait donc que l'équation de $\mathcal{P}$ est de la forme \[
						\mathcal{P}:\qquad 2x -y +2z = d
					.\] Déterminons la valeur de $d$. Comme $A \in \mathcal{P}$, les coordonnées de $A$ vérifient l'équation du plan, d'où \[
						d = 2\times -1 - 1 + 2 \times 3 = 3
					.\] On en déduit que $\mathcal{P}$ a pour équation cartésienne \[
						\mathcal{P} : \qquad 2x - y + 2z - 3 = 0
					.\]
				\item En utilisant les expressions de $x$, $y$ et $z$, provenant du système $\mathcal{D}$, dans l'équation de $\mathcal{P}$, on a \[
						2(1+2t) - 2 + t + 2(2 + 2t) - 3 = 0
					.\] On résout pour cette valeur de $t$ :
					\[
						\cancel2 + 4t - \cancel2 + t + 4 + 4t - 3 = 0 \iff 9t = - 1 \iff t = -\frac{1}{9}
					.\] On en déduit les coordonnées de $H$ : \[
						\begin{cases}
							x = 1 - \frac{2}{9} = \frac{7}{9}\\
							y = 2  + \frac{1}{9} = \frac{19}{9}\\
							z = 2 - \frac{2}{9} = \frac{16}{9}
						\end{cases}
					.\] On a donc $H\left( \frac{7}{9},\frac{19}{9},\frac{16}{9} \right)$.
				\item
					\begin{align*}
						AH^2 &= \left( \frac{7}{9}+1 \right)^2 + \left( \frac{19}{9}-1 \right)^2 + \left( \frac{16}{9} -3 \right)^2\\
						&= \left( \frac{16}{9} \right)^2 + \left( \frac{10}{9} \right)^2 + \left( -\frac{11}{9} \right)^2 \\
						&= \frac{256 + 100 + 121}{9^2} \\
						&= \frac{477}{9^2} \\
						&= \left( \frac{\sqrt{53}}{3} \right)^2 \\
					\end{align*}
					Ainsi, $AH = \frac{\sqrt{53}}{3}$.
			\end{enumerate}
		\item
			\begin{enumerate}
				\item On sait que $H$ et $B$ sont tous les deux sur la droite. Donc, $\vec{HB}$ et $\vec{u}$ sont colinéaires. Il existe donc $k \in \R$ tel que $\vec{HB} = k\vec{u}$.
				\item On a $\vec{AB} \cdot  \vec{u} = \vec{AH} \cdot \vec{u} + \vec{HB} \cdot \vec{u} = \vec{HB} \cdot \vec{u}$ car $\vec{AH}$ et $\vec{u}$ sont colinéaires (donc $\vec{AB} \cdot  \vec{u} = 0$).

					Or, $\vec{HB} = k \vec{u}$. Donc, $\vec{AB} \cdot \vec{u} = k \vec{u} \cdot \vec{u}$. En divisant par $\vec{u}\cdot \vec{u} = \|\vec{u}\|^2$, on obtient \[
						k = \frac{\vec{AB} \cdot  \vec{u}}{\|\vec{u}\|^2}
					.\]
				\item On a $\vec{AB} \cdot \vec{u} = -8$, et $\|\vec{u}\|^2 = 2^2 + 1 + 2^2 = 9$. On en déduit donc la valeur de $k$ : en effet, on a \[
						k = \frac{\vec{AB} \cdot \vec{u}}{\|\vec{u}\|^2} = -\frac{8}{9}
					.\] Or, $\vec{HB} = k \vec{u}$, donc $\vec{OH} = -k \vec{u} + \vec{OB}$ et donc \[
						\vec{OH} = \frac{8}{9}\begin{pmatrix}
							2\\-1\\2
						\end{pmatrix} + \begin{pmatrix}
							-1\\3\\0
						\end{pmatrix} = \begin{pmatrix}
							\frac{16}{9} - 1\\
							3-\frac{8}{9}\\
							\frac{16}{9}
						\end{pmatrix} = \begin{pmatrix}
							\frac{7}{9}\\[2mm]
							\frac{19}{9}\\[2mm]
							\frac{16}{9}
						\end{pmatrix}
					.\]
			\end{enumerate}
			D'où $H\left( \frac{7}{9},\frac{19}{9},\frac{16}{9} \right)$.
		\item On a $V_{ABCH} = \frac{8}{9}$. Or, $V_{ABCH} = \frac{1}{3}\: A_{ACH}\: B\!\:\!H$. On calcule donc $B\!\;\!H$ :
			\[
				B\!\;\!H^2 = k^2 \|\vec{u}\|^2 = \frac{64}{81} (2^2 + 1 + 2^2) = \frac{8^2}{9}
			.\] On en déduit que $B\!\;\!H = \frac{8}{3}$. Ainsi, \[
				A_{ACH} = 3\times V_{ABCH}\times B\!\;\!H = 3\times \frac{8}{9}\times \frac{3}{8} = 1
			.\]
	\end{enumerate}
	\section{Exercice 3 \hfill (7 points)}
	\begin{enumerate}
		\item
			\begin{enumerate}
				\item On a $P(S) = \frac{25}{100}$.
				\item
					\begin{tikzpicture}[grow=right, sloped]
					\node[bag] {}
							child {
									node[bag] {$\overline{F}$}
											child {
													node[end, label=right:{$\overline{S}$}] {}
													edge from parent
											}
											child {
													node[end, label=right:{$S$}] {}
													edge from parent
											}
											edge from parent
											node[below]  {$\frac{48}{100}$}
							}
							child {
									node[bag] {$F$}
									child {
													node[end, label=right:{$\overline{S}$}] {}
													edge from parent
													node[below]  {$\frac{60}{100}$}
											}
											child {
													node[end, label=right: {$S$}] {}
													edge from parent
													node[above] {$\frac{40}{100}$}
											}
									edge from parent
											node[above] {$\frac{52}{100}$}
							};
					\end{tikzpicture}
				\item On cherche $P(F \cap S) = P_F(S) \times P(F) = \frac{40}{100}\times \frac{52}{100} = \frac{208}{1000} = 0,\!208$.
				\item On cherche $P_S(F) = \frac{P(F \cap S)}{P(S)} = \frac{\sfrac{208}{1000}}{\sfrac{25}{100}} = 0,\!052$.
				\item On a $P(S) = P_{\overline{F}}(S) \times P(\overline{F}) + P_F(S) \times P(F)$. D'où,
					\begin{align*}
						P_{\overline{F}}(S) &= \frac{P(S) - P_F(S) \times P(F)}{P(\overline{F})}\\
						&= \frac{0,\!25 - 0,\!40 \times 0,\!52}{1 - 0,\!58} \\
						&= \frac{0,\!25 - 0,\!208}{0,\!42} \\
						&= \frac{0,\!042}{0,\!42} \\
						&= 0,\!10. \\
					\end{align*}
					Ce qui valide l'affirmation du directeur.
			\end{enumerate}
		\item
			\begin{enumerate}
				\item La variable aléatoire $X$ suit la loi binomiale $\textstyle\cal{B}\left(20,\;\frac{1}{4}\right)$. En effet, le choix d'un salarié est indépendant des autres (tirage avec remise). De plus, on demande de manière identique à chaque salariés.
				\item On a \[
						P(X = 5) = {20\choose 5}\; \frac{1}{4^5}\; \left(\frac{3}{4}\right)^{15} \simeq 0,\!202
					.\]
				\item Le programme retourne (environ): \centered{{\tt {>}{>}{>}~~~0.617}.} Cela correspond à $P(X \le 5)$.
				\item On a $P(X \le 6) = P(X \le 5) + P(X = 6) = 0,\!617 + 0,\!169 = 0,\!786$.
			\end{enumerate}
		\item On note $s$ le salaire moyen d'un employé. Le nouveau salaire moyen d'un employé, $s'$, est \[
			s' = \frac{1}{4} \times (1.05) \times s + \frac{3}{4} \times (1.02) \times s
			= 1.0275 \times s
		.\] En moyenne, le salaire augment de $2,\!75$ \%.
	\end{enumerate}
	\section{Exercice 4 \hfill (7 points)}
	\begin{center}
		\def\arraystretch{2.2}
		\begin{tabular}{c|l}
			Question&\hfill Réponse \hfill~\\ \hline
			1.&c. $y = -2$\\ \hline
			2.&b. $F(x) = \frac{1}{2}e^{x^2}$ \\ \hline
			3.&d. Convexe sur $[2, +\infty[$ \\ \hline
			4.&a. Toutes sont croissantes sur $\R$ \\ \hline
			5.&d. $0$ \\ \hline
			6.&c. Une seule solution
		\end{tabular}
	\end{center}
\end{document}

