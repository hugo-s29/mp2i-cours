\part{Projection orthogonale}

Dans ce paragraphe, $\big(E, \left<\cdot  \mid \cdot  \right>\big)$ est un espace préhilbertien (de dimension quelconque).

\begin{defn}
	Soit $A \in \mathcal{P}(E)$. L'\underline{orthogonal} de $A$ est \[
		A^\perp = \{u \in E \mid \forall a\in A,\,a\perp u\}
	.\]
	\index{orthognal d'une partie}
\end{defn}

\begin{exm}
	\begin{enumerate}
		\item $\O^\perp = E$; $\{0_E\}^\perp = E$; $E^\perp = \{0_E\}$.

			Attention \danger, $\left( \O^\perp \right)^\perp = \{0_E\} \neq \O$.
		\item Avec $E = \R^3$, et $A = \{e_3\}$.
			\begin{multicols}{2}
				\begin{figure}[H]
					\centering
					\begin{asy}
						import three;
						size(2cm);
						draw(O--X, Arrow3(TeXHead2));
						draw(O--Y, Arrow3(TeXHead2));
						draw(O--Z, Arrow3(TeXHead2));
						label("$e_1$", X/2, align=S);
						label("$e_2$", Y/2, align=S);
						label("$e_3$", Z/2, align=W);
					\end{asy}
				\end{figure}
				\[
					A^\perp = \Vect(e_1, e_2)
				\] et \[
					A^{\perp\perp} = \Vect(e_1, e_2)^\perp = \Vect(e_3) \neq A.
				\]
			\end{multicols}
	\end{enumerate}
\end{exm}

\begin{prop}
	\[
		\forall A \in \mathcal{P}(E),\, A^\perp \text{ est un sous-espace vectoriel de } E.
	\]
\end{prop}

\begin{prv}
	Soit $A \in \mathcal{P}(E)$.
	\begin{itemize}
		\item  $\forall a \in A,  \left<a \mid 0_E \right> = 0$ donc $0_E \in A^\perp$ et donc $A^\perp \neq \O$.
		\item Soient $(u,v) \in A^\perp$, $(\alpha, \beta) \in \R^2$. Soit $a \in A$.
			\begin{align*}
				\left<\alpha u + \beta v  \mid a \right> &= \alpha \left<u \mid a \right> + \beta \left<v  \mid a \right> \\
				&= \alpha \times 0 + \beta \times 0 \\
				&= 0. \\
			\end{align*}
	\end{itemize}
\end{prv}

\begin{thm}
	Soit $F$ un sous-espace vectoriel de \underline{dimension finie} de $E$. Alors \[
		F\oplus F^\perp = E.
	\]
\end{thm}

\begin{prv}
	Soit $\mathcal{B} = (e_1, \ldots, e_p)$ une base orthonormée de $F$.
	\begin{itemize}
		\item[\underline{\sc Analyse}] Soit $x \in E$. On suppose $x = u + v$ avec $u \in F$ et $v \in F^\perp$. On pose $u = \sum_{i=1}^p u_i e_i$. De plus, \[
				\forall i \in \left\llbracket 1,p \right\rrbracket,\,\left<v \mid e_i \right> = 0.
			\] Soit $i \in \left\llbracket 1,p \right\rrbracket$.
			\begin{align*}
				\left<x \mid e_i \right> &= \left<u + v  \mid e_i \right> \\
				&= \left<u \mid e_i \right> + \left<v \mid e_i \right> \\
				&= \left<\sum_{j=1}^p u_j e_j  \mid e_i \right> \\
				&= \sum_{j=1}^p u_j \left<e_j \mid e_i \right> \\
				&= u_i \\
			\end{align*}
			D'où \[
				\begin{cases}
					u = \sum_{i=1}^p \left<x \mid e_i \right> e_i\\
					v = x - \sum_{i=1}^p \left<x \mid e_i \right> e_i.
				\end{cases}
			\]
		\item[\underline{\sc Synthèse}] Soit $x \in E$. On pose \[
				\begin{cases}
					u = \sum_{i=1}^p \left<x \mid e_i \right>e_i\\
					v = x - \sum_{i=1}^p \left<x \mid e_i \right>e_i
				\end{cases}
			\]
			On a clairement $u+v = x$ et $u \in F$.

			Soit $a \in F$. On pose $a = \sum_{i=1}^p a_i e_i$.
			\begin{align*}
				\left<v \mid a \right> &= \left<x \mid a \right> - \sum_{i=1}^p \left<x \mid e_i \right> \underbrace{\left<e_i  \mid a \right>}_{=a_i}\\
				&= \sum_{i=1}^p a_i \left<x \mid e_i \right> - \sum_{i=1}^p a_i \left<x \mid e_i \right> \\
				&= 0 \\
			\end{align*}
			donc $v \in F^\perp$.
	\end{itemize}
\end{prv}

