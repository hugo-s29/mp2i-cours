\part{Exercice 12}

Soit $\mathcal{B} = (e_1, \ldots, e_n)$ une base orthonormée de $E$.

\[
	\forall i,j,\;\left<f(e_i) \mid f(e_j) \right> = \left<e_i \mid e_j \right> = \delta_{i,j}
\]

donc $\big(f(e_1), \ldots,f(e_n)\big)$ est orthonormée, elle est donc libre; c'est donc une base orthonormée.

D'où
\begin{align*}
	\forall x \in E,\,f(x) &= \sum_{i=1}^n \left<f(x) \mid f(e_i) \right>f(e_i)\\
	&= \sum_{i=1}^n \left<x \mid e_i \right>f(e_i). \\
\end{align*}

Donc,
\begin{align*}
	\forall (x,y) \in E^2, \forall (\lambda, \mu) \in \R^2,\,f(\lambda x + \mu y) &= \sum_{i=1}^n \left<\lambda x + \mu y \mid e_i \right>f(e_i) \\
	&= \sum_{i=1}^n \big(\lambda\left<x \mid e_i \right> + \mu \left<x \mid e_i \right>\big)f(e_i) \\
	&= \lambda \sum_{i=1}^n \left<x \mid e_i \right> f(e_i) + \mu \sum_{i=1}^n \left<y \mid e_i \right>f(e_i) \\
	&= \lambda f(x) + \mu f(y). \\
\end{align*}

