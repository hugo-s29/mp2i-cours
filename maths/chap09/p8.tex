\part{}

\missingpart

\begin{prv}
	Soit $a \in I$ et \begin{align*}
		\tau_a: I\setminus \{a\} &\longrightarrow \R \\
		x &\longmapsto \frac{f(x)-f(a)}{x-a}.
	\end{align*} D'après le lemme des pentes, $\tau_a$ est croissante. En effet, soient $x,y \in I\setminus \{a\}$ tels que $x < y$.
	\begin{itemize}
		\item[\sc Cas 1] $a < x < y$. On a alors \[
				\frac{f(x)-f(a)}{x-a}\le \frac{f(y)-f(a)}{y-a}
			\] et donc $\tau_a(x) \le \tau_a(y)$.
		\item[\sc Cas 2] $x < a < y$. On a alors \[
				\frac{f(x) - f(a)}{x-a}\le \frac{f(x)-f(y)}{x-y} \le \frac{f(y) -f(a)}{y-a}
			\] et donc $\tau_a(x) \le \tau_a(y)$.
		\item[\sc Cas 3] $x<y<a$. On a alors \[
			\frac{f(x)-f(y)}{x-y} \le \frac{f(x) - f(a)}{x-a} \le \frac{f(y) - f(a)}{y-a}
		\] et donc $\tau_a(x) \le \tau_a(y)$.
	\end{itemize}

	Soit $x > a$ avec $x \in I$. On fixe $z_0 < a$ un élément de $I$. $z_0$ existe car $I$ est ouvert. On a alors\ldots
\end{prv}

\missingpart
