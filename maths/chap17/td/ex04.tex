\part{Exercice 4}

$E = \R^4$ \\
$a = (0,1,-1, 2)$\\
$b = (1,3,0,2)$\\
$c = (2,1,-3, 4)$\\
$d = (0,0,2, 1)$\\
$e = (-1, 1, 0, 3)$
\\[5mm]
 
$F = \Vect(a,b,c)$ et $G = \Vect(d,e)$\\

Soient  $(\lambda, \mu, \nu) \in \R^3$.
\begin{align*}
	\lambda a + \mu b + \nu c = 0 & \iff \begin{cases}
		\mu + 2\nu = 0\\
		\lambda + 3\mu + \nu = 0\\
		-\lambda - 3\nu = 0\\
		2\lambda + 2\mu + 4\nu = 0
	\end{cases}\\
	&\iff \begin{cases}
		\mu = -2\nu\\
		\lambda = -3\nu\\
		-8\nu = 0\\
		-6\nu = 0
	\end{cases}\\
	&\iff \begin{cases}
		\nu = 0\\
		\lambda = 0\\
		\mu = 0
	\end{cases}
\end{align*}

Donc $(a,b,c)$ est libre et donc $(a,b,c)$ est une base de $F$. Donc, $\boxed{\dim(F) = 3}$ \\[2mm]

$d$, $e$ ne sont pas colinéaires donc $(d,e)$ est une base de $G$. Donc, $\boxed{\dim(G) = 2}$\\[2mm]

$F + G = \Vect(a,b,c,d,e)$

\let\bx\boxed

\begin{itemize}
	\item[\underline{\sc Méthode 1}] Soient $\lambda, \mu, \nu, \alpha, \beta \in \R$.\\
		\begin{align*}
			\lambda a + \mu b + \nu c + \alpha d + \beta e = 0
			\iff&\begin{cases}
				\bx \mu + 2\nu - \beta = 0\\
				\lambda + 3\mu + \nu + \beta = 0\\
				-\lambda - 3\nu  +2\alpha = 0\\
				2\lambda + 2\mu + 4\nu + \alpha + 3\beta = 0
			\end{cases}\\
			\begin{array}{c}
				\iff\\
				L_2 \leftarrow L_2 - 3L_1\\
				L_4 \leftarrow L_4 -2L_1
			\end{array}& \begin{cases}
				\bx \mu + 2\nu - \beta = 0\\
				\lambda - 5\nu + 4\beta = 0\\
				-\lambda - 3\nu + 2\alpha = 0\\
				2\lambda + \bx\alpha + 5\beta = 0
			\end{cases}\\
			\begin{array}{c}
				\iff\\
				L_3\leftarrow L_3 - 2L_4
			\end{array}&\begin{cases}
				\bx \mu + 2\nu - \beta = 0\\
				\bx \lambda -5 \nu + 4\beta = 0\\
				-5\lambda -3\nu -10\beta = 0\\
				2\lambda + \bx \alpha + 5\beta = 0
			\end{cases}\\
			\begin{array}{c}
				\iff\\
				L_3 \leftarrow L_3 + 5L_2\\
				L_4 \leftarrow L_4 - 2L_2
			\end{array}& \begin{cases}
				\bx \mu + 2\nu - \beta = 0\\
				\bx \lambda - 5\nu + 4\beta = 0\\
				-28\nu + \bx{10\beta} = 0\\
				\bx \alpha + 10\nu - 3\beta = 0
			\end{cases}\\
			\iff& \begin{cases}
				\beta = \frac{28}{10}\nu\\
				\mu = \cdots\\
				\lambda = \cdots\\
				\alpha = \cdots
			\end{cases}
		\end{align*}

		Avec $\nu = 1$, on a $\lambda a + \mu b + c + \alpha d + \beta e = 0$ et donc $c = -\lambda a -\mu b -\alpha d -\beta e$ donc $c \in \Vect(a,b,d,e)$\\
		Avec $\nu = 0$, on a $\lambda a + \mu b + \alpha d + \beta e = 0 \iff \beta = \mu = \lambda = \alpha = 0$ et docnc $(a,b,d,e)$ est libre.\\
		Donc, $(a,b,d,e)$ est une base de $F+G$ et donc $\boxed{\dim(F+G) = 4}$ (donc $F+G = \R^4$\\
		D'après la formule de Grassmann, \[
			\boxed{\dim(F \cap G) = 3 + 2 - 4 = 1}
		\] 
	\item[\underline{\sc Méthode 2}] $\dim(F+G) = \rg(a,b,c,d,e)$ \\
		\[
			M = \begin{pmatrix}
				0&1&2&0&-1\\
				\bx1&3&1&0&1\\
				-1&0&-3&2&0\\
				2&2&4&1&3
			\end{pmatrix}
		\] \[ \rg(a,b,c,d,e) = \rg(M) \]
		\begin{align*}
				M \begin{array}{c}
					\sim\\
					C_2\leftarrow C_2-3C_1\\
					C_3\leftarrow C_3 - C_1\\
					C_5 \leftarrow C_5 - C_1
				\end{array}&
				\begin{pmatrix}
					0&1&2&0&-1\\
					\bx1&0&0&0&0\\
					-1&2&-2&2&\bx 1\\
					2&-4&2&1&1
				\end{pmatrix}\\
				\begin{array}{c}
					\sim\\
					C_2\leftarrow C_2-3C_5\\
					C_3\leftarrow C_3 + 2C_5\\
					C_4\leftarrow C_4 + 2 C_5
				\end{array} &
				\begin{pmatrix}
					0&4&0&-2&1\\
					\bx1&0&0&0&0\\
					-1&0&0&0&\bx1\\
					2&-7&4&\bx1&1
				\end{pmatrix}\\
				\begin{array}{c}
					\sim\\
					C_2\leftarrow C_2 + 7C_4\\
					C_3\leftarrow C_3 - 4C_4
				\end{array}&
				\begin{pmatrix}
					0&-10&\bx8&-2&-1\\
					\bx1&0&0&0&0\\
					-1&0&0&0&\bx1\\
					2&0&0&\bx1&1
				\end{pmatrix}
		\end{align*}
		Donc $\rg(M) = 4$ \[
			\boxed{
				\begin{array}{c}
					\dim(F+G) = 4\\
					\dim(F \cap G) = 1
				\end{array} 
			}
		\]
	\item[\underline{\sc Méthode 3}] $F + G \subset R^4$ donc $\dim(F +G) \le 4$ \\
		$F \subset F+G$ donc $\dim(F+G) \ge 3$ \\
		Donc $\dim(F+G) \in \{3, 4\}$\\

		On suppose $\dim(F+G) = 3$. Alors $F = F + G$. Or, $G \subset  F+G = F$ \\
		On va caractériser $F$ par un système d'équations.\\
		Soit $u = (x,y,z,t) \in \R^4$ 
		\begin{align*}
			u \in F &\iff \exists (\lambda, \mu, \nu) \in \R^3, u = \lambda a + \mu b + \nu c\\
			&\iff\exists (\lambda, \mu, \nu) \in \R^3 \begin{cases}
				x = \mu + 2\nu\\
				y = \bx\lambda + 3\mu + \nu\\
				z = -\lambda -3 \nu\\
				t = 2\lambda + 2\mu + 4\nu
			\end{cases}\\
			&\begin{array}{c}
				\iff\\
				L_3\leftarrow L_3 + L_2\\
				L_4 \leftarrow L_4 -2L_2
			\end{array} \exists (\lambda, \mu, \nu) \in \R^3, \begin{cases}
				\bx\mu + 2\nu = x\\
				\bx \lambda + 3\mu + \nu = y\\
				3\mu - 2\nu = y + z\\
				-4\mu + 2\nu = t - 2y
			\end{cases}\\
			&\begin{array}{c}
				\iff\\
				L_2\leftarrow L_2 -3L_1\\
				L_3 \leftarrow L_3 - 3L_1\\
				L_4 \leftarrow \frac{L_4 + 4L_2}{10}
			\end{array}  \exists (\lambda, \mu, \nu) \in \R^3, \begin{cases}
				\bx \mu  + 2\nu = x\\
				\bx \lambda - 5\nu = y - 3x\\
				-8\nu = y + z - 3x\\
				\bx \nu = \frac{t-2y + 4x}{10}
			\end{cases}\\
			&\begin{array}{c}
				\iff\\
				L_3 \leftarrow L_3 + 8L_4
			\end{array}  \exists (\lambda, \mu, \nu) \in \R^3, \begin{cases}
				\mu = \cdots\\
				\lambda = \cdots\\
				0 = \frac{1}{5}x -\frac{3}{5}y  + z + \frac{16}{5}t\\
				\nu = \cdots\\
			\end{cases}\\
			&\iff x + 3y + 5z + 16t = 0
		\end{align*}

		\[
			(x,y,z,t) \in F \iff x - 3y + 5z + 16t = 0
		\]
		Or, $0- 3\times 0 + 5 \times  2 + 16 = 26 \neq 0$ donc $d \not\in F$ $\lightning$\\
		Donc  $\dim(F+G) = 4$ et domc $\dim(F \cap G) = 1$\\
	\item[\underline{\sc Méthode 4}] On caractérise $F$ et $G$ par des équations. On reprend les calculs de la méthode 3. \[
			(x,y,z,t) \in F \iff x - 3y + 5z + 16t = 0
		\]

		\begin{align*}
			(x,y,z,t) \in G &\iff \exists (\alpha, \beta) \in \R^2, (x,y,z,t) = \alpha d+ \beta e\\
			&\iff \exists (\alpha, \beta) \in \R^2, \begin{cases}
				-\beta = x\\
				\beta = y\\
				2\alpha = z\\
				\alpha + 3\beta = t
			\end{cases}\\
			&\iff \exists (\alpha, \beta) \in \R^2, \begin{cases}
				\alpha = \frac{1}{2}z\\
				\beta = y\\
				x+y = 0\\
				\frac{1}{2}z + 3y = t
			\end{cases}\\
			&\iff \begin{cases}
				x+y= 0\\
				z + 6y - 2t = 0
			\end{cases}
		\end{align*}

		\begin{align*}
			(x,y,z,t) \in F \cap G \iff& \begin{cases}
				x - 3y + 5z + 16t = 0\\
				\bx x + y = 0\\
				z + 6y - 2t = 0
			\end{cases}\\
			\begin{array}{c}
				\iff\\
				L_1\leftarrow L_1 - L_2
			\end{array}&\begin{cases}
				-4y + 5z + 16t = 0\\
				\bx x + y = 0\\
				\bx z + 6y -2t = 0
			\end{cases}\\
			\begin{array}{c}
				\iff\\
				L_1 \leftarrow \frac{L_1 - 5 L_3}{26}
			\end{array}&\begin{cases}
				-\frac{34}{26}y + t = 0\\
				\bx x + y = 0\\
				\bx z + 6y - 2t = 0
			\end{cases}\\
			\begin{array}{c}
				\iff\\
				L_3\leftarrow L_3 + 2L_1
			\end{array}&\begin{cases}
				t = \frac{17}{13}y\\
				x = -y\\
				z = -\frac{44}{13}y
			\end{cases}\\
				\iff& (x,y,z,t) = \left(-y,y, -\frac{44}{13}y, \frac{17}{13}\right)\\
				\iff& (x,y,z,t) = \frac{y}{13}(-13, 13, -44, 17)
		\end{align*}
		Donc, $F \cap G = \Vect((-13, 13, -44, 17))$ donc $\dim(F \cap G) = 1$\\
		Et donc, $\dim(F + G) = 4$
\end{itemize}
