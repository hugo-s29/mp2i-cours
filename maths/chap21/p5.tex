\part{Conséquences}

\begin{prop}
	La multiplication matricielle est associative.
\end{prop}

\begin{prv}
	\[
		\begin{cases}
			A \in \mathcal{M}_{n,p}(\mathbbm{K})\\
			B \in \mathcal{M}_{p,q}(\mathbbm{K})\\
			C \in \mathcal{M}_{q,r}(\mathbbm{K})
		\end{cases}
	\]
	Soient $f \in \mathcal{L}(\mathbbm{K}^p, \mathbbm{K}^n)$, $g \in \mathcal{L}(\mathbbm{K}^q, \mathbbm{K}^p)$ et $h \in \mathcal{L}(\mathbbm{K}^r, \mathbbm{K}^q)$ telles que \[
		\begin{cases}
			A = \Mat_{\mathcal{B},\mathcal{C}}(f)\\
			B = \Mat_{\mathcal{E}, \mathcal{B}}(g)\\
			C = \Mat_{\mathcal{D},\mathcal{E}}(h)
		\end{cases}
	\] où 
	$\mathcal{B}$ est la base canonique de $\mathbbm{K}^p$,
	$\mathcal{C}$ est la base canonique de $\mathbbm{K}^n$,
	$\mathcal{D}$ est la base canonique de $\mathbbm{K}^r$,
	$\mathcal{E}$ est la base canonique de $\mathbbm{K}^q$.\\
	\begin{align*}
		A(BC) &= \Mat_{\mathcal{D}\mathcal{C}}\big(f \circ (g \circ h)\big)\\
					&= \Mat_{\mathcal{D},\mathcal{C}}\big((f \circ g) \circ h\big)  \\
					&= (AB) C \\
	\end{align*}
\end{prv}

\begin{prop}
	Soit $(A,B) \in \mathcal{M}_n(\mathbbm{K})^2$. On suppose que $AB = I_n$. Alors $(A,B) \in \mathrm{GL}_n(\mathbbm{K})^2$ et $A^{-1} = B$.
\end{prop}

\begin{prv}
	Soit $\mathcal{B}$ la base canonique de $\mathbbm{K}^n$, $f \in \mathcal{L}(\mathbbm{K}^n)$ telle que $\Mat_\mathcal{B}(f) = A$, $g \in \mathcal{L}(\mathbbm{K}^n)$ telle que $\Mat_{\mathcal{B}}(g) = B$.
	\begin{align*}
		AB = I_n &\text{ donc } \Mat_\mathcal{B}(f \circ g) = \Mat_\mathcal{B}(\id_{\mathbbm{K}^n})
						 &\text{ donc } f \circ g = \id_{\mathbbm{K}^n}\\
						 &\text{ donc } f \circ g \text{ est injective}\\
						 &\text{ donc } g \text{ est injective}\\
						 &\text{ donc } g \text{ est un isomorphisme}
	\end{align*}
	Or, $f \circ g = \id$ donc $f = f \circ g \circ g^{-1} = g^{-1}$
	\begin{align*}
		BA &= \Mat_\mathcal{B}(g) = \Mat_\mathcal{B}(f)\\
		&= \Mat_\mathcal{B}(g \circ f) \\
		&= \Mat_\mathcal{B}(g \circ g^{-1}) \\
		&= \Mat_\mathcal{B}(\id_{\mathbbm{K}^n}) \\
		&= I_n \\
	\end{align*}
	Donc $A = B^{-1}$.
\end{prv}

\begin{rmk}
	Au passage, on a montré que \[
		f \in \mathrm{GL}(E) \iff \Mat_\mathcal{B}(f) \in \mathrm{GL}_n(\mathbbm{K})
	\] et, dans ce cas, \[
		\Mat_\mathcal{B}(f^{-1}) = \Mat_\mathcal{B}(f)^{-1}
	\]
\end{rmk}

\begin{prop}
	Soit $A \in \mathcal{M}_{p,n}(\mathbbm{K})$.

	Le nombre maximal de lignes linéairement indépendantes de $A$ est égal au rang de $A$.
\end{prop}

\begin{prv}
	On appelle \underline{rang par lignes} le nombre exact de lignes linéairement indépendantes.
	
	Ce rang par ligne est invariant quand on effectue une opération élémentaire sur les lignes.

	En appliquant la méthode du pivot de \Gauss, on obtient une matrice de la forme \[
		\left(\begin{array}{c c c c c c c c}
				0&\ldots&0&\multicolumn{1}{|c}{\bx1} & * & \ldots &\ldots& *\\ \cline{4-5}
				0&\ldots&\ldots&0&\multicolumn{1}{|c}{\bx1} & * & \ldots & *\\ \cline{5-8}
				0&\ldots&\ldots&\ldots&\ldots&\ldots&\ldots&0\\
		\end{array}\right)
	\] qui a le même rang par lignes que $A$.

	On obsèrve que ce rang $r$ est égal au nombre de pivots.

	Soit $S$ le système homogène \[
		AX = 0
	\] où $X = \begin{pmatrix}
		x_1\\
		\vdots\\
		x_n\\
	\end{pmatrix}$. D'après l'algorithme du pivot, la résolution de ce système fournit $r$ inconnues principales et $n-r$ paramètres.

	Sans perte de généralité, on peut supposer que $x_1, \ldots, x_{n-r}$ sont les paramètres et $x_{n-r+1},\ldots, x_n$ les inconnues principales.

	Soit $\mathcal{B}=(e_1, \ldots, e_n)$ la base canonique de $\mathbbm{K}^n$ et $\mathcal{C} = (f_1, \ldots, f_p)$ la base canonique de $\mathbbm{K}^p$ et $f \in \mathcal{L}(\mathbbm{K}^n, \mathbbm{K}^p)$ telle que \[
		A = \Mat_{\mathcal{B},\mathcal{C}}(f)
	.\]
	Soit $x = (x_1, \ldots, x_n) \in \mathbbm{K}^n$ et $X = \Mat_\mathcal{B}(x)$.

	\begin{align*}
		AX = 0 \iff& \Mat_{\mathcal{B}, \mathcal{C}}(f) \Mat_\mathcal{B}(f) = \Mat_\mathcal{C}(0)\\
		\iff& \Mat_\mathcal{C}\big(f(x)\big) = \Mat_\mathcal{C}(0)\\
		\iff& f(x) = 0\\
		\iff& x \in \Ker f
	\end{align*}

	Ainsi, l'ensemble $E$ des solutions de $(S)$ est un $\mathbbm{K}$-espace vectoriel isomorphe à $\Ker(f)$.

	De plus,
	\begin{align*}
		g: E &\longrightarrow \mathbbm{K}^{n-r} \\
		\begin{pmatrix}
			x_1\\
			\vdots\\
			x_n
		\end{pmatrix} &\longmapsto (x_1, \ldots, x_{n-r})
	\end{align*} est un isomorphisme. Donc, $\dim(\Ker f) = \dim(E) = n - r$.
	
	D'après le théorème du rang, \[
		\rg(A) = \rg(f) = \dim(\mathbbm{K}^n) - \dim(\Ker f) = n - (n - r) = r
	.\]
\end{prv}

\begin{defn}
	Soit $A = (a_{i,j})_{\substack{1\le i\le p\\1\le j\le n}} \in \mathcal{M}_{p,n}(\mathbbm{K})$.

	La \underline{transposée} de $A$, notée $\t A = (b_{j,i})_{\substack{1\le j\le n\\1\le i\le p}} \in \mathcal{M}_{n,p}(\mathbbm{K})$ est définie par \[
		\forall i \in \left\llbracket 1,p \right\rrbracket, \forall j \in \left\llbracket 1,n \right\rrbracket, b_{j,i} = a_{i,j}.
	\] Les lignes de $\t A$ sont les colonnes de $A$. Les colonnes de $\t A$ sont les lignes de $A$.
	\index{transposée (matrice)}
\end{defn}

\begin{exm}
	$A = \begin{pmatrix}
		1&0\\
		2&1\\
		3&-1
	\end{pmatrix}$ donc $\t A = \begin{pmatrix}
		1&2&3\\
		0&1&-1
	\end{pmatrix}$.
\end{exm}

\begin{crlr}
	\[
		\forall A \in \mathcal{M}_{n,p}(\mathbbm{K}), \rg(A) = \rg(\t A)
	\] \qed
\end{crlr}

\begin{prop}
	$\begin{array}{rcl}
		\mathcal{M}_n(\mathbbm{K}) &\longrightarrow& \mathcal{M}_n(\mathbbm{K}) \\
		A &\longmapsto& \t A
	\end{array}$ est la symétrie par rapport à $S_n(\mathbbm{K})$ parallèlement à $A_n(\mathbbm{K})$ où \[
		S_n(\mathbbm{K}) = \{A \in \mathcal{M}_n(\mathbbm{K})  \mid  \t A = A\} = 
		\small
		\begin{pNiceArray}{c c c c c}
			\phantom{0}&\phantom{0}&\phantom{0}&\phantom{0}&\phantom{0}\\
			\phantom{0}&\phantom{0}&\phantom{0}&(u)&\phantom{0}\\
			\phantom{0}&\phantom{0}&\phantom{0}&\phantom{0}&\phantom{0}\\
			\phantom{0}&(u)&\phantom{0}&\phantom{0}&\phantom{0}\\
			\phantom{0}&\phantom{0}&\phantom{0}&\phantom{0}&\phantom{0}\\
			\CodeAfter
			\tikz \draw (1-1) -- (5-5);
			\tikz \draw (2-1) -- (5-4) -- (5-1) -- (2-1);
			\tikz \draw (1-2) -- (4-5) -- (1-5) -- (1-2);
		\end{pNiceArray}
	\]
	et \[
		A_n(\mathbbm{K}) = \{A \in \mathcal{M}_n(\mathbbm{K})  \mid \t A = -A\}  = 
		\small\begin{pNiceArray}{c c c c c}
			0&\phantom{0}&\phantom{0}&\phantom{0}&\phantom{0}\\
			\phantom{0}&\phantom{0}&\phantom{0}&(u)&\phantom{0}\\
			\phantom{0}&\phantom{0}&\phantom{0}&\phantom{0}&\phantom{0}\\
			\phantom{0}&(-u)&\phantom{0}&\phantom{0}&\phantom{0}\\
			\phantom{0}&\phantom{0}&\phantom{0}&\phantom{0}&0\\
			\CodeAfter
			\tikz \draw[dotted] (1-1) -- (5-5);
			\tikz \draw (2-1) -- (5-4) -- (5-1) -- (2-1);
			\tikz \draw (1-2) -- (4-5) -- (1-5) -- (1-2);
		\end{pNiceArray}
	\]
	et donc \[
		S_n(\mathbbm{K}) \oplus A_n(\mathbbm{K}) = \mathcal{M}_n(\mathbbm{K})
	.\]
\end{prop}

\begin{prv}
	Soient $A = (a_{i,j})$, $B = (b_{i,j})$ et $(\alpha, \beta) \in \mathbbm{K}^2$.
	\begin{align*}
		\t (\alpha A + \beta B) &= (\alpha a_{j,i} + \beta b_{j,i})_{1 \le i,j \le n} = \alpha (a_{j,i}) + \beta (b_{j,i}) \\
		&= \alpha \t A + \beta \t B \\
	\end{align*}
	
	Clairement, \[
		\forall A \in \mathcal{M}_n(\mathbbm{K}), \t\left( \t A \right) = A
	\] donc $f: A \mapsto \t A$ est la symétrie par rapport à $\Ker(f - \id_{\mathcal{M}_n(\mathbbm{K})})$ parallèlement à $\Ker(f + \id_{\mathcal{M}_n(\mathbbm{K})})$.

	Or, \begin{align*}
		\Ker(f - \id_{\mathcal{M}_n(\mathbbm{K})}) &= \{A \in \mathcal{M}_n(\mathbbm{K})  \mid  f(A) - A = 0\}  \\
		&= \{A \in \mathcal{M}_n(\mathbbm{K})  \mid  \t A = A\} \\
		&= S_n(\mathbbm{K}) \\
	\end{align*} et
	\begin{align*}
		\Ker(f + \id_{\mathcal{M}_n(\mathbbm{K})}) &= \{A \in \mathcal{M}_n(\mathbbm{K})  \mid f(A) + A = 0\}\\
		&= \{A \in \mathcal{M}_n(\mathbbm{K})  \mid \t A = A\} \\
		&= A_n(\mathbbm{K}) \\
	\end{align*}
\end{prv}

\begin{prop}
	Soit $A \in \mathcal{M}_{n,p}(\mathbbm{K})$ et $B \in \mathcal{M}_{p,q}(\mathbbm{K})$. \[
		\t (AB) = \t B \t A
	\]
\end{prop}

\begin{prv}
	On pose \[
		\begin{cases}
			A = (a_{i,j})_{\substack{1\le i\le n\\1\le j\le p}},\\
			B = (b_{j,k})_{\substack{1\le j\le p\\1\le k\le q}},\\
			\t B \t A = (c_{k,i})_{\substack{1\le k\le q\\1\le i\le n}}.
		\end{cases}
	\]

	\begin{align*}
		\forall k \in \left\llbracket 1,q \right\rrbracket, \forall i \in \left\llbracket 1,n \right\rrbracket, 
		c_{k,i} &= \sum_{j=1}^p (\t B)_{k,j} (\t A)_{j,i}\\
		&= \sum_{j=1}^p b_{j,k} a_{i,j} \\
		&= \sum_{j=1}^p a_{i,j} b_{j,k} \\
		&= (AB)_{i,k} \\
		&= \left( \t(AB)_{k,i} \right) \\
	\end{align*}
	Donc, $\t(AB) = \t B \t A$.
\end{prv}

\begin{crlr}
	Si $A \in \mathrm{GL}_n(\mathbbm{K})$ alors $\t A \in \mathrm{GL}_n(\mathbbm{K})$ et \[
		\left( \t A \right) ^{-1} = \t \left( A^{-1} \right) 
	\]
\end{crlr}

\begin{prv}
	On suppose $A \in \mathrm{GL}_n(\mathbbm{K})$.

	$A A^{-1} = I_n$ donc $\t(A A^{-1}) = \t I_n$

	donc $\t(A^{-1}) \t A = I_n$ 

	donc \[
		\t A \in \mathrm{GL}_n(\mathbbm{K})\\
		\t (A^{-1}) = (\t A)^{-1}
	\]
\end{prv}












