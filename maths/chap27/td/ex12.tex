\part{Exercice 12}

\begin{align*}
	p_{n+1} &= q(1-p_n) + p p_n \\
	&= (1-p)(1-p_n) + pp_n \\
	&= p_n (2p-1) + 1 - p \\
\end{align*}

On a \[
	p_n = \alpha (2p-1)^{n} + \frac{1}{2}
\] car
\begin{align*}
	&x = x(2p-1) + 1- p\\
	\iff& x(2p-2) = p-1\\
	\iff& x = \frac{1}{2}
\end{align*}

On a $p_1 = 1$ donc
\begin{align*}
	&\alpha(2p-1) + \frac{1}{2} = 1\\
	\iff& \alpha = \frac{\frac{1}{2}}{2p - 1} = \frac{1}{4p-2} \\
\end{align*}

On a donc \[
	p_n = \frac{1}{4p-2}(2p-1)^n + \frac{1}{2} = \frac{1}{2} (2p-1)^{n_1} + \frac{1}{2} \tendsto{n\to +\infty} \frac{1}{2}
.\]

\begin{mdframed}
	\centered{\Large Méthode : suites arithmético-géométriques}

	On a \[
		\forall n,\,u_{n+1} = a u_n + b
	.\] L'application \begin{align*}
		\varphi: \C^\N &\longrightarrow \C^\N \\
		(u_n) &\longmapsto (u_{n+1}-au_n)
	\end{align*} est linéaire.
	On a donc, \[
		\varphi(u) = (b)_{n\in\N} \iff \overset{\substack{\Ker \varphi\\\vrt\in\\~}}{u} + \underbrace{u_0}_{\text{suite constante}}
	.\] On a $\Ker \varphi = \Vect\big((a^n)\big)$.
\end{mdframed}

