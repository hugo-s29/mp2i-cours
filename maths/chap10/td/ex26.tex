\part{Exercice 26}
\begin{enumerate}
	\item 
		\begin{align*}
			I_{p,q} &= \int_{0}^{1} x^p(1-x)^p dx\\
			&= \left[ \frac{x^{p+1}}{p+1} (1-x)^{q} \right]_0^1+ \int_{0}^{1} \frac{x^{p+1}}{p+1} q(1-x)^{q-1} dx  \\
			&= 0 + \frac{q}{q+1} I_{p+1,q-1} \\
		\end{align*}
		$I_{p,0} = \frac{1}{p+1}$\\
		\begin{align*}
			I_{p,q} &= \frac{q}{p+1}I_{p+1,q-1} \\
			&= \frac{q}{p+1}\times \frac{q-1}{p+2}I_{p+2,q-2} \\
			&= \frac{q}{p+1}\times \frac{q-1}{p+2}\times \ldots \times \frac{1}{p+q}\times I_{p+q,0} \\
			&= \frac{q!~p!}{(p+q)!} \times  \frac{1}{p+q+1} \\
			&= \frac{p!~q!}{(p+q+1)!} \\
		\end{align*}
	\item 
		\begin{align*}
			I_{n,n} &= \int_{0}^{1} x^n(1-x)^n dx\\ 
			&= \int_{0}^{1} \sum_{k=0}^{n} {n \choose k} x^n(-x)^{n-k}dx \\
			&= \sum_{k=0}^{n} (-1)^{n-k} {n \choose k} \int_{0}^{1} x^{2n-k}dx  \\
			&= \sum_{k=0}^{n} (-1)^{n-k} {n \choose k} \left[ \frac{2^{2n-k+1}}{2n-k+1} \right] _0^1 \\
			(*) &= \sum_{k=0}^{n} (-1)^{n-k} {n \choose k} \frac{1}{2n-k+1} \\
		\end{align*}
		Or, $I_{n,n} = \frac{\left( n! \right) ^2}{\left( 2n+1 \right)!}$.\\
		Avec $(*)$, $I_{n,n}$ est une somme de rationnels (donc $I_{n,n} \in \Q$) dont les dénominateurs sont $n+1, n+2, \ldots, 2n+1$ \\
		Comme $D_n = (n+1)\vee(n+2)\vee\ldots\vee(2n+1)$, il existe $a \in \Z$ te lque $I_{n,n}= \frac{a}{D_n}$.\\
		Or, $I_{n,n} >0$ donc $a \in \N_*$ donc $a \ge 1$
		\begin{align*}
			\text{donc }& I_{n,n} \ge \frac{1}{D_n}\\
			\text{donc }& D_n \ge \frac{1}{I_{n,n}}=\frac{(2n+1)!}{\left( n! \right) ^2}
		\end{align*}
	\item $D_n = \prod_{p in \mathcal{P}} p^{\max(\alpha_1(p),\ldots,\alpha_n(p))}$ où \[
			\forall k \in \left\llbracket 1,n+1 \right\rrbracket, \frac{n}{k-1} = \prod_{p \in \mathcal{P}} p^{\alpha_k(p)}
		\]
	Ce produit fait intervenir des nombres premiers qui divisent $n+1$ ou $n+2$ ou \ldots ou $2n+1$, il y en a au plus $\pi(2n+1)$.\\
	Pour chacun de ces nombres premiers, $q = p^{\max(\alpha_k(p) \mid k \in \left\llbracket 1,n+1 \right\rrbracket)}$ apparaît dans la décomposition de l'un des nombres $n+1,n+2, \ldots, 2n+1$ donc $q \le 2n+1$. Donc, \[
	D_n \le (2n+1)^{\pi(2n+1)}
	\] D'où, \[
	\frac{(2n+1)!}{(n!)^2}\le (2n+1)^{\pi(2n+1)}
	\] donc 
	\begin{align*}
		\frac{(2n+1)!}{(n!)^2} &= (2n+1) \frac{(2n)!}{n!~n!} \\
		&= (2n+1) {2n  \choose n} \in \N \\
		&\ge (2n+1) \frac{2^{2n}}{2n+1}
	\end{align*}
\end{enumerate}
