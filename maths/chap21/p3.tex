\part{Matrices d'une application linéaire}

Soit $f \in \mathcal{L}(E,F)$, $\mathcal{B} = (e_1, \ldots, e_n)$ une base de $E$ et $\mathcal{C} = (f_1, \ldots, f_p)$ une base de $F$.
\vspace{5mm}

\begin{prop-defn}
	Soit $x \in E$ et $X = \begin{pmatrix}
		x_1\\
		\vdots\\
		x_n
	\end{pmatrix} = \Mat_\mathcal{B}(x)$. Soit $y \in F$ et $Y = \begin{pmatrix}
		y_1\\
		\vdots\\
		y_p
	\end{pmatrix} = \Mat_\mathcal{C}(y)$. On pose $A = \begin{pmatrix}
		a_{11}&\ldots&a_{1,n}\\
		\vdots&\ddots&\vdots\\
		a_{p,1}&\ldots&a_{p,n}
	\end{pmatrix} \in \mathcal{M}_{p,n}(\mathbbm{K})$ telle que \[
		\forall j \in \left\llbracket 1,n \right\rrbracket, \begin{pmatrix}
			a_{1,j}\\
			\vdots\\
			a_{p,j}
		\end{pmatrix} = \Mat_\mathcal{C}\big(f(e_j)\big)
	\] Alors \[
		y = f(x) \iff Y = AX
	\] On dit que $A$ est la \underlin{matrice de $f$ dans les bases $\mathcal{B}$ et $\mathcal{C}$}. On la note $\Mat_{\mathcal{B},\mathcal{C}}(f)$.
	\index{matrice d'une application linéaire}
\end{prop-defn}

\begin{prv}
	\[
		x = \sum_{j=1}^n x_j e_j \qquad y = \sum_{i=1}^p y_i f_i
	\]
	\begin{align*}
		y = f(x) \iff& \sum_{i=1}^p y_i f_i = f\left( \sum_{j=1}^n x_j e_j \right)\\
		\iff& \sum_{i=1}^p y_i f_i = \sum_{j=1}^n x_j f(e_j)
	\end{align*}
	Or, \[
		\forall j \in \left\llbracket 1,n \right\rrbracket, f(e_j) = \sum_{i=1}^p a_{i,j} f_i
	\] D'où,
	\begin{align*}
		y = f(x) \iff& \sum_{i=1}^p y_i f_i = \sum_{j=1}^n x_j \sum_{i = 1}^p a_{i,j}f_i\\
		&\phantom{\sum_{i=1}^p y_i f_i} = \sum_{i=1}^p \left( \sum_{j=1}^n a_{i,j}x_j \right) f_i\\
		\iff& \forall i \in \left\llbracket 1,p \right\rrbracket, \sum_{j=1}^n a_{i,j} x_i = y_i\\
		\iff& AX = Y
	\end{align*}
\end{prv}

\begin{exm}
	$E = \R^3$ et $f$ la projection sur $R = \big\{(x,y,z)  \mid  x + y = 0\big\}$ parallèlement à $G = \Vect\big((1,1,1)\big)$.
	\begin{itemize} %todo
		\item $\mathcal{B} = (e_1, e_2, e_3)$ base canonique.
			\[
				\Mat_{\mathcal{B}, \mathcal{B}}(f) = \begin{pmatrix}
					\cdot &\cdot &\cdot \\
					\cdot &\cdot &\cdot \\
					\cdot &\cdot &\cdot \\
				\end{pmatrix}
			\]
			\[
				e_1 = (1,0,0) = \underbrace{\left( \frac{1}{2},-\frac{1}{2},-\frac{1}{2} \right)}_{\in F} + \underbrace{\left( \frac{1}{2},\frac{1}{2},\frac{1}{2} \right)}_{\in G}
			\] \[
				f(e_1) = \left( \frac{1}{2}, -\frac{1}{2}, -\frac{1}{2} \right) = \frac{1}{2}e_1 - \frac{1}{2}e_2 - \frac{1}{2}e_3
			\] \[
				e_2 = (0, 1, 0) = \left( -\frac{1}{2}, \frac{1}{2}, -\frac{1}{2} \right) + \left( \frac{1}{2}, \frac{1}{2}, \frac{1}{2} \right)
			\] \[
				e_3 = (0,0,1) = (0,0,1) + (0,0,0)
			\] Donc \[
				\Mat_{\mathcal{B}, \mathcal{B}}(f) = \begin{pmatrix}
					\frac{1}{2}&-\frac{1}{2}&0\\
					-\frac{1}{2}&\frac{1}{2}&0\\
					-\frac{1}{2}&-\frac{1}{2}&1
				\end{pmatrix}
			\]
		\item $\mathcal{C} = (e_3, e_1 - e_2, e_1 + e_2 + e_3) =  (u_1, u_2, u_3)$ % todo
			\[
				\Mat_{\mathcal{C},\mathcal{C}}(f) = \begin{pmatrix}
					1&0&1\\
					0&1&0\\
					0&0&0
				\end{pmatrix}
			\] car \[
				f(u_1) = u_1\\
				f(u_2) = u_2\\
				f(u_3) = 0
			\]
		\item \[
			\Mat_{\mathcal{C},\mathcal{B}} = \begin{pmatrix}
					0&1&0\\
					0&-1&0\\
					1&0&0
				\end{pmatrix}
			\]
		\item \[
			\Mat_{\mathcal{B},\mathcal{C}}(f) = \begin{pmatrix}
				-\frac{1}{2}&-\frac{1}{2}&1\\
				\frac{1}{2}&-\frac{1}{2}&0\\
				0&0&0
			\end{pmatrix}
			\] 
		\item Soit $(x,y,z) \in \R^3$
			\begin{align*}
				\Mat_{\mathcal{B}}\big(f\big((x,y,z)\big)\big) &= \Mat_{\mathcal{B}, \mathcal{B}}(f) \times \Mat_{\mathcal{B}}\big((x,y,z)\big)  \\
				&= \begin{pmatrix}
					-\frac{1}{2}&-\frac{1}{2}&1\\
					\frac{1}{2}&-\frac{1}{2}&0\\
					0&0&0
				\end{pmatrix} \begin{pmatrix}
					x\\
					y\\
					z
				\end{pmatrix} \\
				&= \begin{pmatrix}
					\frac{x-y}{2}\\
					\frac{y - x}{2}\\
					\frac{-x-y}{2}+z
				\end{pmatrix} \\
			\end{align*}
			Donc \[
				f\big((x,y,z)\big) = \left( \frac{x-y}{2}, \frac{y - x}{2}, z - \frac{x + y}{2} \right)
			\]% todo
			\begin{align*}
				\Mat_\mathcal{C}\big(f(x,y,z)\big) &= \Mat_{\mathcal{B},\mathcal{C}}(f) \times \Mat_\mathcal{B}\big((x,y,z)\big)\\
				&= \begin{pmatrix}
					-\frac{1}{2}&-\frac{1}{2}&1\\
					\frac{1}{2}&-\frac{1}{2}&0\\
					0&0&0
				\end{pmatrix} \begin{pmatrix}
					x\\
					y\\
					z
				\end{pmatrix} \\
				&= \begin{pmatrix}
					\frac{-x-y}{2}+z\\
					\frac{x - y}{2}\\
					0
				\end{pmatrix} \\
			\end{align*}
			Donc \[
				f\big((x,y,z)\big)  = \left( z - \frac{x+y}{2} \right)u_1 + \left( \frac{x-y}{2} \right)u_2
			\]
	\end{itemize}
\end{exm}

\begin{exm}
	Soient $E = \C$, $\mathcal{B} = (1,i)$, $\theta \in \R$ et $f$ la rotation de centre $O$ et d'angle $\theta$. \begin{align*}
		f: \C &\longrightarrow \C \\
		z &\longmapsto e^{i\theta}z
	\end{align*}
	$f$ est linéaire :
	\begin{align*}
		\forall (\lambda,\mu) \in \R^2, \forall (z, w) \in \C^2
		f(\lambda z + \mu w) &= e^{i\theta}(\lambda z + \mu w) \\
		&= \lambda e^{i\theta} z + \mu e^{i\theta}w \\
		&= \lambda f(z) + \mu f(w) \\
	\end{align*}
	\[
		\Mat_\mathcal{B}(f) = \begin{pmatrix}
			\cos \theta & -\sin\theta\\
			\sin\theta & \cos\theta
		\end{pmatrix}
	\] car \[
		\begin{cases}
			f(1) = e^{i\theta} = \cos\theta + i \sin\theta\\
			f(i) = ie^{i\theta} = -\sin\theta + i \cos\theta\\
		\end{cases}
	\] %todo graph
	\[z = x + iy \text{ avec } x,y \in \R\]
	\[
		\begin{pmatrix}
			\cos \theta & -\sin\theta\\
			\sin\theta & \cos\theta
		\end{pmatrix} 
		\begin{pmatrix}
			x\\y
		\end{pmatrix} = 
		\begin{pmatrix}
			x\cos \theta & -y\sin\theta\\
			x\sin\theta & y\cos\theta
		\end{pmatrix} 
	\] 
\end{exm}

\begin{thm}
	Soient $E$ et $F$ deux $\mathbbm{K}$-espaces vectoriels de dimension finie, $\mathcal{B} = (e_1, \ldots, e_n)$ une base de $E$ et $\mathcal{C} = (f_1, \ldots, f_p)$ une base de $F$.
	\begin{align*}
		\Phi: \mathcal{L}(E,F) &\longrightarrow \mathcal{M}_{p,n}(\mathbbm{K}) \\
		f &\longmapsto \Mat_{\mathcal{B},\mathcal{C}}(f)
	\end{align*}
	$\Phi$ est un $\mathbbm{K}$-isomorphisme linéaire.
\end{thm}

\begin{prv}
	\begin{itemize}
		\item Soient $(f,g) \in \mathcal{L}(E,F)^2$ et $(\lambda, \mu) \in \mathbbm{K}^2$. On pose $A = \Mat_{\mathcal{B},\mathcal{C}}(f) = \Phi(f) = (a_{i,j})$ et $B = \Mat_{\mathcal{B},\mathcal{C}}(g) = \Phi(g) = (b_{i,j})$. On pose également $C = \Mat_{\mathcal{B}, \mathcal{C}}(\lambda f + \mu g) = \Phi(\lambda f + \mu g) = (c_{i,j})$.
			\begin{align*}
				\forall j \in \left\llbracket 1,n \right\rrbracket, \begin{pmatrix}
					c_{1,j}\\
					\vdots\\
					c_{p,j}
				\end{pmatrix} &= \Mat_{\mathcal{C}}\big((\lambda f + \mu g)(e_j)\big)\\
					&= \Mat_{\mathcal{C}}\big(\lambda f(e_j) + \mu g(e_j)\big) \\
			\end{align*}
			Or, \[
				\Mat_{\mathcal{C}}\big(f(e_j)\big)  = \begin{pmatrix}
					a_{1,j}\\
					\vdots\\
					a_{p,j}
				\end{pmatrix}
			\] et \[
				\Mat_{\mathcal{C}}\big(g(e_j)\big) = \begin{pmatrix}
					b_{1,j}\\
					\vdots\\
					b_{p,j}
				\end{pmatrix} 
			\] Donc \[
				\Mat_\mathcal{C} \big(\lambda f(e_j) + \mu g(e_j)\big) = \lambda \begin{pmatrix}
					a_{1,j}\\
					\vdots\\
					a_{p,j}
				\end{pmatrix} + \mu \begin{pmatrix}
					b_{1,j}\\
					\vdots\\
					b_{p,j}
				\end{pmatrix}
			\] Donc $C = \lambda A + \mu B$ et donc $\Phi$ est linéaire.
		\item Soit $f \in \Ker\Phi$ : \[
				\Mat_{\mathcal{B},\mathcal{C}}(f) =
				\begin{pNiceMatrix}
					0 &\Cdots& 0\\
					\Vdots && \Vdots\\
					0 &\Cdots& 0
				\end{pNiceMatrix}
			\] % todo
			Donc \[
				\forall j \in \left\llbracket 1,n \right\rrbracket, f(e_j) = \sum_{i=1}^p 0 \cdot f_i = 0_F
			\] Soit $x \in E$. On pose $x = \sum_{j=1}^n x_j e_j$. \[
				f(x) = \sum_{j=1}^n x_j f(e_j) = 0_F
			\] Donc $f = 0$. Donc $\Phi$ est injective.\\
			Soit $A \in \mathcal{M}_{p,n}(\mathbbm{K})$. On pose $A = (a_{i,j})_{\substack{1\le i\le p\\1\le j\le n}}$.\\
			On définit $F : E \to  F$ de la fa\c con suivante : pour tout $x \in E$, on décompose $x = \sum_{j=1}^n x_j e_j$. On pose alors
			\begin{align*}
				f(x) &= \sum_{j=1}^n x_j \sum_{i=1}^p a_{i,j}f_i \\
				&= \sum_{i=1}^p \sum_{j=1}^n a_{i,j} x_j f_i \\
			\end{align*} Montrons que $f \in \mathcal{L}(E,F)$ et $\Phi(f) = A$ 
		\item Soit $(x,y) \in E^2$ et $(\alpha, \beta) \in \mathbbm{K}^2$. On pose \[
				\begin{cases}
					x = \sum_{j=1}^n x_j e_j &\text{ avec } (x_1, \ldots, x_n) \in \mathbbm{K}^n\\
					y = \sum_{j=1}^n y_j e_j &\text{ avec } (y_1, \ldots, y_n) \in \mathbbm{K}^n\\
				\end{cases}
			\] Donc \[
				\alpha x + \beta y = \sum_{j=1}^n (\alpha x_j + \beta y_j)(e_j
			\] D'où
			\begin{align*}
				f(\alpha x + \beta y) &= \sum_{i=1}^p \sum_{j=1}^n a_{i,j}(\alpha x_j + \beta x_j)f_i \\
				&= \alpha \sum_{i=1}^p \sum_{j=1}^n a_{i,j} x_j f_i + \beta \sum_{i=1}^p \sum_{j=1}^n b_{i,j} y_j f_i \\
				&= \alpha f(x) + \beta f(y) \\
			\end{align*}
			Soit $j \in \left\llbracket 1,p \right\rrbracket$. \[
				f(e_j) = \sum_{i=1}^p a_{i,j}f_i
			\] et \[
				\Mat_{\mathcal{C}}\big(f(e_j)\big) = \begin{pmatrix}
					a_{1,j}\\
					\vdots\\
					a_{p,j}
				\end{pmatrix} 
			\] donc \[
				\Mat_{\mathcal{B},\mathcal{C}}(f) = A
			\]
	\end{itemize}
\end{prv}

\begin{crlr}
	Si $E$ et $F$ sont deux $\mathbbm{K}$-espaces vectoriels de dimension finie, alors \[
		\dim\big(\mathcal{L}(E,F)\big) = \dim(E) \times \dim(F)
	\]
	\qed
\end{crlr}

\begin{exm}[\itshape Trouver tous les endomorphismes de $\R^2$]
	Soit $\mathcal{B} = (e_1, e_2)$ la base canonique de $\R^2$.
	\begin{align*}
		\R^2 &\longrightarrow \R^2 \\
		(x,y) &\longmapsto (ax + cy, bx + dy)
	\end{align*} avec $(a,b,c,d) \in \R^4$.\\
	Une base de $\mathcal{L}(\R^2)$ est $(f_1, f_2, f_3, f_4)$ où \[
		\begin{cases}
			f_1 : (x,y) \mapsto (x,0)\\
			f_2 : (x,y) \mapsto (0,x)\\
			f_3 : (x,y) \mapsto (y,0)\\
			f_4 : (x,y) \mapsto (0,y)\\
		\end{cases}
	\] 
\end{exm}

\begin{thm}
	Soient $f \in \mathcal{L}(E,F)$ et $g \in \mathcal{L}(F,G)$, $\mathcal{B}$ une base de $E$, $\mathcal{C}$ une base de $F$ et $\mathcal{D}$ une base de $G$. Alors \[
		\Mat_{\mathcal{B},\mathcal{D}}(g \circ f) = \Mat_{\mathcal{C},\mathcal{D}}(g) \times \Mat_{\mathcal{B},\mathcal{C}}(f)
	\] % todo
\end{thm}

\begin{prv}
	On pose
	\begin{align*}
		\mathcal{B} = (e_1, \ldots, e_n)\\
		\mathcal{C} = (f_1, \ldots, f_p)\\
		\mathcal{D} = (g_1, \ldots, g_q).
	\end{align*}
	et
	\begin{align*}
		A = (a_{i,j})_{\substack{1\le i\le p\\1\le j\le n}} = \Mat_{\mathcal{B},\mathcal{C}}(f)\\
		B = (b_{k,i})_{\substack{1\le k\le q\\1\le i\le p}} = \Mat_{\mathcal{C},\mathcal{D}}(g)\\
		C = (c_{k,j})_{\substack{1\le k\le q\\1\le j\le n}} = \Mat_{\mathcal{B},\mathcal{D}}(g \circ f)\\
	\end{align*}
	Soit $j \in \left\llbracket 1,n \right\rrbracket$. La $j$-ième colonne de $C$ est
	\begin{align*}
		\Mat_{\mathcal{D}}\big(g \circ f(e_j)\big) &= \Mat_{\mathcal{D}}\big(g\big(f(e_j\big)\big) \\
		&= \Mat_{\mathcal{C},\mathcal{D}}(g) \times \Mat_{\mathcal{C}}\big(f(e_j)\big) \\
		&= B A_j \\
	\end{align*} où $A_j$ est la $j$-ème colonne de $A$.\\
	Or, la $j$-ème colonne de $BA$ est aussi $BA_j$.\\
	Donc, $C = BA$.
\end{prv}
