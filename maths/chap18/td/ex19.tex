\part{Exercice 19 (supplémentaire)}

{\itshape
	Trouver tous les $P \in \R[X]$ tels que \[
		XP(X+1) = (X+4)P(X)
	\]
}

\begin{itemize}
	\item[\underline{\sc Analyse}] Soit $P \in \R[X]$ tel que \[
			X P(X+1) = (X+4) P(X)
		\] En remplaçant $X$ par 0, on obtient \[
			4P(0) = 0
		\] donc $0$ est racine de $P$. \\
		En replaçant $X$ par $-4$, on obtient \[
			-4P(-3) = 0
		\] donc $-3$ est une racine de $P$.\\
		En remplaçant $X$ par $-3$, on obtient \[
			-3 P(-2) = -P(-3) = 0
		\] donc $-2$ est racine de $P$.\\
		En replaçant $X$ par $-2$, on obtient \[
			-2P(-1) = -2P(-2) = 0
		\] donc $-1$ est racine de $P$. \\
		D'où, \[
			P = X(X+1)(X+2)(X+3)\,Q(X)
		\] avec $Q \in \R[X]$.\\
		Donc,
		\begin{align*}
			X (X+1)(X+2)(X+3)(X+4)\,Q(X+1)\\
			= (X+4)X(X+1)(X+2)(X+3)\,Q(X).
		\end{align*}
		Comme $\R[X]$ est intègre, \[
			Q(X) = Q(X+1).
		\] Si $Q$ n'est pas constant, alors $Q$ a au moins une racine $a \in \C$ et alors \[
			\forall k \in \N, Q(a+k) = Q(a) = 0.
		\] Donc, $Q$ a une infinité de racines donc $Q = 0$.\\
		Donc $Q$ est constant.
	\item[\underline{\sc Synthèse}] Soit $\lambda \in \R$ et \[
			P = \lambda X (X+1) (X+2) (X+3).
		\] On vérifie aisément que \[
			X P(X+1) = (X+4)P(X).
		\]
\end{itemize}
