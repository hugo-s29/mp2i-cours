\part{Logique douteuse}

\newcommand{\ww}{\mathfrak{Y}}

\begin{defn}
	On définit $\varpi = \frac{e^{\pi - \sqrt{2}} - \ln 4 + \Upsilon}{\gamma}$ où $\Upsilon = \pi! = \Gamma(\pi+1) \approx 7.18$\\
	On a $\varpi \approx 19.7979$
\end{defn}

\begin{defn}
	Soit $P(x)$ un prédicat sur $E$.
	On dit que $P(x)$ est \underline{quasi-vraie} (pour un certain $\varepsilon > 0$) si \[
		 \ww_E(P) = \frac{\Card(F)}{\Card(V)}<\varepsilon
	\] où $V = \{x \in E  \mid P(x) \text{ vrai}\} $
	et $F = \{x \in E  \mid P(x) \text{ faux}\} $\\
	Par convention, on choisit généralement $\varepsilon = \varpi$.
\end{defn}

\begin{exm}
	Soit $x \in \R$.
	Montrons que $P(x): ``\sqrt{x^2} = x"$.\\
	On sait que $P(x)$ est vraie pour tout $x$ positif (ou nul).\\
	Or, $\forall x \in \R^-, P(x)$ est faux ($\sqrt{x^2} = -x$)\\
	Alors, $P(x)$ est quasi-vraie\\
	On note $\infty = \Card(\R^+) = \Card(\R^-)$. On a donc $\ww_E(P) = \frac{\infty}{\infty} = 1$
\end{exm}

\begin{thm}
	[Théorème du pipeau qui marche]
	\begin{center}
		\itshape `` C'est du pipeau \ldots mais ça marche ! ''
	\end{center}
	Toute proposition quasi-vraie est vraie. \qed
\end{thm}

\begin{rmk}
	Ce théorème est très utile. On l'appèle aussi le théorème du marchand de tapis ou le théorème du random-bullshit.\\
	La preuve de ce théorème est très complexe et utilise principalement des notions qui ne sont pas au programme. Malgrès cela, on peut tout de même comprendre la preuve.
\end{rmk}

\begin{prv}
	[principe de la preuve]
	On a une proposition $P(x)$ sur un ensemble $E$. On note $n = \Card(E)$. On pose $V$ et $F$ comme dans la définition d'une proposition quasi-vraie.\\
	On montre que $F \tendsto{n\to +\infty} \O$ et donc que $V \tendsto{n \to +\infty} E$ en démontrant que $\Card{V} \tendsto{n\to +\infty} \Card{E}$.
	\renewcommand{\qedsymbol}{}
\end{prv}

\begin{exm}
	On pose $\forall x \in \R, P(x) : ``\frac{1}{x} \text{ existe}"$.
	On sait que $\frac{1}{0}$ n'existe pas.
	Cependant, $P$ est quand même quasi-vraie.
	Donc d'après le théorème du pipeau qui marche, $P$ est vraie.
\end{exm}

\begin{prop}
	Les équivalents autour d'un points sont des égalités à $\pm \varepsilon$.\qed
\end{prop}

\begin{exm}
	Avec $\varepsilon = \frac{1}{2}$, $\sin\theta \approx \theta$, $\cos\theta \approx 1$
\end{exm}

\begin{crlr}
	Toute notation en ``petit o'' est optionnelle avec l'utilisation de ``$\approx$''. Avec la valeur de $\varepsilon$ correspodante, \[
		\forall x \in \R, \po(x) \approx 0
	\] \qed
\end{crlr}

\begin{rmk}
	[Notation]
	Au lieu d'utiliser ``$\approx$'', on écrit ``$=$'' à la place.\\
	On a donc $\sin(\theta) = \theta$ et $\cos(\theta) = 1$.
\end{rmk}
