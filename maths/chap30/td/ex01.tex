\part{Exercice 1}

Notons $\varphi : x \mapsto \int_{x}^{x+T} f(t)~\mathrm{d}t$.

On pose $F$ une primitive de $f$ ($f$ est continue). D'où \[
	\forall x,\,\varphi(X) = F(x+T) - F(x)
.\] On dérive cette expression : \[
	0 = \varphi'(x) = f(x+T) - f(x)
.\]

Ainsi, $f$ est $T$-périodique.

