\part{Déduction naturelle}
Dans ce paragraphe, $A$ et $B$ sont deux propositions.\\

\vspace{5mm}
\underline{Comment démontrer $A\et B$ ?}\marginpar{\fbox{$A\et B$}}
\begin{itemize}
	\item On démontre $A$ 
	\item On démontre $B$
\end{itemize}
\vspace{2mm}

\underline{Comment utiliser l'hypothèse $A \et B$ ?}\\
On utilise $A$ ou on utilise $B$.\\

\vspace{5mm}
\underline{Comment démontrer $A\ou B$ ?}\marginpar{\fbox{$A\ou B$}}\\
On essaie de démontrer $A$. Si on y arrive, alors on a prouvé $A\ou B$ sinon on démontre $B$.\\

\underline{Variante}\\
On suppose $A$ faux. On démontre $B$.\\
\vspace{2mm}

\underline{Comment utiliser l'hypothèse $A \ou B$ ?}\\
On fait une disjonction des cas:\\
\begin{itemize}
	\item {\sc Cas 1 : } On suppose $A$\\
	\item {\sc Cas 2 : } On suppose $B$\\
\end{itemize}

\vspace{5mm}
\underline{Comment démontrer $A\implies B$ ?}\marginpar{\fbox{$A\implies B$}} \\
On suppose $A$. On démontre $B$.\\
\vspace{2mm}

\underline{Comment utiliser l'hypothèse $A \implies B$ ?}\\
On démontre $A$. On utilise $B$.\\
