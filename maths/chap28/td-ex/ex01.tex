\part{Exercice 1}

On sait que $\R^3$ est un $\R$-espace affine. On pose $A = (0, 1, 0) \in F$ et
\begin{align*}
	\vec{F} &= \left\{ (x,y,z) \in \R^3  \mid x + y + z = 0 \et 2x + y + z = 0 \right\}\\
	&= \left\{ (0, y, z)  \mid (y,z) \in \R^2 \et y = -z \right\} \\
	&= \left\{ (0, -z, z)  \mid z \in \R \right\}. \\
\end{align*}

$\vec{F}$ est un $\R$-espace vectoriel. En effet, pour tout $(0, -a, a),\,(0,-b,b) \in \vec{F}$, pour tout $\alpha, \beta \in \R$, on a 
\begin{align*}
	\alpha (0, -a, a) + \beta (0, -b, b) &= \big(0, -(\alpha a + \beta b), \alpha a + \beta b\big) \in F \\
\end{align*}


