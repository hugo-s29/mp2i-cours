\part{Exercice 7}

Soit $\mathcal{B} = (e_1,e_2,e_3)$ la base canonique de $E$.

\begin{itemize}
	\item $p(e_1) = \frac{5}{6}e_1 -\frac{2}{6} e_2 + \frac{1}{6}e_3$;
	\item $p(e_2) = -\frac{2}{6}e_1 + \frac{2}{6} e_2 + \frac{2}{6}e_3$;
	\item $p(e_3) = \frac{1}{6}e_1 + \frac{2}{6}e_2 + \frac{5}{6}e_3$.
\end{itemize}

On remarque $p(e_1) + 2p(e_2) = \frac{1}{6}e_1 + \frac{2}{6}e_2 + \frac{5}{6}e_3 = p(e_3)$.

D'où $\mathrm{Im}(p) = \Vect\big(p(e_1),p(e_2)\big)$.

