\part{Intégrale d'une fonction en escaliers}

\begin{defn}
	Une \underline{subdivision} du segment $[a,b]$ est une suite finie $a = x_0 < x_1 < x_2 < \cdots < x_n = b$.
	\index{subdivision d'un segment}

	\begin{figure}[H]
		\centering
		\begin{asy}
			real a = -5;
			real b = 5;
			size(6cm);
			draw((a,0) -- (b,0));
			real epsx = 0.3;
			real epsy = 0.6;

			void bracket(real x, int sign) {
				draw((x + sign * epsx, epsy) -- (x, epsy) -- (x, -epsy) -- (x + sign * epsx, -epsy));
			}

			bracket(a, 1);
			bracket(b, -1);

			real[] alphas = {0.1, 0.35, 0.6, 0.7, 0.9};

			label("$\substack{\displaystyle a\\[0.4mm]\displaystyle \vrt=\\[0.4mm]\displaystyle x_0}$", (a, -0.8epsy), align=S);
			label("$\substack{\displaystyle b\\[0.4mm]\displaystyle \vrt=\\[0.4mm]\displaystyle x_6}$", (b, -0.8epsy), align=S);

			epsy /= 2;

			for(int i = 0; i < alphas.length; ++i) {
				real alpha = alphas[i];
				real x = b * alpha + a * (1 - alpha);
				bracket(x, 0);
				label("$x_" + string(i+1) + "$", (x, -2epsy));
			}
		\end{asy}
	\end{figure}
	\vspace{3mm}
\end{defn}

\begin{rmk}[Notation]
	Dans ce chapitre, l'ensemble des subdivisions de $[a,b]$ est noté $\mathfrak{S}_{[a,b]}$.
\end{rmk}

\begin{defn}
	Soit $f : [a,b] \longrightarrow \R$.

	On dit que $f$ est en \underline{escalier} s'il existe $\sigma = (x_0, \ldots, x_n) \in \mathfrak{S}_{[a,b]}$ et $(c_0, \ldots, c_{n-1}) \in \R^n$ tels que \[
		\forall i \in \left\llbracket 0,n-1 \right\rrbracket,\;\forall x \in ]x_i,\,x_{i+1}[,\;f(x) = c_i
	.\] On dit alors que $\sigma$ est \underline{adaptée} à $f$.
	\index{fonction en escalier}
	\index{escalier (fonction)}
	\index{subdivision adaptée}
	\index{adaptation (subdivision)}

	\begin{figure}[H]
		\centering
		\begin{asy}
			import graph;

			real a = -0.3;
			real b = 5;
			size(6cm);
			draw((a-0.5,0) -- (b+1,0), Arrow(TeXHead));
			draw((0,-0.5)--(0, 4), Arrow(TeXHead));
			real eps = 0.15;

			real[] alphas = {0,0.1, 0.35, 0.5, 0.7, 0.9,1};
			real[] constants = {0.2, 0.2, 0.4, 0.9, 0.7};

			void tick(real x) { draw((x, -eps) -- (x, eps)); }

			real f(real x) {
				real alpha = (x-a) / (b-a);
				int index = 0;
				while(index < alphas.length && alphas[index] <= alpha) { ++index; }
				if(index <= constants.length && index > 0)
					return constants[index - 1] * 4;
				else
					return constants[constants.length - 1] * 4;
			}

			for(int i = 0; i < alphas.length; ++i) {
				real alpha = alphas[i];
				real x = b * alpha + a * (1 - alpha);
				tick(x);
				label("$x_" + string(i) + "$", (x, -2eps));
				draw((x, 0) -- (x, min(f(x), f(x - 0.2))), dashed);
			}

			draw(graph(f, a, b, 1000), magenta);
		\end{asy}
	\end{figure}
\end{defn}

\begin{defn}
	Soient $\sigma = (x_0, x_1, \ldots, x_n) \in \mathfrak{S}_{[a,b]}$ et $\sigma' = (x_0', x_1', \ldots, x_p') \in \mathfrak{S}_{[a,b]}$. On dit alors que $\sigma'$ est \underline{plus fine}\index{subdivision plus fine}\index{finesse (subdivision)} que $\sigma$ si $\{x_0, x_1, \ldots, x_n\} \subset \{x_0',x_1', \ldots, x_n'\}$. On note alors $\sigma' \prec \sigma$.
\end{defn}

\begin{prop}
	Soient $\sigma_1, \sigma_2$ deux subdivisions de $[a,b]$. Alors il existe une subdivision $\sigma_3$ plus fine que $\sigma_1$ et $\sigma_2$.
\end{prop}

\begin{prv}
	Soit $\sigma_1 = (x_0, x_1, \ldots,x_n) \in \mathfrak{S}_{[a,b]}$ et $\sigma_2 = (x_0', x_1', \ldots, x_p')\in \mathfrak{S}_{[a,b]}$.

	On pose $A = \{x_0,\ldots,x_n\} \cup \{x_0',\ldots,x_p'\}$. On ordonne dans l'ordre croissant les éléments de $A$ : \[
		A = \{x_0'',x_1'',\ldots,x_q''\}
	\] avec $a = x_0'' < x_1'' < \cdots < x_q'' = b$. On pose $\sigma_3 = (x_0'', x_1'', \ldots, x_q'') \in \mathfrak{S}_{[a,b]}$. On a bien $\sigma_3 \prec \sigma_2$ et $\sigma_3 \prec \sigma_1$.
\end{prv}

\begin{prop-defn}
	Soit $f: [a,b] \longrightarrow \R$ une fonction en escaliers.
	Soit $\sigma = (x_0, \ldots, x_n)$ une subdivision adaptée à $f$. Pour $i \in \left\llbracket 0,n-1 \right\rrbracket$, on pose $c_i$ la valeur constante de $f$ sur $]x_i, x_{i+1}[$.

	Alors, \[
		\sum_{i=1}^{n-1} (x_{i+1}- x_i) c_i
	\] ne dépend pas de la subdivision adaptée. On dit que c'est l'\underline{intégrale} de $f$ sur $[a,b]$. On note ce nombre $\int_{[a,b]} f$.
	\index{intégrale (fonction en escaliers)}

	\begin{figure}[H]
		\centering
		\begin{asy}
			import graph;
			import patterns;

			add("p1", hatch(2mm, magenta));
			add("p2", hatch(2mm, NW, magenta));

			real a = -0.3;
			real b = 5;
			size(6cm);
			draw((a-0.5,0) -- (b+1,0), Arrow(TeXHead));
			draw((0,-0.5)--(0, 4), Arrow(TeXHead));
			real eps = 0.15;

			real[] alphas = {0,0.1, 0.35, 0.5, 0.7, 0.9,1};
			real[] constants = {0.2, 0.2, 0.4, 0.9, 0.7};

			void tick(real x) { draw((x, -eps) -- (x, eps)); }

			real f(real x) {
				real alpha = (x-a) / (b-a);
				int index = 0;
				while(index < alphas.length && alphas[index] <= alpha) { ++index; }
				if(index <= constants.length && index > 0)
					return constants[index - 1] * 4;
				else
					return constants[constants.length - 1] * 4;
			}

			for(int i = 0; i < alphas.length; ++i) {
				real alpha = alphas[i];
				real x = b * alpha + a * (1 - alpha);
				real y = min(f(x), f(x - 0.2));
				tick(x);
				label("$x_" + string(i) + "$", (x, -2eps));
				draw((x, 0) -- (x, y), dashed);

				if(i == alphas.length - 1) { continue; }

				real alpha2 = alphas[i + 1];
				real x2 = b * alpha2 + a * (1 - alpha2);
				y = f(x);
				fill((x2, 0) -- (x, 0) -- (x, y) -- (x2, y) -- cycle, pattern("p" + string(i % 2 + 1)));
			}

			draw(graph(f, a, b, 1000), magenta);
		\end{asy}
	\end{figure}
\end{prop-defn}

\begin{prv}
	Soit $\sigma ' = (x_0', \ldots, x_p')$ une subdivision adaptée à $f$. On considère $\sigma''$ une subdivision de $[a,b]$ plus fine que $\sigma$ et $\sigma'$. On pose
	\begin{align*}
		\phantom{\sigma'' = (a, x_{1,1}, x_{1,2},\ldots,}\mathrlap{x_{2,0}}\phantom{x_{1,i_1},x_{2,1},\ldots,}\mathrlap{x_{3,0}}\phantom{x_{2,i_2},\ldots,b)}\\[-1mm]
		\phantom{\sigma'' = (a, x_{1,1}, x_{1,2},\ldots,}\mathrlap{~~\vrt=}\phantom{x_{1,i_1},x_{2,1},\ldots,}\mathrlap{~~\vrt=}\phantom{x_{2,i_2},\ldots,b)}\\[-2mm]
		\sigma'' = (a, x_{1,1}, x_{1,2},\ldots,x_{1,i_1},x_{2,1},\ldots,x_{2,i_2},\ldots,b)\\
		\phantom{\sigma'' = (a, x_{1,1}, x_{1,2},\ldots,}\mathrlap{~~\vrt=}\phantom{x_{1,i_1},x_{2,1},\ldots,}\mathrlap{~~\vrt=}\phantom{x_{2,i_2},\ldots,b)}\\[-2mm]
		\phantom{\sigma'' = (a, x_{1,1}, x_{1,2},\ldots,}\mathrlap{~x_1}\phantom{x_{1,i_1},x_{2,1},\ldots,}\mathrlap{~x_2}\phantom{x_{2,i_2},\ldots,b)}
	\end{align*}
	On a \[
		\forall k \in \left\llbracket 0,n-1 \right\rrbracket,\, \forall j \in \left\llbracket 0,u_k - 1 \right\rrbracket,\,\forall x \in \,]x_{k,j},x_{k,j+1}[,\,f(x) = c_k
	.\]
	\begin{align*}
		\sum_{k=0}^{n-1}\sum_{j=0}^{i_k - 1} (x_{k,j+1} - x_{k,j}) c_k
		&= \sum_{k=0}^{n-1}c_k \sum_{j=0}^{i_k - 1} (x_{k,j+1} - x_{k,j}) \\
		&= \sum_{k=0}^{n-1}(x_{k,i_k} - x_{k,0}) \\
		&= \sum_{k=0}^{n-1}c_k (x_{k+1} - x_k) \\
	\end{align*}

	De même, comme $\sigma \prec \sigma'$, on a aussi l'égalité avec $\sum_{k=0}^{p-1} c_k'(x'_{k+1} - x'_k)$ où $c'_k$ est la valeur de $f$ sur $]x'_k,x'_{k+1}[$.
\end{prv}

Pour définir l'intégrale d'une fonction continue, on peut utiliser deux méthodes :
\begin{enumerate}
	\item L'intégrale d'une fonction continue est la limite d'une suite de fonctions en escaliers. Ces résultats seront vus l'année prochaine en {\it MPI}.
	\item Les sommes de Darboux; c'est la définition que l'on va utiliser.
\end{enumerate}

