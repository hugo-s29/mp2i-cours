\part{Exercice 3}

On pose
\[
\begin{cases}
	\ell_1 = \lim u_{2n}\\
	\ell_2 = \lim u_{2n + 1}\\
	\ell_3 = \lim u_{3n}\\
\end{cases}
\] 
$(u_{6n})_{n\in\N}$ est une sous-suite de $(u_{2n})$ et de $(u_{3n})$ donc

\begin{align*}
	u_{6n} &\tendsto{n\to +\infty} \ell_1\\
	u_{6n} &\tendsto{n\to +\infty} \ell_3
\end{align*}
	
Par unicité de la limite, $\ell_1 = \ell_3$.\\
$(u_{6n+3})_{n\in\N}$ est une sous suite de $(u_{2n+1})$ et de $(u_{3n})$ donc $\ell_2 = \ell_3$\\

D'où $\ell_1 = \ell_2$, $(u_n)$ converge.
