\part{Séries à termes positifs}

\begin{prop}
	Soit $(u_n) \in \big(\R^+\big)^\N$. Alors $\left( \sum_{k=0}^n u_k \right)_{n \in \N}$ est croissante.
\end{prop}

\begin{prv}
	\[
		\forall n \in \N,\, \sum_{k=0}^{n+1} u_k - \sum_{k=0}^n u_k = u_{n+1} \ge 0.
	\]
\end{prv}

\begin{thm}
	Soient $u$ et $v$ deux suites réelles telles que \[
		\forall n \in \N, \, \color{red} \boxed{\color{black} 0 \le u_n } \color{black} \le v_n.
	\]
	\begin{enumerate}
		\item {\large\sc\color{orange} Si} $\Sigma v_n$ converge, {\large\sc\color{orange} Alors} $\Sigma u_n$ converge\\
		\item {\large\sc\color{orange} Si} $\Sigma u_n$ diverge, {\large\sc\color{orange} Alors} $\Sigma v_n$ diverge.
	\end{enumerate}
\end{thm}

\begin{prv}
	\begin{enumerate}
		\item On suppose que $\Sigma v_n$ converge. On note \[
				\forall n \in \N, S_n(v) = \sum_{k=0}^n v_k.
			\] Donc $\big(S_n(v)\big)$ est majorée. Soit $V$ un majorant et $n \in \N$.

			\[\forall k, 0 \le u_k \le v_k\] donc \[
				\sum_{k=0}^n u_k \le \sum_{k=0}^n v_k = S_n(v) \le V
			\] donc $\left(\sum_{k=0}^n u_k\right)_{n\in \N}$ est majorée.
			Or, elle est croissante, donc elle converge.
		\item C'est la contraposée du 1.
	\end{enumerate}
\end{prv}

\begin{cexm}
	$\textstyle \sum \frac{1}{n}$ diverge,
	$\textstyle \sum \frac{1}{n^2}$ converge (vers $\textstyle \frac{\pi^2}{6}$) et \[
		\forall n \in \N,\, 0\le \frac{1}{\textstyle n^2} \le \frac{1}{n}.
	\] 
\end{cexm}

\begin{crlr}
	Soient $u,v$ deux suites réelles {\large\sc\color{orange} positives} telles que $u = O(v)$.

	\begin{enumerate}
		\item Si $\Sigma v_n$ converge, alors $\Sigma u_n$ converge.
		\item Si $\Sigma u_n$ diverge, alors $\Sigma v_n$ diverge.
	\end{enumerate}
	\qed
\end{crlr}

\begin{thm}
	Soient $u$ et $v$ deux suites réelles {\large\sc\color{orange} positives} telles que $u = \po(v)$.

	\begin{enumerate}
		\item Si $\Sigma v_n$ converge, alors $\Sigma u_n$ converge.
		\item Si $\Sigma u_n$ diverge, alors $\Sigma v_n$ diverge.
	\end{enumerate}
	\qed
\end{thm}

\begin{thm}
	[règle des équivalents] Soient $u$ et $v$ deux suites réelles {\large\sc\color{orange} positives} telles que $u \sim v$. Alors \[
		\Sigma u_n \text{ converge} \iff \Sigma v_n \text{ converge}.
	\] 
\end{thm}

\begin{prv}
	On suppose \[
		u_n = v_n + \po(v_n)
	\] et donc \[
		\forall \varepsilon > 0, \exists N \in \N, \forall n \ge N, |u_n - v_n| \le \varepsilon |v_n|
	\] En particulier, on peut considerer $N \in \N$ tel que \[
		\forall n \ge N, -\frac{1}{2} v_n \le u_n - v_n \le \frac{1}{2} v_n
	\] et donc \[
		\forall n \ge N, \frac{1}{2} v_n \le u_n \le \frac{3}{2} v_n
	\] et donc $\begin{cases}
		u = O(v),\\
		v = O(u).
	\end{cases}$ 

	Si $\Sigma v_n$ converge, alors $\Sigma u_n$ converge car $u = O(v)$.

	Si $\Sigma u_n$ converge, alors $\Sigma v_n$ converge car $v = O(u)$.
\end{prv}

\begin{exm}
	{\itshape Déterminer la nature de $\sum \frac{1}{n^3 + n \ln(n)}$ ?}
	
	\[
		\forall n \ge 3, \frac{1}{n^3 + n\ln n} \ge 0.
	\] et \[
		\frac{1}{n^3 + n\ln n} \simi_{n\to +\infty} \frac{1}{n^3}.
	\]

	\[
		\forall n \ge 1, \frac{1}{n^3} \le \frac{1}{n^2}
	\]

	$\textstyle \sum\frac{1}{n^2}$ converge donc $\textstyle \sum \frac{1}{n^3}$ converge donc $\sum \frac{1}{n^3 + n \ln n}$ converge.
\end{exm}
