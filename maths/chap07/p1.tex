\begin{defn}
	Soit $f$ une fonction définie au voisinage d'un point $a \in \R$.

	On dit que $f$ possède un \underline{développment limité d'ordre $n$ au voisinage de $a$} s'il existe des réels $(c_0, \ldots, c_n) \in \R^{n+1}$ tels que \[
		f(x) = c_0 + c_1(x-a) + c_2(x-a)^2 + \cdots + c_n (x-a)^{n} + \po_{x\to a}\big((x-a)^n\big).
	\]

	En particulier, avec $a = 0$, on a \[
		f(x) = \underbrace{c_0 + c_1x + c_2x^2 + \underset{\phantom{x}}{\cdots} + c_n x^n}_{\text{\underline{développement de Taylor}}} + \underbrace{\po_{x\to 0}(x^n)}_{\text{\underline{reste}}}.
	\]

	\index{développement limité d'ordre $n$ au voisinage de $a$}
\end{defn}

\begin{thm}[Taylor-Young]
	Si $f$ est de classe $\mathcal{C}^n$ (i.e. $f$ définie et dérivable $n$ fois et $f^{(n)}$ est continue) au voisinage de $a$, alors $f$ admet un développement limité d'ordre $n$ au voisinage de $a$ est \[
		f(x) = f(a) + (x-a)f'(a) + (x-a)^2 \frac{f''(a)}{2!} + \cdots + (x-a)^n \frac{f^{(n)}(a)}{n!} + \po_{x\to a}\big((x-a)^n\big).
	\] \qed
\end{thm}

\begin{rmk}
	Cette formule est à éviter en pratique : il est bien trop difficile de calculer $f^{(n)}$ pour tout $n$.

	Cependant, on peut quand même en déduire le développement limité de $\exp$, $\cos$, $\sin$ et $x\mapsto (1+x)^\alpha$ en 0.
\end{rmk}

\begin{crlr}
	\begin{align*}
		&e^{x} = 1 + x + \frac{x^2}{2} + \cdots + \frac{x^n}{n!} + \po_{x\to 0}(x^n),\\
		&\cos x = 1 - \frac{x^2}{2} + \frac{x^4}{4!} - \frac{x^6}{6!} + \cdots + (-1)^n \frac{x^{2n}}{(2n)!} + \po_{x\to 0}(x^{2n}),\\
		&\sin x = x - \frac{x^3}{3!} + \frac{x^5}{5!} - \frac{x^7}{7!} + \cdots + (-1)^n \frac{x^{2n+1}}{(2n+1)!} + \po_{x\to 0}(x^{2n+1}),\\
		\forall \alpha \in \R,\, &(1+x)^\alpha = 1 + \alpha x + \alpha (\alpha -1) \frac{x^2}{2!} + \alpha(\alpha-1)(\alpha-2)\frac{x^3}{3!} + \cdots\\
		&\qquad+ \alpha(\alpha-1)\cdots\big(\alpha - (\alpha- 1)\big) \frac{x^n}{n!} + \po_{x\to 0}(x^n).
	\end{align*}
\end{crlr}

\begin{rmk}
	Avec $\alpha = -1$, on obtient le développement limité de $\frac{1}{1+x}$ en 0 : \[
		\frac{1}{1+x} = 1 - x + x^2 - x^3 + \cdots + (-1)^nx^n + \po_{x\to 0}(x^n).
	\]

	On en déduit donc le développement limité en $0$ de $\frac{1}{1-x}$ : \[
		\frac{1}{1-x} = 1 + x + x^2+ x^3 + \cdots + x^n + \po_{x\to 0}(x^n)
	\]

	Avec $\alpha = \frac{1}{2}$, on obtient le développement limité à l'ordre 2 de $\sqrt{1+x}$ : \[
		\sqrt{1+x}  = 1 + \frac{x}{2} - \frac{x^2}{8} + \po_{x\to 0}(x^2).
	\]
\end{rmk}

\begin{thm}[primitivation]
	Soit $f$ une fonction continue en $a$ ayant un développement limité d'ordre $n$ au voisinage de $a$. Soient $(c_0, c_1, \ldots, c_n)\in \R^{n+1}$ tels que \[
		f(x)  = c_0  + c_1(x-a) + c_2(x-a)^2 + \cdots + c_n(x-a)^n + \po_{x\to a}\big((x-a)^n\big).
	\]

	Soit $F$ une primitive de $f$. Alors $F$ a un développement limité d'ordre $n+1$ au voisinage de $a$ et \[
		F(x) = F(a) = c_0(x-a) + c_1 \frac{(x-a)^2}{2} + \cdots + c_n\frac{(x-a)^{n+1}}{n+1} + \po_{x\to a}\big((x-a)^{n+1}\big).
	\]\qed
\end{thm}

\begin{crlr}
	En primitivant le développement limité de $\frac{1}{x+1}$, on obtient le développement limité de $\ln(1+x)$ : \[
		\ln(1+x) = x - \frac{x^2}{2} + \frac{x^3}{3} + \cdots + (-1)^n \frac{x^{n+1}}{n+1} + \po_{x\to 0}(x^{n+1}).
	\] On en déduit aussi le développement limité de $\Arctan$ : \[
		\Arctan x = x - \frac{x^3}{3} + \frac{x^5}{5} - \frac{x^7}{7}+ \cdots + (-1)^n \frac{x^{2n+1}}{2n+1} + \po_{x\to 0}(x^{2n+1}).
	\] 
	\qed
\end{crlr}

\begin{exo}[{\itshape Calculer $\mathit{DL_5(0)}$ de $\tan$}]~\\[-3mm]
	\begin{itemize}
		\item[\sc Méthode 1] (quotient) : $\tan x = \frac{\sin x}{\cos x}$.

			On a \[
				\begin{cases}
					\sin x = x - \frac{x^3}{6} + \frac{x^5}{120} + \po(x^5),\\
					\cos x = 1 - \frac{x^2}{2} + \frac{x^4}{24} + \po(x^5).
				\end{cases}
			\]

			On calcule d'abord le développement limité de $\frac{1}{\cos x}$ :
			\begin{align*}
				\frac{1}{\cos x} &= \frac{1}{1 - \frac{x^2}{2} + \frac{x^4}{24} + \po(x^5)} \\
				&= \frac{1}{1+u} \text{ avec } u = -\frac{x^2}{2} + \frac{x^4}{24} + \po(x^5)\tendsto{x \to 0} 0 \\
				&= 1 - u + u^2 + \po(u^2) \\
				&= 1 - \left( -\frac{x^2}{2} + \frac{x^4}{24} + \po(x^5) \right) \\
				&\phantom{ = 1 -} \mathclap{+} \phantom{+} \left( -\frac{x^2}{2} + \frac{x^4}{24} + \po(x^5) \right)\\
				&\phantom{ = 1 -} \mathclap{+} \phantom{+} \po\left( \left( -\frac{x^2}{2} + \frac{x^4}{24} + \po(x^5) \right)^2 \right)\\
				&= 1+\frac{x^2}{2} -\frac{x^4}{24} + \frac{x^4}{4} + \po(x^5) \\
				&= 1 + \frac{x^2}{2} + \frac{5x^4}{24} + \po(x^5). \\
			\end{align*}

			On en déduit le développement limité de $\tan x$ :
			\begin{align*}
				\tan x &= \left(x - \frac{x^3}{6}+ \frac{x^5}{120} + \po(x^5)\right) \left( 1 + \frac{x^2}{2} + \frac{5x^4}{24} + \po(x^5) \right) \\
				&= x + \frac{x^3}{2} + \frac{5x^5}{24} - \frac{x^3}{6} - \frac{x^5}{12} + \frac{x^5}{120} + \po(x^5) \\
				&= x + \frac{x^3}{3} + \frac{2x^5}{15} + \po(x^5). \\
			\end{align*}

			{
				\color{red}
				À connaître:
				\[
					\boxed{\color{black} \tan x  = x + \frac{x^3}{3} + \po_{x\to 0}(x^3).}
				\]
			}
		\item[\sc Méthode 2](déterminer les coefficiants)

			On identifie les coefficants: 
			\begin{align*}
				\sin x &= (\tan x)(\cos x)\\
				&= \big(c_0 + c_1x + c_2x^2 + c_3x^3 + c_4x^4 + c_5x^5 + \po(x^5)\big)\\
				&\times \left( 1 - \frac{x^2}{2} + \frac{x^4}{24} + \po(x^5) \right)\\
				&= c_0 + c_1x + c_2 x^2 + c_3x^3 + c_4x^4 + c_5x^5 + \po(x^5)\\
				&- c_0 \frac{x^2}{2}-c_1 \frac{x^3}{2} - c_2 \frac{x^4}{2} - c_3 \frac{x^5}{2} + c_0\frac{x^4}{24} + c_1 \frac{x^5}{24} \\
			\end{align*}

			Par unicité du développement limité, \[
				\begin{cases}
					c_0 = 0\\
					c_1 = 1\\
					c_2 - \frac{c_0}{2} = 0\\
					c_3 - \frac{c_1}{2} = -\frac{1}{6}\\
					c_4 - \frac{c_2}{2} + \frac{c_0}{24} = 0\\
					c_5 - \frac{c_3}{2} + \frac{c_1}{24} = \frac{1}{120}
				\end{cases}
			\] et donc \[
				\begin{cases}
					c_0 = c_2 = c_4 = 0\\
					c_1 = 1\\
					c_3 = \frac{1}{3}\\
					c_5 = \frac{2}{15}
				\end{cases}
			\]
		\item[\sc Méthode 3](primitivation)

			On sait que \[
				\frac{\tan x - \tan 0}{x - 0} \tendsto{x\to 0} \tan'(0) = 1
			\] donc $\tan x \sim x$ et donc $\tan x = x + \po(x)$.

			D'où
			\begin{align*}
				\tan'(x) = 1 + \tan^2(x) &= 1 + \big(x + \po(x)\big)^2 \\
				&= 1 + x^2 + \po(x^2). \\
			\end{align*}
			En intégrant, on en déduit que \[
				\tan x = \tan 0 + x + \frac{x^3}{3} + \po(x^3)
			\]

			Donc,
			\begin{align*}
				\tan'(x) &= 1 + \tan^2 x \\
				&= 1 + \left( x + \frac{x^3}{3} + \po(x^3) \right)^2 \\
				&= 1 + x^2 + \frac{x^6}{9} + \po(x^6) + 2 \frac{x^4}{3} + \po(x^4) + \po(x^6) \\
				&= 1 + x^2 + \frac{2}{3}x^4 + \po(x^4) \\
			\end{align*}

			On en déduit donc le développement limité à l'ordre $5$ de $\tan$ : \[
				\tan x = \underbrace{\tan 0}_{\substack{\,\vrt=\\0}} + \frac{x^3}{3} + \frac{2}{15}x^5 + \po(x^5)
			\] 
	\end{itemize}
\end{exo}


