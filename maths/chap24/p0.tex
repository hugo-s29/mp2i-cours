\begin{defn}
	Soit $n \in \N^*$.

	Le \underline{groupe symétrique} est noté $S_n$ : l'ensemble des permutations de $\left\llbracket 1,n \right\rrbracket$ muni de $\circ$.

	\[
		\#S_n = n!
	\]
	\index{groupe symétrique}
\end{defn}

\part{Mise en situation}

\begin{center}
	{\Large \scshape Bon mélange d'un jeu de cartes :}
\end{center}

Soit un jeu neuf de $N$ cartes. On procède à un \underline{mélange par insertion} :
on place la carte qui est au-dessus n'importe où dans le paquet, étape que l'on répète $t$ fois.
\vspace{3mm}

\begin{center}
	{\itshape Pour quelles valeurs de $t$ obtient-on un jeu bien mélangé ?}
\end{center}

\vspace{3mm}
\underline{Modélisation} : On numérote les cartes de $1$ à $N$ dans l'ordre initial du jeu.

\begin{figure}[H]
	\centering
	\begin{asy}
		import three;

		size(8cm);

		settings.render = 0;
		settings.prc = false;
		currentprojection = obliqueX;

		real a = 1;
		real b = 16/9;

		guide3 card = (-b,a,0) -- (-b,-a,0) -- (b,-a,0) -- (b,a,0) -- cycle;

		void drawCardAt(triple A, string s, bool small = false) { 
			guide3 p = shift(A) * card;
			draw(surface(p), white);
			draw(p, black);

			if(small) {
				label("$" + s + "$", A + (b, a + 0.3, 0), fontsize(3), align=NW);
			} else {
				label("$" + s + "$", A + (b, a, 0), fontsize(6), align=NW);
			}

		}

		pen dots = linetype(new real[] {0,5})+2*linewidth(currentpen);

		triple A = (0, 0, 0);
		drawCardAt(A + (0, 0, 3), "3");
		drawCardAt(A + (0, 0, 4), "2");
		drawCardAt(A + (0, 0, 5), "1");
		draw(A + (0,0,1.5) -- A + (0,0,-11.5), dots);
		drawCardAt(A + (0,0, -13), "N");

		triple B = (0, 10, 0);
		drawCardAt(B + (0, 0, 4), "3");
		drawCardAt(B + (0, 0, 5), "2");
		draw(B + (0,0,2.5) -- B + (0,0,-1.5), dots);
		drawCardAt(B + (0,0,-5), "k+1", true);
		drawCardAt(B + (0,0,-4), "1");
		drawCardAt(B + (0,0,-3), "k");
		draw(B + (0,0,-6.5) -- B + (0,0,-11.5), dots);
		drawCardAt(B + (0,0, -13), "N");

		triple C = (0, 20, 0);
		drawCardAt(C + (0, 0, 4), "4");
		drawCardAt(C + (0, 0, 5), "3");
		draw(C + (0,0,2.3) -- C + (0,0,1.7), dots);
		drawCardAt(C + (0,0,-2), "\ell+1", true);
		drawCardAt(C + (0,0,-1), "2");
		drawCardAt(C + (0,0, 0), "\ell");
		draw(C + (0,0,-3.7) -- C + (0,0,-4.8), dots);
		drawCardAt(C + (0,0,-8.5), "k+1", true);
		drawCardAt(C + (0,0,-7.5), "1");
		drawCardAt(C + (0,0,-6.5), "k");
		draw(C + (0,0,-10.2) -- C + (0,0,-11.3), dots);
		drawCardAt(C + (0,0, -13), "N");

		triple F = ((0,0,5) + (0,0,-13)) / 2;
		triple L = (0,-2,0); triple R = (0,2,0);

		draw(A + F + R -- (A+B)/2 + F + L/7, Arrow3(TeXHead2));
		draw((A+B)/2 + F + L/7-- B + F + 1.5L);

		draw(B + F + 1.5R -- (B+C)/2 + F, Arrow3(TeXHead2));
		draw((B+C)/2 + F -- C + F + 1.5L);

		label("insertion", (A+B)/2 + F + L/7, fontsize(8), align=2N);
		label("insertion", (B+C)/2 + F, fontsize(8), align=2N);

		label("$\mathrm{id} \in S_n$", A + (0,0,-15), fontsize(8), align=S);
		label("$\gamma_k\in S_n$", B + (0,0,-15), fontsize(8), align=S);
		label("$\gamma_k\circ\gamma_\ell\in S_n$", C + (0,0,-15), fontsize(8), align=S);
	\end{asy}
\end{figure}

On peut modéliser par un arbre le mélange dont les nœuds sont des permutations des éléments de $S_N$.

\begin{figure}[H]
	\centering
	\begin{asy}
		size(5cm);
		label("$\id$", (0,0));

		pair P = (0,0); pair B = (1,0);

		void drawl() {
			real t = 0.3;
			pair U = (1-t) * B + t * P;
			pair V = (1-t) * P + t * B;
			draw(U--V);
		}

		void drawm() {
			real t = 0.2;
			pair U = (1-t) * P + t * B;
			draw(U--B);
		}

		void drawo() {
			real t = 0.1;
			pair U = (1-t) * P + t * B;
			draw(U--B);
		}

		pair w(pair A, real x) { return (x + P.x, A.y); }

		B = P + w(dir(60), 0.4)/2; label("\tiny$\frac{1}{N}$", B);
		B = P + w(dir(-30), 0.5); drawl(); label("$\gamma_N$", B);
		B = P + w(dir(-10), 0.5); label("$\vdots$", B);
		B = P + w(dir(0), 0.5); drawl(); label("$\gamma_3$", B);
		B = P + w(dir(15), 0.5); drawl(); label("$\gamma_2$", B);
		B = P + w(dir(30), 0.5); drawl(); label("$\gamma_1$", B);

		P = B;

		B = P + w(dir(-30), 0.05); drawm();
		B = P + w(dir(-10), 0.05); label("$\vdots$", B);
		B = P + w(dir(0), 0.05); drawm();
		B = P + w(dir(15), 0.05); drawm();
		B = P + w(dir(30), 0.05); drawm();

		P = B;

		B = P + w(dir(-30), 0.01); drawo();
		B = P + w(dir(-10), 0.01); label("$\vdots$", B);
		B = P + w(dir(0), 0.01); drawo();
		B = P + w(dir(15), 0.01); drawo();
		B = P + w(dir(30), 0.01); drawo();

		draw(shift((0,-0.7)) * brace((B.x + 0.01, 0), (0,0)));
		label("$t$ fois", (B.x + 0.01, -2) / 2, align=S);
	\end{asy}
\end{figure}

On dit que le jeu est bien mélangé après $t$ insertions si chaque élément de $S_N$ est une \underline{feuille} de cet arbre et la probabilité d'obtenir cette permutation est $\textstyle\frac{1}{N!}$.

Avec $N=4$, on a
\begin{align*}
	&\gamma_1= \id\qquad
	&\gamma_2 = \begin{array}{c}
		1\\2\\3\\4
	\end{array} \to \begin{array}{c}
		2\\1\\3\\4
	\end{array}\\
	&\gamma_3 = \begin{array}{c}
		1\\2\\3\\4
	\end{array} \to \begin{array}{c}
		3\\1\\2\\4
	\end{array} \qquad
	&\gamma_4 = \begin{array}{c}
		1\\2\\3\\4
	\end{array} \to \begin{array}{c}
		4\\1\\2\\3
	\end{array}.
\end{align*}

Avec $k = 2$ et $\ell = 1$,
\[
	\begin{array}{c}
		1\\2\\3\\4
	\end{array} \xrightarrow{\gamma_2 \circ} \underbrace{\begin{array}{c}
		2\\1\\3\\4
	\end{array}}_{\gamma_2} \xrightarrow{\gamma_1 \circ} \underbrace{\begin{array}{c}
		2\\1\\3\\4
	\end{array}}_{\gamma_2}
\]

Avec $k = 2$ et $\ell = 2$, on a \[
	\begin{array}{c}
		1\\2\\3\\4
	\end{array} \to \begin{array}{c}
		2\\1\\3\\4
	\end{array} \to \begin{array}{c}
		1\\2\\3\\4
	\end{array}.
\]

Et avec $k = 2$ et $\ell = 3$, on a \[
	\begin{array}{c}
		1\\2\\3\\4
	\end{array} \to \begin{array}{c}
		2\\1\\3\\4
	\end{array} \to \begin{array}{c}
		1\\3\\2\\4
	\end{array}.
\] 

\[
	\gamma_2 \circ \gamma_3 : \begin{array}{c}
		1\mapsto1\\
		2\mapsto3\\
		3\mapsto2\\
		4\mapsto4\\
	\end{array} 
\] 

\begin{center}
	{\itshape Est-ce-que toute permutation peut s'écrire comme un produit des $\gamma_k$ avec $k \in \left\llbracket 1,N \right\rrbracket$ ?}
\end{center}
