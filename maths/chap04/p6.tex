\part{Fonctions trigonométriques réciproques}

\begin{prop-defn}
	L'application $\begin{array}{rcl}
		\left[ -\frac{\pi}{2}, \frac{\pi}{2} \right] &\longrightarrow& [-1, 1] \\
		x &\longmapsto& \sin x
	\end{array}$ est bijective. On appelle arcsinus la réciproque de cette bijection et on la note $\Arcsin$.

	Pour tout $x \in [-1, 1]$, $\Arcsin x$ est le seul angle \cbox{red}{compris entre $-\frac{\pi}{2}$ et $\frac{\pi}{2}$} donc le sinus vaut $x$.
	\qed
	\index{arcsinus (trigonométrie)}
\end{prop-defn}

\begin{exm}
	\begin{itemize}
		\item $\Arcsin 0 = 0$,
		\item $\Arcsin 1 = \frac{\pi}{2}$,
		\item $\Arcsin\left( \sin \frac{3\pi}{2} \right) \color{red}{\neq \frac{3\pi}{2}}$ car $\Arcsin(-1) = -\frac{\pi}{2}$.
		\item $\Arcsin\left( \sin \frac{7\pi}{5} \right) = \frac{-2\pi}{5}$ car $\sin \frac{7\pi}{5} = \sin \left( \pi - \frac{7\pi}{5} \right) = \sin\left( -\frac{2\pi}{5} \right)$ et $-\frac{2\pi}{5} \in \left[ -\frac{\pi}{2}, \frac{\pi}{2} \right]$.
	\end{itemize}
\end{exm}

\begin{prop}
	\begin{enumerate}
		\item $\forall x \in [-1, 1], \Arcsin x \in \left[ -\frac{\pi}{2}, \frac{\pi}{2} \right] $ 
		\item $\forall \theta \in \left[ -\frac{\pi}{2}, \frac{\pi}{2} \right], \Arcsin(\sin \theta) = \theta$ 
		\item $\forall x \in [-1, 1], \sin(\Arcsin x) = x$
		\item $\forall x \in [-1, 1], \cos(\Arcsin x) = \sqrt{1-x^2}$
	\end{enumerate}
\end{prop}

\begin{prv}
	\begin{enumerate}
		\item[1.]~\kern-2.5mm, 2. et 3. correspondent à la définition de $\Arcsin$.
		\item[4.] Soit $x \in [-1, 1]$. On pose $\theta = \Arcsin x \in \left[ -\frac{\pi}{2}, \frac{\pi}{2} \right]$. On sait que $\cos^2 \theta + \sin^2 \theta = 1$ donc $\cos^2 \theta = 1 - \sin^2 \theta = 1 - \sin^2(\Arcsin x) = 1-x^2$. On en déduit que $\big|\cos \theta\big| = \sqrt{1 - x^2}$. Comme $\theta \in \left[ -\frac{\pi}{2}, \frac{\pi}{2} \right]$, $\cos \theta \ge 0$ et donc \[
			\cos(\Arcsin x) = \sqrt{1-x^2}.
		\]
	\end{enumerate}
\end{prv}

\begin{prop}
	\begin{enumerate}
		\item $\Arcsin$ est impaire
		\item $\Arcsin$ est continue sur $[-1,1]$ 
		\item $\Arcsin$ est dérivable sur $]-1,1[$ et \[
				\forall x \in ]-1, 1[, \Arcsin' x = \frac{1}{\sqrt{1-x^2}} > 0
			\]
		\item $\Arcsin$ n'est pas dérivable en 1 et en -1.
	\end{enumerate}
\end{prop}

\begin{figure}[H]
	\centering
	\begin{asy}
		import graph;

		size(5cm);

		real f(real x) { return asin(x); }

		draw(graph(f, -1, 1, 300), magenta);

		draw(scale(0.4) * (dir(45) -- dir(45 + 180)), red, Arrows(TeXHead));
		draw((-1, -pi/2)--(-1, -pi/2 + 0.7), red, Arrow(TeXHead));
		draw((1, pi/2)--(1, pi/2 - 0.7), red, Arrow(TeXHead));

		label("$-1$", (-1, 0), align=S);
		label("$1$", (1, 0), align=S);
		label("$\sfrac{-\pi}{2}$", (0, -pi/2), align=W);
		label("$\sfrac{\pi}{2}$", (0, pi/2), align=W);

		real eps = 0.03;
		draw((-1, -eps)--(-1,eps));
		draw((1, -eps)--(1,eps));
		draw((-eps, pi/2)--(eps, pi/2));
		draw((-eps, -pi/2)--(eps, -pi/2));

		draw((0,-pi/2 - 0.2) -- (0, pi/2 + 0.2), Arrow(TeXHead));
		draw((-1.2,0) -- (1.2,0), Arrow(TeXHead));
	\end{asy}
\end{figure}

\begin{prv}
	\begin{enumerate}
		\item Soit $x \in [-1, 1]$. Alors $-x \in [-1, 1]$. On pose $\theta = \Arcsin(-x)$. $\theta$ est le seul nombre compris entre $-\frac{\pi}{2}$ et $\frac{\pi}{2}$ vérifiant $\sin \theta = -x$.

			$\Arcsin x \in \left[ -\frac{\pi}{2}, \frac{\pi}{2} \right]$ donc $-\Arcsin x \in \left[ -\frac{\pi}{2}, \frac{\pi}{2} \right]$. On a \[
				\sin(-\Arcsin x) = -\sin(\Arcsin x) = -x
			\] donc \[
				\theta = \Arcsin(-x) = -\Arcsin x.
			\]
		\item $\sin$ est continue et strictement croissante sur $[\sfrac{-\pi}{2}, \sfrac{\pi}{2}]$ à valeurs dans $[-1, 1]$ donc $\Arcsin$ est continue sur $[-1, 1]$.
		\item Soit $x \in [-1, 1]$. $\sin'(\Arcsin x) = \cos(\Arcsin x) = \sqrt{1-x^2}$ donc \[
				\forall x \in ]-1,1[, \sin'(\Arcsin x) \neq 0.
			\] Donc $\Arcsin$ est dérivable sur $]-1, 1[$ et \[
				\forall x \in ]-1, 1[, \Arcsin' x = \frac{1}{\sqrt{1-x^2}}
			\]
		\item $\Arcsin$ est continue sur $[-1, 1]$, dérivable sur $]-1,1[$, et \[
			\forall x \in ]-1, 1[, \Arcsin' x = \frac{1}{\sqrt{1-x^2}}
			\tendsto{x\to \pm 1} +\infty.
		\]
		
		D'après le théorème de la limite de la dérivée, $\Arcsin$ n'est pas dérivable en $\pm 1$.
	\end{enumerate}
\end{prv}

\begin{rmk}
	Une primitive de $x \mapsto \frac{1}{\sqrt{1-x}}$ est $\Arcsin$.
\end{rmk}

\begin{prop-defn}
	L'application $\begin{array}{rcl}
		[0,\pi] &\longrightarrow& [-1,1] \\
		x &\longmapsto& \cos x
	\end{array}$ est bijective. On note sa réciproque $\Arccos$.

	En d'autres termes, pour $x \in [-1, 1]$, $\Arccos x$ est le seul angle \cbox{red}{compris entre $0$ et $\pi$} dont le cosinus vaut $x$.
	\index{arccosinus (trigonométrie)}
\end{prop-defn}

\begin{prv}
	Théorème de la bijection continue.
\end{prv}

\begin{exm}
	$\Arccos 0 = \frac{\pi}{2}$, $\Arccos \frac{1}{2} = \frac{\pi}{3}$, $\Arccos \left( -\frac{1}{2} \right) = \frac{2\pi}{3}$,\\
	$\Arccos\left( \cos \frac{75\pi}{67} \right) = \frac{59\pi}{67} \in [0, \pi]$ car $\cos\frac{75\pi}{67} = \cos\left(\frac{75\pi}{67} - 2\pi\right) = \cos\left( -\frac{59\pi}{67} \right) = \cos \frac{59\pi}{67}$.
\end{exm}

\begin{prop}
	\begin{enumerate}
		\item $\forall x \in [-1, 1],\,\Arccos x \in [0,\pi]$,
		\item $\forall \theta \in [0,\pi],\,\Arccos(\cos\theta) = \theta$,
		\item $\forall x \in [-1, 1],\,\cos(\Arccos x) = x$,
		\item $\forall x \in [-1, 1],\, \sin(\Arccos x) = \sqrt{1-x^2}$.
	\end{enumerate}
\end{prop}

\begin{prv}
	\begin{enumerate}
		\item[1.]~\kern-2.5mm, 2. et 3. correspondent à la définition de $\Arccos$.
		\item[4.] Soit $x \in [-1, 1]$.
			\begin{align*}
				&\sin^2(\Arccos x) = 1-\cos^2(\Arccos x) = 1-x^2\\
				\text{donc }& \big|\sin(\Arccos x)\big| = \sqrt{1-x^2}.
			\end{align*}

			Or, $\Arccos x \in [0,\pi]$ donc $\sin(\Arccos x) \ge 0$ et donc \[
				\sin(\Arccos x) = \sqrt{1-x^2}.
			\] 
	\end{enumerate}
\end{prv}

\begin{prop}
	\begin{enumerate}
		\item $\Arccos$ est continue sur $[-1,1]$,
		\item $\Arccos$ est dérivable sur $]-1,1[$ et \[
				\forall x \in ]-1, 1[, \Arccos' x = - \frac{1}{\sqrt{1-x^2}}.
			\]
	\end{enumerate}
\end{prop}

\begin{prv}
	\begin{enumerate}
		\item $\cos$ est coninue et strictement décroissante sur $[0,\pi]$ à valeurs dans $[-1,1]$ donc $\Arccos$ continue dans $[-1,1]$.
		\item
			\begin{align*}
				\forall x \in ]-1,1[,\,
				\cos'(\Arccos x) &= -\sin(\Arccos x) \\
				&= -\sqrt{1-x^2} \neq 0 \\
			\end{align*}

			donc $\Arccos$ est dérivable sur $]-1,1[$ et \[
				\Arccos' x = -\frac{1}{\sqrt{1-x^2}} < 0.
			\]
	\end{enumerate}
\end{prv}

\begin{crlr}
	\[
		\forall x \in [-1,1],\,\Arccos x + \Arcsin x = \frac{\pi}{2}
	\] 
\end{crlr}

\begin{prv}
	\begin{itemize}
		\item[\underline{\sc Méthode 1}]
			Soit $f : \begin{array}{rcl}
				[-1,1] &\longrightarrow& \R \\
				x &\longmapsto& \Arcsin x + \Arccos x
			\end{array}$ continue sur $[-1,1]$ et dérivable sur $]-1,1[$.
			On a \[
				\forall x \in ]-1,1[,\, f'(x) = \frac{1}{\sqrt{1-x^2}} - \frac{1}{\sqrt{1-x^2}} = 0.
			\] donc $f$ est constante sur $]-1, 1[$ : \[
				\exists C \in \R,\, \forall x \in ]-1,1[, f(x) = C.
			\] En particulier, \[
				f(0) = \Arccos 0 + \Arcsin 0 = \frac{\pi}{2} + 0 = \frac{\pi}{2}.
			\] 
		\item[\underline{\sc Méthode 2}] Soit $x \in [0, 1]$.
			\begin{align*}
				\sin(\Arccos x) = \sqrt{1-x^2} &= \cos(\Arcsin x)\\
				&= \sin\left( \frac{\pi}{2} - \Arcsin x \right) \\
			\end{align*} donc \[
				\Arccos x \equiv \frac{\pi}{2} - \Arcsin x \mod {2\pi} \ou \pi - \Arccos x \equiv \frac{\pi}{2} - \Arcsin x \mod{2\pi}
			\] donc
			\begin{align*}
				&\exists k \in \Z,\,\Arccos x + \Arcsin x = \frac{\pi}{2} + 2k\pi\\
				\ou &\exists  k \in \Z,\, -\Arccos x + \Arcsin x = -\frac{\pi}{2} + 2k\pi.
			\end{align*}

			Or, $\begin{cases}
				0 \le  \Arccos x \le \pi\\
				0 \le \Arcsin x \le \frac{\pi}{2}
			\end{cases}$ donc $0 \le \Arccos x + \Arcsin x \le  \pi$\\
			$\iff \begin{cases}
				-\frac{\pi}{2} \le -\Arccos x \le 0\\
				0 \le \Arcsin x \le \frac{\pi}{2}
			\end{cases}$ donc $-\frac{\pi}{2} \le -\Arccos x + \Arcsin x \le \frac{\pi}{2}$ 

			D'où \[
				\Arccos x + \Arcsin x = \frac{\pi}{2} \ou -\Arccos x + \Arcsin x = -\frac{\pi}{2}
			\]

			\begin{itemize}
				\item Si $-\Arccos x > -\frac{\pi}{2}$ ou $\Arcsin x > 0$, alors $-\Arccos x + \Arcsin x > -\frac{\pi}{2}$ et donc \[
						\Arccos x + \Arcsin x = \frac{\pi}{2}.
					\]
				\item Si $-\Arccos x = \frac{\pi}{2}$ et $\Arcsin x = 0$, alors $x = 0$ et donc \[
						\Arccos x + \Arcsin x = \frac{\pi}{2} + 0 = \frac{\pi}{2}.
					\]
			\end{itemize}

			Soit $x \in [-1, 0[$. On pose $y = -x \in [0, 1]$
			\begin{itemize}
				\item $\Arcsin x = - \Arcsin y$,
				\item $\Arccos x = \pi - \Arccos y$ : en effet, $\cos(\pi - \Arccos y) = -\cos(\Arccos y) = -y = x$ et $0 \le \Arccos y \le \pi$ donc $-\pi \le \Arccos y \le 0$ et donc $0 \le \pi - \Arccos y \le \pi$
			\end{itemize}

			D'où,
			\begin{align*}
				\Arccos x + \Arcsin x &= \pi - \Arccos y - \Arcsin y \\
				&= \pi - (\Arccos y + \Arcsin y) \\
				&= \pi - \frac{\pi}{2} \\
				&= \frac{\pi}{2}. \\
			\end{align*}
	\end{itemize}
\end{prv}

\begin{figure}[H]
	\centering
	\begin{asy}
		import graph;

		size(5cm);

		real f(real x) { return acos(x); }

		draw(graph(f, -1, 1, 300), magenta);

		draw(shift((0, pi/2)) * scale(0.4) * (dir(-45) -- dir(-45 + 180)), red, Arrows(TeXHead));
		draw((-1, pi)--(-1, pi - 0.7), red, Arrow(TeXHead));
		draw((1, 0)--(1, 0.7), red, Arrow(TeXHead));

		label("$-1$", (-1, 0), align=S);
		label("$1$", (1, 0), align=S);
		label("$\sfrac{\pi}{2}$", (0, pi/2), align=SW);
		label("$0$", (0, 0), align=SW);
		label("$\pi$", (0, pi), align=W);

		real eps = 0.03;
		draw((-1, -eps)--(-1,eps));
		draw((1, -eps)--(1,eps));
		draw((-eps, pi/2)--(eps, pi/2));
		draw((-eps, pi)--(eps, pi));

		draw((0,-0.2) -- (0, pi + 0.2), Arrow(TeXHead));
		draw((-1.2,0) -- (1.2,0), Arrow(TeXHead));
	\end{asy}
\end{figure}

\begin{prop-defn}
	L'application $\begin{array}{rcl}
		\left]-\frac{\pi}{2}, \frac{\pi}{2}\right[ &\longrightarrow& \R \\
		x &\longmapsto& \tan x
	\end{array}$ est bijective. On note $\Arctan$ la réciproque de cette bijection.

	C'est à dire, pour tout $x \in \R$, $\Arctan x$ est le seul angle \cbox{red}{compris entre $-\frac{\pi}{2}$ et $\frac{\pi}{2}$ (exclus)} dont la tangente vaut $x$.
	\index{arctangente (trigonométrie)}
\end{prop-defn}

\begin{prv}
	Théorème de la bijection continue.
\end{prv}

\begin{exm}
	$\Arctan 0 = 0$, $\Arctan 1 = \frac{\pi}{4}$, $\Arctan \sqrt{3} = \frac{\pi}{3}$,\\
	$\Arctan(\tan \frac{9\pi}{7}) = \frac{2\pi}{7} \in \left]-\frac{\pi}{2},\frac{\pi}{2}\right[$ car $\tan\frac{9\pi}{7} = \tan\left( \frac{9\pi}{7} - \pi \right) = \tan \frac{2\pi}{7}$.
\end{exm}

\begin{prop}
	\begin{enumerate}
		\item $\Arctan$ est dérivable sur $\R$ et \[
				\forall x \in \R,\, \Arctan' x = \frac{1}{1+x^2},
			\]
		\item $\begin{cases}
				\lim_{x\to +\infty} \Arctan x = \frac{\pi}{2},\\
				\lim_{x\to -\infty} \Arctan x = -\frac{\pi}{2},
			\end{cases}$ 
		\item $\Arctan$ est impaire.
	\end{enumerate}
\end{prop}

\begin{figure}[H]
	\centering
	\begin{asy}
		import graph;

		size(8cm);

		real f(real x) { return atan(x); }

		draw(graph(f, -5, 5, 300), magenta);

		draw(scale(0.8) * (dir(45) -- dir(45 + 180)), red, Arrows(TeXHead));

		label("$\sfrac{\pi}{2}$", (0, pi/2), align=W);
		label("$\sfrac{-\pi}{2}$", (0, -pi/2), align=W);

		real eps = 0.1;
		draw((-eps, pi/2)--(eps, pi/2));
		draw((-eps, -pi/2)--(eps, -pi/2));

		draw((0,-pi/2-0.4) -- (0, pi/2 + 0.4), Arrow(TeXHead));
		draw((-5,0) -- (5,0), Arrow(TeXHead));
	\end{asy}
\end{figure}

\begin{prv}
	\begin{enumerate}
		\item $\forall x \in \R,\,\tan'(\Arctan x) = 1 + \tan^2(\Arctan x) = 1+x^2 \neq 0$ donc $\Arctan$ est dérivable sur $\R$ et \[
				\forall x \in \R, \Arctan'(x) = \frac{1}{1+x^2}.
			\]
		\item On déduit des limites de $\tan$ les limites de $\Arctan$.
		\item Comme pour $\Arcsin$.
	\end{enumerate}
\end{prv}

\begin{prop}
	\[
		\forall x \in \R^*,\, \Arctan x + \Arctan \frac{1}{x} = \begin{cases}
			\frac{\pi}{2} &\text{si } x > 0\\
			-\frac{\pi}{2}&\text{si } x < 0
		\end{cases}
	\]
\end{prop}

\begin{prv}
	On pose $f : \begin{array}{rcl}
		\R^* &\longrightarrow& \R \\
		x &\longmapsto& \Arctan x + \Arctan \frac{1}{x}
	\end{array}$ dérivable sur $\R^*$ par
	\begin{align*}
		f'(x) &= \frac{1}{1+x^2} - \frac{1}{1+\left( \frac{1}{x} \right)^2} \times \frac{1}{x^2} \\
		&= \frac{1}{1+x^2} - \frac{1}{1 + x^2} \\
		&= 0 \\
	\end{align*}

	\danger $\R^*$ n'est pas un intervalle !
	\begin{itemize}
		\item $\R^+_*$ est un intervalle donc $\exists C^+ \in \R, \forall x \in \R^+_*, f(x) =C^+$
		\item $\R^-_*$ est un intervalle donc $\exists C^- \in \R, \forall x \in \R^-_*, f(x) =C^-$
	\end{itemize}

	$f(1) = \frac{\pi}{4} + \frac{\pi}{4} = \frac{\pi}{2}$ donc $C^+ = \frac{\pi}{2}$,\\
	$f(-1) = -\frac{\pi}{4} - \frac{\pi}{4} = -\frac{\pi}{2}$ donc $C^- = -\frac{\pi}{2}$.
\end{prv}
