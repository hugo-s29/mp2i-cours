\part{Exercice 8}

Soit $n \ge 2$.

{\it $f$ nilpotent} : $\exists k \in \N^*, \underbrace{f \circ \ldots \circ f}_{k}$
 et $k$ est l'{\it indice de nilpotence} de $f$.

$N_n(\R)$ ensemble des matrices nilpotentes à coefficants dans $\R$.

\subsection*{Partie I}

\begin{enumerate}
	\item On pose $\mathcal{B} = (e_1, \ldots, e_n)$ la base canonique de $\mmathcal{M}_n(\mathbbm{K})$
		\begin{enumerate}
			\item
					\begin{itemize}
						\item[\sc Pour $A$] On a \[
								\forall k < n, f(e_k) = 0
							\] et $f(e_n) = e_1 \neq 0$. Donc, \[
								\Ker(f) = \Vect(e_1, \ldots, e_{n-1})
							\] et \[
								\mathrm{Im}(f) = \Vect(e_1)
							.\] Donc \[
								\mathrm{Im}(f) \subset \Ker(f)
							\]
						\item[\sc Pour $B$] On a \[
								\forall k < n, g(e_k) = 0
							\] et $g(e_n) = e_1 \neq 0$. Donc, \[
								\Ker(g) = \Vect(e_2, \ldots, e_n)
							\] et \[
								\mathrm{Im}\, g = \Vect(e_n)
							\] Donc \[
								\Ker(g) \subset \mathrm{Im}(f)
							.\]
				\end{itemize}
			\item On a $A^2 = (0)$ et $B^2 = 0_{\mmathcal{M}_n(\R)}$. Donc $A$ et $B$ sont nilpotentes d'indice 2.
			\item On a \[
				A + B = 
					\begin{pNiceMatrix}
						1&0&\Cdots&0\\
						0&\Ddots&\Ddots&\Vdots\\
						\Vdots&\Ddots&&0\\
						0&\Cdots&0&1
					\end{pNiceMatrix}
				\]
				On remarque que \[
					M = (A + B)^2 = 
					\begin{pNiceMatrix}
						1&0&\Cdots&0\\
						0&\Ddots&\Ddots&\Vdots\\
						\Vdots&\Ddots&&0\\
						0&\Cdots&0&1
					\end{pNiceMatrix}
				\] et $(A+B)^3 = A + B$. Donc, par récurrence, \[
					(A+B)^q = \begin{cases}
						(A + B)^2 &\text{ si } q = 2k \text{ avec } k \in \N^*\\
						A + B &\text{ si } q = 2k + 1 \text{ avec } k\in \N^*.
					\end{cases}
				\] Bah du coup non.
			\item $AB = A$ et $BA = B$. Donc $(AB)^k = A^k = 0$ et $(BA)^k = B^k = 0$ pour $k \ge 2$. Donc $A, B \in N_n(\R)$.
			\item $I_n \not\in N_n(\R)$ car $\forall k \in \N, {I_n}\!^k = I_n \neq (0)$. Donc $N_n$ n'est pas une $\R$-algèbre de $\mmathcal{M}_n(\R)$. Bon, flemme de rédiger.
		\end{enumerate}

		\subsection*{Partie II}
		\item
			\begin{enumerate}
				\item
			\end{enumerate}
\end{enumerate}




