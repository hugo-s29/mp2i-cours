\part{Évaluation}

\begin{defn}
	Soit $A$ une $\mathbbm{K}$-algèbre et $P \in \mathbbm{K}[X]$. On pose $P =  \sum_{k=0}^n e_k X^k$. Soit $a \in A$.\\
	On pose
	\begin{align*}
		P(a) &= \sum_{k=0}^n e_k a^k\\
		&= e_0 1_A + e_1 a + e_2 a^2 + \cdots + e_n a^n \in A \\
	\end{align*}
	On dit qu'on a \underline{évalué} $P$ en $a$, ou \underline{spécialisé} $X$ avec la valeur de $a$, ou \underline{remplacé} $X$ par $a$, \underline{substitué} $a$ à $X$.
	\index{évaluer (polynôme)}
	\index{spécialiser (polynôme)}
	\index{remplacer (polynôme)}
	\index{substituer (polynôme)}
\end{defn}

\begin{defn}
	Soit $P \in \mathbbm{K}[X]$ et $a \in \mathbbm{K}$. \\
	On dit que $a$ est une \underline{racine de $P$} si $P(a) = 0_{\mathbbm{K}}$.
	\index{racine (polynôme)}
\end{defn}

\begin{defn}
	Soit $P \in \mathbbm{K}[X] \in \mathcal{M}_n(\mathbbm{K})$. On dit que c'est un \underline{polynôme de matrices}.
	\index{polynôme de matrices}
\end{defn}

\begin{exm}
	$P = 1+2X - 3X^2$, $A = \begin{pmatrix}
		1&0\\
		1&1
	\end{pmatrix}$ 
	\begin{align*}
		P(A) &= \begin{pmatrix}
			1&0\\
			0&1
		\end{pmatrix} + 2\begin{pmatrix}
			1&0\\
			1&1
		\end{pmatrix} - 3 \begin{pmatrix}
			1&0\\
			1&1
		\end{pmatrix}^2 \\
		&= \begin{pmatrix}
			1&0\\
			0&1
		\end{pmatrix} + \begin{pmatrix}
			2&0\\
			2&2
		\end{pmatrix} - 3 \begin{pmatrix}
			1&0\\
			2&1
		\end{pmatrix} \\
		&= \begin{pmatrix}
			0&0\\
			-4&0
		\end{pmatrix} \\
	\end{align*}
\end{exm}


\begin{defn}
	Soient $P,Q \in \mathbbm{K}[X]$, $P = \sum_{k=0}^n a_k X^k$.\\
	Alors $P(Q) = \sum_{k=0}^n a_k Q^k \quad \in \mathbbm{K}[X]$\\
	C'est la \underline{composée} de $P$ et $Q$.
	\index{composition (polynômes)}
\end{defn}

\begin{rmk}
	[\danger Attention]
	Ne pas confondre $\underbrace{P(X+1)}_{\text{composée}}$ et $\underbrace{P(X+1)}_{\text{produit}}$.\\
	On a $\underbrace{P(X+1)}_{\text{produit}} = (X+1)\,P = P(X)\,(X+1) = P \times (X+1)$
\end{rmk}

\begin{prop}
	Soient $P,Q \in \mathbbm{K}[X]$ avec $\begin{cases}
		Q \neq 0\\
		P\neq 0
	\end{cases}$. On a \[
		\deg\big(P(Q)\big) = \deg(P) \times \deg(Q)
	\] 
	\qed
\end{prop}

\begin{exm}
	$\mathbbm{K} = {}^\Z / {}_{2\Z} = \left\{\overline{0}, \overline{1}\right\}$ \\

	$P = X^2 + X + 1 \in \mathbbm{K}[X]$ et $Q = 1 \in \mathbbm{K}[X]$\\
	$P \neq Q$\\

	\begin{align*}
		f_P: {}^\Z / {}_{2\Z} &\longrightarrow {}^\Z / {}_{2\Z} \\
		x &\longmapsto P(x)
	\end{align*}

	\begin{align*}
		f_Q: {}^\Z / {}_{2\Z} &\longrightarrow {}^\Z / {}_{2\Z} \\
		x &\longmapsto Q(x)
	\end{align*}

	$f_P\left( \overline{0} \right) = \overline{1} = f_Q\left( \overline{0} \right)$ \\
	$f_P\left( \overline{1} \right) = \overline{1} + \overline{1} + \overline{1} = \overline{1} = f_Q\left( \overline{1} \right)$ \\
	donc $f_P = f_Q$ alors que $P \neq Q$
\end{exm}

\begin{thm}
	Soit $A$ une $\mathbbm{K}$-algèbre. L'application \begin{align*}
		\varphi: \mathbbm{K}[X] &\longrightarrow A^A \\
		P &\longmapsto f_P : \begin{array}{rcl}
			A &\longrightarrow& A \\
			a &\longmapsto& P(a)
		\end{array}
	\end{align*} vérifie
	
	\begin{enumerate}
		\item $\forall P,Q \in \mathbbm{K}[X], \varphi(P+Q) = \varphi(P) + \varphi(Q)$
		\item $\forall P,Q \in \mathbbm{K}[X], \varphi(PQ) = \varphi(P) \times \varphi(Q)$
		\item $\forall \lambda \in \mathbbm{K}, \forall P \in \mathbbm{K}[X], \varphi(\lambda P ) = \lambda \varphi(P)$
	\end{enumerate}

	\qed
\end{thm}

\begin{exm}
	$\mathbbm{K} = \R$ \\
	$X^2  - 1 = (X-1)(X+1)$ 
	\begin{itemize}
		\item $\C$ est une $\R$-algèbre donc \[
				\forall z \in \C, z^2-1 = (z-1)(z+1)
			\] 
		\item $\mathcal{M}_2(\R)$ est une $\R$-algèbre \[
			\forall A \in \mathcal{M}_2(\R), A^2 - I_2 = (A-I_2) (A+I_2)
		\] 
	\end{itemize}
\end{exm}

\begin{defn}
	Soit $P \in \mathbbm{K}[X]$, \[
		P = \sum_{k=0}^{n} a_k X^k
	\] Le \underline{polynôme dérivé} de $P$ est \[
		P' = \sum_{k=0}^{n} k a_k X^{k-1} = \sum_{k=1}^n k a_k X^{k-1}
	\] où \[
		\forall k \in \left\llbracket 1,n \right\rrbracket, ka_k = \underbrace{a_k + \cdots + a_k}_{k \text{ fois}}
	\] \[
		0_\N a_k = 0_\mathbbm{K}
	\]
	\index{dérivé (polynôme)}
\end{defn}

\begin{rmk}
	Si $P \in \R[X]$, $f_p : \begin{array}{rcl}
		\R &\longrightarrow& \R \\
		x &\longmapsto& P(x)
	\end{array}$\\
	$f_{P'} : \begin{array}{rcl}
		\R &\longrightarrow& \R \\
		x &\longmapsto& P'(x)
	\end{array}$ alors $f_{P'} = f'_P$
\end{rmk}

\begin{prop}
	\[
		\forall P \in \mathbbm{K}[X], \deg(P') = \begin{cases}
			\deg(P) - 1 &\text{ si } \deg(P) > 0\\
			-\infty &\text{ sinon}
		\end{cases}
	\] 
\end{prop}

\begin{prop}
	Soient $P,Q \in \mathbbm{K}[X]$ et $\lambda \in \mathbbm{K}$.
	\begin{enumerate}
		\item $(P+Q)' = P' + Q'$ 
		\item $(PQ)' = P'Q + PQ'$ 
		\item $(\lambda P)' = \lambda P'$
	\end{enumerate}
\end{prop}

\begin{prv}
	On pose
	\begin{align*}
		P = \sum_{k=0}^{p} a_k X^k&\qquad&
		Q = \sum_{k=0}^{q} b_k X^k
	\end{align*}
	\begin{enumerate}
		\item On peut supposer $p \ge q$ \\
			Si $p > q$, on pose $b_{q+1} = \cdots = b_p = 0$ \\
			\[
				P + Q = \sum_{k=0}^p (a_k + b_k) X^k
			\] donc 
			\begin{align*}
				(P+Q)' &= \sum_{k=0}^p k(a_k + b_k) X^{k-1} \\
				&= \sum_{k=0}^p ka_kX^{k-1} + \sum_{k=0}^p kb_kX^{k-1} \\
				&= P' + Q' \\
			\end{align*}
		\item \[
				PQ = \sum_{k=0}^p \sum_{\ell=0}^q a_k b_\ell X^{k+\ell}
			\] D'après $1.$, 
			\begin{align*}
				(PQ)' &= \sum_{k=0}^p \sum_{\ell = 0}^q \left(a_k b_\ell X^{k+\ell}\right)'\\
				&= \sum_{k=0}^p \sum_{\ell = 0}^q a_k b_\ell (k+\ell) X^{k+\ell - 1}\\
				&= \sum_{k=0}^p \sum_{\ell=0}^q ka_k b_\ell X^{k-1+\ell} + \sum_{k=0}^p \sum_{\ell=0}^q \ell a_k b_\ell X^{k+\ell-1} \\
				&= \sum_{k=0}^p k_k X^{l-1}\sum_{\ell=0}^q b_\ell X^{\ell} + \sum_{k=0}^p a_k X^k \sum_{\ell=0}^q \ell b_\ell X^{\ell-1} \\
				&= P'Q+ PQ' \\
			\end{align*}
		\item \[
			\lambda P = \sum_{k=0}^p \lambda a_k X^{k}
		\] donc \[
			(\lambda P)' = \sum_{k=0}^p \lambda a_k k X^{k-1} = \lambda \sum_{k=0}^p k a_k X^{k-1} = \lambda P'
		\]
	\end{enumerate}
\end{prv}

\begin{defn}
	Pour $k\in \N$, on définit la dérivée $k$-ième d'un polynôme $P \in \mathbbm{K}[X]$ par
	\begin{itemize}
		\item si $k=0$, $P^{(k)} = P$ 
		\item si $k=1$, $P^{(1)} = P'$ 
		\item si $k>1$, $P^{(k)} = \left( P^{(k-1)} \right)'$
	\end{itemize}
	\index{dérivée $n$-ième (polynôme)}
\end{defn}

\begin{prop}
	\[
		\forall k,j \in \N^2,
		\left(X^{k}\right)^{(j)} = \begin{cases}
			0 &\text{ si } j > k\\
			k(k-1) \cdots (k-j+1)X^{k-j} = \frac{k!}{(k-j)!}X^{k} &\text{ si } j \le k\\
		\end{cases}
	\] 
\end{prop}

\begin{prv}
	[par récurrence sur $j$ à $k$ fixé]
\end{prv}

\begin{prop}
	Soient $P,Q \in \mathbbm{K}[X]$, $\lambda \in \mathbbm{K}$
	\begin{enumerate}
		\item $\forall k \in \N, (P+Q)^{(k)} = P^{(k)} + Q^{(k)}$ 
		\item $\forall k \in \N, (PQ)^{(k)} = \sum_{i=0}^k {k\choose i} P^{(i)} Q^{(k-i)}$ 
		\item $\forall k \in \N, (\lambda P)^{(k)} = \lambda P^{(k)}$
	\end{enumerate}
\end{prop}

\begin{prv}
	[par récurrence sur $k$]
\end{prv}
