\part{Matrices par blocs}

\begin{exm}
	Soit $p$ un projecteur de $E$ : \[
		E = \Ker p \oplus \mathrm{Im}\ p
	\] Soit $\mathcal{B} = (e_1, \ldots, e_k, e_{k+1}, \ldots, e_n)$ une base de $E$ avec $\begin{cases}
		\mathrm{Im}(p) = \Vect(e_1, \ldots, e_k)\\
		\Ker(p) = \Vect(e_{k+1}, \ldots, e_n)\\
	\end{cases}$

	Alors, 
	\begin{align*}
		\Mat_\mathcal{B}(p) =
		\left(\begin{NiceArray}{c c c | c c c}
				1&&&0&\Cdots&0\\
				 &\Ddots&&\Vdots&&\Vdots\\
				&&1&0&\Cdots&0\\\hline
				0&\Cdots&0&0&\Cdots&0\\
				\Vdots&&\Vdots&\Vdots&&\Vdots\\
				0&\Cdots&0&0&\Cdots&0\\
		\end{NiceArray}\right)
		= \left( \begin{array}{c|c}
				I_k & 0\\ \hline
				0&0
		\end{array}\right) \\
	\end{align*}

	De même, si $\s$ est une symétrie de $E$, \[
		E = \Ker(\s - \id_E) \oplus \Ker(\s + \id_E)
	.\] Soit $\mathcal{C} = (e_1', \ldots, e_\ell', e_{\ell+1}', \ldots, e'_n)$ avec $\begin{cases}
		\Vect(e'_1, \ldots, e'_\ell) = \Ker(\s - \id_E),\\
		\Vect(e'_{\ell+1}, \ldots, e'_n) = \Ker(\s + \id_E).\\
	\end{cases}$

	Alors
	\[
		\Mat_\mathcal{C}(\s) = \left(\begin{array}{c|c}
				I_\ell &0\\ \hline
				0&-I_{n-\ell}
		\end{array}\right) 
	\]
\end{exm}

\begin{prop}
	Soient $F$ et $G$ supplémentaires dans $E$ : \[
		E = F \oplus G.
	\] Soit $f \in \mathcal{L}(F)$ et $g \in \mathcal{L}(G)$. Alors \[
	\exists !h \in \mathcal{L}(E) h_{|F} = f,\ h_{|G} = g \et h = f \circ p + g \circ q
	\] où $\begin{cases}
		p \text{ est la projection sur $F$ parallèlement à $G$}\\
		q \text{ est la projection sur $G$ parallèlement à $F$}\\
	\end{cases}$.

	On a aussi $q = \id_E - p$.
\end{prop}

\begin{prv}
	\begin{itemize}
		\item[\sc \underline{Analyse}] Soit $h \in \mathcal{L}(E)$ tel que $\begin{cases}
				h_{|F}=f\\
				h_{|G}=g
			\end{cases}$.

			Soit $x \in E$. Alors \[
				x = \underbrace{p(x)}_{\in F} + \underbrace{q(x)}_{\in G}
			\]

			Donc,
			\begin{align*}
				h(x) &= h\big(p(x)\big) + h\big(q(x)\big)\\
				&= f\big(p(x)\big) + g\big(q(x)\big) \\
				&= (f \circ p + g \circ q)(x) \\
			\end{align*}
			Si $h$ existe, alors \[
				h = f \circ p + g \circ q
			\]
		\item[\underline{\sc Synthèse}] On pose $h = f \circ p + g  \circ q$.

			$p$, $q$, $f$ et $g$ sont linéaires donc $h$ aussi.

			Soit $x \in E$.
			\begin{align*}
				h(x) &= f\big(p(x)\big) + g\big(q(x)\big) \\
				&= f(x) + g(0_E) \\
				&= f(x) \\
			\end{align*}
			donc $h_{|F} = f$ et de même $h_{|G}=g$.
	\end{itemize}
\end{prv}

\begin{prop}
	On reprend les notations et hypothèses précédentes. Soit $(e_1, \ldots, e_p)$ une base de $F$, et $(f_1, \ldots, f_q)$ une base de $G$. Alors, $\mathcal{B} = (e_1, \ldots, e_p, f_1, \ldots, f_q)$ est une base de $E$ et \[
		\Mat_\mathcal{B}(h) = \left(
		\begin{array}{c|c}
			A&0\\ \hline
			0&B
		\end{array}\right)
	\] où $\begin{cases}
		A = \Mat_{(e_1, \ldots e_p)}(f)\\
		B = \Mat_{(f_1, \ldots, f_q)}(g)
	\end{cases}$
	\qed
\end{prop}

\begin{prop}
	Soient $(A,A') \in \mathcal{M}_n(\mathbbm{K})^2$ et $(B,B') \in \mathcal{M}_p(\mathbbm{K})^2$.
	\begin{enumerate}
		\item \[
				\left(\begin{array}{c|c}
					A&0\\ \hline
					0&B
				\end{array}\right)
				\left(\begin{array}{c|c}
					A'&0\\ \hline
					0&B'
				\end{array}\right) = 
				\left(\begin{array}{c|c}
					AA'&0\\ \hline
					0&BB'
				\end{array}\right)
			\]
		\item \[
				\left(\begin{array}{c|c}
					A&0\\ \hline
					0&B
				\end{array}\right) \in \mathrm{GL}_{n+p}(\mathbbm{K})	 \iff \begin{cases}
					 A \in \mathrm{GL}_n(\mathbbm{K})\\
					 B \in \mathrm{GL}_p(\mathbbm{K})
				\end{cases}
			\] et dans ce cas, \[
				\left(\begin{array}{c|c}
					A&0\\ \hline
					0&B
				\end{array}\right)^{-1} =
				\left(\begin{array}{c|c}
					A^{-1}&0\\ \hline
					0&B^{-1}
				\end{array}\right)
			\]
		\item \[
				\tr \left(\begin{array}{c|c}
					A&0\\ \hline
					0&B
				\end{array}\right) = \tr A + \tr B
			\]
	\end{enumerate}
\end{prop}

\begin{prv}
	\begin{enumerate}
		\item Soit $\begin{cases}
				f \in \mathcal{L}(F) \text{ tel que } \Mat_\mathcal{B}(f) = A,
				f' \in \mathcal{L}(F) \text{ tel que } \Mat_\mathcal{B}(f') = A',
				g \in \mathcal{L}(G) \text{ tel que } \Mat_\mathcal{C}(g) = B,
				g' \in \mathcal{L}(G) \text{ tel que } \Mat_\mathcal{C}(g') = B'
			\end{cases}$ où $\begin{cases}
				F \oplus G = \mathbbm{K}^{n+p},\\
				\dim(F) = n, \dim(G) = p,\\
				\mathcal{B} \text{ base de } F,\\
				\mathcal{C} \text{ base de } G.\\
			\end{cases}$
			Soit $\begin{cases}
				h \in \mathcal{L}(\mathbbm{K}^{n+p}) \text{ tel que } \begin{cases}
					h_{|F} = f\\
					h_{|G} = g
				\end{cases}\\
				h' \in \mathcal{L}(\mathbbm{K}^{n+p}) \text{ tel que } \begin{cases}
					h'_{|F} = f'\\
					h'_{|G} = g'\\
				\end{cases}
			\end{cases}$
			Soit $\mathcal{D} = \mathcal{B} \cup \mathcal{C}$ une base de $\mathbbm{K}^{n+p}$.
			\begin{align*}
				\left(\begin{array}{c|c}
					A&0\\ \hline
					0&B
				\end{array}\right)
				\left(\begin{array}{c|c}
					A'&0\\ \hline
					0&B'
				\end{array}\right) &= \Mat_{\mathcal{D}}(h) \Mat_{\mathcal{D}}(h')\\
				&= \Mat_{\mathcal{D}}(h \circ h') \\
			\end{align*}
			Or, $(h \circ h')_{|F} = f \circ f'$ et $(h \circ h')_{|G} = g \circ g'$.

			Donc,
			\begin{align*}
				\Mat_\mathcal{D}(h \circ h') &=
					\left(\begin{array}{c|c}
						\Mat_\mathcal{B}(f \circ f')&0\\ \hline
						0&\Mat_\mathcal{C}(g \circ g')
					\end{array}\right)\\
				&=\left(\begin{array}{c|c}
					AA'&0\\ \hline
					0&BB'
				\end{array}\right).
			\end{align*}
	\end{enumerate}
\end{prv}

\begin{prop}
	Soient $A,A' \in \mathcal{M}_n(\mathbbm{K})$, $B,B' \in \mathcal{M}_{n,p}(\mathbbm{K})$, $C,C' \in \mathcal{M}_{p,n}(\mathbbm{K})$ et $D, D' \in \mathcal{M}_p(\mathbbm{K})$.

	\[
		\left(\begin{array}{c|c}
			A&B\\ \hline
			C&D
		\end{array}\right)
		\left(\begin{array}{c|c}
			A'&B'\\ \hline
			C'&D'
		\end{array}\right) = 
		\left(\begin{array}{c|c}
			AA' + BC'& AB' + BD'\\ \hline
			CA' + DC'&CB' + DD'
		\end{array}\right)
	\] Cette formule se généralise à un nombre quelconque de blocs : \[
		\left(\begin{array}{c|c|c|c}
				A_{11}&A_{12}&\cdots&A_{1,n}\\ \hline
				A_{21}&A_{22}&\cdots&A_{2,n}\\ \hline
				\vdots&\vdots&\ddots&\vdots\\ \hline
				A_{p,1}&A_{p,2}&\cdots&A_{p,n}
		\end{array}\right)
		\left(\begin{array}{c|c|c|c}
				A'_{11}&A'_{12}&\cdots&A'_{1,n}\\ \hline
				A'_{21}&A'_{22}&\cdots&A'_{2,n}\\ \hline
				\vdots&\vdots&\ddots&\vdots\\ \hline
				A'_{p,1}&A'_{p,2}&\cdots&A'_{p,n}
		\end{array}\right)
	\] Cette matrice se calcyle comme on s'y attend si les dimensions des blocs autorisent les produits.
\end{prop}

\begin{prop}
	Le rang d'une matrice $A$, c'est la taille de la plus grande matrice carrée inversible que l'on peut extraire de $A$.
	\qed
\end{prop}



