\part{Parallèlisme et hyperplans}

\begin{defn}
	Soit $\left( E, \vec{E}, \tau \right)$ un espace affine, $F$ et $G$ deux sous-espaces affines de $E$.

	\begin{enumerate}
		\item On dit que $F$ et $G$ sont \underline{fortement parallèles}\index{parallélisme fort (espace affine)} si $\vec{F} = \vec{G}$.
		\item On dit que $F$ et $G$ sont \underline{faiblement parallèles}\index{parallélisme faible (espace affine)} si $\vec{F} \subset \vec{G}$ ou $\vec{G} \subset \vec{F}$.
	\end{enumerate}
\end{defn}

\begin{figure}[H]
	\centering
	\begin{subfigure}{5cm}
		\centering
		\begin{asy}
			import three;
			size(5cm);
	
			real k = 0.73 * 1.5;
			triple o1 = (X-Y)/2;

			draw(surface(X+Y -- X-Y -- -X-Y -- -X+Y -- cycle), white);
			draw(X+Y -- X-Y -- -X-Y -- -X+Y -- cycle, black);
			draw(o1 -- o1-X*k, magenta, Arrow3(TeXHead2));
			draw(o1 -- o1+Y*k, magenta, Arrow3(TeXHead2));
	

			triple off = 2.5Z/3;

			draw(surface(X+Y+off -- X-Y+off -- -X-Y+off -- -X+Y+off -- cycle), white);
			draw(X+Y+off -- X-Y+off -- -X-Y+off -- -X+Y+off -- cycle, black);
			off += o1;
			draw(off-- -X*k+off, magenta, Arrow3(TeXHead2));
			draw(off--Y*k+off, magenta, Arrow3(TeXHead2));
		\end{asy}
		\caption{parallélisme fort}
	\end{subfigure}
	$\qquad$
	\begin{subfigure}{5cm}
		\centering
		\begin{asy}
			import three;
			currentlight = nolight;
			size(5cm);
			triple o1 = (X-Y)/2;

			triple off = -2.5Z/3;

			draw(surface(X+Y+off -- X-Y+off -- -X-Y+off -- -X+Y+off -- cycle), rgb("faf4ed"));
			draw(X+Y+off -- X-Y+off -- -X-Y+off -- -X+Y+off -- cycle, black);

			draw(scale3(1.2) * (X-Y -- -X + Y), black);
	
			draw(o1 + off-- -X+Y+o1 + off, magenta, Arrow3(TeXHead2));
			draw(o1-- -X+Y+o1, magenta, Arrow3(TeXHead2));
		\end{asy}
		\caption{parallélisme faible}
	\end{subfigure}
\end{figure}

\begin{defn}
	Soit $F$ un sous-espace affine de $\left( E, \vec{E}, \tau \right)$. L'espace \[
		\vec{F} = \left\{ \vec{AB}  \mid A,B \in F \right\}
	\] est appelé \underline{direction} de $F$.\index{direction (sous-espace affine)}

	On dit que
	\begin{itemize}
		\item $F$ est une \underline{droite affine}\index{droite affine (espace affine)} si $\vec{F}$ est une droite vectorielle,
		\item $F$ est une \underline{plan affine}\index{plan affine (espace affine)} si $\vec{F}$ est un plan vectorielle,
		\item $F$ est une \underline{hyperplan affine}\index{hyperplan affine (espace affine)} si $\vec{F}$ est un hyperplan vectorielle.
	\end{itemize}
\end{defn}

