\part{Exercice 1}

Soit $\varphi$ l'endomorphisme de $\R_n[X]$ défini par $\varphi(P) = P(X+1)$.

\begin{enumerate}
	\item La base canonique de $\R_n[X]$ est $\mathcal{B} = (1, X, X^2, \ldots, X^n)$.

		\begin{align*}
			\forall k \in \left\llbracket 1,n \right\rrbracket,\,
			\varphi(X^k) &= (X+1)^k \\
			&= \sum_{i=0}^k {k \choose i} X^i \\
		\end{align*}
		
		Donc, \[
			A = \begin{pNiceMatrix}
				1&1&1&\Cdots&1\\
				0&1&2&\Cdots&n\\
				\Vdots&\Vdots&\Vdots&\Ddots&\Vdots\\
				0&0&0&\Cdots&1\\
			\end{pNiceMatrix}
		\] 
	\item \[
			A \text{ est inversible } \iff \varphi \text{ bijective }
		\]

		On pose $\psi : \begin{array}{rcl}
			\mathbbm{K}_n[X] &\longrightarrow& \mathbbm{K}_n[X] \\
			P &\longmapsto& P(X-1)
		\end{array}$

		 $\varphi$ et $\psi$ sont réciproques l'une de l'autre donc \[
		 	A^{-1} = \Mat_{\mathcal{B}}(\psi).
		 \]

		 \begin{align*}
			 \forall k \in \left\llbracket 1,n \right\rrbracket, \,
		 	\psi(X^k) &= (X-1)^k \\
			&= \sum_{i=0}^k {k \choose i} X^i (-1)^{k-i} \\
		 \end{align*}

		 Donc, \[
			A^{-1} = \begin{pNiceMatrix}
				1&-1&1&\Cdots&(-1)^n\\
				0&1&-2&\Cdots&(-1)^{n-1}\\
				\Vdots&\Vdots&\Vdots&\Ddots&\Vdots\\
				0&0&0&\Cdots&1\\
			\end{pNiceMatrix}
		 \] 
\end{enumerate}
