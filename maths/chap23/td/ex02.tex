\part{Exercice 2}

\begin{enumerate}
	\item[0.] On peut représenter un nombre de 4 chiffres comme un élément de $ \left\llbracket 0,9 \right\rrbracket ^4$; il y en a donc $10^4 = 10\,000$.

		On peut aussi représenter un tel nombre par un entier de $ \left\llbracket 0,9\,999 \right\rrbracket$; il y en a donc $10\,000$.

	\item On représente un tel nombre par un élément de $ \left\llbracket 1,9 \right\rrbracket \times \left\llbracket 0,9 \right\rrbracket^3$; il y en a $9 \times 10^3 = 9\,000$.
	\item~\\
		\underline{Cours} : 
		\begin{align*}
			A_{\#E}^n = \#\{(x_1, \ldots, x_n) \in E^n  \mid \forall i \neq j, x_i \neq x_j\}
			&= (\#E)(\#E - 1)\cdots(\#E - n + 1) \\
			&= \frac{(\#E)!}{(\#E - n)!} \\
		\end{align*}

		\[
			\bigcupdot_{x_1 = 1}^9 \underbrace{\big\{(x_1, x_2, x_3, x_4)  \mid (x_2,x_3,x_4) \in \mathcal{A}_3(\left\llbracket 0,9 \right\rrbracket \setminus \{x_1\})\big\}}_{F_{x_1}}
		\]

		On a $\#F = \sum_{x_1=1}^9 \#F_{x_1}$.

		Soit $\varphi : \begin{array}{rcl}
			F_{x_1} &\longrightarrow& \mathcal{A}_{3}(\left\llbracket 0,9 \right\rrbracket \setminus \{x_1\}) \\
			(x_1,x_2,x_3,x_4) &\longmapsto& (x_2,x_3,x_4),
		\end{array}$\\[3mm]
		et sa réciproque $\psi : \begin{array}{rcl}
			\mathcal{A}_3(\left\llbracket 0,9 \right\rrbracket \setminus \{x_1\}) &\longrightarrow& F_{x_1} \\
			(x_2,x_3,x_4) &\longmapsto& (x_1,x_2,x_3,x_4).
		\end{array}$

		Donc \[
			\#F_1 = A^3_9 = \frac{9!}{6!} = 7 \times 8 \times 9
		\] et donc \[
			\#F = \sum_{x_1 = 1}^9 504 = 9 \times 504 = 4\,536.
		\]
	\item
		\begin{itemize}
			\item[\underline{\sc Méthode 1}]
				On note \[
					F_1 = \{(x_1,x_2,x_3,x_4) \in \left\llbracket 1,9 \right\rrbracket \times \left\llbracket 0,9 \right\rrbracket^3  \mid \text{ parmi } (x_1,x_2,x_3,x_4), \text{ il y a 2 valeurs identiques exactement} \}
				\] \[
					F_2 = \{(x_1,x_2,x_3,x_4) \in \left\llbracket 1,9 \right\rrbracket \times \left\llbracket 0,9 \right\rrbracket^3  \mid \text{ parmi } (x_1,x_2,x_3,x_4), \text{ il y a 3 valeurs identiques exactement} \}
				\] \[
					F_3 = \{(x_1,x_2,x_3,x_4) \in \left\llbracket 1,9 \right\rrbracket \times \left\llbracket 0,9 \right\rrbracket^3  \mid \text{ parmi } (x_1,x_2,x_3,x_4), \text{ il y a 4 valeurs identiques exactement} \}.
				\]

				$\begin{array}{rcl}
					F_3 &\longrightarrow& \left\llbracket 1,9 \right\rrbracket \\
					(x_1, x_2, x_3, x_4) &\longmapsto& x_1
				\end{array}$ est une bijection donc $\#F_3 = 9$.

				\begin{align*}
					F_2 &= \{(x_1,x_1,x_1,x_4) \mid x_1 \in \left\llbracket 1,9 \right\rrbracket \et x_4 \in \left\llbracket 0,9 \right\rrbracket \text{ avec } x_1 \neq x_4\} \\
					&\cupdot \{(x_1,x_1,x_3,x_1)  \mid x_1 \in \left\llbracket 1,9 \right\rrbracket\et x_3 \in \left\llbracket 0,9 \right\rrbracket \text{ avec } x_1 \neq x_3\}  \\
					&\cupdot \{(x_1,x_2,x_1,x_1)  \mid x_1 \in \left\llbracket 1,9 \right\rrbracket\et x_2 \in \left\llbracket 0,9 \right\rrbracket \text{ avec } x_1 \neq x_2\}  \\
					&\cupdot \{(x_1,x_2,x_2,x_2)  \mid x_1 \in \left\llbracket 1,9 \right\rrbracket\et x_2 \in \left\llbracket 0,9 \right\rrbracket \text{ avec } x_1 \neq x_2\}  \\
				\end{align*}

				On a
				\begin{align*}
					\#F_2 &= 9 \times 9 + 9 \times  9 + 9 \times 9 + 9 \times 9\\
					&= 4\times 9\times 9 = 324. \\
				\end{align*}
				De même,
				\begin{align*}
					\#F_1 &= 6 \times 9 \times 9 \times 8 + 3 \times 9 \times 9 \\
					&= 4\,131? \\
				\end{align*}
				Il y en a $4464$.
			\item[\underline{\sc Méthode 2}] On passe au complémentaire :
				\[
					9\,000 - 4\,536 = 4\,464.
				\]
			\item[\underline{\sc Méthode 3}](fausse) \[
					6 \times 9 \times 10 \times 10 = 5\,400 \neq 4\,464.
				\]
		\end{itemize}
	\item \[
			7 \times 7 \times 6 \times 5 = 1470.
		\]
\end{enumerate}
