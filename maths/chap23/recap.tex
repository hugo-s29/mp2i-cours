\begin{multicols}{2}
	\begin{recap-box}[frametitle={Principe des tiroirs}]
		Soit $f: E \to F$ avec $\#E > \#F$. Alors \[
			\exists x\neq y \in E,\;f(x) = f(y)
		.\]
	\end{recap-box}
	\begin{recap-box}
		Soit $f : E \to F$ avec $\#E = \#F$. Alors
		\begin{align*}
			f \text{ injective} &\iff f \text{ surjective}\\
			&\iff f \text{ bijective}.
		\end{align*}
	\end{recap-box}
	\begin{recap-box}
		\[
			\#\left( \prod_{i=1}^{n} E_i \right) = \prod_{i=1}^{n} (\#E_i).
		\]
	\end{recap-box}
	\begin{recap-box}
		\[
			\#(E^n) = (\#E)^n
		.\]
	\end{recap-box}
	\begin{recap-box}
		Une $p$-liste est de la forme $(x_1, \ldots, x_p) \in E^p$.\\
	\end{recap-box}
	\begin{recap-box}
		Un $p$-arrangement est une $p$-liste d'éléments de $E$ distincts. Il y en a, avec $n = \#E$, \[
			\frac{n!}{(n-p)!}
		.\]
	\end{recap-box}
	\begin{recap-box}[frametitle={Principe des bergers}]
		Soit $f : E\to F$ telle que chaque élément $y \in F$ ait exactement $k$ antécédants dans $F$. Alors, \[
			\#E = k\#F
		.\] 
	\end{recap-box}
	\begin{recap-box}
		Une permutation est une bijection de $E$ dans $E$. Il y en a $n!$ si $\#E = n$.
	\end{recap-box}
	\begin{recap-box}
		Une $p$-combinaision de $E$ est une partie de $E$ de cardinal $p$. Il y en a ${n\choose p}$ si $n = \#E$.
	\end{recap-box}
	\begin{recap-box}
		\[
			\#(F^E) = (\#F)^{\#E}
		.\]
	\end{recap-box}
	\begin{recap-box}
		\[
			\#\mathcal{P}(E) = 2^{\#E}
		.\]
	\end{recap-box}
\end{multicols}
