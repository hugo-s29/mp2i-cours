\part{Exercice 6}

Soient $\lambda_1, \lambda_2, \lambda_3, \lambda_4, \lambda_5 \in \R$. On suppose \[
	\lambda_1 f_1 + \lambda_2 f_2 + \lambda_3 f_3 + \lambda_4 f_4 + \lambda_5 f_5 = 0
\] 

\begin{align*}
	\forall x > 0, f(x) = \lambda_1 \ln x + \lambda_2 x + \lambda_3 e^x + \lambda_4 e^{x+3} + \lambda_5 \frac{1}{x}= 0\\
\end{align*}

Si $\lambda_5 \neq 0$, alors $f(x) \sim_{x \to 0^+} \frac{\lambda_5}{x} \tendsto{x \to 0^+} \pm \infty$ $\lightning$\\

Donc $\boxed{\lambda_5 = 0}$ \\

Si $\lambda_1 \neq 0$, alors $f(x) \sim_{x \to 0^+} \lambda_1 \ln(x) \tendsto{x \to 0^+} \pm \infty$ $\lightning$\\

Donc $\boxed{\lambda_1 = 0}$ \\

Si $\lambda_3 + e^{3}\lambda_4 \neq 0$, alors $f(x) \sim_{x \to +\infty} \left( \lambda_3 + e^3 \lambda_4 \right)e^{x} \tendsto{x \to +\infty} \pm \infty$\\


Donc $\boxed{\lambda_3 + e^3 \lambda_4 = 0}$ \\

D'où, \[
	\forall x > 0, \lambda_2 x = 0
\] donc $\boxed{\lambda_2 = 0}$\\

$f_4 = e^3 f_3$ donc $(f_1, f_2, f_3, f_4, f_5)$ n'est pas libre\\
Mais, $(f_1, f_2, f_3, f_5)$ est libre ($\lambda_4 = 0$)








