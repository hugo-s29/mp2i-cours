\part{Fonctions à valeurs dans $\C$}

\begin{defn}

	\begin{center}
		\begin{asy}
			import graph;

			axes("$\Re$", "$\Im$", EndArrow);
			size(5cm);

			pair l = (2,3);
			dot("$\ell$", l);

			dot((-1/2,-1/2), white+0);
			dot((6,6), white+0);

			draw(circle(l, 1), deepcyan);
			label("$D$", l + (-1,0), deepcyan, align=W);

			draw((1,1)..(3,2)..(4,3)..(4,5)..(2,4.5)..(0,5)..(-1/2, 3)..(0,2.3)..cycle, red);
			label("$V$", (1,1), red, align=S);
		\end{asy}
	\end{center}

	$V$ est un \underline{voisinage} de $\ell$ s'il existe $r > 0$ tel que $V \supset D(\ell, r)$\\
	où $D(l,r) = \left\{z \in \C \mid \left| z - \ell \right| < r \right\}$
	\index{voisinage (complexe)}
\end{defn}

\begin{prop}
	Soit $f: I \to \C$ et $a \in I$, $\ell \in \C$. \[
		f(x) \tendsto{x \to a} \ell \iff \begin{cases}
			\Re(f(x)) \tendsto{x \to a} \Re(\ell)\\
			\Im(f(x)) \tendsto{x \to a} \Im(\ell)\\
		\end{cases}
	\] \qed
\end{prop}

\begin{rmk}
	[Rappel]
	On dit que $: I \to \C$ est \underline{bornée} s'il existe $M \in \R$ tel que \[
		\forall x \in I, \left| f(x) \right| \le M
	\]
\end{rmk}

