\part{Déterminant d'une matrice carrée}

\begin{defn}
	Soit $A = (a_{i,j})_{\substack{1\le i\le n\\1\le j\le n}} \in \mathcal{M}_n(\mathbbm{K})$.

	Le \underline{déterminant} de $A$ est \[
		\det(A) =
		\begin{vmatrix}
			a_{11}&\cdots&a_{1,n}\\
			\vdots&\ddots&\vdots\\
			a_{n,1}&\cdots&a_{n,n}
		\end{vmatrix} = 
		\sum_{\sigma \in S_n} \varepsilon(\sigma) \prod_{j=1}^n a_{\sigma(j), j}
	\]

	\index{déterminant (d'une matrice)}
\end{defn}

\begin{prop}
	Soit $E$ un $\mathbbm{K}$-espace vectoriel de dimension $n$, $\mathcal{B} = (e_1, \ldots, e_n)$ une base de $E$, $(u_1, \ldots, u_n) \in E^n$ et $A = \Mat_\mathcal{B}(u_1, \ldots,u_n)$.

	Alors, \[
		\det_\mathcal{B}(u_1, \ldots, u_n) = \det(A).
	\]
\end{prop}

\begin{prop}
	Soit $E$ un $\mathbbm{K}$-espace vectoriel de dimension $n$, $\mathcal{B} = (e_1, \ldots, e_n)$ une base de $E$, $f \in \mathcal{L}(E)$ et $A = \Mat_\mathcal{B}(f)$. Alors, $\det(A) = \det(f)$.
\end{prop}

\begin{prv}
	$\det(f) = \det_\mathcal{B}\big(f(e_1), \ldots, f(e_n)\big)$, $A = \Mat_\mathcal{B}\big(f(e_1), \ldots, f(e_n) \big)$.

	\begin{align*}
		\det A &= \det_\mathcal{B}\big(f(e_1), \ldots, f(e_n)\big) \\
		&= \det(f). \\
	\end{align*}
\end{prv}

\begin{prop}
	Soit $(A,B) \in \mathcal{M}_n(\mathbbm{K})^2$. \[
		\det(AB) = \det(A)\,\det(B).
	\]
\end{prop}

\begin{prv}
	Soit $E = \mathbbm{K}^n$, $\mathcal{B} = (e_1, \ldots, e_n)$ la base canonique de $E$, $f \in \mathcal{L}(E)$ tel que $A = \Mat_\mathcal{B}(f)$ et $g \in \mathcal{L}(E)$ tel que $B = \Mat_\mathcal{B}(g)$.

	\begin{align*}
		\det(AB) = \det(f \circ g) &= (\det f)\,(\deg g)\\
		&= \det(A)\,\det(B). \\
	\end{align*}
\end{prv}

\begin{prop}
	Soit $A \in \mathcal{M}_n(\mathbbm{K})$. \[
		A \in \mathrm{GL}_n(\mathbbm{K}) \iff \det(A) \neq 0
	\] Dans ce cas, $\det(A^{-1}) = \det(A)^{-1}$.
	\qed
\end{prop}

\begin{prop}
	\[
		\forall A \in \mathcal{M}_n(\mathbbm{K}), \det(\t A) = \det(A).
	\]
\end{prop}

\begin{prv}
	Soit $A \in \mathcal{M}_n(\mathbbm{K})$.

	\begin{align*}
		\det(\t A) &= \sum_{\sigma \in S_n} \varepsilon(\sigma) \prod_{j=1}^n a_{j, \sigma(j)}\\
		&= \sum_{\sigma \in S_n} \varepsilon(\sigma) \prod_{k=1}^n a_{\sigma^{-1}(k), k}  \\
		&= \sum_{\sigma' \in S_n}\varepsilon\left({\sigma'}^{-1}\right) \prod_{k=1}^n a_{\sigma'(k), k} \\
		&= \sum_{\sigma' \in S_n} \varepsilon(\sigma') \prod_{k=1}^n a_{\sigma'(k),k} \\
		&= \det(A) \\
	\end{align*}
	car
	\begin{align*}
		\forall \sigma' \in S_n, \varepsilon\left( {\sigma'}^{-1} \right) = \varepsilon(\sigma')^{-1} &= \begin{cases}
			1 &\text{ si } \varepsilon(\sigma') = 1\\
			-1 &\text{ si } \varepsilon(\sigma') = -1
		\end{cases} \\
		&= \varepsilon(\sigma').
	\end{align*}
\end{prv}

\begin{prop}
	Soit $A \in \mathcal{M}_n(\mathbbm{K})$, $C$ une opération sur les colonnes et $A'$ la matrice obtenue en appliquant $C$ sur les colonnes $A$.

	\begin{enumerate}
		\item Si $C = c_i \longleftrightarrow c_j$ (avec $i\neq j$), alors $\det(A') = -\det(A)$.
		\item Si $C = c_i \longleftrightarrow \lambda c_i$, alors $\det(A') = \lambda \det(A)$.
		\item Si $C = c_i \longleftrightarrow c_i + \lambda c_j$ (avec $i \neq j$), alors $\det(A') = \det(A)$.
	\end{enumerate}
\end{prop}

\begin{prv}
	Soit $(u_1, \ldots, u_n) \in \left(\mathbbm{K}^n\right)^n$ tel que \[
		\forall i \in \left\llbracket 1,n \right\rrbracket, c_i = \Mat_\mathcal{B}(u_i)
	\] où $\mathcal{B}$ est la base canonique de $\mathbbm{K}^n$ et $c_i$ la $i$-ème colonne de $A$.

	\begin{enumerate}
		\item
			\begin{align*}
				-\det(A) = -\det_{\mathcal{B}}(u_1&, \ldots, u_n)\\
				&\vrt=\\
				\det(A') = \det_{\mathcal{B}}(u_1, \ldots, u_{i-1}, u_j, u_{i+1}, \ldots, u_{j-1}&, u_i, u_{j+1}, \ldots, u_n).
			\end{align*}
		\item
			\begin{align*}
				\det(A') &= \det_\mathcal{B}(u_1, \ldots, u_{i-1}, \lambda u_i, u_{i+1}, \ldots, u_n) \\
				&= \lambda \det_\mathcal{B}(u_1, \ldots, u_{i-1}, u_i, u_{i+1}, \ldots, u_n) \\
				&= \lambda\,\det(A). \\
			\end{align*}
		\item
			\begin{align*}
				\det(A') &= \det_\mathcal{B}(u_1, \ldots, u_{i-1}, u_i + \lambda u_j, u_{i+1}, \ldots, u_n) \\
				&= \det_\mathcal{B}(u_1, \ldots, u_{i-1}, u_i, u_{i+1}, \ldots, u_n) + \lambda \det_\mathcal{B}(u_1, \ldots, u_{i-1}, u_j, u_{i+1}, \ldots, u_n) \\
				&= \det(A) \\
			\end{align*}
			car, comme $u_j$ apparaît deux fois dans le second détermiant, il vaut $0$.
	\end{enumerate}
\end{prv}

\begin{crlr}
	Le déterminant d'une matrice triangulaire est le produit de ses coefficients diagonaux.
\end{crlr}

\begin{prv}
	Soit $T = \begin{pNiceMatrix}
		\lambda_1 & * & \cdots & *\\
		&\ddots&\ddots&\vdots\\
		(0)&&&*\\
		&&\lambda_n
	\end{pNiceMatrix}$. S'il existe $i \in \left\llbracket 1,n \right\rrbracket$, avec $\lambda_i = 0$, alors $\rg(T) < n$ et donc $T \not\in \mathrm{GL}_n(\mathbbm{K})$ et donc $\det(T) = 0 = \prod_{k=1}^n \lambda_k$.

	On suppose que $\forall i, \lambda_i \neq 0$. On a \[
		\frac{1}{\lambda_1} \det(T) = \det \begin{pNiceMatrix}
			1&*&\cdots&*\\
			0&\lambda_2&\ddots&\vdots\\
			\vdots&\ddots&\ddots&*\\
			0&\cdots&0&\lambda_n
		\end{pNiceMatrix}
	\] et \[
		\frac{1}{\lambda_1 \lambda_2} \det(T) = \det \begin{pmatrix}
			1&*&\cdots&\cdots&*\\
			0&1&\ddots&&\vdots\\
			\vdots&0&\lambda_3&\ddots&\vdots\\
			\vdots&\vdots&\ddots&\ddots&*\\
			0&0&\cdots&0&\lambda_n
		\end{pmatrix}
	\]
	D'où
	\begin{align*}
		\frac{1}{\lambda_1 \cdots \lambda_n} \det(T) &= \begin{vNiceMatrix}
			1&&\Block{2-2}{*}&\\
			&\Ddots&&\\
			\Block{2-2}{(0)}&&&\\
			&&&1
		\end{vNiceMatrix} \\
		&= \begin{vNiceMatrix}
			1&&\Block{2-2}{(0)}&\\
			&\Ddots&&\\
			\Block{2-2}{(0)}&&&\\
			&&&1
		\end{vNiceMatrix}\\
		&= 1 \\
	\end{align*}
	donc $\det(T) = \lambda_1 \cdots \lambda_n$.
\end{prv}

\begin{prop}
	Soit $A \in \mathcal{M}_n(\mathbbm{K})$, $L$ une opération sur les lignes, $A'$ la matrice obtenue en appliquant $L$ sur les lignes de $A$.

	\begin{enumerate}
		\item Si $L = \ell_i \longleftrightarrow \ell_j$ (avec $i \neq j$), $\det(A') = -\det(A)$.
		\item Si $L = \ell_i \longleftrightarrow \lambda\ell_i$, $\det(A') = \lambda\det(A)$.
		\item Si $L = \ell_i \longleftrightarrow \ell_i + \lambda \ell_j$ (avec $i \neq j$), $\det(A') = \det(A)$.
	\end{enumerate}
	\qed
\end{prop}

\begin{exm}
	$(a,b,c) \in \C^3$

	\begin{align*}
		\begin{vmatrix}
			a&b&c\\
			b&a&c\\
			c&a&b
		\end{vmatrix} &=
		\begin{vmatrix}
			a+b+c & b & c\\
			a+b+c & a & c\\
			a+b+c & a & b\\
		\end{vmatrix}\\
		&= (a+b+c) \begin{vmatrix}
			1&b&c\\
			1&a&c\\
			1&a&b
		\end{vmatrix}\\
		&= (a+b+c) \begin{vmatrix}
			1&b&c\\
			0&a-b&0\\
			0&0&b-c
		\end{vmatrix}\\
		&= (a+b+c)(a-b)(b-c) \\
	\end{align*}
\end{exm}

