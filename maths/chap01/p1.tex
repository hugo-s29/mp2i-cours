\part{Sommes}

\begin{rmk}[Notation]
	Soit $(u_n)_{n\in\N}$ une suite de nombres complexes. Pour $p \le q \in \N$, on note \[
		\sum_{k=p}^q u_k
	\] le nombre $u_p + u_{p+1} + \cdots + u_q$.

	Par convention, \[
		\sum_{k=p}^q u_k = 0 \qquad \text{ si } q < p.
	\]
\end{rmk}

\begin{exm}
	\[
		\forall n \in \N^*, \sum_{k=1}^n k = 1 + 2 + \cdots + n = \frac{n(n+1)}{2}.
	\]

	\begin{figure}[H]
		\centering
		\begin{asy}
			size(3cm);
			for(int i = 0; i < 5; ++i) {
				for(int j = 0; j < 4; ++j) {
					if(i >= (4 - j)) {
						draw(shift((i,j)) * unitsquare, gray);
					}
				}
			}
			for(int i = 0; i < 5; ++i) {
				for(int j = 0; j < 4; ++j) {
					if(i < (4 - j)) {
						draw(shift((i,j)) * unitsquare);
					}
				}
			}
			
			draw(shift((0.3, 0)) * brace((5, 4), (5, 0)));
			draw(shift((0, -0.3)) * brace((5, 0), (0, 0)));

			label("$n$", (5.8, 2), align=E);
			label("$n+1$", (2.5, -0.8), align=S);
		\end{asy}
	\end{figure}
\end{exm}

\begin{prop}
	Soit $(u_n)_{n\in\N}$ une suite de nombres complexes. \[
		\forall n \in \N, \sum_{k = 0}^n u_{k} = \sum_{k=0}^n u_{n-k}
	\]
\end{prop}

\begin{prv}
	Soit $n \in \N$.

	\begin{align*}
		\sum_{k=0}^n u_{n-k} &= u_n + u_{n-1} + u_{n-2} + \cdots + u_0 \\
		&= u_0 + u_1 + u_2 + \cdots + u_n \\
		&= \sum_{k=0}^n u_k \\
	\end{align*}
\end{prv}

\begin{prop}
	Soient $(u_n)_{n\in\N}$ une suite de nombres complexes, $(p,q,r,s) \in \N^4$ et $\varphi: \left\llbracket p,q \right\rrbracket \to \left\llbracket r,s \right\rrbracket$ une bijection
	(i.e. $\forall y \in \left\llbracket r,s \right\rrbracket, \exists!x \in \left\llbracket p,q \right\rrbracket, \varphi(x) = y$).

	Alors, \[
		\sum_{k=r}^s u_k = \sum_{k=p}^q u_{\varphi(k)}.
	\]
\end{prop}


\begin{prv}
	\[
		\sum_{k=p}^q u_{\varphi(k)} = u_{\varphi(p)} + u_{\varphi(p+1)} + \cdots + u_{\varphi(q)}
	\] \[
		\sum_{k=r}^s u_k = u_r + u_{r+1} + \cdots + u_s
	\]
	Comme $\varphi$ est bijective, chaque terme $u_k$ avec $k \in \left\llbracket r,s \right\rrbracket$ apparaît une fois et une seule fois dans la somme \[
		u_{\varphi(p)} + u_{\varphi(p+1)} + \cdots + u_{\varphi(q)}.
	\] Ainsi, les deux sommes sont identiques.
\end{prv}

\begin{exm}
	On pose \[
		\forall k \in \N, u_k = \frac{1}{k - 4}.
	\] On pose également $\varphi: \left\llbracket 1,5 \right\rrbracket \to \left\llbracket -1, 3 \right\rrbracket$ : une bijection.

	Alors, \[
		\sum_{k=-1}^3 \frac{1}{k-4} = -\frac{1}{5} - \frac{1}{4} - \frac{1}{3} - \frac{1}{2} - 1
	\] et
	\begin{align*}
		\sum_{k=1}^5 u_{\varphi(k)} &= \frac{1}{\varphi(1) - 4}  + \frac{1}{\varphi(2)-4} + \frac{1}{\varphi(3)-4} + \frac{1}{\varphi(4)-4} + \frac{1}{\varphi(5)-4}\\
		&= -\frac{1}{5} - \frac{1}{3} - \frac{1}{3} - \frac{1}{2} - 1. \\
	\end{align*}

\end{exm}

\begin{exm}
	Soit $\varphi$ la bijection définie par \begin{align*}
		\varphi: \left\llbracket 1,5 \right\rrbracket  &\longrightarrow \left\llbracket 1,5 \right\rrbracket  \\
		k &\longmapsto \begin{cases}
			2 &\text{ si } k = 1\\
			3 &\text{ si } k = 2\\
			1 &\text{ si } k = 3\\
			4 &\text{ si } k = 4\\
			5 &\text{ si } k = 5\\
		\end{cases}
	\end{align*}

	\begin{figure}[H]
		\centering
		\begin{asy}
			size(7cm);
			real y = -1.5; real eps = 0.25; real eps2 = eps;
			real e = 0.2;

			draw((2-e,-eps2) -- (1+e,y+eps2), Arrows(TeXHead));
			draw((e,-eps2) -- (2-e,y+eps2), white + 5); draw((e,-eps2) -- (2-e,y+eps2), Arrows(TeXHead));
			draw((1-e,-eps2) -- (e,y+eps2), white + 5); draw((1-e,-eps2) -- (e,y+eps2), Arrows(TeXHead));
			draw((4,-eps2) -- (4,y+eps2), Arrows(TeXHead));
			draw((3,-eps2) -- (3,y+eps2), Arrows(TeXHead));

			label("$\sum_{k=1}^5 u_k$", (-1.5, eps));
			label("$\sum_{k=1}^5 u_{\varphi(k)}$", (-1.5, y + eps));

			label("$=$", (-0.6, 0)); label("$=$", (-0.5, y));
			label("$u_1$", (0, 0)); label("$u_2$", (0, y));
			label("$+$", (0.5, 0)); label("$+$", (0.5, y));
			label("$u_2$", (1, 0)); label("$u_3$", (1, y));
			label("$+$", (1.5, 0)); label("$+$", (1.5, y));
			label("$u_3$", (2, 0)); label("$u_1$", (2, y));
			label("$+$", (2.5, 0)); label("$+$", (2.5, y));
			label("$u_4$", (3, 0)); label("$u_4$", (3, y));
			label("$+$", (3.5, 0)); label("$+$", (3.5, y));
			label("$u_5$", (4, 0)); label("$u_5$", (4, y));
		\end{asy}
	\end{figure}
\end{exm}

\begin{prop}[téléscopage]
	Soit $(u_k)_{k\in\N}$ une suite de nombres complexes. \[
		\forall p \le q \in \N, \sum_{k=p}^q (u_{k+1}- u_k) = u_{q+1}- u_p.
	\]
\end{prop}

\begin{prv}
	\begin{itemize}
		\item[\underline{\sc Méthode 1}] Soient $p \le q$.
			\begin{align*}
				\sum_{k=p}^q(u_{k+1} - u_k) &= \cancel{u_{p+1}} - u_p + \cancel{u_{p+2}} - \cancel{u_{p+1}} \cdots + u_{q+1}- \cancel{u_{q}} \\
				&= u_{q+1} - u_p \\
			\end{align*}
		\item[\underline{\sc Méthode 2}] Soient $p \le q$.
			\[
				\sum_{k=p}^q (u_{k+1} - u_k) = \sum_{k=p}^q u_{k+1} - \sum_{k=p}^q u_k
			\] Soit  $\varphi : \begin{array}{rcl}
				\left\llbracket p,q \right\rrbracket &\longrightarrow&\left\llbracket p+1,q+1 \right\rrbracket  \\
				k &\longmapsto& k+1
			\end{array}$. $\varphi$ est bijective donc \[
				\sum_{k=p}^q u_{k+1} = \sum_{k=p}^q u_{\varphi(k)} = \sum_{k=p+1}^{q+1} u_k.
			\] D'où,
			\begin{align*}
				\sum_{k=p}^q (u_{k+1} - u_k) &= \sum_{k=p+1}^{q+1} u_k - \sum_{k=p}^q u_k \\
				&= \left( u_{q+1} + \sum_{k=p+1}^q u_k \right) - \left( u_p  + \sum_{k=p+1}^q u_k \right) \\
				&= u_{q+1}-u_p \\
			\end{align*}
	\end{itemize}
\end{prv}

\begin{rmk}[Analogie avec le calcul intégral]
	\[
		\int_{a}^{b} f'(x)~\mathrm{d}x = f(b) - f(a).
	\]
\end{rmk}

\begin{exm}
	Calculer $\sum_{k=1}^{n} \frac{1}{k(k+1)}$ pour $n \in \N^*$.

	Soit $n \in \N^*$.
	\begin{align*}
		\forall k \in \left\llbracket 1,n \right\rrbracket, \frac{1}{k(k+1)} &= \frac{(1+k) - k}{k(k+1)} \\
		&= \frac{k+1}{k(k+1)} - \frac{k}{k(k+1)} \\
		&= \frac{1}{k} - \frac{1}{k+1} \\
	\end{align*}
	et, par téléscopage, on obtient donc
	\begin{align*}
		\sum_{k=1}^n \frac{1}{k(k+1)} &= \sum_{k=1}^n \left( \frac{1}{k} - \frac{1}{k+1} \right) \\
		&= \frac{1}{1} - \frac{1}{n+1} \\
		&= 1 - \frac{1}{n+1}. \\
	\end{align*}

	Par contre, on n'a pas de formule simple pour $\sum_{k=1}^n \frac{1}{k^2}$ (mais on sait que $\sum_{k=1}^n \tendsto{n \to +\infty} \frac{\pi^2}{6}).$
\end{exm}

\begin{exm}[à connaître]
	Calculer $\sum_{k=1}^n k^2$ et $\sum_{k=1}^n k^3$ pour $n \in \N^*$.

	On cherche $(u_n)_{n\in\N}$ telle que \[
		\forall k \in \N^*, u_{k+1} - u_k = k^2.
	\] On cherche donc $(u_k)$ sous la forme \[
		\forall k \in N^*, u_k = ak^3 + bk^2 + ck + d
	\] avec $(a,b,c,d) \in \R^4$.

	Soit $k \in \N^*$.

	\begin{align*}
		u_{k+1} - u_k &= a(k+1)^3 + b(k+1)^2 + c(k+1) + \cancel d - ak^3 - bk^2 + ck + \cancel d \\
		&= a(\cancel{k^3} + 3k^2 + 3k + 1) + b(\cancel{k^2} + 2k + 1) + c(\cancel k + 1) - \cancel{ak^3} - \cancel{bk^2} - \cancel{ck} \\
		&= k^2\times a + k(3a + 2b) + (a + b + c) \\
	\end{align*}

	On résout le système
	\begin{align*}
		(S): \quad &\begin{cases}
			3a = 1,\\
			3a + 2b = 0,\\
			a + b + c.
		\end{cases}
		\iff& \begin{cases}
			a = \frac{1}{3},\\
			b = -\frac{1}{2},\\
			c = \frac{1}{2} - \frac{1}{3} = \frac{1}{6}.
		\end{cases}
	\end{align*}

	On vient de montrer que, \[
		\forall k \in N^*, k^2 = u_{k+1} - u_k \qquad \text{ avec } u_k = \frac{1}{3}k^3 - \frac{1}{2}k^2 + \frac{1}{6}k.
	\] Donc, par téléscopage,
	\begin{align*}
		\forall n \in \N^*, \sum_{k=1}^n k^2 &= u_{n+1} - u_1\\
		&= \frac{1}{3}(n+1)^3 - \frac{1}{2}(n+1)^2 + \frac{1}{6}(n+1) - \cancel{\frac{1}{3}} + \cancel{\frac{1}{2}} - \cancel{\frac{1}{6}} \\
		&= \frac{n+1}{6}\big(2(n+1)^2 - 3(n+1) + 1\big) \\
		&= \frac{n+1}{6}(2n^2 + 4n + 2 - 3n - 3 +1) \\
		&= \frac{n(n-1)(2n+1)}{6} \\
	\end{align*}
\end{exm}

\begin{prop}
	Soit $n \in \N$ et $q \in \C$, \[
		\sum_{k=0}^q q^k = \begin{cases}
			n+1 &\text{ si } q = 1\\
			\frac{1-q^{n+1}}{1-q} &\text{ sinon}.
		\end{cases} 
	\] 
\end{prop}

\begin{prv}
	Soit $n \in \N$.
	\begin{enumerate}
		\item 
			\begin{align*}
				(1-q) \sum_{k=0}^n q^k &= \sum_{k=0}^n q^k - q\sum_{k=0}^n q^k \\
				&= \sum_{k=0}^n q^k - \sum_{k=0}^n q^{k+1} \\
				&= \sum_{k=0}^n (q^k - q^{k+1}) \\
				&= 1 - q^{n+1} \\
			\end{align*}

			Si $q \neq 1$, \[
				\sum_{k=0}^n q^k = \frac{1-q^{n+1}}{1-q}.
			\]

			Si $q = 1$, \[
				\sum_{k=0}^n q^k = \sum_{k=0}^n 1 = n+1.
			\]
		\item On pose, pour $n \in \N$, $S_n = \sum_{k=0}^n q^k$.

			On a, d'une part :
			\[
				\forall n \in \N, S_{n+1} = \sum_{k=0}^{n+1} q^k = S_n + q^{n+1}
			\] et d'autre part \[
				\forall n \in \N, S_{n+1} = 1 + \sum_{k=1}^{n+1} q^k = 1 + qS_n.
			\] Et donc,
			\begin{align*}
				\forall n \in \N, 1+q S_n = S_n + q^{n+1} \iff& 1 + (q - 1) S_n = q^{n+1}\\
				\iff& S_n(q-1) = q^{n+1}- 1\\
				\iff& S_n = \frac{q^{n+1} - 1}{q-1} \text{ pour } q \neq 1.
			\end{align*}
	\end{enumerate}
\end{prv}
