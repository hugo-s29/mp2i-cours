\part{Formules à connaître}

\begin{defn}
	Soient $k, n \in \N$. On définit ``$k$ parmi $n$'' par \[
		{n \choose k} = \begin{cases}
			\frac{n!}{k!~(n-k)!} &\text{ si } k\le n,\\
			0 &\text{ sinon}.
		\end{cases}
	\]
\end{defn}

\begin{prop}
	Avec les notations précédentes, 
	\begin{enumerate}
		\item ${n\choose k} = {n \choose n - k}$
		\item ${n+1\choose k+1} = {n \choose k + 1} + {n \choose k}$
	\end{enumerate}
\end{prop}

\begin{prv}
	\begin{enumerate}
		\item ${n\choose n-k} = \frac{n!}{(n-k)!\big(n-(n-k)\big)!} = \frac{n!}{(n-k)!~k!} = {n \choose k}$
		\item
			\begin{align*}
				{n \choose k + 1} + {n \choose k} &= \frac{n!(n-k)}{(k+1)!(n-k-1)!(n-k)} + \frac{n!(k+1)}{k!(n-k)!(k+1)} \\
				&= \frac{n!(n - \cancel k + \cancel k + 1)}{(k+1)!(n-k)!} \\
				&= \frac{(n+1)!}{(k+1)!(n-k)!} \\
				&= {n+1\choose k+1} \\
			\end{align*}
	\end{enumerate}
\end{prv}

\begin{prop}[binôme de Newton]
	Soient $(a,b) \in \C^2$ et $n \in \N$. Alors \[
		(a+b)^n = \sum_{k=0}^n {n \choose k} a^k b^{n-k}
	\] sauf si $\begin{cases}
		a + b = 0,\\
		n = 0.
	\end{cases}$
\end{prop}

\begin{rmk}[triangle de Pascal]
	\begin{tabular}{>{$n=}l<{$\hspace{12pt}}*{13}{c}}
		0 &&&&&&&1&&&&&&\\
		1 &&&&&&1&&1&&&&&\\
		2 &&&&&1&&2&&1&&&&\\
		3 &&&&1&&3&&3&&1&&&\\
		4 &&&1&&4&&6&&4&&1&&\\
		5 &&1&&5&&10&&10&&5&&1&\\
		6 &1&&6&&15&&20&&15&&6&&1
	\end{tabular}
\end{rmk}

\begin{exm}
	\[
		\forall n \in \N, \sum_{k=0}^n {n \choose k} = \sum_{k=0}^n {n \choose k} 1^k 1^{n-k} = (1 + 1)^n = 2^n.
	\] 
\end{exm}

\begin{prv}
	Soient $(a,b) \in \C^2$.

	Pour tout $n \in \N$, on pose \[
		P(n): ``(a+b)^n = \sum_{k=0}^n {n\choose k} a^k b^{n-k}"
	\] avec la convention $\forall z \in \C, z^0 = 1$.

	\begin{itemize}
		\item Soit $n \in \N$. On suppose $P(n)$ vraie. Montrons $P(n+1)$.

			\begin{align*}
				(a+b)^{n+1} &= (a+b)(a+b)^n \\
				&= (a+b)\sum_{k=0}^n {n \choose k} a^k b^{n-k} \\
				&= a \sum_{k=0}^n{n\choose k}a^k b^{n-k} + b \sum_{k=0}^n{n\choose k}a^k b^{n-k}\\
				&= \sum_{k=0}^n {n\choose k} a^{k+1} b^{n-k} + \sum_{k=0}^{n} a^k b^{n+1-k} \\
			\end{align*}

			On pose $\varphi : \begin{array}{rcl}
				\left\llbracket 0,n \right\rrbracket  &\longrightarrow&\left\llbracket 1, n+1 \right\rrbracket\\
				k &\longmapsto& k+1
			\end{array}$ bijective.

			Donc,
			\begin{align*}
				\sum_{k=0}^n {n\choose k} a^k b^{n-k} &= \sum_{k=0}^n u_{\varphi(k)} \\
				&= \sum_{k=1}^{n+1} u_k \text{ où } \forall n \in \left\llbracket 1,n+1 \right\rrbracket, u_k = {n \choose k - 1} a^k b^{n-k+1}. \\
			\end{align*}

			D'où
			\begin{align*}
				(a+b)^{n+1} &= \sum_{k=1}^{n+1} {n \choose k-1}a^k + b^{n+1-k} + \sum_{k=0}^n a^k b^{n+1-k} \\
				&= {n \choose n} a^{n+1}b + \sum_{k=1}^n {n \choose k-1} a^k b^{n+1-k} \\
				&+ {n \choose 0} a^0 b^{n+1} + \sum_{k=1}^n {n \choose k} a^k b^{n+1-k} \\
				&= \sum_{k=1}^n \left( {n \choose k} + {n \choose k -1} \right) a^k b^{n+1-k} + a^{n+1} + b^{n+1} \\
				&= \sum_{k=1}^n  {n+1\choose k}a^k b^{n+1-k}. \\
			\end{align*}
		\item Montrons $P(0)$ $(a+b)^0 = 1$.
			\[
				\sum_{k=0}^0 {0 \choose k} a^k b^{0-k} = {0\choose 0} a^0 b^0 = 1 \times  1 \times 1
			\] Donc, \[
				(a+b)^0 = \sum_{k=0}^0 {0\choose k} a^k b^{0-k}
			\]
	\end{itemize}
\end{prv}

\begin{exm}
	Calculer, pour tout $n \in \N$, $\sum_{k=0}^n (-1)^k {n\choose k}$. On applique la formule du binôme de Newton avec $a = -1$ et $b = 1$:
	\[
		\forall n \in \N, \sum_{k=0}^n (-1)^k {n\choose k} = (-1 + 1)^n = \begin{cases}
			0 \text{ si } n > 0,\\
			1 \text{ si } n = 0.
		\end{cases}
	\]
\end{exm}

\begin{prop}
	Soient $(a,b) \in \C^2$ et $n \in \N$. Alors, \[
		a^n - b^n = (a-b) \sum_{k=0}^{n-1}a^k b^{n-1-k}.
	\] 
\end{prop}

\begin{prv}
	\begin{align*}
		(a-b) \sum_{k=0}^{n-1}a^k &= \sum_{k=0}^{n-1} a^{k+1} b^{n-1-k} + \sum_{k=0}^{n-1}a^k b^{n-k} \\
			&= \sum_{k=0}^{n-1}\big(\underbrace{a^{k+1}b^{n-(k+1)}}_{u_{k+1}}-\underbrace{a^k b^{n-k}}_{u_k}\big)  \\
			&= u_n-u_0 \\
			&= a^n - b^n \\
	\end{align*}
\end{prv}



