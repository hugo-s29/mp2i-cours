\begin{multicols}{2}
	\begin{recap-box}[frametitle={Groupe}]
		On dit que $(G, \diamond)$ est un groupe si
		\begin{itemize}
			\item $\diamond$ est associative;
			\item $\diamond$ a un neutre $e \in G$;
			\item tout élément $x \in E$ a un inverse $y \in E$ :  \[
					x\diamond y = y \diamond x = e
				.\]
		\end{itemize}
		\mdfsubtitle{Sous-groupe}~\\[-5mm]
		On dit que $H \subset G$ est un sous-groupe de $G$ si
		\begin{itemize}
			\item $e \in H$;
			\item $\forall x,y \in H,\;x\diamond y \in H$;
			\item $\forall x \in H,\;x^{-1} \in H$.
		\end{itemize}
	\end{recap-box}
	\begin{recap-box}
		Si $\diamond$ est commutative, on dit que $(G, \diamond)$ est un groupe commutatif ou abélien.
	\end{recap-box}
	\begin{recap-box}
		Pour monter que $H$ est un sous-groupe de $G$, on montre
		\begin{itemize}
			\item $H \neq \O$;
			\item $\forall x,y \in H,\;x \diamond y^{-1} \in H$.
		\end{itemize}
	\end{recap-box}
	\begin{recap-box}
		L'intersection de sous-groupes est un sous-groupe. Attention, l'union de sous-groupes n'est pas forcément un sous-groupe.
	\end{recap-box}
	\begin{recap-box}[frametitle={Sous-groupe engendré}]
		Le sous-groupe engendré par $A$, $\left<A \right>$, est le plus petit sous groupe de $G$ contenant $A$.

		S'il existe $a \in G$ tel que $G = \left<a \right>$, on dit que $G$ est monogène et $a$ est un générateur de $G$.

		Soit $a \in G$. L'ordre de $a$ est $\#\left<a \right>$ i.e. $a^n=e$.
	\end{recap-box}
	\begin{recap-box}[frametitle={Morphisme de groupes}]
		Soit $f : G_1 \to G_2$ où $(G_1, \cdot)$ et $(G_2, \times )$ sont des groupes. $f$ est un morphisme de groupes si \[
			\forall x,y \in G_1,\;f(x \cdot y) = f(x) \times f(y)
		.\] L'image directe d'un sous-groupe de $G_1$ est un sous-groupe de $G_2$.
			L'image réciproque d'un sous-groupe de $G_2$ est un sous-groupe de $G_1$. \[
				\forall  u\in G_1;\,f\left( u^{-1} \right) = f(u)^{-1}
			.\]
	\end{recap-box}
	\begin{recap-box}
		\[
			f \text{ injective} \iff \Ker f = \{e_1\}
		.\]
	\end{recap-box}
\end{multicols}
