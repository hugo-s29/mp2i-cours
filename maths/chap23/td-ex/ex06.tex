\part{Exercice 6}

On note le monde $W = (P_1, \ldots, P_n)$ où, pour tout $i$, $P_i$ est un pays. On note $n$ le nombre de pays dans le monde. On pose l'application $f: W \to \left\llbracket 1,n-1 \right\rrbracket$ qui associe le pays à son nombre de voisins. 

Comme $\#W = n$ et $\#\left\llbracket 1,n-1 \right\rrbracket = n - 1$, et d'après le principe des tiroirs, il existe $P$ et $P'$ deux pays différents avec $f(P) = f(P')$. Donc, les pays $P$ et $P'$ ont le même nombre de voisin.

S'il est possible d'avoir un (ou plusieurs) pays qui n'ont pas de voisins (une île par exemple), alors le raisonement précédent est faux. Il est donc impossible pour un pays d'avoir $n - 1$ voisins et donc $\mathrm{Im}\,f \subset \left\llbracket 0,n-2 \right\rrbracket$ et donc, le raisonement précédent s'applique presque de la même manière.

