\let\larr\leftarrow

\begin{exm}
	\begin{align*}
		(S_1): &\left\{\begin{array}{c c c c c}
			\quad \overbrace{\fbox{$x$}}^{\text{pivot}} &+ y &+ z &- t &= 1\\
			\quad x &+ 2y &+ 3z &+ t &= 0 \\
			\quad x &&+ z &= 2 \\
			\quad
		\end{array}\right.\\
		\begin{array}{c}
			\iff\\
			L_2 \leftarrow L_2 - L_1\\
			L_3 \leftarrow L_3 - L_1
		\end{array}
		&\left\{\begin{array}{c c c c c}
			\fbox{$x$} &+ y &+ z &- t &= 1\\
			&y &+2z &+2t = -1\\
		\end{array}\right.\\
		\begin{array}{c}
			\iff\\
			L_1 \leftarrow L_1 - L_2\\
			L_3 \leftarrow L_3 + L_2
		\end{array}
		&\left\{
		\begin{array}
			{c c c c c}
			\fbox{$x$} &&-z &-3t &=2\\
			& \fbox{$y$} &+2z &+2t &=-1\\
			&&2z &+3t &= 0\\
		\end{array}
		\right.\\
		\begin{array}{c}
			\iff\\
			L_1 \leftarrow L_1 + L_3\\
			L_2 \leftarrow L_2 + \frac{2}{3} L_3
		\end{array}
		&\begin{cases}
			\fbox{$x$} + z = 2\\
			\fbox{$y$} + \frac{2}{3}z = -1\\
			\fbox{$3t$} + 2z = 0\\
		\end{cases}
		&\iff \begin{cases}
			x = 2 - z\\
			y - -1 - \frac{2}{3}z\\
			t = -\frac{2}{3}z
		\end{cases}
	\end{align*}
	L'ensemble des solutions est 
	\[
		\left\{ \left( 2-z, -1-\frac{2}{3}z, z, -\frac{2}{3}z \right) \mid z \in \mathbbm{K} \right\} 
	\] 
	\begin{align*}
		&\begin{cases}
			x + y + z - t = 1\\
			x + 2y + 3z + t = 0\\
			x + \bx{z} = 2
		\end{cases}\\
		\begin{array}{c}
			\iff\\
			L_1 \larr L_1 - L_3\\
			L_2 \larr L_2 - 3L_3\\
		\end{array}&
		\begin{cases}
			\bx{y} - t = -1\\
			-2x + 2y +t = -6\\
			x + \bx{z} = 2
		\end{cases}\\
		\begin{array}{c}
			\iff\\
			L_2 \larr \frac{L_2 - 2L_1}{-2}
		\end{array}&
		\begin{cases}
			\bx{y} - t = -1\\
			\bx{x} - \frac{3}{2}t = 2\\
			x + \bx{z} = 2
		\end{cases}\\
		\begin{array}{c}
			\iff \\
			L_3 \larr L_3 - L_2
		\end{array}&
		\begin{cases}
			\bx{y} - t = -1\\
			\bx{x} - \frac{3}{2}t = 2\\
			\bx{z} + \frac{3}{2}t = 0
		\end{cases}\\
		\iff&
		\begin{cases}
			y = -1+t\\
			x = 2 + \frac{3}{2}t\\
			z = -\frac{3}{2}t
		\end{cases}
	\end{align*}

	\begin{align*}
		\mathcal{S} &= \left\{(2+\frac{3}{2}t, -1+t, -\frac{3}{2}t, t \mid  t \in \mathbbm{K}\right\}  \\
		&= \{\underbrace{(2, -1, 0, 0)}_{A} + t\underbrace{\left( \frac{3}{2}, 1, -\frac{3}{2}, 1 \right)}_{u}  \mid  t \in \mathbbm{K}\}  \\
	\end{align*}
	% Ajouter graphique
\end{exm}

\begin{exm}
	\begin{align*}
		&\begin{cases}
			x + y +z = 0\\
			x - y + z = 1\\
			2x - y + z = 2\\
			x - y - z = 3\\
			y + 3z = 1
		\end{cases}\\
		\begin{array}{c}
			\iff\\
			L_2\larr \frac{L_2-L_1}{-2}\\
			L_3\larr L_3 - 2L_1\\
			L_4\larr L_4 - L_1\\
		\end{array}&
		\begin{cases}
			\bx{x} + y + z = 0\\
			\bx{y} = \frac{1}{2}\\
			-3y - z = 2\\
			-2y -2z = 3\\
			y + 3z = 1
		\end{cases}\\
		\begin{array}{c}
			\iff\\
			L_1\larr L_1-L_2\\
			L_3 \larr -\left( L_3 - 3L_2 \right)\\
			L_4 \larr L_4 +3 L_2\\
			L_5 \larr L_5 - L_2
		\end{array} &
		\begin{cases}
			\bx{x} + z = \frac{1}{2}\\
			\bx{y} = -\frac{1}{2}\\
			\bx{z} = -\frac{1}{2}\\
			-2z = 2\\
			3z = \frac{3}{2}
		\end{cases}\\
		\begin{array}{c}
			\iff\\
			L_1 \larr L_1 - L_3\\
			L_4 \larr L_4 + 2L_3\\
			L_5 \larr L_5 - 3L_3
		\end{array} &
		\begin{cases}
			\bx{x} = 1\\
			\bx{y} = -\frac{1}{2}\\
			\bx{z} = -\frac{1}{2}\\
			\begin{array}{|c|}
				\hline
				0 = 1\\
				0 = 3\\
				\hline
			\end{array} 
			\text{ incompatibilité }
		\end{cases}
	\end{align*}
	Il n'y a pas de solution!
\end{exm}

\begin{exm}
	\begin{align*}
		(S_2): &\begin{cases}
			\bx{x} + y - z = 1\\
			y + z = 0\\
			x + 2y = 0
		\end{cases}\\
		\begin{array}{c}
			\iff\\
			L_3 \larr L_3 - L_1
		\end{array}&
		(S_2'): \begin{cases}
			\bx{x} + y - z = 1\\
			\bx{y} + z = 0\\
			y + z = -1
		\end{cases}\\
		\begin{array}{c}
			\iff\\
			L_1 \larr L_1 - L_2\\
			L_3 \larr L_3 - L_2
		\end{array}&
		\begin{cases}
			\bx{x} - 2z = 1\\
			\bx{y} + z = 0\\
			\bx{0 = -1}
		\end{cases}
	\end{align*}
\end{exm}

\begin{exm}
	\begin{align*}
		(S_1) &\iff 
		\underbrace{\begin{pmatrix}
			1 & 1 & 1\\
			1 & -1 & 1\\
			2 & -1 & 1\\
			1 & -1 & -1\\
			0 & 1 & 3
		\end{pmatrix}} _{A}
		\underbrace{\begin{pmatrix}
			x\\y\\z
		\end{pmatrix}}_{X} =
		\underbrace{\begin{pmatrix}
			0\\1\\2\\3\\1
		\end{pmatrix}}_{B}\\
		&\iff AX = B
	\end{align*}

	\begin{align*}
		(S_2) \iff 
		\underbrace{\begin{pmatrix}
			1&1&-1\\
			0&1&1\\
			1&2&0
		\end{pmatrix}}_{A}
		\begin{pmatrix}
			x\\y\\z
		\end{pmatrix}
		=
		\begin{pmatrix}
			1\\0\\0
		\end{pmatrix} 
	\end{align*}

	\begin{align*}
		(S_2') \iff \underbrace{
			\begin{pmatrix}
				1&1&-1\\
				0&1&1\\
				0&1&1\\
			\end{pmatrix} 
		}_{A'}
		\begin{pmatrix}
			x\\y\\z
		\end{pmatrix} =
		\begin{pmatrix}
			1\\0\\-1
		\end{pmatrix} 
	\end{align*}
	\begin{align*}
		\begin{pmatrix}
			1&0&0\\
			0&1&0\\
			-1&0&1
		\end{pmatrix}
		\underbrace{\begin{pmatrix}
			1&1&-1\\
			0&1&1\\
			1&2&0
		\end{pmatrix}}_{A} = 
		\underbrace{\begin{pmatrix}
			1&1&-1\\
			0&1&1\\
			0&1&1
		\end{pmatrix}}_{A'}
	\end{align*}

	\begin{align*}
		\begin{pmatrix}
			1&-1&0\\
			0&1&0\\
			0&-1&1
		\end{pmatrix}
		\begin{pmatrix}
			1&1&-1\\
			0&1&1\\
			0&0&0
		\end{pmatrix}
		=
		\begin{pmatrix}
			1 & 0 & -2\\
			0 & 1 & 1\\
			0 & 0 & 0
		\end{pmatrix} 
	\end{align*}

	\begin{align*}
		\underbrace{
		\begin{pmatrix}
			1&0&0\\
			0&1&0\\
			-1&0&1
		\end{pmatrix}
		}_{\in~ \mathrm{GL}_3(\mathbbm{K})}
		\underbrace{\begin{pmatrix}
			1&-1&0\\
			0&1&0\\
			0&-1&1
		\end{pmatrix}}_{\in~ \mathrm{GL}_3(\mathbbm{K})}
		A = 
		\begin{pmatrix}
			1 & 0 & -2\\
			0 & 1 & 1\\
			0 & 0 & 0
		\end{pmatrix}
	\end{align*}
	\begin{align*}
		A  \in \mathrm{GL}_3(\mathbbm{K})
		\iff
		\begin{pmatrix}
			1&0&-2\\
			0&1&1\\
			0&0&0
		\end{pmatrix} \in \mathrm{GL}_3 (\mathbbm{K})
	\end{align*}
\end{exm}

\begin{exm}
	\[
		A = \begin{pmatrix}
			\bx1&0&1\\
			0&1&1\\
			1&1&0
		\end{pmatrix} 
	\]

	\begin{align*}
		A
		\begin{array}{c}
			\sim\\
			C_3 \larr C_3 - C_1
		\end{array}
		&
		\begin{pmatrix}
			\bx1 & 0 & 0\\
			0&1&1\\
			0&\bx1&-1
		\end{pmatrix}\\
		\begin{array}{c}
			\sim\\
			C_1 \larr C_1 - C_2\\
			C_3 \larr \frac{C_3 - C_2}{2}
		\end{array}&
		\begin{pmatrix}
			\bx 1&0&0\\
			-1&1&\bx 1\\
			0&\bx 1 & 0
		\end{pmatrix}\\
		\begin{array}{c}
			\sim\\
			C_1\larr C_1 + C_3\\
			C_2 \larr C_2 - C_3
		\end{array}&
		\begin{pmatrix}
			\bx 1 & 0&0\\
			0&0&\bx 1\\
			0&\bx 1&0
		\end{pmatrix}\\
		\begin{array}{c}
			\sim\\
			C_2 \leftrightarrow C_3
		\end{array} & I_3
	\end{align*}

	\begin{align*}
		\begin{pmatrix}
			1&0&1\\
			0&1&1\\
			1&1&0
		\end{pmatrix}
		\begin{pmatrix}
			1&0&-1\\
			0&1&0\\
			0&0&1
		\end{pmatrix} = \begin{pmatrix}
			1&0&0\\
			0&1&1\\
			1&1&-1
		\end{pmatrix}
	\end{align*}

	\begin{align*}
		I_3 = A
		\underbrace{
		\begin{pmatrix}
			1&0&-1\\
			0&1&0\\
			0&0&1
		\end{pmatrix}
		\begin{pmatrix}
			1&0&0\\
			-1&1&\frac{1}{2}\\
			0&0&\frac{1}{2}
		\end{pmatrix}
		\begin{pmatrix}
			1&0&0\\
			0&1&0\\
			1&-1&1
		\end{pmatrix} 
		\begin{pmatrix}
			1&0&0\\
			0&0&1\\
			0&1&0
		\end{pmatrix}}_{B}
	\end{align*}
	\begin{align*}
		A \in \mathrm{GL}_3(\mathbbm{K})\text{ et } A^{-1} = I_3 \times B
	\end{align*}

	\begin{align*}
		&\begin{pmatrix}
			1&0&0\\
			0&1&0\\
			0&0&1\\
		\end{pmatrix}\\
		\begin{array}{c}
			\sim\\
			C_3 \larr C_3 - C_1
		\end{array}&
		\begin{pmatrix}
			1&0&-1\\
			0&1&0\\
			0&0&1
		\end{pmatrix}\\
		\begin{array}{c}
			\sim\\
			C_1 \larr C_1 - C_2\\
			C_3 \larr \frac{C_3 + C_2}{2}
		\end{array}&
		\begin{pmatrix}
			1&0&-\frac{1}{2}\\
			-1&1&\frac{1}{2}\\
			0&0&\frac{1}{2}
		\end{pmatrix}\\
		\begin{array}{c}
			\sim\\
			C_1 \larr C_1 + C_3\\
			C_2 \larr C_2 - C_3
		\end{array}&
		\begin{pmatrix}
			\frac{1}{2}&\frac{1}{2}&-\frac{1}{2}\\
			-\frac{1}{2}&\frac{1}{2}&\frac{1}{2}\\
			\frac{1}{2}&-\frac{1}{2}&\frac{1}{2}
		\end{pmatrix}\\
		\begin{array}{c}
			\sim\\
			C_2 \leftrightarrow C_3
		\end{array}&
		\underbrace{
			\begin{pmatrix}
				\frac{1}{2}&-\frac{1}{2}&\frac{1}{2}\\
				-\frac{1}{2}&\frac{1}{2}&\frac{1}{2}\\
				\frac{1}{2}&\frac{1}{2}&-\frac{1}{2}
			\end{pmatrix}
		}_{A^{-1}}
	\end{align*}
\end{exm}

\begin{rmk}
	[Résumé]
	\begin{itemize}
		\item Méthode du pivot: opérations sur les lignes:
			\begin{enumerate}
				\item $L_i \larr L_i + \lambda L_j$ ($\lambda \in \mathbbm{K}$ )
				\item $L_i \larr \mu L_i$ ($\mu \in \mathbbm{K}\setminus \{0\}$ )
				\item $L_i \leftrightarrow L_j$
			\end{enumerate}

			En appliquant la méthode du pivot on obtient
			\[
				(S) \iff \begin{cases}
					x_{i_1} &= \ell_1(x_{j_1},\ldots,x_{j_n-r})\\
					x_{i_2} &= \ell_2(x_{j_1},\ldots,x_{j_n-r})\\
					\vdots & \vdots\\
					x_{i_r} &= \ell_r(x_{j_1},\ldots,x_{j_n-r})\\
					0 &= *
				\end{cases}
			\]

			Les inconnues $x_{i_1}, \ldots, x_{i_r}$ sont les \underline{inconnues principales}, les autres sont appelées \underline{paramètre}.\\
			On peut supprimer les équations $0 = 0$. S'il y a une équation $0 = \lambda$ avec $\lambda \neq 0$, il n'y a pas de solution: le système est \underline{incompatible}.\\
			Les inconnues principales dépendent des choix de pivots!
		\item Représentation matricielle
			\[
				(S) \iff AX = B
			\] où $A$ est la \underline{matrice du système}, $X = \begin{pmatrix} x_1\\ \vdots\\ x_p \end{pmatrix}$ et $B$ est le \underline{second membre}\\
			$(S)$ a $n$ équations et $p$ inconnues donc $A$ a $n$ lignes et $p$ colonnes.\\
			La matrice $(A\mid B)$ est la \underline{matrice augmentée du système}.
		\item Faire un opération $L$ sur les lignes d'une matrice $M$ revient à multiplier $M$ à gauche par une matrice $R$ où $R$ est obtenue en appliquant $L$ sur $I_n$.
		\item La méthode du pivot matriciel par lignes:
			\[
				% A FAIRE
			\]
	\end{itemize}
\end{rmk}

\begin{exm}
	$A = \begin{pmatrix}
		5&1&4&4\\
		2&1&0&0\\
		7&2&3&4\\
	\end{pmatrix}$

	\begin{align*}
		\begin{pmatrix}
			5 & \bx 1 & 3 & 4\\
			2 & 1 & 0 & 0\\
			7 & 2 & 3 & 4
		\end{pmatrix}
		\begin{array}{c}
			\sim \\
			L_2 \leftarrow L_2 - L_1\\
			L_3 \leftarrow L_3 - 2L_1
		\end{array}&
		\begin{pmatrix}
			5&\bx 1 & 3 & 4\\
			-3 & 0 & -3 & -4\\
			-3 & 0 & -3 & -4\\
		\end{pmatrix}\\
		\begin{array}{c}
			\sim\\
			L_2 \leftarrow -\frac{1}{3} L_2
		\end{array}&
		\begin{pmatrix}
			5&\bx 1&3&4\\
			\bx 1&0&1&\frac{4}{3}\\
			-3&0&-3&-4
		\end{pmatrix}\\
		\begin{array}{c}
			\sim\\
			L_1 \leftarrow L_1 - 5L_2\\
			L_3 \leftarrow L_3 + 3L_2
		\end{array} &
		\begin{pmatrix}
			0 & \bx 1 & -2 & -\frac{8}{3}\\
			\bx 1 & 0 & 1 & \frac{4}{3}\\
			0&0&0&0
		\end{pmatrix}\\
		\begin{array}{c}
			\sim\\
			L_1 \leftrightarrow L_2
		\end{array}&
		\underbrace{
			\left(\begin{array}{c c c c c}
				\multicolumn{1}{|c}1 & 0 & 1 & {4}/{3}\\ \cline{1-1}
				0 & \multicolumn{1}{|c}1 & -2 & -{8}/{3}\\ \cline{2-4}
				0&0&0&0\\
			\end{array}\right)
		}_{\text{matrice échelonnée réduite par lignes}}
	\end{align*}
\end{exm}

\begin{defn}
	[Rang d'une matrice]
	Soit $M$ une matrice et $R$ la matrice échelonnée réduite par lignes associée à $M$. Le nombre de lignes non nulles de $R$ (le nombre de pivots) est appelée \underline{rang} de $M$.\\
	Soit $S$ un système de matrice augmentée $(A \mid B)$. Le \underline{rang} de $S$ est le rang de la matrice $A$.\\
	Le rang est noté $\rg$.
\end{defn}

\begin{prop}
	[Interprétation]
	\begin{itemize}
		\item Soit $S$ un système de $n$ équations, $p$ inconnues de rang $r $.\\
			$r$ est le nombre d'inconnues principales, il y a $p - r$ paramètres.
		\item Soit $M \in \mathcal{M}_{n,p}(\mathbbm{K})$ de rang $r$.\\
			$r$ est le nombre de lignes indépendantes: il y a $n - r$ lignes combinaisons linéaires des $r$ lignes indépendantes.
	\end{itemize}
\end{prop}

\begin{crlr}
	Soit $S$ un système de \underline{$n$ équations}, $p$ inconnues de \underline{rang $n$}. Alors $S$ a au moins une solution.\\
	Si $n = p$ alors $S$ a exactement une solution.\\
	Si $p > n$, il y a une infinité de solutions.\\
	\qed
\end{crlr}

\begin{defn}
	Soit $S$ un système à $n$ équations, $n$ inconnues et de rang $n$. On dit que $S$ est un \underline{système de Cramer} (il a une unique solution)
\end{defn}

\begin{prop}
	Soit $S$ un système de $n$ équations, $p$ inconnues de rang $r$.
	\begin{itemize}
		\item Si $r < n$ alors le système peut-être incompatible: il y a $n - r$ équations de la forme $0 = *$ après la méthode du pivot.
		\item Si $r < p$ alors il y a $p - r$ paramètres: si le système n'est pas incompatible, il y aura une infinité de solutions.
	\end{itemize}
	\qed
\end{prop}

\begin{exm}
	 \[
		A = \begin{pmatrix}
			1 & 2 & 3\\
			5 & 1 & 2\\
			1 & 1 & 1
		\end{pmatrix}
	\] 

	\begin{align*}
		\begin{pmatrix}
			1 & 2 & 3\\
			5 & \bx 1 & 2\\
			1 & 1 & 1
		\end{pmatrix}
		\begin{array}{c}
			\sim\\
			L_1 \leftarrow L_1 - 2L_2\\
			L_3 \leftarrow L_3 + L_2
		\end{array}&
		\begin{pmatrix}
			-9&0&-1\\
			5&\bx 1&2\\
			4&0&\bx 1
		\end{pmatrix}\\
		\begin{array}{c}
			\sim\\
			L_1 \leftarrow L_1 + L_3
		\end{array}&
		\begin{pmatrix}
			\bx {-5} & 0 & 0\\
			5&\bx 1 &2\\
			4 & 0 & \bx 1
		\end{pmatrix}\\
		\begin{array}{c}
			\sim\\
			L_1 \leftarrow -L_1/5
		\end{array}&
		\begin{pmatrix}
			\bx 1 & 0 & 0\\
			5 & \bx 1 & 2\\
			4 & 0 & \bx 1
		\end{pmatrix}\\
		\begin{array}{c}
			\sim\\
			L_2 \leftarrow L_2 - 5L_1\\
			L_3 \leftarrow L_3  - 4L_1\\
			L_2 \leftarrow L_2 + 2 L_3
		\end{array}&
		\left(\begin{array}{c c c}
			\multicolumn{1}{|c}1 & 0 & 0\\ \cline{1-1}
			0 & \multicolumn{1}{|c}1 & 0\\ \cline{2-1}
			0 & 0 & \multicolumn{1}{|c}1\\ \cline{3-1}
		\end{array}\right)
	\end{align*}
\end{exm}

\begin{exm}
	\[
		(S): \begin{cases}
			x + y + z + t = 1\\
			x - y + z + 2t = 0\\
			2y - t = 1\\
			2x + 2z + 3t = 1
		\end{cases}
	\] 

	\begin{align*}
		\begin{cases}
			\bx x + y + z + t = 1\\
			x - y + z + 2t = 0\\
			2y - t = 1\\
			2x + 2z + 3t = 1
		\end{cases}
		\begin{array}{c}
			\iff\\
			L_2 \leftarrow L_2 - L_1\\
			L_4 \leftarrow L_4 - 2L_1
		\end{array}&
		\begin{cases}
			\bx x +y +z + t = 1\\
			-2y + \bx t = -1\\
			2y - t = 1\\
			-2y + t = -1
		\end{cases}\\
		\begin{array}{c}
			\iff\\
			L_3 \leftarrow L_3 + L_2\\
			L_4 \leftarrow L_4 - L_2
		\end{array}&
		\fbox{$
			\begin{cases}
				\bx x + y + z + t = 1\\
				-2y + \bx t = -1\\
				0=0\\
				0=0
			\end{cases}
		$} \text{ \underline{Système triangulaire} }\\
		\iff& \begin{cases}
			t = -1+2y\\
			x = 2-3y -z
		\end{cases}
	\end{align*}

	La matrice du système triangulaire est \[
		\begin{pmatrix}
			1 & 1 & 1 & 1\\
			0 & -2 & 0 & 1\\
			0 & 0 & 0 & 0\\
			0 & 0 & 0 & 0\\
		\end{pmatrix} 
	\]
	Elle n'est pas échelonnée réduite par lignes!
\end{exm}

\begin{prop}
	Soit $A \in \mathcal{M}_{n,p}(\mathbbm{K})$ et $C$ une opération élémentaire sur les colonnes de $A$. On pose $A'$ la matrice obtenue en appliquant $C$ sur les colonnes de $A$.
	\[
		\rg(A) = \rg(A')
	\] \qed
\end{prop}

\begin{exm}
	\[
		A = \begin{pmatrix}
			1&2&3\\
			5&1&2\\
			1&1&1\\
		\end{pmatrix}
	\]

	\begin{align*}
		\begin{pmatrix}
			1&2&3\\
			5&1&2\\
			\bx 1&1&1\\
		\end{pmatrix}
		\begin{array}{c}
			\sim\\
			C_2 \leftarrow C_2 - C_1\\
			C_3 \leftarrow C_3 - C_1
		\end{array}&
		\begin{pmatrix}
			1&\bx 1&2\\
			5&-4&-3\\
			\bx 1 & 0 & 0
		\end{pmatrix} \\
		\begin{array}{c}
			\sim\\
			C_3 \leftarrow C_3 - 2C_2
		\end{array}&
		\begin{pmatrix}
			1&\bx 1 & 0\\
			5 & -4 &\bx 5\\
			\bx 1 & 0 & 0
		\end{pmatrix} 
	\end{align*}

	Donc $\rg(A) = 3$
\end{exm}

\begin{exm}
	\[
		\rg \left( \begin{pmatrix}
			1&1&1\\
			1&1&1\\
			1&1&1\\
		\end{pmatrix} \right)  = 1
	\] 
\end{exm}

\begin{prop}
	Le rang d'une matrice est aussi le nombre de colonnes indépendantes.
	\qed
\end{prop}

\begin{defn}
	Une \underline{matrice triangulaire supérieure} est une matrice carrée avec des coefficients nuls sous sa diagonale: \[
		T = 
		\begin{pNiceArray}{>{\strut}cccc}[margin,extra-margin = 1pt]
			* &\Ldots&& * \\
			0 &\Ddots&&\Vdots\\
			\Vdots & \Ddots&&\\
			0 & \Ldots & 0 & * \\
			\CodeAfter
			\begin{tikzpicture}
				\node [draw=red, rounded corners=2pt, inner ysep = 4pt,
						 rotate fit=-42, fit = (1-1) (4-4)] {};
				\draw[red, <-, bend right, thick] (4-4)+(0.15,-0.15) to (2,0);
				\node[draw, red, above] at (2,0) {diagonale};
			\end{tikzpicture}
		\end{pNiceArray}
	\]
	et \underline{triangulaire inférieure} si les coefficients au-dessus de la diagonale sont nuls:
	\[
		T = 
		\begin{pNiceArray}{>{\strut}cccc}[margin,extra-margin = 1pt]
			* &0&\Ldots& 0 \\
			\Vdots&\Ddots&\Ddots&\Vdots\\
			&&&0\\
			* & \Ldots && * \\
		\end{pNiceArray}
	\]

	Un \underline{système triangulaire} est de la forme \[ 
		\left\{\begin{array}
			{c c c c c c c}
			a_{11}x_1 &+ a_{12}x_2 &+ \ldots &+ a_{1p}x_p &=& b_1 &+ \ldots\\
						&+ a_{22}x_2 &+ \ldots &+ a_{2p}x_p &=& b_2 &+ \ldots\\
						&        &  \ddots &        &\vdots&  &  \\
						&        &    & a_{pp}x_p  &=& b_p &+ \ldots\\
						&        &    & 0      &=& \ldots
		\end{array}\right.
	\]
\end{defn}

\begin{rmk}
	\begin{align*}
		(S) &\iff \begin{cases}
			AX = B\\
			X \in \mathcal{M}_{p,1}(\mathbbm{K}), B \in \mathcal{M}_{n,1}(\mathbbm{K}), A \in \mathcal{M}_{n,p}(\mathbbm{K})
		\end{cases}
	\end{align*}

	On pose \begin{align*}
		\varphi: \mathcal{M}_{p,1}(\mathbbm{K}) &\longrightarrow \mathcal{M}_{n,1}(\mathbbm{K}) \\
		X &\longmapsto AX
	\end{align*}

	On cherche $\varphi^{-1} \left( \left\{ B \right\}  \right) $ \\

	On sait que
	\begin{itemize}
		\item $(\mathcal{M}_{p,1}(\mathbbm{K}), +)$ est un groupe
		\item $(\mathcal{M}_{n,1}(\mathbbm{K}), +)$ aussi
	\end{itemize}

	Soient $X, Y \in \mathcal{M}_{p,1}(\mathbbm{K})$ \[
		\varphi(X+Y) = A(X+Y) = AX + AY = \varphi(X) + \varphi(Y)
	\]Donc $\varphi$ est un morphisme de groupes.\\
	On peut résoudre $(S)$ de la façon suivante:
	\begin{itemize}
		\item On cherche $X_0 \in \mathcal{M}_{p,1}(\mathbbm{K})$ tel que $\varphi(X_0) = B$ 
		\item On résout $\varphi(X) = 0$ ($X \in \Ker(\varphi)$)
	\end{itemize}\vspace{5mm}

	C'est le système homogène associé: \[
		\varphi(X) = B \iff \exists H \in \Ker(\varphi), X = X_0+H
	\]
\end{rmk}

\begin{prop}
	\begin{align*}
		A = \left(a_{i,j}\right)_{\substack{1 \le j\le n\\1\le j\le p}} \in \mathcal{M}_{n,p}(\mathbbm{K})\\
		B = \left(b_{k,\ell}\right)_{\substack{1 \le k\le p\\1\le \ell\le q}} \in \mathcal{M}_{p,q}(\mathbbm{K})
	\end{align*}

	\begin{align*}
		AB = \left( c_{i,\ell} \right) _{\substack{1\le i\le n\\1\le \ell\le q}} \in \mathcal{M}_{n,q}(\mathbbm{K})\\
		c_{i,\ell} = \sum_{j = 1}^n = a_{i,j}b_{j,\ell}
	\end{align*}
\end{prop}
