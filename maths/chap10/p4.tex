\part{Arithmétique modulaire}

\begin{defn}
	Soient $a, b \in \Z$ et $c \in \N^*$. On dit que $a$ est \underline{congrus} à $b$ modulo $c$ si $a$ et $b$ ont le même reste dans la division euclidienne par $c$. Dans ce cas, on écrit $a \equiv b \mod c$. \index{congruence (entiers)}
	\index{entiers congrus}
\end{defn}

\begin{prop}
	La congruence modulo $c$ est une relation d'équivalence. \qed
\end{prop}

\begin{rmk}[Notation]
	On note $\sfrac{\Z}{c\Z}$ l'ensemble des classes d'équivalences modulo $c$.
	
	Par exemple, $\sfrac{\Z}{5\Z} = \left\{ \overline{0},\overline{1},\overline{2},\overline{3},\overline{4} \right\}$.
\end{rmk}

\begin{prop}
	Soient $a,b \in \Z$ et $c \in \N^*$. \[
		a \equiv b \mod c \iff c  \mid (b-a)
	.\]
\end{prop}

\begin{prv}
	\begin{itemize}
		\item[``$\implies$''] Soient $q,q' \in \Z$ et $r \in \N$ tels que \[
				\begin{cases}
					a = cq + r\\
					b = cq' + r
				\end{cases}
			\] avec $0 < r < c$.
			En soustrayant les égalités, on obtient \[
				b - a = c \underbrace{(q' - q')}_{\in \Z}
			.\] Ainsi, $c  \mid (b-a)$.
		\item[``$\impliedby$''] On pose $\begin{cases}
				a = cq + r\\
				b = cq' + r'
			\end{cases}$ avec $(q,q') \in \Z$ et $\begin{cases}
				0 \le r < c\\
				0 \le r' < c.
			\end{cases}$ En soustrayant les égalités et inégalités, on obtient \[
				\begin{cases}
					b - a = c (q' - q) + r' - r\\
					-c < r' - r < c.
				\end{cases}
			\]
	\end{itemize}
\end{prv}

