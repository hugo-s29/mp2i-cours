\part{Exercice 4}

\begin{center}
	\begin{asy}
		import graph;
		size(5cm);

		axes(EndArrow);

		real f(real x){return  cos(x) + 1.5 - sin(3x + sqrt(2))/6;}
		real g(real x){return -cos(x) + 1.5 + sin(3x)/6;}

		draw(graph(f,-1,pi), magenta);
		draw(graph(g,-1,pi), deepcyan);

		real a = 1;
		dot("$a$", (a,0), red, align=S);

		dot((a,f(a)), red);
		dot((a,g(a)), red);

		draw((a,0)--(a,max(f(a),g(a))), dashed+red);

		real eta = 0.2;

		draw((a-0.4,f(a)-eta)--(a+0.4,f(a)-eta));
		draw((a-0.4,f(a)+eta)--(a+0.4,f(a)+eta));

		draw((a-0.4,g(a)-eta)--(a+0.4,g(a)-eta));
		draw((a-0.4,g(a)+eta)--(a+0.4,g(a)+eta));
	\end{asy}
\end{center}

On suppose, sans perte de généralité, $f(a) < g(a)$.\\
On pose $\varepsilon=\frac{g(a) - f(a)}{3}>0$.\\
$g(x) \tendsto{x\to a} g(a)$ donc il existe $\eta_1>0$ tel que \[
	\forall x \in ]a-\eta_1, a+\eta_1[\cap I, \left| g(x) - g(a)\right| \le \varepsilon
\] De même, il existe $\eta_2 > 0$ tel que \[
	\forall x \in ]a-\eta_2, a+\eta_2[\cap I, \left| f(x)-f(a) \right| \le \varepsilon
\] On suppose $a \in \mathring A$
(sinon $J = \{a\}$ conviendrait\ldots)\\
donc il existe $\eta_3 > 0$ tel que $]a-\eta_3,a+\eta_3[ \subset I$.\\
On pose $J = ]a-\eta_1,a+\eta_1[\cap ]a-\eta_2,a+\eta_2[\cap ]a-\eta_3,a+\eta_3[$.\\
$J$ est un intervalle ouvert non vide (car $a \in J$) inclus dans I.
\begin{align*}
	\forall x \in J,
	f(x) \le f(a) + \varepsilon &= \frac{2f(a) + g(a)}{3}\\
															&<\frac{f(a) +2g(a)}{3}\\
															&\le g(a) -\varepsilon\\
															&\le g(x)
\end{align*}

Donc, 
\[
	\forall x \in J, f(x) \neq g(x)
\] 
