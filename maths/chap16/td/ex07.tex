\part{Exercice 7}

\marginpar{\underline{Idée de la solution}}

\[
	f(2x) = f(x) + ax + \po(x)
\]
\begin{align*}
	f(x) \xleftarrow[n \to +\infty]{} f(x)
	&= f\left( \frac{x}{2} \right) a\frac{x}{2} + \po(x)\\
	&= f\left( \frac{x}{4} \right) + a \frac{x}{2} + a \frac{x}{4} + \po(x) \\
	&= f\left( \frac{x}{2^n} \right) + a\sum_{k = 1}^n \frac{x}{2^k} + \po(x) \\
	&\tendsto{n\to +\infty} f(0) + ax + \po(x)
\end{align*}

Donc, $f(x) = f(0) + ax + \po(x)$

\marginpar{\underline{Solution}}

$\varepsilon(x) \to 0$
\[
	f(2x) = f(x) + ax + x\varepsilon(x)
\]
\begin{align*}
	f(x) \xleftarrow[n \to +\infty]{} f(x)
	&= f\left( \frac{x}{2} \right) a\frac{x}{2} + \frac{x}{2}\varepsilon\left( \frac{x}{2} \right) \\
	&= f\left( \frac{x}{4} \right) + a \frac{x}{2} + a \frac{x}{4} + \frac{x}{4}\varepsilon\left( \frac{x}{4} \right) + \frac{x}{2} \varepsilon\left( \frac{x}{2} \right) \\
	&= f\left( \frac{x}{2^n} \right) + a\sum_{k = 1}^n \frac{x}{2^k} + \underbrace{x\sum_{k = 1}^n \frac{1}{2^k} \varepsilon\left( \frac{x}{2^k} \right)}_{\tendsto{n\to +\infty} ?}\\
	&\tendsto{n\to +\infty} f(0) + ax + ?
\end{align*}

On pose \[
	\forall x >0,  \varepsilon(x) = \frac{f(2x) - f(x)}{x}-a
\] D'après l'énoncé, $\varepsilon(x) \tendsto{\substack{x \to 0\\>}}$ et \[
	(\mathcal{E}):\quad \forall x > 0, f(2x) = f(x) + ax + x\varepsilon(x)
\] On en déduit par récurrence que
\begin{align*}
	\forall n \in \N_*, P(n) \text{ est vraie, avec }\\
	P(n): ``\forall x > 0, f(x) = f\left( \frac{x}{2^n} \right) + ax \sum_{k=1}^n \frac{1}{2^k} + x\sum_{k=1}^n \frac{1}{2^k} \varepsilon\left( \frac{x}{2^k} \right)"
\end{align*}

\begin{itemize}
	\item D'après $(\mathcal{E})$, \[
			\forall x>0, f(x) = f\left( \frac{x}{2} \right)  + \frac{ax}{2} + \frac{x}{2}\, \varepsilon\left( \frac{x}{2} \right)
		\] donc $P(1)$ est vraie
	\item Soit $n\in \N_*$. On suppose $P(n)$ vraie.
		\begin{align*}
			\forall x >0,
			f(x) &= f\left( \frac{x}{2^n} \right) + ax\sum_{k=1}^n \frac{1}{2^k} +x\sum_{k=1}^n \frac{1}{2^k} \varepsilon\left( \frac{x}{2^k} \right)\\
			&= f\left( \frac{x}{2^{n+1}} \right) + a \frac{x}{2^{n+1}} + \frac{x}{2^{n+1}}\, \varepsilon\left( \frac{x}{2^{n+1}} \right) \\
			&+ ax \sum_{k=1}^n \frac{1}{2^k} + x\sum_{k=1}^n \frac{1}{2^k} \varepsilon\left( \frac{x}{2^k} \right) \\
			&= f\left( \frac{x}{2^{n+1}} \right) + ax\sum_{k=1}^{n+1}\frac{1}{2^k} + x\sum_{k=1}^{n+1} \frac{1}{2^{k}} \varepsilon\left( \frac{x}{2^k} \right) \\
		\end{align*}
		Donc, $P(n+1)$ est vraie
\end{itemize}

On fixe $x > 0$. \[
	f\left( \frac{x}{2^n} \right) \tendsto{n\to +\infty} f(0) \text{ car $f$ est continue en $0$}
\] 
\begin{align*}
	\forall n \in \N_*, \sum_{k=1}^n \frac{1}{2^k} &= \frac{1}{2} \times  \frac{1- \left( \frac{1}{2} \right)^n}{1 - \frac{1}{2}}
	&= 1 - \frac{1}{2^n} \tendsto{n \to +\infty} 1 \\
\end{align*}

Comme $\lim_{\substack{x\to 0\\>}} \varepsilon(x)  0$, on sait que \[
	\forall \varepsilon >0, \exists \eta > 0, \forall x \in ]0,\eta[, \left| \varepsilon(x) \right| \le \varepsilon
\]

Soit $\varepsilon>0$. On considère $\eta>0$ tel que \[
	\forall x \in ]0,\eta[, \left| \varepsilon(x) \right| \le \varepsilon
\]
Soit $x \in ]0,\eta[$. \[
	\forall k \in \N, \frac{x}{2^k} \le x
\] donc \[
	\forall k \in \N, \frac{x}{2^k} \in ]0,\eta[
\] donc \[
	\forall x \in \N, \left| \varepsilon\left( \frac{x}{2} \right)  \right| \le \varepsilon
\]  Donc 
\begin{align*}
	\forall n \in \N_*, \left| \sum_{k=1}^n \frac{1}{2^k}\varepsilon\left( \frac{x}{2^k} \right) \right|
	&\le \sum_{k=1}^n \frac{1}{2^k} \left| \varepsilon\left( \frac{x}{2^k} \right) \right|\\
	&\le  \sum_{k=1}^n \frac{1}{2^k} \varepsilon\\
	&\le \left( 1 - \frac{1}{2^n} \right) \varepsilon\\
	&\le \varepsilon
\end{align*}

De plus, pour tout $x >0$, il existe $N'_x \in \N_*$ tel que \[
	\forall n \in \N_*', \left| f\left( \frac{x}{2^n} \right) - f(0) \right| \le \varepsilon x
\] et il existe $N_x \in \N_*$ tel que \[
	\forall n \ge N_x, \left| a\sum_{k=1}^n \frac{1}{2^k} - a \right| \le \varepsilon
\] donc \[
	\forall n \ge N, ax - \varepsilon x \le ax \sum_{k=1}^n \frac{1}{2^k} \le ax + \varepsilon x
\]
\begin{align*}
	\forall x \in ]0,\eta''[, \forall n \ge \max(N, N'_x), \\
	f(0) - \varepsilon x + ax - \varepsilon x - \varepsilon x \le f(x) \le f(0) + \varepsilon x + ax + \varepsilon x + \varepsilon x
\end{align*}
donc \[
	-3\varepsilon \le \frac{f(x) - f(0)}{x} - a \le 3\varepsilon
\]
Donc $\lim_{\substack{x\to 0 \\ >}} \frac{f(x) - f(0)}{x}-a = 0$
et donc $f$ est dérivable à droite en 0 et $f'(0) = a$
