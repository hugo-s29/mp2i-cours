\part{Sous-espaces affines}

\begin{defn}
	Soit $\left( E, \vec{E}, \tau \right)$ un $\mathbbm{K}$-espace affine et $F \in \mathcal{P}(E) \setminus \{\O\}$.

	Pour tout $A \in F$, on pose $\vec{F_A} = \left\{\vec{AB}  \mid  B \in F\right\}$. On dit que $F$ est un \underline{sous-espace affine}\index{sous-espace affine} de $\left( E, \vec{E}, \tau \right)$ s'il existe $A \in F$ tel que $\vec{F_A}$ est un sous-espace vectoriel de $\vec{E}$.
\end{defn}

\begin{prop}
	Avec les notations précédentes, $\left( F, \vec{F_A}, \tau_{\left|F \times \vec{F_A}\right.} \right)$ est un espace affine.
\end{prop}

\begin{prv}
	\begin{itemize}
		\item Soit $M \in F$, $\vec{u} \in \vec{F_A}$. \[
				M + \vec{u} = A + \underbrace{\underbrace{\vec{AM}}_{\displaystyle\in \vec{F_A}} \underbrace{\vec{u}}_{\displaystyle\in \vec{F_A}}}_{\displaystyle \in \vec{F_A}} = A + \underbrace{\vec{AB}}_{B \in F} = B \in F.
			\]
		\item Soit $M \in F$, $\vec{u}, \vec{v} \in \vec{F_A}$. \[
					\left(M + \vec{u}\right) + \vec{v} = M + \left(\vec{u} + \vec{v}\right)
			\] car $M \in E$ et $\vec{u}, \vec{v} \in \vec{E}$.
		\item Soient $M,N \in F \subset E$. On sait que \[
				\exists ! \vec{u} \in \vec{E},\, M + \vec{u} = N.
			\]
			\begin{align*}
				\vec{u} = \vec{MN} &= \vec{MA} + \vec{AN}\\
				&= \underbrace{\vec{AN}}_{\displaystyle\in \vec{F_A}} - \underbrace{\vec{AM}}_{\displaystyle\in  \vec{F_A}} \in \vec{F_A} \\
			\end{align*}
	\end{itemize}
\end{prv}

\begin{prop}
	Soit $F$ un sous-espace affine de $\left( E, \vec{E}, \tau \right)$. Alors \[
		\forall (A,B) \in F^2,\; \vec{F_A} = \vec{F_B}.
	\]
\end{prop}

\begin{prv}
	Soit $A$ comme dans la définition $\left( \vec{F_A} \text{ sous-espace vectoriel de } \vec{E} \right)$ et $B \in F$. Soit $\vec{u} \in \vec{F_B}$. Alors $\vec{u} = \vec{BM}$ avec $M \in F$. \[
		\vec{u} = \vec{BA} + \vec{AM} = \vec{AM} - \vec{AB} \in \vec{F_A}.
	\] Soit $\vec{v} \in \vec{F_A}$. On pose $M = \underbrace{M} + \underbrace{\vec{v}}_{\in \vec{F_A}} \in F$. Donc $\vec{v} = \vec{BM} \in \vec{F_B}$.
\end{prv}

\begin{crlr}
	Soit $f \in \mathcal{L}(E, F)$ et $y \in F$.

	Les solutions de l'équation $f(x) = y$ est un sous-espace affine de direction $\Ker f$.
	\qed
\end{crlr}

\begin{prop}
	Soit $(F_i)_{i\in I}$ une famille de sous-espaces affines de $F$. Alors, $\bigcap_{i \in  I} F_i$ est soit vide, soit un sous-espace affine de $F$.
\end{prop}

De même que pour les groupes et les espaces vectoriels, on peut définir le sous-espace engendré par une partie de $E$.

\begin{prop-defn}
	Soit $A \in \mathcal{P}(E)$. Le \underline{sous-espace affine engendré par $A$}\index{sous-espace affine engendré} est \[
		\bigcap_{\substack{F \text{ sous-espace affine de } E\\A \subset F}} F
	.\]
	C'est le plus petit (au sens de l'inclusion) sous-espace affine contenant $A$.
\end{prop-defn}

\begin{rmk}
	Si $E$ est un $\mathbbm{K}$-espace vectoriel et $F$ un sous-espace affine de $E$, alors \[
		\begin{cases}
			\forall A \in F,\, F = A + \vec{F} = \{ A + \vec{u}  \mid  \vec{u} \in \vec{F}\},\\
			\vec{F} \text{ sous-espace vectoriel de } E.
		\end{cases}
	\]
\end{rmk}

