\part{Théorie naïve des ensembles}

\begin{defn}
	Un \underline{ensemble} est une collection finie ou infinie d'objets de même nature ou non. L'ordre de ces objets n'a pas d'importance.
	\index{ensemble}
\end{defn}

\begin{exm}
	\begin{enumerate}
		\item $\left\{ 1, x\mapsto x^2, \{1\} \right\}$ est un ensemble: ses éléments dont l'entier 1, la fonction $x\mapsto x^2$ et un ensemble contenant uniquement 1 (un \underline{singleton}).
		\item $\N$ est un ensemble infini
	\end{enumerate}
\end{exm}

\begin{rmk}
	[Notation]
	Soit $E$ un ensemble et $x$ un objet de $E$.\\
	On écrit $x \in E$ ou bien $x \ni E$.
\end{rmk}

\begin{rmk}
	[\danger Paradoxe]
	On note $\Omega$ l'ensemble de tous les ensembles. Alors, $\Omega \in \Omega$.\\
	Ce n'est pas le cas de tous les ensembles: \[
		\N \not\in \N \text{ car $\N$ n'est pas un entier}
	\] On distingue donc 2 types d'ensembles:
	\begin{itemize}
		\item ceux qui vérifient $E \not\in E$, on dit qu'ils sont \underline{ordinaires}
		\item ceux qui vérifient $E \in E$, on dit qu'ils sont \underline{extra-ordinaires}
	\end{itemize}
	On note $O$ l'ensemble de tous les ensembles ordinaires.
	\begin{itemize}
		\item Supposons $O$ ordinaire. Alors, $O\not\in O$ \\
			Or, $O$ est ordinaire et donc $O \in O$ $\lightning$ 
		\item Supposons $O$ extra-ordinaire.\\
			Alors $O \in O$ et donc $O$ ordinaire $\lightning$
	\end{itemize}

	\color{red}\fbox{C'est un paradoxe}
\end{rmk}

Pour éviter ce type de paradoxe, on a donné une définition axiomatique qui explique quelles sont les opérations permettant de combiner des ensembles pour en faire un autre.
\vspace{1cm}

\begin{defn}
	Soit $E$ un ensemble et $F$ un autre ensemble. On dit que $E$ et $F$ sont \underline{égaux} (noté $E = F$) si $E$ et $F$ contiennent les mêmes objets.
	\index{égalité (ensemble)}
\end{defn}

\begin{exm}
	\begin{enumerate}
		\item $E = \{1,2,3\}$ et $F = \{3,2,1,2\}$ \\
			On a bien $E = F$.
		\item $\N \neq \Z$ car $\begin{cases}
				-1 \in \Z\\
				-1 \not\in \N
			\end{cases}$ 
		\item $E = \left\{ 0, \{0\}  \right\} \neq \{0\} = F$\\
			car $\begin{cases}
				\{0\} \in E\\
				\{0\} \not\in F
			\end{cases}$ \\
			mais, $F \in E$
	\end{enumerate}
\end{exm}

\begin{defn}
	L'ensemble \underline{vide}, noté $\O$ est le seul ensemble à n'avoir aucun élément.
	\index{ensemble vide}
	\index{ensemble vide@$\O$}
\end{defn}

\begin{defn}
	Soient $E$ et $F$ deux ensembles. On dit que $F$ est \underline{inclus} dans $E$, noté $F\subset E$ ou $E \supset F$ si tous les éléments de $F$ sont aussi des éléments de $E$.\\
	\begin{align*}
		\forall x\in F, x \in E
	\end{align*}
	\index{inclusion (ensemble)}
\end{defn}

\begin{figure}[H]
	\centering
	\incfig{inclusion-ensemble}
\end{figure}

\begin{prop}
	Pour tout ensemble $E$, $\O \subset E$
\end{prop}

\begin{prv}
	[par l'absurde]
	Si $\O \not\subset E$ alors $\exists x \in \O, x \not\in E$: une contradiction $\lightning$ \\
\end{prv}

\begin{exm}
	\begin{enumerate}
		\item $E = \{1,2,3\}$ et $F = \{1,3\}$ \\
			On a $F \subset E$ mais pas $E \subset F$ car $\begin{cases}
				2 \in E\\
				2\not\in F
			\end{cases}$ 
		\item $F = \{0\}$ et $E = \{0, \{0\} \}$ \\
			\begin{itemize}
				\item  $F \in E$ car $\{0\} \in E$ 
				\item $F \subset  E$ car $0 \in E$
			\end{itemize}
		\item $E = \left\{ \{0\}  \right\}; F = \{0\}$ 
			\begin{itemize}
				\item $F \not\subset E$ car $0 \not\in E$
				\item $F \in E$
			\end{itemize}
		\item $E = \{\{\{0\} \} \}; F = \{0\}$ 
			\begin{itemize}
				\item $F\not\in E$
				\item $F\not\subset E$ 
				\item $\O \subset F$ 
				\item $\O\subset E$
			\end{itemize}
	\end{enumerate}
\end{exm}

\begin{defn}
	Soit $E$ un ensemble. On peut former \underline{l'ensemble de toutes les parties de $E$}
	(une \underline{partie} de $E$ est un ensemble $F$ avec $F\subset E$). On le note $\mathcal{P}(E)$\\
	\[
		A \in \mathcal{P}(E) \iff A \subset E
	\]

	\index{partie d'un ensemble}
	\index{ensemble de toutes les parties d'un ensemble}
\end{defn}

\begin{exm}
	\begin{enumerate}
		\item $E = \{42\}$ \\
			Les sous-ensembles de $E$ sont $\O$ et $\{42\} = E$ donc \[
				\mathcal{P}(E) = \{\O, \{42\} \} 
			\] 
		\item $\mathcal{P}(\O) = \{\O\}$ 
		\item $E = \{0,1\}$ donc $\mathcal{P}(E) = \{\O, \{0\}, \{1\}, \{0, 1\} \}$ 
		\item $E = \{\O, \{\O\}\}$ donc $\mathcal{P}(E) = \{\O, \{\O\}, \{\{\O\}\}, \{\O, \{\O\} \} \} $ 
		\item $E = \{\O, \{\O\}\}$ \\
			\begin{align*}
				\mathcal{P}(\mathcal{P}(E)) &= \mathcal{P}(\{\O, \{\O\}, \{\{\O\}\}, E)\\
				&= \{\O, \{\O\}, \{\{\O\}\}, \{\{\{\O\}\}\}, \{E\}, \{\O, \{\O\}\}, \{\O, \{\{\O\}\}\}, \{\O, E\}, \{\{\O\},\\
				&\quad\{\{\O\}\}\}, \{\{\O\}, E\}, \{\{\{\O\}\}, E\}, \{\O, \{\O\}, \{\{\O\}\}\}, \{\O, \{\O\}, E\},\\
				&\quad\{\O, \{\{\O\}\}, E\}, \{\{\O\}, \{\{\O\}\}, E\}, \{\O, \{\O\}, \{\{\O\}\}, E\}\}\
			\end{align*}
	\end{enumerate}
\end{exm}

\begin{defn}
	Soit $E$ un ensemble et $A, B \in \mathcal{P}(E)$
	\begin{enumerate}
		\item ~\\
			\begin{minipage}
				{\linewidth}
				\begin{wrapfigure}{r}{3cm}
					\centering
					\vspace{-7mm}
					\begin{asy}
						import patterns;
						add("hatch",hatch(1mm, cmyk(0, 0, 1, 0.5)));
						size(3cm);

						guide main_set = (-1,1)..(-0.8,-0.8)..(0,-0.9)..(0.7,-1.2)..(0.8, 0.9)..cycle;
						guide set_a = (-0.6, 0.6)..(0.2,-0.2)..(0.2,-0.4)..(-0.6,-0.2)..cycle;
						guide set_b = (0.8, -0.6)..(1.1,-0.2)..(0.2,0.5)..(0.2,-0.8)..cycle;

						draw(main_set, magenta); label("$E$", (0.8,0.9),magenta, align=NE);
						draw(set_a, deepcyan); label("$A$", (-0.6,0.6), deepcyan, align=NW);
						draw(set_b, heavygreen); label("$B$", (0.8,-0.6), heavygreen, align=SE);

						fill(set_a, pattern("hatch"));
						fill(set_b, pattern("hatch"));
					\end{asy}
				\end{wrapfigure}
				La \underline{réunion} de $A$ et $B$ est \[
					A \cup B = \{x \in E  \mid x \in A \ou x \in B\}
				\]
				\index{réunion (ensemble)}
			\end{minipage}
			\vspace{2cm}
		\item ~\\
			\begin{minipage}
				{\linewidth}
				\begin{wrapfigure}{r}{3cm}
					\centering
					\vspace{-7mm}
					\begin{asy}
						import patterns;
						add("hatch",hatch(1mm, cmyk(0, 0, 1, 0.5)));
						size(3cm);

						guide main_set = (-1,1)..(-0.8,-0.8)..(0,-0.9)..(0.7,-1.2)..(0.8, 0.9)..cycle;
						guide set_a = (-0.6, 0.6)..(0.2,-0.2)..(0.2,-0.4)..(-0.6,-0.2)..cycle;
						guide set_b = (0.8, -0.6)..(1.1,-0.2)..(0.2,0.5)..(0.2,-0.8)..cycle;

						draw(main_set, magenta); label("$E$", (0.8,0.9),magenta, align=NE);
						draw(set_a, deepcyan); label("$A$", (-0.6,0.6), deepcyan, align=NW);
						draw(set_b, heavygreen); label("$B$", (0.8,-0.6), heavygreen, align=SE);

						picture p;
						fill(p, set_a, pattern("hatch"));
						clip(p, set_b);

						add(p);
					\end{asy}
				\end{wrapfigure}
				L'\underline{intersection} de $A$ et $B$ est \[
					A \cap B = \{x \in E  \mid x \in A \et x \in B\}
				\]
				\index{intersection (ensemble)}
			\end{minipage}
			\vspace{2cm}
		\item ~\\
			\begin{minipage}
				{\linewidth}
				\begin{wrapfigure}{r}{3cm}
					\centering
					\vspace{-7mm}
					\begin{asy}
						import patterns;
						add("hatch",hatch(1mm, cmyk(0, 0, 1, 0.5)));
						size(3cm);

						guide main_set = (-1,1)..(-0.8,-0.8)..(0,-0.9)..(0.7,-1.2)..(0.8, 0.9)..cycle;
						guide set_a = (-0.6, 0.6)..(0.2,-0.2)..(0.2,-0.4)..(-0.6,-0.2)..cycle;
						guide set_b = (0.8, -0.6)..(1.1,-0.2)..(0.2,0.5)..(0.2,-0.8)..cycle;

						draw(main_set, magenta); label("$E$", (0.8,0.9),magenta, align=NE);
						draw(set_a, deepcyan); label("$A$", (-0.6,0.6), deepcyan, align=NW);

						picture p;
						fill(p, main_set, pattern("hatch"));
						unfill(p, set_a);

						add(p);
					\end{asy}
				\end{wrapfigure}
				Le \underline{complémentaire} de $A$ dans $E$ est \[
					E \setminus A = \{x \in E \mid x \not\in A\} = C_E A
				\]
				\index{complémentaire (ensemble)}
			\end{minipage}
			\vspace{2cm}
		\item ~\\
			\begin{minipage}
				{\linewidth}
				\begin{wrapfigure}{r}{3cm}
					\centering
					\vspace{-7mm}
					\begin{asy}
						import patterns;
						add("hatch",hatch(1mm, cmyk(0, 0, 1, 0.5)));
						size(3cm);

						guide main_set = (-1,1)..(-0.8,-0.8)..(0,-0.9)..(0.7,-1.2)..(0.8, 0.9)..cycle;
						guide set_a = (-0.6, 0.6)..(0.2,-0.2)..(0.2,-0.4)..(-0.6,-0.2)..cycle;
						guide set_b = (0.8, -0.6)..(1.1,-0.2)..(0.2,0.5)..(0.2,-0.8)..cycle;

						draw(main_set, magenta); label("$E$", (0.8,0.9),magenta, align=NE);
						draw(set_a, deepcyan); label("$A$", (-0.6,0.6), deepcyan, align=NW);
						draw(set_b, heavygreen); label("$B$", (0.8,-0.6), heavygreen, align=SE);

						picture p2;
						fill(p2, set_a, pattern("hatch"));
						unfill(p2, set_b);

						picture p;
						fill(p, set_b, pattern("hatch"));
						unfill(p, set_a);

						add(p);
						add(p2);
					\end{asy}
				\end{wrapfigure}
				La \underline{différence symétrique} de $A$ et $B$ est
				\begin{align*}
					A \Delta B &= \left\{ x \in E  \mid (x \in A\et x \not\in B)\ou(x\not\in A\et x \in B) \right\}\\
					&= (A\cup B)\setminus(A\cap B) \\
				\end{align*}
				\index{différence symétrique (ensemble)}
			\end{minipage}
	\end{enumerate}
\end{defn}

\begin{prop}
	Soit $E$ un ensemble et $A,B,C \in \mathcal{P}(E)$
	
	\begin{multicols}{2}
		\begin{enumerate}
			\item $A\cap A = A$
			\item $B \cap A = A\cap B$ 
			\item $A \cap (B\cap C) = (A\cap B)\cap C$ 
			\item $A \cap \O = \O$ 
			\item $A\cap E = A$ 
			\item $A \cup A = A$ 
			\item $B \cup A = A\cup B$ 
			\item $A \cup (B\cup C) = (A\cup B) \cup C$
			\item $A \cup \O = A$ 
			\item $A\cup E = E$ 
			\item $(E\setminus A)\setminus A = E\setminus A$
			\item $E \setminus(E\setminus A) = A$
			\item $E\setminus\O = E$ 
			\item $E\setminus E = \O$ 
			\item $A \cup (B \cap C) = (A\cup B) \cap (A \cup C)$
			\item $A \cap (B \cup C) = (A\cap B) \cup (A \cap C)$
			\item $E \setminus (A\cup B) = (E\setminus A) \cap (E\setminus B)$
			\item $E \setminus (A \cap B) = (E\setminus A) \cup  (E\setminus B)$
		\end{enumerate}
	\end{multicols}
\end{prop}

\begin{prv}
	\begin{enumerate}
		\item[16.] $A \cap (B \cup C)= (A\cap B)\cup (A\cap C)$ 
			\begin{itemize}
				\item Soit $x \in A \cap (B \cup C)$ donc $x \in A$ et $x \in B \cup C$ \\
					\begin{itemize}
						\item[\underline{\sc Cas 1}] $x \in B$, alors $x \in A \cap B$ et donc $x \in (A\cap B)\cup (A\cap C)$ 
						\item[\underline{\sc Cas 2}] $x \in C$, alors $x \in A \cap C$ et donc $x \in (A\cap B)\cup (A\cap C)$
					\end{itemize}
					On a prouvé \[
						A \cap (B \cup C) \subset (A \cap B) \cup (A \cap C)
					\]
				\item Soit $x \in (A \cap B) \cup (A \cap C)$ 
					\begin{itemize}
						\item[\underline{\sc Cas 1}]$x \in A\cap B$ donc $x \in A$ et $x \in B$ donc $x \in B \cup C$ et donc $x \in A \cap (B\cup C)$
						\item[\underline{\sc Cas 2}]$x \in A\cap C$ donc $x \in A$ et $x \in C$ donc $x \in B \cup C$ et donc $x \in A \cap (B\cup C)$
					\end{itemize}
					On a prouvé \[
						A \cap (B \cup C) \supset (A \cap B) \cup (A \cap C)
					\]
			\end{itemize}
		\item[17.] $E \setminus (A\cup B) = (E\setminus A) \cap (E\setminus B)$ 
			\begin{itemize}
				\item Montrons que $x \in E\setminus(A\cup B) \implies x \in (E\setminus A)\cap (E\setminus B)$ \\
					Soit $x \in  E\setminus(A\cup B)$ donc $x \not\in A \cup B$
					\begin{itemize}
						\item Si $x \in A$, alors $x \in A \cup B$ $\lightning$\\
							donc $x \not\in A$ i.e. $x \in E\setminus A$ 
						\item Si $x \in B$ alors, $x \in A\cup B$ $\lightning$ \\
							Donc $x \not\in B$ i.e. $x \in E\setminus B$
					\end{itemize}
					On en déduit que $x \in (E\setminus A)\cap (E\setminus B)$
				\item $x \in (E\setminus A) \cap (E\setminus B)$. Montrons que $x \in  E \setminus(A\cup B)$ \\
					On suppose que $x \not\in E \setminus (A\cup B)$ donc $x \in A\cup B$ 
					\begin{itemize}
						\item Si $x \in A$, on a une contradiction car $x \in E\setminus A$
						\item Si $x \in B$, on a une contradiction car $x \in E\setminus B$
					\end{itemize}
					donc $x \in E \setminus(A\cup B)$
			\end{itemize}
	\end{enumerate}
\end{prv}
