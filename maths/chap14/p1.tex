\part{Définition d'une limite de fonctions}

\begin{defn}
	Soit $a \in \overline{\R} = \R \cup \{ +\infty, -\infty\}$ et $V \in \mathcal{P}(\R)$.

	\index{voisinage}

	\begin{enumerate}
		\item Si $a \in \R$, on dit que $V$ est un \underline{voisinage} si \[
				\exists \eta > 0,\,]a-\eta, a+\eta[ \subset V
			.\]
			\begin{figure}[H]
				\centering
				\begin{asy}
					size(5cm);

					draw((-5, 0) -- (5, 0), Arrow(TeXHead));
					
					real vmin = -2;
					real vmax = 3;
					real epsx = 0.2;
					real epsy = 0.2;
					real eta = 1;

					draw((vmin, 0) -- (vmax, 0), magenta);
					draw((-eta, 0) -- (eta, 0), magenta + 1);
					draw((-eta, 0) -- (eta, 0), deepcyan);
					draw((-eta - epsx, epsy) -- (-eta, epsy) -- (-eta, -epsy) -- (-eta - epsx, -epsy), orange);
					draw((eta + epsx, epsy) -- (eta, epsy) -- (eta, -epsy) -- (eta + epsx, -epsy), orange);

					draw((vmin,epsy) -- (vmin, -epsy), magenta);
					draw((vmax,epsy) -- (vmax, -epsy), magenta);

					dot("$a$", (0, 0), deepcyan, align=1.3S);
					label("$\R$", (5, 0), align=SE);
					label("$V$", (vmax, 0), magenta, align=SW);

					draw((0, 2epsy) -- (eta, 2epsy), orange, Arrows(TeXHead));
					label("$\eta$", (eta / 2, 2epsy), orange, align=N);
				\end{asy}
			\end{figure}
		\item Si $a = +\infty$, on dit que $V$ est un \underline{voisinage} de $a$ si \[
				\exists M \in \R,\,]M, +\infty[\;\subset V
			.\]
		\item Si $a = -\infty$, on dit que $V$ est un \underline{voisinage} de $a$ si \[
				\exists m \in \R,\,]\!-\!\infty, m[\;\subset V
			.\]
	\end{enumerate}

	On note $\mathcal{V}_a$ l'ensemble des voisinages de $a$.
\end{defn}

L'utilisation des voisinages permet d'exprimer une limite finie ou infinie plus simplement, sans disjonction de cas.

\begin{exm}
	Soit $u \in \R^\N$ de limite $a \in \overline{\R}$. 

	\begin{itemize}
		\item Si $a \in \R$,
			\begin{align*}
				&\forall \varepsilon > 0,\,\exists N \in \N,\,\forall n \ge N,\,|u_n - a| \le \varepsilon\\
				\text{i.e. }&\forall \varepsilon > 0,\,\exists N \in \N,\,\forall n \ge N,\,a - \varepsilon \le u_n \le a + \varepsilon\\
				\text{i.e. }&\forall V \in \mathcal{V}_a,\,\exists N \in \N,\,\forall n \ge N,\,u_n \in V.
			\end{align*}
		\item Si $a = +\infty$,
			\begin{align*}
				&\forall M \in \R,\,\exists N \in \N,\,\forall n \ge N,\,u_n \ge M\\
				\text{i.e. }&\forall M \in \R,\,\exists N \in \N,\,\forall n \ge N,\, u_n \in [M, +\infty[\\
				\text{i.e. }&\forall V \in \mathcal{V}_a,\,\exists N \in \N,\,\forall n \ge N,\,u_n \in V.
			\end{align*}
		\item Si $a = -\infty$,
			\begin{align*}
				&\forall m \in \R,\,\exists N \in \N,\,\forall n \ge N,\,u_n \le m\\
				\text{i.e. }&\forall m \in \R,\,\exists N \in \N,\,\forall n \ge N,\, u_n \in \,]-\infty,m]\\
				\text{i.e. }&\forall V \in \mathcal{V}_a,\,\exists N \in \N,\,\forall n \ge N,\,u_n \in V.
			\end{align*}
	\end{itemize}

	Même si, dans chacun des cas, la définition de limite de la suite $u$ est différente, en utilisant les voisinages, la notation est identique : \[
		\forall V \in \mathcal{V}_a,\,\exists N \in \N,\,\forall n \ge N,\,u_n \in V
	.\]
\end{exm}

En utilisant cette nouvelle notation, on peut redéfinir la limite plus simplement.

\begin{defn}
	Soit $f$ une fonction définie sur $D \subset \R$ à valeurs réelles.
	Soit $a \in \overline{D} = \{x \in \overline{\R} \mid \forall V \in \mathcal{V}_x,\,V \cap D \neq \O \}$ (on ``rajoute'' chacune des bornes exclues des intervalles composant $D$) : 

	\begin{figure}[H]
		\centering
			\begin{asy}
				size(5cm);

				draw((-5, 0) -- (5, 0), Arrow(TeXHead));
				
				real epsx = 0.2;
				real epsy = 0.2;

				void drawInterval(real a, real b, pen p, real oa, real ob) {
					draw((a, 0) -- (b, 0), p + 1);
					draw((a + oa * epsx, epsy) -- (a, epsy) -- (a, -epsy) -- (a + oa * epsx, -epsy), p);
					draw((b - ob * epsx, epsy) -- (b, epsy) -- (b, -epsy) -- (b - ob * epsx, -epsy), p);
				}

				void drawIntervalLine(real a, real b, pen p) {
					draw((a, 0) -- (b, 0), p);
				}

				void drawCross(real x, pen p) {
					draw((x-epsx, -epsy) -- (x+epsx, epsy), p);
					draw((x+epsx, -epsy) -- (x-epsx, epsy), p);
				}

				label("$\R$", (5, 0), align=SE);
				label("$D$", (4, 0), magenta, align=SW);
				label("$\overline{D}$", (-4, 0), deepcyan, align=SE);
				drawInterval(-4, -1, magenta, 1, -1);
				drawInterval(0, 1, magenta, 1, 1);
				drawInterval(2, 4, magenta, -1, 1);


				drawIntervalLine(-4, -1, deepcyan);
				drawIntervalLine(0, 1, deepcyan);
				drawIntervalLine(2, 4, deepcyan);

				epsx *= 0.3;
				epsy *= 0.3;

				drawCross(-1, deepcyan);
				drawCross(2, deepcyan);
			\end{asy}
	\end{figure}

	On dit que $f(x)$ \underline{tends vers} $\ell$ si \[
		\forall V \in \mathcal{V}_\ell,\,\exists W \in \mathcal{V}_a,\,\forall x \in W \cap D,\,f(x) \in V
	.\]
\end{defn}

\begin{exm}
	Soit $f: \R\to \R$ telle que $f(x) \tendsto{x\to -1} 1$. \[
		\forall V \in \mathcal{V}_{1},\,\exists W \in \mathcal{V}_{-1},\,\forall x \in W,\,f(x) \in V
	\] donc \[
		\forall \varepsilon > 0,\,\exists \eta > 0,\,\forall x \in\; ]-1-\eta,\,-1+\eta[,\, f(x) \in [1-\varepsilon, 1+\varepsilon[
	\] i.e. \[
		\forall \varepsilon > 0,\,\exists \eta > 0,\,\forall x \in \R,\,\big(|x-(-1)| < \eta \implies |f(x) - 1| \le \varepsilon\big)
	.\]

	Si $f(x) \tendsto{x\to -\infty}+\infty$, \[
		\forall V \in \mathcal{V}_{+\infty},\,\exists W \in \mathcal{V}_{-\infty},\,\forall x \in W,\,f(x) \in V
	\] i.e. \[
		\forall M \in \R,\,\exists m \in \R,\,\forall x \in \,]-\infty, m],\,f(x) \in [M,+\infty[
	\] i.e. \[
		\forall M \in \R,\,\exists m \in \R,\,\forall x \in \R,\,\big(x \le m \implies f(x) \ge M\big)
	.\] 
\end{exm}

\begin{defn}
	Soit $a \in \R$.

	Un \underline{voisinage à gauche}\index{voisinage à gauche} de $a$ est une partie de $\R$ qui contient un inveralle $]a-\eta,a]$ avec $\eta > 0$.

	Un \underline{voisinage à droite}\index{voisinage à droite} de $a$ est une partie de $\R$ qui contient un inveralle $[a,a+\eta[$ avec $\eta > 0$.
\end{defn}

\begin{exm}\relax
	{\it
		Soit $f: x \mapsto \begin{cases}
			\frac{1}{x} &\text{ si } x \neq  0,\\
			0 &\text{ si } x = 0.
		\end{cases}$
		\begin{enumerate}
			\item $\lim_{x\to 0}f(x)$ existe ? Si oui, que vaut elle ?
			\item $\lim_{\substack{x\to 0\\\le }}f(x)$ existe ? Si oui, que vaut elle ?
		\end{enumerate}
	}

	Réponse : $\lim_{x \to 0}f(x)$ n'existe pas. Pour le prouver, on raisonne par l'absurde.
	\[
		\ell = \bigcap_{V \in \mathcal{V}_\ell} V
	\]
	Si $\ell$ existe, alors $\ell = f(0)$.\\
	Or, $0 \neq \lim_{x \to 0} f(x)$

	\begin{center}
		\begin{asy}
			size(5cm);
			import graph;

			real f(real x) {
				return 1/x;
			}

			draw(graph(f, 0.3, 5), deepcyan);
			draw(graph(f, -5, -0.3), deepcyan);

			dot((0,0),deepcyan);

			axes(EndArrow);

			real w = 0.1;

			void t(real epsilon) {
				real eta = f(epsilon);

				draw((epsilon,w)--(epsilon,-w));
				draw((-epsilon,w)--(-epsilon,-w));

				draw((w,eta)--(-w,eta));
				draw((w,-eta)--(-w,-eta));
			}

			t(4);
			t(2);
			t(1);

			draw((1.5,0)--(0.25,0), Arrow(TeXHead));
			draw((1.5,0)--(0.5,0), Arrow(TeXHead));
			draw((-1.5,-0)--(-0.25,-0), Arrow(TeXHead));
			draw((-1.5,-0)--(-0.5,-0), Arrow(TeXHead));

			pair O = (0,0);
			draw(O--(0,2), Arrow(TeXHead));
			draw(O--(0,-2), Arrow(TeXHead));
			draw(O--(0,2.25), Arrow(TeXHead));
			draw(O--(0,-2.25), Arrow(TeXHead));
		\end{asy}
	\end{center}
\end{exm}

\begin{prop}
	Si $f$ admet une limite finie en $a \in I$, alors cette limite vaut $f(a)$.
\end{prop}

\begin{prv}
	Soit $\ell = \lim_{x \to a}$ et $a \in \mathcal{D}$.

	On sait que  \[
		\forall V \in \mathcal{V}_{\ell}, \exists W \in \mathcal{V}_a, \forall x \in W \cap \mathcal{D}, f(x) \in V.
	\]

	Soit $V \in \mathcal{V}_\ell$. Alors, $f(a) \in V$.

	\[
		f(a) \in \bigcap_{V \in \mathcal{V}_\ell} V = \begin{cases}
			\{\ell\} &\text{ si } \ell \in \R,\\
			\O &\text{ si } \ell = \pm\infty.
		\end{cases}
	\]
	Donc $\ell \in \R$ et $\ell = f(a)$.
\end{prv}

\begin{rmk}
	De même si $a \in \mathcal{D}$ et si $\lim_{\substack{
			x\to a\\
			 \le	
	}} f(x)$ existe (resp. $\lim_{\substack{
			x\to a\\
			 \le	
	}} f(x)$) alors $f(a) = \lim_{\substack{
		x \to a\\
		\le 
	}} f(x)$ (resp $f(a) = \lim_{\substack{
		x \to a\\
		\ge 
	}} f(x)$ )
\end{rmk}

\begin{exm}\relax
	\begin{figure}[H]
		\centering
		\begin{asy}
			import graph;

			size(5cm);

			label("$\delta$", (4, 0), magenta, align=SE);
			draw((-5.3, 0) -- (5.3, 0), Arrow(TeXHead));
			draw((-5,0) -- (5,0), magenta);
			dot((0,0), white);
			draw((0, -0.3) -- (0, 1.5), Arrow(TeXHead));
			dot((0,1), magenta);
		\end{asy}
	\end{figure}

	$\lim_{x \to 0} \delta(x)$ n'existe pas\\
	$\lim_{x \to 0^+} \delta(x)$ et $\lim_{x \to 0^-}$ n'existent pas non plus.\\

	$\lim_{\substack{
		x \to 0\\
		\neq
	}} \delta(x)$\\

	$\lim_{\substack{
		x \to  0\\
		<
	}} \delta(x) = 0$, $\lim_{\substack{
		x \to 0\\
		>
	}} \delta(x) = 0$\\

	\[
		\lim_{x \to a}\frac{f(x) - f(a)}{x-a} = \lim_{\substack{
			x \to a\\
			\neq
		}} \frac{f(x) - f(a)}{x - a}
	\]
\end{exm}

\begin{defn}
	Soit $f$ définie sur $D$ et $a \in D$. On dit que $f$ est \underline{continue en $a$} si $\lim_{x \to a} f(x)$ existe ou si $\lim_{\substack{
		x \to a\\
		\neq 
	}}f(x) = f(a)$.
	\index{continuité en un point}
\end{defn}

\begin{exm}
	Soit $f: x \mapsto \begin{cases}
		\frac{\sin(x)}{x} & \text{ si } x\neq 0\\
		1 &\text{ si } x = 0
	\end{cases}$\\

	$\lim_{\substack{
		x \to 0\\
		\neq
	}} f(x) = \lim_{x \to  0} \frac{\sin(x)}{x} = 1 = f(0)$ \\
	Donc $f$ est continue en $0$.
\end{exm}

\begin{prop}
	$f$ est continue en $a$ si et seulement si \[
		\lim_{\substack{
			x \to a\\
			<
		}} f(x) = \lim_{\substack{
			x \to a\\
			>
		}} = f(a)
	\]
\end{prop}

\begin{prv}
	[unicité de la limite]
	On suppose que $f(x) \tendsto{x \to u}a$, $f(x) \tendsto{x \to u}b$ avec $a \neq b$.\\
	Soient $V$ et $W$ conne dans le lemme (suivant), \[
		\begin{cases}
			\exists W_1 \in \mathcal{V}_u, \forall x \in W_1 \cap  \mathcal{D}, f(x) \in V\\
			\exists W_2 \in \mathcal{V}_u, \forall x \in W_2 \cap  \mathcal{D}, f(x) \in W\\
		\end{cases}
	\] 
	Donc \[
		\forall x \in \underbrace{W_1 \cap W_2 \cap \mathcal{D}}_{\neq \O \text{ car } W_1\cap W_2 \in \mathcal{V}_u} f(x)  \in V \cap W = \O
	\] 
\end{prv}

\begin{lem}
	Soient $a \neq  b$ deux éléments de $\overline{\R}$ \\
	Alors $\exists V \in \mathcal{V}_a, \exists W \in \mathcal{V}_b, V \cap W = \O$
\end{lem}

\begin{prv}
	\begin{itemize}
		\item[\sc Cas 1] $(a,b) \in \R^2$. On suppose que $a < b$.\\
			On pose $\varepsilon = \frac{b-a}{2}$, \[
				\begin{cases}
					V = ]a-\varepsilon;a+\varepsilon[\\
					W = ]b-\varepsilon;b+\varepsilon[
				\end{cases}
			\] 
			On vérifie que $V \cap W = \O$

			\begin{center}
				\begin{asy}
					import graph;

					xaxis(EndArrow);
					size(5cm);

					dot((-10,1), white+0);
					dot((10,-1), white+0);

					real a = -3;
					real b = 3;

					dot("$a$", (a,0), green, align=N);
					dot("$b$", (b,0), magenta, align=N);

					real eps = 1.6;

					pair a1 = (a - eps, 0);
					pair a2 = (a + eps, 0);

					pair b1 = (b - eps, 0);
					pair b2 = (b + eps, 0);
					
					real w = 0.4;
					void t(pair z) {draw((z.x, w) -- (z.x, -w));}

					t(a1); t(a2);
					t(b1); t(b2);

					pair s = (0, -1.8);
					draw(brace(a2 + s, a1 + s, .65));
					draw(brace(b2 + s, b1 + s, .65));

					label("$V$", (a1+a2)/2 + 2 * s);
					label("$W$", (b1+b2)/2 + 2 * s);

					draw((a, -0.8)--(a+eps, -0.8), heavycyan, Arrows(TeXHead));
					draw((b, -0.8)--(b+eps, -0.8), heavycyan, Arrows(TeXHead));
					label("$\varepsilon$", (a+eps/2, -1.35), heavycyan);
					label("$\varepsilon$", (b+eps/2, -1.35), heavycyan);
				\end{asy}
			\end{center}

		\item[\sc Cas 2] $a \in \R$ et $b = +\infty$ \\
			\[
				\begin{cases}
					V = ]a-1;a+1[\\
					W = ]a+2;+\infty[
				\end{cases}
			\]

			\begin{center}
				\begin{asy}
					import graph;

					xaxis(EndArrow);
					size(5cm);

					dot((-10,1), white+0);
					dot((10,-1), white+0);

					real a = -3;
					real b = 7;

					dot("$a$", (a,0), green, align=N);

					real eps = 2.3;

					pair a1 = (a - eps, 0);
					pair a2 = (a + eps, 0);

					pair b1 = (b - eps, 0);
					pair b2 = (b + eps, 0);
					
					real w = 0.8;
					void t(pair z, real k=-0.2) {draw((z.x+k,w)--(z.x, w)--(z.x, -w)--(z.x+k,-w));}

					t(a1); t(a2,k=0.2);
					t(b1);

					pair s = (0, -2.3);
					draw(brace(a2 + s, a1 + s, .65));
					draw(brace((10, s.y), b1 + s, .65));

					label("$V$", (a1+a2)/2 + 2 * s);
					label("$W$", (b1+(10,0))/2 + 2 * s);

					draw((a, -1.2)--(a+eps, -1.2), heavycyan, Arrows(TeXHead));
					label("$1$", (a+eps/2, -1.8), heavycyan+fontsize(7pt));
				\end{asy}
			\end{center}
		\item[\sc Cas 3] $a = -\infty$, $b = +\infty$ \\
			\[
				\begin{cases}
					V = ]-\infty; 0[\\
					W = ]0;+\infty[
				\end{cases}
			\] 
	\end{itemize}
\end{prv}

\begin{thm}
	Soit $f$ définie sur $\mathcal{D}$ et $a \in \overline{\mathcal{D}}$, $\ell \in \overline{\R}$ \[
		f(x) \tendsto{x \to a} \ell \iff \forall (x_n) \in \mathcal{D}^\N \left( x_n \tendsto{n\to +\infty} a \implies f(x_n) \tendsto{n \to +\infty} \ell \right) 
	\]
\end{thm}

\begin{prv}
	 \begin{itemize}
		\item[$``\implies"$] On suppose que $f(x) \tendsto{x \to a} \ell$.\\
			Soit $(x_n) \in \mathcal{D}^\N$ telle que $x_n \tendsto{n\to +\infty} \ell$\\
			Soit $V \in \mathcal{V}_\ell$. Soit $W \in \mathcal{V}_a$ tel que \[
				\forall x \in W \cap \mathcal{D}, f(x) \in V
			\] $x_n \tendsto{n \to +\infty} a$ donc il existe $N \in \N$ tel que \[
				\forall n \ge N, x_n \in W \cap \mathcal{D}
			\] Donc  \[
				\forall n \ge  N, f(x_n) \in V
			\] D'où, $f(x_n) \tendsto{n \to +\infty} \ell$\\
		\item[$``\impliedby"$ ] On suppose que $f(x) \centernot{\tendsto{x \to a}} \ell$\\
			\[
				\exists V \in \mathcal{V}_\ell, \forall W \in \mathcal{V}_a, \exists x \in W \cap  \mathcal{D}, f(x) \not\in V
			\] Soit $V$ comme ci dessus. Soit $W_1 \in \mathcal{V}_a$.\\
			\begin{itemize}
				\item[\sc Cas 1] $a \in \mathcal{D}$ et $\forall x \in W \cap \mathcal{D} \setminus \{a\}, f(x)  \in V$.\\
					On le prouve par la contraposée. On suppose $f(x) \centernot{\tendsto{x\to a}} \ell \in \R$\\
					Donc, \[
						\exists \varepsilon > 0, \forall \eta > 0, \exists x \in ]a-\eta, a+\eta[, \left| f(x) - \ell \right| > \varepsilon
					\] On considère un tel $\varepsilon$ donc \[
						\forall n \in \N_*, \exists x_n \in \left]a-\frac{1}{n}, a+\frac{1}{n}\right[, \left| f(x_n) - \ell \right| > \varepsilon
					\] Par encadrement, $x_n \tendsto{n\to +\infty} a$ et $f(x_n) \centernot{\tendsto{n\to +\infty}} \ell$
				\item[\sc Cas 2]
					Soit $x_1 \in W_1 \cap \mathcal{D}$ tel que $f(x_1) \not\in V$ \\
					$\begin{cases}
						x_1\in \mathcal{D}\\
						a \not\in \mathcal{D}
					\end{cases}$ donc $x_1 \neq a$ \\
				\item[\sc Cas 3] $\exists x \in W_1 \cap  \mathcal{D}\setminus \{a\}, f(x) \not\in V$\\
					Soit $x_1$ un tel élément:
					\begin{align*}
						&x_1 \in W_1 \cap \mathcal{D}\\
						&x_1 \neq a\\
						&f(x_1) \not\in V
					\end{align*}
			\end{itemize}
			Dans les cas 2 et 3, on pose $W_2 \in \mathcal{V}_a$ tel que \[
				W_2 \subset W_1 \setminus \{x_1\}
			\] En itérant ce procédé, on construit une suite $(x_n)$ qui tend vers $a$ et telle que \[
				\forall n \in \N, f(x_n) \not\in V
			\] et donc $f(x_n) \centernot{\tendsto{n\to +\infty}} \ell$
	\end{itemize}
\end{prv}


\begin{prop}
	Si  $f(x) \tendsto{x \to a} \ell$ et $g(x) \tendsto{x \to a} \ell_2$\\
	alors
	\begin{enumerate}
		\item $f(x) + g(x) \tendsto{x \to a} \ell_1 + \ell_2$\\
		\item $f(x) \times g(x) \tendsto{x \to a} \ell_1 \times \ell_2$\\
		\item Si $\ell_2 \neq 0$, $\frac{f(x)}{g(x)} \tendsto{x \to a} \frac{\ell_1}{\ell_2}$
	\end{enumerate}
\end{prop}

\begin{prv}
	\begin{enumerate}
		\item Soit $(x_n)$ une suite qui tends vers $a$ alors $f(x_n) \tendsto{n \to +\infty} \ell_1$ et $g(x_n) \tendsto{n \to +\infty} \ell_2$\\
			Donc, $f(x_n) + g(x_n) \tendsto{n \to +\infty} \ell_1 + \ell_2$ \\
			Donc $f(x) + g(x) \tendsto{n \to +\infty} \ell_1 + \ell_2$
	\end{enumerate}
\end{prv}

\begin{prop}
	Si $f(x) \tendsto{x \to a} \ell_1$ et $g(x)\tendsto{x \to \ell_1} \ell_2$ alors $g(f(x)) \tendsto{x \to a} \ell_2$
\end{prop}

\begin{prv}
	Soit $(x_n)$ une suite qui tend vers $a$. Alors, $f(x_n) \tendsto{n \to +\infty} \ell_1$ donc $g(f(x_n)) \tendsto{n \to +\infty} \ell_2$ donc $g(f(x)) \tendsto{x \to a} \ell_2$
\end{prv}

\begin{crlr}
	Une somme, un produit, une composée de fonctions continues sont continues.
	\qed
\end{crlr}

\begin{rmk}
	Pour démontrer que $f(x)$ n'a pas de limite quands $x$ tend vers $a$. On cherche deux suites $(x_n)$ et $(y_n)$ de limite $a$ avec \[
		\begin{cases}
			f(x_n) \longrightarrow \ell_1\\
			f(y_n) \longrightarrow \ell_2\\
			\ell_1\neq \ell_2
		\end{cases}
	\]
\end{rmk}

\begin{exm}
	\begin{align*}
		\forall n \in \N, &\sin(2\pi n) = 0 \tendsto{n \to +\infty} 0\\
		&\sin\left(\frac{\pi}{2} + 2\pi n\right) = 1 \tendsto{n \to +\infty} 1\\
	\end{align*}
	Or,
	\begin{align*}
		2\pi n \tendsto{n \to +\infty} +\infty\\
		\frac{\pi}{2} + 2\pi n \tendsto{n \to +\infty} +\infty\\
	\end{align*}

	Donc, $\sin$ n'a pas de limite en $+\infty$
\end{exm}

\begin{thm}
	[Limite monotone]
	Soit $f$ une fonction croissante sur $]a,b[$ avec $a \neq b \in \overline{\R}$.
	 \begin{enumerate}
		\item Si $f$ est majorée, \[
				\exists M \in \R, \forall x \in ]a,b[, f(x) \le M
			\] alors $\lim_{\substack{
				x \to b\\
				<
			}}f(x) = \sup_{x \in ]a,b[}f(x) \in \R$ 
		\item Si $f$ n'est pas majorée, \[
				\lim_{\substack{
					x \to b\\
					<
				}} f(x) = +\infty
			\]
		\item Si $f$ est minorée, \[
				\exists  m \in \R, \forall x \in ]a,b[, f(x) \le m
			\] alors $\lim_{\substack{
				x \to a\\
				>
			}} f(x) = \inf_{]a,b[}f \in \R$ 
		\item Si $f$ n'est pas minorée, $\lim_{\substack{
			x \to a\\
			>
		}} f(x) = -\infty$
	\end{enumerate}
\end{thm}

\begin{prv}
	\begin{enumerate}
		\item $\sup_{]a,b[}f$ existe\\
			\[
				\forall \varepsilon>0, \exists x \in ]a,b[, f(x) > \sup_{]a,b[}(f) - \varepsilon
			\] donc \[
				\forall \varepsilon >0, \exists x \in ]a,b[, \forall y \in [x,b[, \sup_{]a,b[}(f) - \varepsilon < f(y) \le \sup_{]a,b[}(f) < \sup_{]a,b[}(f) + \varepsilon
			\] donc $f(x) \tendsto{\substack{
				x \to b\\
				<
			}} \sup_{]a,b[}(f)$ 
		\item $f$ n'est pas majorée \[
			\forall M \in \R, \exists x \in ]a,b[, f(x) > M
		\] donc \[
			\forall M \in \R, \exists x \in ]a,b[, \forall  y \in [x,b[, f(y) \in [M, +\infty[
		\]
	\end{enumerate}
\end{prv}

\begin{rmk}
	Avec les hypothèses ci-dessus, pour tout $x \in ]a,b[$,\\
	$f$ est croissante sur $]a,x[$, et majorée par  $f(x)$ donc $\lim_{\substack{
		t \to x\\
		<
	}}f(t) \in \R$ \\
	$f$ est croissante sur $]x,b[$ et minorée par $f(x)$ donc $\lim_{\substack{
		t \to x\\
		>
	}} f(t) \in \R$\\
	$
	\lim_{\substack{
		t \to x\\
		<
	}} f(t)\le 
	f(x)
	\le \lim_{\substack{
		t \to x\\
		>
	}} f(t)$

	\begin{center}
		\begin{asy}
			import graph;
			size(5cm);

			axes(EndArrow);

			real f(real x) {
				return (5/24)*x^5 - (11/4)*x^4 + (101/8)*x^3 - (93/4)*x^2 + (91/6)*x;
			}

			real x = 2.8;
			real eps = 0.06;
			draw(graph(f,0, x-eps), magenta);
			draw(graph(f,x+eps,5), magenta);
			
			dot((x, f(x)), magenta);

			dot((-0.4,-0.4), white+0);
		\end{asy}
	\end{center}
\end{rmk}

\begin{thm}
	[Théorème des valeurs intermédiaires]

	Soit $f$ une fonction continue sur un intervalle $I$, $a<b$ deux éléments de $I$. \[
		\forall y \in \left[ f(a),f(b) \right] \cup \left[ f(b),f(a) \right],~
		\exists x \in [a,b],~ y = f(x)
	\] 
	\begin{center}
		\begin{asy}
			import graph;
			size(5cm);
			
			axes(EndArrow);

			real a = 1.5;
			real b = 8;

			real f(real xe) {
				real x = xe - 6.5;
				return (x^3+5*x^2)/3+2;
			}

			path f_graph = graph(f,a,b);

			draw(f_graph, magenta);

			real y = 5;
			path y_line = (0,y)--(10,y);

			draw(y_line, dashed+red);
			label("$y$", (10,y), align=E, red);

			draw((a,0)--(a,f(a)), dashed+deepcyan);
			draw((b,0)--(b,f(b)), dashed+deepcyan);

			label("$a$", (a,0), align=S, deepcyan);
			label("$b$", (b,0), align=S, deepcyan);

			pair[] roots = intersectionpoints(y_line, f_graph);

			for(int i=0; i < 3; ++i) {
				real x = roots[i].x;
				draw((x,0)--(x,y), dashed+heavygreen);
				label("$x$", (x,0), heavygreen, align=S);
			}
		\end{asy}
	\end{center}
\end{thm}

\begin{lem}
	Soit $f$ une fonction continue sur un intervalle $I$, $a<b$ deux éléments de $I$ tels que $f(a) \le 0 \le f(b)$.
	Alors, \[
		\exists x \in [a,b], f(x) = 0
	\]
\end{lem}

\begin{prv}
	[du lemme]
	On pose $A = \left\{x \in [a,b]  \mid  f(x) \le 0 \right\} $ \\
	$A \neq \O$ car $a \in A$ et $A$ est majorée par $b$.\\
	On pose $u = \sup(A)$. Soit $(x_n) \in A^\N$ qui converge vers $u$.\\
	$\forall n \in \N, x_n \in A$ donc $\forall n \in \N, \begin{cases}
		a \le x_n \le b\\
		f(x_n) \le 0
	\end{cases}$ \\
	On sait que $x_n \longrightarrow u$ et $f(x_n) \longrightarrow f(u)$ par continuité de $f$.\\
	Donc, $\begin{cases}
		a \le u \le b\\
		f(u) \le 0
	\end{cases}$ (donc $u = \max(A)$)\\
	De plus, \[
		\forall x \in ]u,b], f(x) > 0
	\] donc \[
		\begin{cases}
			\lim_{\substack{x \to u\\>}}f(x) = f(u)\\
			\\
			\lim_{\substack{x \to u\\>}}f(x) \ge 0\\
		\end{cases}
	\]
	Donc, $f(u) \ge 0$ donc $f(u) = 0$ \\
\end{prv}

\begin{prv}
	[du théorème]

	On pose $g: x \mapsto f(x) - y$. $g$ est continue sur $I$.\\

	\underline{Si} $f(a) < f(b)$ alors $\begin{cases}
		g(a) \le 0\\
		g(b) \ge 0
	\end{cases}$ \\
	D'après le lemme, il existe $x \in [a,b]$ tel que $g(x) = 0$ et donc $f(x) = y$ \\
	\vspace{4mm}

	\underline{Si} $f(a) < f(b)$ alors $\begin{cases}
		h(a) \le 0\\
		h(b) \ge 0
	\end{cases}$ où $h: x\mapsto -g(x) = y - f(x)$ est continue\\
	D'après le lemme, il existe $x \in [a,b]$ tel que $h(x) = 0$ et donc $f(x) = y$
\end{prv}

\begin{crlr}
	Soit $f$ continue sur un intervalle $I$. Alors, $f(I)$ est un intervalle.

	\begin{center}
		\begin{asy}
			import graph;
			size(5cm);
			
			axes(EndArrow);

			real a = 1.7;
			real b = 8;

			real f(real xe) {
				real x = xe - 6.5;
				return (x^3+5*x^2)/3+2;
			}

			path f_graph = graph(f,a,b);

			draw(f_graph, magenta);

			draw((a,0)--(a,f(a)), dashed+deepcyan);
			draw((b,0)--(b,f(b)), dashed+deepcyan);

			draw((0,f(a))--(a,f(a)), dashed+deepcyan);
			draw((0,f(b))--(b,f(b)), dashed+deepcyan);

			label("$a$", (a,0), align=S, deepcyan);
			label("$b$", (b,0), align=S, deepcyan);

			label("$f(a)$", (0,f(a)), align=W, deepcyan);
			label("$f(b)$", (0,f(b)), align=W, deepcyan);

			draw(brace((a-0.5,-1.2),(b+0.5,-1.2), -0.6));
			label("$I$", ((a+b)/2, -1.7), align=S);

			dot((b+2,0),white+0);
		\end{asy}
	\end{center}
\end{crlr}

\begin{prv}
	Montrons que $f(I)$ est convexe\\
	Soit $\alpha \in f(I), \beta\in f(I)$ avec $\alpha<\beta$. Montrons que \[
		\forall \gamma \in [\alpha,\beta], f(\gamma) \in f(I)
	\] Or, $\begin{cases}
		\alpha \in f(I) ~~ \exists  a \in I, \alpha = f(a)\\
		\beta \in f(I) ~~ \exists  b \in I, \beta = f(b)
	\end{cases}$ \\
	D'après le théorème des valeurs intermédiaires, il existe $x \in [a,b]$ tel que $\gamma = f(x)$ donc, $f(\gamma) \in f(I)$
\end{prv}

\begin{crlr}
	On peut généraliser le théorème des valeurs intermédiaires au cas où $\begin{cases}
		a \in \overline{R}\\
		b \in \overline{\R}
	\end{cases}$ en remplaçant $f(a)$ par $\lim_{\substack{x \to a\\>}}f(x)$ et $f(b)$ par $\lim_{\substack{x \to b\\<}}f(x)$ \\
	\qed
\end{crlr}

\begin{thm}
	[Théorème de la bijection]

	Soit $f$ continue, strictement monotone sur un intervalle $I$.
	Alors, $J = f(I)$ est un intervalle de même nature (ouvert, semi-ouvert ou fermé) et $f$ établit une bijection de $I$ sur $J$.
\end{thm}

\begin{prv}
	D'après le théorème des valeurs intermédiaires, $J$ est un intervalle.
	$f$ est strictement monotone donc $f$ injective.
	Donc $f$ établit une bijection de $I$ sur $J$.\\

	\begin{itemize}
		\item[\sc Cas 1] $I = [a,b]$ et $f$ croissante\\
			$\forall x \in I, a \le x \le b$\\
			donc  $\forall x \in I, f(a) \le f(x) \le f(b)$ \\
			donc $J \subset [f(a), f(b)]$\\
			Or, $[f(a), f(b)] \subset J$ d'après le théorème des valeurs intermédiaires\\
			Donc $J = [f(a), f(b)]$
	\end{itemize}
	~~~~ Les autres cas se démontrent de la même façon.
\end{prv}

\begin{thm}
	Soit $f$ continue sur un segment $[a,b]$. Alors, $f$ est bornée et atteint ses bornes, i.e. \[
		\exists (m,M) \in \R^2, f([a,b]) = [m,M]
	\]\\
	\danger On peut avoir $m \neq f(a)$ et $M \neq f(b)$
	\begin{center}
		\begin{asy}
			import graph;
			size(5cm);
			
			axes(EndArrow);

			real a = 1.7;
			real b = 8;

			real f(real xe) {
				real x = xe - 6.5;
				return (x^3+5*x^2)/3+2;
			}

			path f_graph = graph(f,a,b);

			draw(f_graph, magenta);

			draw((a,0)--(a,f(a)), dashed+deepcyan);
			draw((b,0)--(b,f(b)), dashed+deepcyan);

			draw((0,f(a))--(a,f(a)), dashed+deepcyan);
			draw((0,f(b))--(b,f(b)), dashed+deepcyan);

			label("$a$", (a,0), align=S, deepcyan);
			label("$b$", (b,0), align=S, deepcyan);

			label("$f(a)$", (0,f(a)), align=W, deepcyan);
			label("$f(b)$", (0,f(b)), align=W, deepcyan);

			dot((b+2,0),white+0);

			real M = max(f_graph).y;
			real m = min(f_graph).y;
			pair Mp = intersectionpoint(f_graph, (0,M)--(10,M));
			pair mp = intersectionpoint(f_graph, (0,m)--(10,m));

			draw(Mp--(0,M), heavyred + dashed);
			draw(mp--(0,m), heavyred + dashed);
			label("$M$", (0,M), heavyred, align=W);
			label("$m$", (0,m), heavyred, align=W);

			dot((0,M+2),white+0);
		\end{asy}
	\end{center}
\end{thm}

\begin{prv}
	On suppose que $f$ n'est pas majorée: \[
		\forall M \in \R, \exists x \in [a,b], f(x) \ge M
	\] En particulier, \[
		\forall n \in \N, \exists x_n \in [a,b], f(x_n) \ge n
	\] Donc, $f(x_n) \tendsto{n \to +\infty} +\infty$ \\
	La suite $(x_n)_{n\in\N}$ est minorée apr $a$ et majorée par $b$ donc bornée.\\
	D'après le théorème de Bolzano-Weierstrass, il existe $\varphi: \N \longrightarrow \N$ strictement croissante telle que $(x_{\varphi(n)})_{\in\N}$ converge. On pose $\ell = \lim_{n \to +\infty} x_{\varphi(n)}$. On a bien $\ell \in [a,b]$ et $f(\ell) = \lim_{n \to +\infty} f\left(x_{\varphi(n)}\right)$ par continuité de $f$.\\
	Or, $\left( f\left( x_{\varphi(n)} \right)  \right) _{n \in \N}$ est une sous-suite de $\left( f(x_n) \right)$  donc $f\left( x_{\varphi(n)} \right) \tendsto{n \to +\infty}+\infty$: une contradiction\\
	Donc $f$ est majorée et on pose \[
		M = \sup_{a \le x \le b}f(x)
	\] On prouve de même que $f$ est minorée. On pose donc \[
		m = \inf_{a \le x \le b} f(x)
	\]

	Soit $(y_{n}) \in [a,b]^\N$ telle que $f(y_{n}) \tendsto{n \to +\infty} M$.\\
	$(y_{n})$ est bornée donc il existe une sous-suite $\left( y_{\psi(n)} \right)$ de $(y_n)$ convergente. On pose $y = \lim_{n \to +\infty} y_{\psi(n)} \in [a,b]$ \\
	Comme $f$ continue sur $y$, \[
		f(y) = \lim_{n \to +\infty} f\left( y_{\psi(n)} \right) 
	\] Or, $\left( f\left( y_{\psi(n)} \right)  \right) $ est une sous-suite de $\left( f(y_n) \right)$ donc \[
		M = \lim_{n \to +\infty} f\left(y_{\psi(n)}\right)
	\] Par unicité de la limite, $M = f(y)$\\
	Donc, $M = \max_{a \le x \le b} f(x)$. De même, $m \in f([a,b])$ \\
	Enfin, en posant $\begin{cases}
		M = f(y) &\text{ avec } y \in [a,b]\\
		m = f(z) &\text{ avec } z \in [a,b]\\
	\end{cases}$, on obtient \[
		[m,M] = [f(z),f(y)] \underbrace{\subset}_{\mathllap{\text{théorème des valeurs intermédiaires}}} f([a,b]) \underbrace{\subset}_{\mathclap{\substack{m \text{ minimum}\\ \\ M \text{ maximum}}}} [m,M]
	\] donc $f([a,b]) = [m,M]$
\end{prv}
