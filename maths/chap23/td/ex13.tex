\part{Exercice 13}

\begin{enumerate}
	\item $f \circ f = \id_{\mathcal{P}(E)}$.
	\item $g : \begin{array}{rcl}
			\mathcal{P}_0 &\longrightarrow& \mathcal{P}_1 \\
			X &\longmapsto& f(X)
		\end{array}$ bijective donc $\#\mathcal{P}_0 = \#\mathcal{P}_1$.
	\item On en déduit que si $n > 0$ \[
			\#\mathcal{P}_0 = \frac{1}{2} \#\mathcal{P}(E) = \frac{2^n}{2} = 2^{n-1}.
		\]
		On a \[
			\#\mathcal{P}_0 = \sum_{\substack{0 \le i \le n \\ i \text{ pair}}} {n \choose i}
		\] et \[
			\#\mathcal{P}_1 = \sum_{\substack{0 \le i \le n \\ i \text{ impair}}} {n \choose i}.
		\]

		\begin{align*}
			\sum_{k=0}^n (-1)^k{n\choose k} &= \sum_{\substack{0 \le k \le n\\ k \text{ pair}}} {n \choose k} + \sum_{\substack{0 \le k \le n\\ k \text{ impair}}} \left( - {n \choose k} \right)  \\
			&= \#\mathcal{P}_0 - \#\mathcal{P}_1 \\
			&= 0 \\
		\end{align*}
\end{enumerate}
