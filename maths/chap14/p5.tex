\part{Annexe}

\begin{thm}
	{\it Théorème 2.11}\\
	$f: I \to J$ bijective monotone avec $I$ et $J$ deux intervalles.\\
	Alors, $f^{-1}$ est continue (et $f$ aussi)
\end{thm}

\begin{prv}
	$f$ monotone donc $f(I) = J$\\
	donc $f$ continue (d'après 2.10).\\
	$f^{-1}$ monotone, $f^{-1}(J) = I$\\
	donc $f^{-1}$ est continue
\end{prv}

\begin{defn}
	Un \underline{homéomorphisme} est une application bijective, continue dont la réciproque est continue.
	\index{homéomorphisme}
\end{defn}

\begin{rmk}
	Preuve du programme de colle
\end{rmk}

\begin{prv}
	\[
		\exists \eta > 0, \forall h \in ]-\eta, +\eta[, f(a) \ge f(a + h)
	\]

	\begin{center}
		\begin{asy}
			import graph;
			size(5cm);

			axes(EndArrow);

			real[] coeffs = {-0.028381812436924655,0.6474407036068452,-3.7312604354868317,0.054371321739291545};

			real f(real x) {
				real y = 0;
				int k = coeffs.length;

				for(int i = 0; i < k; ++i) {
					y -= x^i * coeffs[k-i-1];
				}

				return y;
			}

			draw(graph(f, -1, 16), deepmagenta);
			real a = 3.863;
			real h = 1.5;

			dot("$a$", (a,0), red, align=S);
			draw((a,0) -- (a, f(a)), dashed + red);

			draw((a-h,0) -- (a - h, f(a-h)), dashed + deepred);
			draw((a+h,0) -- (a + h, f(a+h)), dashed + deepred);

			draw((a,-2)--(a-h,-2), Arrows(TeXHead));
			label("$h$", (a-h/2, -2), align=S);
		\end{asy}
	\end{center}


	\begin{align*}
		f'(a) = \lim_{h\to 0} \frac{f(a+h)-f(a)}{h} &= \lim_{\substack{h\to 0\\>}} \frac{f(a+h)-f(a)}{h} \le 0 \\
		&= \lim_{\substack{h\to 0\\<}} \frac{f(a + h) - f(a)}{h} \ge 0 \\
	\end{align*}

	Donc, $f(a) = 0$
\end{prv}

