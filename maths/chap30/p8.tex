\part{Fonctions réglées}

Soit $f : [a,b] \to \R$ continue. Cette fonction est donc uniformément continue (théorème de Heine) : \[
	\forall \varepsilon > 0,\, \exists \eta > 0,\, \forall (x,y) \in [a,b]^2,\,|x-y|\le \eta \implies \big|f(x) - f(y)\big| \le \varepsilon
.\]

D'où \[
	\forall n \in \N^*,\,\exists \eta_n > 0,\,|x-y| \le \eta_n \implies \big|f(x) - f(y)\big| \le \frac{1}{n}
.\]

\begin{figure}[H]
	\centering
	\begin{asy}
		import graph;
		import patterns;

		add("hatch", hatch(2mm, magenta));

		real[] coeffs = {-0.028394180582212297,0.27523209017393424,0.24929746599526315,-6.4461353755869855};
		real f(real x) {
			x -= 3;
			real y = 0;
			for(int i = 0; i < coeffs.length; ++i) {
				y -= x^i * coeffs[coeffs.length-i-1];
			}
			return y;
		}

		size(5cm);
		path p = graph(f, 1, 9, 500);
		draw((-1, 0) -- (11, 0), Arrow(TeXHead));
		draw((0, min(min(p).y, -1)) -- (0, max(p).y + 1), Arrow(TeXHead));
		for(int i = 1; i < 9; ++i) {
			filldraw((i, f(i)) -- (i+1, f(i)) -- (i+1, 0) -- (i, 0) -- cycle, pattern("hatch"), magenta);
			draw((i, -0.5) -- (i+1, -0.5), Arrows(TeXHead));
			label("$\eta_n$", (i+0.5, -1), fontsize(6pt));
		}
		draw(p, red + 0.6);
	\end{asy}
\end{figure}

Cependant, contrairement au sommes de Darboux, cette construction utilise des notions que l'on admet : notamment, la convergence d'une suite de fonctions.

\begin{exm}[convergence d'une suite de fonctions]
	Pour $n \in \N$, on pose \begin{align*}
		f_n: [0,1] &\longrightarrow \R \\
		x &\longmapsto x^n
	\end{align*}

	\[
		\forall x \in [0,1],\,f_n(x) \tendsto{n\to +\infty} \begin{cases}
			0 & \text{ si } x < 1\\
			1 &\text{ sinon}.
		\end{cases}
	\]

	Pour $n \in \N$, $f_n$ est continue mais $\lim_{n\to +\infty} f_n$ ne l'est pas :
	\begin{figure}[H]
		\centering
		\begin{asy}
			import graph;
			draw((-0.1, 0) -- (1.1, 0), Arrow(TeXHead));
			draw((0,-0.1) -- (0, 1.1), Arrow(TeXHead));
			size(5cm);
			pen a = orange;
			pen b = yellow;
			for(int n = 1; n < 10; ++n) {
				real f(real x) { return x^n; }
				real t = n / 9;
				pen p = b * t + a * (1 - t);
				draw(graph(f, 0, 1), p);
			}
			draw((0,0)--(1,0), magenta);
			dot((1,1), magenta);
		\end{asy}
	\end{figure}
\end{exm}

