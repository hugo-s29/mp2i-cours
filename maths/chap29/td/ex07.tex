\part{Exercice 7}

On montre que $p$ est une projection : on montre $p \circ p = p$ et on calcule donc $M^2$ :
\begin{align*}
	M^2 &= \frac{1}{36} \begin{pmatrix}
		5&-2&1\\
		-2&2&2\\
		1&2&5
	\end{pmatrix}\\
	&= \frac{1}{36} \begin{pmatrix}
		30&-12&6\\
		-12&12&12\\
		6&12&30
	\end{pmatrix}\\
	&= M \\
\end{align*}

Donc $p \circ p = p$ et donc $p$ est la projection sur $\mathrm{Im}(p)$ parallèlement à $\Ker(p)$.

\[
	\rg(p) = \tr(p) = \frac{1}{6}(5+2+5) = 2
.\]

Donc, $\mathrm{Im}(p) = \Vect\big(p(e_1),p(e_2)\big)$\\
\phantom{Donc, }$\phantom{\mathrm{Im}(p)}= \Vect\big((5,-2,1),(-1,1,1)\big)$.

On remarque que $p(e_1 + 2e_2 - e_3) = 0$. D'où $\Ker(p) = \Vect(e_1 + 2e_2 - e_3)$.

\begin{gather*}
	\left<e_1+2e_2-e_3 \mid 5e_1-2e_2+e_3 \right> = 5-4-1 = 0\\
	\left<e_1+2e_2-e_3 \mid -e_1 + e_2 + e_3 \right> = -1 + 2 - 1 = 0
\end{gather*}

donc $\underbrace{\Ker p}_{\dim(\cdot) = 1} \subset \underbrace{\left( \mathrm{Im}\, p \right)^\perp}_{\dim(\cdot) = 3 - 2  =1}$ donc $\Ker p = \left( \mathrm{Im}\,p \right)^\perp$.

\begin{align*}
	(x,y,z) \in \mathrm{Im}\, p \iff& (x,y,z) \in (\Ker p)^\perp\\
	\iff& (x,y,z) \perp (1,2,-1)\\
	\iff& x + 2y - z = 0
\end{align*}

