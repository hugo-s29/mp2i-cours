\part{Résumé et exemples}

\begin{landscape}
	\begin{figure}[H]
		\centering
		\begin{asy}
			usepackage("mathtools");
			size(18cm);

			texpreamble("\renewcommand{\c}[1]{\begin{tabular}{c}#1\end{tabular}}");

			defaultpen(fontsize(4pt));

			real margin=1.5mm;

			object alpha = draw("$u_n\xrightarrow[n\to+\infty]{\phantom{n\to+\infty}} 0$ ?", box, (-0.5,0), margin, black, FillDraw(palegray));

			object beta = draw("$\sum u_n$ diverge", box, (1, 1), margin, cyan, FillDraw(palecyan));

			object gamma = draw("\c{$u_n$ de signe\\ constant à partir\\ d'un certain rang ?}", box, (1, -1), margin, black, FillDraw(palegray));

			object delta = draw("\c{On remplace $(u_n)$\\ par un équivalent\\ plus simple}", box, (3, -0.5), 0, white, FillDraw(white));

			object epsilon = draw("Comparer avec une intégrale", box, (4.5, 0.5), 0, white, FillDraw(white));

			object sigma = draw("\c{$v_n = O(w_n)$\\$\sum w_n$ converge ?}", box, (4.5, 0), margin, black, FillDraw(palegray));

			object tau = draw("\c{$w_n = O(v_n)$\\$\sum w_n$ diverge ?}", box, (4.5, -0.5), margin, black, FillDraw(palegray));

			object iota = draw("$\sum u_n$ diverge", box, (6, -0.5), margin, magenta, FillDraw(palemagenta));

			object pie = draw("$\sum u_n$ converge", box, (6, 0), margin, cyan, FillDraw(palecyan));

			object xi = draw("\c{$\sum u_n$ alternée\\$\big(|u_n|\big)\longrightarrow 0$\\$\big(|u_n|\big)$ décroissante}\!\!\large?", box, (3, -1.5), margin, black, FillDraw(palegray));

			object kappa = draw("$\sum u_n$ converge", box, (5, -1.2), margin, cyan, FillDraw(palecyan));

			object zeta = draw("$\sum\,|u_n|$ converge ?", box, (5, -1.8), margin, black, FillDraw(palegray));

			object mu = draw("$\sum u_n$ converge", box, (7, -1.5), margin, cyan, FillDraw(palecyan));

			object eta = draw("\c{On forme un dévelop-\\pement asymptotique\\ de $u_n$}", box, (7.1,-2.1), 0, white, FillDraw(white));

			pair A = (7.8, -2.3);
			/*pair B = A + (0.3, -0.3);
			int n = 5;
			for(real t = 0; t < 1; t += 1/n) {
				pair M = (1-t) * A + t * B;
				fill(circle(M, 0.008), black);
			}*/
			label("$\ddots$", A);


			add(new void(frame f, transform t) {
					picture pic;
					draw(pic,"\sc non",point(alpha,E,t)--point(beta,W,t),RightSide,Arrow(TeXHead),PenMargin);
					draw(pic,"\sc oui",point(alpha,E,t)--point(gamma,W,t),RightSide,Arrow(TeXHead),PenMargin);

					draw(pic,"\sc oui",point(gamma,E,t)--point(delta,W,t),LeftSide,Arrow(TeXHead),PenMargin);
					draw(pic,"\sc non",point(gamma,E,t)--point(xi,W,t),RightSide,Arrow(TeXHead),PenMargin);

					draw(pic,"",point(delta,E,t * shift((0, 0.2)))--point(sigma,W,t * shift((0, 0.05))),RightSide,Arrow(TeXHead),PenMargin);
					draw(pic,"",point(delta,E,t * shift((0, 0.05)))--point(tau,W,t * shift((0, 0.05))),RightSide,Arrow(TeXHead),PenMargin);
					draw(pic,"",point(delta,E,t * shift((-0.1, 0.3)))--point(epsilon,SW,t),RightSide,Arrow(TeXHead),PenMargin);

					draw(pic,"\sc non", point(sigma,W,t*shift((0, -0.05)))--point(delta,E,t*shift((0, 0.1))),LeftSide,Arrow(TeXHead),PenMargin);
					draw(pic,"\sc oui",point(sigma,E,t)--point(pie,W,t),RightSide,Arrow(TeXHead),PenMargin);


					draw(pic,"\sc non", point(tau,W,t*shift((0, -0.05)))--point(delta,E,t*shift((0, -0.05))),LeftSide,Arrow(TeXHead),PenMargin);
					draw(pic,"\sc oui",point(tau,E,t)--point(iota,W,t),RightSide,Arrow(TeXHead),PenMargin);

					draw(pic,"\sc oui",point(xi,E,t)--point(kappa,W,t),LeftSide,Arrow(TeXHead),PenMargin);
					draw(pic,"\sc non",point(xi,E,t)--point(zeta,W,t),RightSide,Arrow(TeXHead),PenMargin);

					draw(pic,"\sc oui",point(zeta,E,t)--point(mu,W,t),LeftSide,Arrow(TeXHead),PenMargin);
					draw(pic,"\sc non",point(zeta,E,t)--point(eta,W,t),RightSide,Arrow(TeXHead),NoMargin);

					add(f,pic.fit());
				});
		\end{asy}
	\end{figure}
\end{landscape}

\begin{exo}
	\begin{enumerate}
		\item $\forall n \in \N^*,\, u_n = \sqrt[n]{n+1} - \sqrt[n]{n}$,
		\item $\forall n \ge 2,\,u_n = u_n = \ln\left( 1 + \frac{(-1)^n}{n} \right)$,
		\item $\forall n \in \N^*,\, u_n = \sin \left( \frac{\pi}{2n} \right)$,
		\item $\forall n \in \N^*, u_n = \frac{1}{\sum_{k=1}^n \frac{1}{k^\alpha}}$.
	\end{enumerate}

	\begin{enumerate}
		\item Pour $n \in \N^*$, \[
				f_n : x \mapsto x^{\frac{1}{n}} = e^{\frac{1}{n} \ln x}
			\] donc \[
				\forall x > 0,  f'_n(x)  =  \frac{1}{nx} e^{\frac{1}{n}\ln(x)}  = \frac{1}{nx} e^{\frac{1}{n}\ln x}.
			\]
			D'après le théorème des accroissements finis, \[
				\exists c_n \in [n, n+1], f_n(n+1) - f(n) = f'_n(c_n)
			\] donc \[
				u_n = \frac{1}{c_n}e^{\frac{1}{n} \ln c_n}.
			\]

			\[
				\forall n, 1 \le \frac{c_n}{n} \le 1 + \frac{1}{n}
			\] donc $c_n \simi_{n\to +\infty} n$ donc $c_n = n + \po(n)$ :
			\begin{align*}
				\ln c_n &= \ln\big(n + \po(n)\big) \\
				&= \ln\Big(n\big(1 + \po(1)\big)\Big) \\
				&= \ln n + \ln\big(1 + \po(1)\big) \\
				&= \ln(n) + \po(1) \\
				&\sim \ln n
			\end{align*}
			donc $\frac{\ln c_n}{n} \sim \frac{\ln n}{n} \tendsto{n\to +\infty} 0$\\
			donc $e^{\frac{1}{n} \ln c_n} \simi_{n\to +\infty} 1$\\
			donc $u_n \sim \frac{1}{n^2}$\\
			donc $\Sigma u_n$ converve.
		\item \[
				\forall n \ge 2, u_n = \frac{(-1)^n}{n} + O\left( \frac{1}{n^2} \right)
			\] $\sum \frac{(-1)^n}{n}$ converge et $\sum O\left( \frac{1}{n^2} \right)$ converge absolument. Donc, $\Sigma u_n$ converge.
		\item $u_n \simi_{n\to +\infty} \frac{\pi}{2n} \ge 0$ donc $\Sigma u_n$ diverge.
		\item \[
				\lim_{n\to +\infty} \sum_{k=1}^n \frac{1}{k^\alpha} = \begin{cases}
					\underbrace{\zeta(\alpha)}_{>0} &\text{ si } \alpha > 1,\\
					+\infty &\text{ si } \alpha \le 1,
				\end{cases}
			\] donc \[
				u_n \longrightarrow \begin{cases}
					\frac{1}{\zeta(\alpha)} \neq 0 &\text{ si } \alpha > 1,\\
					0 &\text{ si } \alpha \le 1
				\end{cases}
			\]
			Si $\alpha > 1$, alors $\Sigma u_n$ diverge. On suppose $\alpha \le 1$.

			{\itshape Équivalent de $\sum_{k=1}^n \frac{1}{k^\alpha} ?$}

			Si $\alpha < 1$, \[
				\underbrace{\frac{1}{1-\alpha} \left( (n+1)^{1-\alpha} - 2^{1-\alpha} \right)}_{\frac{n^{1-\alpha}}{1-\alpha}}
				= \int_{2}^{n+1} \frac{1}{x^{\alpha}}~\mathrm{d}x
				\le \sum_{k=2}^n \frac{1}{k^\alpha}
				\le \int_{1}^{n} \frac{1}{x^\alpha}~\mathrm{d}x
				= \underbrace{\frac{1}{1-\alpha} \left( n^{1-\alpha} - 1 \right)}_
				{\sim \frac{n^{1-\alpha}}{1-\alpha}}
			\] donc \[
				\sum_{k=1}^n \frac{1}{k^\alpha} \sim \frac{n^{1-\alpha}}{1-\alpha}
			\] donc \[
				u_n \sim \frac{1-\alpha}{n^{1-\alpha}}
			\]
			\begin{align*}
				\Sigma u_n \text{ converge } \iff& 1 - \alpha > 1\\
				\iff& \alpha < 0
			\end{align*}

			Si $\alpha = 1$, $\sum_{k=1}^n \frac{1}{k} \sim \ln n$ et donc $u_n \sim \frac{1}{\ln n}$.

			A-t-on $u_n = \po\left( \frac{1}{n^\beta} \right)$ avec $\beta>1$ ou $\frac{1}{n^\beta} = \po(u_n)$ avec $\beta \le 1$ ?

			Si $\beta \ge 0$, \[
				n^\beta u_n \sim \frac{n^\beta}{\ln(n)} \longrightarrow +\infty
			\] En particulier avec $\beta = \frac{1}{2}$ : \[
				\frac{1}{n^{\frac{1}{2}}} = \po(u_n)
			\] et $\sum \frac{1}{n^{\frac{1}{2}}}$ diverge car $\frac{1}{2} < 1$. Donc, $\Sigma u_n$ diverge.

			On a donc 
			\begin{itemize}
				\item avec $\alpha \le 0$, $\Sigma u_n$ converge,
				\item avec $\alpha > 0$, $\Sigma u_n$ diverge.
			\end{itemize}
	\end{enumerate}
\end{exo}
