\documentclass[a4paper]{report}
\let\mmathcal\mathcal

\usepackage[utf8]{inputenc}
\usepackage[T1]{fontenc}
\usepackage{textcomp}
\usepackage[bookmarks]{hyperref}
\usepackage[french]{babel}
\usepackage{amsmath, amssymb}
\usepackage{amsthm}
\usepackage{tikz}
\usepackage{pgfplots}
\usepackage{mathtools}
\usepackage{tkz-tab}
\usepackage[inline]{asymptote}
\usepackage{frcursive}
\usepackage{verbatim}
\usepackage{moresize}
\usepackage{algorithm}
\usepackage{algpseudocode}
\usepackage{pifont}
\usepackage{calligra}
\usepackage{thmtools}
\usepackage{diagbox}
\usepackage{centernot}
\usepackage{multicol}
\usepackage{nicematrix}
\usepackage{stmaryrd}
\usepackage{setspace}
\usepackage{chngpage}
\usepackage{cancel}
\usepackage{esvect}
\usepackage{wrapfig}
\usepackage{floatflt}
\usepackage{calligra}
\usepackage[cuteinductors,european,straightvoltages,europeanresistors]{circuitikz}
\usepackage{cellspace}
\usepackage{dsfont}
\usepackage{subcaption}
\usepackage{pdflscape}
\usepackage{contour}
\usepackage{soulutf8}

\frenchspacing
\reversemarginpar

% better underline
\setuldepth{a}
\contourlength{0.8pt}

\let\mathbbm\mathds

\setlength\cellspacetoplimit{4pt}
\setlength\cellspacebottomlimit{4pt}
\newcolumntype{D}{>{$}Sr<{$}}

\usetikzlibrary{babel}
\usetikzlibrary{tikzmark,calc,fit,arrows}

\newif\ifsimple
\newif\iffull
\simplefalse\fullfalse
\let\underline\ul
\let\underlin\underline

\usepackage{graphicx}
\newcommand\longvdots[1]{\raisebox{1em}{\rotatebox{-90}{\hbox to #1 {\dotfill}}}}

\usepackage[framemethod=TikZ]{mdframed}
\theoremstyle{definition}
\makeatletter

\pgfplotsset{compat=1.17}
\let\vec\vv

\definecolor{green}{HTML}{60A917}

\def\asydir{asy}

\newcommand{\cwd}{.}

% figure support
\usepackage{import}
\usepackage{xifthen}
\pdfminorversion=7
\usepackage{pdfpages}
\usepackage{transparent}
\newcommand{\incfig}[1]{%
	\def\svgwidth{\columnwidth}
	\import{\cwd/figures/}{#1.pdf_tex}
}

\newcommand{\mathnode}[2]{%
  \mathord{\tikz[baseline=(#1.base), inner sep = 0pt]{\node (#1) {$#2$};}}}

\usepackage{calrsfs}
\usepackage{mathrsfs}
\usepackage{stmaryrd}
\usepackage{float}
\usepackage{tikz-cd}
\usepackage{thmtools}
\usepackage{thm-restate}
\usepackage{etoolbox}

\setlength{\parindent}{0em}
\setlength{\parskip}{0em}

\let\oldemptyset\emptyset
\let\emptyset\varnothing

\let\ge\geqslant
\let\le\leqslant

\newcommand{\C}{\mathbbm{C}}
\newcommand{\R}{\mathbbm{R}}
\newcommand{\Z}{\mathbbm{Z}}
\newcommand{\N}{\mathbbm{N}}
\newcommand{\Q}{\mathbbm{Q}}
\renewcommand{\O}{\emptyset}

\renewcommand\Re{\expandafter\mathfrak{Re}}
\renewcommand\Im{\expandafter\mathfrak{Im}}

\renewcommand{\thepart}{\Roman{part}} 
\newcommand{\centered}[1]{\begin{center}#1\end{center}}

\DeclareMathOperator{\Arctan}{Arctan}
\DeclareMathOperator{\Card}{Card}
\DeclareMathOperator{\Ker}{Ker}
\DeclareMathOperator{\Aut}{Aut}
\DeclareMathOperator{\id}{id}
\DeclareMathOperator{\rg}{rg}
\DeclareMathOperator{\rk}{rk}
\DeclareMathOperator{\argmax}{argmax}
\DeclareMathOperator{\argmin}{argmin}
\DeclareMathOperator{\Vect}{Vect}
\DeclareMathOperator{\cotan}{cotan}
\DeclareMathOperator{\Mat}{Mat}
\DeclareMathOperator{\tr}{tr}
\DeclareMathOperator{\Cov}{Cov}
\DeclareMathOperator{\Supp}{Supp}
\DeclareMathOperator{\Cl}{\mathcal{C}\!\ell}
\DeclareMathOperator*{\po}{\text{\cursive o}}
\DeclareMathOperator*{\dom}{dom}
\DeclareMathOperator*{\codim}{codim}
\DeclareMathOperator*{\simi}{\sim}

\pdfsuppresswarningpagegroup=1

\newcommand{\emptyenv}[2][{}] {
	\newenvironment{#2}[1][{}] {
		\vspace{-16pt}
		#1
		\vspace{16pt}
		\expandafter\noindent\comment
	}{
		\expandafter\noindent\endcomment
	}
}

\mdfsetup{skipabove=1em,skipbelow=1em, innertopmargin=6pt, innerbottommargin=6pt,}

\declaretheoremstyle[
	mdframed={ },
	headpunct={:},
	numbered=no,
	headfont=\normalfont\bfseries,
	bodyfont=\normalfont,
	postheadspace=1em]{defnstyle}

\declaretheoremstyle[
	mdframed={
				rightline=false, topline=false, bottomline=false,
		innerlinewidth=0.4pt,outerlinewidth=0.4pt,
		middlelinewidth=2pt,
		linecolor=black,middlelinecolor=white,
	},
	headpunct={:},
	numbered=no,
	headfont=\normalfont\bfseries,
	bodyfont=\normalfont,
	postheadspace=1em]{thmstyle}

\declaretheoremstyle[
	headpunct={:},
	postheadspace=\newline,
	numbered=no,
	headfont=\normalfont\scshape]{rmkstyle}

\declaretheoremstyle[
	headfont=\normalfont\itshape,
	numbered=no,
	postheadspace=\newline,
	mdframed={ rightline=false, topline=false, bottomline=false },
	headpunct={:},
	qed=\qedsymbol]{prvstyle}

\declaretheorem[style=defnstyle, name=Définition]{defn}
\declaretheorem[style=defnstyle, name=Proposition\\Définition]{prop-defn}

% \declaretheorem[style=plain, thmbox={style=M, bodystyle=\normalfont}, name=Théorème]{thm}
% \declaretheorem[style=plain, thmbox={style=M, bodystyle=\normalfont}, name=Proposition]{prop}
% \declaretheorem[style=plain, thmbox={style=M, bodystyle=\normalfont}, name=Corollaire]{crlr}
% \declaretheorem[style=plain, thmbox={style=M, bodystyle=\normalfont}, name=Lemme]{lem}

\declaretheorem[style=thmstyle, name=Théorème]{thm}
\declaretheorem[style=thmstyle, name=Proposition]{prop}
\declaretheorem[style=thmstyle, name=Corollaire]{crlr}
\declaretheorem[style=thmstyle, name=Lemme]{lem}

\declaretheorem[style=rmkstyle, name=Remarque]{rmk}
\declaretheorem[style=rmkstyle, name=Rappel]{rap}

\AtBeginDocument{
	\ifsimple
		\emptyenv{exm}
		\emptyenv{exo}
		\emptyenv[\hfill$\blacksquare$]{prv}
	\else
		\declaretheorem[style=rmkstyle, name=Exemple]{exm}
		\declaretheorem[style=rmkstyle, name=Exercice]{exo}
		\declaretheorem[style=prvstyle, name=Preuve]{prv}
	\fi
}

\makeatother
\usepackage{fancyhdr}
\pagestyle{fancy}

\fancyhead[R]{}
\fancyhead[L]{\thepart}
\fancyhead[C]{\parttitle}

\fancyfoot[R]{\thepage}
\fancyfoot[L]{}
\fancyfoot[C]{}

\newcommand*\parttitle{}
\let\origpart\part
\renewcommand*{\part}[2][]{%
   \ifx\\#1\\% optional argument not present?
      \origpart{#2}%
      \renewcommand*\parttitle{#2}%
   \else
      \origpart[#1]{#2}%
      \renewcommand*\parttitle{#1}%
   \fi
}

\makeatletter

\newcommand{\tendsto}[1]{\xrightarrow[#1]{}}
\newcommand{\danger}{{\large\fontencoding{U}\fontfamily{futs}\selectfont\char 66\relax}\;}
\newcommand{\ex}{\fbox{ex}\;}
\renewcommand{\mod}[1]{~\left[ #1 \right]}
\newcommand{\todo}[1]{{\color{blue} À faire : #1}}
\newcommand*{\raisesign}[2][.7\normalbaselineskip]{\smash{\llap{\raisebox{#1}{$#2$\hspace{2\arraycolsep}}}}}
\newcommand{\vrt}[1]{\rotatebox{90}{$#1$}}

\DeclareMathOperator{\ou}{\text{ ou }}
\DeclareMathOperator{\et}{\text{ et }}
\DeclareMathOperator{\si}{\text{ si }}
\DeclareMathOperator{\non}{\text{ non }}

\renewcommand{\title}[2]{
	\AtBeginDocument{
		\begin{titlepage}
			\begin{center}
				\vspace{10cm}
				{\Large \sc Chapitre #1}\\
				\vspace{1cm}
				{\HUGE \cursive #2}\\
				\vfill
				Hugo {\sc Salou} MP2I\\
				{\ssmall Dernière mise à jour le \@date }
			\end{center}
		\end{titlepage}
	}
}

\let\bx\boxed
\newcommand{\s}{\text{\cursive s}}
\renewcommand{\t}{{}^t\!}
\newcommand{\eme}{\ensuremath{{}^{\text{ème}}}~}
%\let\oldfract\fract
%\renewcommand{\fract}[2]{\oldfract{\displaystyle #1}{\displaystyle #2}}
% \let\textstyle\displaystyle
% \let\scriptstyle\displaystyle
% \let\scriptscriptstyle\displaystyle
\everymath{\displaystyle}


\makeatletter
\def\moverlay{\mathpalette\mov@rlay}
\def\mov@rlay#1#2{\leavevmode\vtop{%
   \baselineskip\z@skip \lineskiplimit-\maxdimen
   \ialign{\hfil$\m@th#1##$\hfil\cr#2\crcr}}}
\newcommand{\charfusion}[3][\mathord]{
    #1{\ifx#1\mathop\vphantom{#2}\fi
        \mathpalette\mov@rlay{#2\cr#3}
      }
    \ifx#1\mathop\expandafter\displaylimits\fi}
\makeatother

\newcommand{\cupdot}{\charfusion[\mathbin]{\cup}{\cdot}}
\newcommand{\bigcupdot}{\charfusion[\mathop]{\bigcup}{\cdot}}
\newcommand{\plusbar}{\charfusion[\mathbin]{+}{\color{blue}/}}


\title{??}{\'Energie}{Hugo \sc Salou}

\begin{document}
	\begin{align*}
		\text{PFD: } \bigg( m \frac{d \vec{v}}{dt} &= \vec{F} \bigg) \cdot \vec{v}\\
		m\frac{d}{dt}\left[ \frac{1}{2} \|\vec{v}\|^2 \right] &= \mathcal{P}(\vec{F})\\
		\frac{d \mathcal{E}_c}{dt} = \mathcal{P}(\vec{F})
	\end{align*}

	force $\vec{F}$ motrice: $\mathcal{P}(\vec{F}) > 0 \iff \frac{d \mathcal{E}_c}{dt} > 0$ \\

	\begin{align*}
		\int_{t_1}^{t_2} \frac{d \mathcal{E}_c}{dt} \; dt &= \int_{t_1}^{t_2} \mathcal{P}(\vec{F}) \;dt\\
		&= \int_{M_1 \to M_2} \delta W(\vec{F})  \\
		&= W_{M_1M_2}(\vec{F}) \\
	\end{align*}
	Or,  \[
		\int_{t_1}^{t_2} \frac{d \mathcal{E}_c}{dt}\; dt = \left[ \mathcal{E}_c \right]_{t_1}^{t_2} = \mathcal{E}_c(t_2) - \mathcal{E}_c(t_1) = \Delta \mathcal{E}_c
	\] Donc, \[
		\Delta \mathcal{E}_c = W_{M_1M_2}(\vec{F})
	\]

	\begin{center}
		\begin{asy}
			size(5cm);

			pair pol(real a, real r) {return (cos(a) * -r, sin(a) * -r);}

			pair O = (0,0);
			real t = 5 * pi / 7;
			real l = 3;

			pair M = pol(t, l);

			dot(O);
			dot("$M$", M);
			draw(O -- M, black + 1);
			draw(O -- (0, -l), dashed);

			draw((0, -1) .. pol(t, 1), EndArrow);
			label("$\theta$", ((0, -1) + pol(t, 1)) / 2 + (0,-0.2));

			draw(M -- M + (0, -1), red, EndArrow);
			label("$\vec{P}$", M + (-0.3,-0.4), red);

			draw(M -- 2.5 * M / 4 , deepcyan, EndArrow);
			label("$\vec{T}$", (6.5/8) * M + (-0.3, 0), deepcyan);

			draw(M -- M + M / 3, purple, EndArrow);
			label("$\vec{u_{r}}$", 3.5/3 * M + (0.2,0),purple);

			draw(M -- M + pol(t + pi/2, l/3), pink, EndArrow);
			label("$\vec{u_{\theta}}$", M + pol(t + pi/2, l/6) + (0, 0.2), pink);

			draw(M + (0, -0.5) .. (3.5/3) * M, EndArrow);
			label("$\theta$", M + ((0, -1) + pol(t, 1)) / 4 + (0,-0.2));
		\end{asy}
	\end{center}

	\begin{align*}
		\mathcal{P}(m \vec{g}) &= \left(mg\cos\theta \vec{u_r} - mg\sin\theta \vec{u_{\theta}}\right) \cdot \ell\dot{\theta} \vec{u_\theta} \\
		&= -mg\ell\dot{\theta}\sin\theta \\
	\end{align*}

	\[
		\mathcal{E}_c = \frac{1}{2}mv^2 = \frac{1}{2} m\ell^2 \dot{\theta}^2
	\] \[
		\frac{d \mathcal{E}_c}{dt} = \mathcal{P}(m \vec{g}) + \underbrace{\mathcal{P}(\vec{T})}_{= 0}
	\] Donc \[
	\frac{1}{2}m\ell^2 \times 2 \dot{\theta} \ddot{\theta} = -mg\ell\dot{\theta}\sin(\theta)
	\] \[
		\dot{\theta} \neq 0: \qquad \fbox{$\ddot{\theta} + \frac{g}{\ell}\sin\theta = 0$}
	\]

	\begin{align*}
		\mathcal{E}_c(M_2) - \mathcal{E}_c(M_1) &= W_{M_1M_2}(\vec{F}) \\
		\frac{1}{2}mv_2^2 - \frac{1}{2} m v_1^2 &= W_{M_1M_2}(m \vec{g}) + \underbrace{W_{M_1M_2}(\vec{T})}_{ = 0}\\
		W_{M_1M_2}(m \vec{g}) &= mg(z_1-z_2)\\
		v_2^2 &= v_1^2 + 2g(z_1- z_2)
	\end{align*}

	\[
		\frac{d \mathcal{E}_p}{dx} dx = -f dx \implies f = -\frac{d \mathcal{E}_p}{dx}
	\] La force dérive l'énergie potentielle.
	\[
		\int_{M_0}^{M} d \mathcal{E}_p = -\int_{M_0}^{}f\; dx
	\]
	On a $\begin{cases}
		M_0: x_0\\
		M: x
	\end{cases}$ \\
	\[
		\mathcal{E}_p(M) - \mathcal{E}_p(M_0) = -(x - x_0)
	\] \[
		\vec{f} = -k \vec{u_r}
	\] 

	\begin{asy}
		import spring;

		size(3cm);

		real r = 8;
		real t1 = -pi;
		real t2 = 10*pi;

		real lambda = (t2 - t1 + 1) / (t2 - t1);

		pair O = coilpoint(lambda,r,t1) - 20;
		pair M = coilpoint(lambda,r,t2) + 20;

		drawspring(1, "$r$");

		dot("$O$", O, left);
		dot("$M$", M);
	\end{asy}

	\[
		\delta W = \vec{f} \cdot \mathrm{d} \vec{OM} = \vec{f} \cdot \left(\mathrm{d}r \; \vec{u_r} + r \mathrm{d}\theta \; \vec{u_\theta} + \mathrm{d}z \; \vec{u_z} \right) = -kr\; \mathrm{d}r
	\] 

	\begin{align*}
		\mathrm{d}\mathcal{E}_p &= -\delta W = kr\mathrm{d}r\\
		\int_{M_0}^{M} \mathrm{d}\mathcal{E}_p &= \mathcal{E}_p(M) - \mathcal{E}_p(M_0) \\
		&= \int_{r_0}^{r} kr\;\mathrm{d}r = \frac{k}{2}(r^2 - r_0^2)  \\
	\end{align*}
\end{document}
