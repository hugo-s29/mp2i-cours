\begin{defn}
	Une \underline{équation différentielle} est une égalité faisant intervenir une fonction inconnue $y$ ainsi que ses dérivées successives $y', y'', y^{(3)}, \ldots, y^{(n)}$.
\end{defn}

\begin{exm}
	\begin{enumerate}
		\item $y^{(3)} + \ln\left( y' \right) = e^{y}$ 
		\item ~\\
			\begin{minipage}
				{\linewidth}
				\begin{wrapfigure}
					{l}{0.3\linewidth}
					\vspace{-1.5cm}
					\centering
					\begin{asy}
						size(4cm);
						real theta = 30;
						real r = 5;

						pair M = r * dir(theta) * (-I);
						pair O = (0,0);

						dot(O);
						fill(circle(M, 0.2));
						draw(O--M);

						pair B = (r+0.3) * (-I);
						draw(O--B, gray);

						draw(B-(0.5, 0)--B+(0.55, 0), gray);
						for(real x = -0.5; x <= 0.55; x += 0.15) {
							pair A = B + (x, 0);
							pair A2 = A + (-0.1, -0.3);
							draw(A--A2, gray);
						}

						draw(arc(O, r/3, 270, 270+theta), deepcyan, Arrow(TeXHead));
						label("$\theta$", dir(270+theta/2) * r/3, deepcyan, align=S + E/2);
					\end{asy}
				\end{wrapfigure}

				\vspace{1cm}
				On a $\ddot\theta + \sin(\theta) = 0$ i.e. $\frac{d^2\theta}{dt^2} + \sin(\theta) = 0$\\
				Pour les ``petits angles'', $\sin(\theta) \simeq 0$. On résout donc \[
					\ddot\theta = -\theta
				\] 
				\vspace{1cm}
			\end{minipage}
		\item
			\begin{minipage}
				{\linewidth}
				\begin{wrapfigure}
					{r}{0.3\linewidth}
					\vspace{-1cm}
					\begin{circuitikz}
						\draw (0,0) to[american, isource, l=$e(t)$] (0,2) to[R=$R$] (3,2) to[C=$C$] (3,0) to[L=$L$] (0,0);
					\end{circuitikz}
				\end{wrapfigure}
				\begin{align*}
					L\frac{d^2q}{dt^2} + R \frac{dq}{dt} + \frac{1}{C} q = e(t)\\
					L\ddot q + R \dot q + \frac{1}{C}q = e(t)
				\end{align*}
			\end{minipage}
		\item Modèle de population : $\frac{dN}{dt} = \lambda N (1-N)$
	\end{enumerate}
\end{exm}

\begin{defn}
	Une \underline{équation différentielle linéaire d'ordre $n$} est de la forme \[
		a_n y^{(n)} + a_{n-1} y^{(n-1)} + \cdots + a_0y = b(t)
	\] où $b, a_0, a_1, \ldots, a_n$ sont des fonctions connues et continues sur un intervalle $I$. On dit que $b$ est le \underline{second membre} de l'équation.
\end{defn}

\begin{exm}
	[$\cos(t)y'' + \sin(t)y' = \tan(t)$]
\end{exm}

\begin{prop}
	[Principe de superposition]
	Soient $b_1$ et $b_2$ continues sur $I$. Soient $a_0, a_1, \ldots, a_n$ également continues sur $I$.
	\begin{align*}
		(E_1):&\quad\sum_{k=0}^{n} a_k y^{(k)} = b_1\\
		(E_2):&\quad\sum_{k=0}^{n} a_k y^{(k)} = b_2\\
	\end{align*}
	Soient $(\lambda_1, \lambda_2) \in \C^2$. \[
		(E): \quad \sum_{k=1}^{n} a_k y^{(k)} = \lambda_1b_1 + \lambda_2b_2
	\] 
	\[
		\begin{rcases*}
			y_1 \text{ solution de } (E_1)\\
			y_2 \text{ solution de } (E_2)\\
		\end{rcases*} \implies \lambda_1 y_1 + \lambda_2 y_2 \text{ solution de } (E)
	\] 
\end{prop}

\begin{prv}
	On pose $y = \lambda_1 y_1 + \lambda_2 y_2$ dérivable $n$ fois car c'est le cas de $y_1$ et $y_2$
	Donc, \[
		\forall k \in \left\llbracket 0, n \right\rrbracket, y^{(k)} = \lambda_1 y_1^{(k)} + \lambda_2 y_2^{(k)}
	\]
	D'où,
	\begin{align*}
		\sum_{k=0}^{n} a_ky^{(k)}
		&= \sum_{k=0}^{n} a_k \left( \lambda_1 y_1^{(k)} + \lambda_2 y_2^{(k)} \right) \\
		&= \lambda_1 \sum_{k=1}^{n} a_ky_1^{(k)} + \lambda_2 \sum_{k=1}^{n} a_ky_2^{(k)} \\
		&= \lambda_1 b_1 + \lambda_2 b_2 \\
	\end{align*}
\end{prv}

\begin{prop}
	Soit $(E)$ l'équation différentielle $\sum_{k=0}^{n} a_k y^{(k)} = b$ où $a_0, a_1, \ldots, a_n$ sont des fonctions homogènes. L'\underline{équation homogène} associée à $(E)$ est \[
		(H): \sum_{k=0}^{n} a_k y^{(k)} = 0
	\] 
	Les solutions de $(E)$ sont toutes de la forme $h + y_0$ où $h$ est solution de $(H)$ et $y_0$ solution de $(E)$.
\end{prop}

\begin{prv}
	Soit $y$ une solution de $(E)$ et $y_0$ une solution particulière de $(E)$. On pose $h = y - y_0$.\\
	D'après le principe de superposition, $h$ est une solution de $(H)$.\\
	Réciproquement, si  $h$ est une solution de $(H)$ et $y_0$ une solution de $(E)$ alors $h+y_0$ est aussi solution de $(E)$.
\end{prv}

\begin{thm}
	[Théorème de Cauchy]
	Soit $(E)$ une équation linéaire différentielle. \[
		(E): \quad y^{(n)} + \sum_{k=0}^{n-1} a_k y^{(k)} = b
	\] où $a_0, a_1, \ldots, a_n$ sont \underline{continues} sur un \underline{intervalle} $I$.\\
	Soit $t_0 \in I$ et $\left( \alpha_0, \alpha_1, \ldots, \alpha_n \right) \in \C^{n}$.\\
	Il existe {\bf une et une seule} fonction $y$ telle que \[
		\begin{cases}
			y \text{ solution de } (E)\\
			\forall i \in \left\llbracket 0,n-1 \right\rrbracket, y^{(i)}(t_0) = \alpha_i
		\end{cases}
	\] 
\end{thm}

\begin{exm}
	\begin{minipage}
		{\linewidth}
		\begin{wrapfigure}
			{l}{0.3\linewidth}
			\centering
			\begin{asy}
				import solids;
				settings.render=0;
				settings.prc=false;

				size(4cm);
				revolution r = cylinder(O, 0.5, 1, Z);

				draw(r,1);

				draw(circle((0,0,0), 0.5), cyan);
				draw(circle((0,0,0.5), 0.5), cyan);

				draw(r.silhouette());

				draw(circle((0,0), 0.05));
				draw((0,0.01)--(0,-0.4), cyan, Arrow(TeXHead));
			\end{asy}
		\end{wrapfigure}
		On ne peut pas déduire le passé d'un sceau percé, son équation est non-linéaire. \[
			h' = -c\sqrt{h} \text{ avec } c \in \R_*^+
		\] 
	\end{minipage}
\end{exm}
