\documentclass[a4paper]{report}
\let\mmathcal\mathcal

\usepackage[utf8]{inputenc}
\usepackage[T1]{fontenc}
\usepackage{textcomp}
\usepackage[bookmarks]{hyperref}
\usepackage[french]{babel}
\usepackage{amsmath, amssymb}
\usepackage{amsthm}
\usepackage{tikz}
\usepackage{pgfplots}
\usepackage{mathtools}
\usepackage{tkz-tab}
\usepackage[inline]{asymptote}
\usepackage{frcursive}
\usepackage{verbatim}
\usepackage{moresize}
\usepackage{algorithm}
\usepackage{algpseudocode}
\usepackage{pifont}
\usepackage{calligra}
\usepackage{thmtools}
\usepackage{diagbox}
\usepackage{centernot}
\usepackage{multicol}
\usepackage{nicematrix}
\usepackage{stmaryrd}
\usepackage{setspace}
\usepackage{chngpage}
\usepackage{cancel}
\usepackage{esvect}
\usepackage{wrapfig}
\usepackage{floatflt}
\usepackage{calligra}
\usepackage[cuteinductors,european,straightvoltages,europeanresistors]{circuitikz}
\usepackage{cellspace}
\usepackage{dsfont}
\usepackage{subcaption}
\usepackage{pdflscape}
\usepackage{contour}
\usepackage{soulutf8}

\frenchspacing
\reversemarginpar

% better underline
\setuldepth{a}
\contourlength{0.8pt}

\let\mathbbm\mathds

\setlength\cellspacetoplimit{4pt}
\setlength\cellspacebottomlimit{4pt}
\newcolumntype{D}{>{$}Sr<{$}}

\usetikzlibrary{babel}
\usetikzlibrary{tikzmark,calc,fit,arrows}

\newif\ifsimple
\newif\iffull
\simplefalse\fullfalse
\let\underline\ul
\let\underlin\underline

\usepackage{graphicx}
\newcommand\longvdots[1]{\raisebox{1em}{\rotatebox{-90}{\hbox to #1 {\dotfill}}}}

\usepackage[framemethod=TikZ]{mdframed}
\theoremstyle{definition}
\makeatletter

\pgfplotsset{compat=1.17}
\let\vec\vv

\definecolor{green}{HTML}{60A917}

\def\asydir{asy}

\newcommand{\cwd}{.}

% figure support
\usepackage{import}
\usepackage{xifthen}
\pdfminorversion=7
\usepackage{pdfpages}
\usepackage{transparent}
\newcommand{\incfig}[1]{%
	\def\svgwidth{\columnwidth}
	\import{\cwd/figures/}{#1.pdf_tex}
}

\newcommand{\mathnode}[2]{%
  \mathord{\tikz[baseline=(#1.base), inner sep = 0pt]{\node (#1) {$#2$};}}}

\usepackage{calrsfs}
\usepackage{mathrsfs}
\usepackage{stmaryrd}
\usepackage{float}
\usepackage{tikz-cd}
\usepackage{thmtools}
\usepackage{thm-restate}
\usepackage{etoolbox}

\setlength{\parindent}{0em}
\setlength{\parskip}{0em}

\let\oldemptyset\emptyset
\let\emptyset\varnothing

\let\ge\geqslant
\let\le\leqslant

\newcommand{\C}{\mathbbm{C}}
\newcommand{\R}{\mathbbm{R}}
\newcommand{\Z}{\mathbbm{Z}}
\newcommand{\N}{\mathbbm{N}}
\newcommand{\Q}{\mathbbm{Q}}
\renewcommand{\O}{\emptyset}

\renewcommand\Re{\expandafter\mathfrak{Re}}
\renewcommand\Im{\expandafter\mathfrak{Im}}

\renewcommand{\thepart}{\Roman{part}} 
\newcommand{\centered}[1]{\begin{center}#1\end{center}}

\DeclareMathOperator{\Arctan}{Arctan}
\DeclareMathOperator{\Card}{Card}
\DeclareMathOperator{\Ker}{Ker}
\DeclareMathOperator{\Aut}{Aut}
\DeclareMathOperator{\id}{id}
\DeclareMathOperator{\rg}{rg}
\DeclareMathOperator{\rk}{rk}
\DeclareMathOperator{\argmax}{argmax}
\DeclareMathOperator{\argmin}{argmin}
\DeclareMathOperator{\Vect}{Vect}
\DeclareMathOperator{\cotan}{cotan}
\DeclareMathOperator{\Mat}{Mat}
\DeclareMathOperator{\tr}{tr}
\DeclareMathOperator{\Cov}{Cov}
\DeclareMathOperator{\Supp}{Supp}
\DeclareMathOperator{\Cl}{\mathcal{C}\!\ell}
\DeclareMathOperator*{\po}{\text{\cursive o}}
\DeclareMathOperator*{\dom}{dom}
\DeclareMathOperator*{\codim}{codim}
\DeclareMathOperator*{\simi}{\sim}

\pdfsuppresswarningpagegroup=1

\newcommand{\emptyenv}[2][{}] {
	\newenvironment{#2}[1][{}] {
		\vspace{-16pt}
		#1
		\vspace{16pt}
		\expandafter\noindent\comment
	}{
		\expandafter\noindent\endcomment
	}
}

\mdfsetup{skipabove=1em,skipbelow=1em, innertopmargin=6pt, innerbottommargin=6pt,}

\declaretheoremstyle[
	mdframed={ },
	headpunct={:},
	numbered=no,
	headfont=\normalfont\bfseries,
	bodyfont=\normalfont,
	postheadspace=1em]{defnstyle}

\declaretheoremstyle[
	mdframed={
				rightline=false, topline=false, bottomline=false,
		innerlinewidth=0.4pt,outerlinewidth=0.4pt,
		middlelinewidth=2pt,
		linecolor=black,middlelinecolor=white,
	},
	headpunct={:},
	numbered=no,
	headfont=\normalfont\bfseries,
	bodyfont=\normalfont,
	postheadspace=1em]{thmstyle}

\declaretheoremstyle[
	headpunct={:},
	postheadspace=\newline,
	numbered=no,
	headfont=\normalfont\scshape]{rmkstyle}

\declaretheoremstyle[
	headfont=\normalfont\itshape,
	numbered=no,
	postheadspace=\newline,
	mdframed={ rightline=false, topline=false, bottomline=false },
	headpunct={:},
	qed=\qedsymbol]{prvstyle}

\declaretheorem[style=defnstyle, name=Définition]{defn}
\declaretheorem[style=defnstyle, name=Proposition\\Définition]{prop-defn}

% \declaretheorem[style=plain, thmbox={style=M, bodystyle=\normalfont}, name=Théorème]{thm}
% \declaretheorem[style=plain, thmbox={style=M, bodystyle=\normalfont}, name=Proposition]{prop}
% \declaretheorem[style=plain, thmbox={style=M, bodystyle=\normalfont}, name=Corollaire]{crlr}
% \declaretheorem[style=plain, thmbox={style=M, bodystyle=\normalfont}, name=Lemme]{lem}

\declaretheorem[style=thmstyle, name=Théorème]{thm}
\declaretheorem[style=thmstyle, name=Proposition]{prop}
\declaretheorem[style=thmstyle, name=Corollaire]{crlr}
\declaretheorem[style=thmstyle, name=Lemme]{lem}

\declaretheorem[style=rmkstyle, name=Remarque]{rmk}
\declaretheorem[style=rmkstyle, name=Rappel]{rap}

\AtBeginDocument{
	\ifsimple
		\emptyenv{exm}
		\emptyenv{exo}
		\emptyenv[\hfill$\blacksquare$]{prv}
	\else
		\declaretheorem[style=rmkstyle, name=Exemple]{exm}
		\declaretheorem[style=rmkstyle, name=Exercice]{exo}
		\declaretheorem[style=prvstyle, name=Preuve]{prv}
	\fi
}

\makeatother
\usepackage{fancyhdr}
\pagestyle{fancy}

\fancyhead[R]{}
\fancyhead[L]{\thepart}
\fancyhead[C]{\parttitle}

\fancyfoot[R]{\thepage}
\fancyfoot[L]{}
\fancyfoot[C]{}

\newcommand*\parttitle{}
\let\origpart\part
\renewcommand*{\part}[2][]{%
   \ifx\\#1\\% optional argument not present?
      \origpart{#2}%
      \renewcommand*\parttitle{#2}%
   \else
      \origpart[#1]{#2}%
      \renewcommand*\parttitle{#1}%
   \fi
}

\makeatletter

\newcommand{\tendsto}[1]{\xrightarrow[#1]{}}
\newcommand{\danger}{{\large\fontencoding{U}\fontfamily{futs}\selectfont\char 66\relax}\;}
\newcommand{\ex}{\fbox{ex}\;}
\renewcommand{\mod}[1]{~\left[ #1 \right]}
\newcommand{\todo}[1]{{\color{blue} À faire : #1}}
\newcommand*{\raisesign}[2][.7\normalbaselineskip]{\smash{\llap{\raisebox{#1}{$#2$\hspace{2\arraycolsep}}}}}
\newcommand{\vrt}[1]{\rotatebox{90}{$#1$}}

\DeclareMathOperator{\ou}{\text{ ou }}
\DeclareMathOperator{\et}{\text{ et }}
\DeclareMathOperator{\si}{\text{ si }}
\DeclareMathOperator{\non}{\text{ non }}

\renewcommand{\title}[2]{
	\AtBeginDocument{
		\begin{titlepage}
			\begin{center}
				\vspace{10cm}
				{\Large \sc Chapitre #1}\\
				\vspace{1cm}
				{\HUGE \cursive #2}\\
				\vfill
				Hugo {\sc Salou} MP2I\\
				{\ssmall Dernière mise à jour le \@date }
			\end{center}
		\end{titlepage}
	}
}

\let\bx\boxed
\newcommand{\s}{\text{\cursive s}}
\renewcommand{\t}{{}^t\!}
\newcommand{\eme}{\ensuremath{{}^{\text{ème}}}~}
%\let\oldfract\fract
%\renewcommand{\fract}[2]{\oldfract{\displaystyle #1}{\displaystyle #2}}
% \let\textstyle\displaystyle
% \let\scriptstyle\displaystyle
% \let\scriptscriptstyle\displaystyle
\everymath{\displaystyle}


\makeatletter
\def\moverlay{\mathpalette\mov@rlay}
\def\mov@rlay#1#2{\leavevmode\vtop{%
   \baselineskip\z@skip \lineskiplimit-\maxdimen
   \ialign{\hfil$\m@th#1##$\hfil\cr#2\crcr}}}
\newcommand{\charfusion}[3][\mathord]{
    #1{\ifx#1\mathop\vphantom{#2}\fi
        \mathpalette\mov@rlay{#2\cr#3}
      }
    \ifx#1\mathop\expandafter\displaylimits\fi}
\makeatother

\newcommand{\cupdot}{\charfusion[\mathbin]{\cup}{\cdot}}
\newcommand{\bigcupdot}{\charfusion[\mathop]{\bigcup}{\cdot}}
\newcommand{\plusbar}{\charfusion[\mathbin]{+}{\color{blue}/}}


\newcommand{\chap}[2][0]{
    \setcounter{chapter}{#1 - 1}
    \chapter{#2}
    \renewcommand*\parttitle{#2}
}

\renewcommand{\part}[1]{\section{#1}}
\AtBeginDocument{\fulltrue}

\makeatother

\usepackage{fancyhdr}
\pagestyle{fancy}

\fancyhead[R]{\itshape MP2I}
\fancyhead[L]{\thesection}
\fancyhead[C]{\parttitle}

\fancyfoot[C]{\thepage}
\fancyfoot[L]{}
\fancyfoot[R]{}

\makeatletter

\usepackage{pgfornament}

\begin{document}
    \begin{titlepage}
        \begin{center}
            \vspace{10cm}
            {\Large \itshape 2021-2022}\\
            \vspace{3cm}
            \pgfornament[width=8cm]{88}\\
            \vspace{2mm}
            \vspace{0.5cm}
            {\HUGE Mathématiques}\\
            \vspace{0.5cm}
            {\fontsize{240pt}{260pt}\selectfont MP2I}\\
            \vspace{0.5cm}
            \pgfornament[width=8cm]{88}\\
            \vfill
            Hugo {\sc Salou}\\
        \end{center}
    \end{titlepage}
    \tableofcontents


	\chap[00]{Logique (rudiments)}
	\renewcommand{\cwd}{../chap00}
	\part{Modes de définition}

\begin{defn}
	Une suite peut être définie
	\begin{itemize}
		\item \underline{Explicitement}
			On dispose pour tout $n \in \N$ de l'expression de $u_n$ en fonction de $n$.\\
			\ex $\forall n \in \N_*, u_n = \frac{\ln(n)}{n}e^{-n}$\\
		\item \underline{Par récurrence}
			On connait $u_{n+1}$ en fonction de  $u_0, u_1, \ldots, u_n$\\
			\ex $\begin{cases}
				u_0=1\\
				\forall n \in \N, u_{n+1} = \sin(u_n)
			\end{cases}$\\
		\item \underline{Implicitement}
			$\forall n \in \N, u_n$ est le seul nombre verifiant une certaine propriété\\
			\ex $u_n$ est le seul réel vérifiant  $x^5 + nx - 1 = 0$
	\end{itemize}
\end{defn}

	\part{Topologie de $\R^2$}

\begin{defn}
	La \underline{norme (euclidienne)} de $\R^2$ est l'application définie par \[
		\forall (x,y) \in \R^2, \|(x,y)\| = \sqrt{x^2 + y^2}.
	\]

	\begin{figure}[H]
		\centering
		\begin{asy}
			import graph;
			axes(EndArrow);
			size(4cm);
			pair A = (3,2);
			dot(A);
			draw((3,0)--A, dashed);
			draw((0,2)--A, dashed);
			label("$x$", (3,0), align=S);
			label("$y$", (0,2), align=W);
			draw((0,0)--A);
			dot((4,3), white+0);
		\end{asy}
	\end{figure}
	\index{norme (de $\R^2$)}
	\index{norme euclidienne (de $\R^2$)}
\end{defn}

\begin{prop}
	La norme euclidienne vérifie:
	\begin{enumerate}
		\item (séparation) \[
			\forall (x,y) \in \R^2, \|(x,y)\| = 0 \iff x = y = 0,
			\]
		\item (homogénéité positive) \[
				\forall \lambda \in \R, \forall (x,y) \in \R^2, \|\lambda(x,y)\|= \left| \lambda \right| \|(x,y)\|
			\]
		\item (inégalité triangulaire) \[
			\forall (x,y), (a,b) \in \R^2,
			\|(x,y)+(a,b)\|\le \|(x,y)\|+\|(a,b)\|.
		\]
	\end{enumerate}
\end{prop}

\begin{prv}
	Déjà vue en replaçant $(x,y)$ par $x+iy \in \C$ et $\|(x,y)\|$ par |x+iy|
\end{prv}

\begin{defn}
	Soit $(a,b) \in \R^2$ et $r \in \R_+$.

	La \underline{boule ouverte} (ou \underline{disque ouvert}) de centre $(a,b)$ et de rayon $r$ est \[
		B_{(a,b)}(r) = \big\{ (x,y) \in \R^2  \mid \|(x,y) - (a,b)\| < r \big\}.
	\]

	La \underline{boule fermée} (ou \underline{disque fermé}) de centre $(a,b)$ et de rayon $r$ est \[
		\overline{B_{(a,b)}}(r) = \big\{ (x,y)\in \R^2  \mid \|(x,y) - (a,b)\| \le r \big\}.
	\]

	La \underline{sphère} (ou \underline{boule}) de centre $(a,b)$ et de rayon $r$ est \[
		S_{(a,b)}(r) = \partial \overline{B_{(a,b)}}(r) = \big\{ (x,y) \in \R^2  \mid \|(x,y) - (a,b)\| = r \big\}.
	\]
	\index{boule ouverte (de $\R^2$)}
	\index{disque ouverte (de $\R^2$)}
	\index{boule fermée (de $\R^2$)}
	\index{disque fermée (de $\R^2$)}
	\index{boule (de $\R^2$)}
	\index{sphère (de $\R^2$)}
\end{defn}

\begin{figure}[H]
		\centering
		\incfig{boule}
\end{figure}

\begin{rmk}
	On parle de boule en dimension quelconque.
\end{rmk}

\begin{defn}
	Une \underline{partie ouverte} $O$ de $\R^2$ (ou \underline{un ouvert}) si \[
		\forall (x,y) \in O, \exists r > 0, B_{(a,b)}(r) \subset O.
	\]
	Une partie $F$ est \underline{fermée} su $\R^2\setminus F$ est ouverte.
	\index{partie ouverte (de $\R^2$)}
	\index{ouvert (de $\R^2$)}
	\index{partie fermée (de $\R^2$)}
\end{defn}

\begin{figure}[H]
	\centering
	\incfig{partie-ouverte}
\end{figure}

\begin{prop}
	Une boule ouverte est ouverte. Une boule fermée est fermée.
\end{prop}

\begin{figure}[H]
	\centering
	\begin{subfigure}{4cm}
		\centering
		\begin{asy}
			import patterns;

			pair n(pair a) {return a / length(a);}

			add("hatch",hatch(2mm, SW, red));
			size(4cm);

			draw(circle((0,0), 1));
			dot('$(a_0, b_0)$', (0,0), align=S);

			draw((0,0) -- n((-1, 1)), dashed);
			label("$r$", n((-1, 1)) / 2, align=1.5S);

			pair A = n((1,3)) * (2/3);
			real rho = (1 - length(A)) * (2 / 3);

			dot("$(a,b)$", A, red, align=3SE);
			filldraw(circle(A, rho), pattern("hatch"), red);

			label("$O$", n((1,-1))*2.5/3);
		\end{asy}
	\end{subfigure}
	\begin{subfigure}{1cm}
		\centering~\\
	\end{subfigure}
	\begin{subfigure}{5cm}
		\centering
		\begin{asy}
			import patterns;

			pair n(pair a) {return a / length(a);}

			add("hatch",hatch(1mm, SW, red));
			add("hatch2",hatch(3mm, SE, blue));
			size(5cm);

			guide around = (-1.5, -1.5) -- (-1.5, 1.5) -- (2.5, 1.5) -- (2.5, -1.5) -- cycle;

			pair A = n((3, 1)) * 5/3; 
			real rho = 2 / 9;

			picture inter;
			fill(inter, around, pattern("hatch2"));
			fill(inter, circle((0,0), 1), white);
			add(inter);

			draw(circle((0,0), 1));
			dot('$(a_0, b_0)$', (0,0), align=S);

			draw((0,0) -- n((-1, 1)), dashed);
			label("$r$", n((-1, 1)) / 2, align=1.5S);

			dot("$(a,b)$", A, red, align=2SE);
			filldraw(circle(A, rho), pattern("hatch"), red);

			label("$F$", n((1,-1))*2.5/3);
		\end{asy}
	\end{subfigure}
\end{figure}

\begin{prv}
	$\O$ est un ouvert.

	Soit $B$ la boule ouverte de centre $(a_0, b_0) \in \R^2$ et de rayon $r > 0$.

	On pose $\rho = \frac{1}{2}\big(r - \|(a,b) - (a_0,b_0)\|\big)$.
	Montrons que \[
		B_{(a,b)}(\rho) \subset  B_{(a,b)}(r).
	\]

	Soit $(x,y) \in B_{(a,b)}(\rho)$.
	\begin{align*}
		\|(x,y) - (a_0,b_0)\|&= \|(x,y)- (a,b) + (a,b) - (a_0,b_0)\| \\
		&\le \|(x,y) - (a,b)\| + \|(a,b) - (a_0, b_0)\|\\
		&< \rho + \|(a,b) - (a_0, b_0)\| = \frac{1}{2}r + \frac{1}{2} \|(a,b) - (a_0, b_0)\|\\
		&< r
	\end{align*}
	
	Soit $F$ la boule fermée de centre $(a_0, b_0)$ et de rayon $r \ge 0$.

	Soit $(a,b) \not\in F$. On pose \[
		\rho = \frac{1}{2}\big(\|(a,b) - (a_0, b_0)\| - r\big) > 0.
	\]

	Montrons que $B_{(a,b)}(\rho) \subset \R^2\setminus F$.

	Soit $(x,y) \in B_{(a,b)}(\rho)$.

	\begin{align*}
		\|(x,y) - (a_0, b_0)\| &= \|(x,y) - (a,b) + (a,b) - (a_0, b_0)\| \\
		&\ge \big| \underbrace{\|(x,y) - (a,b)\|}_{\le \rho} - \underbrace{\|(a,b) - (a_0, b_0)\|}_{> r} \big|\\
		&\ge \|(a,b) - (a_0, b_0)\|- \|(x,y) - (a,b)\|\\
		&> \|(a,b) - (a_0, b_0)\|- \rho\\
		&> \frac{1}{2} \|(a,b) - (a_0, b_0)\| + \frac{1}{2}r\\
		&> r
	\end{align*}

	donc $(x,y) \not\in F$.
\end{prv}

\begin{exm}
	\begin{enumerate}
		\item $\O$ est ouvert.\\
			$\R^2$ est ouvert.
		\item $\O$ est fermé.\\
			$\R^2$ est fermé.\\
		\item $\big\{(x,0)  \mid x > 0\big\}$ n'est ni ouverte ni fermé.
	\end{enumerate}
\end{exm}

\begin{figure}[H]
	\centering
	\begin{asy}
		size(3cm);

		draw((0, -1) -- (0, 3), Arrow(TeXHead));
		draw((-1, 0) -- (3, 0), Arrow(TeXHead));
		
		draw((0,0) -- (0, 2.97), red);
		draw(circle((0,1.5), 0.5), deepred);
		draw(circle((0,0.5), 0.1), deepred);
	\end{asy}
\end{figure}

\begin{defn}
	Soit $(a,b) \in \R^2$ et $V \in \mathcal{P}(\R^2)$.

	On dit que $V$ est un voisinage de $(a,b)$ s'il existe $r > 0$ tel que \[
		B_{(a,b)}(r) \subset V.
	\]
	\index{voisinage (dans $\R^2$)}
\end{defn}

\begin{prop}
	Un ouvert non vide est un voisinage en chacun de ces points. \qed
\end{prop}

\begin{defn}
	Soit $D \subset \R^2$. Un \underline{point intérieur} de $D$ est un couple $(a,b) \in D$ tel que \[
		\exists r > 0, B_{(a,b)}(r) \subset D.
	\] en d'autres termes, si $D$ est un voisinage de $(a,b)$.

	On note $\mathring D$ l'ensemble des points intérieurs à $D$. C'est \underline{l'intérieur} de $D$.
	\index{point intérieur (dans $\R^2$)}
	\index{intérieur (dans $\R^2$)}
\end{defn}

\begin{prop}
	$\mathring D$ est le plus grand ouvert $O$ de $\R^2$ tel que $O \subset D$.
\end{prop}

\begin{figure}[H]
	\centering
	\incfig{interieur-plus-grand-ouvert}
\end{figure}


\begin{prv}
	Soit $(a,b) \in \mathring D$.

	Par définition, il existe $r > 0$ tel que \[
		B_{(a,b)}(r) \subset D.
	\] Montrons que $B_{(a,b)}(r) \subset \mathring D$.

	Soit $(x,y) \in B_{(a,b)}(r)$. Comme $B_{(a,b)}(r)$ est un ouvert de $\R^2$, il existe $\rho > 0$ tel que \[
		B_{(x,y)}(\rho) \subset B_{(a,b)}(r)
	\] donc $(x,y) \in \mathring D$.

	Donc $\mathring D$ est ouvert, $\mathring D \subset D$.

	Soit $O$ un ouvert de $\R^2$ tel que $O \subset D$. Montrons que $O \subset \mathring D$.

	Soit $(x,y) \in O$. Soit $r > 0$ tel que \[
		B_{(x,y)}(r) \subset O \subset D
	\] donc $(x,y) \in \mathring D$.
\end{prv}

\begin{defn}
	Soit $f: D \subset \R^2 \to \R$, $\ell \in \R$, $(a,b) \in \mathring D$.

	On dit que \underline{$f(x,y)$ tend vers $\ell$ quand $(x,y)$ tend vers $(a,b)$} ou que $\ell$ est \underline{une limite} de $f$ en $(a,b)$ si \[
		\forall \varepsilon > 0, \exists r > 0, \forall (x,y) \in D, \|(x,y) - (a,b)\| < r \implies \left| f(x,y) - \ell \right| \le \varepsilon.
	\] en d'autres termes si \[
		\forall V \in \mathcal{V}_{\ell}, \exists W \in \mathcal{V}_{(a,b)}, \forall (x,y) \in W \cap D, f(x,y) \in V.
	\]
	\index{limite (dans $\R^2$)}
	\index{tendre vers (dans $\R^2$)}
\end{defn}

\begin{prop}
	[unicité de la limite]
	Soit $f: D \to \R$, $(a,b) \in \mathring D$, $\ell_1, \ell_2 \in \R$ telles que $\ell_1$ et $\ell_2$ sont des limites de $f$ en $(a,b)$.

	Alors $\ell_1 = \ell_2$.
\end{prop}

\begin{figure}[H]
	\centering
	\incfig{preuve-unicité-de-la-limite}
\end{figure}

\begin{prv}
	On suppose $\ell_1 < \ell_2$. On pose $\varepsilon = \frac{\ell_2 - \ell_1}{2} > 0$.

	Soit $r_1 > 0$ tel que \[
		f\big(B_{(a,b)}(r_1)\big) \subset ]\ell_1 - \varepsilon, \ell_1 + \varepsilon[.
	\] Soit $r_2 > 0$ tel que \[
		f\big(B_{(a,b)}(r_2)\big) \subset ]\ell_2 - \varepsilon, \ell_2 + \varepsilon [.
	\] On pose $r = \min(r_1, r_2)$ donc \[
		B_{(a,b)}(r_1) \cap B_{(a,b)}(r_2) = B_{(a,b)}(r) \neq \O.
	\] Soit $(x,y) \in B_{(a,b)}(r)$. Alors, \[
		f(x,y) \in ]\ell_1 - \varepsilon, \ell_1 + \varepsilon[ \cap ]\ell_2 - \varepsilon, \ell_2 + \varepsilon[ = \O.
	\] $\lightning$
\end{prv}

\begin{defn}
	Soit $f : D \to \R$, $(a,b) \in \mathring D$.

	On dit que $f$ est \underline{continue} en $(a,b)$ si \[
		f(x,y) \tendsto{(x,y) \to (a,b)}f(a,b).
	\]
	\index{continuité (dans $\R^2$)}
\end{defn}

\begin{prop}
	\underline{Si} $f(x,y) \tendsto{(x,y) \to (a,b)} \ell$ \\
	\underline{alors} $\begin{cases}
		f(x,b) \tendsto{x \to a} \ell\\
		f(a,y) \tendsto{y \to b} \ell.\\
	\end{cases}$
\end{prop}

\begin{prv}~\\
	\begin{figure}[H]
		\centering
		\incfig{limite-x-en-a-et-y-en-b}
	\end{figure}
\end{prv}

\underline{Contre-exemple} : exercice 3.

\begin{exm}
	\begin{enumerate}
		\item $f : \begin{array}{rcl}
				\R^2 &\longrightarrow& \R \\
				(x,y) &\longmapsto& x
			\end{array}$ limite en $(0,0)$ ?

			Soit $\varepsilon > 0$. On pose $r = \varepsilon$. \[
				\forall (x,y) \in B_{(0,0)}(r),
				\left| f(x,y) \right| = \left| x \right| \le \|(x,y)\| < r = \varepsilon
			\] Donc $f(x,y) \tendsto{(x,y) \to (a,b)} 0$.
		\item limite $f : \begin{array}{rcl}
				\R^2 &\longrightarrow& \R \\
				(x,y) &\longmapsto& x^3
			\end{array}$ en $(0,0)$ ?

			Soit $\varepsilon > 0$. On pose $r = \sqrt[3]{r} > 0$. \[
				\forall (x,y) \in B_{(0,0)}(r),
				\left| f(x,y) \right| = \left| x^3 \right| \le \|(x,y)\|^3 < r^3 = \varepsilon.
			\]
		\item limite de $f : \begin{array}{rcl}
			\R^2 &\longrightarrow& \R \\
			(x,y) &\longmapsto& x^3y^2
		\end{array}$ en $(0,0)$ ?

		Soit $\varepsilon > 0$. On pose $r = \sqrt[5]{\varepsilon} > 0$. \[
			\forall (x,y) \in B_{(0,0)}(r), \left| f(x,y) \right| = \left| x^3 y^2 \right| \le \|(x,y)\|^3 \|(x,y)\|^2 < r^5 = \varepsilon.
		\]
	\end{enumerate}
\end{exm}

\begin{defn}
	Soient $D \subset \R^2$ et $(x,y) \in \R^2$.

	\begin{figure}[H]
    \centering
    \incfig{point-adhérent}
	\end{figure}
	
	On dit que $(x,y)$ est \underline{adhérent} à $D$ si \[
		\forall r > 0, B_{(x,y)}(r) \cap D \neq \O.
	\] L'ensemble des points adhérents à $D$ est noté $\overline{D}$. On dit que $\overline{D}$ est \underline{l'adhérence} de $D$.
	\index{point adhérent (dans $\R^2$)}
	\index{adhérent (dans $\R^2$)}
\end{defn}

\begin{defn}
	Soit $f: D \subset \R^2 \to \R$ et $(a,b) \in \overline{D}$, $\ell \in \R$. On dit que $f$ tend vers $\ell$ quand $(x,y)$ tend vers $(a,b)$ si \[
		\forall \varepsilon > 0, \exists r > 0, \forall (x,y) \in B_{(a,b)}(r) \cap D,
		\left| f(x,y) - \ell \right| \le \varepsilon.
	\]
	\index{limite (dans $\R^2$)}
	\index{tendre vers (dans $\R^2$)}
\end{defn}

\begin{prop}
	\begin{enumerate}
		\item Dans ce contexte, il y a unicité de la limite
		\item La limite d'une somme, d'un produit, d'un quotien, d'une composée se comporte comme dans le cas d'une seule variable.
		\item Soit $f: D \to \R$ continue. Soient $g: I \to \R$ et $h: I \to \R$ continues telles que \[
			\forall t \in I, \big(g(t), h(t)\big) \in D.
		\] Alors \[
			t \in I \mapsto f\big(g(t), h(t)\big) \in \R
		\] est continue.
	\end{enumerate}
\end{prop}

\begin{figure}[H]
	\centering
	\begin{asy}
		import three;
		import graph3;
		size(5cm);

		settings.render = 0;
		settings.prc = false;
		currentprojection = obliqueX;

		draw(O -- X, Arrow3(TeXHead2));
		draw(O -- Y, Arrow3(TeXHead2));
		draw(O -- Z, Arrow3(TeXHead2));

		triple f(real x, real y, real z = 0) { return (x,y,cos(x - 0.5) * cos(y - 0.5)/1.2 + 0.15); }

		real inc = 1 / 5;

		for(real x = 0; x <= 1; x += inc) {
			draw(graph(
				new real(real t) { return x; }, // x
				new real(real y) { return y; }, // y
				new real(real y) { return f(x,y).z; }, // z
				0, 1
			), gray);
		}

		for(real y = 0; y <= 1; y += inc) {
			draw(graph(
				new real(real x) { return x; }, // x
				new real(real t) { return y; }, // y
				new real(real x) { return f(x,y).z; }, // z
				0, 1
			), gray);
		}

		path3 path1 = (0.3, 0.2, 0) .. (0.5, 0.5, 0) .. (0.6, 0.7, 0) .. (0.9, 0.8, 0);
		path3 path2 = (0.3, 0.8, 0) .. (0.5, 0.5, 0) .. (0.6, 0.3, 0) .. (0.9, 0.2, 0);
		path3 pathA = f(0.3, 0.2, 0) .. f(0.5, 0.5, 0) .. f(0.6, 0.7, 0) .. f(0.9, 0.8, 0);
		path3 pathB = f(0.3, 0.8, 0) .. f(0.5, 0.5, 0) .. f(0.6, 0.3, 0) .. f(0.9, 0.2, 0);

		draw(path1, red, Arrow3(TeXHead2, position=0.5));
		draw(pathA, red, Arrow3(TeXHead2, position=0.5));
		draw(path2, deepcyan, Arrow3(TeXHead2, position=0.3));
		draw(pathB, deepcyan, Arrow3(TeXHead2, position=0.3));

		dot((0.5, 0.5, 0));
		dot(f(0.5, 0.5, 0));
		draw((0.5, 0.5, 0) -- f(0.5, 0.5, 0), dashed);
	\end{asy}
\end{figure}


	\part{Transpositions}

\begin{defn}
	Une \underline{transposition} est un cycle de longueur 2 : $\begin{pmatrix}
		a&b
	\end{pmatrix}$ avec $a \neq b$.
	\index{transposition (permutation)}
\end{defn}

\begin{exm}
	Avec $n = 5$ et $\gamma = \begin{pmatrix}
		2&4&1
	\end{pmatrix}$.

	\begin{figure}[H]
		\centering

		\begin{asy}
			size(5cm);

			real rho = 0.15; // circles

			void draw_cycle(pair O, real r ...int[] nums) {
				int n = nums.length;
				real eps = (15 / r) * 0.8;

				for(int i = 0; i < n; ++i) {
					real theta_1 = (360/n) * (i+1);
					real theta_2 = (360/n) * i;

					pair C = O + dir(theta_2) * r;

					draw(circle(C, rho));
					label("$" + string(nums[i]) + "$", C);
					draw(arc(O, r, theta_2+eps, theta_1-eps), Arrow(TeXHead));
				}
			}

			draw_cycle((-1,0), 0.8, 1, 2, 4);
			draw_cycle((1,0), 0.3, 3);
			draw_cycle((2,0), 0.3, 5);
		\end{asy}
	\end{figure}

	\[
		\gamma = \begin{pmatrix}
			1&4
		\end{pmatrix} \begin{pmatrix}
			1&2
		\end{pmatrix}
	\]

	Avec $n = 6$ et $\gamma = \begin{pmatrix}
		1&3&5&6&2
	\end{pmatrix} = \begin{pmatrix}
		1&2&3&4&5&6\\
		3&1&5&4&6&2
	\end{pmatrix}$.

	Donc, \[
		\gamma = \begin{pmatrix}
			1&2
		\end{pmatrix} \begin{pmatrix}
			1&6
		\end{pmatrix} \begin{pmatrix}
			1&5
		\end{pmatrix} \begin{pmatrix}
			1&3
		\end{pmatrix}
	\] 
	\[
		\begin{pmatrix}
			1&2&3&4&5&6\\
			3&2&1&4&5&6\\
			3&2&5&4&1&6\\
			3&2&5&4&6&1\\
			3&1&5&4&6&2\\
		\end{pmatrix}
	\]

	Et, \[
		\gamma = \begin{pmatrix}
			1&3
		\end{pmatrix} \begin{pmatrix}
			2&3
		\end{pmatrix} \begin{pmatrix}
			3&5
		\end{pmatrix} \begin{pmatrix}
			5&6
		\end{pmatrix} 
	\]

	\[
		\begin{pmatrix}
			1&2&3&4&5&6\\
			1&2&3&4&6&5\\
			1&2&5&4&6&3\\
			1&3&5&4&6&2\\
			3&1&5&4&6&2\\
		\end{pmatrix} 
	\] 
\end{exm}

\begin{exm}
	\[
		\begin{pmatrix}
			1&4
		\end{pmatrix} = \begin{pmatrix}
			1&2
		\end{pmatrix} \begin{pmatrix}
			2&3
		\end{pmatrix} \begin{pmatrix}
			3&4
		\end{pmatrix} \begin{pmatrix}
			2&3
		\end{pmatrix} \begin{pmatrix}
			1&2
		\end{pmatrix}
	\]
	On n'a pas toujours le même nombre de transpositions mais la parité du nombre reste la même (proposition plus loin).
\end{exm}

\begin{thm}
	Toute permutation se décompose en produit de transpositions.
\end{thm}

\begin{prv}
	Soit $\gamma = \begin{pmatrix}
		a_1&\cdots&a_k
	\end{pmatrix}$ un $k$-cycle.

	On remarque que
	\[
		\gamma = \begin{pmatrix}
			a_1&a_k
		\end{pmatrix} \cdots \begin{pmatrix}
			a_1&a_4
		\end{pmatrix} \begin{pmatrix}
			a_1&a_3
		\end{pmatrix} \begin{pmatrix}
			a_1&a_2
		\end{pmatrix}
	\] C'est un produit de transpositions.
\end{prv}

\begin{exm}
	Avec $n = 10$ et $\sigma = \begin{pmatrix}
		1&2&3&4&5&6&7&8&9&10\\
		9&8&1&7&2&3&4&5&10&6
	\end{pmatrix}$.

	On a
	\begin{align*}
		\sigma &= \begin{pmatrix}
			1&9&10&6&3
		\end{pmatrix} \begin{pmatrix}
			2&8&5
		\end{pmatrix} \begin{pmatrix}
			4&7
		\end{pmatrix}\\
		&= \begin{pmatrix}
			1&3
		\end{pmatrix} \begin{pmatrix}
			1&6
		\end{pmatrix} \begin{pmatrix}
			1&10
		\end{pmatrix} \begin{pmatrix}
			1&9
		\end{pmatrix} \begin{pmatrix}
			2&5
		\end{pmatrix} \begin{pmatrix}
			2&8
		\end{pmatrix} \begin{pmatrix}
			4&7
		\end{pmatrix} \\
	\end{align*}

	Vérification : \[
		\begin{pmatrix}
			1&2&3&4&5&6&7&8&9&10\\
			1&2&3&7&5&6&4&8&9&10\\
			1&8&3&7&5&6&4&2&9&10\\
			1&8&3&7&2&6&4&5&9&10\\
			9&8&3&7&2&6&4&5&1&10\\
			9&8&3&7&2&6&4&5&10&1\\
			9&8&3&7&2&1&4&5&10&6\\
			9&8&1&7&2&3&4&5&10&6\\
		\end{pmatrix} 
	\] 
\end{exm}

	\part{Familles orthogonales}

\begin{thm}[Pythagore]
	Soit $(x,y) \in E^2$. \[
		\|x+y\|^2 = \|x\|^2 + \|y\|^2 \iff x \perp y
	.\]
	\begin{figure}[H]
		\centering
		\begin{asy}
			size(4cm);
			pair u = (1, 0.5);
			pair v = 1.5 * (0, 1) * u;
			draw((0,0)--u, Arrow(TeXHead));
			label("$x$", u/2, align=S);
			draw(u--v+u, Arrow(TeXHead));
			label("$y$", u + v/2, align=NE);
			draw((0,0) -- u + v, Arrow(TeXHead));
			draw(u + v / 7.5 -- u + v / 7.5 - u / 5 -- u - u / 5 -- u -- cycle);
		\end{asy}
	\end{figure}
\end{thm}

\begin{prv}
	\[
		\|x + y\|^2 = \|x\|^2 + \|y\|^2 \iff 2\left<x \mid y \right> = 0 \iff x \perp y
	.\]
\end{prv}

\begin{defn}
	Soit $(e_i)_{i\in I}$ une famille de vecteurs. On dit que cette famille est \underline{orthogonale} si \[
		\forall i \neq j\, e_i \perp e_j
	.\] Si, en plus, on a \[
		\forall i \in I,\,\|e_i\| = 1,
	\] alors on dit que la famille est \underline{orthonormale} ou \underline{orthonormée}.
	\index{famille orthogonale}
	\index{famille orthonormale}
	\index{famille orthonormée}
\end{defn}

\begin{prop}[Pythagore]
	Soit $(e_1, \ldots, e_n)$ une famille orthogonale. Alors \[
		\left\| \sum_{i=1}^n e_i \right\|^2 = \sum_{i=1}^n \|e_i\|^2
	.\]
\end{prop}

\begin{thm}
	Toute famille orthogonale de vecteurs non nuls est libre.
\end{thm}

\begin{prv}
	Soit $(e_i)_{i\in I}$ une famille orthogonale telle que \[
		\forall i \in I,\,e_i \neq 0_E
	.\] Soit $n \in \N^*$, $(\lambda_1, \ldots, \lambda_n) \in \R^n$. On suppose \[
		\sum_{k=1}^n \lambda_k e_{i_k} = 0_E
	.\] Soit $j \in \left\llbracket 1,n \right\rrbracket$.
	\begin{align*}
		0 &= \left<\sum_{k=1}^n \lambda_k e_{i_k}  \mid e_{i_j} \right>\\
		&= \sum_{k=1}^n \lambda_k \left<e_{i_k}  \mid e_{i_j} \right> \\
		&= \lambda_j \underbrace{\|e_{i_j}\|^2}_{\neq 0} \\
	\end{align*}
	donc $\lambda_j = 0$.
\end{prv}

\begin{algo}[Orthonormalisation de Gran--Schmidt]
	On suppose $E$ de dimension finie. Soit $\mathcal{B} = (e_1, \ldots, e_n)$ une base de $E$.

	\begin{itemize}
		\item\underline{\it Étape 1}: On pose $v_1 = \frac{e_1}{\|e_1\|}$ de sorte que $\|v_1\| = 1$.
		\item\underline{\it Étape 2} : On pose \[
				u_2 = e_2 - \left<e_2  \mid v_1 \right> v_1
			.\] Ainsi,
			\begin{align*}
				\left<u_2 \mid v_1 \right> &= \big<e_2 - \left<e_2 \mid v_1 \right> v_1  \mid v_1 \big>\\
				&= \left<e_2 \mid v_1 \right> - \left<e_2 \mid v_1 \right> \left<v_1 \mid v_1 \right> \\
				&= 0. \\
			\end{align*}
			On pose $v_2 = \frac{u_2}{\|u_2\|}$ donc $v_2 \perp v_1$ et $\|v_2\| = 1$.
		\item\underline{\it Étape 3} : On pose \[
				u_2 = e_3 - \left<e_3 \mid v_1 \right>v_1 - \left<e_3 \mid v_2 \right>v_2
			.\] Ainsi,
			\begin{align*}
				\left<u_3  \mid v_1 \right> &= \left<e_3  \mid v_1 \right> - \left<e_3 \mid v_1 \right>\underbrace{\left<v_1 \mid v_1 \right>}_{=1} - \left<e_3 \mid v_2 \right>\underbrace{\left<v_2 \mid v_1 \right>}_{=0} \\
				&= 0 \\
			\end{align*}
			et 
			\begin{align*}
				\left<u_3 \mid v_2 \right> &= \left<e_3  \mid  v_2 \right> - \left<e_3 \mid v_1 \right> \underbrace{\left<v_1 \mid v_2 \right>}_{=0} - \left<e_3 \mid v_2 \right> \underbrace{\left<v_2 \mid v_2 \right>}_{=1}\\
				&= 0. \\
			\end{align*}
			On pose $v_3 = \frac{u_3}{\|u_3\|}$ de sorte que $v_3 \perp v_1$, $v_3 \perp v_2$ et $\|v_3\| = 1$.
		\item\underline{\it Étape $i+1$}: On pose \[
			u_{i+1} = e_{i+1} - \sum_{k=1}^i \left<e_{i+1}  \mid v_k \right> v_k
		.\] Ainsi, pour tout $j \in \left\llbracket 1,i \right\rrbracket,$ on a
		\begin{align*}
			\left<u_{i+1}  \mid v_j \right> &= \left<e_{i+1}  \mid v_j \right> - \sum_{k=1}^i \left<e_{i+1} \mid v_k \right> \left<v_k \mid v_j \right> \\
			&= \left<e_{i+1} \mid v_j \right> - \left<e_{i+1} \mid v_j \right> \|v_j\|^2 \\
			&= 0. \\
		\end{align*}
		On pose $v_{i+1} = \frac{u_{i+1}}{\|u_{i+1}\|}$.
	\end{itemize}
\end{algo}

\begin{exm}
	Avec $E = \R_3[X]$, $\left<P \mid Q \right> = \int_{0}^{1} P(t)\,Q(t)~\mathrm{d}t$ et $\mathcal{B} = (1, X, X^2, X^3)$.
	\begin{enumerate}
		\item $\|1\|^2 = \left<1 \mid 1 \right> = \int_{0}^{1} 1~\mathrm{d}t = 1$ et donc $v_1 = 1$.
		\item $u_2 = X - \left<X  \mid v_1 \right>v_1$. Or, $\left<X \mid v_1 \right> = \int_{0}^{1} t~\mathrm{d}t = \frac{1}{2}$. D'où $u_2 = X - \frac{1}{2}$.
			\begin{align*}
				\|u_2\|^2 &= \int_{0}^{1} \left( t - \frac{1}{2} \right)^2~\mathrm{d}t \\
				&= \int_{0}^{1} \left( t^2 - t + \frac{1}{4} \right)~\mathrm{d}t \\
				&= \frac{1}{3} - \frac{1}{2} + \frac{1}{4} \\
				&= \frac{1}{12} \\
			\end{align*} On en déduit que $v_2 = \sqrt{12}\left( X - \frac{1}{2} \right)$.
		\item $u_3 = X^2 - \left<X^2 \mid v_1 \right>v_1 - \left<X^2 \mid v_2 \right>v_2$.
			On a \[
				\left<X^2 \mid v_1 \right> = \int_{0}^{1} t^2~\mathrm{d}t = \frac{1}{3}
			\] et
			\begin{align*}
				\left<X^2 \mid v_2 \right> &= \sqrt{12} \int_{0}^{1} t^2\left( t - \frac{1}{2} \right)~\mathrm{d}t \\
				&= \frac{\sqrt{12}}{12}. \\
			\end{align*}
			D'où
			\begin{align*}
				u_3 &= X^2 - \frac{1}{3} - \frac{\sqrt{12}}{12}\sqrt{12} \left( X - \frac{1}{2} \right)\\
				&= X^2 - \frac{1}{3} - X + \frac{1}{2} \\
				&= X^2 - X + \frac{1}{6}. \\
			\end{align*}
			\begin{align*}
				\|u_3\|^2 &= \int_{0}^{1} \left( t^2 - t + \frac{1}{6} \right)~\mathrm{d}t\\
				&= \int_{0}^{1} \left( t^4 + t^2 + \frac{1}{36} - 2t^3 + \frac{t^2}{3} - \frac{t}{3} \right) ~\mathrm{d}t \\
				&= \frac{1}{5} + \frac{1}{3} + \frac{1}{36} - \frac{1}{2} + \frac{1}{9} - \frac{1}{6} \\
				&= \frac{36 + 60 + 5 - 90 + 20 - 30}{180} \\
				&= \frac{1}{180} \\
			\end{align*}
			On en déduit que \[
				v_3 = 6\sqrt{5}\left( X^2 - X + \frac{1}{6} \right).
			\]
		\item Exercice : calculer $v_4$.
	\end{enumerate}
\end{exm}

\begin{prop}
	Soit $\mathcal{B} = (e_1, \ldots, e_n)$ une base de $E$ et $\mathcal{C}$ la base obtenue par le procédé d'orthonormalisation de Gram--Schmidt. Alors, \[
		\forall i \in \left\llbracket 1,n \right\rrbracket,\,\Vect(e_1,\ldots, e_i) = \Vect(v_1, \ldots, v_i)
	.\]\qed
\end{prop}

\begin{exm}[orthogonalisation]
	\begin{itemize}
		\item $u_1 = 1$.
		\item
			\begin{align*}
				\begin{rcases*}
					u_2 \in \Vect(e_1, e_2)\\
					u_2 \perp u_1
				\end{rcases*}
				\iff& \begin{cases}
					u_2 = ae_1 + be_2\quad (a,b) \in \R^2\\
					\left<u_1 \mid u_2 \right> = 0
				\end{cases}\\
				\iff& \begin{cases}
					u_2 = a + bX\\
					\int_{0}^{1} (a+bt)~\mathrm{d}t = 0.
				\end{cases}\\
			\end{align*}
			\begin{align*}
				\int_{0}^{1} (a+bt)~\mathrm{d}t = 0 \iff& a + \frac{b}{2} = 0\\
				\iff& a = -\frac{b}{2}\\
				\iff& u_2 = -\frac{b}{2} + bX.
			\end{align*}
			Par exemple, $u_2 = -1 + 2X$.
		\item $\begin{cases}
				u_3 \in \Vect(e_1, e_2, e_3)\\
				u_3 \perp u_1\\
				u_3 \perp u_2
			\end{cases}$

			On pose $u_3 = a + bX + cX^2$ avec $(a,b,c)\in \R^3$.
			\begin{align*}
				\begin{rcases*}
					\int_{0}^{1} \left( a+bt + ct^2 \right)~\mathrm{d}t = 0\\
					\int_{0}^{1} \left(a + bt+ct^2\right)(2t - 1)~\mathrm{d}t = 0
				\end{rcases*} \iff& \begin{cases}
					a + \frac{b}{2} + \frac{c}{3} = 0\\
					\int_{0}^{1} \left( 2ct^3 + (-c + 2b)t^2 + (2a - b)t - a \right) ~\mathrm{d}t = 0
				\end{cases}\\
				\iff& \begin{cases}
					a + \frac{b}{2} + \frac{c}{3} = 0\\
					\frac{c}{2} + \frac{2b - c}{3} + \frac{2\cancel{a} - b}{2} - \cancel{a} = 0
				\end{cases}\\
				\iff&  \begin{cases}
					a = -\frac{b}{2} - \frac{c}{3} = \frac{c}{2} - \frac{c}{3} = \frac{c}{6}\\
					b = -c.
				\end{cases}
			\end{align*}
			On en déduit que \[
				u_3 = 1 - 6X + 6X^2
			.\]
	\end{itemize}
\end{exm}

\begin{crlr}[théorème de la base orthonormée incomplète] Soit $(e_1, \ldots, e_k)$ une base orthonormée d'un espace euclidien. On peut trouver $e_{k+1},\ldots,e_n$ tels que $(e_1, \ldots, e_k, e_{k+1},\ldots,e_n)$ soit une base orthonormée de $E$.
\end{crlr}

\begin{prv}
	On sait que $(e_1, \ldots, e_k)$ est libre. On complète $(e_1, \ldots, e_k)$ en une base $\mathcal{B}$ de $E$. On orthonormalise $\mathcal{B}$ : on obtient une base orthonormée $\mathcal{C}$ de $E$. En détaillant l'algorithme de Gram--Schmidt, on s'aper\c coit que les $k$ premiers vecteurs de $\mathcal{C}$ sont ceux de $\mathcal{B}$.
\end{prv}

\begin{thm}
	Soit $E$ un espace euclidien et $\mathcal{B} = (e_1, \ldots, e_n)$ une base orthonormée de $E$. Soit $(x,y) \in E^2$. On pose $(x_1, \ldots, x_n) \in \R^n$ et $(y_1, \ldots, y_n) \in \R^n$ tels que \[
		x = \sum_{i=1}^n x_i e_i \qquad\qquad y = \sum_{i=1}^n y_i e_i
	.\] Alors \[
		\left<x \mid y \right> = \sum_{i=1}^n x_i y_i
	.\]
	\vspace{3mm}
	Soit $X = \mat{x_1\\\vdots\\x_n}$ et $Y = \mat{y_1\\ \vdots \\ y_n}$. Alors, \[
		\left<x \mid y \right> = X^\T\,Y
	.\]
\end{thm}

\begin{prv}
	\begin{align*}
		\left<x \mid y \right> &= \left<\sum_{i=1}^n x_ie_i  \mid y \right>\\
		&= \sum_{i=1}^n x_i \left<e_i  \mid y \right> \\
		&= \sum_{i=1}^n x_i \left<e_i  \mid \sum_{j=1}^n y_j e_j \right> \\
		&= \sum_{i=1}^n x_i \sum_{j=1}^n y_j \underbrace{\left<e_i \mid e_j \right>}_{\delta_i^j} \\
		&= \sum_{i=1}^n x_i y_i. \\
	\end{align*}
\end{prv}

\begin{prop}
	Soit $E$ un espace euclidien et $\mathcal{B} = (e_1, \ldots, e_n)$ une base orthonormée de $E$. Alors, \[
		\forall x \in E,\,x = \sum_{i=1}^n \left<x \mid e_i \right>e_i
	.\]
\end{prop}

\begin{prv}
	Soit $x \in E$. On pose \[
		x = \sum_{i=1}^n x_i e_i
	\] avec $(x_1, \ldots, x_n) \in \R^n$. Soit $j \in \left\llbracket 1,n \right\rrbracket$. On a
	\begin{align*}
		\left<x \mid e_j \right> &= \left<\sum_{i=1}^n x_i e_i  \mid e_j \right>\\
		&= \sum_{i=1}^n x_i \left<e_i \mid e_j \right> \\
		&= x_j. \\
	\end{align*}
\end{prv}

	\part{Lois de composition}

\begin{defn}
	Une \underline{loi de composition interne} \index{loi de composition interne} est une application $f$ de $E \times E$ dans $E$.
	
	On la note $x * y$ au lieu de $f(x,y)$ (on est libre de choisir le symbôle).
\end{defn}

\begin{defn}
	Soit $E$ un ensemble muni d'une loi de composition interne $\boxtimes$.

	On dit que $\boxtimes$ est \underline{associative} \index{associativité (loi de composition interne)} si \[
		\forall (x,y,z) \in E^3,\;(x\boxtimes y)\boxtimes z = x \boxtimes (y \boxtimes z).
	\] Dans ce cas, on écrit plutôt $x \boxtimes y \boxtimes z$.
\end{defn}

\begin{exm}
	\begin{itemize}
		\item $+$ et $\times $ dans $\C$ sont associatives;
		\item $ \circ$ est associative;
		\item  la multiplication matricielle est aussi associative.
	\end{itemize}
\end{exm}

\begin{defn}
	On dit que $\boxtimes$ est \underline{commutative} \index{commutativité (loi de composition interne)} si \[
		\forall (x,y) \in E^2, x\boxtimes y = y\boxtimes x.
	\]
\end{defn}

\begin{exm}
	\begin{itemize}
		\item $+$ et $\times $ dans $\C$ sont commuatives;
		\item $ \circ $ n'est pas commutative;
		\item  la multiplication matricielle n'est pas commutative.
	\end{itemize}
\end{exm}

\begin{defn}
	Soit $e \in E$. On dit que $e$ est un
	\begin{itemize}
		\item \underline{élément neutre à gauche}\index{élément neutre à gauche (loi de composition interne)} si  \[
				\forall x \in E,\; e\boxtimes x = x;
			\]
		\item \underline{élément neutre à droite}\index{élément neutre à droite (loi de composition interne)} si  \[
				\forall x \in E,\; x\boxtimes e = x;
			\]
		\item \underline{élément neutre}\index{élément neutre (loi de composition interne)} si  \[
				\forall x \in E,\; e\boxtimes x = x\boxtimes e = x.
			\]
	\end{itemize}
\end{defn}

\begin{prop}
	Sous reserve d'existence, il y a unicité de l'élément neutre.
\end{prop}

\begin{prv}
	Soient $e$ et $e'$ deux éléments neutre.
	\begin{itemize}
		\item $e \boxtimes e' = e'$ car $e$ est neutre,
		\item $e \boxtimes e' = e$ car $e'$ est neutre.
	\end{itemize} On a donc $e = e'$.
\end{prv}

\begin{axm}[axiome du choix]
	Soit $E$ un ensemble non vide. Il existe $f : \mathcal{P}(E) \setminus \{\O\} \to E$ telle que \[
		\forall A \in \mathcal{P}(E) \setminus \{\O\},\; f(A) \in A.
	\]
\end{axm}

\begin{defn}
	Soit $f: E \to F$. Le \underline{graphe} \index{graphe (application)} de $f$ est \[
		\Big\{\big(x,f(x)\big)  \mid x \in E\Big\} \subset E \times F.
	\]
\end{defn}

\begin{prop}
	Soit $G \subset E\times F$. $G$ est le graphe d'une application si et seulement si \[
		\forall x \in E,\,\exists! y \in F,\, (x,y) \in G.
	\]
\end{prop}

\begin{prv}
	\begin{itemize}
		\item[``$\implies$''] par définition d'une application
		\item[``$\impliedby$''] On pose $f(x)$ le seul élément $y$ de $F$ qui vérifie $(x,y) \in G$. Alors $f \in F^E$ et son graphe vaut $G$.
	\end{itemize}
\end{prv}

\begin{defn}
	Soit $A \in \mathcal{P}(E)$. L'\underline{indicatrice}\index{indicatrice (ensemble)} de $A$ est \begin{align*}
		\mathbbm{1}_A: E &\longrightarrow \{0,1\} \\
		x &\longmapsto \begin{cases}
			1 &\text{ si } x \in A,\\
			0 & \text{ si } x \not\in A.
		\end{cases}
	\end{align*}
\end{defn}

\begin{exm}
	\begin{enumerate}
		\item Dans $\C$, le neutre de $+$ est $0$ et le neutre de $\times$ est $1$.
		\item Dans $E^E$, le neutre de $ \circ $ est $\id_E$.
		\item Dans $\mathcal{M}_n(\C)$ (l'ensemble des matrices carrées $n \times n$ à valeurs dans $\C$), le neutre de $\times $ est $I_n$ : \[
				I_n =
				\begin{pNiceMatrix}
					1&&(0)\\
					&\Ddots&\\
					(0)&&1
				\end{pNiceMatrix}
			\] 
	\end{enumerate}
\end{exm}

\begin{defn}
	Soit $E$ un ensemble muni d'une loi de composition interne $\boxtimes$ et $x \in E$.

	\begin{enumerate}
		\item On dit que $x$ est \underline{simplifiable à gauche}\index{simplifiabilité à gauche} si \[
				\forall (y,z) \in E^2,\,(x\boxtimes y = x \boxtimes z) \implies x = z.
			\] et que $x$ est \underline{simplifiable à droite}\index{simplifiabilité à droite} si \[
				\forall (y,z) \in E^2,\,(y\boxtimes x = z \boxtimes y) \implies x = z.
			\]
		\item On dit que $x$ est \underline{symétrisable à gauche}\index{symétrisabilité à gauche} s'il exiiste $y \in E$ tel que $y\boxtimes x = e$ où $e$ est l'élément neutre de $\boxtimes$.

			De même, on dit que $x$ est \underline{symétrisable à droite}\index{symétrisabilité à droite} s'il existe $y \in E$ tel que $x \boxtimes y = e$.

			On dit que $x$ est \underline{symétrisable}\index{symétrisabilité} s'il est symétrisable à gauche et à droite, donc s'il existe $y \in E$ tel que $x \boxtimes y = y \boxtimes x = e$.
	\end{enumerate}
\end{defn}

\begin{exm}
	$E = \N$ muni de la loi $+$, tous les éléments de $E$ sont simplifiables. $0$ est le seuele élément de $E$ symétrisable.
\end{exm}

\begin{prop}
	Avec les notations précédentes, si $\boxtimes$ est associative, et $x$ est symétrisable, alors $x$ est simplifiable.
\end{prop}

\begin{prv}
	Soient $y, z \in E$.
	\begin{itemize}
		\item On suppose $x \boxtimes y = x \boxtimes z$. Soit $a \in E$ tel que $a\in E$ tel que $a \boxtimes x = e$. Alors \[
				a \boxtimes (x\boxtimes y) = a \boxtimes (x \boxtimes z).
			\] Or,
			\begin{align*}
				a \boxtimes (x \boxtimes y) &= (a \boxtimes x) \boxtimes y \\
				&= e \boxtimes y \\
				&= y. \\
			\end{align*}

			De même, $a \boxtimes (x \boxtimes z) = z$.

			Donc $y = z$.
		\item De même, si $y \boxtimes x = z \boxtimes x$, on ``multiplie'' $x$ à droite par $a$ et on obtient $y = z$.
	\end{itemize}
\end{prv}

\begin{prop-defn}
	On suppose $\boxtimes$ associative. Soit $x \in E$ symétrisable. Alors \[
		\exists ! y \in E,\; x \boxtimes y = y \boxtimes x = e.
	\] On dit que $y$ est le \underline{symétrique}\index{symétrique (loi de composition interne)} de $x$ et on le note $y = x^*$.
\end{prop-defn}

\begin{prv}
	Soeint $x,y,z \in E$ tels que \[
		\begin{cases}
			 x \boxtimes y = y \boxtimes x = e\\
			 x \boxtimes z = z \boxtimes x = e\\
		\end{cases}
	\] Alors, $x \boxtimes y = x \boxtimes z$ et, en simplifiant par $x$, on a $y = z$.
\end{prv}

\begin{exm}
	Les fonctions symétrisables de $(E^E,  \circ)$ sont les bijections et le symétrique d'une bijection est sa réciproque.
\end{exm}

\begin{rmk}
	\begin{enumerate}
		\item Si la loi est notée $+$, on parle d'\underline{opposé}\index{opposé (loi de composition interne)} plutôt que de symétrique et on le note $-x$ au lieu de $x^*$.
			L'élément neutre est noté $0_E$.
		\item Si la loi est notée $\times$, on parle d'élément \underline{inversible}\index{inversibilité (loi de composition interne)} au lieu de symétrisable, d'\underline{inverse}\index{inverse (loi de composition interne)} au lieu de symétrique et on note $x^{-1}$ au lieu de $x^*$. On note le neutre $1_E$.
	\end{enumerate}
\end{rmk}

\begin{exo}
	Soient $x,y \in E = \R^+_*$. On définit la loi de composition interne $\oplus$ : \[
		x \oplus y = \frac{1}{\frac{1}{x}\oplus \frac{1}{y}}.
	\] Cette loi peut-être utile en physique pour le calcul de résistances équivalentes en parallèles.
	\begin{itemize}
		\item {\sc Associativité} : soient $x,y,z \in E$.

			D'une part, on a \[
				x \oplus (y \oplus z) = \frac{1}{\frac{1}{x} + \frac{1}{\frac{1}{\frac{1}{x}+ \frac{1}{y}}}} = \frac{1}{\frac{1}{x}+\frac{1}{y}+\frac{1}{z}}.
			\] D'autre part, on a \[
			(x \oplus y) \oplus z = \frac{1}{\frac{1}{\frac{1}{\frac{1}{x}+\frac{1}{y}}}+\frac{1}{z}} = \frac{1}{\frac{1}{x}+ \frac{1}{y}+\frac{1}{z}}.
			\] La loi $\oplus$ est associative.
		\item {\sc Commutativité} : soient $x, y \in E$. \[
				x \oplus y = \frac{1}{\frac{1}{x}+\frac{1}{y}} = \frac{1}{\frac{1}{y}+\frac{1}{x}} = y\oplus x.
			\] Donc la loi $\oplus$ est commutative.
		\item {\sc Élément neutre} : soit $e$ l'élément neutre de $\oplus$. \[
				\forall x \in E,\; x \oplus e = e \oplus x = x.
			\] Comme la loi est commutative, seul l'égalité $x \oplus e = x$ est utile.

			Soit $x \in E$. On a donc $\frac{1}{\frac{1}{x}+\frac{1}{e}}=x$ donc $\frac{ex}{e+x}=x$ donc $ex = x(e+x)$ et donc $\cancel{ex} = \cancel{ex} + x^2$. On en déduit que $x^2 = 0$, ce qui n'est pas possible car $x \in \R^+_*$. Donc, il n'y a pas d'élément neutre pour $\oplus$.
	\end{itemize}
\end{exo}


	\chap[01]{Calculs algébriques}
	\renewcommand{\cwd}{../chap01}
	\part{Topologie de $\R^2$}

\begin{defn}
	La \underline{norme (euclidienne)} de $\R^2$ est l'application définie par \[
		\forall (x,y) \in \R^2, \|(x,y)\| = \sqrt{x^2 + y^2}.
	\]

	\begin{figure}[H]
		\centering
		\begin{asy}
			import graph;
			axes(EndArrow);
			size(4cm);
			pair A = (3,2);
			dot(A);
			draw((3,0)--A, dashed);
			draw((0,2)--A, dashed);
			label("$x$", (3,0), align=S);
			label("$y$", (0,2), align=W);
			draw((0,0)--A);
			dot((4,3), white+0);
		\end{asy}
	\end{figure}
	\index{norme (de $\R^2$)}
	\index{norme euclidienne (de $\R^2$)}
\end{defn}

\begin{prop}
	La norme euclidienne vérifie:
	\begin{enumerate}
		\item (séparation) \[
			\forall (x,y) \in \R^2, \|(x,y)\| = 0 \iff x = y = 0,
			\]
		\item (homogénéité positive) \[
				\forall \lambda \in \R, \forall (x,y) \in \R^2, \|\lambda(x,y)\|= \left| \lambda \right| \|(x,y)\|
			\]
		\item (inégalité triangulaire) \[
			\forall (x,y), (a,b) \in \R^2,
			\|(x,y)+(a,b)\|\le \|(x,y)\|+\|(a,b)\|.
		\]
	\end{enumerate}
\end{prop}

\begin{prv}
	Déjà vue en replaçant $(x,y)$ par $x+iy \in \C$ et $\|(x,y)\|$ par |x+iy|
\end{prv}

\begin{defn}
	Soit $(a,b) \in \R^2$ et $r \in \R_+$.

	La \underline{boule ouverte} (ou \underline{disque ouvert}) de centre $(a,b)$ et de rayon $r$ est \[
		B_{(a,b)}(r) = \big\{ (x,y) \in \R^2  \mid \|(x,y) - (a,b)\| < r \big\}.
	\]

	La \underline{boule fermée} (ou \underline{disque fermé}) de centre $(a,b)$ et de rayon $r$ est \[
		\overline{B_{(a,b)}}(r) = \big\{ (x,y)\in \R^2  \mid \|(x,y) - (a,b)\| \le r \big\}.
	\]

	La \underline{sphère} (ou \underline{boule}) de centre $(a,b)$ et de rayon $r$ est \[
		S_{(a,b)}(r) = \partial \overline{B_{(a,b)}}(r) = \big\{ (x,y) \in \R^2  \mid \|(x,y) - (a,b)\| = r \big\}.
	\]
	\index{boule ouverte (de $\R^2$)}
	\index{disque ouverte (de $\R^2$)}
	\index{boule fermée (de $\R^2$)}
	\index{disque fermée (de $\R^2$)}
	\index{boule (de $\R^2$)}
	\index{sphère (de $\R^2$)}
\end{defn}

\begin{figure}[H]
		\centering
		\incfig{boule}
\end{figure}

\begin{rmk}
	On parle de boule en dimension quelconque.
\end{rmk}

\begin{defn}
	Une \underline{partie ouverte} $O$ de $\R^2$ (ou \underline{un ouvert}) si \[
		\forall (x,y) \in O, \exists r > 0, B_{(a,b)}(r) \subset O.
	\]
	Une partie $F$ est \underline{fermée} su $\R^2\setminus F$ est ouverte.
	\index{partie ouverte (de $\R^2$)}
	\index{ouvert (de $\R^2$)}
	\index{partie fermée (de $\R^2$)}
\end{defn}

\begin{figure}[H]
	\centering
	\incfig{partie-ouverte}
\end{figure}

\begin{prop}
	Une boule ouverte est ouverte. Une boule fermée est fermée.
\end{prop}

\begin{figure}[H]
	\centering
	\begin{subfigure}{4cm}
		\centering
		\begin{asy}
			import patterns;

			pair n(pair a) {return a / length(a);}

			add("hatch",hatch(2mm, SW, red));
			size(4cm);

			draw(circle((0,0), 1));
			dot('$(a_0, b_0)$', (0,0), align=S);

			draw((0,0) -- n((-1, 1)), dashed);
			label("$r$", n((-1, 1)) / 2, align=1.5S);

			pair A = n((1,3)) * (2/3);
			real rho = (1 - length(A)) * (2 / 3);

			dot("$(a,b)$", A, red, align=3SE);
			filldraw(circle(A, rho), pattern("hatch"), red);

			label("$O$", n((1,-1))*2.5/3);
		\end{asy}
	\end{subfigure}
	\begin{subfigure}{1cm}
		\centering~\\
	\end{subfigure}
	\begin{subfigure}{5cm}
		\centering
		\begin{asy}
			import patterns;

			pair n(pair a) {return a / length(a);}

			add("hatch",hatch(1mm, SW, red));
			add("hatch2",hatch(3mm, SE, blue));
			size(5cm);

			guide around = (-1.5, -1.5) -- (-1.5, 1.5) -- (2.5, 1.5) -- (2.5, -1.5) -- cycle;

			pair A = n((3, 1)) * 5/3; 
			real rho = 2 / 9;

			picture inter;
			fill(inter, around, pattern("hatch2"));
			fill(inter, circle((0,0), 1), white);
			add(inter);

			draw(circle((0,0), 1));
			dot('$(a_0, b_0)$', (0,0), align=S);

			draw((0,0) -- n((-1, 1)), dashed);
			label("$r$", n((-1, 1)) / 2, align=1.5S);

			dot("$(a,b)$", A, red, align=2SE);
			filldraw(circle(A, rho), pattern("hatch"), red);

			label("$F$", n((1,-1))*2.5/3);
		\end{asy}
	\end{subfigure}
\end{figure}

\begin{prv}
	$\O$ est un ouvert.

	Soit $B$ la boule ouverte de centre $(a_0, b_0) \in \R^2$ et de rayon $r > 0$.

	On pose $\rho = \frac{1}{2}\big(r - \|(a,b) - (a_0,b_0)\|\big)$.
	Montrons que \[
		B_{(a,b)}(\rho) \subset  B_{(a,b)}(r).
	\]

	Soit $(x,y) \in B_{(a,b)}(\rho)$.
	\begin{align*}
		\|(x,y) - (a_0,b_0)\|&= \|(x,y)- (a,b) + (a,b) - (a_0,b_0)\| \\
		&\le \|(x,y) - (a,b)\| + \|(a,b) - (a_0, b_0)\|\\
		&< \rho + \|(a,b) - (a_0, b_0)\| = \frac{1}{2}r + \frac{1}{2} \|(a,b) - (a_0, b_0)\|\\
		&< r
	\end{align*}
	
	Soit $F$ la boule fermée de centre $(a_0, b_0)$ et de rayon $r \ge 0$.

	Soit $(a,b) \not\in F$. On pose \[
		\rho = \frac{1}{2}\big(\|(a,b) - (a_0, b_0)\| - r\big) > 0.
	\]

	Montrons que $B_{(a,b)}(\rho) \subset \R^2\setminus F$.

	Soit $(x,y) \in B_{(a,b)}(\rho)$.

	\begin{align*}
		\|(x,y) - (a_0, b_0)\| &= \|(x,y) - (a,b) + (a,b) - (a_0, b_0)\| \\
		&\ge \big| \underbrace{\|(x,y) - (a,b)\|}_{\le \rho} - \underbrace{\|(a,b) - (a_0, b_0)\|}_{> r} \big|\\
		&\ge \|(a,b) - (a_0, b_0)\|- \|(x,y) - (a,b)\|\\
		&> \|(a,b) - (a_0, b_0)\|- \rho\\
		&> \frac{1}{2} \|(a,b) - (a_0, b_0)\| + \frac{1}{2}r\\
		&> r
	\end{align*}

	donc $(x,y) \not\in F$.
\end{prv}

\begin{exm}
	\begin{enumerate}
		\item $\O$ est ouvert.\\
			$\R^2$ est ouvert.
		\item $\O$ est fermé.\\
			$\R^2$ est fermé.\\
		\item $\big\{(x,0)  \mid x > 0\big\}$ n'est ni ouverte ni fermé.
	\end{enumerate}
\end{exm}

\begin{figure}[H]
	\centering
	\begin{asy}
		size(3cm);

		draw((0, -1) -- (0, 3), Arrow(TeXHead));
		draw((-1, 0) -- (3, 0), Arrow(TeXHead));
		
		draw((0,0) -- (0, 2.97), red);
		draw(circle((0,1.5), 0.5), deepred);
		draw(circle((0,0.5), 0.1), deepred);
	\end{asy}
\end{figure}

\begin{defn}
	Soit $(a,b) \in \R^2$ et $V \in \mathcal{P}(\R^2)$.

	On dit que $V$ est un voisinage de $(a,b)$ s'il existe $r > 0$ tel que \[
		B_{(a,b)}(r) \subset V.
	\]
	\index{voisinage (dans $\R^2$)}
\end{defn}

\begin{prop}
	Un ouvert non vide est un voisinage en chacun de ces points. \qed
\end{prop}

\begin{defn}
	Soit $D \subset \R^2$. Un \underline{point intérieur} de $D$ est un couple $(a,b) \in D$ tel que \[
		\exists r > 0, B_{(a,b)}(r) \subset D.
	\] en d'autres termes, si $D$ est un voisinage de $(a,b)$.

	On note $\mathring D$ l'ensemble des points intérieurs à $D$. C'est \underline{l'intérieur} de $D$.
	\index{point intérieur (dans $\R^2$)}
	\index{intérieur (dans $\R^2$)}
\end{defn}

\begin{prop}
	$\mathring D$ est le plus grand ouvert $O$ de $\R^2$ tel que $O \subset D$.
\end{prop}

\begin{figure}[H]
	\centering
	\incfig{interieur-plus-grand-ouvert}
\end{figure}


\begin{prv}
	Soit $(a,b) \in \mathring D$.

	Par définition, il existe $r > 0$ tel que \[
		B_{(a,b)}(r) \subset D.
	\] Montrons que $B_{(a,b)}(r) \subset \mathring D$.

	Soit $(x,y) \in B_{(a,b)}(r)$. Comme $B_{(a,b)}(r)$ est un ouvert de $\R^2$, il existe $\rho > 0$ tel que \[
		B_{(x,y)}(\rho) \subset B_{(a,b)}(r)
	\] donc $(x,y) \in \mathring D$.

	Donc $\mathring D$ est ouvert, $\mathring D \subset D$.

	Soit $O$ un ouvert de $\R^2$ tel que $O \subset D$. Montrons que $O \subset \mathring D$.

	Soit $(x,y) \in O$. Soit $r > 0$ tel que \[
		B_{(x,y)}(r) \subset O \subset D
	\] donc $(x,y) \in \mathring D$.
\end{prv}

\begin{defn}
	Soit $f: D \subset \R^2 \to \R$, $\ell \in \R$, $(a,b) \in \mathring D$.

	On dit que \underline{$f(x,y)$ tend vers $\ell$ quand $(x,y)$ tend vers $(a,b)$} ou que $\ell$ est \underline{une limite} de $f$ en $(a,b)$ si \[
		\forall \varepsilon > 0, \exists r > 0, \forall (x,y) \in D, \|(x,y) - (a,b)\| < r \implies \left| f(x,y) - \ell \right| \le \varepsilon.
	\] en d'autres termes si \[
		\forall V \in \mathcal{V}_{\ell}, \exists W \in \mathcal{V}_{(a,b)}, \forall (x,y) \in W \cap D, f(x,y) \in V.
	\]
	\index{limite (dans $\R^2$)}
	\index{tendre vers (dans $\R^2$)}
\end{defn}

\begin{prop}
	[unicité de la limite]
	Soit $f: D \to \R$, $(a,b) \in \mathring D$, $\ell_1, \ell_2 \in \R$ telles que $\ell_1$ et $\ell_2$ sont des limites de $f$ en $(a,b)$.

	Alors $\ell_1 = \ell_2$.
\end{prop}

\begin{figure}[H]
	\centering
	\incfig{preuve-unicité-de-la-limite}
\end{figure}

\begin{prv}
	On suppose $\ell_1 < \ell_2$. On pose $\varepsilon = \frac{\ell_2 - \ell_1}{2} > 0$.

	Soit $r_1 > 0$ tel que \[
		f\big(B_{(a,b)}(r_1)\big) \subset ]\ell_1 - \varepsilon, \ell_1 + \varepsilon[.
	\] Soit $r_2 > 0$ tel que \[
		f\big(B_{(a,b)}(r_2)\big) \subset ]\ell_2 - \varepsilon, \ell_2 + \varepsilon [.
	\] On pose $r = \min(r_1, r_2)$ donc \[
		B_{(a,b)}(r_1) \cap B_{(a,b)}(r_2) = B_{(a,b)}(r) \neq \O.
	\] Soit $(x,y) \in B_{(a,b)}(r)$. Alors, \[
		f(x,y) \in ]\ell_1 - \varepsilon, \ell_1 + \varepsilon[ \cap ]\ell_2 - \varepsilon, \ell_2 + \varepsilon[ = \O.
	\] $\lightning$
\end{prv}

\begin{defn}
	Soit $f : D \to \R$, $(a,b) \in \mathring D$.

	On dit que $f$ est \underline{continue} en $(a,b)$ si \[
		f(x,y) \tendsto{(x,y) \to (a,b)}f(a,b).
	\]
	\index{continuité (dans $\R^2$)}
\end{defn}

\begin{prop}
	\underline{Si} $f(x,y) \tendsto{(x,y) \to (a,b)} \ell$ \\
	\underline{alors} $\begin{cases}
		f(x,b) \tendsto{x \to a} \ell\\
		f(a,y) \tendsto{y \to b} \ell.\\
	\end{cases}$
\end{prop}

\begin{prv}~\\
	\begin{figure}[H]
		\centering
		\incfig{limite-x-en-a-et-y-en-b}
	\end{figure}
\end{prv}

\underline{Contre-exemple} : exercice 3.

\begin{exm}
	\begin{enumerate}
		\item $f : \begin{array}{rcl}
				\R^2 &\longrightarrow& \R \\
				(x,y) &\longmapsto& x
			\end{array}$ limite en $(0,0)$ ?

			Soit $\varepsilon > 0$. On pose $r = \varepsilon$. \[
				\forall (x,y) \in B_{(0,0)}(r),
				\left| f(x,y) \right| = \left| x \right| \le \|(x,y)\| < r = \varepsilon
			\] Donc $f(x,y) \tendsto{(x,y) \to (a,b)} 0$.
		\item limite $f : \begin{array}{rcl}
				\R^2 &\longrightarrow& \R \\
				(x,y) &\longmapsto& x^3
			\end{array}$ en $(0,0)$ ?

			Soit $\varepsilon > 0$. On pose $r = \sqrt[3]{r} > 0$. \[
				\forall (x,y) \in B_{(0,0)}(r),
				\left| f(x,y) \right| = \left| x^3 \right| \le \|(x,y)\|^3 < r^3 = \varepsilon.
			\]
		\item limite de $f : \begin{array}{rcl}
			\R^2 &\longrightarrow& \R \\
			(x,y) &\longmapsto& x^3y^2
		\end{array}$ en $(0,0)$ ?

		Soit $\varepsilon > 0$. On pose $r = \sqrt[5]{\varepsilon} > 0$. \[
			\forall (x,y) \in B_{(0,0)}(r), \left| f(x,y) \right| = \left| x^3 y^2 \right| \le \|(x,y)\|^3 \|(x,y)\|^2 < r^5 = \varepsilon.
		\]
	\end{enumerate}
\end{exm}

\begin{defn}
	Soient $D \subset \R^2$ et $(x,y) \in \R^2$.

	\begin{figure}[H]
    \centering
    \incfig{point-adhérent}
	\end{figure}
	
	On dit que $(x,y)$ est \underline{adhérent} à $D$ si \[
		\forall r > 0, B_{(x,y)}(r) \cap D \neq \O.
	\] L'ensemble des points adhérents à $D$ est noté $\overline{D}$. On dit que $\overline{D}$ est \underline{l'adhérence} de $D$.
	\index{point adhérent (dans $\R^2$)}
	\index{adhérent (dans $\R^2$)}
\end{defn}

\begin{defn}
	Soit $f: D \subset \R^2 \to \R$ et $(a,b) \in \overline{D}$, $\ell \in \R$. On dit que $f$ tend vers $\ell$ quand $(x,y)$ tend vers $(a,b)$ si \[
		\forall \varepsilon > 0, \exists r > 0, \forall (x,y) \in B_{(a,b)}(r) \cap D,
		\left| f(x,y) - \ell \right| \le \varepsilon.
	\]
	\index{limite (dans $\R^2$)}
	\index{tendre vers (dans $\R^2$)}
\end{defn}

\begin{prop}
	\begin{enumerate}
		\item Dans ce contexte, il y a unicité de la limite
		\item La limite d'une somme, d'un produit, d'un quotien, d'une composée se comporte comme dans le cas d'une seule variable.
		\item Soit $f: D \to \R$ continue. Soient $g: I \to \R$ et $h: I \to \R$ continues telles que \[
			\forall t \in I, \big(g(t), h(t)\big) \in D.
		\] Alors \[
			t \in I \mapsto f\big(g(t), h(t)\big) \in \R
		\] est continue.
	\end{enumerate}
\end{prop}

\begin{figure}[H]
	\centering
	\begin{asy}
		import three;
		import graph3;
		size(5cm);

		settings.render = 0;
		settings.prc = false;
		currentprojection = obliqueX;

		draw(O -- X, Arrow3(TeXHead2));
		draw(O -- Y, Arrow3(TeXHead2));
		draw(O -- Z, Arrow3(TeXHead2));

		triple f(real x, real y, real z = 0) { return (x,y,cos(x - 0.5) * cos(y - 0.5)/1.2 + 0.15); }

		real inc = 1 / 5;

		for(real x = 0; x <= 1; x += inc) {
			draw(graph(
				new real(real t) { return x; }, // x
				new real(real y) { return y; }, // y
				new real(real y) { return f(x,y).z; }, // z
				0, 1
			), gray);
		}

		for(real y = 0; y <= 1; y += inc) {
			draw(graph(
				new real(real x) { return x; }, // x
				new real(real t) { return y; }, // y
				new real(real x) { return f(x,y).z; }, // z
				0, 1
			), gray);
		}

		path3 path1 = (0.3, 0.2, 0) .. (0.5, 0.5, 0) .. (0.6, 0.7, 0) .. (0.9, 0.8, 0);
		path3 path2 = (0.3, 0.8, 0) .. (0.5, 0.5, 0) .. (0.6, 0.3, 0) .. (0.9, 0.2, 0);
		path3 pathA = f(0.3, 0.2, 0) .. f(0.5, 0.5, 0) .. f(0.6, 0.7, 0) .. f(0.9, 0.8, 0);
		path3 pathB = f(0.3, 0.8, 0) .. f(0.5, 0.5, 0) .. f(0.6, 0.3, 0) .. f(0.9, 0.2, 0);

		draw(path1, red, Arrow3(TeXHead2, position=0.5));
		draw(pathA, red, Arrow3(TeXHead2, position=0.5));
		draw(path2, deepcyan, Arrow3(TeXHead2, position=0.3));
		draw(pathB, deepcyan, Arrow3(TeXHead2, position=0.3));

		dot((0.5, 0.5, 0));
		dot(f(0.5, 0.5, 0));
		draw((0.5, 0.5, 0) -- f(0.5, 0.5, 0), dashed);
	\end{asy}
\end{figure}


	\part{Transpositions}

\begin{defn}
	Une \underline{transposition} est un cycle de longueur 2 : $\begin{pmatrix}
		a&b
	\end{pmatrix}$ avec $a \neq b$.
	\index{transposition (permutation)}
\end{defn}

\begin{exm}
	Avec $n = 5$ et $\gamma = \begin{pmatrix}
		2&4&1
	\end{pmatrix}$.

	\begin{figure}[H]
		\centering

		\begin{asy}
			size(5cm);

			real rho = 0.15; // circles

			void draw_cycle(pair O, real r ...int[] nums) {
				int n = nums.length;
				real eps = (15 / r) * 0.8;

				for(int i = 0; i < n; ++i) {
					real theta_1 = (360/n) * (i+1);
					real theta_2 = (360/n) * i;

					pair C = O + dir(theta_2) * r;

					draw(circle(C, rho));
					label("$" + string(nums[i]) + "$", C);
					draw(arc(O, r, theta_2+eps, theta_1-eps), Arrow(TeXHead));
				}
			}

			draw_cycle((-1,0), 0.8, 1, 2, 4);
			draw_cycle((1,0), 0.3, 3);
			draw_cycle((2,0), 0.3, 5);
		\end{asy}
	\end{figure}

	\[
		\gamma = \begin{pmatrix}
			1&4
		\end{pmatrix} \begin{pmatrix}
			1&2
		\end{pmatrix}
	\]

	Avec $n = 6$ et $\gamma = \begin{pmatrix}
		1&3&5&6&2
	\end{pmatrix} = \begin{pmatrix}
		1&2&3&4&5&6\\
		3&1&5&4&6&2
	\end{pmatrix}$.

	Donc, \[
		\gamma = \begin{pmatrix}
			1&2
		\end{pmatrix} \begin{pmatrix}
			1&6
		\end{pmatrix} \begin{pmatrix}
			1&5
		\end{pmatrix} \begin{pmatrix}
			1&3
		\end{pmatrix}
	\] 
	\[
		\begin{pmatrix}
			1&2&3&4&5&6\\
			3&2&1&4&5&6\\
			3&2&5&4&1&6\\
			3&2&5&4&6&1\\
			3&1&5&4&6&2\\
		\end{pmatrix}
	\]

	Et, \[
		\gamma = \begin{pmatrix}
			1&3
		\end{pmatrix} \begin{pmatrix}
			2&3
		\end{pmatrix} \begin{pmatrix}
			3&5
		\end{pmatrix} \begin{pmatrix}
			5&6
		\end{pmatrix} 
	\]

	\[
		\begin{pmatrix}
			1&2&3&4&5&6\\
			1&2&3&4&6&5\\
			1&2&5&4&6&3\\
			1&3&5&4&6&2\\
			3&1&5&4&6&2\\
		\end{pmatrix} 
	\] 
\end{exm}

\begin{exm}
	\[
		\begin{pmatrix}
			1&4
		\end{pmatrix} = \begin{pmatrix}
			1&2
		\end{pmatrix} \begin{pmatrix}
			2&3
		\end{pmatrix} \begin{pmatrix}
			3&4
		\end{pmatrix} \begin{pmatrix}
			2&3
		\end{pmatrix} \begin{pmatrix}
			1&2
		\end{pmatrix}
	\]
	On n'a pas toujours le même nombre de transpositions mais la parité du nombre reste la même (proposition plus loin).
\end{exm}

\begin{thm}
	Toute permutation se décompose en produit de transpositions.
\end{thm}

\begin{prv}
	Soit $\gamma = \begin{pmatrix}
		a_1&\cdots&a_k
	\end{pmatrix}$ un $k$-cycle.

	On remarque que
	\[
		\gamma = \begin{pmatrix}
			a_1&a_k
		\end{pmatrix} \cdots \begin{pmatrix}
			a_1&a_4
		\end{pmatrix} \begin{pmatrix}
			a_1&a_3
		\end{pmatrix} \begin{pmatrix}
			a_1&a_2
		\end{pmatrix}
	\] C'est un produit de transpositions.
\end{prv}

\begin{exm}
	Avec $n = 10$ et $\sigma = \begin{pmatrix}
		1&2&3&4&5&6&7&8&9&10\\
		9&8&1&7&2&3&4&5&10&6
	\end{pmatrix}$.

	On a
	\begin{align*}
		\sigma &= \begin{pmatrix}
			1&9&10&6&3
		\end{pmatrix} \begin{pmatrix}
			2&8&5
		\end{pmatrix} \begin{pmatrix}
			4&7
		\end{pmatrix}\\
		&= \begin{pmatrix}
			1&3
		\end{pmatrix} \begin{pmatrix}
			1&6
		\end{pmatrix} \begin{pmatrix}
			1&10
		\end{pmatrix} \begin{pmatrix}
			1&9
		\end{pmatrix} \begin{pmatrix}
			2&5
		\end{pmatrix} \begin{pmatrix}
			2&8
		\end{pmatrix} \begin{pmatrix}
			4&7
		\end{pmatrix} \\
	\end{align*}

	Vérification : \[
		\begin{pmatrix}
			1&2&3&4&5&6&7&8&9&10\\
			1&2&3&7&5&6&4&8&9&10\\
			1&8&3&7&5&6&4&2&9&10\\
			1&8&3&7&2&6&4&5&9&10\\
			9&8&3&7&2&6&4&5&1&10\\
			9&8&3&7&2&6&4&5&10&1\\
			9&8&3&7&2&1&4&5&10&6\\
			9&8&1&7&2&3&4&5&10&6\\
		\end{pmatrix} 
	\] 
\end{exm}

	\part{Familles orthogonales}

\begin{thm}[Pythagore]
	Soit $(x,y) \in E^2$. \[
		\|x+y\|^2 = \|x\|^2 + \|y\|^2 \iff x \perp y
	.\]
	\begin{figure}[H]
		\centering
		\begin{asy}
			size(4cm);
			pair u = (1, 0.5);
			pair v = 1.5 * (0, 1) * u;
			draw((0,0)--u, Arrow(TeXHead));
			label("$x$", u/2, align=S);
			draw(u--v+u, Arrow(TeXHead));
			label("$y$", u + v/2, align=NE);
			draw((0,0) -- u + v, Arrow(TeXHead));
			draw(u + v / 7.5 -- u + v / 7.5 - u / 5 -- u - u / 5 -- u -- cycle);
		\end{asy}
	\end{figure}
\end{thm}

\begin{prv}
	\[
		\|x + y\|^2 = \|x\|^2 + \|y\|^2 \iff 2\left<x \mid y \right> = 0 \iff x \perp y
	.\]
\end{prv}

\begin{defn}
	Soit $(e_i)_{i\in I}$ une famille de vecteurs. On dit que cette famille est \underline{orthogonale} si \[
		\forall i \neq j\, e_i \perp e_j
	.\] Si, en plus, on a \[
		\forall i \in I,\,\|e_i\| = 1,
	\] alors on dit que la famille est \underline{orthonormale} ou \underline{orthonormée}.
	\index{famille orthogonale}
	\index{famille orthonormale}
	\index{famille orthonormée}
\end{defn}

\begin{prop}[Pythagore]
	Soit $(e_1, \ldots, e_n)$ une famille orthogonale. Alors \[
		\left\| \sum_{i=1}^n e_i \right\|^2 = \sum_{i=1}^n \|e_i\|^2
	.\]
\end{prop}

\begin{thm}
	Toute famille orthogonale de vecteurs non nuls est libre.
\end{thm}

\begin{prv}
	Soit $(e_i)_{i\in I}$ une famille orthogonale telle que \[
		\forall i \in I,\,e_i \neq 0_E
	.\] Soit $n \in \N^*$, $(\lambda_1, \ldots, \lambda_n) \in \R^n$. On suppose \[
		\sum_{k=1}^n \lambda_k e_{i_k} = 0_E
	.\] Soit $j \in \left\llbracket 1,n \right\rrbracket$.
	\begin{align*}
		0 &= \left<\sum_{k=1}^n \lambda_k e_{i_k}  \mid e_{i_j} \right>\\
		&= \sum_{k=1}^n \lambda_k \left<e_{i_k}  \mid e_{i_j} \right> \\
		&= \lambda_j \underbrace{\|e_{i_j}\|^2}_{\neq 0} \\
	\end{align*}
	donc $\lambda_j = 0$.
\end{prv}

\begin{algo}[Orthonormalisation de Gran--Schmidt]
	On suppose $E$ de dimension finie. Soit $\mathcal{B} = (e_1, \ldots, e_n)$ une base de $E$.

	\begin{itemize}
		\item\underline{\it Étape 1}: On pose $v_1 = \frac{e_1}{\|e_1\|}$ de sorte que $\|v_1\| = 1$.
		\item\underline{\it Étape 2} : On pose \[
				u_2 = e_2 - \left<e_2  \mid v_1 \right> v_1
			.\] Ainsi,
			\begin{align*}
				\left<u_2 \mid v_1 \right> &= \big<e_2 - \left<e_2 \mid v_1 \right> v_1  \mid v_1 \big>\\
				&= \left<e_2 \mid v_1 \right> - \left<e_2 \mid v_1 \right> \left<v_1 \mid v_1 \right> \\
				&= 0. \\
			\end{align*}
			On pose $v_2 = \frac{u_2}{\|u_2\|}$ donc $v_2 \perp v_1$ et $\|v_2\| = 1$.
		\item\underline{\it Étape 3} : On pose \[
				u_2 = e_3 - \left<e_3 \mid v_1 \right>v_1 - \left<e_3 \mid v_2 \right>v_2
			.\] Ainsi,
			\begin{align*}
				\left<u_3  \mid v_1 \right> &= \left<e_3  \mid v_1 \right> - \left<e_3 \mid v_1 \right>\underbrace{\left<v_1 \mid v_1 \right>}_{=1} - \left<e_3 \mid v_2 \right>\underbrace{\left<v_2 \mid v_1 \right>}_{=0} \\
				&= 0 \\
			\end{align*}
			et 
			\begin{align*}
				\left<u_3 \mid v_2 \right> &= \left<e_3  \mid  v_2 \right> - \left<e_3 \mid v_1 \right> \underbrace{\left<v_1 \mid v_2 \right>}_{=0} - \left<e_3 \mid v_2 \right> \underbrace{\left<v_2 \mid v_2 \right>}_{=1}\\
				&= 0. \\
			\end{align*}
			On pose $v_3 = \frac{u_3}{\|u_3\|}$ de sorte que $v_3 \perp v_1$, $v_3 \perp v_2$ et $\|v_3\| = 1$.
		\item\underline{\it Étape $i+1$}: On pose \[
			u_{i+1} = e_{i+1} - \sum_{k=1}^i \left<e_{i+1}  \mid v_k \right> v_k
		.\] Ainsi, pour tout $j \in \left\llbracket 1,i \right\rrbracket,$ on a
		\begin{align*}
			\left<u_{i+1}  \mid v_j \right> &= \left<e_{i+1}  \mid v_j \right> - \sum_{k=1}^i \left<e_{i+1} \mid v_k \right> \left<v_k \mid v_j \right> \\
			&= \left<e_{i+1} \mid v_j \right> - \left<e_{i+1} \mid v_j \right> \|v_j\|^2 \\
			&= 0. \\
		\end{align*}
		On pose $v_{i+1} = \frac{u_{i+1}}{\|u_{i+1}\|}$.
	\end{itemize}
\end{algo}

\begin{exm}
	Avec $E = \R_3[X]$, $\left<P \mid Q \right> = \int_{0}^{1} P(t)\,Q(t)~\mathrm{d}t$ et $\mathcal{B} = (1, X, X^2, X^3)$.
	\begin{enumerate}
		\item $\|1\|^2 = \left<1 \mid 1 \right> = \int_{0}^{1} 1~\mathrm{d}t = 1$ et donc $v_1 = 1$.
		\item $u_2 = X - \left<X  \mid v_1 \right>v_1$. Or, $\left<X \mid v_1 \right> = \int_{0}^{1} t~\mathrm{d}t = \frac{1}{2}$. D'où $u_2 = X - \frac{1}{2}$.
			\begin{align*}
				\|u_2\|^2 &= \int_{0}^{1} \left( t - \frac{1}{2} \right)^2~\mathrm{d}t \\
				&= \int_{0}^{1} \left( t^2 - t + \frac{1}{4} \right)~\mathrm{d}t \\
				&= \frac{1}{3} - \frac{1}{2} + \frac{1}{4} \\
				&= \frac{1}{12} \\
			\end{align*} On en déduit que $v_2 = \sqrt{12}\left( X - \frac{1}{2} \right)$.
		\item $u_3 = X^2 - \left<X^2 \mid v_1 \right>v_1 - \left<X^2 \mid v_2 \right>v_2$.
			On a \[
				\left<X^2 \mid v_1 \right> = \int_{0}^{1} t^2~\mathrm{d}t = \frac{1}{3}
			\] et
			\begin{align*}
				\left<X^2 \mid v_2 \right> &= \sqrt{12} \int_{0}^{1} t^2\left( t - \frac{1}{2} \right)~\mathrm{d}t \\
				&= \frac{\sqrt{12}}{12}. \\
			\end{align*}
			D'où
			\begin{align*}
				u_3 &= X^2 - \frac{1}{3} - \frac{\sqrt{12}}{12}\sqrt{12} \left( X - \frac{1}{2} \right)\\
				&= X^2 - \frac{1}{3} - X + \frac{1}{2} \\
				&= X^2 - X + \frac{1}{6}. \\
			\end{align*}
			\begin{align*}
				\|u_3\|^2 &= \int_{0}^{1} \left( t^2 - t + \frac{1}{6} \right)~\mathrm{d}t\\
				&= \int_{0}^{1} \left( t^4 + t^2 + \frac{1}{36} - 2t^3 + \frac{t^2}{3} - \frac{t}{3} \right) ~\mathrm{d}t \\
				&= \frac{1}{5} + \frac{1}{3} + \frac{1}{36} - \frac{1}{2} + \frac{1}{9} - \frac{1}{6} \\
				&= \frac{36 + 60 + 5 - 90 + 20 - 30}{180} \\
				&= \frac{1}{180} \\
			\end{align*}
			On en déduit que \[
				v_3 = 6\sqrt{5}\left( X^2 - X + \frac{1}{6} \right).
			\]
		\item Exercice : calculer $v_4$.
	\end{enumerate}
\end{exm}

\begin{prop}
	Soit $\mathcal{B} = (e_1, \ldots, e_n)$ une base de $E$ et $\mathcal{C}$ la base obtenue par le procédé d'orthonormalisation de Gram--Schmidt. Alors, \[
		\forall i \in \left\llbracket 1,n \right\rrbracket,\,\Vect(e_1,\ldots, e_i) = \Vect(v_1, \ldots, v_i)
	.\]\qed
\end{prop}

\begin{exm}[orthogonalisation]
	\begin{itemize}
		\item $u_1 = 1$.
		\item
			\begin{align*}
				\begin{rcases*}
					u_2 \in \Vect(e_1, e_2)\\
					u_2 \perp u_1
				\end{rcases*}
				\iff& \begin{cases}
					u_2 = ae_1 + be_2\quad (a,b) \in \R^2\\
					\left<u_1 \mid u_2 \right> = 0
				\end{cases}\\
				\iff& \begin{cases}
					u_2 = a + bX\\
					\int_{0}^{1} (a+bt)~\mathrm{d}t = 0.
				\end{cases}\\
			\end{align*}
			\begin{align*}
				\int_{0}^{1} (a+bt)~\mathrm{d}t = 0 \iff& a + \frac{b}{2} = 0\\
				\iff& a = -\frac{b}{2}\\
				\iff& u_2 = -\frac{b}{2} + bX.
			\end{align*}
			Par exemple, $u_2 = -1 + 2X$.
		\item $\begin{cases}
				u_3 \in \Vect(e_1, e_2, e_3)\\
				u_3 \perp u_1\\
				u_3 \perp u_2
			\end{cases}$

			On pose $u_3 = a + bX + cX^2$ avec $(a,b,c)\in \R^3$.
			\begin{align*}
				\begin{rcases*}
					\int_{0}^{1} \left( a+bt + ct^2 \right)~\mathrm{d}t = 0\\
					\int_{0}^{1} \left(a + bt+ct^2\right)(2t - 1)~\mathrm{d}t = 0
				\end{rcases*} \iff& \begin{cases}
					a + \frac{b}{2} + \frac{c}{3} = 0\\
					\int_{0}^{1} \left( 2ct^3 + (-c + 2b)t^2 + (2a - b)t - a \right) ~\mathrm{d}t = 0
				\end{cases}\\
				\iff& \begin{cases}
					a + \frac{b}{2} + \frac{c}{3} = 0\\
					\frac{c}{2} + \frac{2b - c}{3} + \frac{2\cancel{a} - b}{2} - \cancel{a} = 0
				\end{cases}\\
				\iff&  \begin{cases}
					a = -\frac{b}{2} - \frac{c}{3} = \frac{c}{2} - \frac{c}{3} = \frac{c}{6}\\
					b = -c.
				\end{cases}
			\end{align*}
			On en déduit que \[
				u_3 = 1 - 6X + 6X^2
			.\]
	\end{itemize}
\end{exm}

\begin{crlr}[théorème de la base orthonormée incomplète] Soit $(e_1, \ldots, e_k)$ une base orthonormée d'un espace euclidien. On peut trouver $e_{k+1},\ldots,e_n$ tels que $(e_1, \ldots, e_k, e_{k+1},\ldots,e_n)$ soit une base orthonormée de $E$.
\end{crlr}

\begin{prv}
	On sait que $(e_1, \ldots, e_k)$ est libre. On complète $(e_1, \ldots, e_k)$ en une base $\mathcal{B}$ de $E$. On orthonormalise $\mathcal{B}$ : on obtient une base orthonormée $\mathcal{C}$ de $E$. En détaillant l'algorithme de Gram--Schmidt, on s'aper\c coit que les $k$ premiers vecteurs de $\mathcal{C}$ sont ceux de $\mathcal{B}$.
\end{prv}

\begin{thm}
	Soit $E$ un espace euclidien et $\mathcal{B} = (e_1, \ldots, e_n)$ une base orthonormée de $E$. Soit $(x,y) \in E^2$. On pose $(x_1, \ldots, x_n) \in \R^n$ et $(y_1, \ldots, y_n) \in \R^n$ tels que \[
		x = \sum_{i=1}^n x_i e_i \qquad\qquad y = \sum_{i=1}^n y_i e_i
	.\] Alors \[
		\left<x \mid y \right> = \sum_{i=1}^n x_i y_i
	.\]
	\vspace{3mm}
	Soit $X = \mat{x_1\\\vdots\\x_n}$ et $Y = \mat{y_1\\ \vdots \\ y_n}$. Alors, \[
		\left<x \mid y \right> = X^\T\,Y
	.\]
\end{thm}

\begin{prv}
	\begin{align*}
		\left<x \mid y \right> &= \left<\sum_{i=1}^n x_ie_i  \mid y \right>\\
		&= \sum_{i=1}^n x_i \left<e_i  \mid y \right> \\
		&= \sum_{i=1}^n x_i \left<e_i  \mid \sum_{j=1}^n y_j e_j \right> \\
		&= \sum_{i=1}^n x_i \sum_{j=1}^n y_j \underbrace{\left<e_i \mid e_j \right>}_{\delta_i^j} \\
		&= \sum_{i=1}^n x_i y_i. \\
	\end{align*}
\end{prv}

\begin{prop}
	Soit $E$ un espace euclidien et $\mathcal{B} = (e_1, \ldots, e_n)$ une base orthonormée de $E$. Alors, \[
		\forall x \in E,\,x = \sum_{i=1}^n \left<x \mid e_i \right>e_i
	.\]
\end{prop}

\begin{prv}
	Soit $x \in E$. On pose \[
		x = \sum_{i=1}^n x_i e_i
	\] avec $(x_1, \ldots, x_n) \in \R^n$. Soit $j \in \left\llbracket 1,n \right\rrbracket$. On a
	\begin{align*}
		\left<x \mid e_j \right> &= \left<\sum_{i=1}^n x_i e_i  \mid e_j \right>\\
		&= \sum_{i=1}^n x_i \left<e_i \mid e_j \right> \\
		&= x_j. \\
	\end{align*}
\end{prv}

	\part{Lois de composition}

\begin{defn}
	Une \underline{loi de composition interne} \index{loi de composition interne} est une application $f$ de $E \times E$ dans $E$.
	
	On la note $x * y$ au lieu de $f(x,y)$ (on est libre de choisir le symbôle).
\end{defn}

\begin{defn}
	Soit $E$ un ensemble muni d'une loi de composition interne $\boxtimes$.

	On dit que $\boxtimes$ est \underline{associative} \index{associativité (loi de composition interne)} si \[
		\forall (x,y,z) \in E^3,\;(x\boxtimes y)\boxtimes z = x \boxtimes (y \boxtimes z).
	\] Dans ce cas, on écrit plutôt $x \boxtimes y \boxtimes z$.
\end{defn}

\begin{exm}
	\begin{itemize}
		\item $+$ et $\times $ dans $\C$ sont associatives;
		\item $ \circ$ est associative;
		\item  la multiplication matricielle est aussi associative.
	\end{itemize}
\end{exm}

\begin{defn}
	On dit que $\boxtimes$ est \underline{commutative} \index{commutativité (loi de composition interne)} si \[
		\forall (x,y) \in E^2, x\boxtimes y = y\boxtimes x.
	\]
\end{defn}

\begin{exm}
	\begin{itemize}
		\item $+$ et $\times $ dans $\C$ sont commuatives;
		\item $ \circ $ n'est pas commutative;
		\item  la multiplication matricielle n'est pas commutative.
	\end{itemize}
\end{exm}

\begin{defn}
	Soit $e \in E$. On dit que $e$ est un
	\begin{itemize}
		\item \underline{élément neutre à gauche}\index{élément neutre à gauche (loi de composition interne)} si  \[
				\forall x \in E,\; e\boxtimes x = x;
			\]
		\item \underline{élément neutre à droite}\index{élément neutre à droite (loi de composition interne)} si  \[
				\forall x \in E,\; x\boxtimes e = x;
			\]
		\item \underline{élément neutre}\index{élément neutre (loi de composition interne)} si  \[
				\forall x \in E,\; e\boxtimes x = x\boxtimes e = x.
			\]
	\end{itemize}
\end{defn}

\begin{prop}
	Sous reserve d'existence, il y a unicité de l'élément neutre.
\end{prop}

\begin{prv}
	Soient $e$ et $e'$ deux éléments neutre.
	\begin{itemize}
		\item $e \boxtimes e' = e'$ car $e$ est neutre,
		\item $e \boxtimes e' = e$ car $e'$ est neutre.
	\end{itemize} On a donc $e = e'$.
\end{prv}

\begin{axm}[axiome du choix]
	Soit $E$ un ensemble non vide. Il existe $f : \mathcal{P}(E) \setminus \{\O\} \to E$ telle que \[
		\forall A \in \mathcal{P}(E) \setminus \{\O\},\; f(A) \in A.
	\]
\end{axm}

\begin{defn}
	Soit $f: E \to F$. Le \underline{graphe} \index{graphe (application)} de $f$ est \[
		\Big\{\big(x,f(x)\big)  \mid x \in E\Big\} \subset E \times F.
	\]
\end{defn}

\begin{prop}
	Soit $G \subset E\times F$. $G$ est le graphe d'une application si et seulement si \[
		\forall x \in E,\,\exists! y \in F,\, (x,y) \in G.
	\]
\end{prop}

\begin{prv}
	\begin{itemize}
		\item[``$\implies$''] par définition d'une application
		\item[``$\impliedby$''] On pose $f(x)$ le seul élément $y$ de $F$ qui vérifie $(x,y) \in G$. Alors $f \in F^E$ et son graphe vaut $G$.
	\end{itemize}
\end{prv}

\begin{defn}
	Soit $A \in \mathcal{P}(E)$. L'\underline{indicatrice}\index{indicatrice (ensemble)} de $A$ est \begin{align*}
		\mathbbm{1}_A: E &\longrightarrow \{0,1\} \\
		x &\longmapsto \begin{cases}
			1 &\text{ si } x \in A,\\
			0 & \text{ si } x \not\in A.
		\end{cases}
	\end{align*}
\end{defn}

\begin{exm}
	\begin{enumerate}
		\item Dans $\C$, le neutre de $+$ est $0$ et le neutre de $\times$ est $1$.
		\item Dans $E^E$, le neutre de $ \circ $ est $\id_E$.
		\item Dans $\mathcal{M}_n(\C)$ (l'ensemble des matrices carrées $n \times n$ à valeurs dans $\C$), le neutre de $\times $ est $I_n$ : \[
				I_n =
				\begin{pNiceMatrix}
					1&&(0)\\
					&\Ddots&\\
					(0)&&1
				\end{pNiceMatrix}
			\] 
	\end{enumerate}
\end{exm}

\begin{defn}
	Soit $E$ un ensemble muni d'une loi de composition interne $\boxtimes$ et $x \in E$.

	\begin{enumerate}
		\item On dit que $x$ est \underline{simplifiable à gauche}\index{simplifiabilité à gauche} si \[
				\forall (y,z) \in E^2,\,(x\boxtimes y = x \boxtimes z) \implies x = z.
			\] et que $x$ est \underline{simplifiable à droite}\index{simplifiabilité à droite} si \[
				\forall (y,z) \in E^2,\,(y\boxtimes x = z \boxtimes y) \implies x = z.
			\]
		\item On dit que $x$ est \underline{symétrisable à gauche}\index{symétrisabilité à gauche} s'il exiiste $y \in E$ tel que $y\boxtimes x = e$ où $e$ est l'élément neutre de $\boxtimes$.

			De même, on dit que $x$ est \underline{symétrisable à droite}\index{symétrisabilité à droite} s'il existe $y \in E$ tel que $x \boxtimes y = e$.

			On dit que $x$ est \underline{symétrisable}\index{symétrisabilité} s'il est symétrisable à gauche et à droite, donc s'il existe $y \in E$ tel que $x \boxtimes y = y \boxtimes x = e$.
	\end{enumerate}
\end{defn}

\begin{exm}
	$E = \N$ muni de la loi $+$, tous les éléments de $E$ sont simplifiables. $0$ est le seuele élément de $E$ symétrisable.
\end{exm}

\begin{prop}
	Avec les notations précédentes, si $\boxtimes$ est associative, et $x$ est symétrisable, alors $x$ est simplifiable.
\end{prop}

\begin{prv}
	Soient $y, z \in E$.
	\begin{itemize}
		\item On suppose $x \boxtimes y = x \boxtimes z$. Soit $a \in E$ tel que $a\in E$ tel que $a \boxtimes x = e$. Alors \[
				a \boxtimes (x\boxtimes y) = a \boxtimes (x \boxtimes z).
			\] Or,
			\begin{align*}
				a \boxtimes (x \boxtimes y) &= (a \boxtimes x) \boxtimes y \\
				&= e \boxtimes y \\
				&= y. \\
			\end{align*}

			De même, $a \boxtimes (x \boxtimes z) = z$.

			Donc $y = z$.
		\item De même, si $y \boxtimes x = z \boxtimes x$, on ``multiplie'' $x$ à droite par $a$ et on obtient $y = z$.
	\end{itemize}
\end{prv}

\begin{prop-defn}
	On suppose $\boxtimes$ associative. Soit $x \in E$ symétrisable. Alors \[
		\exists ! y \in E,\; x \boxtimes y = y \boxtimes x = e.
	\] On dit que $y$ est le \underline{symétrique}\index{symétrique (loi de composition interne)} de $x$ et on le note $y = x^*$.
\end{prop-defn}

\begin{prv}
	Soeint $x,y,z \in E$ tels que \[
		\begin{cases}
			 x \boxtimes y = y \boxtimes x = e\\
			 x \boxtimes z = z \boxtimes x = e\\
		\end{cases}
	\] Alors, $x \boxtimes y = x \boxtimes z$ et, en simplifiant par $x$, on a $y = z$.
\end{prv}

\begin{exm}
	Les fonctions symétrisables de $(E^E,  \circ)$ sont les bijections et le symétrique d'une bijection est sa réciproque.
\end{exm}

\begin{rmk}
	\begin{enumerate}
		\item Si la loi est notée $+$, on parle d'\underline{opposé}\index{opposé (loi de composition interne)} plutôt que de symétrique et on le note $-x$ au lieu de $x^*$.
			L'élément neutre est noté $0_E$.
		\item Si la loi est notée $\times$, on parle d'élément \underline{inversible}\index{inversibilité (loi de composition interne)} au lieu de symétrisable, d'\underline{inverse}\index{inverse (loi de composition interne)} au lieu de symétrique et on note $x^{-1}$ au lieu de $x^*$. On note le neutre $1_E$.
	\end{enumerate}
\end{rmk}

\begin{exo}
	Soient $x,y \in E = \R^+_*$. On définit la loi de composition interne $\oplus$ : \[
		x \oplus y = \frac{1}{\frac{1}{x}\oplus \frac{1}{y}}.
	\] Cette loi peut-être utile en physique pour le calcul de résistances équivalentes en parallèles.
	\begin{itemize}
		\item {\sc Associativité} : soient $x,y,z \in E$.

			D'une part, on a \[
				x \oplus (y \oplus z) = \frac{1}{\frac{1}{x} + \frac{1}{\frac{1}{\frac{1}{x}+ \frac{1}{y}}}} = \frac{1}{\frac{1}{x}+\frac{1}{y}+\frac{1}{z}}.
			\] D'autre part, on a \[
			(x \oplus y) \oplus z = \frac{1}{\frac{1}{\frac{1}{\frac{1}{x}+\frac{1}{y}}}+\frac{1}{z}} = \frac{1}{\frac{1}{x}+ \frac{1}{y}+\frac{1}{z}}.
			\] La loi $\oplus$ est associative.
		\item {\sc Commutativité} : soient $x, y \in E$. \[
				x \oplus y = \frac{1}{\frac{1}{x}+\frac{1}{y}} = \frac{1}{\frac{1}{y}+\frac{1}{x}} = y\oplus x.
			\] Donc la loi $\oplus$ est commutative.
		\item {\sc Élément neutre} : soit $e$ l'élément neutre de $\oplus$. \[
				\forall x \in E,\; x \oplus e = e \oplus x = x.
			\] Comme la loi est commutative, seul l'égalité $x \oplus e = x$ est utile.

			Soit $x \in E$. On a donc $\frac{1}{\frac{1}{x}+\frac{1}{e}}=x$ donc $\frac{ex}{e+x}=x$ donc $ex = x(e+x)$ et donc $\cancel{ex} = \cancel{ex} + x^2$. On en déduit que $x^2 = 0$, ce qui n'est pas possible car $x \in \R^+_*$. Donc, il n'y a pas d'élément neutre pour $\oplus$.
	\end{itemize}
\end{exo}

	\part{Divers}

\begin{defn}
	Soient $E$ et $F$ deux ensembles. Un \underline{couple}\index{couple} $(x,y)$ est la donnée d'un élément $x$ de $E$ et d'un élément $y$ de $F$ où \[
		\forall x,x' \in E,\,\forall y,y' \in F,\qquad (x,y) = (x',y') \iff \begin{cases}
			x=x',\\
			y=y'.
		\end{cases}
	\] On note $E \times F$ l'ensemble des couples; c'est le \underline{produit cartésien}\index{produit cartésion (ensembles)} de $E$ et $F$.
\end{defn}

\begin{exm}
	$D \times [0,1]$ est un cylindre plein où $D$ est le disque unité fermé i.e. \[
		D = \Big\{(x,y) \in \R^2 \mid x^2+y^2 \le 1\Big\}.
	\]
	\begin{figure}[H]
		\centering
		\begin{subfigure}[b]{3cm}
			\centering
			\begin{asy}
				size(3cm);
				draw(unitcircle);
				draw((0,0)--(1,0), red);
				label("$1$",(0.5,0), red, align=S);
			\end{asy}
		\end{subfigure}
		\begin{subfigure}[b]{3cm}
			\centering
			\begin{asy}
				size(3cm);
				label("$\times\; [0,1]\; =$", (0,0), fontsize(10));
				draw(unitcircle, white+0);
			\end{asy}
		\end{subfigure}
		\begin{subfigure}[b]{3cm}
			\centering
			\begin{asy}
				import solids;
				size(3cm);
				draw(shift((0, 0.5)) * unitcircle, white+0);
				revolution r = cylinder(O, 1, 1.5, Z);
				draw(r);
				triple M = (-1/2, sqrt(3)/2, 0);
				draw((0,0,0) -- M, red);
				label("$1$", M/2, red, align=S);
				draw(M*1.1--M*1.1+(0,0,1.5), magenta, Arrows3(TeXHead2));
				label("$1$", M*1.1+(0,0,0.75), magenta, align=E);
			\end{asy}
		\end{subfigure}
	\end{figure}

	$C \times C$ où $C = \Big\{(x,y) \in \R^2  \mid x^2 + y^2 = 1\Big\}$ est un tore (creu).

	\begin{figure}[H]
		\centering
		\begin{subfigure}[b]{3cm}
			\centering
			\begin{asy}
				size(3cm);
				draw(unitcircle);
				draw((0,0)--(1,0), red);
				label("$1$",(0.5,0), red, align=S);
			\end{asy}
		\end{subfigure}
		\begin{subfigure}[b]{1cm}
			\centering
			\begin{asy}
				size(3cm);
				label("$\times$", (0,0), fontsize(10));
				dot((0.1, 1), white+0);
				dot((-0.1, -1), white+0);
			\end{asy}
		\end{subfigure}
		\begin{subfigure}[b]{3cm}
			\centering
			\begin{asy}
				size(3cm);
				draw(unitcircle);
				draw((0,0)--(1,0), red);
				label("$1$",(0.5,0), red, align=S);
			\end{asy}
		\end{subfigure}
		\begin{subfigure}[b]{1cm}
			\centering
			\begin{asy}
				size(3cm);
				label("$=$", (0,0), fontsize(10));
				dot((0.1, 1), white+0);
				dot((-0.1, -1), white+0);
			\end{asy}
		\end{subfigure}
		\begin{subfigure}[b]{3cm}
			\centering
			\begin{asy}
				import three;
				import graph3;

				size(3cm,3cm);
				surface torus = surface(Circle(c=2Y,normal=X,r=0.5,n=32), c=O, axis=Z, n=32);

				draw(torus, white + opacity(0), meshpen=black + 0.2pt, nolight, render(merge=true));
			\end{asy}
			\vspace{0.7cm}
		\end{subfigure}
	\end{figure}
\end{exm}

\begin{defn}
	Soient $E$ et $F$ deux ensembles. On dit que $E$ et $F$ sont \underline{équipotents} s'il existe une bijection de $E$ dans $F$.
	\index{équipotence (ensembles)}
\end{defn}

\begin{exm}
	\begin{enumerate}
		\item $\N$ et $\N^*$ sont équipotents car  $f : \begin{array}{rcl}
				\N &\longrightarrow& \N^* \\
				k &\longmapsto& k + 1
			\end{array}$ est bijective.
		\item $P = \{n \in \N  \mid n \text{ pair}\}$ et $I= \{n \in \N \mid n \text{ impair}\}$ sont équipotents car $f : \begin{array}{rcl}
				P &\longrightarrow& I \\
				x &\longmapsto& x+1
			\end{array}$ est bijective.
		\item $\N$ et $P$ sont équipotents car $f : \begin{array}{rcl}
				\N &\longrightarrow& P \\
				k &\longmapsto& 2k
			\end{array}$ est bijective.
		\item $[0,1]$ et $[0,1[$ sont équipotents car \begin{align*}
			f: [0,1] &\longrightarrow [0,1[ \\
			x &\longmapsto \begin{cases}
				\frac{1}{n+1} &\text{ si } x = \frac{1}{n} \text{ avec } n \in \N^*\\
				x &\text{ sinon}
			\end{cases}
		\end{align*} est bijective.
		\item De même, $]0,1[$ et $]0,1]$ sont équipotents.
		\item $]0,1[$ et $[0,1[$ sont équipotents : $f : \begin{array}{rcl}
					]0,1] &\longrightarrow& [0,1[ \\
				x &\longmapsto& 1-x
			\end{array}$ est bijective.
		\item $\forall a < b$, $[a,b]$ et $[0,1]$ sont équipotents : \begin{align*}
				f: [0,1] &\longrightarrow [a,b] \\
				\alpha &\longmapsto \alpha b + (1 - \alpha) a
			\end{align*} est bijective (interpolation linéaire).
		\item $\R$ et $]0,1[$ sont équipotents : \begin{align*}
				f: \R &\longrightarrow ]0,1[ \\
				x &\longmapsto \frac{1}{2} + \frac{\Arctan x}{\pi}
			\end{align*} est bijective.
		\item $[0,1[$ et $\N$ ne sont pas équipotents (argument de Cantor). Soit $f: \N \to [0,1[$ une bijection :
			\[
				\begin{array}{c|l}
					k&\hfill f(k)\hfill~ \\ \hline
					0&0,\hfill \!0\hfill 0\hfill 0\hfill 0\hfill\ldots\\
					1&0,\hfill a_1\hfill a_2\hfill a_3\hfill a_4\hfill\ldots\\
					2&0,\hfill b_1\hfill b_2\hfill b_3\hfill b_4\hfill\ldots\\
					\vdots&\hfill\vdots\hfill\ddots
				\end{array}
			\] On considère le nombre \[
				x = 0,\,(a_0+1)(b_1+1)(c_2+1)\cdots
			\] $f(1) \neq x$ car ils n'ont pas le même chiffre des dizaines.\\
			$f(2) \neq x$ car ils n'ont pas le même chiffre des centaines.

			Par le même raisonement, on en déduit que \[
				\forall n \in \N, f(n) \neq x
			\] donc $x$ n'a pas d'antécédant : une contradiction.
		\item On verra en exercice que $E$ et $\mathcal{P}(E)$ ne sont pas équipotents. $\R$ et $\mathcal{P}(\R)$ ne sont pas équipotents mais $\R$ et $\mathcal{P}(\N)$ le sont (développement dyadique).
		\item $\R^2$ et $\R$ sont équipotents; $\C$ et $\R$ sont équipotents.
	\end{enumerate}
\end{exm}

\begin{exo}
	Soit $E$ un ensemble. L'application \begin{align*}
		f: \mathcal{P}(E) &\longrightarrow {0,1}^E \\
		A &\longmapsto \mathbbm{1}_A
	\end{align*} est bijective.

	Soit $g : E \to \{0,1\}$.
	\begin{itemize}
		\item[\underline{\sc Analyse}] Soit $A \in \mathcal{P}(E)$ tel que $f(A) = g$. Alors $g = \mathbbm{1}_A$.
			donc  \[
				\forall x \in E,\; g(x) = \mathbbm{1}_A(x)
			\] et donc \[
				\begin{cases}
					\forall x \in A,\, g(x) = 1\\
					\forall x \in E \setminus A,\,g(x) = 0
				\end{cases}
			\] On en déduit que \[
				A = \{ x \in E  \mid  g(x) = 1\}  = g^{-1}\big(\{1\}\big).
			\]
		\item[\underline{\sc Synthèse}] On pose $A = g^{-1}\big(\{1\}\big)$. Montrons que $f(A) = g$.
			\[
				\forall x \in E,\,g(x) = \begin{cases}
					1 &\text{ si } x \in A\\
					0 &\text{ si } x \not\in A
				\end{cases} = \mathbbm{1}_A
			\] donc $g = \mathbbm{1}_A$.
	\end{itemize}

	On aurait aussi pu rédiger de la fa\c con suivante : on pose \begin{align*}
		u: \{0,1\}^E &\longrightarrow \mathcal{P}(E) \\
		g &\longmapsto g^{-1}\big(\{1\}\big).
	\end{align*} On montre que $u$ est la réciproque de $f$ : \[
		\begin{cases}
			f \circ u = \id_{\{0,1\}^E},\\
			u \circ f = \id_{\mathcal{P}(E)}.
		\end{cases}
	\]
\end{exo}

\begin{defn}
	Soit $f : E \to F$. L'\underline{image de $f$}\index{image (application)} est \[
		\mathrm{Im}(f) = f(E) = \big\{f(x) \mid x \in E\big\}.
	\]
\end{defn}

\begin{prop}
	Soit $f: E \to F$. \[
		f \text{ est surjective } \iff f(E) = F.
	\]
\end{prop}

\begin{defn}
	Une \underline{suite de $E$}\index{suite (ensemble)} est une application de $\N$ dans $E$.
\end{defn}

\begin{rmk}[Notation]
	Soit $u \in E^\N$. Pour $n \in \N$, on écrit $u_n$ à la place de $u(n)$.
\end{rmk}

\begin{defn}
	Soient $E$ et $I$ deux ensembles. Une \underline{famille de $E$ indéxée par $I$}\index{famille (ensemble)} est une application de $I$ dans $E$.

	À la place de $u(i)$ (avec $i \in I$), on écrit $u_i$.
\end{defn}

\begin{defn}
	Soit $E$ un ensemble et $(A_i)_{i \in I}$ une famille de parties de $E$. On suppose $I \neq \O$. On pose \[
		\bigcup_{i \in  I} A_i = \{x \in E  \mid \exists i \in I,\, x \in A_i\}
	\] et \[
		\bigcap_{i \in  I} A_i = \{x \in E  \mid \forall i \in I,\, x \in A_i\}.
	\] On pose aussi $\bigcup_{i \in \O} A_i = \O$ et $\bigcap_{i \in \O}  A_i = E$.
\end{defn}

\begin{rmk}
	De même que pour les sommes et produits de complexes, on peut intervertir des réunions doubles.
\end{rmk}

\begin{prop}
	Soit $E$ un ensemble, $(A,B) \in \mathcal{P}(E)^2$. \[
		A \subset (E \setminus B) \iff A \cap B = \O.
	\]
\end{prop}

\begin{figure}[H]
	\centering
	\begin{asy}
		import patterns;
		add("hatch",hatch(1mm, deepcyan));
		add("hatch2",hatch(1mm, heavygreen));
		size(3cm);

		guide main_set = scale(1.3) * ((-1,1)..(-0.8,-0.8)..(0,-0.9)..(0.7,-1.2)..(0.8, 0.9)..cycle);
		guide set_a = shift((-0.5, -0.2)) * ((-0.6, 0.6)..(0.2,-0.2)..(0.2,-0.4)..(-0.6,-0.2)..cycle);
		guide set_b = shift((0.3, 0.4)) * ((0.8, -0.6)..(1.1,-0.2)..(0.2,0.5)..(0.2,-0.8)..cycle);

		draw(main_set, magenta); label("$E$", 1.3*(0.8,0.9),magenta, align=NE);
		draw(set_a, deepcyan); label("$A$", (-0.6,0.6), deepcyan, align=NW);
		draw(set_b, heavygreen); label("$B$", (0.8,-0.6), heavygreen, align=SE);

		fill(set_a, pattern("hatch"));
		fill(set_b, pattern("hatch2"));
	\end{asy}
\end{figure}

\begin{prv}
	\begin{itemize}
		\item[``$\implies$''] Soit $x \in A \cap B$. Alors $x \in A$ et $x \in B$. Comme $x \in A \subset (E \setminus B)$, alors $x \in E \setminus B$ i.e. $x \not\in B$ : une contradiction. Donc $A \cap B = \O$.
		\item[``$\impliedby$''] On suppose $A \cap B = \O$. Soit $x \in A$. Si $x \in B$, alors $x \in A \cap B = \O$ : faux.
			Donc $x \not\in B$ et donc $x \in E \setminus B$.
	\end{itemize}
\end{prv}

\begin{prop}
	Si $f: E\to F$ et $g: F \to G$ sont bijectives, alors $g \circ f$ est bijective et \[
		(g \circ f)^{-1} = f^{-1} \circ g^{-1}.
	\] \qed
\end{prop}

\begin{rmk}[\danger Attention]
	$g \circ f$ peut-être bijective alors que $f$ et $g$ ne le sont pas.
\end{rmk}


	\part{Rappels sur $\ln$ et $\exp$}

\begin{prop}~\\
	\begin{itemize}
		\item Soit $(a_i)_{i\in I}$ une famille finie de réels strictement positifs. Alors,
			\[
				\ln\left( \prod_{i \in I} a_i \right) = \sum_{i \in I} \ln a_i.
			\]
		\item Soit $(b_i)_{i\in I}$ une famille de réels. Alors \[
			\exp\left( \sum_{i \in I} b_i \right) = \prod_{i \in I} \exp(b_i).
		\]
	\end{itemize}
\end{prop}

\begin{rmk}
	Soit $f: I \to \R^*$ dérivable. On pose $g: x \mapsto \ln \left| f(x) \right|$.

	Alors $g$ est dérivable sur $I$ et \[
		\forall x \in I, g'(x) = \frac{f'(x)}{f(x)}
	\]

	On dit que $\frac{f'}{f}$ est la \underline{dérivée logarithmique} de $f$.

	Soient $f_1,f_2: I\to \R^*$ dérivables. Alors \[
		\frac{(f_1\,f_2)'}{f_1\,f_2} = \frac{f_1'}{f_1} + \frac{f_2'}{f_2}.
	\]
\end{rmk}

\begin{rmk}
	Soit $a \in \R$.
	\begin{itemize}
		\item Soit $n \in \N^*$. Alors, $a^n = \overbrace{a\times a\times a\times \cdots \times a}^{n \text{ fois }}$.
		\item Soit $n \in \Z^-_*$. Si $a \neq 0$, alors $a^n = \frac{1}{a^{-n}}$.
		\item Si $a \neq 0$, $a^{0} = 1$ et \[
				\forall p,q \in \Z, a^{p}\times a^{q} = a^{p+q}.
			\]
		\item Soit $p \in \Z$ et $a > 0$. \[
			a^{p} = \exp(\ln a^{p}) = \exp(p \ln a) = e^{p \ln a}.
		\]
	\end{itemize}
\end{rmk}

\begin{defn}
	Soit $a \in \R^+_*$ et $p \in \R$. On pose $a^p = e^{p \ln a}$.
\end{defn}


	\chap[02]{Nombres complexes}
	\renewcommand{\cwd}{../chap02}
	\part{Topologie de $\R^2$}

\begin{defn}
	La \underline{norme (euclidienne)} de $\R^2$ est l'application définie par \[
		\forall (x,y) \in \R^2, \|(x,y)\| = \sqrt{x^2 + y^2}.
	\]

	\begin{figure}[H]
		\centering
		\begin{asy}
			import graph;
			axes(EndArrow);
			size(4cm);
			pair A = (3,2);
			dot(A);
			draw((3,0)--A, dashed);
			draw((0,2)--A, dashed);
			label("$x$", (3,0), align=S);
			label("$y$", (0,2), align=W);
			draw((0,0)--A);
			dot((4,3), white+0);
		\end{asy}
	\end{figure}
	\index{norme (de $\R^2$)}
	\index{norme euclidienne (de $\R^2$)}
\end{defn}

\begin{prop}
	La norme euclidienne vérifie:
	\begin{enumerate}
		\item (séparation) \[
			\forall (x,y) \in \R^2, \|(x,y)\| = 0 \iff x = y = 0,
			\]
		\item (homogénéité positive) \[
				\forall \lambda \in \R, \forall (x,y) \in \R^2, \|\lambda(x,y)\|= \left| \lambda \right| \|(x,y)\|
			\]
		\item (inégalité triangulaire) \[
			\forall (x,y), (a,b) \in \R^2,
			\|(x,y)+(a,b)\|\le \|(x,y)\|+\|(a,b)\|.
		\]
	\end{enumerate}
\end{prop}

\begin{prv}
	Déjà vue en replaçant $(x,y)$ par $x+iy \in \C$ et $\|(x,y)\|$ par |x+iy|
\end{prv}

\begin{defn}
	Soit $(a,b) \in \R^2$ et $r \in \R_+$.

	La \underline{boule ouverte} (ou \underline{disque ouvert}) de centre $(a,b)$ et de rayon $r$ est \[
		B_{(a,b)}(r) = \big\{ (x,y) \in \R^2  \mid \|(x,y) - (a,b)\| < r \big\}.
	\]

	La \underline{boule fermée} (ou \underline{disque fermé}) de centre $(a,b)$ et de rayon $r$ est \[
		\overline{B_{(a,b)}}(r) = \big\{ (x,y)\in \R^2  \mid \|(x,y) - (a,b)\| \le r \big\}.
	\]

	La \underline{sphère} (ou \underline{boule}) de centre $(a,b)$ et de rayon $r$ est \[
		S_{(a,b)}(r) = \partial \overline{B_{(a,b)}}(r) = \big\{ (x,y) \in \R^2  \mid \|(x,y) - (a,b)\| = r \big\}.
	\]
	\index{boule ouverte (de $\R^2$)}
	\index{disque ouverte (de $\R^2$)}
	\index{boule fermée (de $\R^2$)}
	\index{disque fermée (de $\R^2$)}
	\index{boule (de $\R^2$)}
	\index{sphère (de $\R^2$)}
\end{defn}

\begin{figure}[H]
		\centering
		\incfig{boule}
\end{figure}

\begin{rmk}
	On parle de boule en dimension quelconque.
\end{rmk}

\begin{defn}
	Une \underline{partie ouverte} $O$ de $\R^2$ (ou \underline{un ouvert}) si \[
		\forall (x,y) \in O, \exists r > 0, B_{(a,b)}(r) \subset O.
	\]
	Une partie $F$ est \underline{fermée} su $\R^2\setminus F$ est ouverte.
	\index{partie ouverte (de $\R^2$)}
	\index{ouvert (de $\R^2$)}
	\index{partie fermée (de $\R^2$)}
\end{defn}

\begin{figure}[H]
	\centering
	\incfig{partie-ouverte}
\end{figure}

\begin{prop}
	Une boule ouverte est ouverte. Une boule fermée est fermée.
\end{prop}

\begin{figure}[H]
	\centering
	\begin{subfigure}{4cm}
		\centering
		\begin{asy}
			import patterns;

			pair n(pair a) {return a / length(a);}

			add("hatch",hatch(2mm, SW, red));
			size(4cm);

			draw(circle((0,0), 1));
			dot('$(a_0, b_0)$', (0,0), align=S);

			draw((0,0) -- n((-1, 1)), dashed);
			label("$r$", n((-1, 1)) / 2, align=1.5S);

			pair A = n((1,3)) * (2/3);
			real rho = (1 - length(A)) * (2 / 3);

			dot("$(a,b)$", A, red, align=3SE);
			filldraw(circle(A, rho), pattern("hatch"), red);

			label("$O$", n((1,-1))*2.5/3);
		\end{asy}
	\end{subfigure}
	\begin{subfigure}{1cm}
		\centering~\\
	\end{subfigure}
	\begin{subfigure}{5cm}
		\centering
		\begin{asy}
			import patterns;

			pair n(pair a) {return a / length(a);}

			add("hatch",hatch(1mm, SW, red));
			add("hatch2",hatch(3mm, SE, blue));
			size(5cm);

			guide around = (-1.5, -1.5) -- (-1.5, 1.5) -- (2.5, 1.5) -- (2.5, -1.5) -- cycle;

			pair A = n((3, 1)) * 5/3; 
			real rho = 2 / 9;

			picture inter;
			fill(inter, around, pattern("hatch2"));
			fill(inter, circle((0,0), 1), white);
			add(inter);

			draw(circle((0,0), 1));
			dot('$(a_0, b_0)$', (0,0), align=S);

			draw((0,0) -- n((-1, 1)), dashed);
			label("$r$", n((-1, 1)) / 2, align=1.5S);

			dot("$(a,b)$", A, red, align=2SE);
			filldraw(circle(A, rho), pattern("hatch"), red);

			label("$F$", n((1,-1))*2.5/3);
		\end{asy}
	\end{subfigure}
\end{figure}

\begin{prv}
	$\O$ est un ouvert.

	Soit $B$ la boule ouverte de centre $(a_0, b_0) \in \R^2$ et de rayon $r > 0$.

	On pose $\rho = \frac{1}{2}\big(r - \|(a,b) - (a_0,b_0)\|\big)$.
	Montrons que \[
		B_{(a,b)}(\rho) \subset  B_{(a,b)}(r).
	\]

	Soit $(x,y) \in B_{(a,b)}(\rho)$.
	\begin{align*}
		\|(x,y) - (a_0,b_0)\|&= \|(x,y)- (a,b) + (a,b) - (a_0,b_0)\| \\
		&\le \|(x,y) - (a,b)\| + \|(a,b) - (a_0, b_0)\|\\
		&< \rho + \|(a,b) - (a_0, b_0)\| = \frac{1}{2}r + \frac{1}{2} \|(a,b) - (a_0, b_0)\|\\
		&< r
	\end{align*}
	
	Soit $F$ la boule fermée de centre $(a_0, b_0)$ et de rayon $r \ge 0$.

	Soit $(a,b) \not\in F$. On pose \[
		\rho = \frac{1}{2}\big(\|(a,b) - (a_0, b_0)\| - r\big) > 0.
	\]

	Montrons que $B_{(a,b)}(\rho) \subset \R^2\setminus F$.

	Soit $(x,y) \in B_{(a,b)}(\rho)$.

	\begin{align*}
		\|(x,y) - (a_0, b_0)\| &= \|(x,y) - (a,b) + (a,b) - (a_0, b_0)\| \\
		&\ge \big| \underbrace{\|(x,y) - (a,b)\|}_{\le \rho} - \underbrace{\|(a,b) - (a_0, b_0)\|}_{> r} \big|\\
		&\ge \|(a,b) - (a_0, b_0)\|- \|(x,y) - (a,b)\|\\
		&> \|(a,b) - (a_0, b_0)\|- \rho\\
		&> \frac{1}{2} \|(a,b) - (a_0, b_0)\| + \frac{1}{2}r\\
		&> r
	\end{align*}

	donc $(x,y) \not\in F$.
\end{prv}

\begin{exm}
	\begin{enumerate}
		\item $\O$ est ouvert.\\
			$\R^2$ est ouvert.
		\item $\O$ est fermé.\\
			$\R^2$ est fermé.\\
		\item $\big\{(x,0)  \mid x > 0\big\}$ n'est ni ouverte ni fermé.
	\end{enumerate}
\end{exm}

\begin{figure}[H]
	\centering
	\begin{asy}
		size(3cm);

		draw((0, -1) -- (0, 3), Arrow(TeXHead));
		draw((-1, 0) -- (3, 0), Arrow(TeXHead));
		
		draw((0,0) -- (0, 2.97), red);
		draw(circle((0,1.5), 0.5), deepred);
		draw(circle((0,0.5), 0.1), deepred);
	\end{asy}
\end{figure}

\begin{defn}
	Soit $(a,b) \in \R^2$ et $V \in \mathcal{P}(\R^2)$.

	On dit que $V$ est un voisinage de $(a,b)$ s'il existe $r > 0$ tel que \[
		B_{(a,b)}(r) \subset V.
	\]
	\index{voisinage (dans $\R^2$)}
\end{defn}

\begin{prop}
	Un ouvert non vide est un voisinage en chacun de ces points. \qed
\end{prop}

\begin{defn}
	Soit $D \subset \R^2$. Un \underline{point intérieur} de $D$ est un couple $(a,b) \in D$ tel que \[
		\exists r > 0, B_{(a,b)}(r) \subset D.
	\] en d'autres termes, si $D$ est un voisinage de $(a,b)$.

	On note $\mathring D$ l'ensemble des points intérieurs à $D$. C'est \underline{l'intérieur} de $D$.
	\index{point intérieur (dans $\R^2$)}
	\index{intérieur (dans $\R^2$)}
\end{defn}

\begin{prop}
	$\mathring D$ est le plus grand ouvert $O$ de $\R^2$ tel que $O \subset D$.
\end{prop}

\begin{figure}[H]
	\centering
	\incfig{interieur-plus-grand-ouvert}
\end{figure}


\begin{prv}
	Soit $(a,b) \in \mathring D$.

	Par définition, il existe $r > 0$ tel que \[
		B_{(a,b)}(r) \subset D.
	\] Montrons que $B_{(a,b)}(r) \subset \mathring D$.

	Soit $(x,y) \in B_{(a,b)}(r)$. Comme $B_{(a,b)}(r)$ est un ouvert de $\R^2$, il existe $\rho > 0$ tel que \[
		B_{(x,y)}(\rho) \subset B_{(a,b)}(r)
	\] donc $(x,y) \in \mathring D$.

	Donc $\mathring D$ est ouvert, $\mathring D \subset D$.

	Soit $O$ un ouvert de $\R^2$ tel que $O \subset D$. Montrons que $O \subset \mathring D$.

	Soit $(x,y) \in O$. Soit $r > 0$ tel que \[
		B_{(x,y)}(r) \subset O \subset D
	\] donc $(x,y) \in \mathring D$.
\end{prv}

\begin{defn}
	Soit $f: D \subset \R^2 \to \R$, $\ell \in \R$, $(a,b) \in \mathring D$.

	On dit que \underline{$f(x,y)$ tend vers $\ell$ quand $(x,y)$ tend vers $(a,b)$} ou que $\ell$ est \underline{une limite} de $f$ en $(a,b)$ si \[
		\forall \varepsilon > 0, \exists r > 0, \forall (x,y) \in D, \|(x,y) - (a,b)\| < r \implies \left| f(x,y) - \ell \right| \le \varepsilon.
	\] en d'autres termes si \[
		\forall V \in \mathcal{V}_{\ell}, \exists W \in \mathcal{V}_{(a,b)}, \forall (x,y) \in W \cap D, f(x,y) \in V.
	\]
	\index{limite (dans $\R^2$)}
	\index{tendre vers (dans $\R^2$)}
\end{defn}

\begin{prop}
	[unicité de la limite]
	Soit $f: D \to \R$, $(a,b) \in \mathring D$, $\ell_1, \ell_2 \in \R$ telles que $\ell_1$ et $\ell_2$ sont des limites de $f$ en $(a,b)$.

	Alors $\ell_1 = \ell_2$.
\end{prop}

\begin{figure}[H]
	\centering
	\incfig{preuve-unicité-de-la-limite}
\end{figure}

\begin{prv}
	On suppose $\ell_1 < \ell_2$. On pose $\varepsilon = \frac{\ell_2 - \ell_1}{2} > 0$.

	Soit $r_1 > 0$ tel que \[
		f\big(B_{(a,b)}(r_1)\big) \subset ]\ell_1 - \varepsilon, \ell_1 + \varepsilon[.
	\] Soit $r_2 > 0$ tel que \[
		f\big(B_{(a,b)}(r_2)\big) \subset ]\ell_2 - \varepsilon, \ell_2 + \varepsilon [.
	\] On pose $r = \min(r_1, r_2)$ donc \[
		B_{(a,b)}(r_1) \cap B_{(a,b)}(r_2) = B_{(a,b)}(r) \neq \O.
	\] Soit $(x,y) \in B_{(a,b)}(r)$. Alors, \[
		f(x,y) \in ]\ell_1 - \varepsilon, \ell_1 + \varepsilon[ \cap ]\ell_2 - \varepsilon, \ell_2 + \varepsilon[ = \O.
	\] $\lightning$
\end{prv}

\begin{defn}
	Soit $f : D \to \R$, $(a,b) \in \mathring D$.

	On dit que $f$ est \underline{continue} en $(a,b)$ si \[
		f(x,y) \tendsto{(x,y) \to (a,b)}f(a,b).
	\]
	\index{continuité (dans $\R^2$)}
\end{defn}

\begin{prop}
	\underline{Si} $f(x,y) \tendsto{(x,y) \to (a,b)} \ell$ \\
	\underline{alors} $\begin{cases}
		f(x,b) \tendsto{x \to a} \ell\\
		f(a,y) \tendsto{y \to b} \ell.\\
	\end{cases}$
\end{prop}

\begin{prv}~\\
	\begin{figure}[H]
		\centering
		\incfig{limite-x-en-a-et-y-en-b}
	\end{figure}
\end{prv}

\underline{Contre-exemple} : exercice 3.

\begin{exm}
	\begin{enumerate}
		\item $f : \begin{array}{rcl}
				\R^2 &\longrightarrow& \R \\
				(x,y) &\longmapsto& x
			\end{array}$ limite en $(0,0)$ ?

			Soit $\varepsilon > 0$. On pose $r = \varepsilon$. \[
				\forall (x,y) \in B_{(0,0)}(r),
				\left| f(x,y) \right| = \left| x \right| \le \|(x,y)\| < r = \varepsilon
			\] Donc $f(x,y) \tendsto{(x,y) \to (a,b)} 0$.
		\item limite $f : \begin{array}{rcl}
				\R^2 &\longrightarrow& \R \\
				(x,y) &\longmapsto& x^3
			\end{array}$ en $(0,0)$ ?

			Soit $\varepsilon > 0$. On pose $r = \sqrt[3]{r} > 0$. \[
				\forall (x,y) \in B_{(0,0)}(r),
				\left| f(x,y) \right| = \left| x^3 \right| \le \|(x,y)\|^3 < r^3 = \varepsilon.
			\]
		\item limite de $f : \begin{array}{rcl}
			\R^2 &\longrightarrow& \R \\
			(x,y) &\longmapsto& x^3y^2
		\end{array}$ en $(0,0)$ ?

		Soit $\varepsilon > 0$. On pose $r = \sqrt[5]{\varepsilon} > 0$. \[
			\forall (x,y) \in B_{(0,0)}(r), \left| f(x,y) \right| = \left| x^3 y^2 \right| \le \|(x,y)\|^3 \|(x,y)\|^2 < r^5 = \varepsilon.
		\]
	\end{enumerate}
\end{exm}

\begin{defn}
	Soient $D \subset \R^2$ et $(x,y) \in \R^2$.

	\begin{figure}[H]
    \centering
    \incfig{point-adhérent}
	\end{figure}
	
	On dit que $(x,y)$ est \underline{adhérent} à $D$ si \[
		\forall r > 0, B_{(x,y)}(r) \cap D \neq \O.
	\] L'ensemble des points adhérents à $D$ est noté $\overline{D}$. On dit que $\overline{D}$ est \underline{l'adhérence} de $D$.
	\index{point adhérent (dans $\R^2$)}
	\index{adhérent (dans $\R^2$)}
\end{defn}

\begin{defn}
	Soit $f: D \subset \R^2 \to \R$ et $(a,b) \in \overline{D}$, $\ell \in \R$. On dit que $f$ tend vers $\ell$ quand $(x,y)$ tend vers $(a,b)$ si \[
		\forall \varepsilon > 0, \exists r > 0, \forall (x,y) \in B_{(a,b)}(r) \cap D,
		\left| f(x,y) - \ell \right| \le \varepsilon.
	\]
	\index{limite (dans $\R^2$)}
	\index{tendre vers (dans $\R^2$)}
\end{defn}

\begin{prop}
	\begin{enumerate}
		\item Dans ce contexte, il y a unicité de la limite
		\item La limite d'une somme, d'un produit, d'un quotien, d'une composée se comporte comme dans le cas d'une seule variable.
		\item Soit $f: D \to \R$ continue. Soient $g: I \to \R$ et $h: I \to \R$ continues telles que \[
			\forall t \in I, \big(g(t), h(t)\big) \in D.
		\] Alors \[
			t \in I \mapsto f\big(g(t), h(t)\big) \in \R
		\] est continue.
	\end{enumerate}
\end{prop}

\begin{figure}[H]
	\centering
	\begin{asy}
		import three;
		import graph3;
		size(5cm);

		settings.render = 0;
		settings.prc = false;
		currentprojection = obliqueX;

		draw(O -- X, Arrow3(TeXHead2));
		draw(O -- Y, Arrow3(TeXHead2));
		draw(O -- Z, Arrow3(TeXHead2));

		triple f(real x, real y, real z = 0) { return (x,y,cos(x - 0.5) * cos(y - 0.5)/1.2 + 0.15); }

		real inc = 1 / 5;

		for(real x = 0; x <= 1; x += inc) {
			draw(graph(
				new real(real t) { return x; }, // x
				new real(real y) { return y; }, // y
				new real(real y) { return f(x,y).z; }, // z
				0, 1
			), gray);
		}

		for(real y = 0; y <= 1; y += inc) {
			draw(graph(
				new real(real x) { return x; }, // x
				new real(real t) { return y; }, // y
				new real(real x) { return f(x,y).z; }, // z
				0, 1
			), gray);
		}

		path3 path1 = (0.3, 0.2, 0) .. (0.5, 0.5, 0) .. (0.6, 0.7, 0) .. (0.9, 0.8, 0);
		path3 path2 = (0.3, 0.8, 0) .. (0.5, 0.5, 0) .. (0.6, 0.3, 0) .. (0.9, 0.2, 0);
		path3 pathA = f(0.3, 0.2, 0) .. f(0.5, 0.5, 0) .. f(0.6, 0.7, 0) .. f(0.9, 0.8, 0);
		path3 pathB = f(0.3, 0.8, 0) .. f(0.5, 0.5, 0) .. f(0.6, 0.3, 0) .. f(0.9, 0.2, 0);

		draw(path1, red, Arrow3(TeXHead2, position=0.5));
		draw(pathA, red, Arrow3(TeXHead2, position=0.5));
		draw(path2, deepcyan, Arrow3(TeXHead2, position=0.3));
		draw(pathB, deepcyan, Arrow3(TeXHead2, position=0.3));

		dot((0.5, 0.5, 0));
		dot(f(0.5, 0.5, 0));
		draw((0.5, 0.5, 0) -- f(0.5, 0.5, 0), dashed);
	\end{asy}
\end{figure}


	\part{Transpositions}

\begin{defn}
	Une \underline{transposition} est un cycle de longueur 2 : $\begin{pmatrix}
		a&b
	\end{pmatrix}$ avec $a \neq b$.
	\index{transposition (permutation)}
\end{defn}

\begin{exm}
	Avec $n = 5$ et $\gamma = \begin{pmatrix}
		2&4&1
	\end{pmatrix}$.

	\begin{figure}[H]
		\centering

		\begin{asy}
			size(5cm);

			real rho = 0.15; // circles

			void draw_cycle(pair O, real r ...int[] nums) {
				int n = nums.length;
				real eps = (15 / r) * 0.8;

				for(int i = 0; i < n; ++i) {
					real theta_1 = (360/n) * (i+1);
					real theta_2 = (360/n) * i;

					pair C = O + dir(theta_2) * r;

					draw(circle(C, rho));
					label("$" + string(nums[i]) + "$", C);
					draw(arc(O, r, theta_2+eps, theta_1-eps), Arrow(TeXHead));
				}
			}

			draw_cycle((-1,0), 0.8, 1, 2, 4);
			draw_cycle((1,0), 0.3, 3);
			draw_cycle((2,0), 0.3, 5);
		\end{asy}
	\end{figure}

	\[
		\gamma = \begin{pmatrix}
			1&4
		\end{pmatrix} \begin{pmatrix}
			1&2
		\end{pmatrix}
	\]

	Avec $n = 6$ et $\gamma = \begin{pmatrix}
		1&3&5&6&2
	\end{pmatrix} = \begin{pmatrix}
		1&2&3&4&5&6\\
		3&1&5&4&6&2
	\end{pmatrix}$.

	Donc, \[
		\gamma = \begin{pmatrix}
			1&2
		\end{pmatrix} \begin{pmatrix}
			1&6
		\end{pmatrix} \begin{pmatrix}
			1&5
		\end{pmatrix} \begin{pmatrix}
			1&3
		\end{pmatrix}
	\] 
	\[
		\begin{pmatrix}
			1&2&3&4&5&6\\
			3&2&1&4&5&6\\
			3&2&5&4&1&6\\
			3&2&5&4&6&1\\
			3&1&5&4&6&2\\
		\end{pmatrix}
	\]

	Et, \[
		\gamma = \begin{pmatrix}
			1&3
		\end{pmatrix} \begin{pmatrix}
			2&3
		\end{pmatrix} \begin{pmatrix}
			3&5
		\end{pmatrix} \begin{pmatrix}
			5&6
		\end{pmatrix} 
	\]

	\[
		\begin{pmatrix}
			1&2&3&4&5&6\\
			1&2&3&4&6&5\\
			1&2&5&4&6&3\\
			1&3&5&4&6&2\\
			3&1&5&4&6&2\\
		\end{pmatrix} 
	\] 
\end{exm}

\begin{exm}
	\[
		\begin{pmatrix}
			1&4
		\end{pmatrix} = \begin{pmatrix}
			1&2
		\end{pmatrix} \begin{pmatrix}
			2&3
		\end{pmatrix} \begin{pmatrix}
			3&4
		\end{pmatrix} \begin{pmatrix}
			2&3
		\end{pmatrix} \begin{pmatrix}
			1&2
		\end{pmatrix}
	\]
	On n'a pas toujours le même nombre de transpositions mais la parité du nombre reste la même (proposition plus loin).
\end{exm}

\begin{thm}
	Toute permutation se décompose en produit de transpositions.
\end{thm}

\begin{prv}
	Soit $\gamma = \begin{pmatrix}
		a_1&\cdots&a_k
	\end{pmatrix}$ un $k$-cycle.

	On remarque que
	\[
		\gamma = \begin{pmatrix}
			a_1&a_k
		\end{pmatrix} \cdots \begin{pmatrix}
			a_1&a_4
		\end{pmatrix} \begin{pmatrix}
			a_1&a_3
		\end{pmatrix} \begin{pmatrix}
			a_1&a_2
		\end{pmatrix}
	\] C'est un produit de transpositions.
\end{prv}

\begin{exm}
	Avec $n = 10$ et $\sigma = \begin{pmatrix}
		1&2&3&4&5&6&7&8&9&10\\
		9&8&1&7&2&3&4&5&10&6
	\end{pmatrix}$.

	On a
	\begin{align*}
		\sigma &= \begin{pmatrix}
			1&9&10&6&3
		\end{pmatrix} \begin{pmatrix}
			2&8&5
		\end{pmatrix} \begin{pmatrix}
			4&7
		\end{pmatrix}\\
		&= \begin{pmatrix}
			1&3
		\end{pmatrix} \begin{pmatrix}
			1&6
		\end{pmatrix} \begin{pmatrix}
			1&10
		\end{pmatrix} \begin{pmatrix}
			1&9
		\end{pmatrix} \begin{pmatrix}
			2&5
		\end{pmatrix} \begin{pmatrix}
			2&8
		\end{pmatrix} \begin{pmatrix}
			4&7
		\end{pmatrix} \\
	\end{align*}

	Vérification : \[
		\begin{pmatrix}
			1&2&3&4&5&6&7&8&9&10\\
			1&2&3&7&5&6&4&8&9&10\\
			1&8&3&7&5&6&4&2&9&10\\
			1&8&3&7&2&6&4&5&9&10\\
			9&8&3&7&2&6&4&5&1&10\\
			9&8&3&7&2&6&4&5&10&1\\
			9&8&3&7&2&1&4&5&10&6\\
			9&8&1&7&2&3&4&5&10&6\\
		\end{pmatrix} 
	\] 
\end{exm}

	\part{Familles orthogonales}

\begin{thm}[Pythagore]
	Soit $(x,y) \in E^2$. \[
		\|x+y\|^2 = \|x\|^2 + \|y\|^2 \iff x \perp y
	.\]
	\begin{figure}[H]
		\centering
		\begin{asy}
			size(4cm);
			pair u = (1, 0.5);
			pair v = 1.5 * (0, 1) * u;
			draw((0,0)--u, Arrow(TeXHead));
			label("$x$", u/2, align=S);
			draw(u--v+u, Arrow(TeXHead));
			label("$y$", u + v/2, align=NE);
			draw((0,0) -- u + v, Arrow(TeXHead));
			draw(u + v / 7.5 -- u + v / 7.5 - u / 5 -- u - u / 5 -- u -- cycle);
		\end{asy}
	\end{figure}
\end{thm}

\begin{prv}
	\[
		\|x + y\|^2 = \|x\|^2 + \|y\|^2 \iff 2\left<x \mid y \right> = 0 \iff x \perp y
	.\]
\end{prv}

\begin{defn}
	Soit $(e_i)_{i\in I}$ une famille de vecteurs. On dit que cette famille est \underline{orthogonale} si \[
		\forall i \neq j\, e_i \perp e_j
	.\] Si, en plus, on a \[
		\forall i \in I,\,\|e_i\| = 1,
	\] alors on dit que la famille est \underline{orthonormale} ou \underline{orthonormée}.
	\index{famille orthogonale}
	\index{famille orthonormale}
	\index{famille orthonormée}
\end{defn}

\begin{prop}[Pythagore]
	Soit $(e_1, \ldots, e_n)$ une famille orthogonale. Alors \[
		\left\| \sum_{i=1}^n e_i \right\|^2 = \sum_{i=1}^n \|e_i\|^2
	.\]
\end{prop}

\begin{thm}
	Toute famille orthogonale de vecteurs non nuls est libre.
\end{thm}

\begin{prv}
	Soit $(e_i)_{i\in I}$ une famille orthogonale telle que \[
		\forall i \in I,\,e_i \neq 0_E
	.\] Soit $n \in \N^*$, $(\lambda_1, \ldots, \lambda_n) \in \R^n$. On suppose \[
		\sum_{k=1}^n \lambda_k e_{i_k} = 0_E
	.\] Soit $j \in \left\llbracket 1,n \right\rrbracket$.
	\begin{align*}
		0 &= \left<\sum_{k=1}^n \lambda_k e_{i_k}  \mid e_{i_j} \right>\\
		&= \sum_{k=1}^n \lambda_k \left<e_{i_k}  \mid e_{i_j} \right> \\
		&= \lambda_j \underbrace{\|e_{i_j}\|^2}_{\neq 0} \\
	\end{align*}
	donc $\lambda_j = 0$.
\end{prv}

\begin{algo}[Orthonormalisation de Gran--Schmidt]
	On suppose $E$ de dimension finie. Soit $\mathcal{B} = (e_1, \ldots, e_n)$ une base de $E$.

	\begin{itemize}
		\item\underline{\it Étape 1}: On pose $v_1 = \frac{e_1}{\|e_1\|}$ de sorte que $\|v_1\| = 1$.
		\item\underline{\it Étape 2} : On pose \[
				u_2 = e_2 - \left<e_2  \mid v_1 \right> v_1
			.\] Ainsi,
			\begin{align*}
				\left<u_2 \mid v_1 \right> &= \big<e_2 - \left<e_2 \mid v_1 \right> v_1  \mid v_1 \big>\\
				&= \left<e_2 \mid v_1 \right> - \left<e_2 \mid v_1 \right> \left<v_1 \mid v_1 \right> \\
				&= 0. \\
			\end{align*}
			On pose $v_2 = \frac{u_2}{\|u_2\|}$ donc $v_2 \perp v_1$ et $\|v_2\| = 1$.
		\item\underline{\it Étape 3} : On pose \[
				u_2 = e_3 - \left<e_3 \mid v_1 \right>v_1 - \left<e_3 \mid v_2 \right>v_2
			.\] Ainsi,
			\begin{align*}
				\left<u_3  \mid v_1 \right> &= \left<e_3  \mid v_1 \right> - \left<e_3 \mid v_1 \right>\underbrace{\left<v_1 \mid v_1 \right>}_{=1} - \left<e_3 \mid v_2 \right>\underbrace{\left<v_2 \mid v_1 \right>}_{=0} \\
				&= 0 \\
			\end{align*}
			et 
			\begin{align*}
				\left<u_3 \mid v_2 \right> &= \left<e_3  \mid  v_2 \right> - \left<e_3 \mid v_1 \right> \underbrace{\left<v_1 \mid v_2 \right>}_{=0} - \left<e_3 \mid v_2 \right> \underbrace{\left<v_2 \mid v_2 \right>}_{=1}\\
				&= 0. \\
			\end{align*}
			On pose $v_3 = \frac{u_3}{\|u_3\|}$ de sorte que $v_3 \perp v_1$, $v_3 \perp v_2$ et $\|v_3\| = 1$.
		\item\underline{\it Étape $i+1$}: On pose \[
			u_{i+1} = e_{i+1} - \sum_{k=1}^i \left<e_{i+1}  \mid v_k \right> v_k
		.\] Ainsi, pour tout $j \in \left\llbracket 1,i \right\rrbracket,$ on a
		\begin{align*}
			\left<u_{i+1}  \mid v_j \right> &= \left<e_{i+1}  \mid v_j \right> - \sum_{k=1}^i \left<e_{i+1} \mid v_k \right> \left<v_k \mid v_j \right> \\
			&= \left<e_{i+1} \mid v_j \right> - \left<e_{i+1} \mid v_j \right> \|v_j\|^2 \\
			&= 0. \\
		\end{align*}
		On pose $v_{i+1} = \frac{u_{i+1}}{\|u_{i+1}\|}$.
	\end{itemize}
\end{algo}

\begin{exm}
	Avec $E = \R_3[X]$, $\left<P \mid Q \right> = \int_{0}^{1} P(t)\,Q(t)~\mathrm{d}t$ et $\mathcal{B} = (1, X, X^2, X^3)$.
	\begin{enumerate}
		\item $\|1\|^2 = \left<1 \mid 1 \right> = \int_{0}^{1} 1~\mathrm{d}t = 1$ et donc $v_1 = 1$.
		\item $u_2 = X - \left<X  \mid v_1 \right>v_1$. Or, $\left<X \mid v_1 \right> = \int_{0}^{1} t~\mathrm{d}t = \frac{1}{2}$. D'où $u_2 = X - \frac{1}{2}$.
			\begin{align*}
				\|u_2\|^2 &= \int_{0}^{1} \left( t - \frac{1}{2} \right)^2~\mathrm{d}t \\
				&= \int_{0}^{1} \left( t^2 - t + \frac{1}{4} \right)~\mathrm{d}t \\
				&= \frac{1}{3} - \frac{1}{2} + \frac{1}{4} \\
				&= \frac{1}{12} \\
			\end{align*} On en déduit que $v_2 = \sqrt{12}\left( X - \frac{1}{2} \right)$.
		\item $u_3 = X^2 - \left<X^2 \mid v_1 \right>v_1 - \left<X^2 \mid v_2 \right>v_2$.
			On a \[
				\left<X^2 \mid v_1 \right> = \int_{0}^{1} t^2~\mathrm{d}t = \frac{1}{3}
			\] et
			\begin{align*}
				\left<X^2 \mid v_2 \right> &= \sqrt{12} \int_{0}^{1} t^2\left( t - \frac{1}{2} \right)~\mathrm{d}t \\
				&= \frac{\sqrt{12}}{12}. \\
			\end{align*}
			D'où
			\begin{align*}
				u_3 &= X^2 - \frac{1}{3} - \frac{\sqrt{12}}{12}\sqrt{12} \left( X - \frac{1}{2} \right)\\
				&= X^2 - \frac{1}{3} - X + \frac{1}{2} \\
				&= X^2 - X + \frac{1}{6}. \\
			\end{align*}
			\begin{align*}
				\|u_3\|^2 &= \int_{0}^{1} \left( t^2 - t + \frac{1}{6} \right)~\mathrm{d}t\\
				&= \int_{0}^{1} \left( t^4 + t^2 + \frac{1}{36} - 2t^3 + \frac{t^2}{3} - \frac{t}{3} \right) ~\mathrm{d}t \\
				&= \frac{1}{5} + \frac{1}{3} + \frac{1}{36} - \frac{1}{2} + \frac{1}{9} - \frac{1}{6} \\
				&= \frac{36 + 60 + 5 - 90 + 20 - 30}{180} \\
				&= \frac{1}{180} \\
			\end{align*}
			On en déduit que \[
				v_3 = 6\sqrt{5}\left( X^2 - X + \frac{1}{6} \right).
			\]
		\item Exercice : calculer $v_4$.
	\end{enumerate}
\end{exm}

\begin{prop}
	Soit $\mathcal{B} = (e_1, \ldots, e_n)$ une base de $E$ et $\mathcal{C}$ la base obtenue par le procédé d'orthonormalisation de Gram--Schmidt. Alors, \[
		\forall i \in \left\llbracket 1,n \right\rrbracket,\,\Vect(e_1,\ldots, e_i) = \Vect(v_1, \ldots, v_i)
	.\]\qed
\end{prop}

\begin{exm}[orthogonalisation]
	\begin{itemize}
		\item $u_1 = 1$.
		\item
			\begin{align*}
				\begin{rcases*}
					u_2 \in \Vect(e_1, e_2)\\
					u_2 \perp u_1
				\end{rcases*}
				\iff& \begin{cases}
					u_2 = ae_1 + be_2\quad (a,b) \in \R^2\\
					\left<u_1 \mid u_2 \right> = 0
				\end{cases}\\
				\iff& \begin{cases}
					u_2 = a + bX\\
					\int_{0}^{1} (a+bt)~\mathrm{d}t = 0.
				\end{cases}\\
			\end{align*}
			\begin{align*}
				\int_{0}^{1} (a+bt)~\mathrm{d}t = 0 \iff& a + \frac{b}{2} = 0\\
				\iff& a = -\frac{b}{2}\\
				\iff& u_2 = -\frac{b}{2} + bX.
			\end{align*}
			Par exemple, $u_2 = -1 + 2X$.
		\item $\begin{cases}
				u_3 \in \Vect(e_1, e_2, e_3)\\
				u_3 \perp u_1\\
				u_3 \perp u_2
			\end{cases}$

			On pose $u_3 = a + bX + cX^2$ avec $(a,b,c)\in \R^3$.
			\begin{align*}
				\begin{rcases*}
					\int_{0}^{1} \left( a+bt + ct^2 \right)~\mathrm{d}t = 0\\
					\int_{0}^{1} \left(a + bt+ct^2\right)(2t - 1)~\mathrm{d}t = 0
				\end{rcases*} \iff& \begin{cases}
					a + \frac{b}{2} + \frac{c}{3} = 0\\
					\int_{0}^{1} \left( 2ct^3 + (-c + 2b)t^2 + (2a - b)t - a \right) ~\mathrm{d}t = 0
				\end{cases}\\
				\iff& \begin{cases}
					a + \frac{b}{2} + \frac{c}{3} = 0\\
					\frac{c}{2} + \frac{2b - c}{3} + \frac{2\cancel{a} - b}{2} - \cancel{a} = 0
				\end{cases}\\
				\iff&  \begin{cases}
					a = -\frac{b}{2} - \frac{c}{3} = \frac{c}{2} - \frac{c}{3} = \frac{c}{6}\\
					b = -c.
				\end{cases}
			\end{align*}
			On en déduit que \[
				u_3 = 1 - 6X + 6X^2
			.\]
	\end{itemize}
\end{exm}

\begin{crlr}[théorème de la base orthonormée incomplète] Soit $(e_1, \ldots, e_k)$ une base orthonormée d'un espace euclidien. On peut trouver $e_{k+1},\ldots,e_n$ tels que $(e_1, \ldots, e_k, e_{k+1},\ldots,e_n)$ soit une base orthonormée de $E$.
\end{crlr}

\begin{prv}
	On sait que $(e_1, \ldots, e_k)$ est libre. On complète $(e_1, \ldots, e_k)$ en une base $\mathcal{B}$ de $E$. On orthonormalise $\mathcal{B}$ : on obtient une base orthonormée $\mathcal{C}$ de $E$. En détaillant l'algorithme de Gram--Schmidt, on s'aper\c coit que les $k$ premiers vecteurs de $\mathcal{C}$ sont ceux de $\mathcal{B}$.
\end{prv}

\begin{thm}
	Soit $E$ un espace euclidien et $\mathcal{B} = (e_1, \ldots, e_n)$ une base orthonormée de $E$. Soit $(x,y) \in E^2$. On pose $(x_1, \ldots, x_n) \in \R^n$ et $(y_1, \ldots, y_n) \in \R^n$ tels que \[
		x = \sum_{i=1}^n x_i e_i \qquad\qquad y = \sum_{i=1}^n y_i e_i
	.\] Alors \[
		\left<x \mid y \right> = \sum_{i=1}^n x_i y_i
	.\]
	\vspace{3mm}
	Soit $X = \mat{x_1\\\vdots\\x_n}$ et $Y = \mat{y_1\\ \vdots \\ y_n}$. Alors, \[
		\left<x \mid y \right> = X^\T\,Y
	.\]
\end{thm}

\begin{prv}
	\begin{align*}
		\left<x \mid y \right> &= \left<\sum_{i=1}^n x_ie_i  \mid y \right>\\
		&= \sum_{i=1}^n x_i \left<e_i  \mid y \right> \\
		&= \sum_{i=1}^n x_i \left<e_i  \mid \sum_{j=1}^n y_j e_j \right> \\
		&= \sum_{i=1}^n x_i \sum_{j=1}^n y_j \underbrace{\left<e_i \mid e_j \right>}_{\delta_i^j} \\
		&= \sum_{i=1}^n x_i y_i. \\
	\end{align*}
\end{prv}

\begin{prop}
	Soit $E$ un espace euclidien et $\mathcal{B} = (e_1, \ldots, e_n)$ une base orthonormée de $E$. Alors, \[
		\forall x \in E,\,x = \sum_{i=1}^n \left<x \mid e_i \right>e_i
	.\]
\end{prop}

\begin{prv}
	Soit $x \in E$. On pose \[
		x = \sum_{i=1}^n x_i e_i
	\] avec $(x_1, \ldots, x_n) \in \R^n$. Soit $j \in \left\llbracket 1,n \right\rrbracket$. On a
	\begin{align*}
		\left<x \mid e_j \right> &= \left<\sum_{i=1}^n x_i e_i  \mid e_j \right>\\
		&= \sum_{i=1}^n x_i \left<e_i \mid e_j \right> \\
		&= x_j. \\
	\end{align*}
\end{prv}

	\part{Lois de composition}

\begin{defn}
	Une \underline{loi de composition interne} \index{loi de composition interne} est une application $f$ de $E \times E$ dans $E$.
	
	On la note $x * y$ au lieu de $f(x,y)$ (on est libre de choisir le symbôle).
\end{defn}

\begin{defn}
	Soit $E$ un ensemble muni d'une loi de composition interne $\boxtimes$.

	On dit que $\boxtimes$ est \underline{associative} \index{associativité (loi de composition interne)} si \[
		\forall (x,y,z) \in E^3,\;(x\boxtimes y)\boxtimes z = x \boxtimes (y \boxtimes z).
	\] Dans ce cas, on écrit plutôt $x \boxtimes y \boxtimes z$.
\end{defn}

\begin{exm}
	\begin{itemize}
		\item $+$ et $\times $ dans $\C$ sont associatives;
		\item $ \circ$ est associative;
		\item  la multiplication matricielle est aussi associative.
	\end{itemize}
\end{exm}

\begin{defn}
	On dit que $\boxtimes$ est \underline{commutative} \index{commutativité (loi de composition interne)} si \[
		\forall (x,y) \in E^2, x\boxtimes y = y\boxtimes x.
	\]
\end{defn}

\begin{exm}
	\begin{itemize}
		\item $+$ et $\times $ dans $\C$ sont commuatives;
		\item $ \circ $ n'est pas commutative;
		\item  la multiplication matricielle n'est pas commutative.
	\end{itemize}
\end{exm}

\begin{defn}
	Soit $e \in E$. On dit que $e$ est un
	\begin{itemize}
		\item \underline{élément neutre à gauche}\index{élément neutre à gauche (loi de composition interne)} si  \[
				\forall x \in E,\; e\boxtimes x = x;
			\]
		\item \underline{élément neutre à droite}\index{élément neutre à droite (loi de composition interne)} si  \[
				\forall x \in E,\; x\boxtimes e = x;
			\]
		\item \underline{élément neutre}\index{élément neutre (loi de composition interne)} si  \[
				\forall x \in E,\; e\boxtimes x = x\boxtimes e = x.
			\]
	\end{itemize}
\end{defn}

\begin{prop}
	Sous reserve d'existence, il y a unicité de l'élément neutre.
\end{prop}

\begin{prv}
	Soient $e$ et $e'$ deux éléments neutre.
	\begin{itemize}
		\item $e \boxtimes e' = e'$ car $e$ est neutre,
		\item $e \boxtimes e' = e$ car $e'$ est neutre.
	\end{itemize} On a donc $e = e'$.
\end{prv}

\begin{axm}[axiome du choix]
	Soit $E$ un ensemble non vide. Il existe $f : \mathcal{P}(E) \setminus \{\O\} \to E$ telle que \[
		\forall A \in \mathcal{P}(E) \setminus \{\O\},\; f(A) \in A.
	\]
\end{axm}

\begin{defn}
	Soit $f: E \to F$. Le \underline{graphe} \index{graphe (application)} de $f$ est \[
		\Big\{\big(x,f(x)\big)  \mid x \in E\Big\} \subset E \times F.
	\]
\end{defn}

\begin{prop}
	Soit $G \subset E\times F$. $G$ est le graphe d'une application si et seulement si \[
		\forall x \in E,\,\exists! y \in F,\, (x,y) \in G.
	\]
\end{prop}

\begin{prv}
	\begin{itemize}
		\item[``$\implies$''] par définition d'une application
		\item[``$\impliedby$''] On pose $f(x)$ le seul élément $y$ de $F$ qui vérifie $(x,y) \in G$. Alors $f \in F^E$ et son graphe vaut $G$.
	\end{itemize}
\end{prv}

\begin{defn}
	Soit $A \in \mathcal{P}(E)$. L'\underline{indicatrice}\index{indicatrice (ensemble)} de $A$ est \begin{align*}
		\mathbbm{1}_A: E &\longrightarrow \{0,1\} \\
		x &\longmapsto \begin{cases}
			1 &\text{ si } x \in A,\\
			0 & \text{ si } x \not\in A.
		\end{cases}
	\end{align*}
\end{defn}

\begin{exm}
	\begin{enumerate}
		\item Dans $\C$, le neutre de $+$ est $0$ et le neutre de $\times$ est $1$.
		\item Dans $E^E$, le neutre de $ \circ $ est $\id_E$.
		\item Dans $\mathcal{M}_n(\C)$ (l'ensemble des matrices carrées $n \times n$ à valeurs dans $\C$), le neutre de $\times $ est $I_n$ : \[
				I_n =
				\begin{pNiceMatrix}
					1&&(0)\\
					&\Ddots&\\
					(0)&&1
				\end{pNiceMatrix}
			\] 
	\end{enumerate}
\end{exm}

\begin{defn}
	Soit $E$ un ensemble muni d'une loi de composition interne $\boxtimes$ et $x \in E$.

	\begin{enumerate}
		\item On dit que $x$ est \underline{simplifiable à gauche}\index{simplifiabilité à gauche} si \[
				\forall (y,z) \in E^2,\,(x\boxtimes y = x \boxtimes z) \implies x = z.
			\] et que $x$ est \underline{simplifiable à droite}\index{simplifiabilité à droite} si \[
				\forall (y,z) \in E^2,\,(y\boxtimes x = z \boxtimes y) \implies x = z.
			\]
		\item On dit que $x$ est \underline{symétrisable à gauche}\index{symétrisabilité à gauche} s'il exiiste $y \in E$ tel que $y\boxtimes x = e$ où $e$ est l'élément neutre de $\boxtimes$.

			De même, on dit que $x$ est \underline{symétrisable à droite}\index{symétrisabilité à droite} s'il existe $y \in E$ tel que $x \boxtimes y = e$.

			On dit que $x$ est \underline{symétrisable}\index{symétrisabilité} s'il est symétrisable à gauche et à droite, donc s'il existe $y \in E$ tel que $x \boxtimes y = y \boxtimes x = e$.
	\end{enumerate}
\end{defn}

\begin{exm}
	$E = \N$ muni de la loi $+$, tous les éléments de $E$ sont simplifiables. $0$ est le seuele élément de $E$ symétrisable.
\end{exm}

\begin{prop}
	Avec les notations précédentes, si $\boxtimes$ est associative, et $x$ est symétrisable, alors $x$ est simplifiable.
\end{prop}

\begin{prv}
	Soient $y, z \in E$.
	\begin{itemize}
		\item On suppose $x \boxtimes y = x \boxtimes z$. Soit $a \in E$ tel que $a\in E$ tel que $a \boxtimes x = e$. Alors \[
				a \boxtimes (x\boxtimes y) = a \boxtimes (x \boxtimes z).
			\] Or,
			\begin{align*}
				a \boxtimes (x \boxtimes y) &= (a \boxtimes x) \boxtimes y \\
				&= e \boxtimes y \\
				&= y. \\
			\end{align*}

			De même, $a \boxtimes (x \boxtimes z) = z$.

			Donc $y = z$.
		\item De même, si $y \boxtimes x = z \boxtimes x$, on ``multiplie'' $x$ à droite par $a$ et on obtient $y = z$.
	\end{itemize}
\end{prv}

\begin{prop-defn}
	On suppose $\boxtimes$ associative. Soit $x \in E$ symétrisable. Alors \[
		\exists ! y \in E,\; x \boxtimes y = y \boxtimes x = e.
	\] On dit que $y$ est le \underline{symétrique}\index{symétrique (loi de composition interne)} de $x$ et on le note $y = x^*$.
\end{prop-defn}

\begin{prv}
	Soeint $x,y,z \in E$ tels que \[
		\begin{cases}
			 x \boxtimes y = y \boxtimes x = e\\
			 x \boxtimes z = z \boxtimes x = e\\
		\end{cases}
	\] Alors, $x \boxtimes y = x \boxtimes z$ et, en simplifiant par $x$, on a $y = z$.
\end{prv}

\begin{exm}
	Les fonctions symétrisables de $(E^E,  \circ)$ sont les bijections et le symétrique d'une bijection est sa réciproque.
\end{exm}

\begin{rmk}
	\begin{enumerate}
		\item Si la loi est notée $+$, on parle d'\underline{opposé}\index{opposé (loi de composition interne)} plutôt que de symétrique et on le note $-x$ au lieu de $x^*$.
			L'élément neutre est noté $0_E$.
		\item Si la loi est notée $\times$, on parle d'élément \underline{inversible}\index{inversibilité (loi de composition interne)} au lieu de symétrisable, d'\underline{inverse}\index{inverse (loi de composition interne)} au lieu de symétrique et on note $x^{-1}$ au lieu de $x^*$. On note le neutre $1_E$.
	\end{enumerate}
\end{rmk}

\begin{exo}
	Soient $x,y \in E = \R^+_*$. On définit la loi de composition interne $\oplus$ : \[
		x \oplus y = \frac{1}{\frac{1}{x}\oplus \frac{1}{y}}.
	\] Cette loi peut-être utile en physique pour le calcul de résistances équivalentes en parallèles.
	\begin{itemize}
		\item {\sc Associativité} : soient $x,y,z \in E$.

			D'une part, on a \[
				x \oplus (y \oplus z) = \frac{1}{\frac{1}{x} + \frac{1}{\frac{1}{\frac{1}{x}+ \frac{1}{y}}}} = \frac{1}{\frac{1}{x}+\frac{1}{y}+\frac{1}{z}}.
			\] D'autre part, on a \[
			(x \oplus y) \oplus z = \frac{1}{\frac{1}{\frac{1}{\frac{1}{x}+\frac{1}{y}}}+\frac{1}{z}} = \frac{1}{\frac{1}{x}+ \frac{1}{y}+\frac{1}{z}}.
			\] La loi $\oplus$ est associative.
		\item {\sc Commutativité} : soient $x, y \in E$. \[
				x \oplus y = \frac{1}{\frac{1}{x}+\frac{1}{y}} = \frac{1}{\frac{1}{y}+\frac{1}{x}} = y\oplus x.
			\] Donc la loi $\oplus$ est commutative.
		\item {\sc Élément neutre} : soit $e$ l'élément neutre de $\oplus$. \[
				\forall x \in E,\; x \oplus e = e \oplus x = x.
			\] Comme la loi est commutative, seul l'égalité $x \oplus e = x$ est utile.

			Soit $x \in E$. On a donc $\frac{1}{\frac{1}{x}+\frac{1}{e}}=x$ donc $\frac{ex}{e+x}=x$ donc $ex = x(e+x)$ et donc $\cancel{ex} = \cancel{ex} + x^2$. On en déduit que $x^2 = 0$, ce qui n'est pas possible car $x \in \R^+_*$. Donc, il n'y a pas d'élément neutre pour $\oplus$.
	\end{itemize}
\end{exo}

	\part{Divers}

\begin{defn}
	Soient $E$ et $F$ deux ensembles. Un \underline{couple}\index{couple} $(x,y)$ est la donnée d'un élément $x$ de $E$ et d'un élément $y$ de $F$ où \[
		\forall x,x' \in E,\,\forall y,y' \in F,\qquad (x,y) = (x',y') \iff \begin{cases}
			x=x',\\
			y=y'.
		\end{cases}
	\] On note $E \times F$ l'ensemble des couples; c'est le \underline{produit cartésien}\index{produit cartésion (ensembles)} de $E$ et $F$.
\end{defn}

\begin{exm}
	$D \times [0,1]$ est un cylindre plein où $D$ est le disque unité fermé i.e. \[
		D = \Big\{(x,y) \in \R^2 \mid x^2+y^2 \le 1\Big\}.
	\]
	\begin{figure}[H]
		\centering
		\begin{subfigure}[b]{3cm}
			\centering
			\begin{asy}
				size(3cm);
				draw(unitcircle);
				draw((0,0)--(1,0), red);
				label("$1$",(0.5,0), red, align=S);
			\end{asy}
		\end{subfigure}
		\begin{subfigure}[b]{3cm}
			\centering
			\begin{asy}
				size(3cm);
				label("$\times\; [0,1]\; =$", (0,0), fontsize(10));
				draw(unitcircle, white+0);
			\end{asy}
		\end{subfigure}
		\begin{subfigure}[b]{3cm}
			\centering
			\begin{asy}
				import solids;
				size(3cm);
				draw(shift((0, 0.5)) * unitcircle, white+0);
				revolution r = cylinder(O, 1, 1.5, Z);
				draw(r);
				triple M = (-1/2, sqrt(3)/2, 0);
				draw((0,0,0) -- M, red);
				label("$1$", M/2, red, align=S);
				draw(M*1.1--M*1.1+(0,0,1.5), magenta, Arrows3(TeXHead2));
				label("$1$", M*1.1+(0,0,0.75), magenta, align=E);
			\end{asy}
		\end{subfigure}
	\end{figure}

	$C \times C$ où $C = \Big\{(x,y) \in \R^2  \mid x^2 + y^2 = 1\Big\}$ est un tore (creu).

	\begin{figure}[H]
		\centering
		\begin{subfigure}[b]{3cm}
			\centering
			\begin{asy}
				size(3cm);
				draw(unitcircle);
				draw((0,0)--(1,0), red);
				label("$1$",(0.5,0), red, align=S);
			\end{asy}
		\end{subfigure}
		\begin{subfigure}[b]{1cm}
			\centering
			\begin{asy}
				size(3cm);
				label("$\times$", (0,0), fontsize(10));
				dot((0.1, 1), white+0);
				dot((-0.1, -1), white+0);
			\end{asy}
		\end{subfigure}
		\begin{subfigure}[b]{3cm}
			\centering
			\begin{asy}
				size(3cm);
				draw(unitcircle);
				draw((0,0)--(1,0), red);
				label("$1$",(0.5,0), red, align=S);
			\end{asy}
		\end{subfigure}
		\begin{subfigure}[b]{1cm}
			\centering
			\begin{asy}
				size(3cm);
				label("$=$", (0,0), fontsize(10));
				dot((0.1, 1), white+0);
				dot((-0.1, -1), white+0);
			\end{asy}
		\end{subfigure}
		\begin{subfigure}[b]{3cm}
			\centering
			\begin{asy}
				import three;
				import graph3;

				size(3cm,3cm);
				surface torus = surface(Circle(c=2Y,normal=X,r=0.5,n=32), c=O, axis=Z, n=32);

				draw(torus, white + opacity(0), meshpen=black + 0.2pt, nolight, render(merge=true));
			\end{asy}
			\vspace{0.7cm}
		\end{subfigure}
	\end{figure}
\end{exm}

\begin{defn}
	Soient $E$ et $F$ deux ensembles. On dit que $E$ et $F$ sont \underline{équipotents} s'il existe une bijection de $E$ dans $F$.
	\index{équipotence (ensembles)}
\end{defn}

\begin{exm}
	\begin{enumerate}
		\item $\N$ et $\N^*$ sont équipotents car  $f : \begin{array}{rcl}
				\N &\longrightarrow& \N^* \\
				k &\longmapsto& k + 1
			\end{array}$ est bijective.
		\item $P = \{n \in \N  \mid n \text{ pair}\}$ et $I= \{n \in \N \mid n \text{ impair}\}$ sont équipotents car $f : \begin{array}{rcl}
				P &\longrightarrow& I \\
				x &\longmapsto& x+1
			\end{array}$ est bijective.
		\item $\N$ et $P$ sont équipotents car $f : \begin{array}{rcl}
				\N &\longrightarrow& P \\
				k &\longmapsto& 2k
			\end{array}$ est bijective.
		\item $[0,1]$ et $[0,1[$ sont équipotents car \begin{align*}
			f: [0,1] &\longrightarrow [0,1[ \\
			x &\longmapsto \begin{cases}
				\frac{1}{n+1} &\text{ si } x = \frac{1}{n} \text{ avec } n \in \N^*\\
				x &\text{ sinon}
			\end{cases}
		\end{align*} est bijective.
		\item De même, $]0,1[$ et $]0,1]$ sont équipotents.
		\item $]0,1[$ et $[0,1[$ sont équipotents : $f : \begin{array}{rcl}
					]0,1] &\longrightarrow& [0,1[ \\
				x &\longmapsto& 1-x
			\end{array}$ est bijective.
		\item $\forall a < b$, $[a,b]$ et $[0,1]$ sont équipotents : \begin{align*}
				f: [0,1] &\longrightarrow [a,b] \\
				\alpha &\longmapsto \alpha b + (1 - \alpha) a
			\end{align*} est bijective (interpolation linéaire).
		\item $\R$ et $]0,1[$ sont équipotents : \begin{align*}
				f: \R &\longrightarrow ]0,1[ \\
				x &\longmapsto \frac{1}{2} + \frac{\Arctan x}{\pi}
			\end{align*} est bijective.
		\item $[0,1[$ et $\N$ ne sont pas équipotents (argument de Cantor). Soit $f: \N \to [0,1[$ une bijection :
			\[
				\begin{array}{c|l}
					k&\hfill f(k)\hfill~ \\ \hline
					0&0,\hfill \!0\hfill 0\hfill 0\hfill 0\hfill\ldots\\
					1&0,\hfill a_1\hfill a_2\hfill a_3\hfill a_4\hfill\ldots\\
					2&0,\hfill b_1\hfill b_2\hfill b_3\hfill b_4\hfill\ldots\\
					\vdots&\hfill\vdots\hfill\ddots
				\end{array}
			\] On considère le nombre \[
				x = 0,\,(a_0+1)(b_1+1)(c_2+1)\cdots
			\] $f(1) \neq x$ car ils n'ont pas le même chiffre des dizaines.\\
			$f(2) \neq x$ car ils n'ont pas le même chiffre des centaines.

			Par le même raisonement, on en déduit que \[
				\forall n \in \N, f(n) \neq x
			\] donc $x$ n'a pas d'antécédant : une contradiction.
		\item On verra en exercice que $E$ et $\mathcal{P}(E)$ ne sont pas équipotents. $\R$ et $\mathcal{P}(\R)$ ne sont pas équipotents mais $\R$ et $\mathcal{P}(\N)$ le sont (développement dyadique).
		\item $\R^2$ et $\R$ sont équipotents; $\C$ et $\R$ sont équipotents.
	\end{enumerate}
\end{exm}

\begin{exo}
	Soit $E$ un ensemble. L'application \begin{align*}
		f: \mathcal{P}(E) &\longrightarrow {0,1}^E \\
		A &\longmapsto \mathbbm{1}_A
	\end{align*} est bijective.

	Soit $g : E \to \{0,1\}$.
	\begin{itemize}
		\item[\underline{\sc Analyse}] Soit $A \in \mathcal{P}(E)$ tel que $f(A) = g$. Alors $g = \mathbbm{1}_A$.
			donc  \[
				\forall x \in E,\; g(x) = \mathbbm{1}_A(x)
			\] et donc \[
				\begin{cases}
					\forall x \in A,\, g(x) = 1\\
					\forall x \in E \setminus A,\,g(x) = 0
				\end{cases}
			\] On en déduit que \[
				A = \{ x \in E  \mid  g(x) = 1\}  = g^{-1}\big(\{1\}\big).
			\]
		\item[\underline{\sc Synthèse}] On pose $A = g^{-1}\big(\{1\}\big)$. Montrons que $f(A) = g$.
			\[
				\forall x \in E,\,g(x) = \begin{cases}
					1 &\text{ si } x \in A\\
					0 &\text{ si } x \not\in A
				\end{cases} = \mathbbm{1}_A
			\] donc $g = \mathbbm{1}_A$.
	\end{itemize}

	On aurait aussi pu rédiger de la fa\c con suivante : on pose \begin{align*}
		u: \{0,1\}^E &\longrightarrow \mathcal{P}(E) \\
		g &\longmapsto g^{-1}\big(\{1\}\big).
	\end{align*} On montre que $u$ est la réciproque de $f$ : \[
		\begin{cases}
			f \circ u = \id_{\{0,1\}^E},\\
			u \circ f = \id_{\mathcal{P}(E)}.
		\end{cases}
	\]
\end{exo}

\begin{defn}
	Soit $f : E \to F$. L'\underline{image de $f$}\index{image (application)} est \[
		\mathrm{Im}(f) = f(E) = \big\{f(x) \mid x \in E\big\}.
	\]
\end{defn}

\begin{prop}
	Soit $f: E \to F$. \[
		f \text{ est surjective } \iff f(E) = F.
	\]
\end{prop}

\begin{defn}
	Une \underline{suite de $E$}\index{suite (ensemble)} est une application de $\N$ dans $E$.
\end{defn}

\begin{rmk}[Notation]
	Soit $u \in E^\N$. Pour $n \in \N$, on écrit $u_n$ à la place de $u(n)$.
\end{rmk}

\begin{defn}
	Soient $E$ et $I$ deux ensembles. Une \underline{famille de $E$ indéxée par $I$}\index{famille (ensemble)} est une application de $I$ dans $E$.

	À la place de $u(i)$ (avec $i \in I$), on écrit $u_i$.
\end{defn}

\begin{defn}
	Soit $E$ un ensemble et $(A_i)_{i \in I}$ une famille de parties de $E$. On suppose $I \neq \O$. On pose \[
		\bigcup_{i \in  I} A_i = \{x \in E  \mid \exists i \in I,\, x \in A_i\}
	\] et \[
		\bigcap_{i \in  I} A_i = \{x \in E  \mid \forall i \in I,\, x \in A_i\}.
	\] On pose aussi $\bigcup_{i \in \O} A_i = \O$ et $\bigcap_{i \in \O}  A_i = E$.
\end{defn}

\begin{rmk}
	De même que pour les sommes et produits de complexes, on peut intervertir des réunions doubles.
\end{rmk}

\begin{prop}
	Soit $E$ un ensemble, $(A,B) \in \mathcal{P}(E)^2$. \[
		A \subset (E \setminus B) \iff A \cap B = \O.
	\]
\end{prop}

\begin{figure}[H]
	\centering
	\begin{asy}
		import patterns;
		add("hatch",hatch(1mm, deepcyan));
		add("hatch2",hatch(1mm, heavygreen));
		size(3cm);

		guide main_set = scale(1.3) * ((-1,1)..(-0.8,-0.8)..(0,-0.9)..(0.7,-1.2)..(0.8, 0.9)..cycle);
		guide set_a = shift((-0.5, -0.2)) * ((-0.6, 0.6)..(0.2,-0.2)..(0.2,-0.4)..(-0.6,-0.2)..cycle);
		guide set_b = shift((0.3, 0.4)) * ((0.8, -0.6)..(1.1,-0.2)..(0.2,0.5)..(0.2,-0.8)..cycle);

		draw(main_set, magenta); label("$E$", 1.3*(0.8,0.9),magenta, align=NE);
		draw(set_a, deepcyan); label("$A$", (-0.6,0.6), deepcyan, align=NW);
		draw(set_b, heavygreen); label("$B$", (0.8,-0.6), heavygreen, align=SE);

		fill(set_a, pattern("hatch"));
		fill(set_b, pattern("hatch2"));
	\end{asy}
\end{figure}

\begin{prv}
	\begin{itemize}
		\item[``$\implies$''] Soit $x \in A \cap B$. Alors $x \in A$ et $x \in B$. Comme $x \in A \subset (E \setminus B)$, alors $x \in E \setminus B$ i.e. $x \not\in B$ : une contradiction. Donc $A \cap B = \O$.
		\item[``$\impliedby$''] On suppose $A \cap B = \O$. Soit $x \in A$. Si $x \in B$, alors $x \in A \cap B = \O$ : faux.
			Donc $x \not\in B$ et donc $x \in E \setminus B$.
	\end{itemize}
\end{prv}

\begin{prop}
	Si $f: E\to F$ et $g: F \to G$ sont bijectives, alors $g \circ f$ est bijective et \[
		(g \circ f)^{-1} = f^{-1} \circ g^{-1}.
	\] \qed
\end{prop}

\begin{rmk}[\danger Attention]
	$g \circ f$ peut-être bijective alors que $f$ et $g$ ne le sont pas.
\end{rmk}



	\chap[03]{Étude de fonctions}
	\renewcommand{\cwd}{../chap03}
	\part{Modes de définition}

\begin{defn}
	Une suite peut être définie
	\begin{itemize}
		\item \underline{Explicitement}
			On dispose pour tout $n \in \N$ de l'expression de $u_n$ en fonction de $n$.\\
			\ex $\forall n \in \N_*, u_n = \frac{\ln(n)}{n}e^{-n}$\\
		\item \underline{Par récurrence}
			On connait $u_{n+1}$ en fonction de  $u_0, u_1, \ldots, u_n$\\
			\ex $\begin{cases}
				u_0=1\\
				\forall n \in \N, u_{n+1} = \sin(u_n)
			\end{cases}$\\
		\item \underline{Implicitement}
			$\forall n \in \N, u_n$ est le seul nombre verifiant une certaine propriété\\
			\ex $u_n$ est le seul réel vérifiant  $x^5 + nx - 1 = 0$
	\end{itemize}
\end{defn}

	\part{Topologie de $\R^2$}

\begin{defn}
	La \underline{norme (euclidienne)} de $\R^2$ est l'application définie par \[
		\forall (x,y) \in \R^2, \|(x,y)\| = \sqrt{x^2 + y^2}.
	\]

	\begin{figure}[H]
		\centering
		\begin{asy}
			import graph;
			axes(EndArrow);
			size(4cm);
			pair A = (3,2);
			dot(A);
			draw((3,0)--A, dashed);
			draw((0,2)--A, dashed);
			label("$x$", (3,0), align=S);
			label("$y$", (0,2), align=W);
			draw((0,0)--A);
			dot((4,3), white+0);
		\end{asy}
	\end{figure}
	\index{norme (de $\R^2$)}
	\index{norme euclidienne (de $\R^2$)}
\end{defn}

\begin{prop}
	La norme euclidienne vérifie:
	\begin{enumerate}
		\item (séparation) \[
			\forall (x,y) \in \R^2, \|(x,y)\| = 0 \iff x = y = 0,
			\]
		\item (homogénéité positive) \[
				\forall \lambda \in \R, \forall (x,y) \in \R^2, \|\lambda(x,y)\|= \left| \lambda \right| \|(x,y)\|
			\]
		\item (inégalité triangulaire) \[
			\forall (x,y), (a,b) \in \R^2,
			\|(x,y)+(a,b)\|\le \|(x,y)\|+\|(a,b)\|.
		\]
	\end{enumerate}
\end{prop}

\begin{prv}
	Déjà vue en replaçant $(x,y)$ par $x+iy \in \C$ et $\|(x,y)\|$ par |x+iy|
\end{prv}

\begin{defn}
	Soit $(a,b) \in \R^2$ et $r \in \R_+$.

	La \underline{boule ouverte} (ou \underline{disque ouvert}) de centre $(a,b)$ et de rayon $r$ est \[
		B_{(a,b)}(r) = \big\{ (x,y) \in \R^2  \mid \|(x,y) - (a,b)\| < r \big\}.
	\]

	La \underline{boule fermée} (ou \underline{disque fermé}) de centre $(a,b)$ et de rayon $r$ est \[
		\overline{B_{(a,b)}}(r) = \big\{ (x,y)\in \R^2  \mid \|(x,y) - (a,b)\| \le r \big\}.
	\]

	La \underline{sphère} (ou \underline{boule}) de centre $(a,b)$ et de rayon $r$ est \[
		S_{(a,b)}(r) = \partial \overline{B_{(a,b)}}(r) = \big\{ (x,y) \in \R^2  \mid \|(x,y) - (a,b)\| = r \big\}.
	\]
	\index{boule ouverte (de $\R^2$)}
	\index{disque ouverte (de $\R^2$)}
	\index{boule fermée (de $\R^2$)}
	\index{disque fermée (de $\R^2$)}
	\index{boule (de $\R^2$)}
	\index{sphère (de $\R^2$)}
\end{defn}

\begin{figure}[H]
		\centering
		\incfig{boule}
\end{figure}

\begin{rmk}
	On parle de boule en dimension quelconque.
\end{rmk}

\begin{defn}
	Une \underline{partie ouverte} $O$ de $\R^2$ (ou \underline{un ouvert}) si \[
		\forall (x,y) \in O, \exists r > 0, B_{(a,b)}(r) \subset O.
	\]
	Une partie $F$ est \underline{fermée} su $\R^2\setminus F$ est ouverte.
	\index{partie ouverte (de $\R^2$)}
	\index{ouvert (de $\R^2$)}
	\index{partie fermée (de $\R^2$)}
\end{defn}

\begin{figure}[H]
	\centering
	\incfig{partie-ouverte}
\end{figure}

\begin{prop}
	Une boule ouverte est ouverte. Une boule fermée est fermée.
\end{prop}

\begin{figure}[H]
	\centering
	\begin{subfigure}{4cm}
		\centering
		\begin{asy}
			import patterns;

			pair n(pair a) {return a / length(a);}

			add("hatch",hatch(2mm, SW, red));
			size(4cm);

			draw(circle((0,0), 1));
			dot('$(a_0, b_0)$', (0,0), align=S);

			draw((0,0) -- n((-1, 1)), dashed);
			label("$r$", n((-1, 1)) / 2, align=1.5S);

			pair A = n((1,3)) * (2/3);
			real rho = (1 - length(A)) * (2 / 3);

			dot("$(a,b)$", A, red, align=3SE);
			filldraw(circle(A, rho), pattern("hatch"), red);

			label("$O$", n((1,-1))*2.5/3);
		\end{asy}
	\end{subfigure}
	\begin{subfigure}{1cm}
		\centering~\\
	\end{subfigure}
	\begin{subfigure}{5cm}
		\centering
		\begin{asy}
			import patterns;

			pair n(pair a) {return a / length(a);}

			add("hatch",hatch(1mm, SW, red));
			add("hatch2",hatch(3mm, SE, blue));
			size(5cm);

			guide around = (-1.5, -1.5) -- (-1.5, 1.5) -- (2.5, 1.5) -- (2.5, -1.5) -- cycle;

			pair A = n((3, 1)) * 5/3; 
			real rho = 2 / 9;

			picture inter;
			fill(inter, around, pattern("hatch2"));
			fill(inter, circle((0,0), 1), white);
			add(inter);

			draw(circle((0,0), 1));
			dot('$(a_0, b_0)$', (0,0), align=S);

			draw((0,0) -- n((-1, 1)), dashed);
			label("$r$", n((-1, 1)) / 2, align=1.5S);

			dot("$(a,b)$", A, red, align=2SE);
			filldraw(circle(A, rho), pattern("hatch"), red);

			label("$F$", n((1,-1))*2.5/3);
		\end{asy}
	\end{subfigure}
\end{figure}

\begin{prv}
	$\O$ est un ouvert.

	Soit $B$ la boule ouverte de centre $(a_0, b_0) \in \R^2$ et de rayon $r > 0$.

	On pose $\rho = \frac{1}{2}\big(r - \|(a,b) - (a_0,b_0)\|\big)$.
	Montrons que \[
		B_{(a,b)}(\rho) \subset  B_{(a,b)}(r).
	\]

	Soit $(x,y) \in B_{(a,b)}(\rho)$.
	\begin{align*}
		\|(x,y) - (a_0,b_0)\|&= \|(x,y)- (a,b) + (a,b) - (a_0,b_0)\| \\
		&\le \|(x,y) - (a,b)\| + \|(a,b) - (a_0, b_0)\|\\
		&< \rho + \|(a,b) - (a_0, b_0)\| = \frac{1}{2}r + \frac{1}{2} \|(a,b) - (a_0, b_0)\|\\
		&< r
	\end{align*}
	
	Soit $F$ la boule fermée de centre $(a_0, b_0)$ et de rayon $r \ge 0$.

	Soit $(a,b) \not\in F$. On pose \[
		\rho = \frac{1}{2}\big(\|(a,b) - (a_0, b_0)\| - r\big) > 0.
	\]

	Montrons que $B_{(a,b)}(\rho) \subset \R^2\setminus F$.

	Soit $(x,y) \in B_{(a,b)}(\rho)$.

	\begin{align*}
		\|(x,y) - (a_0, b_0)\| &= \|(x,y) - (a,b) + (a,b) - (a_0, b_0)\| \\
		&\ge \big| \underbrace{\|(x,y) - (a,b)\|}_{\le \rho} - \underbrace{\|(a,b) - (a_0, b_0)\|}_{> r} \big|\\
		&\ge \|(a,b) - (a_0, b_0)\|- \|(x,y) - (a,b)\|\\
		&> \|(a,b) - (a_0, b_0)\|- \rho\\
		&> \frac{1}{2} \|(a,b) - (a_0, b_0)\| + \frac{1}{2}r\\
		&> r
	\end{align*}

	donc $(x,y) \not\in F$.
\end{prv}

\begin{exm}
	\begin{enumerate}
		\item $\O$ est ouvert.\\
			$\R^2$ est ouvert.
		\item $\O$ est fermé.\\
			$\R^2$ est fermé.\\
		\item $\big\{(x,0)  \mid x > 0\big\}$ n'est ni ouverte ni fermé.
	\end{enumerate}
\end{exm}

\begin{figure}[H]
	\centering
	\begin{asy}
		size(3cm);

		draw((0, -1) -- (0, 3), Arrow(TeXHead));
		draw((-1, 0) -- (3, 0), Arrow(TeXHead));
		
		draw((0,0) -- (0, 2.97), red);
		draw(circle((0,1.5), 0.5), deepred);
		draw(circle((0,0.5), 0.1), deepred);
	\end{asy}
\end{figure}

\begin{defn}
	Soit $(a,b) \in \R^2$ et $V \in \mathcal{P}(\R^2)$.

	On dit que $V$ est un voisinage de $(a,b)$ s'il existe $r > 0$ tel que \[
		B_{(a,b)}(r) \subset V.
	\]
	\index{voisinage (dans $\R^2$)}
\end{defn}

\begin{prop}
	Un ouvert non vide est un voisinage en chacun de ces points. \qed
\end{prop}

\begin{defn}
	Soit $D \subset \R^2$. Un \underline{point intérieur} de $D$ est un couple $(a,b) \in D$ tel que \[
		\exists r > 0, B_{(a,b)}(r) \subset D.
	\] en d'autres termes, si $D$ est un voisinage de $(a,b)$.

	On note $\mathring D$ l'ensemble des points intérieurs à $D$. C'est \underline{l'intérieur} de $D$.
	\index{point intérieur (dans $\R^2$)}
	\index{intérieur (dans $\R^2$)}
\end{defn}

\begin{prop}
	$\mathring D$ est le plus grand ouvert $O$ de $\R^2$ tel que $O \subset D$.
\end{prop}

\begin{figure}[H]
	\centering
	\incfig{interieur-plus-grand-ouvert}
\end{figure}


\begin{prv}
	Soit $(a,b) \in \mathring D$.

	Par définition, il existe $r > 0$ tel que \[
		B_{(a,b)}(r) \subset D.
	\] Montrons que $B_{(a,b)}(r) \subset \mathring D$.

	Soit $(x,y) \in B_{(a,b)}(r)$. Comme $B_{(a,b)}(r)$ est un ouvert de $\R^2$, il existe $\rho > 0$ tel que \[
		B_{(x,y)}(\rho) \subset B_{(a,b)}(r)
	\] donc $(x,y) \in \mathring D$.

	Donc $\mathring D$ est ouvert, $\mathring D \subset D$.

	Soit $O$ un ouvert de $\R^2$ tel que $O \subset D$. Montrons que $O \subset \mathring D$.

	Soit $(x,y) \in O$. Soit $r > 0$ tel que \[
		B_{(x,y)}(r) \subset O \subset D
	\] donc $(x,y) \in \mathring D$.
\end{prv}

\begin{defn}
	Soit $f: D \subset \R^2 \to \R$, $\ell \in \R$, $(a,b) \in \mathring D$.

	On dit que \underline{$f(x,y)$ tend vers $\ell$ quand $(x,y)$ tend vers $(a,b)$} ou que $\ell$ est \underline{une limite} de $f$ en $(a,b)$ si \[
		\forall \varepsilon > 0, \exists r > 0, \forall (x,y) \in D, \|(x,y) - (a,b)\| < r \implies \left| f(x,y) - \ell \right| \le \varepsilon.
	\] en d'autres termes si \[
		\forall V \in \mathcal{V}_{\ell}, \exists W \in \mathcal{V}_{(a,b)}, \forall (x,y) \in W \cap D, f(x,y) \in V.
	\]
	\index{limite (dans $\R^2$)}
	\index{tendre vers (dans $\R^2$)}
\end{defn}

\begin{prop}
	[unicité de la limite]
	Soit $f: D \to \R$, $(a,b) \in \mathring D$, $\ell_1, \ell_2 \in \R$ telles que $\ell_1$ et $\ell_2$ sont des limites de $f$ en $(a,b)$.

	Alors $\ell_1 = \ell_2$.
\end{prop}

\begin{figure}[H]
	\centering
	\incfig{preuve-unicité-de-la-limite}
\end{figure}

\begin{prv}
	On suppose $\ell_1 < \ell_2$. On pose $\varepsilon = \frac{\ell_2 - \ell_1}{2} > 0$.

	Soit $r_1 > 0$ tel que \[
		f\big(B_{(a,b)}(r_1)\big) \subset ]\ell_1 - \varepsilon, \ell_1 + \varepsilon[.
	\] Soit $r_2 > 0$ tel que \[
		f\big(B_{(a,b)}(r_2)\big) \subset ]\ell_2 - \varepsilon, \ell_2 + \varepsilon [.
	\] On pose $r = \min(r_1, r_2)$ donc \[
		B_{(a,b)}(r_1) \cap B_{(a,b)}(r_2) = B_{(a,b)}(r) \neq \O.
	\] Soit $(x,y) \in B_{(a,b)}(r)$. Alors, \[
		f(x,y) \in ]\ell_1 - \varepsilon, \ell_1 + \varepsilon[ \cap ]\ell_2 - \varepsilon, \ell_2 + \varepsilon[ = \O.
	\] $\lightning$
\end{prv}

\begin{defn}
	Soit $f : D \to \R$, $(a,b) \in \mathring D$.

	On dit que $f$ est \underline{continue} en $(a,b)$ si \[
		f(x,y) \tendsto{(x,y) \to (a,b)}f(a,b).
	\]
	\index{continuité (dans $\R^2$)}
\end{defn}

\begin{prop}
	\underline{Si} $f(x,y) \tendsto{(x,y) \to (a,b)} \ell$ \\
	\underline{alors} $\begin{cases}
		f(x,b) \tendsto{x \to a} \ell\\
		f(a,y) \tendsto{y \to b} \ell.\\
	\end{cases}$
\end{prop}

\begin{prv}~\\
	\begin{figure}[H]
		\centering
		\incfig{limite-x-en-a-et-y-en-b}
	\end{figure}
\end{prv}

\underline{Contre-exemple} : exercice 3.

\begin{exm}
	\begin{enumerate}
		\item $f : \begin{array}{rcl}
				\R^2 &\longrightarrow& \R \\
				(x,y) &\longmapsto& x
			\end{array}$ limite en $(0,0)$ ?

			Soit $\varepsilon > 0$. On pose $r = \varepsilon$. \[
				\forall (x,y) \in B_{(0,0)}(r),
				\left| f(x,y) \right| = \left| x \right| \le \|(x,y)\| < r = \varepsilon
			\] Donc $f(x,y) \tendsto{(x,y) \to (a,b)} 0$.
		\item limite $f : \begin{array}{rcl}
				\R^2 &\longrightarrow& \R \\
				(x,y) &\longmapsto& x^3
			\end{array}$ en $(0,0)$ ?

			Soit $\varepsilon > 0$. On pose $r = \sqrt[3]{r} > 0$. \[
				\forall (x,y) \in B_{(0,0)}(r),
				\left| f(x,y) \right| = \left| x^3 \right| \le \|(x,y)\|^3 < r^3 = \varepsilon.
			\]
		\item limite de $f : \begin{array}{rcl}
			\R^2 &\longrightarrow& \R \\
			(x,y) &\longmapsto& x^3y^2
		\end{array}$ en $(0,0)$ ?

		Soit $\varepsilon > 0$. On pose $r = \sqrt[5]{\varepsilon} > 0$. \[
			\forall (x,y) \in B_{(0,0)}(r), \left| f(x,y) \right| = \left| x^3 y^2 \right| \le \|(x,y)\|^3 \|(x,y)\|^2 < r^5 = \varepsilon.
		\]
	\end{enumerate}
\end{exm}

\begin{defn}
	Soient $D \subset \R^2$ et $(x,y) \in \R^2$.

	\begin{figure}[H]
    \centering
    \incfig{point-adhérent}
	\end{figure}
	
	On dit que $(x,y)$ est \underline{adhérent} à $D$ si \[
		\forall r > 0, B_{(x,y)}(r) \cap D \neq \O.
	\] L'ensemble des points adhérents à $D$ est noté $\overline{D}$. On dit que $\overline{D}$ est \underline{l'adhérence} de $D$.
	\index{point adhérent (dans $\R^2$)}
	\index{adhérent (dans $\R^2$)}
\end{defn}

\begin{defn}
	Soit $f: D \subset \R^2 \to \R$ et $(a,b) \in \overline{D}$, $\ell \in \R$. On dit que $f$ tend vers $\ell$ quand $(x,y)$ tend vers $(a,b)$ si \[
		\forall \varepsilon > 0, \exists r > 0, \forall (x,y) \in B_{(a,b)}(r) \cap D,
		\left| f(x,y) - \ell \right| \le \varepsilon.
	\]
	\index{limite (dans $\R^2$)}
	\index{tendre vers (dans $\R^2$)}
\end{defn}

\begin{prop}
	\begin{enumerate}
		\item Dans ce contexte, il y a unicité de la limite
		\item La limite d'une somme, d'un produit, d'un quotien, d'une composée se comporte comme dans le cas d'une seule variable.
		\item Soit $f: D \to \R$ continue. Soient $g: I \to \R$ et $h: I \to \R$ continues telles que \[
			\forall t \in I, \big(g(t), h(t)\big) \in D.
		\] Alors \[
			t \in I \mapsto f\big(g(t), h(t)\big) \in \R
		\] est continue.
	\end{enumerate}
\end{prop}

\begin{figure}[H]
	\centering
	\begin{asy}
		import three;
		import graph3;
		size(5cm);

		settings.render = 0;
		settings.prc = false;
		currentprojection = obliqueX;

		draw(O -- X, Arrow3(TeXHead2));
		draw(O -- Y, Arrow3(TeXHead2));
		draw(O -- Z, Arrow3(TeXHead2));

		triple f(real x, real y, real z = 0) { return (x,y,cos(x - 0.5) * cos(y - 0.5)/1.2 + 0.15); }

		real inc = 1 / 5;

		for(real x = 0; x <= 1; x += inc) {
			draw(graph(
				new real(real t) { return x; }, // x
				new real(real y) { return y; }, // y
				new real(real y) { return f(x,y).z; }, // z
				0, 1
			), gray);
		}

		for(real y = 0; y <= 1; y += inc) {
			draw(graph(
				new real(real x) { return x; }, // x
				new real(real t) { return y; }, // y
				new real(real x) { return f(x,y).z; }, // z
				0, 1
			), gray);
		}

		path3 path1 = (0.3, 0.2, 0) .. (0.5, 0.5, 0) .. (0.6, 0.7, 0) .. (0.9, 0.8, 0);
		path3 path2 = (0.3, 0.8, 0) .. (0.5, 0.5, 0) .. (0.6, 0.3, 0) .. (0.9, 0.2, 0);
		path3 pathA = f(0.3, 0.2, 0) .. f(0.5, 0.5, 0) .. f(0.6, 0.7, 0) .. f(0.9, 0.8, 0);
		path3 pathB = f(0.3, 0.8, 0) .. f(0.5, 0.5, 0) .. f(0.6, 0.3, 0) .. f(0.9, 0.2, 0);

		draw(path1, red, Arrow3(TeXHead2, position=0.5));
		draw(pathA, red, Arrow3(TeXHead2, position=0.5));
		draw(path2, deepcyan, Arrow3(TeXHead2, position=0.3));
		draw(pathB, deepcyan, Arrow3(TeXHead2, position=0.3));

		dot((0.5, 0.5, 0));
		dot(f(0.5, 0.5, 0));
		draw((0.5, 0.5, 0) -- f(0.5, 0.5, 0), dashed);
	\end{asy}
\end{figure}


	\part{Transpositions}

\begin{defn}
	Une \underline{transposition} est un cycle de longueur 2 : $\begin{pmatrix}
		a&b
	\end{pmatrix}$ avec $a \neq b$.
	\index{transposition (permutation)}
\end{defn}

\begin{exm}
	Avec $n = 5$ et $\gamma = \begin{pmatrix}
		2&4&1
	\end{pmatrix}$.

	\begin{figure}[H]
		\centering

		\begin{asy}
			size(5cm);

			real rho = 0.15; // circles

			void draw_cycle(pair O, real r ...int[] nums) {
				int n = nums.length;
				real eps = (15 / r) * 0.8;

				for(int i = 0; i < n; ++i) {
					real theta_1 = (360/n) * (i+1);
					real theta_2 = (360/n) * i;

					pair C = O + dir(theta_2) * r;

					draw(circle(C, rho));
					label("$" + string(nums[i]) + "$", C);
					draw(arc(O, r, theta_2+eps, theta_1-eps), Arrow(TeXHead));
				}
			}

			draw_cycle((-1,0), 0.8, 1, 2, 4);
			draw_cycle((1,0), 0.3, 3);
			draw_cycle((2,0), 0.3, 5);
		\end{asy}
	\end{figure}

	\[
		\gamma = \begin{pmatrix}
			1&4
		\end{pmatrix} \begin{pmatrix}
			1&2
		\end{pmatrix}
	\]

	Avec $n = 6$ et $\gamma = \begin{pmatrix}
		1&3&5&6&2
	\end{pmatrix} = \begin{pmatrix}
		1&2&3&4&5&6\\
		3&1&5&4&6&2
	\end{pmatrix}$.

	Donc, \[
		\gamma = \begin{pmatrix}
			1&2
		\end{pmatrix} \begin{pmatrix}
			1&6
		\end{pmatrix} \begin{pmatrix}
			1&5
		\end{pmatrix} \begin{pmatrix}
			1&3
		\end{pmatrix}
	\] 
	\[
		\begin{pmatrix}
			1&2&3&4&5&6\\
			3&2&1&4&5&6\\
			3&2&5&4&1&6\\
			3&2&5&4&6&1\\
			3&1&5&4&6&2\\
		\end{pmatrix}
	\]

	Et, \[
		\gamma = \begin{pmatrix}
			1&3
		\end{pmatrix} \begin{pmatrix}
			2&3
		\end{pmatrix} \begin{pmatrix}
			3&5
		\end{pmatrix} \begin{pmatrix}
			5&6
		\end{pmatrix} 
	\]

	\[
		\begin{pmatrix}
			1&2&3&4&5&6\\
			1&2&3&4&6&5\\
			1&2&5&4&6&3\\
			1&3&5&4&6&2\\
			3&1&5&4&6&2\\
		\end{pmatrix} 
	\] 
\end{exm}

\begin{exm}
	\[
		\begin{pmatrix}
			1&4
		\end{pmatrix} = \begin{pmatrix}
			1&2
		\end{pmatrix} \begin{pmatrix}
			2&3
		\end{pmatrix} \begin{pmatrix}
			3&4
		\end{pmatrix} \begin{pmatrix}
			2&3
		\end{pmatrix} \begin{pmatrix}
			1&2
		\end{pmatrix}
	\]
	On n'a pas toujours le même nombre de transpositions mais la parité du nombre reste la même (proposition plus loin).
\end{exm}

\begin{thm}
	Toute permutation se décompose en produit de transpositions.
\end{thm}

\begin{prv}
	Soit $\gamma = \begin{pmatrix}
		a_1&\cdots&a_k
	\end{pmatrix}$ un $k$-cycle.

	On remarque que
	\[
		\gamma = \begin{pmatrix}
			a_1&a_k
		\end{pmatrix} \cdots \begin{pmatrix}
			a_1&a_4
		\end{pmatrix} \begin{pmatrix}
			a_1&a_3
		\end{pmatrix} \begin{pmatrix}
			a_1&a_2
		\end{pmatrix}
	\] C'est un produit de transpositions.
\end{prv}

\begin{exm}
	Avec $n = 10$ et $\sigma = \begin{pmatrix}
		1&2&3&4&5&6&7&8&9&10\\
		9&8&1&7&2&3&4&5&10&6
	\end{pmatrix}$.

	On a
	\begin{align*}
		\sigma &= \begin{pmatrix}
			1&9&10&6&3
		\end{pmatrix} \begin{pmatrix}
			2&8&5
		\end{pmatrix} \begin{pmatrix}
			4&7
		\end{pmatrix}\\
		&= \begin{pmatrix}
			1&3
		\end{pmatrix} \begin{pmatrix}
			1&6
		\end{pmatrix} \begin{pmatrix}
			1&10
		\end{pmatrix} \begin{pmatrix}
			1&9
		\end{pmatrix} \begin{pmatrix}
			2&5
		\end{pmatrix} \begin{pmatrix}
			2&8
		\end{pmatrix} \begin{pmatrix}
			4&7
		\end{pmatrix} \\
	\end{align*}

	Vérification : \[
		\begin{pmatrix}
			1&2&3&4&5&6&7&8&9&10\\
			1&2&3&7&5&6&4&8&9&10\\
			1&8&3&7&5&6&4&2&9&10\\
			1&8&3&7&2&6&4&5&9&10\\
			9&8&3&7&2&6&4&5&1&10\\
			9&8&3&7&2&6&4&5&10&1\\
			9&8&3&7&2&1&4&5&10&6\\
			9&8&1&7&2&3&4&5&10&6\\
		\end{pmatrix} 
	\] 
\end{exm}


	\chap[05]{Calcul intégral}
	\renewcommand{\cwd}{../chap05}
	\part{Topologie de $\R^2$}

\begin{defn}
	La \underline{norme (euclidienne)} de $\R^2$ est l'application définie par \[
		\forall (x,y) \in \R^2, \|(x,y)\| = \sqrt{x^2 + y^2}.
	\]

	\begin{figure}[H]
		\centering
		\begin{asy}
			import graph;
			axes(EndArrow);
			size(4cm);
			pair A = (3,2);
			dot(A);
			draw((3,0)--A, dashed);
			draw((0,2)--A, dashed);
			label("$x$", (3,0), align=S);
			label("$y$", (0,2), align=W);
			draw((0,0)--A);
			dot((4,3), white+0);
		\end{asy}
	\end{figure}
	\index{norme (de $\R^2$)}
	\index{norme euclidienne (de $\R^2$)}
\end{defn}

\begin{prop}
	La norme euclidienne vérifie:
	\begin{enumerate}
		\item (séparation) \[
			\forall (x,y) \in \R^2, \|(x,y)\| = 0 \iff x = y = 0,
			\]
		\item (homogénéité positive) \[
				\forall \lambda \in \R, \forall (x,y) \in \R^2, \|\lambda(x,y)\|= \left| \lambda \right| \|(x,y)\|
			\]
		\item (inégalité triangulaire) \[
			\forall (x,y), (a,b) \in \R^2,
			\|(x,y)+(a,b)\|\le \|(x,y)\|+\|(a,b)\|.
		\]
	\end{enumerate}
\end{prop}

\begin{prv}
	Déjà vue en replaçant $(x,y)$ par $x+iy \in \C$ et $\|(x,y)\|$ par |x+iy|
\end{prv}

\begin{defn}
	Soit $(a,b) \in \R^2$ et $r \in \R_+$.

	La \underline{boule ouverte} (ou \underline{disque ouvert}) de centre $(a,b)$ et de rayon $r$ est \[
		B_{(a,b)}(r) = \big\{ (x,y) \in \R^2  \mid \|(x,y) - (a,b)\| < r \big\}.
	\]

	La \underline{boule fermée} (ou \underline{disque fermé}) de centre $(a,b)$ et de rayon $r$ est \[
		\overline{B_{(a,b)}}(r) = \big\{ (x,y)\in \R^2  \mid \|(x,y) - (a,b)\| \le r \big\}.
	\]

	La \underline{sphère} (ou \underline{boule}) de centre $(a,b)$ et de rayon $r$ est \[
		S_{(a,b)}(r) = \partial \overline{B_{(a,b)}}(r) = \big\{ (x,y) \in \R^2  \mid \|(x,y) - (a,b)\| = r \big\}.
	\]
	\index{boule ouverte (de $\R^2$)}
	\index{disque ouverte (de $\R^2$)}
	\index{boule fermée (de $\R^2$)}
	\index{disque fermée (de $\R^2$)}
	\index{boule (de $\R^2$)}
	\index{sphère (de $\R^2$)}
\end{defn}

\begin{figure}[H]
		\centering
		\incfig{boule}
\end{figure}

\begin{rmk}
	On parle de boule en dimension quelconque.
\end{rmk}

\begin{defn}
	Une \underline{partie ouverte} $O$ de $\R^2$ (ou \underline{un ouvert}) si \[
		\forall (x,y) \in O, \exists r > 0, B_{(a,b)}(r) \subset O.
	\]
	Une partie $F$ est \underline{fermée} su $\R^2\setminus F$ est ouverte.
	\index{partie ouverte (de $\R^2$)}
	\index{ouvert (de $\R^2$)}
	\index{partie fermée (de $\R^2$)}
\end{defn}

\begin{figure}[H]
	\centering
	\incfig{partie-ouverte}
\end{figure}

\begin{prop}
	Une boule ouverte est ouverte. Une boule fermée est fermée.
\end{prop}

\begin{figure}[H]
	\centering
	\begin{subfigure}{4cm}
		\centering
		\begin{asy}
			import patterns;

			pair n(pair a) {return a / length(a);}

			add("hatch",hatch(2mm, SW, red));
			size(4cm);

			draw(circle((0,0), 1));
			dot('$(a_0, b_0)$', (0,0), align=S);

			draw((0,0) -- n((-1, 1)), dashed);
			label("$r$", n((-1, 1)) / 2, align=1.5S);

			pair A = n((1,3)) * (2/3);
			real rho = (1 - length(A)) * (2 / 3);

			dot("$(a,b)$", A, red, align=3SE);
			filldraw(circle(A, rho), pattern("hatch"), red);

			label("$O$", n((1,-1))*2.5/3);
		\end{asy}
	\end{subfigure}
	\begin{subfigure}{1cm}
		\centering~\\
	\end{subfigure}
	\begin{subfigure}{5cm}
		\centering
		\begin{asy}
			import patterns;

			pair n(pair a) {return a / length(a);}

			add("hatch",hatch(1mm, SW, red));
			add("hatch2",hatch(3mm, SE, blue));
			size(5cm);

			guide around = (-1.5, -1.5) -- (-1.5, 1.5) -- (2.5, 1.5) -- (2.5, -1.5) -- cycle;

			pair A = n((3, 1)) * 5/3; 
			real rho = 2 / 9;

			picture inter;
			fill(inter, around, pattern("hatch2"));
			fill(inter, circle((0,0), 1), white);
			add(inter);

			draw(circle((0,0), 1));
			dot('$(a_0, b_0)$', (0,0), align=S);

			draw((0,0) -- n((-1, 1)), dashed);
			label("$r$", n((-1, 1)) / 2, align=1.5S);

			dot("$(a,b)$", A, red, align=2SE);
			filldraw(circle(A, rho), pattern("hatch"), red);

			label("$F$", n((1,-1))*2.5/3);
		\end{asy}
	\end{subfigure}
\end{figure}

\begin{prv}
	$\O$ est un ouvert.

	Soit $B$ la boule ouverte de centre $(a_0, b_0) \in \R^2$ et de rayon $r > 0$.

	On pose $\rho = \frac{1}{2}\big(r - \|(a,b) - (a_0,b_0)\|\big)$.
	Montrons que \[
		B_{(a,b)}(\rho) \subset  B_{(a,b)}(r).
	\]

	Soit $(x,y) \in B_{(a,b)}(\rho)$.
	\begin{align*}
		\|(x,y) - (a_0,b_0)\|&= \|(x,y)- (a,b) + (a,b) - (a_0,b_0)\| \\
		&\le \|(x,y) - (a,b)\| + \|(a,b) - (a_0, b_0)\|\\
		&< \rho + \|(a,b) - (a_0, b_0)\| = \frac{1}{2}r + \frac{1}{2} \|(a,b) - (a_0, b_0)\|\\
		&< r
	\end{align*}
	
	Soit $F$ la boule fermée de centre $(a_0, b_0)$ et de rayon $r \ge 0$.

	Soit $(a,b) \not\in F$. On pose \[
		\rho = \frac{1}{2}\big(\|(a,b) - (a_0, b_0)\| - r\big) > 0.
	\]

	Montrons que $B_{(a,b)}(\rho) \subset \R^2\setminus F$.

	Soit $(x,y) \in B_{(a,b)}(\rho)$.

	\begin{align*}
		\|(x,y) - (a_0, b_0)\| &= \|(x,y) - (a,b) + (a,b) - (a_0, b_0)\| \\
		&\ge \big| \underbrace{\|(x,y) - (a,b)\|}_{\le \rho} - \underbrace{\|(a,b) - (a_0, b_0)\|}_{> r} \big|\\
		&\ge \|(a,b) - (a_0, b_0)\|- \|(x,y) - (a,b)\|\\
		&> \|(a,b) - (a_0, b_0)\|- \rho\\
		&> \frac{1}{2} \|(a,b) - (a_0, b_0)\| + \frac{1}{2}r\\
		&> r
	\end{align*}

	donc $(x,y) \not\in F$.
\end{prv}

\begin{exm}
	\begin{enumerate}
		\item $\O$ est ouvert.\\
			$\R^2$ est ouvert.
		\item $\O$ est fermé.\\
			$\R^2$ est fermé.\\
		\item $\big\{(x,0)  \mid x > 0\big\}$ n'est ni ouverte ni fermé.
	\end{enumerate}
\end{exm}

\begin{figure}[H]
	\centering
	\begin{asy}
		size(3cm);

		draw((0, -1) -- (0, 3), Arrow(TeXHead));
		draw((-1, 0) -- (3, 0), Arrow(TeXHead));
		
		draw((0,0) -- (0, 2.97), red);
		draw(circle((0,1.5), 0.5), deepred);
		draw(circle((0,0.5), 0.1), deepred);
	\end{asy}
\end{figure}

\begin{defn}
	Soit $(a,b) \in \R^2$ et $V \in \mathcal{P}(\R^2)$.

	On dit que $V$ est un voisinage de $(a,b)$ s'il existe $r > 0$ tel que \[
		B_{(a,b)}(r) \subset V.
	\]
	\index{voisinage (dans $\R^2$)}
\end{defn}

\begin{prop}
	Un ouvert non vide est un voisinage en chacun de ces points. \qed
\end{prop}

\begin{defn}
	Soit $D \subset \R^2$. Un \underline{point intérieur} de $D$ est un couple $(a,b) \in D$ tel que \[
		\exists r > 0, B_{(a,b)}(r) \subset D.
	\] en d'autres termes, si $D$ est un voisinage de $(a,b)$.

	On note $\mathring D$ l'ensemble des points intérieurs à $D$. C'est \underline{l'intérieur} de $D$.
	\index{point intérieur (dans $\R^2$)}
	\index{intérieur (dans $\R^2$)}
\end{defn}

\begin{prop}
	$\mathring D$ est le plus grand ouvert $O$ de $\R^2$ tel que $O \subset D$.
\end{prop}

\begin{figure}[H]
	\centering
	\incfig{interieur-plus-grand-ouvert}
\end{figure}


\begin{prv}
	Soit $(a,b) \in \mathring D$.

	Par définition, il existe $r > 0$ tel que \[
		B_{(a,b)}(r) \subset D.
	\] Montrons que $B_{(a,b)}(r) \subset \mathring D$.

	Soit $(x,y) \in B_{(a,b)}(r)$. Comme $B_{(a,b)}(r)$ est un ouvert de $\R^2$, il existe $\rho > 0$ tel que \[
		B_{(x,y)}(\rho) \subset B_{(a,b)}(r)
	\] donc $(x,y) \in \mathring D$.

	Donc $\mathring D$ est ouvert, $\mathring D \subset D$.

	Soit $O$ un ouvert de $\R^2$ tel que $O \subset D$. Montrons que $O \subset \mathring D$.

	Soit $(x,y) \in O$. Soit $r > 0$ tel que \[
		B_{(x,y)}(r) \subset O \subset D
	\] donc $(x,y) \in \mathring D$.
\end{prv}

\begin{defn}
	Soit $f: D \subset \R^2 \to \R$, $\ell \in \R$, $(a,b) \in \mathring D$.

	On dit que \underline{$f(x,y)$ tend vers $\ell$ quand $(x,y)$ tend vers $(a,b)$} ou que $\ell$ est \underline{une limite} de $f$ en $(a,b)$ si \[
		\forall \varepsilon > 0, \exists r > 0, \forall (x,y) \in D, \|(x,y) - (a,b)\| < r \implies \left| f(x,y) - \ell \right| \le \varepsilon.
	\] en d'autres termes si \[
		\forall V \in \mathcal{V}_{\ell}, \exists W \in \mathcal{V}_{(a,b)}, \forall (x,y) \in W \cap D, f(x,y) \in V.
	\]
	\index{limite (dans $\R^2$)}
	\index{tendre vers (dans $\R^2$)}
\end{defn}

\begin{prop}
	[unicité de la limite]
	Soit $f: D \to \R$, $(a,b) \in \mathring D$, $\ell_1, \ell_2 \in \R$ telles que $\ell_1$ et $\ell_2$ sont des limites de $f$ en $(a,b)$.

	Alors $\ell_1 = \ell_2$.
\end{prop}

\begin{figure}[H]
	\centering
	\incfig{preuve-unicité-de-la-limite}
\end{figure}

\begin{prv}
	On suppose $\ell_1 < \ell_2$. On pose $\varepsilon = \frac{\ell_2 - \ell_1}{2} > 0$.

	Soit $r_1 > 0$ tel que \[
		f\big(B_{(a,b)}(r_1)\big) \subset ]\ell_1 - \varepsilon, \ell_1 + \varepsilon[.
	\] Soit $r_2 > 0$ tel que \[
		f\big(B_{(a,b)}(r_2)\big) \subset ]\ell_2 - \varepsilon, \ell_2 + \varepsilon [.
	\] On pose $r = \min(r_1, r_2)$ donc \[
		B_{(a,b)}(r_1) \cap B_{(a,b)}(r_2) = B_{(a,b)}(r) \neq \O.
	\] Soit $(x,y) \in B_{(a,b)}(r)$. Alors, \[
		f(x,y) \in ]\ell_1 - \varepsilon, \ell_1 + \varepsilon[ \cap ]\ell_2 - \varepsilon, \ell_2 + \varepsilon[ = \O.
	\] $\lightning$
\end{prv}

\begin{defn}
	Soit $f : D \to \R$, $(a,b) \in \mathring D$.

	On dit que $f$ est \underline{continue} en $(a,b)$ si \[
		f(x,y) \tendsto{(x,y) \to (a,b)}f(a,b).
	\]
	\index{continuité (dans $\R^2$)}
\end{defn}

\begin{prop}
	\underline{Si} $f(x,y) \tendsto{(x,y) \to (a,b)} \ell$ \\
	\underline{alors} $\begin{cases}
		f(x,b) \tendsto{x \to a} \ell\\
		f(a,y) \tendsto{y \to b} \ell.\\
	\end{cases}$
\end{prop}

\begin{prv}~\\
	\begin{figure}[H]
		\centering
		\incfig{limite-x-en-a-et-y-en-b}
	\end{figure}
\end{prv}

\underline{Contre-exemple} : exercice 3.

\begin{exm}
	\begin{enumerate}
		\item $f : \begin{array}{rcl}
				\R^2 &\longrightarrow& \R \\
				(x,y) &\longmapsto& x
			\end{array}$ limite en $(0,0)$ ?

			Soit $\varepsilon > 0$. On pose $r = \varepsilon$. \[
				\forall (x,y) \in B_{(0,0)}(r),
				\left| f(x,y) \right| = \left| x \right| \le \|(x,y)\| < r = \varepsilon
			\] Donc $f(x,y) \tendsto{(x,y) \to (a,b)} 0$.
		\item limite $f : \begin{array}{rcl}
				\R^2 &\longrightarrow& \R \\
				(x,y) &\longmapsto& x^3
			\end{array}$ en $(0,0)$ ?

			Soit $\varepsilon > 0$. On pose $r = \sqrt[3]{r} > 0$. \[
				\forall (x,y) \in B_{(0,0)}(r),
				\left| f(x,y) \right| = \left| x^3 \right| \le \|(x,y)\|^3 < r^3 = \varepsilon.
			\]
		\item limite de $f : \begin{array}{rcl}
			\R^2 &\longrightarrow& \R \\
			(x,y) &\longmapsto& x^3y^2
		\end{array}$ en $(0,0)$ ?

		Soit $\varepsilon > 0$. On pose $r = \sqrt[5]{\varepsilon} > 0$. \[
			\forall (x,y) \in B_{(0,0)}(r), \left| f(x,y) \right| = \left| x^3 y^2 \right| \le \|(x,y)\|^3 \|(x,y)\|^2 < r^5 = \varepsilon.
		\]
	\end{enumerate}
\end{exm}

\begin{defn}
	Soient $D \subset \R^2$ et $(x,y) \in \R^2$.

	\begin{figure}[H]
    \centering
    \incfig{point-adhérent}
	\end{figure}
	
	On dit que $(x,y)$ est \underline{adhérent} à $D$ si \[
		\forall r > 0, B_{(x,y)}(r) \cap D \neq \O.
	\] L'ensemble des points adhérents à $D$ est noté $\overline{D}$. On dit que $\overline{D}$ est \underline{l'adhérence} de $D$.
	\index{point adhérent (dans $\R^2$)}
	\index{adhérent (dans $\R^2$)}
\end{defn}

\begin{defn}
	Soit $f: D \subset \R^2 \to \R$ et $(a,b) \in \overline{D}$, $\ell \in \R$. On dit que $f$ tend vers $\ell$ quand $(x,y)$ tend vers $(a,b)$ si \[
		\forall \varepsilon > 0, \exists r > 0, \forall (x,y) \in B_{(a,b)}(r) \cap D,
		\left| f(x,y) - \ell \right| \le \varepsilon.
	\]
	\index{limite (dans $\R^2$)}
	\index{tendre vers (dans $\R^2$)}
\end{defn}

\begin{prop}
	\begin{enumerate}
		\item Dans ce contexte, il y a unicité de la limite
		\item La limite d'une somme, d'un produit, d'un quotien, d'une composée se comporte comme dans le cas d'une seule variable.
		\item Soit $f: D \to \R$ continue. Soient $g: I \to \R$ et $h: I \to \R$ continues telles que \[
			\forall t \in I, \big(g(t), h(t)\big) \in D.
		\] Alors \[
			t \in I \mapsto f\big(g(t), h(t)\big) \in \R
		\] est continue.
	\end{enumerate}
\end{prop}

\begin{figure}[H]
	\centering
	\begin{asy}
		import three;
		import graph3;
		size(5cm);

		settings.render = 0;
		settings.prc = false;
		currentprojection = obliqueX;

		draw(O -- X, Arrow3(TeXHead2));
		draw(O -- Y, Arrow3(TeXHead2));
		draw(O -- Z, Arrow3(TeXHead2));

		triple f(real x, real y, real z = 0) { return (x,y,cos(x - 0.5) * cos(y - 0.5)/1.2 + 0.15); }

		real inc = 1 / 5;

		for(real x = 0; x <= 1; x += inc) {
			draw(graph(
				new real(real t) { return x; }, // x
				new real(real y) { return y; }, // y
				new real(real y) { return f(x,y).z; }, // z
				0, 1
			), gray);
		}

		for(real y = 0; y <= 1; y += inc) {
			draw(graph(
				new real(real x) { return x; }, // x
				new real(real t) { return y; }, // y
				new real(real x) { return f(x,y).z; }, // z
				0, 1
			), gray);
		}

		path3 path1 = (0.3, 0.2, 0) .. (0.5, 0.5, 0) .. (0.6, 0.7, 0) .. (0.9, 0.8, 0);
		path3 path2 = (0.3, 0.8, 0) .. (0.5, 0.5, 0) .. (0.6, 0.3, 0) .. (0.9, 0.2, 0);
		path3 pathA = f(0.3, 0.2, 0) .. f(0.5, 0.5, 0) .. f(0.6, 0.7, 0) .. f(0.9, 0.8, 0);
		path3 pathB = f(0.3, 0.8, 0) .. f(0.5, 0.5, 0) .. f(0.6, 0.3, 0) .. f(0.9, 0.2, 0);

		draw(path1, red, Arrow3(TeXHead2, position=0.5));
		draw(pathA, red, Arrow3(TeXHead2, position=0.5));
		draw(path2, deepcyan, Arrow3(TeXHead2, position=0.3));
		draw(pathB, deepcyan, Arrow3(TeXHead2, position=0.3));

		dot((0.5, 0.5, 0));
		dot(f(0.5, 0.5, 0));
		draw((0.5, 0.5, 0) -- f(0.5, 0.5, 0), dashed);
	\end{asy}
\end{figure}



	\chap[06]{Équations différentielles linéaires}
	\renewcommand{\cwd}{../chap06}
	\part{Topologie de $\R^2$}

\begin{defn}
	La \underline{norme (euclidienne)} de $\R^2$ est l'application définie par \[
		\forall (x,y) \in \R^2, \|(x,y)\| = \sqrt{x^2 + y^2}.
	\]

	\begin{figure}[H]
		\centering
		\begin{asy}
			import graph;
			axes(EndArrow);
			size(4cm);
			pair A = (3,2);
			dot(A);
			draw((3,0)--A, dashed);
			draw((0,2)--A, dashed);
			label("$x$", (3,0), align=S);
			label("$y$", (0,2), align=W);
			draw((0,0)--A);
			dot((4,3), white+0);
		\end{asy}
	\end{figure}
	\index{norme (de $\R^2$)}
	\index{norme euclidienne (de $\R^2$)}
\end{defn}

\begin{prop}
	La norme euclidienne vérifie:
	\begin{enumerate}
		\item (séparation) \[
			\forall (x,y) \in \R^2, \|(x,y)\| = 0 \iff x = y = 0,
			\]
		\item (homogénéité positive) \[
				\forall \lambda \in \R, \forall (x,y) \in \R^2, \|\lambda(x,y)\|= \left| \lambda \right| \|(x,y)\|
			\]
		\item (inégalité triangulaire) \[
			\forall (x,y), (a,b) \in \R^2,
			\|(x,y)+(a,b)\|\le \|(x,y)\|+\|(a,b)\|.
		\]
	\end{enumerate}
\end{prop}

\begin{prv}
	Déjà vue en replaçant $(x,y)$ par $x+iy \in \C$ et $\|(x,y)\|$ par |x+iy|
\end{prv}

\begin{defn}
	Soit $(a,b) \in \R^2$ et $r \in \R_+$.

	La \underline{boule ouverte} (ou \underline{disque ouvert}) de centre $(a,b)$ et de rayon $r$ est \[
		B_{(a,b)}(r) = \big\{ (x,y) \in \R^2  \mid \|(x,y) - (a,b)\| < r \big\}.
	\]

	La \underline{boule fermée} (ou \underline{disque fermé}) de centre $(a,b)$ et de rayon $r$ est \[
		\overline{B_{(a,b)}}(r) = \big\{ (x,y)\in \R^2  \mid \|(x,y) - (a,b)\| \le r \big\}.
	\]

	La \underline{sphère} (ou \underline{boule}) de centre $(a,b)$ et de rayon $r$ est \[
		S_{(a,b)}(r) = \partial \overline{B_{(a,b)}}(r) = \big\{ (x,y) \in \R^2  \mid \|(x,y) - (a,b)\| = r \big\}.
	\]
	\index{boule ouverte (de $\R^2$)}
	\index{disque ouverte (de $\R^2$)}
	\index{boule fermée (de $\R^2$)}
	\index{disque fermée (de $\R^2$)}
	\index{boule (de $\R^2$)}
	\index{sphère (de $\R^2$)}
\end{defn}

\begin{figure}[H]
		\centering
		\incfig{boule}
\end{figure}

\begin{rmk}
	On parle de boule en dimension quelconque.
\end{rmk}

\begin{defn}
	Une \underline{partie ouverte} $O$ de $\R^2$ (ou \underline{un ouvert}) si \[
		\forall (x,y) \in O, \exists r > 0, B_{(a,b)}(r) \subset O.
	\]
	Une partie $F$ est \underline{fermée} su $\R^2\setminus F$ est ouverte.
	\index{partie ouverte (de $\R^2$)}
	\index{ouvert (de $\R^2$)}
	\index{partie fermée (de $\R^2$)}
\end{defn}

\begin{figure}[H]
	\centering
	\incfig{partie-ouverte}
\end{figure}

\begin{prop}
	Une boule ouverte est ouverte. Une boule fermée est fermée.
\end{prop}

\begin{figure}[H]
	\centering
	\begin{subfigure}{4cm}
		\centering
		\begin{asy}
			import patterns;

			pair n(pair a) {return a / length(a);}

			add("hatch",hatch(2mm, SW, red));
			size(4cm);

			draw(circle((0,0), 1));
			dot('$(a_0, b_0)$', (0,0), align=S);

			draw((0,0) -- n((-1, 1)), dashed);
			label("$r$", n((-1, 1)) / 2, align=1.5S);

			pair A = n((1,3)) * (2/3);
			real rho = (1 - length(A)) * (2 / 3);

			dot("$(a,b)$", A, red, align=3SE);
			filldraw(circle(A, rho), pattern("hatch"), red);

			label("$O$", n((1,-1))*2.5/3);
		\end{asy}
	\end{subfigure}
	\begin{subfigure}{1cm}
		\centering~\\
	\end{subfigure}
	\begin{subfigure}{5cm}
		\centering
		\begin{asy}
			import patterns;

			pair n(pair a) {return a / length(a);}

			add("hatch",hatch(1mm, SW, red));
			add("hatch2",hatch(3mm, SE, blue));
			size(5cm);

			guide around = (-1.5, -1.5) -- (-1.5, 1.5) -- (2.5, 1.5) -- (2.5, -1.5) -- cycle;

			pair A = n((3, 1)) * 5/3; 
			real rho = 2 / 9;

			picture inter;
			fill(inter, around, pattern("hatch2"));
			fill(inter, circle((0,0), 1), white);
			add(inter);

			draw(circle((0,0), 1));
			dot('$(a_0, b_0)$', (0,0), align=S);

			draw((0,0) -- n((-1, 1)), dashed);
			label("$r$", n((-1, 1)) / 2, align=1.5S);

			dot("$(a,b)$", A, red, align=2SE);
			filldraw(circle(A, rho), pattern("hatch"), red);

			label("$F$", n((1,-1))*2.5/3);
		\end{asy}
	\end{subfigure}
\end{figure}

\begin{prv}
	$\O$ est un ouvert.

	Soit $B$ la boule ouverte de centre $(a_0, b_0) \in \R^2$ et de rayon $r > 0$.

	On pose $\rho = \frac{1}{2}\big(r - \|(a,b) - (a_0,b_0)\|\big)$.
	Montrons que \[
		B_{(a,b)}(\rho) \subset  B_{(a,b)}(r).
	\]

	Soit $(x,y) \in B_{(a,b)}(\rho)$.
	\begin{align*}
		\|(x,y) - (a_0,b_0)\|&= \|(x,y)- (a,b) + (a,b) - (a_0,b_0)\| \\
		&\le \|(x,y) - (a,b)\| + \|(a,b) - (a_0, b_0)\|\\
		&< \rho + \|(a,b) - (a_0, b_0)\| = \frac{1}{2}r + \frac{1}{2} \|(a,b) - (a_0, b_0)\|\\
		&< r
	\end{align*}
	
	Soit $F$ la boule fermée de centre $(a_0, b_0)$ et de rayon $r \ge 0$.

	Soit $(a,b) \not\in F$. On pose \[
		\rho = \frac{1}{2}\big(\|(a,b) - (a_0, b_0)\| - r\big) > 0.
	\]

	Montrons que $B_{(a,b)}(\rho) \subset \R^2\setminus F$.

	Soit $(x,y) \in B_{(a,b)}(\rho)$.

	\begin{align*}
		\|(x,y) - (a_0, b_0)\| &= \|(x,y) - (a,b) + (a,b) - (a_0, b_0)\| \\
		&\ge \big| \underbrace{\|(x,y) - (a,b)\|}_{\le \rho} - \underbrace{\|(a,b) - (a_0, b_0)\|}_{> r} \big|\\
		&\ge \|(a,b) - (a_0, b_0)\|- \|(x,y) - (a,b)\|\\
		&> \|(a,b) - (a_0, b_0)\|- \rho\\
		&> \frac{1}{2} \|(a,b) - (a_0, b_0)\| + \frac{1}{2}r\\
		&> r
	\end{align*}

	donc $(x,y) \not\in F$.
\end{prv}

\begin{exm}
	\begin{enumerate}
		\item $\O$ est ouvert.\\
			$\R^2$ est ouvert.
		\item $\O$ est fermé.\\
			$\R^2$ est fermé.\\
		\item $\big\{(x,0)  \mid x > 0\big\}$ n'est ni ouverte ni fermé.
	\end{enumerate}
\end{exm}

\begin{figure}[H]
	\centering
	\begin{asy}
		size(3cm);

		draw((0, -1) -- (0, 3), Arrow(TeXHead));
		draw((-1, 0) -- (3, 0), Arrow(TeXHead));
		
		draw((0,0) -- (0, 2.97), red);
		draw(circle((0,1.5), 0.5), deepred);
		draw(circle((0,0.5), 0.1), deepred);
	\end{asy}
\end{figure}

\begin{defn}
	Soit $(a,b) \in \R^2$ et $V \in \mathcal{P}(\R^2)$.

	On dit que $V$ est un voisinage de $(a,b)$ s'il existe $r > 0$ tel que \[
		B_{(a,b)}(r) \subset V.
	\]
	\index{voisinage (dans $\R^2$)}
\end{defn}

\begin{prop}
	Un ouvert non vide est un voisinage en chacun de ces points. \qed
\end{prop}

\begin{defn}
	Soit $D \subset \R^2$. Un \underline{point intérieur} de $D$ est un couple $(a,b) \in D$ tel que \[
		\exists r > 0, B_{(a,b)}(r) \subset D.
	\] en d'autres termes, si $D$ est un voisinage de $(a,b)$.

	On note $\mathring D$ l'ensemble des points intérieurs à $D$. C'est \underline{l'intérieur} de $D$.
	\index{point intérieur (dans $\R^2$)}
	\index{intérieur (dans $\R^2$)}
\end{defn}

\begin{prop}
	$\mathring D$ est le plus grand ouvert $O$ de $\R^2$ tel que $O \subset D$.
\end{prop}

\begin{figure}[H]
	\centering
	\incfig{interieur-plus-grand-ouvert}
\end{figure}


\begin{prv}
	Soit $(a,b) \in \mathring D$.

	Par définition, il existe $r > 0$ tel que \[
		B_{(a,b)}(r) \subset D.
	\] Montrons que $B_{(a,b)}(r) \subset \mathring D$.

	Soit $(x,y) \in B_{(a,b)}(r)$. Comme $B_{(a,b)}(r)$ est un ouvert de $\R^2$, il existe $\rho > 0$ tel que \[
		B_{(x,y)}(\rho) \subset B_{(a,b)}(r)
	\] donc $(x,y) \in \mathring D$.

	Donc $\mathring D$ est ouvert, $\mathring D \subset D$.

	Soit $O$ un ouvert de $\R^2$ tel que $O \subset D$. Montrons que $O \subset \mathring D$.

	Soit $(x,y) \in O$. Soit $r > 0$ tel que \[
		B_{(x,y)}(r) \subset O \subset D
	\] donc $(x,y) \in \mathring D$.
\end{prv}

\begin{defn}
	Soit $f: D \subset \R^2 \to \R$, $\ell \in \R$, $(a,b) \in \mathring D$.

	On dit que \underline{$f(x,y)$ tend vers $\ell$ quand $(x,y)$ tend vers $(a,b)$} ou que $\ell$ est \underline{une limite} de $f$ en $(a,b)$ si \[
		\forall \varepsilon > 0, \exists r > 0, \forall (x,y) \in D, \|(x,y) - (a,b)\| < r \implies \left| f(x,y) - \ell \right| \le \varepsilon.
	\] en d'autres termes si \[
		\forall V \in \mathcal{V}_{\ell}, \exists W \in \mathcal{V}_{(a,b)}, \forall (x,y) \in W \cap D, f(x,y) \in V.
	\]
	\index{limite (dans $\R^2$)}
	\index{tendre vers (dans $\R^2$)}
\end{defn}

\begin{prop}
	[unicité de la limite]
	Soit $f: D \to \R$, $(a,b) \in \mathring D$, $\ell_1, \ell_2 \in \R$ telles que $\ell_1$ et $\ell_2$ sont des limites de $f$ en $(a,b)$.

	Alors $\ell_1 = \ell_2$.
\end{prop}

\begin{figure}[H]
	\centering
	\incfig{preuve-unicité-de-la-limite}
\end{figure}

\begin{prv}
	On suppose $\ell_1 < \ell_2$. On pose $\varepsilon = \frac{\ell_2 - \ell_1}{2} > 0$.

	Soit $r_1 > 0$ tel que \[
		f\big(B_{(a,b)}(r_1)\big) \subset ]\ell_1 - \varepsilon, \ell_1 + \varepsilon[.
	\] Soit $r_2 > 0$ tel que \[
		f\big(B_{(a,b)}(r_2)\big) \subset ]\ell_2 - \varepsilon, \ell_2 + \varepsilon [.
	\] On pose $r = \min(r_1, r_2)$ donc \[
		B_{(a,b)}(r_1) \cap B_{(a,b)}(r_2) = B_{(a,b)}(r) \neq \O.
	\] Soit $(x,y) \in B_{(a,b)}(r)$. Alors, \[
		f(x,y) \in ]\ell_1 - \varepsilon, \ell_1 + \varepsilon[ \cap ]\ell_2 - \varepsilon, \ell_2 + \varepsilon[ = \O.
	\] $\lightning$
\end{prv}

\begin{defn}
	Soit $f : D \to \R$, $(a,b) \in \mathring D$.

	On dit que $f$ est \underline{continue} en $(a,b)$ si \[
		f(x,y) \tendsto{(x,y) \to (a,b)}f(a,b).
	\]
	\index{continuité (dans $\R^2$)}
\end{defn}

\begin{prop}
	\underline{Si} $f(x,y) \tendsto{(x,y) \to (a,b)} \ell$ \\
	\underline{alors} $\begin{cases}
		f(x,b) \tendsto{x \to a} \ell\\
		f(a,y) \tendsto{y \to b} \ell.\\
	\end{cases}$
\end{prop}

\begin{prv}~\\
	\begin{figure}[H]
		\centering
		\incfig{limite-x-en-a-et-y-en-b}
	\end{figure}
\end{prv}

\underline{Contre-exemple} : exercice 3.

\begin{exm}
	\begin{enumerate}
		\item $f : \begin{array}{rcl}
				\R^2 &\longrightarrow& \R \\
				(x,y) &\longmapsto& x
			\end{array}$ limite en $(0,0)$ ?

			Soit $\varepsilon > 0$. On pose $r = \varepsilon$. \[
				\forall (x,y) \in B_{(0,0)}(r),
				\left| f(x,y) \right| = \left| x \right| \le \|(x,y)\| < r = \varepsilon
			\] Donc $f(x,y) \tendsto{(x,y) \to (a,b)} 0$.
		\item limite $f : \begin{array}{rcl}
				\R^2 &\longrightarrow& \R \\
				(x,y) &\longmapsto& x^3
			\end{array}$ en $(0,0)$ ?

			Soit $\varepsilon > 0$. On pose $r = \sqrt[3]{r} > 0$. \[
				\forall (x,y) \in B_{(0,0)}(r),
				\left| f(x,y) \right| = \left| x^3 \right| \le \|(x,y)\|^3 < r^3 = \varepsilon.
			\]
		\item limite de $f : \begin{array}{rcl}
			\R^2 &\longrightarrow& \R \\
			(x,y) &\longmapsto& x^3y^2
		\end{array}$ en $(0,0)$ ?

		Soit $\varepsilon > 0$. On pose $r = \sqrt[5]{\varepsilon} > 0$. \[
			\forall (x,y) \in B_{(0,0)}(r), \left| f(x,y) \right| = \left| x^3 y^2 \right| \le \|(x,y)\|^3 \|(x,y)\|^2 < r^5 = \varepsilon.
		\]
	\end{enumerate}
\end{exm}

\begin{defn}
	Soient $D \subset \R^2$ et $(x,y) \in \R^2$.

	\begin{figure}[H]
    \centering
    \incfig{point-adhérent}
	\end{figure}
	
	On dit que $(x,y)$ est \underline{adhérent} à $D$ si \[
		\forall r > 0, B_{(x,y)}(r) \cap D \neq \O.
	\] L'ensemble des points adhérents à $D$ est noté $\overline{D}$. On dit que $\overline{D}$ est \underline{l'adhérence} de $D$.
	\index{point adhérent (dans $\R^2$)}
	\index{adhérent (dans $\R^2$)}
\end{defn}

\begin{defn}
	Soit $f: D \subset \R^2 \to \R$ et $(a,b) \in \overline{D}$, $\ell \in \R$. On dit que $f$ tend vers $\ell$ quand $(x,y)$ tend vers $(a,b)$ si \[
		\forall \varepsilon > 0, \exists r > 0, \forall (x,y) \in B_{(a,b)}(r) \cap D,
		\left| f(x,y) - \ell \right| \le \varepsilon.
	\]
	\index{limite (dans $\R^2$)}
	\index{tendre vers (dans $\R^2$)}
\end{defn}

\begin{prop}
	\begin{enumerate}
		\item Dans ce contexte, il y a unicité de la limite
		\item La limite d'une somme, d'un produit, d'un quotien, d'une composée se comporte comme dans le cas d'une seule variable.
		\item Soit $f: D \to \R$ continue. Soient $g: I \to \R$ et $h: I \to \R$ continues telles que \[
			\forall t \in I, \big(g(t), h(t)\big) \in D.
		\] Alors \[
			t \in I \mapsto f\big(g(t), h(t)\big) \in \R
		\] est continue.
	\end{enumerate}
\end{prop}

\begin{figure}[H]
	\centering
	\begin{asy}
		import three;
		import graph3;
		size(5cm);

		settings.render = 0;
		settings.prc = false;
		currentprojection = obliqueX;

		draw(O -- X, Arrow3(TeXHead2));
		draw(O -- Y, Arrow3(TeXHead2));
		draw(O -- Z, Arrow3(TeXHead2));

		triple f(real x, real y, real z = 0) { return (x,y,cos(x - 0.5) * cos(y - 0.5)/1.2 + 0.15); }

		real inc = 1 / 5;

		for(real x = 0; x <= 1; x += inc) {
			draw(graph(
				new real(real t) { return x; }, // x
				new real(real y) { return y; }, // y
				new real(real y) { return f(x,y).z; }, // z
				0, 1
			), gray);
		}

		for(real y = 0; y <= 1; y += inc) {
			draw(graph(
				new real(real x) { return x; }, // x
				new real(real t) { return y; }, // y
				new real(real x) { return f(x,y).z; }, // z
				0, 1
			), gray);
		}

		path3 path1 = (0.3, 0.2, 0) .. (0.5, 0.5, 0) .. (0.6, 0.7, 0) .. (0.9, 0.8, 0);
		path3 path2 = (0.3, 0.8, 0) .. (0.5, 0.5, 0) .. (0.6, 0.3, 0) .. (0.9, 0.2, 0);
		path3 pathA = f(0.3, 0.2, 0) .. f(0.5, 0.5, 0) .. f(0.6, 0.7, 0) .. f(0.9, 0.8, 0);
		path3 pathB = f(0.3, 0.8, 0) .. f(0.5, 0.5, 0) .. f(0.6, 0.3, 0) .. f(0.9, 0.2, 0);

		draw(path1, red, Arrow3(TeXHead2, position=0.5));
		draw(pathA, red, Arrow3(TeXHead2, position=0.5));
		draw(path2, deepcyan, Arrow3(TeXHead2, position=0.3));
		draw(pathB, deepcyan, Arrow3(TeXHead2, position=0.3));

		dot((0.5, 0.5, 0));
		dot(f(0.5, 0.5, 0));
		draw((0.5, 0.5, 0) -- f(0.5, 0.5, 0), dashed);
	\end{asy}
\end{figure}


	\part{Transpositions}

\begin{defn}
	Une \underline{transposition} est un cycle de longueur 2 : $\begin{pmatrix}
		a&b
	\end{pmatrix}$ avec $a \neq b$.
	\index{transposition (permutation)}
\end{defn}

\begin{exm}
	Avec $n = 5$ et $\gamma = \begin{pmatrix}
		2&4&1
	\end{pmatrix}$.

	\begin{figure}[H]
		\centering

		\begin{asy}
			size(5cm);

			real rho = 0.15; // circles

			void draw_cycle(pair O, real r ...int[] nums) {
				int n = nums.length;
				real eps = (15 / r) * 0.8;

				for(int i = 0; i < n; ++i) {
					real theta_1 = (360/n) * (i+1);
					real theta_2 = (360/n) * i;

					pair C = O + dir(theta_2) * r;

					draw(circle(C, rho));
					label("$" + string(nums[i]) + "$", C);
					draw(arc(O, r, theta_2+eps, theta_1-eps), Arrow(TeXHead));
				}
			}

			draw_cycle((-1,0), 0.8, 1, 2, 4);
			draw_cycle((1,0), 0.3, 3);
			draw_cycle((2,0), 0.3, 5);
		\end{asy}
	\end{figure}

	\[
		\gamma = \begin{pmatrix}
			1&4
		\end{pmatrix} \begin{pmatrix}
			1&2
		\end{pmatrix}
	\]

	Avec $n = 6$ et $\gamma = \begin{pmatrix}
		1&3&5&6&2
	\end{pmatrix} = \begin{pmatrix}
		1&2&3&4&5&6\\
		3&1&5&4&6&2
	\end{pmatrix}$.

	Donc, \[
		\gamma = \begin{pmatrix}
			1&2
		\end{pmatrix} \begin{pmatrix}
			1&6
		\end{pmatrix} \begin{pmatrix}
			1&5
		\end{pmatrix} \begin{pmatrix}
			1&3
		\end{pmatrix}
	\] 
	\[
		\begin{pmatrix}
			1&2&3&4&5&6\\
			3&2&1&4&5&6\\
			3&2&5&4&1&6\\
			3&2&5&4&6&1\\
			3&1&5&4&6&2\\
		\end{pmatrix}
	\]

	Et, \[
		\gamma = \begin{pmatrix}
			1&3
		\end{pmatrix} \begin{pmatrix}
			2&3
		\end{pmatrix} \begin{pmatrix}
			3&5
		\end{pmatrix} \begin{pmatrix}
			5&6
		\end{pmatrix} 
	\]

	\[
		\begin{pmatrix}
			1&2&3&4&5&6\\
			1&2&3&4&6&5\\
			1&2&5&4&6&3\\
			1&3&5&4&6&2\\
			3&1&5&4&6&2\\
		\end{pmatrix} 
	\] 
\end{exm}

\begin{exm}
	\[
		\begin{pmatrix}
			1&4
		\end{pmatrix} = \begin{pmatrix}
			1&2
		\end{pmatrix} \begin{pmatrix}
			2&3
		\end{pmatrix} \begin{pmatrix}
			3&4
		\end{pmatrix} \begin{pmatrix}
			2&3
		\end{pmatrix} \begin{pmatrix}
			1&2
		\end{pmatrix}
	\]
	On n'a pas toujours le même nombre de transpositions mais la parité du nombre reste la même (proposition plus loin).
\end{exm}

\begin{thm}
	Toute permutation se décompose en produit de transpositions.
\end{thm}

\begin{prv}
	Soit $\gamma = \begin{pmatrix}
		a_1&\cdots&a_k
	\end{pmatrix}$ un $k$-cycle.

	On remarque que
	\[
		\gamma = \begin{pmatrix}
			a_1&a_k
		\end{pmatrix} \cdots \begin{pmatrix}
			a_1&a_4
		\end{pmatrix} \begin{pmatrix}
			a_1&a_3
		\end{pmatrix} \begin{pmatrix}
			a_1&a_2
		\end{pmatrix}
	\] C'est un produit de transpositions.
\end{prv}

\begin{exm}
	Avec $n = 10$ et $\sigma = \begin{pmatrix}
		1&2&3&4&5&6&7&8&9&10\\
		9&8&1&7&2&3&4&5&10&6
	\end{pmatrix}$.

	On a
	\begin{align*}
		\sigma &= \begin{pmatrix}
			1&9&10&6&3
		\end{pmatrix} \begin{pmatrix}
			2&8&5
		\end{pmatrix} \begin{pmatrix}
			4&7
		\end{pmatrix}\\
		&= \begin{pmatrix}
			1&3
		\end{pmatrix} \begin{pmatrix}
			1&6
		\end{pmatrix} \begin{pmatrix}
			1&10
		\end{pmatrix} \begin{pmatrix}
			1&9
		\end{pmatrix} \begin{pmatrix}
			2&5
		\end{pmatrix} \begin{pmatrix}
			2&8
		\end{pmatrix} \begin{pmatrix}
			4&7
		\end{pmatrix} \\
	\end{align*}

	Vérification : \[
		\begin{pmatrix}
			1&2&3&4&5&6&7&8&9&10\\
			1&2&3&7&5&6&4&8&9&10\\
			1&8&3&7&5&6&4&2&9&10\\
			1&8&3&7&2&6&4&5&9&10\\
			9&8&3&7&2&6&4&5&1&10\\
			9&8&3&7&2&6&4&5&10&1\\
			9&8&3&7&2&1&4&5&10&6\\
			9&8&1&7&2&3&4&5&10&6\\
		\end{pmatrix} 
	\] 
\end{exm}

	\part{Familles orthogonales}

\begin{thm}[Pythagore]
	Soit $(x,y) \in E^2$. \[
		\|x+y\|^2 = \|x\|^2 + \|y\|^2 \iff x \perp y
	.\]
	\begin{figure}[H]
		\centering
		\begin{asy}
			size(4cm);
			pair u = (1, 0.5);
			pair v = 1.5 * (0, 1) * u;
			draw((0,0)--u, Arrow(TeXHead));
			label("$x$", u/2, align=S);
			draw(u--v+u, Arrow(TeXHead));
			label("$y$", u + v/2, align=NE);
			draw((0,0) -- u + v, Arrow(TeXHead));
			draw(u + v / 7.5 -- u + v / 7.5 - u / 5 -- u - u / 5 -- u -- cycle);
		\end{asy}
	\end{figure}
\end{thm}

\begin{prv}
	\[
		\|x + y\|^2 = \|x\|^2 + \|y\|^2 \iff 2\left<x \mid y \right> = 0 \iff x \perp y
	.\]
\end{prv}

\begin{defn}
	Soit $(e_i)_{i\in I}$ une famille de vecteurs. On dit que cette famille est \underline{orthogonale} si \[
		\forall i \neq j\, e_i \perp e_j
	.\] Si, en plus, on a \[
		\forall i \in I,\,\|e_i\| = 1,
	\] alors on dit que la famille est \underline{orthonormale} ou \underline{orthonormée}.
	\index{famille orthogonale}
	\index{famille orthonormale}
	\index{famille orthonormée}
\end{defn}

\begin{prop}[Pythagore]
	Soit $(e_1, \ldots, e_n)$ une famille orthogonale. Alors \[
		\left\| \sum_{i=1}^n e_i \right\|^2 = \sum_{i=1}^n \|e_i\|^2
	.\]
\end{prop}

\begin{thm}
	Toute famille orthogonale de vecteurs non nuls est libre.
\end{thm}

\begin{prv}
	Soit $(e_i)_{i\in I}$ une famille orthogonale telle que \[
		\forall i \in I,\,e_i \neq 0_E
	.\] Soit $n \in \N^*$, $(\lambda_1, \ldots, \lambda_n) \in \R^n$. On suppose \[
		\sum_{k=1}^n \lambda_k e_{i_k} = 0_E
	.\] Soit $j \in \left\llbracket 1,n \right\rrbracket$.
	\begin{align*}
		0 &= \left<\sum_{k=1}^n \lambda_k e_{i_k}  \mid e_{i_j} \right>\\
		&= \sum_{k=1}^n \lambda_k \left<e_{i_k}  \mid e_{i_j} \right> \\
		&= \lambda_j \underbrace{\|e_{i_j}\|^2}_{\neq 0} \\
	\end{align*}
	donc $\lambda_j = 0$.
\end{prv}

\begin{algo}[Orthonormalisation de Gran--Schmidt]
	On suppose $E$ de dimension finie. Soit $\mathcal{B} = (e_1, \ldots, e_n)$ une base de $E$.

	\begin{itemize}
		\item\underline{\it Étape 1}: On pose $v_1 = \frac{e_1}{\|e_1\|}$ de sorte que $\|v_1\| = 1$.
		\item\underline{\it Étape 2} : On pose \[
				u_2 = e_2 - \left<e_2  \mid v_1 \right> v_1
			.\] Ainsi,
			\begin{align*}
				\left<u_2 \mid v_1 \right> &= \big<e_2 - \left<e_2 \mid v_1 \right> v_1  \mid v_1 \big>\\
				&= \left<e_2 \mid v_1 \right> - \left<e_2 \mid v_1 \right> \left<v_1 \mid v_1 \right> \\
				&= 0. \\
			\end{align*}
			On pose $v_2 = \frac{u_2}{\|u_2\|}$ donc $v_2 \perp v_1$ et $\|v_2\| = 1$.
		\item\underline{\it Étape 3} : On pose \[
				u_2 = e_3 - \left<e_3 \mid v_1 \right>v_1 - \left<e_3 \mid v_2 \right>v_2
			.\] Ainsi,
			\begin{align*}
				\left<u_3  \mid v_1 \right> &= \left<e_3  \mid v_1 \right> - \left<e_3 \mid v_1 \right>\underbrace{\left<v_1 \mid v_1 \right>}_{=1} - \left<e_3 \mid v_2 \right>\underbrace{\left<v_2 \mid v_1 \right>}_{=0} \\
				&= 0 \\
			\end{align*}
			et 
			\begin{align*}
				\left<u_3 \mid v_2 \right> &= \left<e_3  \mid  v_2 \right> - \left<e_3 \mid v_1 \right> \underbrace{\left<v_1 \mid v_2 \right>}_{=0} - \left<e_3 \mid v_2 \right> \underbrace{\left<v_2 \mid v_2 \right>}_{=1}\\
				&= 0. \\
			\end{align*}
			On pose $v_3 = \frac{u_3}{\|u_3\|}$ de sorte que $v_3 \perp v_1$, $v_3 \perp v_2$ et $\|v_3\| = 1$.
		\item\underline{\it Étape $i+1$}: On pose \[
			u_{i+1} = e_{i+1} - \sum_{k=1}^i \left<e_{i+1}  \mid v_k \right> v_k
		.\] Ainsi, pour tout $j \in \left\llbracket 1,i \right\rrbracket,$ on a
		\begin{align*}
			\left<u_{i+1}  \mid v_j \right> &= \left<e_{i+1}  \mid v_j \right> - \sum_{k=1}^i \left<e_{i+1} \mid v_k \right> \left<v_k \mid v_j \right> \\
			&= \left<e_{i+1} \mid v_j \right> - \left<e_{i+1} \mid v_j \right> \|v_j\|^2 \\
			&= 0. \\
		\end{align*}
		On pose $v_{i+1} = \frac{u_{i+1}}{\|u_{i+1}\|}$.
	\end{itemize}
\end{algo}

\begin{exm}
	Avec $E = \R_3[X]$, $\left<P \mid Q \right> = \int_{0}^{1} P(t)\,Q(t)~\mathrm{d}t$ et $\mathcal{B} = (1, X, X^2, X^3)$.
	\begin{enumerate}
		\item $\|1\|^2 = \left<1 \mid 1 \right> = \int_{0}^{1} 1~\mathrm{d}t = 1$ et donc $v_1 = 1$.
		\item $u_2 = X - \left<X  \mid v_1 \right>v_1$. Or, $\left<X \mid v_1 \right> = \int_{0}^{1} t~\mathrm{d}t = \frac{1}{2}$. D'où $u_2 = X - \frac{1}{2}$.
			\begin{align*}
				\|u_2\|^2 &= \int_{0}^{1} \left( t - \frac{1}{2} \right)^2~\mathrm{d}t \\
				&= \int_{0}^{1} \left( t^2 - t + \frac{1}{4} \right)~\mathrm{d}t \\
				&= \frac{1}{3} - \frac{1}{2} + \frac{1}{4} \\
				&= \frac{1}{12} \\
			\end{align*} On en déduit que $v_2 = \sqrt{12}\left( X - \frac{1}{2} \right)$.
		\item $u_3 = X^2 - \left<X^2 \mid v_1 \right>v_1 - \left<X^2 \mid v_2 \right>v_2$.
			On a \[
				\left<X^2 \mid v_1 \right> = \int_{0}^{1} t^2~\mathrm{d}t = \frac{1}{3}
			\] et
			\begin{align*}
				\left<X^2 \mid v_2 \right> &= \sqrt{12} \int_{0}^{1} t^2\left( t - \frac{1}{2} \right)~\mathrm{d}t \\
				&= \frac{\sqrt{12}}{12}. \\
			\end{align*}
			D'où
			\begin{align*}
				u_3 &= X^2 - \frac{1}{3} - \frac{\sqrt{12}}{12}\sqrt{12} \left( X - \frac{1}{2} \right)\\
				&= X^2 - \frac{1}{3} - X + \frac{1}{2} \\
				&= X^2 - X + \frac{1}{6}. \\
			\end{align*}
			\begin{align*}
				\|u_3\|^2 &= \int_{0}^{1} \left( t^2 - t + \frac{1}{6} \right)~\mathrm{d}t\\
				&= \int_{0}^{1} \left( t^4 + t^2 + \frac{1}{36} - 2t^3 + \frac{t^2}{3} - \frac{t}{3} \right) ~\mathrm{d}t \\
				&= \frac{1}{5} + \frac{1}{3} + \frac{1}{36} - \frac{1}{2} + \frac{1}{9} - \frac{1}{6} \\
				&= \frac{36 + 60 + 5 - 90 + 20 - 30}{180} \\
				&= \frac{1}{180} \\
			\end{align*}
			On en déduit que \[
				v_3 = 6\sqrt{5}\left( X^2 - X + \frac{1}{6} \right).
			\]
		\item Exercice : calculer $v_4$.
	\end{enumerate}
\end{exm}

\begin{prop}
	Soit $\mathcal{B} = (e_1, \ldots, e_n)$ une base de $E$ et $\mathcal{C}$ la base obtenue par le procédé d'orthonormalisation de Gram--Schmidt. Alors, \[
		\forall i \in \left\llbracket 1,n \right\rrbracket,\,\Vect(e_1,\ldots, e_i) = \Vect(v_1, \ldots, v_i)
	.\]\qed
\end{prop}

\begin{exm}[orthogonalisation]
	\begin{itemize}
		\item $u_1 = 1$.
		\item
			\begin{align*}
				\begin{rcases*}
					u_2 \in \Vect(e_1, e_2)\\
					u_2 \perp u_1
				\end{rcases*}
				\iff& \begin{cases}
					u_2 = ae_1 + be_2\quad (a,b) \in \R^2\\
					\left<u_1 \mid u_2 \right> = 0
				\end{cases}\\
				\iff& \begin{cases}
					u_2 = a + bX\\
					\int_{0}^{1} (a+bt)~\mathrm{d}t = 0.
				\end{cases}\\
			\end{align*}
			\begin{align*}
				\int_{0}^{1} (a+bt)~\mathrm{d}t = 0 \iff& a + \frac{b}{2} = 0\\
				\iff& a = -\frac{b}{2}\\
				\iff& u_2 = -\frac{b}{2} + bX.
			\end{align*}
			Par exemple, $u_2 = -1 + 2X$.
		\item $\begin{cases}
				u_3 \in \Vect(e_1, e_2, e_3)\\
				u_3 \perp u_1\\
				u_3 \perp u_2
			\end{cases}$

			On pose $u_3 = a + bX + cX^2$ avec $(a,b,c)\in \R^3$.
			\begin{align*}
				\begin{rcases*}
					\int_{0}^{1} \left( a+bt + ct^2 \right)~\mathrm{d}t = 0\\
					\int_{0}^{1} \left(a + bt+ct^2\right)(2t - 1)~\mathrm{d}t = 0
				\end{rcases*} \iff& \begin{cases}
					a + \frac{b}{2} + \frac{c}{3} = 0\\
					\int_{0}^{1} \left( 2ct^3 + (-c + 2b)t^2 + (2a - b)t - a \right) ~\mathrm{d}t = 0
				\end{cases}\\
				\iff& \begin{cases}
					a + \frac{b}{2} + \frac{c}{3} = 0\\
					\frac{c}{2} + \frac{2b - c}{3} + \frac{2\cancel{a} - b}{2} - \cancel{a} = 0
				\end{cases}\\
				\iff&  \begin{cases}
					a = -\frac{b}{2} - \frac{c}{3} = \frac{c}{2} - \frac{c}{3} = \frac{c}{6}\\
					b = -c.
				\end{cases}
			\end{align*}
			On en déduit que \[
				u_3 = 1 - 6X + 6X^2
			.\]
	\end{itemize}
\end{exm}

\begin{crlr}[théorème de la base orthonormée incomplète] Soit $(e_1, \ldots, e_k)$ une base orthonormée d'un espace euclidien. On peut trouver $e_{k+1},\ldots,e_n$ tels que $(e_1, \ldots, e_k, e_{k+1},\ldots,e_n)$ soit une base orthonormée de $E$.
\end{crlr}

\begin{prv}
	On sait que $(e_1, \ldots, e_k)$ est libre. On complète $(e_1, \ldots, e_k)$ en une base $\mathcal{B}$ de $E$. On orthonormalise $\mathcal{B}$ : on obtient une base orthonormée $\mathcal{C}$ de $E$. En détaillant l'algorithme de Gram--Schmidt, on s'aper\c coit que les $k$ premiers vecteurs de $\mathcal{C}$ sont ceux de $\mathcal{B}$.
\end{prv}

\begin{thm}
	Soit $E$ un espace euclidien et $\mathcal{B} = (e_1, \ldots, e_n)$ une base orthonormée de $E$. Soit $(x,y) \in E^2$. On pose $(x_1, \ldots, x_n) \in \R^n$ et $(y_1, \ldots, y_n) \in \R^n$ tels que \[
		x = \sum_{i=1}^n x_i e_i \qquad\qquad y = \sum_{i=1}^n y_i e_i
	.\] Alors \[
		\left<x \mid y \right> = \sum_{i=1}^n x_i y_i
	.\]
	\vspace{3mm}
	Soit $X = \mat{x_1\\\vdots\\x_n}$ et $Y = \mat{y_1\\ \vdots \\ y_n}$. Alors, \[
		\left<x \mid y \right> = X^\T\,Y
	.\]
\end{thm}

\begin{prv}
	\begin{align*}
		\left<x \mid y \right> &= \left<\sum_{i=1}^n x_ie_i  \mid y \right>\\
		&= \sum_{i=1}^n x_i \left<e_i  \mid y \right> \\
		&= \sum_{i=1}^n x_i \left<e_i  \mid \sum_{j=1}^n y_j e_j \right> \\
		&= \sum_{i=1}^n x_i \sum_{j=1}^n y_j \underbrace{\left<e_i \mid e_j \right>}_{\delta_i^j} \\
		&= \sum_{i=1}^n x_i y_i. \\
	\end{align*}
\end{prv}

\begin{prop}
	Soit $E$ un espace euclidien et $\mathcal{B} = (e_1, \ldots, e_n)$ une base orthonormée de $E$. Alors, \[
		\forall x \in E,\,x = \sum_{i=1}^n \left<x \mid e_i \right>e_i
	.\]
\end{prop}

\begin{prv}
	Soit $x \in E$. On pose \[
		x = \sum_{i=1}^n x_i e_i
	\] avec $(x_1, \ldots, x_n) \in \R^n$. Soit $j \in \left\llbracket 1,n \right\rrbracket$. On a
	\begin{align*}
		\left<x \mid e_j \right> &= \left<\sum_{i=1}^n x_i e_i  \mid e_j \right>\\
		&= \sum_{i=1}^n x_i \left<e_i \mid e_j \right> \\
		&= x_j. \\
	\end{align*}
\end{prv}


	\chap[08]{Ensembles, applications, relations et lois de composition}
	\renewcommand{\cwd}{../chap08}
	\part{Topologie de $\R^2$}

\begin{defn}
	La \underline{norme (euclidienne)} de $\R^2$ est l'application définie par \[
		\forall (x,y) \in \R^2, \|(x,y)\| = \sqrt{x^2 + y^2}.
	\]

	\begin{figure}[H]
		\centering
		\begin{asy}
			import graph;
			axes(EndArrow);
			size(4cm);
			pair A = (3,2);
			dot(A);
			draw((3,0)--A, dashed);
			draw((0,2)--A, dashed);
			label("$x$", (3,0), align=S);
			label("$y$", (0,2), align=W);
			draw((0,0)--A);
			dot((4,3), white+0);
		\end{asy}
	\end{figure}
	\index{norme (de $\R^2$)}
	\index{norme euclidienne (de $\R^2$)}
\end{defn}

\begin{prop}
	La norme euclidienne vérifie:
	\begin{enumerate}
		\item (séparation) \[
			\forall (x,y) \in \R^2, \|(x,y)\| = 0 \iff x = y = 0,
			\]
		\item (homogénéité positive) \[
				\forall \lambda \in \R, \forall (x,y) \in \R^2, \|\lambda(x,y)\|= \left| \lambda \right| \|(x,y)\|
			\]
		\item (inégalité triangulaire) \[
			\forall (x,y), (a,b) \in \R^2,
			\|(x,y)+(a,b)\|\le \|(x,y)\|+\|(a,b)\|.
		\]
	\end{enumerate}
\end{prop}

\begin{prv}
	Déjà vue en replaçant $(x,y)$ par $x+iy \in \C$ et $\|(x,y)\|$ par |x+iy|
\end{prv}

\begin{defn}
	Soit $(a,b) \in \R^2$ et $r \in \R_+$.

	La \underline{boule ouverte} (ou \underline{disque ouvert}) de centre $(a,b)$ et de rayon $r$ est \[
		B_{(a,b)}(r) = \big\{ (x,y) \in \R^2  \mid \|(x,y) - (a,b)\| < r \big\}.
	\]

	La \underline{boule fermée} (ou \underline{disque fermé}) de centre $(a,b)$ et de rayon $r$ est \[
		\overline{B_{(a,b)}}(r) = \big\{ (x,y)\in \R^2  \mid \|(x,y) - (a,b)\| \le r \big\}.
	\]

	La \underline{sphère} (ou \underline{boule}) de centre $(a,b)$ et de rayon $r$ est \[
		S_{(a,b)}(r) = \partial \overline{B_{(a,b)}}(r) = \big\{ (x,y) \in \R^2  \mid \|(x,y) - (a,b)\| = r \big\}.
	\]
	\index{boule ouverte (de $\R^2$)}
	\index{disque ouverte (de $\R^2$)}
	\index{boule fermée (de $\R^2$)}
	\index{disque fermée (de $\R^2$)}
	\index{boule (de $\R^2$)}
	\index{sphère (de $\R^2$)}
\end{defn}

\begin{figure}[H]
		\centering
		\incfig{boule}
\end{figure}

\begin{rmk}
	On parle de boule en dimension quelconque.
\end{rmk}

\begin{defn}
	Une \underline{partie ouverte} $O$ de $\R^2$ (ou \underline{un ouvert}) si \[
		\forall (x,y) \in O, \exists r > 0, B_{(a,b)}(r) \subset O.
	\]
	Une partie $F$ est \underline{fermée} su $\R^2\setminus F$ est ouverte.
	\index{partie ouverte (de $\R^2$)}
	\index{ouvert (de $\R^2$)}
	\index{partie fermée (de $\R^2$)}
\end{defn}

\begin{figure}[H]
	\centering
	\incfig{partie-ouverte}
\end{figure}

\begin{prop}
	Une boule ouverte est ouverte. Une boule fermée est fermée.
\end{prop}

\begin{figure}[H]
	\centering
	\begin{subfigure}{4cm}
		\centering
		\begin{asy}
			import patterns;

			pair n(pair a) {return a / length(a);}

			add("hatch",hatch(2mm, SW, red));
			size(4cm);

			draw(circle((0,0), 1));
			dot('$(a_0, b_0)$', (0,0), align=S);

			draw((0,0) -- n((-1, 1)), dashed);
			label("$r$", n((-1, 1)) / 2, align=1.5S);

			pair A = n((1,3)) * (2/3);
			real rho = (1 - length(A)) * (2 / 3);

			dot("$(a,b)$", A, red, align=3SE);
			filldraw(circle(A, rho), pattern("hatch"), red);

			label("$O$", n((1,-1))*2.5/3);
		\end{asy}
	\end{subfigure}
	\begin{subfigure}{1cm}
		\centering~\\
	\end{subfigure}
	\begin{subfigure}{5cm}
		\centering
		\begin{asy}
			import patterns;

			pair n(pair a) {return a / length(a);}

			add("hatch",hatch(1mm, SW, red));
			add("hatch2",hatch(3mm, SE, blue));
			size(5cm);

			guide around = (-1.5, -1.5) -- (-1.5, 1.5) -- (2.5, 1.5) -- (2.5, -1.5) -- cycle;

			pair A = n((3, 1)) * 5/3; 
			real rho = 2 / 9;

			picture inter;
			fill(inter, around, pattern("hatch2"));
			fill(inter, circle((0,0), 1), white);
			add(inter);

			draw(circle((0,0), 1));
			dot('$(a_0, b_0)$', (0,0), align=S);

			draw((0,0) -- n((-1, 1)), dashed);
			label("$r$", n((-1, 1)) / 2, align=1.5S);

			dot("$(a,b)$", A, red, align=2SE);
			filldraw(circle(A, rho), pattern("hatch"), red);

			label("$F$", n((1,-1))*2.5/3);
		\end{asy}
	\end{subfigure}
\end{figure}

\begin{prv}
	$\O$ est un ouvert.

	Soit $B$ la boule ouverte de centre $(a_0, b_0) \in \R^2$ et de rayon $r > 0$.

	On pose $\rho = \frac{1}{2}\big(r - \|(a,b) - (a_0,b_0)\|\big)$.
	Montrons que \[
		B_{(a,b)}(\rho) \subset  B_{(a,b)}(r).
	\]

	Soit $(x,y) \in B_{(a,b)}(\rho)$.
	\begin{align*}
		\|(x,y) - (a_0,b_0)\|&= \|(x,y)- (a,b) + (a,b) - (a_0,b_0)\| \\
		&\le \|(x,y) - (a,b)\| + \|(a,b) - (a_0, b_0)\|\\
		&< \rho + \|(a,b) - (a_0, b_0)\| = \frac{1}{2}r + \frac{1}{2} \|(a,b) - (a_0, b_0)\|\\
		&< r
	\end{align*}
	
	Soit $F$ la boule fermée de centre $(a_0, b_0)$ et de rayon $r \ge 0$.

	Soit $(a,b) \not\in F$. On pose \[
		\rho = \frac{1}{2}\big(\|(a,b) - (a_0, b_0)\| - r\big) > 0.
	\]

	Montrons que $B_{(a,b)}(\rho) \subset \R^2\setminus F$.

	Soit $(x,y) \in B_{(a,b)}(\rho)$.

	\begin{align*}
		\|(x,y) - (a_0, b_0)\| &= \|(x,y) - (a,b) + (a,b) - (a_0, b_0)\| \\
		&\ge \big| \underbrace{\|(x,y) - (a,b)\|}_{\le \rho} - \underbrace{\|(a,b) - (a_0, b_0)\|}_{> r} \big|\\
		&\ge \|(a,b) - (a_0, b_0)\|- \|(x,y) - (a,b)\|\\
		&> \|(a,b) - (a_0, b_0)\|- \rho\\
		&> \frac{1}{2} \|(a,b) - (a_0, b_0)\| + \frac{1}{2}r\\
		&> r
	\end{align*}

	donc $(x,y) \not\in F$.
\end{prv}

\begin{exm}
	\begin{enumerate}
		\item $\O$ est ouvert.\\
			$\R^2$ est ouvert.
		\item $\O$ est fermé.\\
			$\R^2$ est fermé.\\
		\item $\big\{(x,0)  \mid x > 0\big\}$ n'est ni ouverte ni fermé.
	\end{enumerate}
\end{exm}

\begin{figure}[H]
	\centering
	\begin{asy}
		size(3cm);

		draw((0, -1) -- (0, 3), Arrow(TeXHead));
		draw((-1, 0) -- (3, 0), Arrow(TeXHead));
		
		draw((0,0) -- (0, 2.97), red);
		draw(circle((0,1.5), 0.5), deepred);
		draw(circle((0,0.5), 0.1), deepred);
	\end{asy}
\end{figure}

\begin{defn}
	Soit $(a,b) \in \R^2$ et $V \in \mathcal{P}(\R^2)$.

	On dit que $V$ est un voisinage de $(a,b)$ s'il existe $r > 0$ tel que \[
		B_{(a,b)}(r) \subset V.
	\]
	\index{voisinage (dans $\R^2$)}
\end{defn}

\begin{prop}
	Un ouvert non vide est un voisinage en chacun de ces points. \qed
\end{prop}

\begin{defn}
	Soit $D \subset \R^2$. Un \underline{point intérieur} de $D$ est un couple $(a,b) \in D$ tel que \[
		\exists r > 0, B_{(a,b)}(r) \subset D.
	\] en d'autres termes, si $D$ est un voisinage de $(a,b)$.

	On note $\mathring D$ l'ensemble des points intérieurs à $D$. C'est \underline{l'intérieur} de $D$.
	\index{point intérieur (dans $\R^2$)}
	\index{intérieur (dans $\R^2$)}
\end{defn}

\begin{prop}
	$\mathring D$ est le plus grand ouvert $O$ de $\R^2$ tel que $O \subset D$.
\end{prop}

\begin{figure}[H]
	\centering
	\incfig{interieur-plus-grand-ouvert}
\end{figure}


\begin{prv}
	Soit $(a,b) \in \mathring D$.

	Par définition, il existe $r > 0$ tel que \[
		B_{(a,b)}(r) \subset D.
	\] Montrons que $B_{(a,b)}(r) \subset \mathring D$.

	Soit $(x,y) \in B_{(a,b)}(r)$. Comme $B_{(a,b)}(r)$ est un ouvert de $\R^2$, il existe $\rho > 0$ tel que \[
		B_{(x,y)}(\rho) \subset B_{(a,b)}(r)
	\] donc $(x,y) \in \mathring D$.

	Donc $\mathring D$ est ouvert, $\mathring D \subset D$.

	Soit $O$ un ouvert de $\R^2$ tel que $O \subset D$. Montrons que $O \subset \mathring D$.

	Soit $(x,y) \in O$. Soit $r > 0$ tel que \[
		B_{(x,y)}(r) \subset O \subset D
	\] donc $(x,y) \in \mathring D$.
\end{prv}

\begin{defn}
	Soit $f: D \subset \R^2 \to \R$, $\ell \in \R$, $(a,b) \in \mathring D$.

	On dit que \underline{$f(x,y)$ tend vers $\ell$ quand $(x,y)$ tend vers $(a,b)$} ou que $\ell$ est \underline{une limite} de $f$ en $(a,b)$ si \[
		\forall \varepsilon > 0, \exists r > 0, \forall (x,y) \in D, \|(x,y) - (a,b)\| < r \implies \left| f(x,y) - \ell \right| \le \varepsilon.
	\] en d'autres termes si \[
		\forall V \in \mathcal{V}_{\ell}, \exists W \in \mathcal{V}_{(a,b)}, \forall (x,y) \in W \cap D, f(x,y) \in V.
	\]
	\index{limite (dans $\R^2$)}
	\index{tendre vers (dans $\R^2$)}
\end{defn}

\begin{prop}
	[unicité de la limite]
	Soit $f: D \to \R$, $(a,b) \in \mathring D$, $\ell_1, \ell_2 \in \R$ telles que $\ell_1$ et $\ell_2$ sont des limites de $f$ en $(a,b)$.

	Alors $\ell_1 = \ell_2$.
\end{prop}

\begin{figure}[H]
	\centering
	\incfig{preuve-unicité-de-la-limite}
\end{figure}

\begin{prv}
	On suppose $\ell_1 < \ell_2$. On pose $\varepsilon = \frac{\ell_2 - \ell_1}{2} > 0$.

	Soit $r_1 > 0$ tel que \[
		f\big(B_{(a,b)}(r_1)\big) \subset ]\ell_1 - \varepsilon, \ell_1 + \varepsilon[.
	\] Soit $r_2 > 0$ tel que \[
		f\big(B_{(a,b)}(r_2)\big) \subset ]\ell_2 - \varepsilon, \ell_2 + \varepsilon [.
	\] On pose $r = \min(r_1, r_2)$ donc \[
		B_{(a,b)}(r_1) \cap B_{(a,b)}(r_2) = B_{(a,b)}(r) \neq \O.
	\] Soit $(x,y) \in B_{(a,b)}(r)$. Alors, \[
		f(x,y) \in ]\ell_1 - \varepsilon, \ell_1 + \varepsilon[ \cap ]\ell_2 - \varepsilon, \ell_2 + \varepsilon[ = \O.
	\] $\lightning$
\end{prv}

\begin{defn}
	Soit $f : D \to \R$, $(a,b) \in \mathring D$.

	On dit que $f$ est \underline{continue} en $(a,b)$ si \[
		f(x,y) \tendsto{(x,y) \to (a,b)}f(a,b).
	\]
	\index{continuité (dans $\R^2$)}
\end{defn}

\begin{prop}
	\underline{Si} $f(x,y) \tendsto{(x,y) \to (a,b)} \ell$ \\
	\underline{alors} $\begin{cases}
		f(x,b) \tendsto{x \to a} \ell\\
		f(a,y) \tendsto{y \to b} \ell.\\
	\end{cases}$
\end{prop}

\begin{prv}~\\
	\begin{figure}[H]
		\centering
		\incfig{limite-x-en-a-et-y-en-b}
	\end{figure}
\end{prv}

\underline{Contre-exemple} : exercice 3.

\begin{exm}
	\begin{enumerate}
		\item $f : \begin{array}{rcl}
				\R^2 &\longrightarrow& \R \\
				(x,y) &\longmapsto& x
			\end{array}$ limite en $(0,0)$ ?

			Soit $\varepsilon > 0$. On pose $r = \varepsilon$. \[
				\forall (x,y) \in B_{(0,0)}(r),
				\left| f(x,y) \right| = \left| x \right| \le \|(x,y)\| < r = \varepsilon
			\] Donc $f(x,y) \tendsto{(x,y) \to (a,b)} 0$.
		\item limite $f : \begin{array}{rcl}
				\R^2 &\longrightarrow& \R \\
				(x,y) &\longmapsto& x^3
			\end{array}$ en $(0,0)$ ?

			Soit $\varepsilon > 0$. On pose $r = \sqrt[3]{r} > 0$. \[
				\forall (x,y) \in B_{(0,0)}(r),
				\left| f(x,y) \right| = \left| x^3 \right| \le \|(x,y)\|^3 < r^3 = \varepsilon.
			\]
		\item limite de $f : \begin{array}{rcl}
			\R^2 &\longrightarrow& \R \\
			(x,y) &\longmapsto& x^3y^2
		\end{array}$ en $(0,0)$ ?

		Soit $\varepsilon > 0$. On pose $r = \sqrt[5]{\varepsilon} > 0$. \[
			\forall (x,y) \in B_{(0,0)}(r), \left| f(x,y) \right| = \left| x^3 y^2 \right| \le \|(x,y)\|^3 \|(x,y)\|^2 < r^5 = \varepsilon.
		\]
	\end{enumerate}
\end{exm}

\begin{defn}
	Soient $D \subset \R^2$ et $(x,y) \in \R^2$.

	\begin{figure}[H]
    \centering
    \incfig{point-adhérent}
	\end{figure}
	
	On dit que $(x,y)$ est \underline{adhérent} à $D$ si \[
		\forall r > 0, B_{(x,y)}(r) \cap D \neq \O.
	\] L'ensemble des points adhérents à $D$ est noté $\overline{D}$. On dit que $\overline{D}$ est \underline{l'adhérence} de $D$.
	\index{point adhérent (dans $\R^2$)}
	\index{adhérent (dans $\R^2$)}
\end{defn}

\begin{defn}
	Soit $f: D \subset \R^2 \to \R$ et $(a,b) \in \overline{D}$, $\ell \in \R$. On dit que $f$ tend vers $\ell$ quand $(x,y)$ tend vers $(a,b)$ si \[
		\forall \varepsilon > 0, \exists r > 0, \forall (x,y) \in B_{(a,b)}(r) \cap D,
		\left| f(x,y) - \ell \right| \le \varepsilon.
	\]
	\index{limite (dans $\R^2$)}
	\index{tendre vers (dans $\R^2$)}
\end{defn}

\begin{prop}
	\begin{enumerate}
		\item Dans ce contexte, il y a unicité de la limite
		\item La limite d'une somme, d'un produit, d'un quotien, d'une composée se comporte comme dans le cas d'une seule variable.
		\item Soit $f: D \to \R$ continue. Soient $g: I \to \R$ et $h: I \to \R$ continues telles que \[
			\forall t \in I, \big(g(t), h(t)\big) \in D.
		\] Alors \[
			t \in I \mapsto f\big(g(t), h(t)\big) \in \R
		\] est continue.
	\end{enumerate}
\end{prop}

\begin{figure}[H]
	\centering
	\begin{asy}
		import three;
		import graph3;
		size(5cm);

		settings.render = 0;
		settings.prc = false;
		currentprojection = obliqueX;

		draw(O -- X, Arrow3(TeXHead2));
		draw(O -- Y, Arrow3(TeXHead2));
		draw(O -- Z, Arrow3(TeXHead2));

		triple f(real x, real y, real z = 0) { return (x,y,cos(x - 0.5) * cos(y - 0.5)/1.2 + 0.15); }

		real inc = 1 / 5;

		for(real x = 0; x <= 1; x += inc) {
			draw(graph(
				new real(real t) { return x; }, // x
				new real(real y) { return y; }, // y
				new real(real y) { return f(x,y).z; }, // z
				0, 1
			), gray);
		}

		for(real y = 0; y <= 1; y += inc) {
			draw(graph(
				new real(real x) { return x; }, // x
				new real(real t) { return y; }, // y
				new real(real x) { return f(x,y).z; }, // z
				0, 1
			), gray);
		}

		path3 path1 = (0.3, 0.2, 0) .. (0.5, 0.5, 0) .. (0.6, 0.7, 0) .. (0.9, 0.8, 0);
		path3 path2 = (0.3, 0.8, 0) .. (0.5, 0.5, 0) .. (0.6, 0.3, 0) .. (0.9, 0.2, 0);
		path3 pathA = f(0.3, 0.2, 0) .. f(0.5, 0.5, 0) .. f(0.6, 0.7, 0) .. f(0.9, 0.8, 0);
		path3 pathB = f(0.3, 0.8, 0) .. f(0.5, 0.5, 0) .. f(0.6, 0.3, 0) .. f(0.9, 0.2, 0);

		draw(path1, red, Arrow3(TeXHead2, position=0.5));
		draw(pathA, red, Arrow3(TeXHead2, position=0.5));
		draw(path2, deepcyan, Arrow3(TeXHead2, position=0.3));
		draw(pathB, deepcyan, Arrow3(TeXHead2, position=0.3));

		dot((0.5, 0.5, 0));
		dot(f(0.5, 0.5, 0));
		draw((0.5, 0.5, 0) -- f(0.5, 0.5, 0), dashed);
	\end{asy}
\end{figure}


	\part{Transpositions}

\begin{defn}
	Une \underline{transposition} est un cycle de longueur 2 : $\begin{pmatrix}
		a&b
	\end{pmatrix}$ avec $a \neq b$.
	\index{transposition (permutation)}
\end{defn}

\begin{exm}
	Avec $n = 5$ et $\gamma = \begin{pmatrix}
		2&4&1
	\end{pmatrix}$.

	\begin{figure}[H]
		\centering

		\begin{asy}
			size(5cm);

			real rho = 0.15; // circles

			void draw_cycle(pair O, real r ...int[] nums) {
				int n = nums.length;
				real eps = (15 / r) * 0.8;

				for(int i = 0; i < n; ++i) {
					real theta_1 = (360/n) * (i+1);
					real theta_2 = (360/n) * i;

					pair C = O + dir(theta_2) * r;

					draw(circle(C, rho));
					label("$" + string(nums[i]) + "$", C);
					draw(arc(O, r, theta_2+eps, theta_1-eps), Arrow(TeXHead));
				}
			}

			draw_cycle((-1,0), 0.8, 1, 2, 4);
			draw_cycle((1,0), 0.3, 3);
			draw_cycle((2,0), 0.3, 5);
		\end{asy}
	\end{figure}

	\[
		\gamma = \begin{pmatrix}
			1&4
		\end{pmatrix} \begin{pmatrix}
			1&2
		\end{pmatrix}
	\]

	Avec $n = 6$ et $\gamma = \begin{pmatrix}
		1&3&5&6&2
	\end{pmatrix} = \begin{pmatrix}
		1&2&3&4&5&6\\
		3&1&5&4&6&2
	\end{pmatrix}$.

	Donc, \[
		\gamma = \begin{pmatrix}
			1&2
		\end{pmatrix} \begin{pmatrix}
			1&6
		\end{pmatrix} \begin{pmatrix}
			1&5
		\end{pmatrix} \begin{pmatrix}
			1&3
		\end{pmatrix}
	\] 
	\[
		\begin{pmatrix}
			1&2&3&4&5&6\\
			3&2&1&4&5&6\\
			3&2&5&4&1&6\\
			3&2&5&4&6&1\\
			3&1&5&4&6&2\\
		\end{pmatrix}
	\]

	Et, \[
		\gamma = \begin{pmatrix}
			1&3
		\end{pmatrix} \begin{pmatrix}
			2&3
		\end{pmatrix} \begin{pmatrix}
			3&5
		\end{pmatrix} \begin{pmatrix}
			5&6
		\end{pmatrix} 
	\]

	\[
		\begin{pmatrix}
			1&2&3&4&5&6\\
			1&2&3&4&6&5\\
			1&2&5&4&6&3\\
			1&3&5&4&6&2\\
			3&1&5&4&6&2\\
		\end{pmatrix} 
	\] 
\end{exm}

\begin{exm}
	\[
		\begin{pmatrix}
			1&4
		\end{pmatrix} = \begin{pmatrix}
			1&2
		\end{pmatrix} \begin{pmatrix}
			2&3
		\end{pmatrix} \begin{pmatrix}
			3&4
		\end{pmatrix} \begin{pmatrix}
			2&3
		\end{pmatrix} \begin{pmatrix}
			1&2
		\end{pmatrix}
	\]
	On n'a pas toujours le même nombre de transpositions mais la parité du nombre reste la même (proposition plus loin).
\end{exm}

\begin{thm}
	Toute permutation se décompose en produit de transpositions.
\end{thm}

\begin{prv}
	Soit $\gamma = \begin{pmatrix}
		a_1&\cdots&a_k
	\end{pmatrix}$ un $k$-cycle.

	On remarque que
	\[
		\gamma = \begin{pmatrix}
			a_1&a_k
		\end{pmatrix} \cdots \begin{pmatrix}
			a_1&a_4
		\end{pmatrix} \begin{pmatrix}
			a_1&a_3
		\end{pmatrix} \begin{pmatrix}
			a_1&a_2
		\end{pmatrix}
	\] C'est un produit de transpositions.
\end{prv}

\begin{exm}
	Avec $n = 10$ et $\sigma = \begin{pmatrix}
		1&2&3&4&5&6&7&8&9&10\\
		9&8&1&7&2&3&4&5&10&6
	\end{pmatrix}$.

	On a
	\begin{align*}
		\sigma &= \begin{pmatrix}
			1&9&10&6&3
		\end{pmatrix} \begin{pmatrix}
			2&8&5
		\end{pmatrix} \begin{pmatrix}
			4&7
		\end{pmatrix}\\
		&= \begin{pmatrix}
			1&3
		\end{pmatrix} \begin{pmatrix}
			1&6
		\end{pmatrix} \begin{pmatrix}
			1&10
		\end{pmatrix} \begin{pmatrix}
			1&9
		\end{pmatrix} \begin{pmatrix}
			2&5
		\end{pmatrix} \begin{pmatrix}
			2&8
		\end{pmatrix} \begin{pmatrix}
			4&7
		\end{pmatrix} \\
	\end{align*}

	Vérification : \[
		\begin{pmatrix}
			1&2&3&4&5&6&7&8&9&10\\
			1&2&3&7&5&6&4&8&9&10\\
			1&8&3&7&5&6&4&2&9&10\\
			1&8&3&7&2&6&4&5&9&10\\
			9&8&3&7&2&6&4&5&1&10\\
			9&8&3&7&2&6&4&5&10&1\\
			9&8&3&7&2&1&4&5&10&6\\
			9&8&1&7&2&3&4&5&10&6\\
		\end{pmatrix} 
	\] 
\end{exm}


	\chap[11]{Suites numériques}
	\renewcommand{\cwd}{../chap11}
	\part{Modes de définition}

\begin{defn}
	Une suite peut être définie
	\begin{itemize}
		\item \underline{Explicitement}
			On dispose pour tout $n \in \N$ de l'expression de $u_n$ en fonction de $n$.\\
			\ex $\forall n \in \N_*, u_n = \frac{\ln(n)}{n}e^{-n}$\\
		\item \underline{Par récurrence}
			On connait $u_{n+1}$ en fonction de  $u_0, u_1, \ldots, u_n$\\
			\ex $\begin{cases}
				u_0=1\\
				\forall n \in \N, u_{n+1} = \sin(u_n)
			\end{cases}$\\
		\item \underline{Implicitement}
			$\forall n \in \N, u_n$ est le seul nombre verifiant une certaine propriété\\
			\ex $u_n$ est le seul réel vérifiant  $x^5 + nx - 1 = 0$
	\end{itemize}
\end{defn}

	\part{Topologie de $\R^2$}

\begin{defn}
	La \underline{norme (euclidienne)} de $\R^2$ est l'application définie par \[
		\forall (x,y) \in \R^2, \|(x,y)\| = \sqrt{x^2 + y^2}.
	\]

	\begin{figure}[H]
		\centering
		\begin{asy}
			import graph;
			axes(EndArrow);
			size(4cm);
			pair A = (3,2);
			dot(A);
			draw((3,0)--A, dashed);
			draw((0,2)--A, dashed);
			label("$x$", (3,0), align=S);
			label("$y$", (0,2), align=W);
			draw((0,0)--A);
			dot((4,3), white+0);
		\end{asy}
	\end{figure}
	\index{norme (de $\R^2$)}
	\index{norme euclidienne (de $\R^2$)}
\end{defn}

\begin{prop}
	La norme euclidienne vérifie:
	\begin{enumerate}
		\item (séparation) \[
			\forall (x,y) \in \R^2, \|(x,y)\| = 0 \iff x = y = 0,
			\]
		\item (homogénéité positive) \[
				\forall \lambda \in \R, \forall (x,y) \in \R^2, \|\lambda(x,y)\|= \left| \lambda \right| \|(x,y)\|
			\]
		\item (inégalité triangulaire) \[
			\forall (x,y), (a,b) \in \R^2,
			\|(x,y)+(a,b)\|\le \|(x,y)\|+\|(a,b)\|.
		\]
	\end{enumerate}
\end{prop}

\begin{prv}
	Déjà vue en replaçant $(x,y)$ par $x+iy \in \C$ et $\|(x,y)\|$ par |x+iy|
\end{prv}

\begin{defn}
	Soit $(a,b) \in \R^2$ et $r \in \R_+$.

	La \underline{boule ouverte} (ou \underline{disque ouvert}) de centre $(a,b)$ et de rayon $r$ est \[
		B_{(a,b)}(r) = \big\{ (x,y) \in \R^2  \mid \|(x,y) - (a,b)\| < r \big\}.
	\]

	La \underline{boule fermée} (ou \underline{disque fermé}) de centre $(a,b)$ et de rayon $r$ est \[
		\overline{B_{(a,b)}}(r) = \big\{ (x,y)\in \R^2  \mid \|(x,y) - (a,b)\| \le r \big\}.
	\]

	La \underline{sphère} (ou \underline{boule}) de centre $(a,b)$ et de rayon $r$ est \[
		S_{(a,b)}(r) = \partial \overline{B_{(a,b)}}(r) = \big\{ (x,y) \in \R^2  \mid \|(x,y) - (a,b)\| = r \big\}.
	\]
	\index{boule ouverte (de $\R^2$)}
	\index{disque ouverte (de $\R^2$)}
	\index{boule fermée (de $\R^2$)}
	\index{disque fermée (de $\R^2$)}
	\index{boule (de $\R^2$)}
	\index{sphère (de $\R^2$)}
\end{defn}

\begin{figure}[H]
		\centering
		\incfig{boule}
\end{figure}

\begin{rmk}
	On parle de boule en dimension quelconque.
\end{rmk}

\begin{defn}
	Une \underline{partie ouverte} $O$ de $\R^2$ (ou \underline{un ouvert}) si \[
		\forall (x,y) \in O, \exists r > 0, B_{(a,b)}(r) \subset O.
	\]
	Une partie $F$ est \underline{fermée} su $\R^2\setminus F$ est ouverte.
	\index{partie ouverte (de $\R^2$)}
	\index{ouvert (de $\R^2$)}
	\index{partie fermée (de $\R^2$)}
\end{defn}

\begin{figure}[H]
	\centering
	\incfig{partie-ouverte}
\end{figure}

\begin{prop}
	Une boule ouverte est ouverte. Une boule fermée est fermée.
\end{prop}

\begin{figure}[H]
	\centering
	\begin{subfigure}{4cm}
		\centering
		\begin{asy}
			import patterns;

			pair n(pair a) {return a / length(a);}

			add("hatch",hatch(2mm, SW, red));
			size(4cm);

			draw(circle((0,0), 1));
			dot('$(a_0, b_0)$', (0,0), align=S);

			draw((0,0) -- n((-1, 1)), dashed);
			label("$r$", n((-1, 1)) / 2, align=1.5S);

			pair A = n((1,3)) * (2/3);
			real rho = (1 - length(A)) * (2 / 3);

			dot("$(a,b)$", A, red, align=3SE);
			filldraw(circle(A, rho), pattern("hatch"), red);

			label("$O$", n((1,-1))*2.5/3);
		\end{asy}
	\end{subfigure}
	\begin{subfigure}{1cm}
		\centering~\\
	\end{subfigure}
	\begin{subfigure}{5cm}
		\centering
		\begin{asy}
			import patterns;

			pair n(pair a) {return a / length(a);}

			add("hatch",hatch(1mm, SW, red));
			add("hatch2",hatch(3mm, SE, blue));
			size(5cm);

			guide around = (-1.5, -1.5) -- (-1.5, 1.5) -- (2.5, 1.5) -- (2.5, -1.5) -- cycle;

			pair A = n((3, 1)) * 5/3; 
			real rho = 2 / 9;

			picture inter;
			fill(inter, around, pattern("hatch2"));
			fill(inter, circle((0,0), 1), white);
			add(inter);

			draw(circle((0,0), 1));
			dot('$(a_0, b_0)$', (0,0), align=S);

			draw((0,0) -- n((-1, 1)), dashed);
			label("$r$", n((-1, 1)) / 2, align=1.5S);

			dot("$(a,b)$", A, red, align=2SE);
			filldraw(circle(A, rho), pattern("hatch"), red);

			label("$F$", n((1,-1))*2.5/3);
		\end{asy}
	\end{subfigure}
\end{figure}

\begin{prv}
	$\O$ est un ouvert.

	Soit $B$ la boule ouverte de centre $(a_0, b_0) \in \R^2$ et de rayon $r > 0$.

	On pose $\rho = \frac{1}{2}\big(r - \|(a,b) - (a_0,b_0)\|\big)$.
	Montrons que \[
		B_{(a,b)}(\rho) \subset  B_{(a,b)}(r).
	\]

	Soit $(x,y) \in B_{(a,b)}(\rho)$.
	\begin{align*}
		\|(x,y) - (a_0,b_0)\|&= \|(x,y)- (a,b) + (a,b) - (a_0,b_0)\| \\
		&\le \|(x,y) - (a,b)\| + \|(a,b) - (a_0, b_0)\|\\
		&< \rho + \|(a,b) - (a_0, b_0)\| = \frac{1}{2}r + \frac{1}{2} \|(a,b) - (a_0, b_0)\|\\
		&< r
	\end{align*}
	
	Soit $F$ la boule fermée de centre $(a_0, b_0)$ et de rayon $r \ge 0$.

	Soit $(a,b) \not\in F$. On pose \[
		\rho = \frac{1}{2}\big(\|(a,b) - (a_0, b_0)\| - r\big) > 0.
	\]

	Montrons que $B_{(a,b)}(\rho) \subset \R^2\setminus F$.

	Soit $(x,y) \in B_{(a,b)}(\rho)$.

	\begin{align*}
		\|(x,y) - (a_0, b_0)\| &= \|(x,y) - (a,b) + (a,b) - (a_0, b_0)\| \\
		&\ge \big| \underbrace{\|(x,y) - (a,b)\|}_{\le \rho} - \underbrace{\|(a,b) - (a_0, b_0)\|}_{> r} \big|\\
		&\ge \|(a,b) - (a_0, b_0)\|- \|(x,y) - (a,b)\|\\
		&> \|(a,b) - (a_0, b_0)\|- \rho\\
		&> \frac{1}{2} \|(a,b) - (a_0, b_0)\| + \frac{1}{2}r\\
		&> r
	\end{align*}

	donc $(x,y) \not\in F$.
\end{prv}

\begin{exm}
	\begin{enumerate}
		\item $\O$ est ouvert.\\
			$\R^2$ est ouvert.
		\item $\O$ est fermé.\\
			$\R^2$ est fermé.\\
		\item $\big\{(x,0)  \mid x > 0\big\}$ n'est ni ouverte ni fermé.
	\end{enumerate}
\end{exm}

\begin{figure}[H]
	\centering
	\begin{asy}
		size(3cm);

		draw((0, -1) -- (0, 3), Arrow(TeXHead));
		draw((-1, 0) -- (3, 0), Arrow(TeXHead));
		
		draw((0,0) -- (0, 2.97), red);
		draw(circle((0,1.5), 0.5), deepred);
		draw(circle((0,0.5), 0.1), deepred);
	\end{asy}
\end{figure}

\begin{defn}
	Soit $(a,b) \in \R^2$ et $V \in \mathcal{P}(\R^2)$.

	On dit que $V$ est un voisinage de $(a,b)$ s'il existe $r > 0$ tel que \[
		B_{(a,b)}(r) \subset V.
	\]
	\index{voisinage (dans $\R^2$)}
\end{defn}

\begin{prop}
	Un ouvert non vide est un voisinage en chacun de ces points. \qed
\end{prop}

\begin{defn}
	Soit $D \subset \R^2$. Un \underline{point intérieur} de $D$ est un couple $(a,b) \in D$ tel que \[
		\exists r > 0, B_{(a,b)}(r) \subset D.
	\] en d'autres termes, si $D$ est un voisinage de $(a,b)$.

	On note $\mathring D$ l'ensemble des points intérieurs à $D$. C'est \underline{l'intérieur} de $D$.
	\index{point intérieur (dans $\R^2$)}
	\index{intérieur (dans $\R^2$)}
\end{defn}

\begin{prop}
	$\mathring D$ est le plus grand ouvert $O$ de $\R^2$ tel que $O \subset D$.
\end{prop}

\begin{figure}[H]
	\centering
	\incfig{interieur-plus-grand-ouvert}
\end{figure}


\begin{prv}
	Soit $(a,b) \in \mathring D$.

	Par définition, il existe $r > 0$ tel que \[
		B_{(a,b)}(r) \subset D.
	\] Montrons que $B_{(a,b)}(r) \subset \mathring D$.

	Soit $(x,y) \in B_{(a,b)}(r)$. Comme $B_{(a,b)}(r)$ est un ouvert de $\R^2$, il existe $\rho > 0$ tel que \[
		B_{(x,y)}(\rho) \subset B_{(a,b)}(r)
	\] donc $(x,y) \in \mathring D$.

	Donc $\mathring D$ est ouvert, $\mathring D \subset D$.

	Soit $O$ un ouvert de $\R^2$ tel que $O \subset D$. Montrons que $O \subset \mathring D$.

	Soit $(x,y) \in O$. Soit $r > 0$ tel que \[
		B_{(x,y)}(r) \subset O \subset D
	\] donc $(x,y) \in \mathring D$.
\end{prv}

\begin{defn}
	Soit $f: D \subset \R^2 \to \R$, $\ell \in \R$, $(a,b) \in \mathring D$.

	On dit que \underline{$f(x,y)$ tend vers $\ell$ quand $(x,y)$ tend vers $(a,b)$} ou que $\ell$ est \underline{une limite} de $f$ en $(a,b)$ si \[
		\forall \varepsilon > 0, \exists r > 0, \forall (x,y) \in D, \|(x,y) - (a,b)\| < r \implies \left| f(x,y) - \ell \right| \le \varepsilon.
	\] en d'autres termes si \[
		\forall V \in \mathcal{V}_{\ell}, \exists W \in \mathcal{V}_{(a,b)}, \forall (x,y) \in W \cap D, f(x,y) \in V.
	\]
	\index{limite (dans $\R^2$)}
	\index{tendre vers (dans $\R^2$)}
\end{defn}

\begin{prop}
	[unicité de la limite]
	Soit $f: D \to \R$, $(a,b) \in \mathring D$, $\ell_1, \ell_2 \in \R$ telles que $\ell_1$ et $\ell_2$ sont des limites de $f$ en $(a,b)$.

	Alors $\ell_1 = \ell_2$.
\end{prop}

\begin{figure}[H]
	\centering
	\incfig{preuve-unicité-de-la-limite}
\end{figure}

\begin{prv}
	On suppose $\ell_1 < \ell_2$. On pose $\varepsilon = \frac{\ell_2 - \ell_1}{2} > 0$.

	Soit $r_1 > 0$ tel que \[
		f\big(B_{(a,b)}(r_1)\big) \subset ]\ell_1 - \varepsilon, \ell_1 + \varepsilon[.
	\] Soit $r_2 > 0$ tel que \[
		f\big(B_{(a,b)}(r_2)\big) \subset ]\ell_2 - \varepsilon, \ell_2 + \varepsilon [.
	\] On pose $r = \min(r_1, r_2)$ donc \[
		B_{(a,b)}(r_1) \cap B_{(a,b)}(r_2) = B_{(a,b)}(r) \neq \O.
	\] Soit $(x,y) \in B_{(a,b)}(r)$. Alors, \[
		f(x,y) \in ]\ell_1 - \varepsilon, \ell_1 + \varepsilon[ \cap ]\ell_2 - \varepsilon, \ell_2 + \varepsilon[ = \O.
	\] $\lightning$
\end{prv}

\begin{defn}
	Soit $f : D \to \R$, $(a,b) \in \mathring D$.

	On dit que $f$ est \underline{continue} en $(a,b)$ si \[
		f(x,y) \tendsto{(x,y) \to (a,b)}f(a,b).
	\]
	\index{continuité (dans $\R^2$)}
\end{defn}

\begin{prop}
	\underline{Si} $f(x,y) \tendsto{(x,y) \to (a,b)} \ell$ \\
	\underline{alors} $\begin{cases}
		f(x,b) \tendsto{x \to a} \ell\\
		f(a,y) \tendsto{y \to b} \ell.\\
	\end{cases}$
\end{prop}

\begin{prv}~\\
	\begin{figure}[H]
		\centering
		\incfig{limite-x-en-a-et-y-en-b}
	\end{figure}
\end{prv}

\underline{Contre-exemple} : exercice 3.

\begin{exm}
	\begin{enumerate}
		\item $f : \begin{array}{rcl}
				\R^2 &\longrightarrow& \R \\
				(x,y) &\longmapsto& x
			\end{array}$ limite en $(0,0)$ ?

			Soit $\varepsilon > 0$. On pose $r = \varepsilon$. \[
				\forall (x,y) \in B_{(0,0)}(r),
				\left| f(x,y) \right| = \left| x \right| \le \|(x,y)\| < r = \varepsilon
			\] Donc $f(x,y) \tendsto{(x,y) \to (a,b)} 0$.
		\item limite $f : \begin{array}{rcl}
				\R^2 &\longrightarrow& \R \\
				(x,y) &\longmapsto& x^3
			\end{array}$ en $(0,0)$ ?

			Soit $\varepsilon > 0$. On pose $r = \sqrt[3]{r} > 0$. \[
				\forall (x,y) \in B_{(0,0)}(r),
				\left| f(x,y) \right| = \left| x^3 \right| \le \|(x,y)\|^3 < r^3 = \varepsilon.
			\]
		\item limite de $f : \begin{array}{rcl}
			\R^2 &\longrightarrow& \R \\
			(x,y) &\longmapsto& x^3y^2
		\end{array}$ en $(0,0)$ ?

		Soit $\varepsilon > 0$. On pose $r = \sqrt[5]{\varepsilon} > 0$. \[
			\forall (x,y) \in B_{(0,0)}(r), \left| f(x,y) \right| = \left| x^3 y^2 \right| \le \|(x,y)\|^3 \|(x,y)\|^2 < r^5 = \varepsilon.
		\]
	\end{enumerate}
\end{exm}

\begin{defn}
	Soient $D \subset \R^2$ et $(x,y) \in \R^2$.

	\begin{figure}[H]
    \centering
    \incfig{point-adhérent}
	\end{figure}
	
	On dit que $(x,y)$ est \underline{adhérent} à $D$ si \[
		\forall r > 0, B_{(x,y)}(r) \cap D \neq \O.
	\] L'ensemble des points adhérents à $D$ est noté $\overline{D}$. On dit que $\overline{D}$ est \underline{l'adhérence} de $D$.
	\index{point adhérent (dans $\R^2$)}
	\index{adhérent (dans $\R^2$)}
\end{defn}

\begin{defn}
	Soit $f: D \subset \R^2 \to \R$ et $(a,b) \in \overline{D}$, $\ell \in \R$. On dit que $f$ tend vers $\ell$ quand $(x,y)$ tend vers $(a,b)$ si \[
		\forall \varepsilon > 0, \exists r > 0, \forall (x,y) \in B_{(a,b)}(r) \cap D,
		\left| f(x,y) - \ell \right| \le \varepsilon.
	\]
	\index{limite (dans $\R^2$)}
	\index{tendre vers (dans $\R^2$)}
\end{defn}

\begin{prop}
	\begin{enumerate}
		\item Dans ce contexte, il y a unicité de la limite
		\item La limite d'une somme, d'un produit, d'un quotien, d'une composée se comporte comme dans le cas d'une seule variable.
		\item Soit $f: D \to \R$ continue. Soient $g: I \to \R$ et $h: I \to \R$ continues telles que \[
			\forall t \in I, \big(g(t), h(t)\big) \in D.
		\] Alors \[
			t \in I \mapsto f\big(g(t), h(t)\big) \in \R
		\] est continue.
	\end{enumerate}
\end{prop}

\begin{figure}[H]
	\centering
	\begin{asy}
		import three;
		import graph3;
		size(5cm);

		settings.render = 0;
		settings.prc = false;
		currentprojection = obliqueX;

		draw(O -- X, Arrow3(TeXHead2));
		draw(O -- Y, Arrow3(TeXHead2));
		draw(O -- Z, Arrow3(TeXHead2));

		triple f(real x, real y, real z = 0) { return (x,y,cos(x - 0.5) * cos(y - 0.5)/1.2 + 0.15); }

		real inc = 1 / 5;

		for(real x = 0; x <= 1; x += inc) {
			draw(graph(
				new real(real t) { return x; }, // x
				new real(real y) { return y; }, // y
				new real(real y) { return f(x,y).z; }, // z
				0, 1
			), gray);
		}

		for(real y = 0; y <= 1; y += inc) {
			draw(graph(
				new real(real x) { return x; }, // x
				new real(real t) { return y; }, // y
				new real(real x) { return f(x,y).z; }, // z
				0, 1
			), gray);
		}

		path3 path1 = (0.3, 0.2, 0) .. (0.5, 0.5, 0) .. (0.6, 0.7, 0) .. (0.9, 0.8, 0);
		path3 path2 = (0.3, 0.8, 0) .. (0.5, 0.5, 0) .. (0.6, 0.3, 0) .. (0.9, 0.2, 0);
		path3 pathA = f(0.3, 0.2, 0) .. f(0.5, 0.5, 0) .. f(0.6, 0.7, 0) .. f(0.9, 0.8, 0);
		path3 pathB = f(0.3, 0.8, 0) .. f(0.5, 0.5, 0) .. f(0.6, 0.3, 0) .. f(0.9, 0.2, 0);

		draw(path1, red, Arrow3(TeXHead2, position=0.5));
		draw(pathA, red, Arrow3(TeXHead2, position=0.5));
		draw(path2, deepcyan, Arrow3(TeXHead2, position=0.3));
		draw(pathB, deepcyan, Arrow3(TeXHead2, position=0.3));

		dot((0.5, 0.5, 0));
		dot(f(0.5, 0.5, 0));
		draw((0.5, 0.5, 0) -- f(0.5, 0.5, 0), dashed);
	\end{asy}
\end{figure}


	\part{Transpositions}

\begin{defn}
	Une \underline{transposition} est un cycle de longueur 2 : $\begin{pmatrix}
		a&b
	\end{pmatrix}$ avec $a \neq b$.
	\index{transposition (permutation)}
\end{defn}

\begin{exm}
	Avec $n = 5$ et $\gamma = \begin{pmatrix}
		2&4&1
	\end{pmatrix}$.

	\begin{figure}[H]
		\centering

		\begin{asy}
			size(5cm);

			real rho = 0.15; // circles

			void draw_cycle(pair O, real r ...int[] nums) {
				int n = nums.length;
				real eps = (15 / r) * 0.8;

				for(int i = 0; i < n; ++i) {
					real theta_1 = (360/n) * (i+1);
					real theta_2 = (360/n) * i;

					pair C = O + dir(theta_2) * r;

					draw(circle(C, rho));
					label("$" + string(nums[i]) + "$", C);
					draw(arc(O, r, theta_2+eps, theta_1-eps), Arrow(TeXHead));
				}
			}

			draw_cycle((-1,0), 0.8, 1, 2, 4);
			draw_cycle((1,0), 0.3, 3);
			draw_cycle((2,0), 0.3, 5);
		\end{asy}
	\end{figure}

	\[
		\gamma = \begin{pmatrix}
			1&4
		\end{pmatrix} \begin{pmatrix}
			1&2
		\end{pmatrix}
	\]

	Avec $n = 6$ et $\gamma = \begin{pmatrix}
		1&3&5&6&2
	\end{pmatrix} = \begin{pmatrix}
		1&2&3&4&5&6\\
		3&1&5&4&6&2
	\end{pmatrix}$.

	Donc, \[
		\gamma = \begin{pmatrix}
			1&2
		\end{pmatrix} \begin{pmatrix}
			1&6
		\end{pmatrix} \begin{pmatrix}
			1&5
		\end{pmatrix} \begin{pmatrix}
			1&3
		\end{pmatrix}
	\] 
	\[
		\begin{pmatrix}
			1&2&3&4&5&6\\
			3&2&1&4&5&6\\
			3&2&5&4&1&6\\
			3&2&5&4&6&1\\
			3&1&5&4&6&2\\
		\end{pmatrix}
	\]

	Et, \[
		\gamma = \begin{pmatrix}
			1&3
		\end{pmatrix} \begin{pmatrix}
			2&3
		\end{pmatrix} \begin{pmatrix}
			3&5
		\end{pmatrix} \begin{pmatrix}
			5&6
		\end{pmatrix} 
	\]

	\[
		\begin{pmatrix}
			1&2&3&4&5&6\\
			1&2&3&4&6&5\\
			1&2&5&4&6&3\\
			1&3&5&4&6&2\\
			3&1&5&4&6&2\\
		\end{pmatrix} 
	\] 
\end{exm}

\begin{exm}
	\[
		\begin{pmatrix}
			1&4
		\end{pmatrix} = \begin{pmatrix}
			1&2
		\end{pmatrix} \begin{pmatrix}
			2&3
		\end{pmatrix} \begin{pmatrix}
			3&4
		\end{pmatrix} \begin{pmatrix}
			2&3
		\end{pmatrix} \begin{pmatrix}
			1&2
		\end{pmatrix}
	\]
	On n'a pas toujours le même nombre de transpositions mais la parité du nombre reste la même (proposition plus loin).
\end{exm}

\begin{thm}
	Toute permutation se décompose en produit de transpositions.
\end{thm}

\begin{prv}
	Soit $\gamma = \begin{pmatrix}
		a_1&\cdots&a_k
	\end{pmatrix}$ un $k$-cycle.

	On remarque que
	\[
		\gamma = \begin{pmatrix}
			a_1&a_k
		\end{pmatrix} \cdots \begin{pmatrix}
			a_1&a_4
		\end{pmatrix} \begin{pmatrix}
			a_1&a_3
		\end{pmatrix} \begin{pmatrix}
			a_1&a_2
		\end{pmatrix}
	\] C'est un produit de transpositions.
\end{prv}

\begin{exm}
	Avec $n = 10$ et $\sigma = \begin{pmatrix}
		1&2&3&4&5&6&7&8&9&10\\
		9&8&1&7&2&3&4&5&10&6
	\end{pmatrix}$.

	On a
	\begin{align*}
		\sigma &= \begin{pmatrix}
			1&9&10&6&3
		\end{pmatrix} \begin{pmatrix}
			2&8&5
		\end{pmatrix} \begin{pmatrix}
			4&7
		\end{pmatrix}\\
		&= \begin{pmatrix}
			1&3
		\end{pmatrix} \begin{pmatrix}
			1&6
		\end{pmatrix} \begin{pmatrix}
			1&10
		\end{pmatrix} \begin{pmatrix}
			1&9
		\end{pmatrix} \begin{pmatrix}
			2&5
		\end{pmatrix} \begin{pmatrix}
			2&8
		\end{pmatrix} \begin{pmatrix}
			4&7
		\end{pmatrix} \\
	\end{align*}

	Vérification : \[
		\begin{pmatrix}
			1&2&3&4&5&6&7&8&9&10\\
			1&2&3&7&5&6&4&8&9&10\\
			1&8&3&7&5&6&4&2&9&10\\
			1&8&3&7&2&6&4&5&9&10\\
			9&8&3&7&2&6&4&5&1&10\\
			9&8&3&7&2&6&4&5&10&1\\
			9&8&3&7&2&1&4&5&10&6\\
			9&8&1&7&2&3&4&5&10&6\\
		\end{pmatrix} 
	\] 
\end{exm}

	\part{Familles orthogonales}

\begin{thm}[Pythagore]
	Soit $(x,y) \in E^2$. \[
		\|x+y\|^2 = \|x\|^2 + \|y\|^2 \iff x \perp y
	.\]
	\begin{figure}[H]
		\centering
		\begin{asy}
			size(4cm);
			pair u = (1, 0.5);
			pair v = 1.5 * (0, 1) * u;
			draw((0,0)--u, Arrow(TeXHead));
			label("$x$", u/2, align=S);
			draw(u--v+u, Arrow(TeXHead));
			label("$y$", u + v/2, align=NE);
			draw((0,0) -- u + v, Arrow(TeXHead));
			draw(u + v / 7.5 -- u + v / 7.5 - u / 5 -- u - u / 5 -- u -- cycle);
		\end{asy}
	\end{figure}
\end{thm}

\begin{prv}
	\[
		\|x + y\|^2 = \|x\|^2 + \|y\|^2 \iff 2\left<x \mid y \right> = 0 \iff x \perp y
	.\]
\end{prv}

\begin{defn}
	Soit $(e_i)_{i\in I}$ une famille de vecteurs. On dit que cette famille est \underline{orthogonale} si \[
		\forall i \neq j\, e_i \perp e_j
	.\] Si, en plus, on a \[
		\forall i \in I,\,\|e_i\| = 1,
	\] alors on dit que la famille est \underline{orthonormale} ou \underline{orthonormée}.
	\index{famille orthogonale}
	\index{famille orthonormale}
	\index{famille orthonormée}
\end{defn}

\begin{prop}[Pythagore]
	Soit $(e_1, \ldots, e_n)$ une famille orthogonale. Alors \[
		\left\| \sum_{i=1}^n e_i \right\|^2 = \sum_{i=1}^n \|e_i\|^2
	.\]
\end{prop}

\begin{thm}
	Toute famille orthogonale de vecteurs non nuls est libre.
\end{thm}

\begin{prv}
	Soit $(e_i)_{i\in I}$ une famille orthogonale telle que \[
		\forall i \in I,\,e_i \neq 0_E
	.\] Soit $n \in \N^*$, $(\lambda_1, \ldots, \lambda_n) \in \R^n$. On suppose \[
		\sum_{k=1}^n \lambda_k e_{i_k} = 0_E
	.\] Soit $j \in \left\llbracket 1,n \right\rrbracket$.
	\begin{align*}
		0 &= \left<\sum_{k=1}^n \lambda_k e_{i_k}  \mid e_{i_j} \right>\\
		&= \sum_{k=1}^n \lambda_k \left<e_{i_k}  \mid e_{i_j} \right> \\
		&= \lambda_j \underbrace{\|e_{i_j}\|^2}_{\neq 0} \\
	\end{align*}
	donc $\lambda_j = 0$.
\end{prv}

\begin{algo}[Orthonormalisation de Gran--Schmidt]
	On suppose $E$ de dimension finie. Soit $\mathcal{B} = (e_1, \ldots, e_n)$ une base de $E$.

	\begin{itemize}
		\item\underline{\it Étape 1}: On pose $v_1 = \frac{e_1}{\|e_1\|}$ de sorte que $\|v_1\| = 1$.
		\item\underline{\it Étape 2} : On pose \[
				u_2 = e_2 - \left<e_2  \mid v_1 \right> v_1
			.\] Ainsi,
			\begin{align*}
				\left<u_2 \mid v_1 \right> &= \big<e_2 - \left<e_2 \mid v_1 \right> v_1  \mid v_1 \big>\\
				&= \left<e_2 \mid v_1 \right> - \left<e_2 \mid v_1 \right> \left<v_1 \mid v_1 \right> \\
				&= 0. \\
			\end{align*}
			On pose $v_2 = \frac{u_2}{\|u_2\|}$ donc $v_2 \perp v_1$ et $\|v_2\| = 1$.
		\item\underline{\it Étape 3} : On pose \[
				u_2 = e_3 - \left<e_3 \mid v_1 \right>v_1 - \left<e_3 \mid v_2 \right>v_2
			.\] Ainsi,
			\begin{align*}
				\left<u_3  \mid v_1 \right> &= \left<e_3  \mid v_1 \right> - \left<e_3 \mid v_1 \right>\underbrace{\left<v_1 \mid v_1 \right>}_{=1} - \left<e_3 \mid v_2 \right>\underbrace{\left<v_2 \mid v_1 \right>}_{=0} \\
				&= 0 \\
			\end{align*}
			et 
			\begin{align*}
				\left<u_3 \mid v_2 \right> &= \left<e_3  \mid  v_2 \right> - \left<e_3 \mid v_1 \right> \underbrace{\left<v_1 \mid v_2 \right>}_{=0} - \left<e_3 \mid v_2 \right> \underbrace{\left<v_2 \mid v_2 \right>}_{=1}\\
				&= 0. \\
			\end{align*}
			On pose $v_3 = \frac{u_3}{\|u_3\|}$ de sorte que $v_3 \perp v_1$, $v_3 \perp v_2$ et $\|v_3\| = 1$.
		\item\underline{\it Étape $i+1$}: On pose \[
			u_{i+1} = e_{i+1} - \sum_{k=1}^i \left<e_{i+1}  \mid v_k \right> v_k
		.\] Ainsi, pour tout $j \in \left\llbracket 1,i \right\rrbracket,$ on a
		\begin{align*}
			\left<u_{i+1}  \mid v_j \right> &= \left<e_{i+1}  \mid v_j \right> - \sum_{k=1}^i \left<e_{i+1} \mid v_k \right> \left<v_k \mid v_j \right> \\
			&= \left<e_{i+1} \mid v_j \right> - \left<e_{i+1} \mid v_j \right> \|v_j\|^2 \\
			&= 0. \\
		\end{align*}
		On pose $v_{i+1} = \frac{u_{i+1}}{\|u_{i+1}\|}$.
	\end{itemize}
\end{algo}

\begin{exm}
	Avec $E = \R_3[X]$, $\left<P \mid Q \right> = \int_{0}^{1} P(t)\,Q(t)~\mathrm{d}t$ et $\mathcal{B} = (1, X, X^2, X^3)$.
	\begin{enumerate}
		\item $\|1\|^2 = \left<1 \mid 1 \right> = \int_{0}^{1} 1~\mathrm{d}t = 1$ et donc $v_1 = 1$.
		\item $u_2 = X - \left<X  \mid v_1 \right>v_1$. Or, $\left<X \mid v_1 \right> = \int_{0}^{1} t~\mathrm{d}t = \frac{1}{2}$. D'où $u_2 = X - \frac{1}{2}$.
			\begin{align*}
				\|u_2\|^2 &= \int_{0}^{1} \left( t - \frac{1}{2} \right)^2~\mathrm{d}t \\
				&= \int_{0}^{1} \left( t^2 - t + \frac{1}{4} \right)~\mathrm{d}t \\
				&= \frac{1}{3} - \frac{1}{2} + \frac{1}{4} \\
				&= \frac{1}{12} \\
			\end{align*} On en déduit que $v_2 = \sqrt{12}\left( X - \frac{1}{2} \right)$.
		\item $u_3 = X^2 - \left<X^2 \mid v_1 \right>v_1 - \left<X^2 \mid v_2 \right>v_2$.
			On a \[
				\left<X^2 \mid v_1 \right> = \int_{0}^{1} t^2~\mathrm{d}t = \frac{1}{3}
			\] et
			\begin{align*}
				\left<X^2 \mid v_2 \right> &= \sqrt{12} \int_{0}^{1} t^2\left( t - \frac{1}{2} \right)~\mathrm{d}t \\
				&= \frac{\sqrt{12}}{12}. \\
			\end{align*}
			D'où
			\begin{align*}
				u_3 &= X^2 - \frac{1}{3} - \frac{\sqrt{12}}{12}\sqrt{12} \left( X - \frac{1}{2} \right)\\
				&= X^2 - \frac{1}{3} - X + \frac{1}{2} \\
				&= X^2 - X + \frac{1}{6}. \\
			\end{align*}
			\begin{align*}
				\|u_3\|^2 &= \int_{0}^{1} \left( t^2 - t + \frac{1}{6} \right)~\mathrm{d}t\\
				&= \int_{0}^{1} \left( t^4 + t^2 + \frac{1}{36} - 2t^3 + \frac{t^2}{3} - \frac{t}{3} \right) ~\mathrm{d}t \\
				&= \frac{1}{5} + \frac{1}{3} + \frac{1}{36} - \frac{1}{2} + \frac{1}{9} - \frac{1}{6} \\
				&= \frac{36 + 60 + 5 - 90 + 20 - 30}{180} \\
				&= \frac{1}{180} \\
			\end{align*}
			On en déduit que \[
				v_3 = 6\sqrt{5}\left( X^2 - X + \frac{1}{6} \right).
			\]
		\item Exercice : calculer $v_4$.
	\end{enumerate}
\end{exm}

\begin{prop}
	Soit $\mathcal{B} = (e_1, \ldots, e_n)$ une base de $E$ et $\mathcal{C}$ la base obtenue par le procédé d'orthonormalisation de Gram--Schmidt. Alors, \[
		\forall i \in \left\llbracket 1,n \right\rrbracket,\,\Vect(e_1,\ldots, e_i) = \Vect(v_1, \ldots, v_i)
	.\]\qed
\end{prop}

\begin{exm}[orthogonalisation]
	\begin{itemize}
		\item $u_1 = 1$.
		\item
			\begin{align*}
				\begin{rcases*}
					u_2 \in \Vect(e_1, e_2)\\
					u_2 \perp u_1
				\end{rcases*}
				\iff& \begin{cases}
					u_2 = ae_1 + be_2\quad (a,b) \in \R^2\\
					\left<u_1 \mid u_2 \right> = 0
				\end{cases}\\
				\iff& \begin{cases}
					u_2 = a + bX\\
					\int_{0}^{1} (a+bt)~\mathrm{d}t = 0.
				\end{cases}\\
			\end{align*}
			\begin{align*}
				\int_{0}^{1} (a+bt)~\mathrm{d}t = 0 \iff& a + \frac{b}{2} = 0\\
				\iff& a = -\frac{b}{2}\\
				\iff& u_2 = -\frac{b}{2} + bX.
			\end{align*}
			Par exemple, $u_2 = -1 + 2X$.
		\item $\begin{cases}
				u_3 \in \Vect(e_1, e_2, e_3)\\
				u_3 \perp u_1\\
				u_3 \perp u_2
			\end{cases}$

			On pose $u_3 = a + bX + cX^2$ avec $(a,b,c)\in \R^3$.
			\begin{align*}
				\begin{rcases*}
					\int_{0}^{1} \left( a+bt + ct^2 \right)~\mathrm{d}t = 0\\
					\int_{0}^{1} \left(a + bt+ct^2\right)(2t - 1)~\mathrm{d}t = 0
				\end{rcases*} \iff& \begin{cases}
					a + \frac{b}{2} + \frac{c}{3} = 0\\
					\int_{0}^{1} \left( 2ct^3 + (-c + 2b)t^2 + (2a - b)t - a \right) ~\mathrm{d}t = 0
				\end{cases}\\
				\iff& \begin{cases}
					a + \frac{b}{2} + \frac{c}{3} = 0\\
					\frac{c}{2} + \frac{2b - c}{3} + \frac{2\cancel{a} - b}{2} - \cancel{a} = 0
				\end{cases}\\
				\iff&  \begin{cases}
					a = -\frac{b}{2} - \frac{c}{3} = \frac{c}{2} - \frac{c}{3} = \frac{c}{6}\\
					b = -c.
				\end{cases}
			\end{align*}
			On en déduit que \[
				u_3 = 1 - 6X + 6X^2
			.\]
	\end{itemize}
\end{exm}

\begin{crlr}[théorème de la base orthonormée incomplète] Soit $(e_1, \ldots, e_k)$ une base orthonormée d'un espace euclidien. On peut trouver $e_{k+1},\ldots,e_n$ tels que $(e_1, \ldots, e_k, e_{k+1},\ldots,e_n)$ soit une base orthonormée de $E$.
\end{crlr}

\begin{prv}
	On sait que $(e_1, \ldots, e_k)$ est libre. On complète $(e_1, \ldots, e_k)$ en une base $\mathcal{B}$ de $E$. On orthonormalise $\mathcal{B}$ : on obtient une base orthonormée $\mathcal{C}$ de $E$. En détaillant l'algorithme de Gram--Schmidt, on s'aper\c coit que les $k$ premiers vecteurs de $\mathcal{C}$ sont ceux de $\mathcal{B}$.
\end{prv}

\begin{thm}
	Soit $E$ un espace euclidien et $\mathcal{B} = (e_1, \ldots, e_n)$ une base orthonormée de $E$. Soit $(x,y) \in E^2$. On pose $(x_1, \ldots, x_n) \in \R^n$ et $(y_1, \ldots, y_n) \in \R^n$ tels que \[
		x = \sum_{i=1}^n x_i e_i \qquad\qquad y = \sum_{i=1}^n y_i e_i
	.\] Alors \[
		\left<x \mid y \right> = \sum_{i=1}^n x_i y_i
	.\]
	\vspace{3mm}
	Soit $X = \mat{x_1\\\vdots\\x_n}$ et $Y = \mat{y_1\\ \vdots \\ y_n}$. Alors, \[
		\left<x \mid y \right> = X^\T\,Y
	.\]
\end{thm}

\begin{prv}
	\begin{align*}
		\left<x \mid y \right> &= \left<\sum_{i=1}^n x_ie_i  \mid y \right>\\
		&= \sum_{i=1}^n x_i \left<e_i  \mid y \right> \\
		&= \sum_{i=1}^n x_i \left<e_i  \mid \sum_{j=1}^n y_j e_j \right> \\
		&= \sum_{i=1}^n x_i \sum_{j=1}^n y_j \underbrace{\left<e_i \mid e_j \right>}_{\delta_i^j} \\
		&= \sum_{i=1}^n x_i y_i. \\
	\end{align*}
\end{prv}

\begin{prop}
	Soit $E$ un espace euclidien et $\mathcal{B} = (e_1, \ldots, e_n)$ une base orthonormée de $E$. Alors, \[
		\forall x \in E,\,x = \sum_{i=1}^n \left<x \mid e_i \right>e_i
	.\]
\end{prop}

\begin{prv}
	Soit $x \in E$. On pose \[
		x = \sum_{i=1}^n x_i e_i
	\] avec $(x_1, \ldots, x_n) \in \R^n$. Soit $j \in \left\llbracket 1,n \right\rrbracket$. On a
	\begin{align*}
		\left<x \mid e_j \right> &= \left<\sum_{i=1}^n x_i e_i  \mid e_j \right>\\
		&= \sum_{i=1}^n x_i \left<e_i \mid e_j \right> \\
		&= x_j. \\
	\end{align*}
\end{prv}

	\part{Lois de composition}

\begin{defn}
	Une \underline{loi de composition interne} \index{loi de composition interne} est une application $f$ de $E \times E$ dans $E$.
	
	On la note $x * y$ au lieu de $f(x,y)$ (on est libre de choisir le symbôle).
\end{defn}

\begin{defn}
	Soit $E$ un ensemble muni d'une loi de composition interne $\boxtimes$.

	On dit que $\boxtimes$ est \underline{associative} \index{associativité (loi de composition interne)} si \[
		\forall (x,y,z) \in E^3,\;(x\boxtimes y)\boxtimes z = x \boxtimes (y \boxtimes z).
	\] Dans ce cas, on écrit plutôt $x \boxtimes y \boxtimes z$.
\end{defn}

\begin{exm}
	\begin{itemize}
		\item $+$ et $\times $ dans $\C$ sont associatives;
		\item $ \circ$ est associative;
		\item  la multiplication matricielle est aussi associative.
	\end{itemize}
\end{exm}

\begin{defn}
	On dit que $\boxtimes$ est \underline{commutative} \index{commutativité (loi de composition interne)} si \[
		\forall (x,y) \in E^2, x\boxtimes y = y\boxtimes x.
	\]
\end{defn}

\begin{exm}
	\begin{itemize}
		\item $+$ et $\times $ dans $\C$ sont commuatives;
		\item $ \circ $ n'est pas commutative;
		\item  la multiplication matricielle n'est pas commutative.
	\end{itemize}
\end{exm}

\begin{defn}
	Soit $e \in E$. On dit que $e$ est un
	\begin{itemize}
		\item \underline{élément neutre à gauche}\index{élément neutre à gauche (loi de composition interne)} si  \[
				\forall x \in E,\; e\boxtimes x = x;
			\]
		\item \underline{élément neutre à droite}\index{élément neutre à droite (loi de composition interne)} si  \[
				\forall x \in E,\; x\boxtimes e = x;
			\]
		\item \underline{élément neutre}\index{élément neutre (loi de composition interne)} si  \[
				\forall x \in E,\; e\boxtimes x = x\boxtimes e = x.
			\]
	\end{itemize}
\end{defn}

\begin{prop}
	Sous reserve d'existence, il y a unicité de l'élément neutre.
\end{prop}

\begin{prv}
	Soient $e$ et $e'$ deux éléments neutre.
	\begin{itemize}
		\item $e \boxtimes e' = e'$ car $e$ est neutre,
		\item $e \boxtimes e' = e$ car $e'$ est neutre.
	\end{itemize} On a donc $e = e'$.
\end{prv}

\begin{axm}[axiome du choix]
	Soit $E$ un ensemble non vide. Il existe $f : \mathcal{P}(E) \setminus \{\O\} \to E$ telle que \[
		\forall A \in \mathcal{P}(E) \setminus \{\O\},\; f(A) \in A.
	\]
\end{axm}

\begin{defn}
	Soit $f: E \to F$. Le \underline{graphe} \index{graphe (application)} de $f$ est \[
		\Big\{\big(x,f(x)\big)  \mid x \in E\Big\} \subset E \times F.
	\]
\end{defn}

\begin{prop}
	Soit $G \subset E\times F$. $G$ est le graphe d'une application si et seulement si \[
		\forall x \in E,\,\exists! y \in F,\, (x,y) \in G.
	\]
\end{prop}

\begin{prv}
	\begin{itemize}
		\item[``$\implies$''] par définition d'une application
		\item[``$\impliedby$''] On pose $f(x)$ le seul élément $y$ de $F$ qui vérifie $(x,y) \in G$. Alors $f \in F^E$ et son graphe vaut $G$.
	\end{itemize}
\end{prv}

\begin{defn}
	Soit $A \in \mathcal{P}(E)$. L'\underline{indicatrice}\index{indicatrice (ensemble)} de $A$ est \begin{align*}
		\mathbbm{1}_A: E &\longrightarrow \{0,1\} \\
		x &\longmapsto \begin{cases}
			1 &\text{ si } x \in A,\\
			0 & \text{ si } x \not\in A.
		\end{cases}
	\end{align*}
\end{defn}

\begin{exm}
	\begin{enumerate}
		\item Dans $\C$, le neutre de $+$ est $0$ et le neutre de $\times$ est $1$.
		\item Dans $E^E$, le neutre de $ \circ $ est $\id_E$.
		\item Dans $\mathcal{M}_n(\C)$ (l'ensemble des matrices carrées $n \times n$ à valeurs dans $\C$), le neutre de $\times $ est $I_n$ : \[
				I_n =
				\begin{pNiceMatrix}
					1&&(0)\\
					&\Ddots&\\
					(0)&&1
				\end{pNiceMatrix}
			\] 
	\end{enumerate}
\end{exm}

\begin{defn}
	Soit $E$ un ensemble muni d'une loi de composition interne $\boxtimes$ et $x \in E$.

	\begin{enumerate}
		\item On dit que $x$ est \underline{simplifiable à gauche}\index{simplifiabilité à gauche} si \[
				\forall (y,z) \in E^2,\,(x\boxtimes y = x \boxtimes z) \implies x = z.
			\] et que $x$ est \underline{simplifiable à droite}\index{simplifiabilité à droite} si \[
				\forall (y,z) \in E^2,\,(y\boxtimes x = z \boxtimes y) \implies x = z.
			\]
		\item On dit que $x$ est \underline{symétrisable à gauche}\index{symétrisabilité à gauche} s'il exiiste $y \in E$ tel que $y\boxtimes x = e$ où $e$ est l'élément neutre de $\boxtimes$.

			De même, on dit que $x$ est \underline{symétrisable à droite}\index{symétrisabilité à droite} s'il existe $y \in E$ tel que $x \boxtimes y = e$.

			On dit que $x$ est \underline{symétrisable}\index{symétrisabilité} s'il est symétrisable à gauche et à droite, donc s'il existe $y \in E$ tel que $x \boxtimes y = y \boxtimes x = e$.
	\end{enumerate}
\end{defn}

\begin{exm}
	$E = \N$ muni de la loi $+$, tous les éléments de $E$ sont simplifiables. $0$ est le seuele élément de $E$ symétrisable.
\end{exm}

\begin{prop}
	Avec les notations précédentes, si $\boxtimes$ est associative, et $x$ est symétrisable, alors $x$ est simplifiable.
\end{prop}

\begin{prv}
	Soient $y, z \in E$.
	\begin{itemize}
		\item On suppose $x \boxtimes y = x \boxtimes z$. Soit $a \in E$ tel que $a\in E$ tel que $a \boxtimes x = e$. Alors \[
				a \boxtimes (x\boxtimes y) = a \boxtimes (x \boxtimes z).
			\] Or,
			\begin{align*}
				a \boxtimes (x \boxtimes y) &= (a \boxtimes x) \boxtimes y \\
				&= e \boxtimes y \\
				&= y. \\
			\end{align*}

			De même, $a \boxtimes (x \boxtimes z) = z$.

			Donc $y = z$.
		\item De même, si $y \boxtimes x = z \boxtimes x$, on ``multiplie'' $x$ à droite par $a$ et on obtient $y = z$.
	\end{itemize}
\end{prv}

\begin{prop-defn}
	On suppose $\boxtimes$ associative. Soit $x \in E$ symétrisable. Alors \[
		\exists ! y \in E,\; x \boxtimes y = y \boxtimes x = e.
	\] On dit que $y$ est le \underline{symétrique}\index{symétrique (loi de composition interne)} de $x$ et on le note $y = x^*$.
\end{prop-defn}

\begin{prv}
	Soeint $x,y,z \in E$ tels que \[
		\begin{cases}
			 x \boxtimes y = y \boxtimes x = e\\
			 x \boxtimes z = z \boxtimes x = e\\
		\end{cases}
	\] Alors, $x \boxtimes y = x \boxtimes z$ et, en simplifiant par $x$, on a $y = z$.
\end{prv}

\begin{exm}
	Les fonctions symétrisables de $(E^E,  \circ)$ sont les bijections et le symétrique d'une bijection est sa réciproque.
\end{exm}

\begin{rmk}
	\begin{enumerate}
		\item Si la loi est notée $+$, on parle d'\underline{opposé}\index{opposé (loi de composition interne)} plutôt que de symétrique et on le note $-x$ au lieu de $x^*$.
			L'élément neutre est noté $0_E$.
		\item Si la loi est notée $\times$, on parle d'élément \underline{inversible}\index{inversibilité (loi de composition interne)} au lieu de symétrisable, d'\underline{inverse}\index{inverse (loi de composition interne)} au lieu de symétrique et on note $x^{-1}$ au lieu de $x^*$. On note le neutre $1_E$.
	\end{enumerate}
\end{rmk}

\begin{exo}
	Soient $x,y \in E = \R^+_*$. On définit la loi de composition interne $\oplus$ : \[
		x \oplus y = \frac{1}{\frac{1}{x}\oplus \frac{1}{y}}.
	\] Cette loi peut-être utile en physique pour le calcul de résistances équivalentes en parallèles.
	\begin{itemize}
		\item {\sc Associativité} : soient $x,y,z \in E$.

			D'une part, on a \[
				x \oplus (y \oplus z) = \frac{1}{\frac{1}{x} + \frac{1}{\frac{1}{\frac{1}{x}+ \frac{1}{y}}}} = \frac{1}{\frac{1}{x}+\frac{1}{y}+\frac{1}{z}}.
			\] D'autre part, on a \[
			(x \oplus y) \oplus z = \frac{1}{\frac{1}{\frac{1}{\frac{1}{x}+\frac{1}{y}}}+\frac{1}{z}} = \frac{1}{\frac{1}{x}+ \frac{1}{y}+\frac{1}{z}}.
			\] La loi $\oplus$ est associative.
		\item {\sc Commutativité} : soient $x, y \in E$. \[
				x \oplus y = \frac{1}{\frac{1}{x}+\frac{1}{y}} = \frac{1}{\frac{1}{y}+\frac{1}{x}} = y\oplus x.
			\] Donc la loi $\oplus$ est commutative.
		\item {\sc Élément neutre} : soit $e$ l'élément neutre de $\oplus$. \[
				\forall x \in E,\; x \oplus e = e \oplus x = x.
			\] Comme la loi est commutative, seul l'égalité $x \oplus e = x$ est utile.

			Soit $x \in E$. On a donc $\frac{1}{\frac{1}{x}+\frac{1}{e}}=x$ donc $\frac{ex}{e+x}=x$ donc $ex = x(e+x)$ et donc $\cancel{ex} = \cancel{ex} + x^2$. On en déduit que $x^2 = 0$, ce qui n'est pas possible car $x \in \R^+_*$. Donc, il n'y a pas d'élément neutre pour $\oplus$.
	\end{itemize}
\end{exo}

	\part{Divers}

\begin{defn}
	Soient $E$ et $F$ deux ensembles. Un \underline{couple}\index{couple} $(x,y)$ est la donnée d'un élément $x$ de $E$ et d'un élément $y$ de $F$ où \[
		\forall x,x' \in E,\,\forall y,y' \in F,\qquad (x,y) = (x',y') \iff \begin{cases}
			x=x',\\
			y=y'.
		\end{cases}
	\] On note $E \times F$ l'ensemble des couples; c'est le \underline{produit cartésien}\index{produit cartésion (ensembles)} de $E$ et $F$.
\end{defn}

\begin{exm}
	$D \times [0,1]$ est un cylindre plein où $D$ est le disque unité fermé i.e. \[
		D = \Big\{(x,y) \in \R^2 \mid x^2+y^2 \le 1\Big\}.
	\]
	\begin{figure}[H]
		\centering
		\begin{subfigure}[b]{3cm}
			\centering
			\begin{asy}
				size(3cm);
				draw(unitcircle);
				draw((0,0)--(1,0), red);
				label("$1$",(0.5,0), red, align=S);
			\end{asy}
		\end{subfigure}
		\begin{subfigure}[b]{3cm}
			\centering
			\begin{asy}
				size(3cm);
				label("$\times\; [0,1]\; =$", (0,0), fontsize(10));
				draw(unitcircle, white+0);
			\end{asy}
		\end{subfigure}
		\begin{subfigure}[b]{3cm}
			\centering
			\begin{asy}
				import solids;
				size(3cm);
				draw(shift((0, 0.5)) * unitcircle, white+0);
				revolution r = cylinder(O, 1, 1.5, Z);
				draw(r);
				triple M = (-1/2, sqrt(3)/2, 0);
				draw((0,0,0) -- M, red);
				label("$1$", M/2, red, align=S);
				draw(M*1.1--M*1.1+(0,0,1.5), magenta, Arrows3(TeXHead2));
				label("$1$", M*1.1+(0,0,0.75), magenta, align=E);
			\end{asy}
		\end{subfigure}
	\end{figure}

	$C \times C$ où $C = \Big\{(x,y) \in \R^2  \mid x^2 + y^2 = 1\Big\}$ est un tore (creu).

	\begin{figure}[H]
		\centering
		\begin{subfigure}[b]{3cm}
			\centering
			\begin{asy}
				size(3cm);
				draw(unitcircle);
				draw((0,0)--(1,0), red);
				label("$1$",(0.5,0), red, align=S);
			\end{asy}
		\end{subfigure}
		\begin{subfigure}[b]{1cm}
			\centering
			\begin{asy}
				size(3cm);
				label("$\times$", (0,0), fontsize(10));
				dot((0.1, 1), white+0);
				dot((-0.1, -1), white+0);
			\end{asy}
		\end{subfigure}
		\begin{subfigure}[b]{3cm}
			\centering
			\begin{asy}
				size(3cm);
				draw(unitcircle);
				draw((0,0)--(1,0), red);
				label("$1$",(0.5,0), red, align=S);
			\end{asy}
		\end{subfigure}
		\begin{subfigure}[b]{1cm}
			\centering
			\begin{asy}
				size(3cm);
				label("$=$", (0,0), fontsize(10));
				dot((0.1, 1), white+0);
				dot((-0.1, -1), white+0);
			\end{asy}
		\end{subfigure}
		\begin{subfigure}[b]{3cm}
			\centering
			\begin{asy}
				import three;
				import graph3;

				size(3cm,3cm);
				surface torus = surface(Circle(c=2Y,normal=X,r=0.5,n=32), c=O, axis=Z, n=32);

				draw(torus, white + opacity(0), meshpen=black + 0.2pt, nolight, render(merge=true));
			\end{asy}
			\vspace{0.7cm}
		\end{subfigure}
	\end{figure}
\end{exm}

\begin{defn}
	Soient $E$ et $F$ deux ensembles. On dit que $E$ et $F$ sont \underline{équipotents} s'il existe une bijection de $E$ dans $F$.
	\index{équipotence (ensembles)}
\end{defn}

\begin{exm}
	\begin{enumerate}
		\item $\N$ et $\N^*$ sont équipotents car  $f : \begin{array}{rcl}
				\N &\longrightarrow& \N^* \\
				k &\longmapsto& k + 1
			\end{array}$ est bijective.
		\item $P = \{n \in \N  \mid n \text{ pair}\}$ et $I= \{n \in \N \mid n \text{ impair}\}$ sont équipotents car $f : \begin{array}{rcl}
				P &\longrightarrow& I \\
				x &\longmapsto& x+1
			\end{array}$ est bijective.
		\item $\N$ et $P$ sont équipotents car $f : \begin{array}{rcl}
				\N &\longrightarrow& P \\
				k &\longmapsto& 2k
			\end{array}$ est bijective.
		\item $[0,1]$ et $[0,1[$ sont équipotents car \begin{align*}
			f: [0,1] &\longrightarrow [0,1[ \\
			x &\longmapsto \begin{cases}
				\frac{1}{n+1} &\text{ si } x = \frac{1}{n} \text{ avec } n \in \N^*\\
				x &\text{ sinon}
			\end{cases}
		\end{align*} est bijective.
		\item De même, $]0,1[$ et $]0,1]$ sont équipotents.
		\item $]0,1[$ et $[0,1[$ sont équipotents : $f : \begin{array}{rcl}
					]0,1] &\longrightarrow& [0,1[ \\
				x &\longmapsto& 1-x
			\end{array}$ est bijective.
		\item $\forall a < b$, $[a,b]$ et $[0,1]$ sont équipotents : \begin{align*}
				f: [0,1] &\longrightarrow [a,b] \\
				\alpha &\longmapsto \alpha b + (1 - \alpha) a
			\end{align*} est bijective (interpolation linéaire).
		\item $\R$ et $]0,1[$ sont équipotents : \begin{align*}
				f: \R &\longrightarrow ]0,1[ \\
				x &\longmapsto \frac{1}{2} + \frac{\Arctan x}{\pi}
			\end{align*} est bijective.
		\item $[0,1[$ et $\N$ ne sont pas équipotents (argument de Cantor). Soit $f: \N \to [0,1[$ une bijection :
			\[
				\begin{array}{c|l}
					k&\hfill f(k)\hfill~ \\ \hline
					0&0,\hfill \!0\hfill 0\hfill 0\hfill 0\hfill\ldots\\
					1&0,\hfill a_1\hfill a_2\hfill a_3\hfill a_4\hfill\ldots\\
					2&0,\hfill b_1\hfill b_2\hfill b_3\hfill b_4\hfill\ldots\\
					\vdots&\hfill\vdots\hfill\ddots
				\end{array}
			\] On considère le nombre \[
				x = 0,\,(a_0+1)(b_1+1)(c_2+1)\cdots
			\] $f(1) \neq x$ car ils n'ont pas le même chiffre des dizaines.\\
			$f(2) \neq x$ car ils n'ont pas le même chiffre des centaines.

			Par le même raisonement, on en déduit que \[
				\forall n \in \N, f(n) \neq x
			\] donc $x$ n'a pas d'antécédant : une contradiction.
		\item On verra en exercice que $E$ et $\mathcal{P}(E)$ ne sont pas équipotents. $\R$ et $\mathcal{P}(\R)$ ne sont pas équipotents mais $\R$ et $\mathcal{P}(\N)$ le sont (développement dyadique).
		\item $\R^2$ et $\R$ sont équipotents; $\C$ et $\R$ sont équipotents.
	\end{enumerate}
\end{exm}

\begin{exo}
	Soit $E$ un ensemble. L'application \begin{align*}
		f: \mathcal{P}(E) &\longrightarrow {0,1}^E \\
		A &\longmapsto \mathbbm{1}_A
	\end{align*} est bijective.

	Soit $g : E \to \{0,1\}$.
	\begin{itemize}
		\item[\underline{\sc Analyse}] Soit $A \in \mathcal{P}(E)$ tel que $f(A) = g$. Alors $g = \mathbbm{1}_A$.
			donc  \[
				\forall x \in E,\; g(x) = \mathbbm{1}_A(x)
			\] et donc \[
				\begin{cases}
					\forall x \in A,\, g(x) = 1\\
					\forall x \in E \setminus A,\,g(x) = 0
				\end{cases}
			\] On en déduit que \[
				A = \{ x \in E  \mid  g(x) = 1\}  = g^{-1}\big(\{1\}\big).
			\]
		\item[\underline{\sc Synthèse}] On pose $A = g^{-1}\big(\{1\}\big)$. Montrons que $f(A) = g$.
			\[
				\forall x \in E,\,g(x) = \begin{cases}
					1 &\text{ si } x \in A\\
					0 &\text{ si } x \not\in A
				\end{cases} = \mathbbm{1}_A
			\] donc $g = \mathbbm{1}_A$.
	\end{itemize}

	On aurait aussi pu rédiger de la fa\c con suivante : on pose \begin{align*}
		u: \{0,1\}^E &\longrightarrow \mathcal{P}(E) \\
		g &\longmapsto g^{-1}\big(\{1\}\big).
	\end{align*} On montre que $u$ est la réciproque de $f$ : \[
		\begin{cases}
			f \circ u = \id_{\{0,1\}^E},\\
			u \circ f = \id_{\mathcal{P}(E)}.
		\end{cases}
	\]
\end{exo}

\begin{defn}
	Soit $f : E \to F$. L'\underline{image de $f$}\index{image (application)} est \[
		\mathrm{Im}(f) = f(E) = \big\{f(x) \mid x \in E\big\}.
	\]
\end{defn}

\begin{prop}
	Soit $f: E \to F$. \[
		f \text{ est surjective } \iff f(E) = F.
	\]
\end{prop}

\begin{defn}
	Une \underline{suite de $E$}\index{suite (ensemble)} est une application de $\N$ dans $E$.
\end{defn}

\begin{rmk}[Notation]
	Soit $u \in E^\N$. Pour $n \in \N$, on écrit $u_n$ à la place de $u(n)$.
\end{rmk}

\begin{defn}
	Soient $E$ et $I$ deux ensembles. Une \underline{famille de $E$ indéxée par $I$}\index{famille (ensemble)} est une application de $I$ dans $E$.

	À la place de $u(i)$ (avec $i \in I$), on écrit $u_i$.
\end{defn}

\begin{defn}
	Soit $E$ un ensemble et $(A_i)_{i \in I}$ une famille de parties de $E$. On suppose $I \neq \O$. On pose \[
		\bigcup_{i \in  I} A_i = \{x \in E  \mid \exists i \in I,\, x \in A_i\}
	\] et \[
		\bigcap_{i \in  I} A_i = \{x \in E  \mid \forall i \in I,\, x \in A_i\}.
	\] On pose aussi $\bigcup_{i \in \O} A_i = \O$ et $\bigcap_{i \in \O}  A_i = E$.
\end{defn}

\begin{rmk}
	De même que pour les sommes et produits de complexes, on peut intervertir des réunions doubles.
\end{rmk}

\begin{prop}
	Soit $E$ un ensemble, $(A,B) \in \mathcal{P}(E)^2$. \[
		A \subset (E \setminus B) \iff A \cap B = \O.
	\]
\end{prop}

\begin{figure}[H]
	\centering
	\begin{asy}
		import patterns;
		add("hatch",hatch(1mm, deepcyan));
		add("hatch2",hatch(1mm, heavygreen));
		size(3cm);

		guide main_set = scale(1.3) * ((-1,1)..(-0.8,-0.8)..(0,-0.9)..(0.7,-1.2)..(0.8, 0.9)..cycle);
		guide set_a = shift((-0.5, -0.2)) * ((-0.6, 0.6)..(0.2,-0.2)..(0.2,-0.4)..(-0.6,-0.2)..cycle);
		guide set_b = shift((0.3, 0.4)) * ((0.8, -0.6)..(1.1,-0.2)..(0.2,0.5)..(0.2,-0.8)..cycle);

		draw(main_set, magenta); label("$E$", 1.3*(0.8,0.9),magenta, align=NE);
		draw(set_a, deepcyan); label("$A$", (-0.6,0.6), deepcyan, align=NW);
		draw(set_b, heavygreen); label("$B$", (0.8,-0.6), heavygreen, align=SE);

		fill(set_a, pattern("hatch"));
		fill(set_b, pattern("hatch2"));
	\end{asy}
\end{figure}

\begin{prv}
	\begin{itemize}
		\item[``$\implies$''] Soit $x \in A \cap B$. Alors $x \in A$ et $x \in B$. Comme $x \in A \subset (E \setminus B)$, alors $x \in E \setminus B$ i.e. $x \not\in B$ : une contradiction. Donc $A \cap B = \O$.
		\item[``$\impliedby$''] On suppose $A \cap B = \O$. Soit $x \in A$. Si $x \in B$, alors $x \in A \cap B = \O$ : faux.
			Donc $x \not\in B$ et donc $x \in E \setminus B$.
	\end{itemize}
\end{prv}

\begin{prop}
	Si $f: E\to F$ et $g: F \to G$ sont bijectives, alors $g \circ f$ est bijective et \[
		(g \circ f)^{-1} = f^{-1} \circ g^{-1}.
	\] \qed
\end{prop}

\begin{rmk}[\danger Attention]
	$g \circ f$ peut-être bijective alors que $f$ et $g$ ne le sont pas.
\end{rmk}


	\part{Rappels sur $\ln$ et $\exp$}

\begin{prop}~\\
	\begin{itemize}
		\item Soit $(a_i)_{i\in I}$ une famille finie de réels strictement positifs. Alors,
			\[
				\ln\left( \prod_{i \in I} a_i \right) = \sum_{i \in I} \ln a_i.
			\]
		\item Soit $(b_i)_{i\in I}$ une famille de réels. Alors \[
			\exp\left( \sum_{i \in I} b_i \right) = \prod_{i \in I} \exp(b_i).
		\]
	\end{itemize}
\end{prop}

\begin{rmk}
	Soit $f: I \to \R^*$ dérivable. On pose $g: x \mapsto \ln \left| f(x) \right|$.

	Alors $g$ est dérivable sur $I$ et \[
		\forall x \in I, g'(x) = \frac{f'(x)}{f(x)}
	\]

	On dit que $\frac{f'}{f}$ est la \underline{dérivée logarithmique} de $f$.

	Soient $f_1,f_2: I\to \R^*$ dérivables. Alors \[
		\frac{(f_1\,f_2)'}{f_1\,f_2} = \frac{f_1'}{f_1} + \frac{f_2'}{f_2}.
	\]
\end{rmk}

\begin{rmk}
	Soit $a \in \R$.
	\begin{itemize}
		\item Soit $n \in \N^*$. Alors, $a^n = \overbrace{a\times a\times a\times \cdots \times a}^{n \text{ fois }}$.
		\item Soit $n \in \Z^-_*$. Si $a \neq 0$, alors $a^n = \frac{1}{a^{-n}}$.
		\item Si $a \neq 0$, $a^{0} = 1$ et \[
				\forall p,q \in \Z, a^{p}\times a^{q} = a^{p+q}.
			\]
		\item Soit $p \in \Z$ et $a > 0$. \[
			a^{p} = \exp(\ln a^{p}) = \exp(p \ln a) = e^{p \ln a}.
		\]
	\end{itemize}
\end{rmk}

\begin{defn}
	Soit $a \in \R^+_*$ et $p \in \R$. On pose $a^p = e^{p \ln a}$.
\end{defn}

	\part{Covariance ({\sc Hors-Programme})}

On se place dans une optique de Big Data.
On dispose d'un tableau à $N$ lignes : chaque ligne correspond à une observation et chaque colonne à une ``mesure'' (ou caractéristique).

Ces caractéristiques peuvent être corrélées plus ou moins fortement et contenir plus ou moins d'information.

Plus la variance est grande, plus il y a d'information.

Soient $X$ et $Y$ deux colonnes. D'après l'inégalité de Cauchy-Schwarz : \[
	\big|\Cov(X,Y)\big| \le \sigma_X\:\sigma_Y \quad \text{ donc } \quad -1 \le \frac{\Cov(X,Y)}{\sigma_X\:\sigma_Y} \le 1
.\]

Si $Y = \alpha X + \beta$ : \hfill$\left| \frac{\Cov(X,Y)}{\sigma_X,\sigma_Y} \right| = 1$.\hfill~\hfill~

L'objectif est de modifier les colonnes de fa\c con à extraire le plus d'informations possible sur le moins de colonnes possibles.

La \underline{matrice de covariance}\index{matrice de covariance} est \[
	A = \begin{pmatrix}
		\Cov(X_i,X_j)
	\end{pmatrix}_{\substack{1\le i\le N\\1\le j\le N}}
.\]

On aimerait que la matrice $A$ soit diagonale avec de grands coefficients diagonaux.

L'année prochaine, nous verrons le théorème suivant :
\begin{thm}[théorème spectral]
	Toute matrice symétrique réelle est diagonalisable dans une base orthonormée de vecteurs propres.
\end{thm}

Il existe donc des variables aléatoires $Y_1, \ldots, Y_k$ combinaisons linéaires des $X_1, \ldots, X_k$ telles que \[
	\begin{pmatrix}
		\Cov(Y_i,Y_j)
	\end{pmatrix} =
	\begin{pNiceMatrix}
		\lambda_1 &&\Block{3-2}{(0)}&\\
		&\Ddots&&\\
		\Block{2-3}{(0)}\\
		&&&\lambda_k
	\end{pNiceMatrix}
.\]

De plus, le maximum de $V(\alpha_1 X_1 + \cdots + \alpha_k X_k)$ avec la condition que $\alpha_1^2 + \cdots + \alpha_k^2 = 1$ est $V(Y_1) = \lambda_1$.

\begin{defn}[Multiplicateurs de Lagrange]
	Soit $f : U \subset \R^n \longrightarrow \R$ où $U$ est un ouvert de $\R^n$.
	On cherche $\max_{(x_i)_{i \in \left\llbracket 1,n \right\rrbracket} \in R^n} f(x_1, \ldots, x_n)$ avec la contrainte $g(x_1, \ldots, x_n)$.
	
	On pose \begin{align*}
		F: \R^{n+1} &\longrightarrow \R \\
		(x_1, \ldots, x_n, \lambda) &\longmapsto f(x_1, \ldots, x_n) + \lambda g(x_1, \ldots, x_n).
	\end{align*}

	On cherche les points critiques de $F$ :
	\begin{align*}
		\nabla F(x_1, \ldots, x_n, \lambda) = 0 \iff& \begin{cases}
			\forall i,\,\frac{\partial F}{\partial x_i}(x_1, \ldots, x_n, \lambda) = 0\\
			\frac{\partial F}{\partial \lambda}(x_1, \ldots, x_n, \lambda) = 0
		\end{cases}\\
		\iff& \begin{cases}
			\forall i,\,\frac{\partial f}{\partial x_i}(x_1, \ldots,x_n) + \lambda \frac{\partial g}{\partial x_i}(x_1, \ldots, x_n) = 0\\
			g(x_1, \ldots, x_n) = 0
		\end{cases}
	\end{align*}

	Si $(x_1, \ldots, x_n, \lambda)$ est le maximum de $F$, alors\\
	$\forall y_1, \ldots,y_n, \mu \in \R^{n+1}$ \hfill $F(x_1, \ldots, x_n, \lambda) \ge F(y_1, \ldots, y_n, \mu)$ \hfill~\\
	\phantom{e}~\hfill $f(x_1, \ldots, x_n) + \underbrace{\lambda g(x_1, \ldots, x_n)}_{=0} \ge f(y_1, \ldots, y_n)$ \hfill ~
\end{defn}




	\chap[12]{Structures algébriques usuelles}
	\renewcommand{\cwd}{../chap12}
	\part{Topologie de $\R^2$}

\begin{defn}
	La \underline{norme (euclidienne)} de $\R^2$ est l'application définie par \[
		\forall (x,y) \in \R^2, \|(x,y)\| = \sqrt{x^2 + y^2}.
	\]

	\begin{figure}[H]
		\centering
		\begin{asy}
			import graph;
			axes(EndArrow);
			size(4cm);
			pair A = (3,2);
			dot(A);
			draw((3,0)--A, dashed);
			draw((0,2)--A, dashed);
			label("$x$", (3,0), align=S);
			label("$y$", (0,2), align=W);
			draw((0,0)--A);
			dot((4,3), white+0);
		\end{asy}
	\end{figure}
	\index{norme (de $\R^2$)}
	\index{norme euclidienne (de $\R^2$)}
\end{defn}

\begin{prop}
	La norme euclidienne vérifie:
	\begin{enumerate}
		\item (séparation) \[
			\forall (x,y) \in \R^2, \|(x,y)\| = 0 \iff x = y = 0,
			\]
		\item (homogénéité positive) \[
				\forall \lambda \in \R, \forall (x,y) \in \R^2, \|\lambda(x,y)\|= \left| \lambda \right| \|(x,y)\|
			\]
		\item (inégalité triangulaire) \[
			\forall (x,y), (a,b) \in \R^2,
			\|(x,y)+(a,b)\|\le \|(x,y)\|+\|(a,b)\|.
		\]
	\end{enumerate}
\end{prop}

\begin{prv}
	Déjà vue en replaçant $(x,y)$ par $x+iy \in \C$ et $\|(x,y)\|$ par |x+iy|
\end{prv}

\begin{defn}
	Soit $(a,b) \in \R^2$ et $r \in \R_+$.

	La \underline{boule ouverte} (ou \underline{disque ouvert}) de centre $(a,b)$ et de rayon $r$ est \[
		B_{(a,b)}(r) = \big\{ (x,y) \in \R^2  \mid \|(x,y) - (a,b)\| < r \big\}.
	\]

	La \underline{boule fermée} (ou \underline{disque fermé}) de centre $(a,b)$ et de rayon $r$ est \[
		\overline{B_{(a,b)}}(r) = \big\{ (x,y)\in \R^2  \mid \|(x,y) - (a,b)\| \le r \big\}.
	\]

	La \underline{sphère} (ou \underline{boule}) de centre $(a,b)$ et de rayon $r$ est \[
		S_{(a,b)}(r) = \partial \overline{B_{(a,b)}}(r) = \big\{ (x,y) \in \R^2  \mid \|(x,y) - (a,b)\| = r \big\}.
	\]
	\index{boule ouverte (de $\R^2$)}
	\index{disque ouverte (de $\R^2$)}
	\index{boule fermée (de $\R^2$)}
	\index{disque fermée (de $\R^2$)}
	\index{boule (de $\R^2$)}
	\index{sphère (de $\R^2$)}
\end{defn}

\begin{figure}[H]
		\centering
		\incfig{boule}
\end{figure}

\begin{rmk}
	On parle de boule en dimension quelconque.
\end{rmk}

\begin{defn}
	Une \underline{partie ouverte} $O$ de $\R^2$ (ou \underline{un ouvert}) si \[
		\forall (x,y) \in O, \exists r > 0, B_{(a,b)}(r) \subset O.
	\]
	Une partie $F$ est \underline{fermée} su $\R^2\setminus F$ est ouverte.
	\index{partie ouverte (de $\R^2$)}
	\index{ouvert (de $\R^2$)}
	\index{partie fermée (de $\R^2$)}
\end{defn}

\begin{figure}[H]
	\centering
	\incfig{partie-ouverte}
\end{figure}

\begin{prop}
	Une boule ouverte est ouverte. Une boule fermée est fermée.
\end{prop}

\begin{figure}[H]
	\centering
	\begin{subfigure}{4cm}
		\centering
		\begin{asy}
			import patterns;

			pair n(pair a) {return a / length(a);}

			add("hatch",hatch(2mm, SW, red));
			size(4cm);

			draw(circle((0,0), 1));
			dot('$(a_0, b_0)$', (0,0), align=S);

			draw((0,0) -- n((-1, 1)), dashed);
			label("$r$", n((-1, 1)) / 2, align=1.5S);

			pair A = n((1,3)) * (2/3);
			real rho = (1 - length(A)) * (2 / 3);

			dot("$(a,b)$", A, red, align=3SE);
			filldraw(circle(A, rho), pattern("hatch"), red);

			label("$O$", n((1,-1))*2.5/3);
		\end{asy}
	\end{subfigure}
	\begin{subfigure}{1cm}
		\centering~\\
	\end{subfigure}
	\begin{subfigure}{5cm}
		\centering
		\begin{asy}
			import patterns;

			pair n(pair a) {return a / length(a);}

			add("hatch",hatch(1mm, SW, red));
			add("hatch2",hatch(3mm, SE, blue));
			size(5cm);

			guide around = (-1.5, -1.5) -- (-1.5, 1.5) -- (2.5, 1.5) -- (2.5, -1.5) -- cycle;

			pair A = n((3, 1)) * 5/3; 
			real rho = 2 / 9;

			picture inter;
			fill(inter, around, pattern("hatch2"));
			fill(inter, circle((0,0), 1), white);
			add(inter);

			draw(circle((0,0), 1));
			dot('$(a_0, b_0)$', (0,0), align=S);

			draw((0,0) -- n((-1, 1)), dashed);
			label("$r$", n((-1, 1)) / 2, align=1.5S);

			dot("$(a,b)$", A, red, align=2SE);
			filldraw(circle(A, rho), pattern("hatch"), red);

			label("$F$", n((1,-1))*2.5/3);
		\end{asy}
	\end{subfigure}
\end{figure}

\begin{prv}
	$\O$ est un ouvert.

	Soit $B$ la boule ouverte de centre $(a_0, b_0) \in \R^2$ et de rayon $r > 0$.

	On pose $\rho = \frac{1}{2}\big(r - \|(a,b) - (a_0,b_0)\|\big)$.
	Montrons que \[
		B_{(a,b)}(\rho) \subset  B_{(a,b)}(r).
	\]

	Soit $(x,y) \in B_{(a,b)}(\rho)$.
	\begin{align*}
		\|(x,y) - (a_0,b_0)\|&= \|(x,y)- (a,b) + (a,b) - (a_0,b_0)\| \\
		&\le \|(x,y) - (a,b)\| + \|(a,b) - (a_0, b_0)\|\\
		&< \rho + \|(a,b) - (a_0, b_0)\| = \frac{1}{2}r + \frac{1}{2} \|(a,b) - (a_0, b_0)\|\\
		&< r
	\end{align*}
	
	Soit $F$ la boule fermée de centre $(a_0, b_0)$ et de rayon $r \ge 0$.

	Soit $(a,b) \not\in F$. On pose \[
		\rho = \frac{1}{2}\big(\|(a,b) - (a_0, b_0)\| - r\big) > 0.
	\]

	Montrons que $B_{(a,b)}(\rho) \subset \R^2\setminus F$.

	Soit $(x,y) \in B_{(a,b)}(\rho)$.

	\begin{align*}
		\|(x,y) - (a_0, b_0)\| &= \|(x,y) - (a,b) + (a,b) - (a_0, b_0)\| \\
		&\ge \big| \underbrace{\|(x,y) - (a,b)\|}_{\le \rho} - \underbrace{\|(a,b) - (a_0, b_0)\|}_{> r} \big|\\
		&\ge \|(a,b) - (a_0, b_0)\|- \|(x,y) - (a,b)\|\\
		&> \|(a,b) - (a_0, b_0)\|- \rho\\
		&> \frac{1}{2} \|(a,b) - (a_0, b_0)\| + \frac{1}{2}r\\
		&> r
	\end{align*}

	donc $(x,y) \not\in F$.
\end{prv}

\begin{exm}
	\begin{enumerate}
		\item $\O$ est ouvert.\\
			$\R^2$ est ouvert.
		\item $\O$ est fermé.\\
			$\R^2$ est fermé.\\
		\item $\big\{(x,0)  \mid x > 0\big\}$ n'est ni ouverte ni fermé.
	\end{enumerate}
\end{exm}

\begin{figure}[H]
	\centering
	\begin{asy}
		size(3cm);

		draw((0, -1) -- (0, 3), Arrow(TeXHead));
		draw((-1, 0) -- (3, 0), Arrow(TeXHead));
		
		draw((0,0) -- (0, 2.97), red);
		draw(circle((0,1.5), 0.5), deepred);
		draw(circle((0,0.5), 0.1), deepred);
	\end{asy}
\end{figure}

\begin{defn}
	Soit $(a,b) \in \R^2$ et $V \in \mathcal{P}(\R^2)$.

	On dit que $V$ est un voisinage de $(a,b)$ s'il existe $r > 0$ tel que \[
		B_{(a,b)}(r) \subset V.
	\]
	\index{voisinage (dans $\R^2$)}
\end{defn}

\begin{prop}
	Un ouvert non vide est un voisinage en chacun de ces points. \qed
\end{prop}

\begin{defn}
	Soit $D \subset \R^2$. Un \underline{point intérieur} de $D$ est un couple $(a,b) \in D$ tel que \[
		\exists r > 0, B_{(a,b)}(r) \subset D.
	\] en d'autres termes, si $D$ est un voisinage de $(a,b)$.

	On note $\mathring D$ l'ensemble des points intérieurs à $D$. C'est \underline{l'intérieur} de $D$.
	\index{point intérieur (dans $\R^2$)}
	\index{intérieur (dans $\R^2$)}
\end{defn}

\begin{prop}
	$\mathring D$ est le plus grand ouvert $O$ de $\R^2$ tel que $O \subset D$.
\end{prop}

\begin{figure}[H]
	\centering
	\incfig{interieur-plus-grand-ouvert}
\end{figure}


\begin{prv}
	Soit $(a,b) \in \mathring D$.

	Par définition, il existe $r > 0$ tel que \[
		B_{(a,b)}(r) \subset D.
	\] Montrons que $B_{(a,b)}(r) \subset \mathring D$.

	Soit $(x,y) \in B_{(a,b)}(r)$. Comme $B_{(a,b)}(r)$ est un ouvert de $\R^2$, il existe $\rho > 0$ tel que \[
		B_{(x,y)}(\rho) \subset B_{(a,b)}(r)
	\] donc $(x,y) \in \mathring D$.

	Donc $\mathring D$ est ouvert, $\mathring D \subset D$.

	Soit $O$ un ouvert de $\R^2$ tel que $O \subset D$. Montrons que $O \subset \mathring D$.

	Soit $(x,y) \in O$. Soit $r > 0$ tel que \[
		B_{(x,y)}(r) \subset O \subset D
	\] donc $(x,y) \in \mathring D$.
\end{prv}

\begin{defn}
	Soit $f: D \subset \R^2 \to \R$, $\ell \in \R$, $(a,b) \in \mathring D$.

	On dit que \underline{$f(x,y)$ tend vers $\ell$ quand $(x,y)$ tend vers $(a,b)$} ou que $\ell$ est \underline{une limite} de $f$ en $(a,b)$ si \[
		\forall \varepsilon > 0, \exists r > 0, \forall (x,y) \in D, \|(x,y) - (a,b)\| < r \implies \left| f(x,y) - \ell \right| \le \varepsilon.
	\] en d'autres termes si \[
		\forall V \in \mathcal{V}_{\ell}, \exists W \in \mathcal{V}_{(a,b)}, \forall (x,y) \in W \cap D, f(x,y) \in V.
	\]
	\index{limite (dans $\R^2$)}
	\index{tendre vers (dans $\R^2$)}
\end{defn}

\begin{prop}
	[unicité de la limite]
	Soit $f: D \to \R$, $(a,b) \in \mathring D$, $\ell_1, \ell_2 \in \R$ telles que $\ell_1$ et $\ell_2$ sont des limites de $f$ en $(a,b)$.

	Alors $\ell_1 = \ell_2$.
\end{prop}

\begin{figure}[H]
	\centering
	\incfig{preuve-unicité-de-la-limite}
\end{figure}

\begin{prv}
	On suppose $\ell_1 < \ell_2$. On pose $\varepsilon = \frac{\ell_2 - \ell_1}{2} > 0$.

	Soit $r_1 > 0$ tel que \[
		f\big(B_{(a,b)}(r_1)\big) \subset ]\ell_1 - \varepsilon, \ell_1 + \varepsilon[.
	\] Soit $r_2 > 0$ tel que \[
		f\big(B_{(a,b)}(r_2)\big) \subset ]\ell_2 - \varepsilon, \ell_2 + \varepsilon [.
	\] On pose $r = \min(r_1, r_2)$ donc \[
		B_{(a,b)}(r_1) \cap B_{(a,b)}(r_2) = B_{(a,b)}(r) \neq \O.
	\] Soit $(x,y) \in B_{(a,b)}(r)$. Alors, \[
		f(x,y) \in ]\ell_1 - \varepsilon, \ell_1 + \varepsilon[ \cap ]\ell_2 - \varepsilon, \ell_2 + \varepsilon[ = \O.
	\] $\lightning$
\end{prv}

\begin{defn}
	Soit $f : D \to \R$, $(a,b) \in \mathring D$.

	On dit que $f$ est \underline{continue} en $(a,b)$ si \[
		f(x,y) \tendsto{(x,y) \to (a,b)}f(a,b).
	\]
	\index{continuité (dans $\R^2$)}
\end{defn}

\begin{prop}
	\underline{Si} $f(x,y) \tendsto{(x,y) \to (a,b)} \ell$ \\
	\underline{alors} $\begin{cases}
		f(x,b) \tendsto{x \to a} \ell\\
		f(a,y) \tendsto{y \to b} \ell.\\
	\end{cases}$
\end{prop}

\begin{prv}~\\
	\begin{figure}[H]
		\centering
		\incfig{limite-x-en-a-et-y-en-b}
	\end{figure}
\end{prv}

\underline{Contre-exemple} : exercice 3.

\begin{exm}
	\begin{enumerate}
		\item $f : \begin{array}{rcl}
				\R^2 &\longrightarrow& \R \\
				(x,y) &\longmapsto& x
			\end{array}$ limite en $(0,0)$ ?

			Soit $\varepsilon > 0$. On pose $r = \varepsilon$. \[
				\forall (x,y) \in B_{(0,0)}(r),
				\left| f(x,y) \right| = \left| x \right| \le \|(x,y)\| < r = \varepsilon
			\] Donc $f(x,y) \tendsto{(x,y) \to (a,b)} 0$.
		\item limite $f : \begin{array}{rcl}
				\R^2 &\longrightarrow& \R \\
				(x,y) &\longmapsto& x^3
			\end{array}$ en $(0,0)$ ?

			Soit $\varepsilon > 0$. On pose $r = \sqrt[3]{r} > 0$. \[
				\forall (x,y) \in B_{(0,0)}(r),
				\left| f(x,y) \right| = \left| x^3 \right| \le \|(x,y)\|^3 < r^3 = \varepsilon.
			\]
		\item limite de $f : \begin{array}{rcl}
			\R^2 &\longrightarrow& \R \\
			(x,y) &\longmapsto& x^3y^2
		\end{array}$ en $(0,0)$ ?

		Soit $\varepsilon > 0$. On pose $r = \sqrt[5]{\varepsilon} > 0$. \[
			\forall (x,y) \in B_{(0,0)}(r), \left| f(x,y) \right| = \left| x^3 y^2 \right| \le \|(x,y)\|^3 \|(x,y)\|^2 < r^5 = \varepsilon.
		\]
	\end{enumerate}
\end{exm}

\begin{defn}
	Soient $D \subset \R^2$ et $(x,y) \in \R^2$.

	\begin{figure}[H]
    \centering
    \incfig{point-adhérent}
	\end{figure}
	
	On dit que $(x,y)$ est \underline{adhérent} à $D$ si \[
		\forall r > 0, B_{(x,y)}(r) \cap D \neq \O.
	\] L'ensemble des points adhérents à $D$ est noté $\overline{D}$. On dit que $\overline{D}$ est \underline{l'adhérence} de $D$.
	\index{point adhérent (dans $\R^2$)}
	\index{adhérent (dans $\R^2$)}
\end{defn}

\begin{defn}
	Soit $f: D \subset \R^2 \to \R$ et $(a,b) \in \overline{D}$, $\ell \in \R$. On dit que $f$ tend vers $\ell$ quand $(x,y)$ tend vers $(a,b)$ si \[
		\forall \varepsilon > 0, \exists r > 0, \forall (x,y) \in B_{(a,b)}(r) \cap D,
		\left| f(x,y) - \ell \right| \le \varepsilon.
	\]
	\index{limite (dans $\R^2$)}
	\index{tendre vers (dans $\R^2$)}
\end{defn}

\begin{prop}
	\begin{enumerate}
		\item Dans ce contexte, il y a unicité de la limite
		\item La limite d'une somme, d'un produit, d'un quotien, d'une composée se comporte comme dans le cas d'une seule variable.
		\item Soit $f: D \to \R$ continue. Soient $g: I \to \R$ et $h: I \to \R$ continues telles que \[
			\forall t \in I, \big(g(t), h(t)\big) \in D.
		\] Alors \[
			t \in I \mapsto f\big(g(t), h(t)\big) \in \R
		\] est continue.
	\end{enumerate}
\end{prop}

\begin{figure}[H]
	\centering
	\begin{asy}
		import three;
		import graph3;
		size(5cm);

		settings.render = 0;
		settings.prc = false;
		currentprojection = obliqueX;

		draw(O -- X, Arrow3(TeXHead2));
		draw(O -- Y, Arrow3(TeXHead2));
		draw(O -- Z, Arrow3(TeXHead2));

		triple f(real x, real y, real z = 0) { return (x,y,cos(x - 0.5) * cos(y - 0.5)/1.2 + 0.15); }

		real inc = 1 / 5;

		for(real x = 0; x <= 1; x += inc) {
			draw(graph(
				new real(real t) { return x; }, // x
				new real(real y) { return y; }, // y
				new real(real y) { return f(x,y).z; }, // z
				0, 1
			), gray);
		}

		for(real y = 0; y <= 1; y += inc) {
			draw(graph(
				new real(real x) { return x; }, // x
				new real(real t) { return y; }, // y
				new real(real x) { return f(x,y).z; }, // z
				0, 1
			), gray);
		}

		path3 path1 = (0.3, 0.2, 0) .. (0.5, 0.5, 0) .. (0.6, 0.7, 0) .. (0.9, 0.8, 0);
		path3 path2 = (0.3, 0.8, 0) .. (0.5, 0.5, 0) .. (0.6, 0.3, 0) .. (0.9, 0.2, 0);
		path3 pathA = f(0.3, 0.2, 0) .. f(0.5, 0.5, 0) .. f(0.6, 0.7, 0) .. f(0.9, 0.8, 0);
		path3 pathB = f(0.3, 0.8, 0) .. f(0.5, 0.5, 0) .. f(0.6, 0.3, 0) .. f(0.9, 0.2, 0);

		draw(path1, red, Arrow3(TeXHead2, position=0.5));
		draw(pathA, red, Arrow3(TeXHead2, position=0.5));
		draw(path2, deepcyan, Arrow3(TeXHead2, position=0.3));
		draw(pathB, deepcyan, Arrow3(TeXHead2, position=0.3));

		dot((0.5, 0.5, 0));
		dot(f(0.5, 0.5, 0));
		draw((0.5, 0.5, 0) -- f(0.5, 0.5, 0), dashed);
	\end{asy}
\end{figure}


	\part{Transpositions}

\begin{defn}
	Une \underline{transposition} est un cycle de longueur 2 : $\begin{pmatrix}
		a&b
	\end{pmatrix}$ avec $a \neq b$.
	\index{transposition (permutation)}
\end{defn}

\begin{exm}
	Avec $n = 5$ et $\gamma = \begin{pmatrix}
		2&4&1
	\end{pmatrix}$.

	\begin{figure}[H]
		\centering

		\begin{asy}
			size(5cm);

			real rho = 0.15; // circles

			void draw_cycle(pair O, real r ...int[] nums) {
				int n = nums.length;
				real eps = (15 / r) * 0.8;

				for(int i = 0; i < n; ++i) {
					real theta_1 = (360/n) * (i+1);
					real theta_2 = (360/n) * i;

					pair C = O + dir(theta_2) * r;

					draw(circle(C, rho));
					label("$" + string(nums[i]) + "$", C);
					draw(arc(O, r, theta_2+eps, theta_1-eps), Arrow(TeXHead));
				}
			}

			draw_cycle((-1,0), 0.8, 1, 2, 4);
			draw_cycle((1,0), 0.3, 3);
			draw_cycle((2,0), 0.3, 5);
		\end{asy}
	\end{figure}

	\[
		\gamma = \begin{pmatrix}
			1&4
		\end{pmatrix} \begin{pmatrix}
			1&2
		\end{pmatrix}
	\]

	Avec $n = 6$ et $\gamma = \begin{pmatrix}
		1&3&5&6&2
	\end{pmatrix} = \begin{pmatrix}
		1&2&3&4&5&6\\
		3&1&5&4&6&2
	\end{pmatrix}$.

	Donc, \[
		\gamma = \begin{pmatrix}
			1&2
		\end{pmatrix} \begin{pmatrix}
			1&6
		\end{pmatrix} \begin{pmatrix}
			1&5
		\end{pmatrix} \begin{pmatrix}
			1&3
		\end{pmatrix}
	\] 
	\[
		\begin{pmatrix}
			1&2&3&4&5&6\\
			3&2&1&4&5&6\\
			3&2&5&4&1&6\\
			3&2&5&4&6&1\\
			3&1&5&4&6&2\\
		\end{pmatrix}
	\]

	Et, \[
		\gamma = \begin{pmatrix}
			1&3
		\end{pmatrix} \begin{pmatrix}
			2&3
		\end{pmatrix} \begin{pmatrix}
			3&5
		\end{pmatrix} \begin{pmatrix}
			5&6
		\end{pmatrix} 
	\]

	\[
		\begin{pmatrix}
			1&2&3&4&5&6\\
			1&2&3&4&6&5\\
			1&2&5&4&6&3\\
			1&3&5&4&6&2\\
			3&1&5&4&6&2\\
		\end{pmatrix} 
	\] 
\end{exm}

\begin{exm}
	\[
		\begin{pmatrix}
			1&4
		\end{pmatrix} = \begin{pmatrix}
			1&2
		\end{pmatrix} \begin{pmatrix}
			2&3
		\end{pmatrix} \begin{pmatrix}
			3&4
		\end{pmatrix} \begin{pmatrix}
			2&3
		\end{pmatrix} \begin{pmatrix}
			1&2
		\end{pmatrix}
	\]
	On n'a pas toujours le même nombre de transpositions mais la parité du nombre reste la même (proposition plus loin).
\end{exm}

\begin{thm}
	Toute permutation se décompose en produit de transpositions.
\end{thm}

\begin{prv}
	Soit $\gamma = \begin{pmatrix}
		a_1&\cdots&a_k
	\end{pmatrix}$ un $k$-cycle.

	On remarque que
	\[
		\gamma = \begin{pmatrix}
			a_1&a_k
		\end{pmatrix} \cdots \begin{pmatrix}
			a_1&a_4
		\end{pmatrix} \begin{pmatrix}
			a_1&a_3
		\end{pmatrix} \begin{pmatrix}
			a_1&a_2
		\end{pmatrix}
	\] C'est un produit de transpositions.
\end{prv}

\begin{exm}
	Avec $n = 10$ et $\sigma = \begin{pmatrix}
		1&2&3&4&5&6&7&8&9&10\\
		9&8&1&7&2&3&4&5&10&6
	\end{pmatrix}$.

	On a
	\begin{align*}
		\sigma &= \begin{pmatrix}
			1&9&10&6&3
		\end{pmatrix} \begin{pmatrix}
			2&8&5
		\end{pmatrix} \begin{pmatrix}
			4&7
		\end{pmatrix}\\
		&= \begin{pmatrix}
			1&3
		\end{pmatrix} \begin{pmatrix}
			1&6
		\end{pmatrix} \begin{pmatrix}
			1&10
		\end{pmatrix} \begin{pmatrix}
			1&9
		\end{pmatrix} \begin{pmatrix}
			2&5
		\end{pmatrix} \begin{pmatrix}
			2&8
		\end{pmatrix} \begin{pmatrix}
			4&7
		\end{pmatrix} \\
	\end{align*}

	Vérification : \[
		\begin{pmatrix}
			1&2&3&4&5&6&7&8&9&10\\
			1&2&3&7&5&6&4&8&9&10\\
			1&8&3&7&5&6&4&2&9&10\\
			1&8&3&7&2&6&4&5&9&10\\
			9&8&3&7&2&6&4&5&1&10\\
			9&8&3&7&2&6&4&5&10&1\\
			9&8&3&7&2&1&4&5&10&6\\
			9&8&1&7&2&3&4&5&10&6\\
		\end{pmatrix} 
	\] 
\end{exm}

	\part{Familles orthogonales}

\begin{thm}[Pythagore]
	Soit $(x,y) \in E^2$. \[
		\|x+y\|^2 = \|x\|^2 + \|y\|^2 \iff x \perp y
	.\]
	\begin{figure}[H]
		\centering
		\begin{asy}
			size(4cm);
			pair u = (1, 0.5);
			pair v = 1.5 * (0, 1) * u;
			draw((0,0)--u, Arrow(TeXHead));
			label("$x$", u/2, align=S);
			draw(u--v+u, Arrow(TeXHead));
			label("$y$", u + v/2, align=NE);
			draw((0,0) -- u + v, Arrow(TeXHead));
			draw(u + v / 7.5 -- u + v / 7.5 - u / 5 -- u - u / 5 -- u -- cycle);
		\end{asy}
	\end{figure}
\end{thm}

\begin{prv}
	\[
		\|x + y\|^2 = \|x\|^2 + \|y\|^2 \iff 2\left<x \mid y \right> = 0 \iff x \perp y
	.\]
\end{prv}

\begin{defn}
	Soit $(e_i)_{i\in I}$ une famille de vecteurs. On dit que cette famille est \underline{orthogonale} si \[
		\forall i \neq j\, e_i \perp e_j
	.\] Si, en plus, on a \[
		\forall i \in I,\,\|e_i\| = 1,
	\] alors on dit que la famille est \underline{orthonormale} ou \underline{orthonormée}.
	\index{famille orthogonale}
	\index{famille orthonormale}
	\index{famille orthonormée}
\end{defn}

\begin{prop}[Pythagore]
	Soit $(e_1, \ldots, e_n)$ une famille orthogonale. Alors \[
		\left\| \sum_{i=1}^n e_i \right\|^2 = \sum_{i=1}^n \|e_i\|^2
	.\]
\end{prop}

\begin{thm}
	Toute famille orthogonale de vecteurs non nuls est libre.
\end{thm}

\begin{prv}
	Soit $(e_i)_{i\in I}$ une famille orthogonale telle que \[
		\forall i \in I,\,e_i \neq 0_E
	.\] Soit $n \in \N^*$, $(\lambda_1, \ldots, \lambda_n) \in \R^n$. On suppose \[
		\sum_{k=1}^n \lambda_k e_{i_k} = 0_E
	.\] Soit $j \in \left\llbracket 1,n \right\rrbracket$.
	\begin{align*}
		0 &= \left<\sum_{k=1}^n \lambda_k e_{i_k}  \mid e_{i_j} \right>\\
		&= \sum_{k=1}^n \lambda_k \left<e_{i_k}  \mid e_{i_j} \right> \\
		&= \lambda_j \underbrace{\|e_{i_j}\|^2}_{\neq 0} \\
	\end{align*}
	donc $\lambda_j = 0$.
\end{prv}

\begin{algo}[Orthonormalisation de Gran--Schmidt]
	On suppose $E$ de dimension finie. Soit $\mathcal{B} = (e_1, \ldots, e_n)$ une base de $E$.

	\begin{itemize}
		\item\underline{\it Étape 1}: On pose $v_1 = \frac{e_1}{\|e_1\|}$ de sorte que $\|v_1\| = 1$.
		\item\underline{\it Étape 2} : On pose \[
				u_2 = e_2 - \left<e_2  \mid v_1 \right> v_1
			.\] Ainsi,
			\begin{align*}
				\left<u_2 \mid v_1 \right> &= \big<e_2 - \left<e_2 \mid v_1 \right> v_1  \mid v_1 \big>\\
				&= \left<e_2 \mid v_1 \right> - \left<e_2 \mid v_1 \right> \left<v_1 \mid v_1 \right> \\
				&= 0. \\
			\end{align*}
			On pose $v_2 = \frac{u_2}{\|u_2\|}$ donc $v_2 \perp v_1$ et $\|v_2\| = 1$.
		\item\underline{\it Étape 3} : On pose \[
				u_2 = e_3 - \left<e_3 \mid v_1 \right>v_1 - \left<e_3 \mid v_2 \right>v_2
			.\] Ainsi,
			\begin{align*}
				\left<u_3  \mid v_1 \right> &= \left<e_3  \mid v_1 \right> - \left<e_3 \mid v_1 \right>\underbrace{\left<v_1 \mid v_1 \right>}_{=1} - \left<e_3 \mid v_2 \right>\underbrace{\left<v_2 \mid v_1 \right>}_{=0} \\
				&= 0 \\
			\end{align*}
			et 
			\begin{align*}
				\left<u_3 \mid v_2 \right> &= \left<e_3  \mid  v_2 \right> - \left<e_3 \mid v_1 \right> \underbrace{\left<v_1 \mid v_2 \right>}_{=0} - \left<e_3 \mid v_2 \right> \underbrace{\left<v_2 \mid v_2 \right>}_{=1}\\
				&= 0. \\
			\end{align*}
			On pose $v_3 = \frac{u_3}{\|u_3\|}$ de sorte que $v_3 \perp v_1$, $v_3 \perp v_2$ et $\|v_3\| = 1$.
		\item\underline{\it Étape $i+1$}: On pose \[
			u_{i+1} = e_{i+1} - \sum_{k=1}^i \left<e_{i+1}  \mid v_k \right> v_k
		.\] Ainsi, pour tout $j \in \left\llbracket 1,i \right\rrbracket,$ on a
		\begin{align*}
			\left<u_{i+1}  \mid v_j \right> &= \left<e_{i+1}  \mid v_j \right> - \sum_{k=1}^i \left<e_{i+1} \mid v_k \right> \left<v_k \mid v_j \right> \\
			&= \left<e_{i+1} \mid v_j \right> - \left<e_{i+1} \mid v_j \right> \|v_j\|^2 \\
			&= 0. \\
		\end{align*}
		On pose $v_{i+1} = \frac{u_{i+1}}{\|u_{i+1}\|}$.
	\end{itemize}
\end{algo}

\begin{exm}
	Avec $E = \R_3[X]$, $\left<P \mid Q \right> = \int_{0}^{1} P(t)\,Q(t)~\mathrm{d}t$ et $\mathcal{B} = (1, X, X^2, X^3)$.
	\begin{enumerate}
		\item $\|1\|^2 = \left<1 \mid 1 \right> = \int_{0}^{1} 1~\mathrm{d}t = 1$ et donc $v_1 = 1$.
		\item $u_2 = X - \left<X  \mid v_1 \right>v_1$. Or, $\left<X \mid v_1 \right> = \int_{0}^{1} t~\mathrm{d}t = \frac{1}{2}$. D'où $u_2 = X - \frac{1}{2}$.
			\begin{align*}
				\|u_2\|^2 &= \int_{0}^{1} \left( t - \frac{1}{2} \right)^2~\mathrm{d}t \\
				&= \int_{0}^{1} \left( t^2 - t + \frac{1}{4} \right)~\mathrm{d}t \\
				&= \frac{1}{3} - \frac{1}{2} + \frac{1}{4} \\
				&= \frac{1}{12} \\
			\end{align*} On en déduit que $v_2 = \sqrt{12}\left( X - \frac{1}{2} \right)$.
		\item $u_3 = X^2 - \left<X^2 \mid v_1 \right>v_1 - \left<X^2 \mid v_2 \right>v_2$.
			On a \[
				\left<X^2 \mid v_1 \right> = \int_{0}^{1} t^2~\mathrm{d}t = \frac{1}{3}
			\] et
			\begin{align*}
				\left<X^2 \mid v_2 \right> &= \sqrt{12} \int_{0}^{1} t^2\left( t - \frac{1}{2} \right)~\mathrm{d}t \\
				&= \frac{\sqrt{12}}{12}. \\
			\end{align*}
			D'où
			\begin{align*}
				u_3 &= X^2 - \frac{1}{3} - \frac{\sqrt{12}}{12}\sqrt{12} \left( X - \frac{1}{2} \right)\\
				&= X^2 - \frac{1}{3} - X + \frac{1}{2} \\
				&= X^2 - X + \frac{1}{6}. \\
			\end{align*}
			\begin{align*}
				\|u_3\|^2 &= \int_{0}^{1} \left( t^2 - t + \frac{1}{6} \right)~\mathrm{d}t\\
				&= \int_{0}^{1} \left( t^4 + t^2 + \frac{1}{36} - 2t^3 + \frac{t^2}{3} - \frac{t}{3} \right) ~\mathrm{d}t \\
				&= \frac{1}{5} + \frac{1}{3} + \frac{1}{36} - \frac{1}{2} + \frac{1}{9} - \frac{1}{6} \\
				&= \frac{36 + 60 + 5 - 90 + 20 - 30}{180} \\
				&= \frac{1}{180} \\
			\end{align*}
			On en déduit que \[
				v_3 = 6\sqrt{5}\left( X^2 - X + \frac{1}{6} \right).
			\]
		\item Exercice : calculer $v_4$.
	\end{enumerate}
\end{exm}

\begin{prop}
	Soit $\mathcal{B} = (e_1, \ldots, e_n)$ une base de $E$ et $\mathcal{C}$ la base obtenue par le procédé d'orthonormalisation de Gram--Schmidt. Alors, \[
		\forall i \in \left\llbracket 1,n \right\rrbracket,\,\Vect(e_1,\ldots, e_i) = \Vect(v_1, \ldots, v_i)
	.\]\qed
\end{prop}

\begin{exm}[orthogonalisation]
	\begin{itemize}
		\item $u_1 = 1$.
		\item
			\begin{align*}
				\begin{rcases*}
					u_2 \in \Vect(e_1, e_2)\\
					u_2 \perp u_1
				\end{rcases*}
				\iff& \begin{cases}
					u_2 = ae_1 + be_2\quad (a,b) \in \R^2\\
					\left<u_1 \mid u_2 \right> = 0
				\end{cases}\\
				\iff& \begin{cases}
					u_2 = a + bX\\
					\int_{0}^{1} (a+bt)~\mathrm{d}t = 0.
				\end{cases}\\
			\end{align*}
			\begin{align*}
				\int_{0}^{1} (a+bt)~\mathrm{d}t = 0 \iff& a + \frac{b}{2} = 0\\
				\iff& a = -\frac{b}{2}\\
				\iff& u_2 = -\frac{b}{2} + bX.
			\end{align*}
			Par exemple, $u_2 = -1 + 2X$.
		\item $\begin{cases}
				u_3 \in \Vect(e_1, e_2, e_3)\\
				u_3 \perp u_1\\
				u_3 \perp u_2
			\end{cases}$

			On pose $u_3 = a + bX + cX^2$ avec $(a,b,c)\in \R^3$.
			\begin{align*}
				\begin{rcases*}
					\int_{0}^{1} \left( a+bt + ct^2 \right)~\mathrm{d}t = 0\\
					\int_{0}^{1} \left(a + bt+ct^2\right)(2t - 1)~\mathrm{d}t = 0
				\end{rcases*} \iff& \begin{cases}
					a + \frac{b}{2} + \frac{c}{3} = 0\\
					\int_{0}^{1} \left( 2ct^3 + (-c + 2b)t^2 + (2a - b)t - a \right) ~\mathrm{d}t = 0
				\end{cases}\\
				\iff& \begin{cases}
					a + \frac{b}{2} + \frac{c}{3} = 0\\
					\frac{c}{2} + \frac{2b - c}{3} + \frac{2\cancel{a} - b}{2} - \cancel{a} = 0
				\end{cases}\\
				\iff&  \begin{cases}
					a = -\frac{b}{2} - \frac{c}{3} = \frac{c}{2} - \frac{c}{3} = \frac{c}{6}\\
					b = -c.
				\end{cases}
			\end{align*}
			On en déduit que \[
				u_3 = 1 - 6X + 6X^2
			.\]
	\end{itemize}
\end{exm}

\begin{crlr}[théorème de la base orthonormée incomplète] Soit $(e_1, \ldots, e_k)$ une base orthonormée d'un espace euclidien. On peut trouver $e_{k+1},\ldots,e_n$ tels que $(e_1, \ldots, e_k, e_{k+1},\ldots,e_n)$ soit une base orthonormée de $E$.
\end{crlr}

\begin{prv}
	On sait que $(e_1, \ldots, e_k)$ est libre. On complète $(e_1, \ldots, e_k)$ en une base $\mathcal{B}$ de $E$. On orthonormalise $\mathcal{B}$ : on obtient une base orthonormée $\mathcal{C}$ de $E$. En détaillant l'algorithme de Gram--Schmidt, on s'aper\c coit que les $k$ premiers vecteurs de $\mathcal{C}$ sont ceux de $\mathcal{B}$.
\end{prv}

\begin{thm}
	Soit $E$ un espace euclidien et $\mathcal{B} = (e_1, \ldots, e_n)$ une base orthonormée de $E$. Soit $(x,y) \in E^2$. On pose $(x_1, \ldots, x_n) \in \R^n$ et $(y_1, \ldots, y_n) \in \R^n$ tels que \[
		x = \sum_{i=1}^n x_i e_i \qquad\qquad y = \sum_{i=1}^n y_i e_i
	.\] Alors \[
		\left<x \mid y \right> = \sum_{i=1}^n x_i y_i
	.\]
	\vspace{3mm}
	Soit $X = \mat{x_1\\\vdots\\x_n}$ et $Y = \mat{y_1\\ \vdots \\ y_n}$. Alors, \[
		\left<x \mid y \right> = X^\T\,Y
	.\]
\end{thm}

\begin{prv}
	\begin{align*}
		\left<x \mid y \right> &= \left<\sum_{i=1}^n x_ie_i  \mid y \right>\\
		&= \sum_{i=1}^n x_i \left<e_i  \mid y \right> \\
		&= \sum_{i=1}^n x_i \left<e_i  \mid \sum_{j=1}^n y_j e_j \right> \\
		&= \sum_{i=1}^n x_i \sum_{j=1}^n y_j \underbrace{\left<e_i \mid e_j \right>}_{\delta_i^j} \\
		&= \sum_{i=1}^n x_i y_i. \\
	\end{align*}
\end{prv}

\begin{prop}
	Soit $E$ un espace euclidien et $\mathcal{B} = (e_1, \ldots, e_n)$ une base orthonormée de $E$. Alors, \[
		\forall x \in E,\,x = \sum_{i=1}^n \left<x \mid e_i \right>e_i
	.\]
\end{prop}

\begin{prv}
	Soit $x \in E$. On pose \[
		x = \sum_{i=1}^n x_i e_i
	\] avec $(x_1, \ldots, x_n) \in \R^n$. Soit $j \in \left\llbracket 1,n \right\rrbracket$. On a
	\begin{align*}
		\left<x \mid e_j \right> &= \left<\sum_{i=1}^n x_i e_i  \mid e_j \right>\\
		&= \sum_{i=1}^n x_i \left<e_i \mid e_j \right> \\
		&= x_j. \\
	\end{align*}
\end{prv}

	\part{Lois de composition}

\begin{defn}
	Une \underline{loi de composition interne} \index{loi de composition interne} est une application $f$ de $E \times E$ dans $E$.
	
	On la note $x * y$ au lieu de $f(x,y)$ (on est libre de choisir le symbôle).
\end{defn}

\begin{defn}
	Soit $E$ un ensemble muni d'une loi de composition interne $\boxtimes$.

	On dit que $\boxtimes$ est \underline{associative} \index{associativité (loi de composition interne)} si \[
		\forall (x,y,z) \in E^3,\;(x\boxtimes y)\boxtimes z = x \boxtimes (y \boxtimes z).
	\] Dans ce cas, on écrit plutôt $x \boxtimes y \boxtimes z$.
\end{defn}

\begin{exm}
	\begin{itemize}
		\item $+$ et $\times $ dans $\C$ sont associatives;
		\item $ \circ$ est associative;
		\item  la multiplication matricielle est aussi associative.
	\end{itemize}
\end{exm}

\begin{defn}
	On dit que $\boxtimes$ est \underline{commutative} \index{commutativité (loi de composition interne)} si \[
		\forall (x,y) \in E^2, x\boxtimes y = y\boxtimes x.
	\]
\end{defn}

\begin{exm}
	\begin{itemize}
		\item $+$ et $\times $ dans $\C$ sont commuatives;
		\item $ \circ $ n'est pas commutative;
		\item  la multiplication matricielle n'est pas commutative.
	\end{itemize}
\end{exm}

\begin{defn}
	Soit $e \in E$. On dit que $e$ est un
	\begin{itemize}
		\item \underline{élément neutre à gauche}\index{élément neutre à gauche (loi de composition interne)} si  \[
				\forall x \in E,\; e\boxtimes x = x;
			\]
		\item \underline{élément neutre à droite}\index{élément neutre à droite (loi de composition interne)} si  \[
				\forall x \in E,\; x\boxtimes e = x;
			\]
		\item \underline{élément neutre}\index{élément neutre (loi de composition interne)} si  \[
				\forall x \in E,\; e\boxtimes x = x\boxtimes e = x.
			\]
	\end{itemize}
\end{defn}

\begin{prop}
	Sous reserve d'existence, il y a unicité de l'élément neutre.
\end{prop}

\begin{prv}
	Soient $e$ et $e'$ deux éléments neutre.
	\begin{itemize}
		\item $e \boxtimes e' = e'$ car $e$ est neutre,
		\item $e \boxtimes e' = e$ car $e'$ est neutre.
	\end{itemize} On a donc $e = e'$.
\end{prv}

\begin{axm}[axiome du choix]
	Soit $E$ un ensemble non vide. Il existe $f : \mathcal{P}(E) \setminus \{\O\} \to E$ telle que \[
		\forall A \in \mathcal{P}(E) \setminus \{\O\},\; f(A) \in A.
	\]
\end{axm}

\begin{defn}
	Soit $f: E \to F$. Le \underline{graphe} \index{graphe (application)} de $f$ est \[
		\Big\{\big(x,f(x)\big)  \mid x \in E\Big\} \subset E \times F.
	\]
\end{defn}

\begin{prop}
	Soit $G \subset E\times F$. $G$ est le graphe d'une application si et seulement si \[
		\forall x \in E,\,\exists! y \in F,\, (x,y) \in G.
	\]
\end{prop}

\begin{prv}
	\begin{itemize}
		\item[``$\implies$''] par définition d'une application
		\item[``$\impliedby$''] On pose $f(x)$ le seul élément $y$ de $F$ qui vérifie $(x,y) \in G$. Alors $f \in F^E$ et son graphe vaut $G$.
	\end{itemize}
\end{prv}

\begin{defn}
	Soit $A \in \mathcal{P}(E)$. L'\underline{indicatrice}\index{indicatrice (ensemble)} de $A$ est \begin{align*}
		\mathbbm{1}_A: E &\longrightarrow \{0,1\} \\
		x &\longmapsto \begin{cases}
			1 &\text{ si } x \in A,\\
			0 & \text{ si } x \not\in A.
		\end{cases}
	\end{align*}
\end{defn}

\begin{exm}
	\begin{enumerate}
		\item Dans $\C$, le neutre de $+$ est $0$ et le neutre de $\times$ est $1$.
		\item Dans $E^E$, le neutre de $ \circ $ est $\id_E$.
		\item Dans $\mathcal{M}_n(\C)$ (l'ensemble des matrices carrées $n \times n$ à valeurs dans $\C$), le neutre de $\times $ est $I_n$ : \[
				I_n =
				\begin{pNiceMatrix}
					1&&(0)\\
					&\Ddots&\\
					(0)&&1
				\end{pNiceMatrix}
			\] 
	\end{enumerate}
\end{exm}

\begin{defn}
	Soit $E$ un ensemble muni d'une loi de composition interne $\boxtimes$ et $x \in E$.

	\begin{enumerate}
		\item On dit que $x$ est \underline{simplifiable à gauche}\index{simplifiabilité à gauche} si \[
				\forall (y,z) \in E^2,\,(x\boxtimes y = x \boxtimes z) \implies x = z.
			\] et que $x$ est \underline{simplifiable à droite}\index{simplifiabilité à droite} si \[
				\forall (y,z) \in E^2,\,(y\boxtimes x = z \boxtimes y) \implies x = z.
			\]
		\item On dit que $x$ est \underline{symétrisable à gauche}\index{symétrisabilité à gauche} s'il exiiste $y \in E$ tel que $y\boxtimes x = e$ où $e$ est l'élément neutre de $\boxtimes$.

			De même, on dit que $x$ est \underline{symétrisable à droite}\index{symétrisabilité à droite} s'il existe $y \in E$ tel que $x \boxtimes y = e$.

			On dit que $x$ est \underline{symétrisable}\index{symétrisabilité} s'il est symétrisable à gauche et à droite, donc s'il existe $y \in E$ tel que $x \boxtimes y = y \boxtimes x = e$.
	\end{enumerate}
\end{defn}

\begin{exm}
	$E = \N$ muni de la loi $+$, tous les éléments de $E$ sont simplifiables. $0$ est le seuele élément de $E$ symétrisable.
\end{exm}

\begin{prop}
	Avec les notations précédentes, si $\boxtimes$ est associative, et $x$ est symétrisable, alors $x$ est simplifiable.
\end{prop}

\begin{prv}
	Soient $y, z \in E$.
	\begin{itemize}
		\item On suppose $x \boxtimes y = x \boxtimes z$. Soit $a \in E$ tel que $a\in E$ tel que $a \boxtimes x = e$. Alors \[
				a \boxtimes (x\boxtimes y) = a \boxtimes (x \boxtimes z).
			\] Or,
			\begin{align*}
				a \boxtimes (x \boxtimes y) &= (a \boxtimes x) \boxtimes y \\
				&= e \boxtimes y \\
				&= y. \\
			\end{align*}

			De même, $a \boxtimes (x \boxtimes z) = z$.

			Donc $y = z$.
		\item De même, si $y \boxtimes x = z \boxtimes x$, on ``multiplie'' $x$ à droite par $a$ et on obtient $y = z$.
	\end{itemize}
\end{prv}

\begin{prop-defn}
	On suppose $\boxtimes$ associative. Soit $x \in E$ symétrisable. Alors \[
		\exists ! y \in E,\; x \boxtimes y = y \boxtimes x = e.
	\] On dit que $y$ est le \underline{symétrique}\index{symétrique (loi de composition interne)} de $x$ et on le note $y = x^*$.
\end{prop-defn}

\begin{prv}
	Soeint $x,y,z \in E$ tels que \[
		\begin{cases}
			 x \boxtimes y = y \boxtimes x = e\\
			 x \boxtimes z = z \boxtimes x = e\\
		\end{cases}
	\] Alors, $x \boxtimes y = x \boxtimes z$ et, en simplifiant par $x$, on a $y = z$.
\end{prv}

\begin{exm}
	Les fonctions symétrisables de $(E^E,  \circ)$ sont les bijections et le symétrique d'une bijection est sa réciproque.
\end{exm}

\begin{rmk}
	\begin{enumerate}
		\item Si la loi est notée $+$, on parle d'\underline{opposé}\index{opposé (loi de composition interne)} plutôt que de symétrique et on le note $-x$ au lieu de $x^*$.
			L'élément neutre est noté $0_E$.
		\item Si la loi est notée $\times$, on parle d'élément \underline{inversible}\index{inversibilité (loi de composition interne)} au lieu de symétrisable, d'\underline{inverse}\index{inverse (loi de composition interne)} au lieu de symétrique et on note $x^{-1}$ au lieu de $x^*$. On note le neutre $1_E$.
	\end{enumerate}
\end{rmk}

\begin{exo}
	Soient $x,y \in E = \R^+_*$. On définit la loi de composition interne $\oplus$ : \[
		x \oplus y = \frac{1}{\frac{1}{x}\oplus \frac{1}{y}}.
	\] Cette loi peut-être utile en physique pour le calcul de résistances équivalentes en parallèles.
	\begin{itemize}
		\item {\sc Associativité} : soient $x,y,z \in E$.

			D'une part, on a \[
				x \oplus (y \oplus z) = \frac{1}{\frac{1}{x} + \frac{1}{\frac{1}{\frac{1}{x}+ \frac{1}{y}}}} = \frac{1}{\frac{1}{x}+\frac{1}{y}+\frac{1}{z}}.
			\] D'autre part, on a \[
			(x \oplus y) \oplus z = \frac{1}{\frac{1}{\frac{1}{\frac{1}{x}+\frac{1}{y}}}+\frac{1}{z}} = \frac{1}{\frac{1}{x}+ \frac{1}{y}+\frac{1}{z}}.
			\] La loi $\oplus$ est associative.
		\item {\sc Commutativité} : soient $x, y \in E$. \[
				x \oplus y = \frac{1}{\frac{1}{x}+\frac{1}{y}} = \frac{1}{\frac{1}{y}+\frac{1}{x}} = y\oplus x.
			\] Donc la loi $\oplus$ est commutative.
		\item {\sc Élément neutre} : soit $e$ l'élément neutre de $\oplus$. \[
				\forall x \in E,\; x \oplus e = e \oplus x = x.
			\] Comme la loi est commutative, seul l'égalité $x \oplus e = x$ est utile.

			Soit $x \in E$. On a donc $\frac{1}{\frac{1}{x}+\frac{1}{e}}=x$ donc $\frac{ex}{e+x}=x$ donc $ex = x(e+x)$ et donc $\cancel{ex} = \cancel{ex} + x^2$. On en déduit que $x^2 = 0$, ce qui n'est pas possible car $x \in \R^+_*$. Donc, il n'y a pas d'élément neutre pour $\oplus$.
	\end{itemize}
\end{exo}


	\chap[13]{Systèmes linéaires et calculs matriciels}
	\renewcommand{\cwd}{../chap13}
	\part{Topologie de $\R^2$}

\begin{defn}
	La \underline{norme (euclidienne)} de $\R^2$ est l'application définie par \[
		\forall (x,y) \in \R^2, \|(x,y)\| = \sqrt{x^2 + y^2}.
	\]

	\begin{figure}[H]
		\centering
		\begin{asy}
			import graph;
			axes(EndArrow);
			size(4cm);
			pair A = (3,2);
			dot(A);
			draw((3,0)--A, dashed);
			draw((0,2)--A, dashed);
			label("$x$", (3,0), align=S);
			label("$y$", (0,2), align=W);
			draw((0,0)--A);
			dot((4,3), white+0);
		\end{asy}
	\end{figure}
	\index{norme (de $\R^2$)}
	\index{norme euclidienne (de $\R^2$)}
\end{defn}

\begin{prop}
	La norme euclidienne vérifie:
	\begin{enumerate}
		\item (séparation) \[
			\forall (x,y) \in \R^2, \|(x,y)\| = 0 \iff x = y = 0,
			\]
		\item (homogénéité positive) \[
				\forall \lambda \in \R, \forall (x,y) \in \R^2, \|\lambda(x,y)\|= \left| \lambda \right| \|(x,y)\|
			\]
		\item (inégalité triangulaire) \[
			\forall (x,y), (a,b) \in \R^2,
			\|(x,y)+(a,b)\|\le \|(x,y)\|+\|(a,b)\|.
		\]
	\end{enumerate}
\end{prop}

\begin{prv}
	Déjà vue en replaçant $(x,y)$ par $x+iy \in \C$ et $\|(x,y)\|$ par |x+iy|
\end{prv}

\begin{defn}
	Soit $(a,b) \in \R^2$ et $r \in \R_+$.

	La \underline{boule ouverte} (ou \underline{disque ouvert}) de centre $(a,b)$ et de rayon $r$ est \[
		B_{(a,b)}(r) = \big\{ (x,y) \in \R^2  \mid \|(x,y) - (a,b)\| < r \big\}.
	\]

	La \underline{boule fermée} (ou \underline{disque fermé}) de centre $(a,b)$ et de rayon $r$ est \[
		\overline{B_{(a,b)}}(r) = \big\{ (x,y)\in \R^2  \mid \|(x,y) - (a,b)\| \le r \big\}.
	\]

	La \underline{sphère} (ou \underline{boule}) de centre $(a,b)$ et de rayon $r$ est \[
		S_{(a,b)}(r) = \partial \overline{B_{(a,b)}}(r) = \big\{ (x,y) \in \R^2  \mid \|(x,y) - (a,b)\| = r \big\}.
	\]
	\index{boule ouverte (de $\R^2$)}
	\index{disque ouverte (de $\R^2$)}
	\index{boule fermée (de $\R^2$)}
	\index{disque fermée (de $\R^2$)}
	\index{boule (de $\R^2$)}
	\index{sphère (de $\R^2$)}
\end{defn}

\begin{figure}[H]
		\centering
		\incfig{boule}
\end{figure}

\begin{rmk}
	On parle de boule en dimension quelconque.
\end{rmk}

\begin{defn}
	Une \underline{partie ouverte} $O$ de $\R^2$ (ou \underline{un ouvert}) si \[
		\forall (x,y) \in O, \exists r > 0, B_{(a,b)}(r) \subset O.
	\]
	Une partie $F$ est \underline{fermée} su $\R^2\setminus F$ est ouverte.
	\index{partie ouverte (de $\R^2$)}
	\index{ouvert (de $\R^2$)}
	\index{partie fermée (de $\R^2$)}
\end{defn}

\begin{figure}[H]
	\centering
	\incfig{partie-ouverte}
\end{figure}

\begin{prop}
	Une boule ouverte est ouverte. Une boule fermée est fermée.
\end{prop}

\begin{figure}[H]
	\centering
	\begin{subfigure}{4cm}
		\centering
		\begin{asy}
			import patterns;

			pair n(pair a) {return a / length(a);}

			add("hatch",hatch(2mm, SW, red));
			size(4cm);

			draw(circle((0,0), 1));
			dot('$(a_0, b_0)$', (0,0), align=S);

			draw((0,0) -- n((-1, 1)), dashed);
			label("$r$", n((-1, 1)) / 2, align=1.5S);

			pair A = n((1,3)) * (2/3);
			real rho = (1 - length(A)) * (2 / 3);

			dot("$(a,b)$", A, red, align=3SE);
			filldraw(circle(A, rho), pattern("hatch"), red);

			label("$O$", n((1,-1))*2.5/3);
		\end{asy}
	\end{subfigure}
	\begin{subfigure}{1cm}
		\centering~\\
	\end{subfigure}
	\begin{subfigure}{5cm}
		\centering
		\begin{asy}
			import patterns;

			pair n(pair a) {return a / length(a);}

			add("hatch",hatch(1mm, SW, red));
			add("hatch2",hatch(3mm, SE, blue));
			size(5cm);

			guide around = (-1.5, -1.5) -- (-1.5, 1.5) -- (2.5, 1.5) -- (2.5, -1.5) -- cycle;

			pair A = n((3, 1)) * 5/3; 
			real rho = 2 / 9;

			picture inter;
			fill(inter, around, pattern("hatch2"));
			fill(inter, circle((0,0), 1), white);
			add(inter);

			draw(circle((0,0), 1));
			dot('$(a_0, b_0)$', (0,0), align=S);

			draw((0,0) -- n((-1, 1)), dashed);
			label("$r$", n((-1, 1)) / 2, align=1.5S);

			dot("$(a,b)$", A, red, align=2SE);
			filldraw(circle(A, rho), pattern("hatch"), red);

			label("$F$", n((1,-1))*2.5/3);
		\end{asy}
	\end{subfigure}
\end{figure}

\begin{prv}
	$\O$ est un ouvert.

	Soit $B$ la boule ouverte de centre $(a_0, b_0) \in \R^2$ et de rayon $r > 0$.

	On pose $\rho = \frac{1}{2}\big(r - \|(a,b) - (a_0,b_0)\|\big)$.
	Montrons que \[
		B_{(a,b)}(\rho) \subset  B_{(a,b)}(r).
	\]

	Soit $(x,y) \in B_{(a,b)}(\rho)$.
	\begin{align*}
		\|(x,y) - (a_0,b_0)\|&= \|(x,y)- (a,b) + (a,b) - (a_0,b_0)\| \\
		&\le \|(x,y) - (a,b)\| + \|(a,b) - (a_0, b_0)\|\\
		&< \rho + \|(a,b) - (a_0, b_0)\| = \frac{1}{2}r + \frac{1}{2} \|(a,b) - (a_0, b_0)\|\\
		&< r
	\end{align*}
	
	Soit $F$ la boule fermée de centre $(a_0, b_0)$ et de rayon $r \ge 0$.

	Soit $(a,b) \not\in F$. On pose \[
		\rho = \frac{1}{2}\big(\|(a,b) - (a_0, b_0)\| - r\big) > 0.
	\]

	Montrons que $B_{(a,b)}(\rho) \subset \R^2\setminus F$.

	Soit $(x,y) \in B_{(a,b)}(\rho)$.

	\begin{align*}
		\|(x,y) - (a_0, b_0)\| &= \|(x,y) - (a,b) + (a,b) - (a_0, b_0)\| \\
		&\ge \big| \underbrace{\|(x,y) - (a,b)\|}_{\le \rho} - \underbrace{\|(a,b) - (a_0, b_0)\|}_{> r} \big|\\
		&\ge \|(a,b) - (a_0, b_0)\|- \|(x,y) - (a,b)\|\\
		&> \|(a,b) - (a_0, b_0)\|- \rho\\
		&> \frac{1}{2} \|(a,b) - (a_0, b_0)\| + \frac{1}{2}r\\
		&> r
	\end{align*}

	donc $(x,y) \not\in F$.
\end{prv}

\begin{exm}
	\begin{enumerate}
		\item $\O$ est ouvert.\\
			$\R^2$ est ouvert.
		\item $\O$ est fermé.\\
			$\R^2$ est fermé.\\
		\item $\big\{(x,0)  \mid x > 0\big\}$ n'est ni ouverte ni fermé.
	\end{enumerate}
\end{exm}

\begin{figure}[H]
	\centering
	\begin{asy}
		size(3cm);

		draw((0, -1) -- (0, 3), Arrow(TeXHead));
		draw((-1, 0) -- (3, 0), Arrow(TeXHead));
		
		draw((0,0) -- (0, 2.97), red);
		draw(circle((0,1.5), 0.5), deepred);
		draw(circle((0,0.5), 0.1), deepred);
	\end{asy}
\end{figure}

\begin{defn}
	Soit $(a,b) \in \R^2$ et $V \in \mathcal{P}(\R^2)$.

	On dit que $V$ est un voisinage de $(a,b)$ s'il existe $r > 0$ tel que \[
		B_{(a,b)}(r) \subset V.
	\]
	\index{voisinage (dans $\R^2$)}
\end{defn}

\begin{prop}
	Un ouvert non vide est un voisinage en chacun de ces points. \qed
\end{prop}

\begin{defn}
	Soit $D \subset \R^2$. Un \underline{point intérieur} de $D$ est un couple $(a,b) \in D$ tel que \[
		\exists r > 0, B_{(a,b)}(r) \subset D.
	\] en d'autres termes, si $D$ est un voisinage de $(a,b)$.

	On note $\mathring D$ l'ensemble des points intérieurs à $D$. C'est \underline{l'intérieur} de $D$.
	\index{point intérieur (dans $\R^2$)}
	\index{intérieur (dans $\R^2$)}
\end{defn}

\begin{prop}
	$\mathring D$ est le plus grand ouvert $O$ de $\R^2$ tel que $O \subset D$.
\end{prop}

\begin{figure}[H]
	\centering
	\incfig{interieur-plus-grand-ouvert}
\end{figure}


\begin{prv}
	Soit $(a,b) \in \mathring D$.

	Par définition, il existe $r > 0$ tel que \[
		B_{(a,b)}(r) \subset D.
	\] Montrons que $B_{(a,b)}(r) \subset \mathring D$.

	Soit $(x,y) \in B_{(a,b)}(r)$. Comme $B_{(a,b)}(r)$ est un ouvert de $\R^2$, il existe $\rho > 0$ tel que \[
		B_{(x,y)}(\rho) \subset B_{(a,b)}(r)
	\] donc $(x,y) \in \mathring D$.

	Donc $\mathring D$ est ouvert, $\mathring D \subset D$.

	Soit $O$ un ouvert de $\R^2$ tel que $O \subset D$. Montrons que $O \subset \mathring D$.

	Soit $(x,y) \in O$. Soit $r > 0$ tel que \[
		B_{(x,y)}(r) \subset O \subset D
	\] donc $(x,y) \in \mathring D$.
\end{prv}

\begin{defn}
	Soit $f: D \subset \R^2 \to \R$, $\ell \in \R$, $(a,b) \in \mathring D$.

	On dit que \underline{$f(x,y)$ tend vers $\ell$ quand $(x,y)$ tend vers $(a,b)$} ou que $\ell$ est \underline{une limite} de $f$ en $(a,b)$ si \[
		\forall \varepsilon > 0, \exists r > 0, \forall (x,y) \in D, \|(x,y) - (a,b)\| < r \implies \left| f(x,y) - \ell \right| \le \varepsilon.
	\] en d'autres termes si \[
		\forall V \in \mathcal{V}_{\ell}, \exists W \in \mathcal{V}_{(a,b)}, \forall (x,y) \in W \cap D, f(x,y) \in V.
	\]
	\index{limite (dans $\R^2$)}
	\index{tendre vers (dans $\R^2$)}
\end{defn}

\begin{prop}
	[unicité de la limite]
	Soit $f: D \to \R$, $(a,b) \in \mathring D$, $\ell_1, \ell_2 \in \R$ telles que $\ell_1$ et $\ell_2$ sont des limites de $f$ en $(a,b)$.

	Alors $\ell_1 = \ell_2$.
\end{prop}

\begin{figure}[H]
	\centering
	\incfig{preuve-unicité-de-la-limite}
\end{figure}

\begin{prv}
	On suppose $\ell_1 < \ell_2$. On pose $\varepsilon = \frac{\ell_2 - \ell_1}{2} > 0$.

	Soit $r_1 > 0$ tel que \[
		f\big(B_{(a,b)}(r_1)\big) \subset ]\ell_1 - \varepsilon, \ell_1 + \varepsilon[.
	\] Soit $r_2 > 0$ tel que \[
		f\big(B_{(a,b)}(r_2)\big) \subset ]\ell_2 - \varepsilon, \ell_2 + \varepsilon [.
	\] On pose $r = \min(r_1, r_2)$ donc \[
		B_{(a,b)}(r_1) \cap B_{(a,b)}(r_2) = B_{(a,b)}(r) \neq \O.
	\] Soit $(x,y) \in B_{(a,b)}(r)$. Alors, \[
		f(x,y) \in ]\ell_1 - \varepsilon, \ell_1 + \varepsilon[ \cap ]\ell_2 - \varepsilon, \ell_2 + \varepsilon[ = \O.
	\] $\lightning$
\end{prv}

\begin{defn}
	Soit $f : D \to \R$, $(a,b) \in \mathring D$.

	On dit que $f$ est \underline{continue} en $(a,b)$ si \[
		f(x,y) \tendsto{(x,y) \to (a,b)}f(a,b).
	\]
	\index{continuité (dans $\R^2$)}
\end{defn}

\begin{prop}
	\underline{Si} $f(x,y) \tendsto{(x,y) \to (a,b)} \ell$ \\
	\underline{alors} $\begin{cases}
		f(x,b) \tendsto{x \to a} \ell\\
		f(a,y) \tendsto{y \to b} \ell.\\
	\end{cases}$
\end{prop}

\begin{prv}~\\
	\begin{figure}[H]
		\centering
		\incfig{limite-x-en-a-et-y-en-b}
	\end{figure}
\end{prv}

\underline{Contre-exemple} : exercice 3.

\begin{exm}
	\begin{enumerate}
		\item $f : \begin{array}{rcl}
				\R^2 &\longrightarrow& \R \\
				(x,y) &\longmapsto& x
			\end{array}$ limite en $(0,0)$ ?

			Soit $\varepsilon > 0$. On pose $r = \varepsilon$. \[
				\forall (x,y) \in B_{(0,0)}(r),
				\left| f(x,y) \right| = \left| x \right| \le \|(x,y)\| < r = \varepsilon
			\] Donc $f(x,y) \tendsto{(x,y) \to (a,b)} 0$.
		\item limite $f : \begin{array}{rcl}
				\R^2 &\longrightarrow& \R \\
				(x,y) &\longmapsto& x^3
			\end{array}$ en $(0,0)$ ?

			Soit $\varepsilon > 0$. On pose $r = \sqrt[3]{r} > 0$. \[
				\forall (x,y) \in B_{(0,0)}(r),
				\left| f(x,y) \right| = \left| x^3 \right| \le \|(x,y)\|^3 < r^3 = \varepsilon.
			\]
		\item limite de $f : \begin{array}{rcl}
			\R^2 &\longrightarrow& \R \\
			(x,y) &\longmapsto& x^3y^2
		\end{array}$ en $(0,0)$ ?

		Soit $\varepsilon > 0$. On pose $r = \sqrt[5]{\varepsilon} > 0$. \[
			\forall (x,y) \in B_{(0,0)}(r), \left| f(x,y) \right| = \left| x^3 y^2 \right| \le \|(x,y)\|^3 \|(x,y)\|^2 < r^5 = \varepsilon.
		\]
	\end{enumerate}
\end{exm}

\begin{defn}
	Soient $D \subset \R^2$ et $(x,y) \in \R^2$.

	\begin{figure}[H]
    \centering
    \incfig{point-adhérent}
	\end{figure}
	
	On dit que $(x,y)$ est \underline{adhérent} à $D$ si \[
		\forall r > 0, B_{(x,y)}(r) \cap D \neq \O.
	\] L'ensemble des points adhérents à $D$ est noté $\overline{D}$. On dit que $\overline{D}$ est \underline{l'adhérence} de $D$.
	\index{point adhérent (dans $\R^2$)}
	\index{adhérent (dans $\R^2$)}
\end{defn}

\begin{defn}
	Soit $f: D \subset \R^2 \to \R$ et $(a,b) \in \overline{D}$, $\ell \in \R$. On dit que $f$ tend vers $\ell$ quand $(x,y)$ tend vers $(a,b)$ si \[
		\forall \varepsilon > 0, \exists r > 0, \forall (x,y) \in B_{(a,b)}(r) \cap D,
		\left| f(x,y) - \ell \right| \le \varepsilon.
	\]
	\index{limite (dans $\R^2$)}
	\index{tendre vers (dans $\R^2$)}
\end{defn}

\begin{prop}
	\begin{enumerate}
		\item Dans ce contexte, il y a unicité de la limite
		\item La limite d'une somme, d'un produit, d'un quotien, d'une composée se comporte comme dans le cas d'une seule variable.
		\item Soit $f: D \to \R$ continue. Soient $g: I \to \R$ et $h: I \to \R$ continues telles que \[
			\forall t \in I, \big(g(t), h(t)\big) \in D.
		\] Alors \[
			t \in I \mapsto f\big(g(t), h(t)\big) \in \R
		\] est continue.
	\end{enumerate}
\end{prop}

\begin{figure}[H]
	\centering
	\begin{asy}
		import three;
		import graph3;
		size(5cm);

		settings.render = 0;
		settings.prc = false;
		currentprojection = obliqueX;

		draw(O -- X, Arrow3(TeXHead2));
		draw(O -- Y, Arrow3(TeXHead2));
		draw(O -- Z, Arrow3(TeXHead2));

		triple f(real x, real y, real z = 0) { return (x,y,cos(x - 0.5) * cos(y - 0.5)/1.2 + 0.15); }

		real inc = 1 / 5;

		for(real x = 0; x <= 1; x += inc) {
			draw(graph(
				new real(real t) { return x; }, // x
				new real(real y) { return y; }, // y
				new real(real y) { return f(x,y).z; }, // z
				0, 1
			), gray);
		}

		for(real y = 0; y <= 1; y += inc) {
			draw(graph(
				new real(real x) { return x; }, // x
				new real(real t) { return y; }, // y
				new real(real x) { return f(x,y).z; }, // z
				0, 1
			), gray);
		}

		path3 path1 = (0.3, 0.2, 0) .. (0.5, 0.5, 0) .. (0.6, 0.7, 0) .. (0.9, 0.8, 0);
		path3 path2 = (0.3, 0.8, 0) .. (0.5, 0.5, 0) .. (0.6, 0.3, 0) .. (0.9, 0.2, 0);
		path3 pathA = f(0.3, 0.2, 0) .. f(0.5, 0.5, 0) .. f(0.6, 0.7, 0) .. f(0.9, 0.8, 0);
		path3 pathB = f(0.3, 0.8, 0) .. f(0.5, 0.5, 0) .. f(0.6, 0.3, 0) .. f(0.9, 0.2, 0);

		draw(path1, red, Arrow3(TeXHead2, position=0.5));
		draw(pathA, red, Arrow3(TeXHead2, position=0.5));
		draw(path2, deepcyan, Arrow3(TeXHead2, position=0.3));
		draw(pathB, deepcyan, Arrow3(TeXHead2, position=0.3));

		dot((0.5, 0.5, 0));
		dot(f(0.5, 0.5, 0));
		draw((0.5, 0.5, 0) -- f(0.5, 0.5, 0), dashed);
	\end{asy}
\end{figure}



	\chap[14]{Continuité}
	\renewcommand{\cwd}{../chap14}
	\part{Topologie de $\R^2$}

\begin{defn}
	La \underline{norme (euclidienne)} de $\R^2$ est l'application définie par \[
		\forall (x,y) \in \R^2, \|(x,y)\| = \sqrt{x^2 + y^2}.
	\]

	\begin{figure}[H]
		\centering
		\begin{asy}
			import graph;
			axes(EndArrow);
			size(4cm);
			pair A = (3,2);
			dot(A);
			draw((3,0)--A, dashed);
			draw((0,2)--A, dashed);
			label("$x$", (3,0), align=S);
			label("$y$", (0,2), align=W);
			draw((0,0)--A);
			dot((4,3), white+0);
		\end{asy}
	\end{figure}
	\index{norme (de $\R^2$)}
	\index{norme euclidienne (de $\R^2$)}
\end{defn}

\begin{prop}
	La norme euclidienne vérifie:
	\begin{enumerate}
		\item (séparation) \[
			\forall (x,y) \in \R^2, \|(x,y)\| = 0 \iff x = y = 0,
			\]
		\item (homogénéité positive) \[
				\forall \lambda \in \R, \forall (x,y) \in \R^2, \|\lambda(x,y)\|= \left| \lambda \right| \|(x,y)\|
			\]
		\item (inégalité triangulaire) \[
			\forall (x,y), (a,b) \in \R^2,
			\|(x,y)+(a,b)\|\le \|(x,y)\|+\|(a,b)\|.
		\]
	\end{enumerate}
\end{prop}

\begin{prv}
	Déjà vue en replaçant $(x,y)$ par $x+iy \in \C$ et $\|(x,y)\|$ par |x+iy|
\end{prv}

\begin{defn}
	Soit $(a,b) \in \R^2$ et $r \in \R_+$.

	La \underline{boule ouverte} (ou \underline{disque ouvert}) de centre $(a,b)$ et de rayon $r$ est \[
		B_{(a,b)}(r) = \big\{ (x,y) \in \R^2  \mid \|(x,y) - (a,b)\| < r \big\}.
	\]

	La \underline{boule fermée} (ou \underline{disque fermé}) de centre $(a,b)$ et de rayon $r$ est \[
		\overline{B_{(a,b)}}(r) = \big\{ (x,y)\in \R^2  \mid \|(x,y) - (a,b)\| \le r \big\}.
	\]

	La \underline{sphère} (ou \underline{boule}) de centre $(a,b)$ et de rayon $r$ est \[
		S_{(a,b)}(r) = \partial \overline{B_{(a,b)}}(r) = \big\{ (x,y) \in \R^2  \mid \|(x,y) - (a,b)\| = r \big\}.
	\]
	\index{boule ouverte (de $\R^2$)}
	\index{disque ouverte (de $\R^2$)}
	\index{boule fermée (de $\R^2$)}
	\index{disque fermée (de $\R^2$)}
	\index{boule (de $\R^2$)}
	\index{sphère (de $\R^2$)}
\end{defn}

\begin{figure}[H]
		\centering
		\incfig{boule}
\end{figure}

\begin{rmk}
	On parle de boule en dimension quelconque.
\end{rmk}

\begin{defn}
	Une \underline{partie ouverte} $O$ de $\R^2$ (ou \underline{un ouvert}) si \[
		\forall (x,y) \in O, \exists r > 0, B_{(a,b)}(r) \subset O.
	\]
	Une partie $F$ est \underline{fermée} su $\R^2\setminus F$ est ouverte.
	\index{partie ouverte (de $\R^2$)}
	\index{ouvert (de $\R^2$)}
	\index{partie fermée (de $\R^2$)}
\end{defn}

\begin{figure}[H]
	\centering
	\incfig{partie-ouverte}
\end{figure}

\begin{prop}
	Une boule ouverte est ouverte. Une boule fermée est fermée.
\end{prop}

\begin{figure}[H]
	\centering
	\begin{subfigure}{4cm}
		\centering
		\begin{asy}
			import patterns;

			pair n(pair a) {return a / length(a);}

			add("hatch",hatch(2mm, SW, red));
			size(4cm);

			draw(circle((0,0), 1));
			dot('$(a_0, b_0)$', (0,0), align=S);

			draw((0,0) -- n((-1, 1)), dashed);
			label("$r$", n((-1, 1)) / 2, align=1.5S);

			pair A = n((1,3)) * (2/3);
			real rho = (1 - length(A)) * (2 / 3);

			dot("$(a,b)$", A, red, align=3SE);
			filldraw(circle(A, rho), pattern("hatch"), red);

			label("$O$", n((1,-1))*2.5/3);
		\end{asy}
	\end{subfigure}
	\begin{subfigure}{1cm}
		\centering~\\
	\end{subfigure}
	\begin{subfigure}{5cm}
		\centering
		\begin{asy}
			import patterns;

			pair n(pair a) {return a / length(a);}

			add("hatch",hatch(1mm, SW, red));
			add("hatch2",hatch(3mm, SE, blue));
			size(5cm);

			guide around = (-1.5, -1.5) -- (-1.5, 1.5) -- (2.5, 1.5) -- (2.5, -1.5) -- cycle;

			pair A = n((3, 1)) * 5/3; 
			real rho = 2 / 9;

			picture inter;
			fill(inter, around, pattern("hatch2"));
			fill(inter, circle((0,0), 1), white);
			add(inter);

			draw(circle((0,0), 1));
			dot('$(a_0, b_0)$', (0,0), align=S);

			draw((0,0) -- n((-1, 1)), dashed);
			label("$r$", n((-1, 1)) / 2, align=1.5S);

			dot("$(a,b)$", A, red, align=2SE);
			filldraw(circle(A, rho), pattern("hatch"), red);

			label("$F$", n((1,-1))*2.5/3);
		\end{asy}
	\end{subfigure}
\end{figure}

\begin{prv}
	$\O$ est un ouvert.

	Soit $B$ la boule ouverte de centre $(a_0, b_0) \in \R^2$ et de rayon $r > 0$.

	On pose $\rho = \frac{1}{2}\big(r - \|(a,b) - (a_0,b_0)\|\big)$.
	Montrons que \[
		B_{(a,b)}(\rho) \subset  B_{(a,b)}(r).
	\]

	Soit $(x,y) \in B_{(a,b)}(\rho)$.
	\begin{align*}
		\|(x,y) - (a_0,b_0)\|&= \|(x,y)- (a,b) + (a,b) - (a_0,b_0)\| \\
		&\le \|(x,y) - (a,b)\| + \|(a,b) - (a_0, b_0)\|\\
		&< \rho + \|(a,b) - (a_0, b_0)\| = \frac{1}{2}r + \frac{1}{2} \|(a,b) - (a_0, b_0)\|\\
		&< r
	\end{align*}
	
	Soit $F$ la boule fermée de centre $(a_0, b_0)$ et de rayon $r \ge 0$.

	Soit $(a,b) \not\in F$. On pose \[
		\rho = \frac{1}{2}\big(\|(a,b) - (a_0, b_0)\| - r\big) > 0.
	\]

	Montrons que $B_{(a,b)}(\rho) \subset \R^2\setminus F$.

	Soit $(x,y) \in B_{(a,b)}(\rho)$.

	\begin{align*}
		\|(x,y) - (a_0, b_0)\| &= \|(x,y) - (a,b) + (a,b) - (a_0, b_0)\| \\
		&\ge \big| \underbrace{\|(x,y) - (a,b)\|}_{\le \rho} - \underbrace{\|(a,b) - (a_0, b_0)\|}_{> r} \big|\\
		&\ge \|(a,b) - (a_0, b_0)\|- \|(x,y) - (a,b)\|\\
		&> \|(a,b) - (a_0, b_0)\|- \rho\\
		&> \frac{1}{2} \|(a,b) - (a_0, b_0)\| + \frac{1}{2}r\\
		&> r
	\end{align*}

	donc $(x,y) \not\in F$.
\end{prv}

\begin{exm}
	\begin{enumerate}
		\item $\O$ est ouvert.\\
			$\R^2$ est ouvert.
		\item $\O$ est fermé.\\
			$\R^2$ est fermé.\\
		\item $\big\{(x,0)  \mid x > 0\big\}$ n'est ni ouverte ni fermé.
	\end{enumerate}
\end{exm}

\begin{figure}[H]
	\centering
	\begin{asy}
		size(3cm);

		draw((0, -1) -- (0, 3), Arrow(TeXHead));
		draw((-1, 0) -- (3, 0), Arrow(TeXHead));
		
		draw((0,0) -- (0, 2.97), red);
		draw(circle((0,1.5), 0.5), deepred);
		draw(circle((0,0.5), 0.1), deepred);
	\end{asy}
\end{figure}

\begin{defn}
	Soit $(a,b) \in \R^2$ et $V \in \mathcal{P}(\R^2)$.

	On dit que $V$ est un voisinage de $(a,b)$ s'il existe $r > 0$ tel que \[
		B_{(a,b)}(r) \subset V.
	\]
	\index{voisinage (dans $\R^2$)}
\end{defn}

\begin{prop}
	Un ouvert non vide est un voisinage en chacun de ces points. \qed
\end{prop}

\begin{defn}
	Soit $D \subset \R^2$. Un \underline{point intérieur} de $D$ est un couple $(a,b) \in D$ tel que \[
		\exists r > 0, B_{(a,b)}(r) \subset D.
	\] en d'autres termes, si $D$ est un voisinage de $(a,b)$.

	On note $\mathring D$ l'ensemble des points intérieurs à $D$. C'est \underline{l'intérieur} de $D$.
	\index{point intérieur (dans $\R^2$)}
	\index{intérieur (dans $\R^2$)}
\end{defn}

\begin{prop}
	$\mathring D$ est le plus grand ouvert $O$ de $\R^2$ tel que $O \subset D$.
\end{prop}

\begin{figure}[H]
	\centering
	\incfig{interieur-plus-grand-ouvert}
\end{figure}


\begin{prv}
	Soit $(a,b) \in \mathring D$.

	Par définition, il existe $r > 0$ tel que \[
		B_{(a,b)}(r) \subset D.
	\] Montrons que $B_{(a,b)}(r) \subset \mathring D$.

	Soit $(x,y) \in B_{(a,b)}(r)$. Comme $B_{(a,b)}(r)$ est un ouvert de $\R^2$, il existe $\rho > 0$ tel que \[
		B_{(x,y)}(\rho) \subset B_{(a,b)}(r)
	\] donc $(x,y) \in \mathring D$.

	Donc $\mathring D$ est ouvert, $\mathring D \subset D$.

	Soit $O$ un ouvert de $\R^2$ tel que $O \subset D$. Montrons que $O \subset \mathring D$.

	Soit $(x,y) \in O$. Soit $r > 0$ tel que \[
		B_{(x,y)}(r) \subset O \subset D
	\] donc $(x,y) \in \mathring D$.
\end{prv}

\begin{defn}
	Soit $f: D \subset \R^2 \to \R$, $\ell \in \R$, $(a,b) \in \mathring D$.

	On dit que \underline{$f(x,y)$ tend vers $\ell$ quand $(x,y)$ tend vers $(a,b)$} ou que $\ell$ est \underline{une limite} de $f$ en $(a,b)$ si \[
		\forall \varepsilon > 0, \exists r > 0, \forall (x,y) \in D, \|(x,y) - (a,b)\| < r \implies \left| f(x,y) - \ell \right| \le \varepsilon.
	\] en d'autres termes si \[
		\forall V \in \mathcal{V}_{\ell}, \exists W \in \mathcal{V}_{(a,b)}, \forall (x,y) \in W \cap D, f(x,y) \in V.
	\]
	\index{limite (dans $\R^2$)}
	\index{tendre vers (dans $\R^2$)}
\end{defn}

\begin{prop}
	[unicité de la limite]
	Soit $f: D \to \R$, $(a,b) \in \mathring D$, $\ell_1, \ell_2 \in \R$ telles que $\ell_1$ et $\ell_2$ sont des limites de $f$ en $(a,b)$.

	Alors $\ell_1 = \ell_2$.
\end{prop}

\begin{figure}[H]
	\centering
	\incfig{preuve-unicité-de-la-limite}
\end{figure}

\begin{prv}
	On suppose $\ell_1 < \ell_2$. On pose $\varepsilon = \frac{\ell_2 - \ell_1}{2} > 0$.

	Soit $r_1 > 0$ tel que \[
		f\big(B_{(a,b)}(r_1)\big) \subset ]\ell_1 - \varepsilon, \ell_1 + \varepsilon[.
	\] Soit $r_2 > 0$ tel que \[
		f\big(B_{(a,b)}(r_2)\big) \subset ]\ell_2 - \varepsilon, \ell_2 + \varepsilon [.
	\] On pose $r = \min(r_1, r_2)$ donc \[
		B_{(a,b)}(r_1) \cap B_{(a,b)}(r_2) = B_{(a,b)}(r) \neq \O.
	\] Soit $(x,y) \in B_{(a,b)}(r)$. Alors, \[
		f(x,y) \in ]\ell_1 - \varepsilon, \ell_1 + \varepsilon[ \cap ]\ell_2 - \varepsilon, \ell_2 + \varepsilon[ = \O.
	\] $\lightning$
\end{prv}

\begin{defn}
	Soit $f : D \to \R$, $(a,b) \in \mathring D$.

	On dit que $f$ est \underline{continue} en $(a,b)$ si \[
		f(x,y) \tendsto{(x,y) \to (a,b)}f(a,b).
	\]
	\index{continuité (dans $\R^2$)}
\end{defn}

\begin{prop}
	\underline{Si} $f(x,y) \tendsto{(x,y) \to (a,b)} \ell$ \\
	\underline{alors} $\begin{cases}
		f(x,b) \tendsto{x \to a} \ell\\
		f(a,y) \tendsto{y \to b} \ell.\\
	\end{cases}$
\end{prop}

\begin{prv}~\\
	\begin{figure}[H]
		\centering
		\incfig{limite-x-en-a-et-y-en-b}
	\end{figure}
\end{prv}

\underline{Contre-exemple} : exercice 3.

\begin{exm}
	\begin{enumerate}
		\item $f : \begin{array}{rcl}
				\R^2 &\longrightarrow& \R \\
				(x,y) &\longmapsto& x
			\end{array}$ limite en $(0,0)$ ?

			Soit $\varepsilon > 0$. On pose $r = \varepsilon$. \[
				\forall (x,y) \in B_{(0,0)}(r),
				\left| f(x,y) \right| = \left| x \right| \le \|(x,y)\| < r = \varepsilon
			\] Donc $f(x,y) \tendsto{(x,y) \to (a,b)} 0$.
		\item limite $f : \begin{array}{rcl}
				\R^2 &\longrightarrow& \R \\
				(x,y) &\longmapsto& x^3
			\end{array}$ en $(0,0)$ ?

			Soit $\varepsilon > 0$. On pose $r = \sqrt[3]{r} > 0$. \[
				\forall (x,y) \in B_{(0,0)}(r),
				\left| f(x,y) \right| = \left| x^3 \right| \le \|(x,y)\|^3 < r^3 = \varepsilon.
			\]
		\item limite de $f : \begin{array}{rcl}
			\R^2 &\longrightarrow& \R \\
			(x,y) &\longmapsto& x^3y^2
		\end{array}$ en $(0,0)$ ?

		Soit $\varepsilon > 0$. On pose $r = \sqrt[5]{\varepsilon} > 0$. \[
			\forall (x,y) \in B_{(0,0)}(r), \left| f(x,y) \right| = \left| x^3 y^2 \right| \le \|(x,y)\|^3 \|(x,y)\|^2 < r^5 = \varepsilon.
		\]
	\end{enumerate}
\end{exm}

\begin{defn}
	Soient $D \subset \R^2$ et $(x,y) \in \R^2$.

	\begin{figure}[H]
    \centering
    \incfig{point-adhérent}
	\end{figure}
	
	On dit que $(x,y)$ est \underline{adhérent} à $D$ si \[
		\forall r > 0, B_{(x,y)}(r) \cap D \neq \O.
	\] L'ensemble des points adhérents à $D$ est noté $\overline{D}$. On dit que $\overline{D}$ est \underline{l'adhérence} de $D$.
	\index{point adhérent (dans $\R^2$)}
	\index{adhérent (dans $\R^2$)}
\end{defn}

\begin{defn}
	Soit $f: D \subset \R^2 \to \R$ et $(a,b) \in \overline{D}$, $\ell \in \R$. On dit que $f$ tend vers $\ell$ quand $(x,y)$ tend vers $(a,b)$ si \[
		\forall \varepsilon > 0, \exists r > 0, \forall (x,y) \in B_{(a,b)}(r) \cap D,
		\left| f(x,y) - \ell \right| \le \varepsilon.
	\]
	\index{limite (dans $\R^2$)}
	\index{tendre vers (dans $\R^2$)}
\end{defn}

\begin{prop}
	\begin{enumerate}
		\item Dans ce contexte, il y a unicité de la limite
		\item La limite d'une somme, d'un produit, d'un quotien, d'une composée se comporte comme dans le cas d'une seule variable.
		\item Soit $f: D \to \R$ continue. Soient $g: I \to \R$ et $h: I \to \R$ continues telles que \[
			\forall t \in I, \big(g(t), h(t)\big) \in D.
		\] Alors \[
			t \in I \mapsto f\big(g(t), h(t)\big) \in \R
		\] est continue.
	\end{enumerate}
\end{prop}

\begin{figure}[H]
	\centering
	\begin{asy}
		import three;
		import graph3;
		size(5cm);

		settings.render = 0;
		settings.prc = false;
		currentprojection = obliqueX;

		draw(O -- X, Arrow3(TeXHead2));
		draw(O -- Y, Arrow3(TeXHead2));
		draw(O -- Z, Arrow3(TeXHead2));

		triple f(real x, real y, real z = 0) { return (x,y,cos(x - 0.5) * cos(y - 0.5)/1.2 + 0.15); }

		real inc = 1 / 5;

		for(real x = 0; x <= 1; x += inc) {
			draw(graph(
				new real(real t) { return x; }, // x
				new real(real y) { return y; }, // y
				new real(real y) { return f(x,y).z; }, // z
				0, 1
			), gray);
		}

		for(real y = 0; y <= 1; y += inc) {
			draw(graph(
				new real(real x) { return x; }, // x
				new real(real t) { return y; }, // y
				new real(real x) { return f(x,y).z; }, // z
				0, 1
			), gray);
		}

		path3 path1 = (0.3, 0.2, 0) .. (0.5, 0.5, 0) .. (0.6, 0.7, 0) .. (0.9, 0.8, 0);
		path3 path2 = (0.3, 0.8, 0) .. (0.5, 0.5, 0) .. (0.6, 0.3, 0) .. (0.9, 0.2, 0);
		path3 pathA = f(0.3, 0.2, 0) .. f(0.5, 0.5, 0) .. f(0.6, 0.7, 0) .. f(0.9, 0.8, 0);
		path3 pathB = f(0.3, 0.8, 0) .. f(0.5, 0.5, 0) .. f(0.6, 0.3, 0) .. f(0.9, 0.2, 0);

		draw(path1, red, Arrow3(TeXHead2, position=0.5));
		draw(pathA, red, Arrow3(TeXHead2, position=0.5));
		draw(path2, deepcyan, Arrow3(TeXHead2, position=0.3));
		draw(pathB, deepcyan, Arrow3(TeXHead2, position=0.3));

		dot((0.5, 0.5, 0));
		dot(f(0.5, 0.5, 0));
		draw((0.5, 0.5, 0) -- f(0.5, 0.5, 0), dashed);
	\end{asy}
\end{figure}


	\part{Familles orthogonales}

\begin{thm}[Pythagore]
	Soit $(x,y) \in E^2$. \[
		\|x+y\|^2 = \|x\|^2 + \|y\|^2 \iff x \perp y
	.\]
	\begin{figure}[H]
		\centering
		\begin{asy}
			size(4cm);
			pair u = (1, 0.5);
			pair v = 1.5 * (0, 1) * u;
			draw((0,0)--u, Arrow(TeXHead));
			label("$x$", u/2, align=S);
			draw(u--v+u, Arrow(TeXHead));
			label("$y$", u + v/2, align=NE);
			draw((0,0) -- u + v, Arrow(TeXHead));
			draw(u + v / 7.5 -- u + v / 7.5 - u / 5 -- u - u / 5 -- u -- cycle);
		\end{asy}
	\end{figure}
\end{thm}

\begin{prv}
	\[
		\|x + y\|^2 = \|x\|^2 + \|y\|^2 \iff 2\left<x \mid y \right> = 0 \iff x \perp y
	.\]
\end{prv}

\begin{defn}
	Soit $(e_i)_{i\in I}$ une famille de vecteurs. On dit que cette famille est \underline{orthogonale} si \[
		\forall i \neq j\, e_i \perp e_j
	.\] Si, en plus, on a \[
		\forall i \in I,\,\|e_i\| = 1,
	\] alors on dit que la famille est \underline{orthonormale} ou \underline{orthonormée}.
	\index{famille orthogonale}
	\index{famille orthonormale}
	\index{famille orthonormée}
\end{defn}

\begin{prop}[Pythagore]
	Soit $(e_1, \ldots, e_n)$ une famille orthogonale. Alors \[
		\left\| \sum_{i=1}^n e_i \right\|^2 = \sum_{i=1}^n \|e_i\|^2
	.\]
\end{prop}

\begin{thm}
	Toute famille orthogonale de vecteurs non nuls est libre.
\end{thm}

\begin{prv}
	Soit $(e_i)_{i\in I}$ une famille orthogonale telle que \[
		\forall i \in I,\,e_i \neq 0_E
	.\] Soit $n \in \N^*$, $(\lambda_1, \ldots, \lambda_n) \in \R^n$. On suppose \[
		\sum_{k=1}^n \lambda_k e_{i_k} = 0_E
	.\] Soit $j \in \left\llbracket 1,n \right\rrbracket$.
	\begin{align*}
		0 &= \left<\sum_{k=1}^n \lambda_k e_{i_k}  \mid e_{i_j} \right>\\
		&= \sum_{k=1}^n \lambda_k \left<e_{i_k}  \mid e_{i_j} \right> \\
		&= \lambda_j \underbrace{\|e_{i_j}\|^2}_{\neq 0} \\
	\end{align*}
	donc $\lambda_j = 0$.
\end{prv}

\begin{algo}[Orthonormalisation de Gran--Schmidt]
	On suppose $E$ de dimension finie. Soit $\mathcal{B} = (e_1, \ldots, e_n)$ une base de $E$.

	\begin{itemize}
		\item\underline{\it Étape 1}: On pose $v_1 = \frac{e_1}{\|e_1\|}$ de sorte que $\|v_1\| = 1$.
		\item\underline{\it Étape 2} : On pose \[
				u_2 = e_2 - \left<e_2  \mid v_1 \right> v_1
			.\] Ainsi,
			\begin{align*}
				\left<u_2 \mid v_1 \right> &= \big<e_2 - \left<e_2 \mid v_1 \right> v_1  \mid v_1 \big>\\
				&= \left<e_2 \mid v_1 \right> - \left<e_2 \mid v_1 \right> \left<v_1 \mid v_1 \right> \\
				&= 0. \\
			\end{align*}
			On pose $v_2 = \frac{u_2}{\|u_2\|}$ donc $v_2 \perp v_1$ et $\|v_2\| = 1$.
		\item\underline{\it Étape 3} : On pose \[
				u_2 = e_3 - \left<e_3 \mid v_1 \right>v_1 - \left<e_3 \mid v_2 \right>v_2
			.\] Ainsi,
			\begin{align*}
				\left<u_3  \mid v_1 \right> &= \left<e_3  \mid v_1 \right> - \left<e_3 \mid v_1 \right>\underbrace{\left<v_1 \mid v_1 \right>}_{=1} - \left<e_3 \mid v_2 \right>\underbrace{\left<v_2 \mid v_1 \right>}_{=0} \\
				&= 0 \\
			\end{align*}
			et 
			\begin{align*}
				\left<u_3 \mid v_2 \right> &= \left<e_3  \mid  v_2 \right> - \left<e_3 \mid v_1 \right> \underbrace{\left<v_1 \mid v_2 \right>}_{=0} - \left<e_3 \mid v_2 \right> \underbrace{\left<v_2 \mid v_2 \right>}_{=1}\\
				&= 0. \\
			\end{align*}
			On pose $v_3 = \frac{u_3}{\|u_3\|}$ de sorte que $v_3 \perp v_1$, $v_3 \perp v_2$ et $\|v_3\| = 1$.
		\item\underline{\it Étape $i+1$}: On pose \[
			u_{i+1} = e_{i+1} - \sum_{k=1}^i \left<e_{i+1}  \mid v_k \right> v_k
		.\] Ainsi, pour tout $j \in \left\llbracket 1,i \right\rrbracket,$ on a
		\begin{align*}
			\left<u_{i+1}  \mid v_j \right> &= \left<e_{i+1}  \mid v_j \right> - \sum_{k=1}^i \left<e_{i+1} \mid v_k \right> \left<v_k \mid v_j \right> \\
			&= \left<e_{i+1} \mid v_j \right> - \left<e_{i+1} \mid v_j \right> \|v_j\|^2 \\
			&= 0. \\
		\end{align*}
		On pose $v_{i+1} = \frac{u_{i+1}}{\|u_{i+1}\|}$.
	\end{itemize}
\end{algo}

\begin{exm}
	Avec $E = \R_3[X]$, $\left<P \mid Q \right> = \int_{0}^{1} P(t)\,Q(t)~\mathrm{d}t$ et $\mathcal{B} = (1, X, X^2, X^3)$.
	\begin{enumerate}
		\item $\|1\|^2 = \left<1 \mid 1 \right> = \int_{0}^{1} 1~\mathrm{d}t = 1$ et donc $v_1 = 1$.
		\item $u_2 = X - \left<X  \mid v_1 \right>v_1$. Or, $\left<X \mid v_1 \right> = \int_{0}^{1} t~\mathrm{d}t = \frac{1}{2}$. D'où $u_2 = X - \frac{1}{2}$.
			\begin{align*}
				\|u_2\|^2 &= \int_{0}^{1} \left( t - \frac{1}{2} \right)^2~\mathrm{d}t \\
				&= \int_{0}^{1} \left( t^2 - t + \frac{1}{4} \right)~\mathrm{d}t \\
				&= \frac{1}{3} - \frac{1}{2} + \frac{1}{4} \\
				&= \frac{1}{12} \\
			\end{align*} On en déduit que $v_2 = \sqrt{12}\left( X - \frac{1}{2} \right)$.
		\item $u_3 = X^2 - \left<X^2 \mid v_1 \right>v_1 - \left<X^2 \mid v_2 \right>v_2$.
			On a \[
				\left<X^2 \mid v_1 \right> = \int_{0}^{1} t^2~\mathrm{d}t = \frac{1}{3}
			\] et
			\begin{align*}
				\left<X^2 \mid v_2 \right> &= \sqrt{12} \int_{0}^{1} t^2\left( t - \frac{1}{2} \right)~\mathrm{d}t \\
				&= \frac{\sqrt{12}}{12}. \\
			\end{align*}
			D'où
			\begin{align*}
				u_3 &= X^2 - \frac{1}{3} - \frac{\sqrt{12}}{12}\sqrt{12} \left( X - \frac{1}{2} \right)\\
				&= X^2 - \frac{1}{3} - X + \frac{1}{2} \\
				&= X^2 - X + \frac{1}{6}. \\
			\end{align*}
			\begin{align*}
				\|u_3\|^2 &= \int_{0}^{1} \left( t^2 - t + \frac{1}{6} \right)~\mathrm{d}t\\
				&= \int_{0}^{1} \left( t^4 + t^2 + \frac{1}{36} - 2t^3 + \frac{t^2}{3} - \frac{t}{3} \right) ~\mathrm{d}t \\
				&= \frac{1}{5} + \frac{1}{3} + \frac{1}{36} - \frac{1}{2} + \frac{1}{9} - \frac{1}{6} \\
				&= \frac{36 + 60 + 5 - 90 + 20 - 30}{180} \\
				&= \frac{1}{180} \\
			\end{align*}
			On en déduit que \[
				v_3 = 6\sqrt{5}\left( X^2 - X + \frac{1}{6} \right).
			\]
		\item Exercice : calculer $v_4$.
	\end{enumerate}
\end{exm}

\begin{prop}
	Soit $\mathcal{B} = (e_1, \ldots, e_n)$ une base de $E$ et $\mathcal{C}$ la base obtenue par le procédé d'orthonormalisation de Gram--Schmidt. Alors, \[
		\forall i \in \left\llbracket 1,n \right\rrbracket,\,\Vect(e_1,\ldots, e_i) = \Vect(v_1, \ldots, v_i)
	.\]\qed
\end{prop}

\begin{exm}[orthogonalisation]
	\begin{itemize}
		\item $u_1 = 1$.
		\item
			\begin{align*}
				\begin{rcases*}
					u_2 \in \Vect(e_1, e_2)\\
					u_2 \perp u_1
				\end{rcases*}
				\iff& \begin{cases}
					u_2 = ae_1 + be_2\quad (a,b) \in \R^2\\
					\left<u_1 \mid u_2 \right> = 0
				\end{cases}\\
				\iff& \begin{cases}
					u_2 = a + bX\\
					\int_{0}^{1} (a+bt)~\mathrm{d}t = 0.
				\end{cases}\\
			\end{align*}
			\begin{align*}
				\int_{0}^{1} (a+bt)~\mathrm{d}t = 0 \iff& a + \frac{b}{2} = 0\\
				\iff& a = -\frac{b}{2}\\
				\iff& u_2 = -\frac{b}{2} + bX.
			\end{align*}
			Par exemple, $u_2 = -1 + 2X$.
		\item $\begin{cases}
				u_3 \in \Vect(e_1, e_2, e_3)\\
				u_3 \perp u_1\\
				u_3 \perp u_2
			\end{cases}$

			On pose $u_3 = a + bX + cX^2$ avec $(a,b,c)\in \R^3$.
			\begin{align*}
				\begin{rcases*}
					\int_{0}^{1} \left( a+bt + ct^2 \right)~\mathrm{d}t = 0\\
					\int_{0}^{1} \left(a + bt+ct^2\right)(2t - 1)~\mathrm{d}t = 0
				\end{rcases*} \iff& \begin{cases}
					a + \frac{b}{2} + \frac{c}{3} = 0\\
					\int_{0}^{1} \left( 2ct^3 + (-c + 2b)t^2 + (2a - b)t - a \right) ~\mathrm{d}t = 0
				\end{cases}\\
				\iff& \begin{cases}
					a + \frac{b}{2} + \frac{c}{3} = 0\\
					\frac{c}{2} + \frac{2b - c}{3} + \frac{2\cancel{a} - b}{2} - \cancel{a} = 0
				\end{cases}\\
				\iff&  \begin{cases}
					a = -\frac{b}{2} - \frac{c}{3} = \frac{c}{2} - \frac{c}{3} = \frac{c}{6}\\
					b = -c.
				\end{cases}
			\end{align*}
			On en déduit que \[
				u_3 = 1 - 6X + 6X^2
			.\]
	\end{itemize}
\end{exm}

\begin{crlr}[théorème de la base orthonormée incomplète] Soit $(e_1, \ldots, e_k)$ une base orthonormée d'un espace euclidien. On peut trouver $e_{k+1},\ldots,e_n$ tels que $(e_1, \ldots, e_k, e_{k+1},\ldots,e_n)$ soit une base orthonormée de $E$.
\end{crlr}

\begin{prv}
	On sait que $(e_1, \ldots, e_k)$ est libre. On complète $(e_1, \ldots, e_k)$ en une base $\mathcal{B}$ de $E$. On orthonormalise $\mathcal{B}$ : on obtient une base orthonormée $\mathcal{C}$ de $E$. En détaillant l'algorithme de Gram--Schmidt, on s'aper\c coit que les $k$ premiers vecteurs de $\mathcal{C}$ sont ceux de $\mathcal{B}$.
\end{prv}

\begin{thm}
	Soit $E$ un espace euclidien et $\mathcal{B} = (e_1, \ldots, e_n)$ une base orthonormée de $E$. Soit $(x,y) \in E^2$. On pose $(x_1, \ldots, x_n) \in \R^n$ et $(y_1, \ldots, y_n) \in \R^n$ tels que \[
		x = \sum_{i=1}^n x_i e_i \qquad\qquad y = \sum_{i=1}^n y_i e_i
	.\] Alors \[
		\left<x \mid y \right> = \sum_{i=1}^n x_i y_i
	.\]
	\vspace{3mm}
	Soit $X = \mat{x_1\\\vdots\\x_n}$ et $Y = \mat{y_1\\ \vdots \\ y_n}$. Alors, \[
		\left<x \mid y \right> = X^\T\,Y
	.\]
\end{thm}

\begin{prv}
	\begin{align*}
		\left<x \mid y \right> &= \left<\sum_{i=1}^n x_ie_i  \mid y \right>\\
		&= \sum_{i=1}^n x_i \left<e_i  \mid y \right> \\
		&= \sum_{i=1}^n x_i \left<e_i  \mid \sum_{j=1}^n y_j e_j \right> \\
		&= \sum_{i=1}^n x_i \sum_{j=1}^n y_j \underbrace{\left<e_i \mid e_j \right>}_{\delta_i^j} \\
		&= \sum_{i=1}^n x_i y_i. \\
	\end{align*}
\end{prv}

\begin{prop}
	Soit $E$ un espace euclidien et $\mathcal{B} = (e_1, \ldots, e_n)$ une base orthonormée de $E$. Alors, \[
		\forall x \in E,\,x = \sum_{i=1}^n \left<x \mid e_i \right>e_i
	.\]
\end{prop}

\begin{prv}
	Soit $x \in E$. On pose \[
		x = \sum_{i=1}^n x_i e_i
	\] avec $(x_1, \ldots, x_n) \in \R^n$. Soit $j \in \left\llbracket 1,n \right\rrbracket$. On a
	\begin{align*}
		\left<x \mid e_j \right> &= \left<\sum_{i=1}^n x_i e_i  \mid e_j \right>\\
		&= \sum_{i=1}^n x_i \left<e_i \mid e_j \right> \\
		&= x_j. \\
	\end{align*}
\end{prv}

	\part{Lois de composition}

\begin{defn}
	Une \underline{loi de composition interne} \index{loi de composition interne} est une application $f$ de $E \times E$ dans $E$.
	
	On la note $x * y$ au lieu de $f(x,y)$ (on est libre de choisir le symbôle).
\end{defn}

\begin{defn}
	Soit $E$ un ensemble muni d'une loi de composition interne $\boxtimes$.

	On dit que $\boxtimes$ est \underline{associative} \index{associativité (loi de composition interne)} si \[
		\forall (x,y,z) \in E^3,\;(x\boxtimes y)\boxtimes z = x \boxtimes (y \boxtimes z).
	\] Dans ce cas, on écrit plutôt $x \boxtimes y \boxtimes z$.
\end{defn}

\begin{exm}
	\begin{itemize}
		\item $+$ et $\times $ dans $\C$ sont associatives;
		\item $ \circ$ est associative;
		\item  la multiplication matricielle est aussi associative.
	\end{itemize}
\end{exm}

\begin{defn}
	On dit que $\boxtimes$ est \underline{commutative} \index{commutativité (loi de composition interne)} si \[
		\forall (x,y) \in E^2, x\boxtimes y = y\boxtimes x.
	\]
\end{defn}

\begin{exm}
	\begin{itemize}
		\item $+$ et $\times $ dans $\C$ sont commuatives;
		\item $ \circ $ n'est pas commutative;
		\item  la multiplication matricielle n'est pas commutative.
	\end{itemize}
\end{exm}

\begin{defn}
	Soit $e \in E$. On dit que $e$ est un
	\begin{itemize}
		\item \underline{élément neutre à gauche}\index{élément neutre à gauche (loi de composition interne)} si  \[
				\forall x \in E,\; e\boxtimes x = x;
			\]
		\item \underline{élément neutre à droite}\index{élément neutre à droite (loi de composition interne)} si  \[
				\forall x \in E,\; x\boxtimes e = x;
			\]
		\item \underline{élément neutre}\index{élément neutre (loi de composition interne)} si  \[
				\forall x \in E,\; e\boxtimes x = x\boxtimes e = x.
			\]
	\end{itemize}
\end{defn}

\begin{prop}
	Sous reserve d'existence, il y a unicité de l'élément neutre.
\end{prop}

\begin{prv}
	Soient $e$ et $e'$ deux éléments neutre.
	\begin{itemize}
		\item $e \boxtimes e' = e'$ car $e$ est neutre,
		\item $e \boxtimes e' = e$ car $e'$ est neutre.
	\end{itemize} On a donc $e = e'$.
\end{prv}

\begin{axm}[axiome du choix]
	Soit $E$ un ensemble non vide. Il existe $f : \mathcal{P}(E) \setminus \{\O\} \to E$ telle que \[
		\forall A \in \mathcal{P}(E) \setminus \{\O\},\; f(A) \in A.
	\]
\end{axm}

\begin{defn}
	Soit $f: E \to F$. Le \underline{graphe} \index{graphe (application)} de $f$ est \[
		\Big\{\big(x,f(x)\big)  \mid x \in E\Big\} \subset E \times F.
	\]
\end{defn}

\begin{prop}
	Soit $G \subset E\times F$. $G$ est le graphe d'une application si et seulement si \[
		\forall x \in E,\,\exists! y \in F,\, (x,y) \in G.
	\]
\end{prop}

\begin{prv}
	\begin{itemize}
		\item[``$\implies$''] par définition d'une application
		\item[``$\impliedby$''] On pose $f(x)$ le seul élément $y$ de $F$ qui vérifie $(x,y) \in G$. Alors $f \in F^E$ et son graphe vaut $G$.
	\end{itemize}
\end{prv}

\begin{defn}
	Soit $A \in \mathcal{P}(E)$. L'\underline{indicatrice}\index{indicatrice (ensemble)} de $A$ est \begin{align*}
		\mathbbm{1}_A: E &\longrightarrow \{0,1\} \\
		x &\longmapsto \begin{cases}
			1 &\text{ si } x \in A,\\
			0 & \text{ si } x \not\in A.
		\end{cases}
	\end{align*}
\end{defn}

\begin{exm}
	\begin{enumerate}
		\item Dans $\C$, le neutre de $+$ est $0$ et le neutre de $\times$ est $1$.
		\item Dans $E^E$, le neutre de $ \circ $ est $\id_E$.
		\item Dans $\mathcal{M}_n(\C)$ (l'ensemble des matrices carrées $n \times n$ à valeurs dans $\C$), le neutre de $\times $ est $I_n$ : \[
				I_n =
				\begin{pNiceMatrix}
					1&&(0)\\
					&\Ddots&\\
					(0)&&1
				\end{pNiceMatrix}
			\] 
	\end{enumerate}
\end{exm}

\begin{defn}
	Soit $E$ un ensemble muni d'une loi de composition interne $\boxtimes$ et $x \in E$.

	\begin{enumerate}
		\item On dit que $x$ est \underline{simplifiable à gauche}\index{simplifiabilité à gauche} si \[
				\forall (y,z) \in E^2,\,(x\boxtimes y = x \boxtimes z) \implies x = z.
			\] et que $x$ est \underline{simplifiable à droite}\index{simplifiabilité à droite} si \[
				\forall (y,z) \in E^2,\,(y\boxtimes x = z \boxtimes y) \implies x = z.
			\]
		\item On dit que $x$ est \underline{symétrisable à gauche}\index{symétrisabilité à gauche} s'il exiiste $y \in E$ tel que $y\boxtimes x = e$ où $e$ est l'élément neutre de $\boxtimes$.

			De même, on dit que $x$ est \underline{symétrisable à droite}\index{symétrisabilité à droite} s'il existe $y \in E$ tel que $x \boxtimes y = e$.

			On dit que $x$ est \underline{symétrisable}\index{symétrisabilité} s'il est symétrisable à gauche et à droite, donc s'il existe $y \in E$ tel que $x \boxtimes y = y \boxtimes x = e$.
	\end{enumerate}
\end{defn}

\begin{exm}
	$E = \N$ muni de la loi $+$, tous les éléments de $E$ sont simplifiables. $0$ est le seuele élément de $E$ symétrisable.
\end{exm}

\begin{prop}
	Avec les notations précédentes, si $\boxtimes$ est associative, et $x$ est symétrisable, alors $x$ est simplifiable.
\end{prop}

\begin{prv}
	Soient $y, z \in E$.
	\begin{itemize}
		\item On suppose $x \boxtimes y = x \boxtimes z$. Soit $a \in E$ tel que $a\in E$ tel que $a \boxtimes x = e$. Alors \[
				a \boxtimes (x\boxtimes y) = a \boxtimes (x \boxtimes z).
			\] Or,
			\begin{align*}
				a \boxtimes (x \boxtimes y) &= (a \boxtimes x) \boxtimes y \\
				&= e \boxtimes y \\
				&= y. \\
			\end{align*}

			De même, $a \boxtimes (x \boxtimes z) = z$.

			Donc $y = z$.
		\item De même, si $y \boxtimes x = z \boxtimes x$, on ``multiplie'' $x$ à droite par $a$ et on obtient $y = z$.
	\end{itemize}
\end{prv}

\begin{prop-defn}
	On suppose $\boxtimes$ associative. Soit $x \in E$ symétrisable. Alors \[
		\exists ! y \in E,\; x \boxtimes y = y \boxtimes x = e.
	\] On dit que $y$ est le \underline{symétrique}\index{symétrique (loi de composition interne)} de $x$ et on le note $y = x^*$.
\end{prop-defn}

\begin{prv}
	Soeint $x,y,z \in E$ tels que \[
		\begin{cases}
			 x \boxtimes y = y \boxtimes x = e\\
			 x \boxtimes z = z \boxtimes x = e\\
		\end{cases}
	\] Alors, $x \boxtimes y = x \boxtimes z$ et, en simplifiant par $x$, on a $y = z$.
\end{prv}

\begin{exm}
	Les fonctions symétrisables de $(E^E,  \circ)$ sont les bijections et le symétrique d'une bijection est sa réciproque.
\end{exm}

\begin{rmk}
	\begin{enumerate}
		\item Si la loi est notée $+$, on parle d'\underline{opposé}\index{opposé (loi de composition interne)} plutôt que de symétrique et on le note $-x$ au lieu de $x^*$.
			L'élément neutre est noté $0_E$.
		\item Si la loi est notée $\times$, on parle d'élément \underline{inversible}\index{inversibilité (loi de composition interne)} au lieu de symétrisable, d'\underline{inverse}\index{inverse (loi de composition interne)} au lieu de symétrique et on note $x^{-1}$ au lieu de $x^*$. On note le neutre $1_E$.
	\end{enumerate}
\end{rmk}

\begin{exo}
	Soient $x,y \in E = \R^+_*$. On définit la loi de composition interne $\oplus$ : \[
		x \oplus y = \frac{1}{\frac{1}{x}\oplus \frac{1}{y}}.
	\] Cette loi peut-être utile en physique pour le calcul de résistances équivalentes en parallèles.
	\begin{itemize}
		\item {\sc Associativité} : soient $x,y,z \in E$.

			D'une part, on a \[
				x \oplus (y \oplus z) = \frac{1}{\frac{1}{x} + \frac{1}{\frac{1}{\frac{1}{x}+ \frac{1}{y}}}} = \frac{1}{\frac{1}{x}+\frac{1}{y}+\frac{1}{z}}.
			\] D'autre part, on a \[
			(x \oplus y) \oplus z = \frac{1}{\frac{1}{\frac{1}{\frac{1}{x}+\frac{1}{y}}}+\frac{1}{z}} = \frac{1}{\frac{1}{x}+ \frac{1}{y}+\frac{1}{z}}.
			\] La loi $\oplus$ est associative.
		\item {\sc Commutativité} : soient $x, y \in E$. \[
				x \oplus y = \frac{1}{\frac{1}{x}+\frac{1}{y}} = \frac{1}{\frac{1}{y}+\frac{1}{x}} = y\oplus x.
			\] Donc la loi $\oplus$ est commutative.
		\item {\sc Élément neutre} : soit $e$ l'élément neutre de $\oplus$. \[
				\forall x \in E,\; x \oplus e = e \oplus x = x.
			\] Comme la loi est commutative, seul l'égalité $x \oplus e = x$ est utile.

			Soit $x \in E$. On a donc $\frac{1}{\frac{1}{x}+\frac{1}{e}}=x$ donc $\frac{ex}{e+x}=x$ donc $ex = x(e+x)$ et donc $\cancel{ex} = \cancel{ex} + x^2$. On en déduit que $x^2 = 0$, ce qui n'est pas possible car $x \in \R^+_*$. Donc, il n'y a pas d'élément neutre pour $\oplus$.
	\end{itemize}
\end{exo}

	\part{Divers}

\begin{defn}
	Soient $E$ et $F$ deux ensembles. Un \underline{couple}\index{couple} $(x,y)$ est la donnée d'un élément $x$ de $E$ et d'un élément $y$ de $F$ où \[
		\forall x,x' \in E,\,\forall y,y' \in F,\qquad (x,y) = (x',y') \iff \begin{cases}
			x=x',\\
			y=y'.
		\end{cases}
	\] On note $E \times F$ l'ensemble des couples; c'est le \underline{produit cartésien}\index{produit cartésion (ensembles)} de $E$ et $F$.
\end{defn}

\begin{exm}
	$D \times [0,1]$ est un cylindre plein où $D$ est le disque unité fermé i.e. \[
		D = \Big\{(x,y) \in \R^2 \mid x^2+y^2 \le 1\Big\}.
	\]
	\begin{figure}[H]
		\centering
		\begin{subfigure}[b]{3cm}
			\centering
			\begin{asy}
				size(3cm);
				draw(unitcircle);
				draw((0,0)--(1,0), red);
				label("$1$",(0.5,0), red, align=S);
			\end{asy}
		\end{subfigure}
		\begin{subfigure}[b]{3cm}
			\centering
			\begin{asy}
				size(3cm);
				label("$\times\; [0,1]\; =$", (0,0), fontsize(10));
				draw(unitcircle, white+0);
			\end{asy}
		\end{subfigure}
		\begin{subfigure}[b]{3cm}
			\centering
			\begin{asy}
				import solids;
				size(3cm);
				draw(shift((0, 0.5)) * unitcircle, white+0);
				revolution r = cylinder(O, 1, 1.5, Z);
				draw(r);
				triple M = (-1/2, sqrt(3)/2, 0);
				draw((0,0,0) -- M, red);
				label("$1$", M/2, red, align=S);
				draw(M*1.1--M*1.1+(0,0,1.5), magenta, Arrows3(TeXHead2));
				label("$1$", M*1.1+(0,0,0.75), magenta, align=E);
			\end{asy}
		\end{subfigure}
	\end{figure}

	$C \times C$ où $C = \Big\{(x,y) \in \R^2  \mid x^2 + y^2 = 1\Big\}$ est un tore (creu).

	\begin{figure}[H]
		\centering
		\begin{subfigure}[b]{3cm}
			\centering
			\begin{asy}
				size(3cm);
				draw(unitcircle);
				draw((0,0)--(1,0), red);
				label("$1$",(0.5,0), red, align=S);
			\end{asy}
		\end{subfigure}
		\begin{subfigure}[b]{1cm}
			\centering
			\begin{asy}
				size(3cm);
				label("$\times$", (0,0), fontsize(10));
				dot((0.1, 1), white+0);
				dot((-0.1, -1), white+0);
			\end{asy}
		\end{subfigure}
		\begin{subfigure}[b]{3cm}
			\centering
			\begin{asy}
				size(3cm);
				draw(unitcircle);
				draw((0,0)--(1,0), red);
				label("$1$",(0.5,0), red, align=S);
			\end{asy}
		\end{subfigure}
		\begin{subfigure}[b]{1cm}
			\centering
			\begin{asy}
				size(3cm);
				label("$=$", (0,0), fontsize(10));
				dot((0.1, 1), white+0);
				dot((-0.1, -1), white+0);
			\end{asy}
		\end{subfigure}
		\begin{subfigure}[b]{3cm}
			\centering
			\begin{asy}
				import three;
				import graph3;

				size(3cm,3cm);
				surface torus = surface(Circle(c=2Y,normal=X,r=0.5,n=32), c=O, axis=Z, n=32);

				draw(torus, white + opacity(0), meshpen=black + 0.2pt, nolight, render(merge=true));
			\end{asy}
			\vspace{0.7cm}
		\end{subfigure}
	\end{figure}
\end{exm}

\begin{defn}
	Soient $E$ et $F$ deux ensembles. On dit que $E$ et $F$ sont \underline{équipotents} s'il existe une bijection de $E$ dans $F$.
	\index{équipotence (ensembles)}
\end{defn}

\begin{exm}
	\begin{enumerate}
		\item $\N$ et $\N^*$ sont équipotents car  $f : \begin{array}{rcl}
				\N &\longrightarrow& \N^* \\
				k &\longmapsto& k + 1
			\end{array}$ est bijective.
		\item $P = \{n \in \N  \mid n \text{ pair}\}$ et $I= \{n \in \N \mid n \text{ impair}\}$ sont équipotents car $f : \begin{array}{rcl}
				P &\longrightarrow& I \\
				x &\longmapsto& x+1
			\end{array}$ est bijective.
		\item $\N$ et $P$ sont équipotents car $f : \begin{array}{rcl}
				\N &\longrightarrow& P \\
				k &\longmapsto& 2k
			\end{array}$ est bijective.
		\item $[0,1]$ et $[0,1[$ sont équipotents car \begin{align*}
			f: [0,1] &\longrightarrow [0,1[ \\
			x &\longmapsto \begin{cases}
				\frac{1}{n+1} &\text{ si } x = \frac{1}{n} \text{ avec } n \in \N^*\\
				x &\text{ sinon}
			\end{cases}
		\end{align*} est bijective.
		\item De même, $]0,1[$ et $]0,1]$ sont équipotents.
		\item $]0,1[$ et $[0,1[$ sont équipotents : $f : \begin{array}{rcl}
					]0,1] &\longrightarrow& [0,1[ \\
				x &\longmapsto& 1-x
			\end{array}$ est bijective.
		\item $\forall a < b$, $[a,b]$ et $[0,1]$ sont équipotents : \begin{align*}
				f: [0,1] &\longrightarrow [a,b] \\
				\alpha &\longmapsto \alpha b + (1 - \alpha) a
			\end{align*} est bijective (interpolation linéaire).
		\item $\R$ et $]0,1[$ sont équipotents : \begin{align*}
				f: \R &\longrightarrow ]0,1[ \\
				x &\longmapsto \frac{1}{2} + \frac{\Arctan x}{\pi}
			\end{align*} est bijective.
		\item $[0,1[$ et $\N$ ne sont pas équipotents (argument de Cantor). Soit $f: \N \to [0,1[$ une bijection :
			\[
				\begin{array}{c|l}
					k&\hfill f(k)\hfill~ \\ \hline
					0&0,\hfill \!0\hfill 0\hfill 0\hfill 0\hfill\ldots\\
					1&0,\hfill a_1\hfill a_2\hfill a_3\hfill a_4\hfill\ldots\\
					2&0,\hfill b_1\hfill b_2\hfill b_3\hfill b_4\hfill\ldots\\
					\vdots&\hfill\vdots\hfill\ddots
				\end{array}
			\] On considère le nombre \[
				x = 0,\,(a_0+1)(b_1+1)(c_2+1)\cdots
			\] $f(1) \neq x$ car ils n'ont pas le même chiffre des dizaines.\\
			$f(2) \neq x$ car ils n'ont pas le même chiffre des centaines.

			Par le même raisonement, on en déduit que \[
				\forall n \in \N, f(n) \neq x
			\] donc $x$ n'a pas d'antécédant : une contradiction.
		\item On verra en exercice que $E$ et $\mathcal{P}(E)$ ne sont pas équipotents. $\R$ et $\mathcal{P}(\R)$ ne sont pas équipotents mais $\R$ et $\mathcal{P}(\N)$ le sont (développement dyadique).
		\item $\R^2$ et $\R$ sont équipotents; $\C$ et $\R$ sont équipotents.
	\end{enumerate}
\end{exm}

\begin{exo}
	Soit $E$ un ensemble. L'application \begin{align*}
		f: \mathcal{P}(E) &\longrightarrow {0,1}^E \\
		A &\longmapsto \mathbbm{1}_A
	\end{align*} est bijective.

	Soit $g : E \to \{0,1\}$.
	\begin{itemize}
		\item[\underline{\sc Analyse}] Soit $A \in \mathcal{P}(E)$ tel que $f(A) = g$. Alors $g = \mathbbm{1}_A$.
			donc  \[
				\forall x \in E,\; g(x) = \mathbbm{1}_A(x)
			\] et donc \[
				\begin{cases}
					\forall x \in A,\, g(x) = 1\\
					\forall x \in E \setminus A,\,g(x) = 0
				\end{cases}
			\] On en déduit que \[
				A = \{ x \in E  \mid  g(x) = 1\}  = g^{-1}\big(\{1\}\big).
			\]
		\item[\underline{\sc Synthèse}] On pose $A = g^{-1}\big(\{1\}\big)$. Montrons que $f(A) = g$.
			\[
				\forall x \in E,\,g(x) = \begin{cases}
					1 &\text{ si } x \in A\\
					0 &\text{ si } x \not\in A
				\end{cases} = \mathbbm{1}_A
			\] donc $g = \mathbbm{1}_A$.
	\end{itemize}

	On aurait aussi pu rédiger de la fa\c con suivante : on pose \begin{align*}
		u: \{0,1\}^E &\longrightarrow \mathcal{P}(E) \\
		g &\longmapsto g^{-1}\big(\{1\}\big).
	\end{align*} On montre que $u$ est la réciproque de $f$ : \[
		\begin{cases}
			f \circ u = \id_{\{0,1\}^E},\\
			u \circ f = \id_{\mathcal{P}(E)}.
		\end{cases}
	\]
\end{exo}

\begin{defn}
	Soit $f : E \to F$. L'\underline{image de $f$}\index{image (application)} est \[
		\mathrm{Im}(f) = f(E) = \big\{f(x) \mid x \in E\big\}.
	\]
\end{defn}

\begin{prop}
	Soit $f: E \to F$. \[
		f \text{ est surjective } \iff f(E) = F.
	\]
\end{prop}

\begin{defn}
	Une \underline{suite de $E$}\index{suite (ensemble)} est une application de $\N$ dans $E$.
\end{defn}

\begin{rmk}[Notation]
	Soit $u \in E^\N$. Pour $n \in \N$, on écrit $u_n$ à la place de $u(n)$.
\end{rmk}

\begin{defn}
	Soient $E$ et $I$ deux ensembles. Une \underline{famille de $E$ indéxée par $I$}\index{famille (ensemble)} est une application de $I$ dans $E$.

	À la place de $u(i)$ (avec $i \in I$), on écrit $u_i$.
\end{defn}

\begin{defn}
	Soit $E$ un ensemble et $(A_i)_{i \in I}$ une famille de parties de $E$. On suppose $I \neq \O$. On pose \[
		\bigcup_{i \in  I} A_i = \{x \in E  \mid \exists i \in I,\, x \in A_i\}
	\] et \[
		\bigcap_{i \in  I} A_i = \{x \in E  \mid \forall i \in I,\, x \in A_i\}.
	\] On pose aussi $\bigcup_{i \in \O} A_i = \O$ et $\bigcap_{i \in \O}  A_i = E$.
\end{defn}

\begin{rmk}
	De même que pour les sommes et produits de complexes, on peut intervertir des réunions doubles.
\end{rmk}

\begin{prop}
	Soit $E$ un ensemble, $(A,B) \in \mathcal{P}(E)^2$. \[
		A \subset (E \setminus B) \iff A \cap B = \O.
	\]
\end{prop}

\begin{figure}[H]
	\centering
	\begin{asy}
		import patterns;
		add("hatch",hatch(1mm, deepcyan));
		add("hatch2",hatch(1mm, heavygreen));
		size(3cm);

		guide main_set = scale(1.3) * ((-1,1)..(-0.8,-0.8)..(0,-0.9)..(0.7,-1.2)..(0.8, 0.9)..cycle);
		guide set_a = shift((-0.5, -0.2)) * ((-0.6, 0.6)..(0.2,-0.2)..(0.2,-0.4)..(-0.6,-0.2)..cycle);
		guide set_b = shift((0.3, 0.4)) * ((0.8, -0.6)..(1.1,-0.2)..(0.2,0.5)..(0.2,-0.8)..cycle);

		draw(main_set, magenta); label("$E$", 1.3*(0.8,0.9),magenta, align=NE);
		draw(set_a, deepcyan); label("$A$", (-0.6,0.6), deepcyan, align=NW);
		draw(set_b, heavygreen); label("$B$", (0.8,-0.6), heavygreen, align=SE);

		fill(set_a, pattern("hatch"));
		fill(set_b, pattern("hatch2"));
	\end{asy}
\end{figure}

\begin{prv}
	\begin{itemize}
		\item[``$\implies$''] Soit $x \in A \cap B$. Alors $x \in A$ et $x \in B$. Comme $x \in A \subset (E \setminus B)$, alors $x \in E \setminus B$ i.e. $x \not\in B$ : une contradiction. Donc $A \cap B = \O$.
		\item[``$\impliedby$''] On suppose $A \cap B = \O$. Soit $x \in A$. Si $x \in B$, alors $x \in A \cap B = \O$ : faux.
			Donc $x \not\in B$ et donc $x \in E \setminus B$.
	\end{itemize}
\end{prv}

\begin{prop}
	Si $f: E\to F$ et $g: F \to G$ sont bijectives, alors $g \circ f$ est bijective et \[
		(g \circ f)^{-1} = f^{-1} \circ g^{-1}.
	\] \qed
\end{prop}

\begin{rmk}[\danger Attention]
	$g \circ f$ peut-être bijective alors que $f$ et $g$ ne le sont pas.
\end{rmk}



	\chap[15]{Espaces vectoriels}
	\renewcommand{\cwd}{../chap15}
	\part{Topologie de $\R^2$}

\begin{defn}
	La \underline{norme (euclidienne)} de $\R^2$ est l'application définie par \[
		\forall (x,y) \in \R^2, \|(x,y)\| = \sqrt{x^2 + y^2}.
	\]

	\begin{figure}[H]
		\centering
		\begin{asy}
			import graph;
			axes(EndArrow);
			size(4cm);
			pair A = (3,2);
			dot(A);
			draw((3,0)--A, dashed);
			draw((0,2)--A, dashed);
			label("$x$", (3,0), align=S);
			label("$y$", (0,2), align=W);
			draw((0,0)--A);
			dot((4,3), white+0);
		\end{asy}
	\end{figure}
	\index{norme (de $\R^2$)}
	\index{norme euclidienne (de $\R^2$)}
\end{defn}

\begin{prop}
	La norme euclidienne vérifie:
	\begin{enumerate}
		\item (séparation) \[
			\forall (x,y) \in \R^2, \|(x,y)\| = 0 \iff x = y = 0,
			\]
		\item (homogénéité positive) \[
				\forall \lambda \in \R, \forall (x,y) \in \R^2, \|\lambda(x,y)\|= \left| \lambda \right| \|(x,y)\|
			\]
		\item (inégalité triangulaire) \[
			\forall (x,y), (a,b) \in \R^2,
			\|(x,y)+(a,b)\|\le \|(x,y)\|+\|(a,b)\|.
		\]
	\end{enumerate}
\end{prop}

\begin{prv}
	Déjà vue en replaçant $(x,y)$ par $x+iy \in \C$ et $\|(x,y)\|$ par |x+iy|
\end{prv}

\begin{defn}
	Soit $(a,b) \in \R^2$ et $r \in \R_+$.

	La \underline{boule ouverte} (ou \underline{disque ouvert}) de centre $(a,b)$ et de rayon $r$ est \[
		B_{(a,b)}(r) = \big\{ (x,y) \in \R^2  \mid \|(x,y) - (a,b)\| < r \big\}.
	\]

	La \underline{boule fermée} (ou \underline{disque fermé}) de centre $(a,b)$ et de rayon $r$ est \[
		\overline{B_{(a,b)}}(r) = \big\{ (x,y)\in \R^2  \mid \|(x,y) - (a,b)\| \le r \big\}.
	\]

	La \underline{sphère} (ou \underline{boule}) de centre $(a,b)$ et de rayon $r$ est \[
		S_{(a,b)}(r) = \partial \overline{B_{(a,b)}}(r) = \big\{ (x,y) \in \R^2  \mid \|(x,y) - (a,b)\| = r \big\}.
	\]
	\index{boule ouverte (de $\R^2$)}
	\index{disque ouverte (de $\R^2$)}
	\index{boule fermée (de $\R^2$)}
	\index{disque fermée (de $\R^2$)}
	\index{boule (de $\R^2$)}
	\index{sphère (de $\R^2$)}
\end{defn}

\begin{figure}[H]
		\centering
		\incfig{boule}
\end{figure}

\begin{rmk}
	On parle de boule en dimension quelconque.
\end{rmk}

\begin{defn}
	Une \underline{partie ouverte} $O$ de $\R^2$ (ou \underline{un ouvert}) si \[
		\forall (x,y) \in O, \exists r > 0, B_{(a,b)}(r) \subset O.
	\]
	Une partie $F$ est \underline{fermée} su $\R^2\setminus F$ est ouverte.
	\index{partie ouverte (de $\R^2$)}
	\index{ouvert (de $\R^2$)}
	\index{partie fermée (de $\R^2$)}
\end{defn}

\begin{figure}[H]
	\centering
	\incfig{partie-ouverte}
\end{figure}

\begin{prop}
	Une boule ouverte est ouverte. Une boule fermée est fermée.
\end{prop}

\begin{figure}[H]
	\centering
	\begin{subfigure}{4cm}
		\centering
		\begin{asy}
			import patterns;

			pair n(pair a) {return a / length(a);}

			add("hatch",hatch(2mm, SW, red));
			size(4cm);

			draw(circle((0,0), 1));
			dot('$(a_0, b_0)$', (0,0), align=S);

			draw((0,0) -- n((-1, 1)), dashed);
			label("$r$", n((-1, 1)) / 2, align=1.5S);

			pair A = n((1,3)) * (2/3);
			real rho = (1 - length(A)) * (2 / 3);

			dot("$(a,b)$", A, red, align=3SE);
			filldraw(circle(A, rho), pattern("hatch"), red);

			label("$O$", n((1,-1))*2.5/3);
		\end{asy}
	\end{subfigure}
	\begin{subfigure}{1cm}
		\centering~\\
	\end{subfigure}
	\begin{subfigure}{5cm}
		\centering
		\begin{asy}
			import patterns;

			pair n(pair a) {return a / length(a);}

			add("hatch",hatch(1mm, SW, red));
			add("hatch2",hatch(3mm, SE, blue));
			size(5cm);

			guide around = (-1.5, -1.5) -- (-1.5, 1.5) -- (2.5, 1.5) -- (2.5, -1.5) -- cycle;

			pair A = n((3, 1)) * 5/3; 
			real rho = 2 / 9;

			picture inter;
			fill(inter, around, pattern("hatch2"));
			fill(inter, circle((0,0), 1), white);
			add(inter);

			draw(circle((0,0), 1));
			dot('$(a_0, b_0)$', (0,0), align=S);

			draw((0,0) -- n((-1, 1)), dashed);
			label("$r$", n((-1, 1)) / 2, align=1.5S);

			dot("$(a,b)$", A, red, align=2SE);
			filldraw(circle(A, rho), pattern("hatch"), red);

			label("$F$", n((1,-1))*2.5/3);
		\end{asy}
	\end{subfigure}
\end{figure}

\begin{prv}
	$\O$ est un ouvert.

	Soit $B$ la boule ouverte de centre $(a_0, b_0) \in \R^2$ et de rayon $r > 0$.

	On pose $\rho = \frac{1}{2}\big(r - \|(a,b) - (a_0,b_0)\|\big)$.
	Montrons que \[
		B_{(a,b)}(\rho) \subset  B_{(a,b)}(r).
	\]

	Soit $(x,y) \in B_{(a,b)}(\rho)$.
	\begin{align*}
		\|(x,y) - (a_0,b_0)\|&= \|(x,y)- (a,b) + (a,b) - (a_0,b_0)\| \\
		&\le \|(x,y) - (a,b)\| + \|(a,b) - (a_0, b_0)\|\\
		&< \rho + \|(a,b) - (a_0, b_0)\| = \frac{1}{2}r + \frac{1}{2} \|(a,b) - (a_0, b_0)\|\\
		&< r
	\end{align*}
	
	Soit $F$ la boule fermée de centre $(a_0, b_0)$ et de rayon $r \ge 0$.

	Soit $(a,b) \not\in F$. On pose \[
		\rho = \frac{1}{2}\big(\|(a,b) - (a_0, b_0)\| - r\big) > 0.
	\]

	Montrons que $B_{(a,b)}(\rho) \subset \R^2\setminus F$.

	Soit $(x,y) \in B_{(a,b)}(\rho)$.

	\begin{align*}
		\|(x,y) - (a_0, b_0)\| &= \|(x,y) - (a,b) + (a,b) - (a_0, b_0)\| \\
		&\ge \big| \underbrace{\|(x,y) - (a,b)\|}_{\le \rho} - \underbrace{\|(a,b) - (a_0, b_0)\|}_{> r} \big|\\
		&\ge \|(a,b) - (a_0, b_0)\|- \|(x,y) - (a,b)\|\\
		&> \|(a,b) - (a_0, b_0)\|- \rho\\
		&> \frac{1}{2} \|(a,b) - (a_0, b_0)\| + \frac{1}{2}r\\
		&> r
	\end{align*}

	donc $(x,y) \not\in F$.
\end{prv}

\begin{exm}
	\begin{enumerate}
		\item $\O$ est ouvert.\\
			$\R^2$ est ouvert.
		\item $\O$ est fermé.\\
			$\R^2$ est fermé.\\
		\item $\big\{(x,0)  \mid x > 0\big\}$ n'est ni ouverte ni fermé.
	\end{enumerate}
\end{exm}

\begin{figure}[H]
	\centering
	\begin{asy}
		size(3cm);

		draw((0, -1) -- (0, 3), Arrow(TeXHead));
		draw((-1, 0) -- (3, 0), Arrow(TeXHead));
		
		draw((0,0) -- (0, 2.97), red);
		draw(circle((0,1.5), 0.5), deepred);
		draw(circle((0,0.5), 0.1), deepred);
	\end{asy}
\end{figure}

\begin{defn}
	Soit $(a,b) \in \R^2$ et $V \in \mathcal{P}(\R^2)$.

	On dit que $V$ est un voisinage de $(a,b)$ s'il existe $r > 0$ tel que \[
		B_{(a,b)}(r) \subset V.
	\]
	\index{voisinage (dans $\R^2$)}
\end{defn}

\begin{prop}
	Un ouvert non vide est un voisinage en chacun de ces points. \qed
\end{prop}

\begin{defn}
	Soit $D \subset \R^2$. Un \underline{point intérieur} de $D$ est un couple $(a,b) \in D$ tel que \[
		\exists r > 0, B_{(a,b)}(r) \subset D.
	\] en d'autres termes, si $D$ est un voisinage de $(a,b)$.

	On note $\mathring D$ l'ensemble des points intérieurs à $D$. C'est \underline{l'intérieur} de $D$.
	\index{point intérieur (dans $\R^2$)}
	\index{intérieur (dans $\R^2$)}
\end{defn}

\begin{prop}
	$\mathring D$ est le plus grand ouvert $O$ de $\R^2$ tel que $O \subset D$.
\end{prop}

\begin{figure}[H]
	\centering
	\incfig{interieur-plus-grand-ouvert}
\end{figure}


\begin{prv}
	Soit $(a,b) \in \mathring D$.

	Par définition, il existe $r > 0$ tel que \[
		B_{(a,b)}(r) \subset D.
	\] Montrons que $B_{(a,b)}(r) \subset \mathring D$.

	Soit $(x,y) \in B_{(a,b)}(r)$. Comme $B_{(a,b)}(r)$ est un ouvert de $\R^2$, il existe $\rho > 0$ tel que \[
		B_{(x,y)}(\rho) \subset B_{(a,b)}(r)
	\] donc $(x,y) \in \mathring D$.

	Donc $\mathring D$ est ouvert, $\mathring D \subset D$.

	Soit $O$ un ouvert de $\R^2$ tel que $O \subset D$. Montrons que $O \subset \mathring D$.

	Soit $(x,y) \in O$. Soit $r > 0$ tel que \[
		B_{(x,y)}(r) \subset O \subset D
	\] donc $(x,y) \in \mathring D$.
\end{prv}

\begin{defn}
	Soit $f: D \subset \R^2 \to \R$, $\ell \in \R$, $(a,b) \in \mathring D$.

	On dit que \underline{$f(x,y)$ tend vers $\ell$ quand $(x,y)$ tend vers $(a,b)$} ou que $\ell$ est \underline{une limite} de $f$ en $(a,b)$ si \[
		\forall \varepsilon > 0, \exists r > 0, \forall (x,y) \in D, \|(x,y) - (a,b)\| < r \implies \left| f(x,y) - \ell \right| \le \varepsilon.
	\] en d'autres termes si \[
		\forall V \in \mathcal{V}_{\ell}, \exists W \in \mathcal{V}_{(a,b)}, \forall (x,y) \in W \cap D, f(x,y) \in V.
	\]
	\index{limite (dans $\R^2$)}
	\index{tendre vers (dans $\R^2$)}
\end{defn}

\begin{prop}
	[unicité de la limite]
	Soit $f: D \to \R$, $(a,b) \in \mathring D$, $\ell_1, \ell_2 \in \R$ telles que $\ell_1$ et $\ell_2$ sont des limites de $f$ en $(a,b)$.

	Alors $\ell_1 = \ell_2$.
\end{prop}

\begin{figure}[H]
	\centering
	\incfig{preuve-unicité-de-la-limite}
\end{figure}

\begin{prv}
	On suppose $\ell_1 < \ell_2$. On pose $\varepsilon = \frac{\ell_2 - \ell_1}{2} > 0$.

	Soit $r_1 > 0$ tel que \[
		f\big(B_{(a,b)}(r_1)\big) \subset ]\ell_1 - \varepsilon, \ell_1 + \varepsilon[.
	\] Soit $r_2 > 0$ tel que \[
		f\big(B_{(a,b)}(r_2)\big) \subset ]\ell_2 - \varepsilon, \ell_2 + \varepsilon [.
	\] On pose $r = \min(r_1, r_2)$ donc \[
		B_{(a,b)}(r_1) \cap B_{(a,b)}(r_2) = B_{(a,b)}(r) \neq \O.
	\] Soit $(x,y) \in B_{(a,b)}(r)$. Alors, \[
		f(x,y) \in ]\ell_1 - \varepsilon, \ell_1 + \varepsilon[ \cap ]\ell_2 - \varepsilon, \ell_2 + \varepsilon[ = \O.
	\] $\lightning$
\end{prv}

\begin{defn}
	Soit $f : D \to \R$, $(a,b) \in \mathring D$.

	On dit que $f$ est \underline{continue} en $(a,b)$ si \[
		f(x,y) \tendsto{(x,y) \to (a,b)}f(a,b).
	\]
	\index{continuité (dans $\R^2$)}
\end{defn}

\begin{prop}
	\underline{Si} $f(x,y) \tendsto{(x,y) \to (a,b)} \ell$ \\
	\underline{alors} $\begin{cases}
		f(x,b) \tendsto{x \to a} \ell\\
		f(a,y) \tendsto{y \to b} \ell.\\
	\end{cases}$
\end{prop}

\begin{prv}~\\
	\begin{figure}[H]
		\centering
		\incfig{limite-x-en-a-et-y-en-b}
	\end{figure}
\end{prv}

\underline{Contre-exemple} : exercice 3.

\begin{exm}
	\begin{enumerate}
		\item $f : \begin{array}{rcl}
				\R^2 &\longrightarrow& \R \\
				(x,y) &\longmapsto& x
			\end{array}$ limite en $(0,0)$ ?

			Soit $\varepsilon > 0$. On pose $r = \varepsilon$. \[
				\forall (x,y) \in B_{(0,0)}(r),
				\left| f(x,y) \right| = \left| x \right| \le \|(x,y)\| < r = \varepsilon
			\] Donc $f(x,y) \tendsto{(x,y) \to (a,b)} 0$.
		\item limite $f : \begin{array}{rcl}
				\R^2 &\longrightarrow& \R \\
				(x,y) &\longmapsto& x^3
			\end{array}$ en $(0,0)$ ?

			Soit $\varepsilon > 0$. On pose $r = \sqrt[3]{r} > 0$. \[
				\forall (x,y) \in B_{(0,0)}(r),
				\left| f(x,y) \right| = \left| x^3 \right| \le \|(x,y)\|^3 < r^3 = \varepsilon.
			\]
		\item limite de $f : \begin{array}{rcl}
			\R^2 &\longrightarrow& \R \\
			(x,y) &\longmapsto& x^3y^2
		\end{array}$ en $(0,0)$ ?

		Soit $\varepsilon > 0$. On pose $r = \sqrt[5]{\varepsilon} > 0$. \[
			\forall (x,y) \in B_{(0,0)}(r), \left| f(x,y) \right| = \left| x^3 y^2 \right| \le \|(x,y)\|^3 \|(x,y)\|^2 < r^5 = \varepsilon.
		\]
	\end{enumerate}
\end{exm}

\begin{defn}
	Soient $D \subset \R^2$ et $(x,y) \in \R^2$.

	\begin{figure}[H]
    \centering
    \incfig{point-adhérent}
	\end{figure}
	
	On dit que $(x,y)$ est \underline{adhérent} à $D$ si \[
		\forall r > 0, B_{(x,y)}(r) \cap D \neq \O.
	\] L'ensemble des points adhérents à $D$ est noté $\overline{D}$. On dit que $\overline{D}$ est \underline{l'adhérence} de $D$.
	\index{point adhérent (dans $\R^2$)}
	\index{adhérent (dans $\R^2$)}
\end{defn}

\begin{defn}
	Soit $f: D \subset \R^2 \to \R$ et $(a,b) \in \overline{D}$, $\ell \in \R$. On dit que $f$ tend vers $\ell$ quand $(x,y)$ tend vers $(a,b)$ si \[
		\forall \varepsilon > 0, \exists r > 0, \forall (x,y) \in B_{(a,b)}(r) \cap D,
		\left| f(x,y) - \ell \right| \le \varepsilon.
	\]
	\index{limite (dans $\R^2$)}
	\index{tendre vers (dans $\R^2$)}
\end{defn}

\begin{prop}
	\begin{enumerate}
		\item Dans ce contexte, il y a unicité de la limite
		\item La limite d'une somme, d'un produit, d'un quotien, d'une composée se comporte comme dans le cas d'une seule variable.
		\item Soit $f: D \to \R$ continue. Soient $g: I \to \R$ et $h: I \to \R$ continues telles que \[
			\forall t \in I, \big(g(t), h(t)\big) \in D.
		\] Alors \[
			t \in I \mapsto f\big(g(t), h(t)\big) \in \R
		\] est continue.
	\end{enumerate}
\end{prop}

\begin{figure}[H]
	\centering
	\begin{asy}
		import three;
		import graph3;
		size(5cm);

		settings.render = 0;
		settings.prc = false;
		currentprojection = obliqueX;

		draw(O -- X, Arrow3(TeXHead2));
		draw(O -- Y, Arrow3(TeXHead2));
		draw(O -- Z, Arrow3(TeXHead2));

		triple f(real x, real y, real z = 0) { return (x,y,cos(x - 0.5) * cos(y - 0.5)/1.2 + 0.15); }

		real inc = 1 / 5;

		for(real x = 0; x <= 1; x += inc) {
			draw(graph(
				new real(real t) { return x; }, // x
				new real(real y) { return y; }, // y
				new real(real y) { return f(x,y).z; }, // z
				0, 1
			), gray);
		}

		for(real y = 0; y <= 1; y += inc) {
			draw(graph(
				new real(real x) { return x; }, // x
				new real(real t) { return y; }, // y
				new real(real x) { return f(x,y).z; }, // z
				0, 1
			), gray);
		}

		path3 path1 = (0.3, 0.2, 0) .. (0.5, 0.5, 0) .. (0.6, 0.7, 0) .. (0.9, 0.8, 0);
		path3 path2 = (0.3, 0.8, 0) .. (0.5, 0.5, 0) .. (0.6, 0.3, 0) .. (0.9, 0.2, 0);
		path3 pathA = f(0.3, 0.2, 0) .. f(0.5, 0.5, 0) .. f(0.6, 0.7, 0) .. f(0.9, 0.8, 0);
		path3 pathB = f(0.3, 0.8, 0) .. f(0.5, 0.5, 0) .. f(0.6, 0.3, 0) .. f(0.9, 0.2, 0);

		draw(path1, red, Arrow3(TeXHead2, position=0.5));
		draw(pathA, red, Arrow3(TeXHead2, position=0.5));
		draw(path2, deepcyan, Arrow3(TeXHead2, position=0.3));
		draw(pathB, deepcyan, Arrow3(TeXHead2, position=0.3));

		dot((0.5, 0.5, 0));
		dot(f(0.5, 0.5, 0));
		draw((0.5, 0.5, 0) -- f(0.5, 0.5, 0), dashed);
	\end{asy}
\end{figure}


	\part{Transpositions}

\begin{defn}
	Une \underline{transposition} est un cycle de longueur 2 : $\begin{pmatrix}
		a&b
	\end{pmatrix}$ avec $a \neq b$.
	\index{transposition (permutation)}
\end{defn}

\begin{exm}
	Avec $n = 5$ et $\gamma = \begin{pmatrix}
		2&4&1
	\end{pmatrix}$.

	\begin{figure}[H]
		\centering

		\begin{asy}
			size(5cm);

			real rho = 0.15; // circles

			void draw_cycle(pair O, real r ...int[] nums) {
				int n = nums.length;
				real eps = (15 / r) * 0.8;

				for(int i = 0; i < n; ++i) {
					real theta_1 = (360/n) * (i+1);
					real theta_2 = (360/n) * i;

					pair C = O + dir(theta_2) * r;

					draw(circle(C, rho));
					label("$" + string(nums[i]) + "$", C);
					draw(arc(O, r, theta_2+eps, theta_1-eps), Arrow(TeXHead));
				}
			}

			draw_cycle((-1,0), 0.8, 1, 2, 4);
			draw_cycle((1,0), 0.3, 3);
			draw_cycle((2,0), 0.3, 5);
		\end{asy}
	\end{figure}

	\[
		\gamma = \begin{pmatrix}
			1&4
		\end{pmatrix} \begin{pmatrix}
			1&2
		\end{pmatrix}
	\]

	Avec $n = 6$ et $\gamma = \begin{pmatrix}
		1&3&5&6&2
	\end{pmatrix} = \begin{pmatrix}
		1&2&3&4&5&6\\
		3&1&5&4&6&2
	\end{pmatrix}$.

	Donc, \[
		\gamma = \begin{pmatrix}
			1&2
		\end{pmatrix} \begin{pmatrix}
			1&6
		\end{pmatrix} \begin{pmatrix}
			1&5
		\end{pmatrix} \begin{pmatrix}
			1&3
		\end{pmatrix}
	\] 
	\[
		\begin{pmatrix}
			1&2&3&4&5&6\\
			3&2&1&4&5&6\\
			3&2&5&4&1&6\\
			3&2&5&4&6&1\\
			3&1&5&4&6&2\\
		\end{pmatrix}
	\]

	Et, \[
		\gamma = \begin{pmatrix}
			1&3
		\end{pmatrix} \begin{pmatrix}
			2&3
		\end{pmatrix} \begin{pmatrix}
			3&5
		\end{pmatrix} \begin{pmatrix}
			5&6
		\end{pmatrix} 
	\]

	\[
		\begin{pmatrix}
			1&2&3&4&5&6\\
			1&2&3&4&6&5\\
			1&2&5&4&6&3\\
			1&3&5&4&6&2\\
			3&1&5&4&6&2\\
		\end{pmatrix} 
	\] 
\end{exm}

\begin{exm}
	\[
		\begin{pmatrix}
			1&4
		\end{pmatrix} = \begin{pmatrix}
			1&2
		\end{pmatrix} \begin{pmatrix}
			2&3
		\end{pmatrix} \begin{pmatrix}
			3&4
		\end{pmatrix} \begin{pmatrix}
			2&3
		\end{pmatrix} \begin{pmatrix}
			1&2
		\end{pmatrix}
	\]
	On n'a pas toujours le même nombre de transpositions mais la parité du nombre reste la même (proposition plus loin).
\end{exm}

\begin{thm}
	Toute permutation se décompose en produit de transpositions.
\end{thm}

\begin{prv}
	Soit $\gamma = \begin{pmatrix}
		a_1&\cdots&a_k
	\end{pmatrix}$ un $k$-cycle.

	On remarque que
	\[
		\gamma = \begin{pmatrix}
			a_1&a_k
		\end{pmatrix} \cdots \begin{pmatrix}
			a_1&a_4
		\end{pmatrix} \begin{pmatrix}
			a_1&a_3
		\end{pmatrix} \begin{pmatrix}
			a_1&a_2
		\end{pmatrix}
	\] C'est un produit de transpositions.
\end{prv}

\begin{exm}
	Avec $n = 10$ et $\sigma = \begin{pmatrix}
		1&2&3&4&5&6&7&8&9&10\\
		9&8&1&7&2&3&4&5&10&6
	\end{pmatrix}$.

	On a
	\begin{align*}
		\sigma &= \begin{pmatrix}
			1&9&10&6&3
		\end{pmatrix} \begin{pmatrix}
			2&8&5
		\end{pmatrix} \begin{pmatrix}
			4&7
		\end{pmatrix}\\
		&= \begin{pmatrix}
			1&3
		\end{pmatrix} \begin{pmatrix}
			1&6
		\end{pmatrix} \begin{pmatrix}
			1&10
		\end{pmatrix} \begin{pmatrix}
			1&9
		\end{pmatrix} \begin{pmatrix}
			2&5
		\end{pmatrix} \begin{pmatrix}
			2&8
		\end{pmatrix} \begin{pmatrix}
			4&7
		\end{pmatrix} \\
	\end{align*}

	Vérification : \[
		\begin{pmatrix}
			1&2&3&4&5&6&7&8&9&10\\
			1&2&3&7&5&6&4&8&9&10\\
			1&8&3&7&5&6&4&2&9&10\\
			1&8&3&7&2&6&4&5&9&10\\
			9&8&3&7&2&6&4&5&1&10\\
			9&8&3&7&2&6&4&5&10&1\\
			9&8&3&7&2&1&4&5&10&6\\
			9&8&1&7&2&3&4&5&10&6\\
		\end{pmatrix} 
	\] 
\end{exm}

	\part{Familles orthogonales}

\begin{thm}[Pythagore]
	Soit $(x,y) \in E^2$. \[
		\|x+y\|^2 = \|x\|^2 + \|y\|^2 \iff x \perp y
	.\]
	\begin{figure}[H]
		\centering
		\begin{asy}
			size(4cm);
			pair u = (1, 0.5);
			pair v = 1.5 * (0, 1) * u;
			draw((0,0)--u, Arrow(TeXHead));
			label("$x$", u/2, align=S);
			draw(u--v+u, Arrow(TeXHead));
			label("$y$", u + v/2, align=NE);
			draw((0,0) -- u + v, Arrow(TeXHead));
			draw(u + v / 7.5 -- u + v / 7.5 - u / 5 -- u - u / 5 -- u -- cycle);
		\end{asy}
	\end{figure}
\end{thm}

\begin{prv}
	\[
		\|x + y\|^2 = \|x\|^2 + \|y\|^2 \iff 2\left<x \mid y \right> = 0 \iff x \perp y
	.\]
\end{prv}

\begin{defn}
	Soit $(e_i)_{i\in I}$ une famille de vecteurs. On dit que cette famille est \underline{orthogonale} si \[
		\forall i \neq j\, e_i \perp e_j
	.\] Si, en plus, on a \[
		\forall i \in I,\,\|e_i\| = 1,
	\] alors on dit que la famille est \underline{orthonormale} ou \underline{orthonormée}.
	\index{famille orthogonale}
	\index{famille orthonormale}
	\index{famille orthonormée}
\end{defn}

\begin{prop}[Pythagore]
	Soit $(e_1, \ldots, e_n)$ une famille orthogonale. Alors \[
		\left\| \sum_{i=1}^n e_i \right\|^2 = \sum_{i=1}^n \|e_i\|^2
	.\]
\end{prop}

\begin{thm}
	Toute famille orthogonale de vecteurs non nuls est libre.
\end{thm}

\begin{prv}
	Soit $(e_i)_{i\in I}$ une famille orthogonale telle que \[
		\forall i \in I,\,e_i \neq 0_E
	.\] Soit $n \in \N^*$, $(\lambda_1, \ldots, \lambda_n) \in \R^n$. On suppose \[
		\sum_{k=1}^n \lambda_k e_{i_k} = 0_E
	.\] Soit $j \in \left\llbracket 1,n \right\rrbracket$.
	\begin{align*}
		0 &= \left<\sum_{k=1}^n \lambda_k e_{i_k}  \mid e_{i_j} \right>\\
		&= \sum_{k=1}^n \lambda_k \left<e_{i_k}  \mid e_{i_j} \right> \\
		&= \lambda_j \underbrace{\|e_{i_j}\|^2}_{\neq 0} \\
	\end{align*}
	donc $\lambda_j = 0$.
\end{prv}

\begin{algo}[Orthonormalisation de Gran--Schmidt]
	On suppose $E$ de dimension finie. Soit $\mathcal{B} = (e_1, \ldots, e_n)$ une base de $E$.

	\begin{itemize}
		\item\underline{\it Étape 1}: On pose $v_1 = \frac{e_1}{\|e_1\|}$ de sorte que $\|v_1\| = 1$.
		\item\underline{\it Étape 2} : On pose \[
				u_2 = e_2 - \left<e_2  \mid v_1 \right> v_1
			.\] Ainsi,
			\begin{align*}
				\left<u_2 \mid v_1 \right> &= \big<e_2 - \left<e_2 \mid v_1 \right> v_1  \mid v_1 \big>\\
				&= \left<e_2 \mid v_1 \right> - \left<e_2 \mid v_1 \right> \left<v_1 \mid v_1 \right> \\
				&= 0. \\
			\end{align*}
			On pose $v_2 = \frac{u_2}{\|u_2\|}$ donc $v_2 \perp v_1$ et $\|v_2\| = 1$.
		\item\underline{\it Étape 3} : On pose \[
				u_2 = e_3 - \left<e_3 \mid v_1 \right>v_1 - \left<e_3 \mid v_2 \right>v_2
			.\] Ainsi,
			\begin{align*}
				\left<u_3  \mid v_1 \right> &= \left<e_3  \mid v_1 \right> - \left<e_3 \mid v_1 \right>\underbrace{\left<v_1 \mid v_1 \right>}_{=1} - \left<e_3 \mid v_2 \right>\underbrace{\left<v_2 \mid v_1 \right>}_{=0} \\
				&= 0 \\
			\end{align*}
			et 
			\begin{align*}
				\left<u_3 \mid v_2 \right> &= \left<e_3  \mid  v_2 \right> - \left<e_3 \mid v_1 \right> \underbrace{\left<v_1 \mid v_2 \right>}_{=0} - \left<e_3 \mid v_2 \right> \underbrace{\left<v_2 \mid v_2 \right>}_{=1}\\
				&= 0. \\
			\end{align*}
			On pose $v_3 = \frac{u_3}{\|u_3\|}$ de sorte que $v_3 \perp v_1$, $v_3 \perp v_2$ et $\|v_3\| = 1$.
		\item\underline{\it Étape $i+1$}: On pose \[
			u_{i+1} = e_{i+1} - \sum_{k=1}^i \left<e_{i+1}  \mid v_k \right> v_k
		.\] Ainsi, pour tout $j \in \left\llbracket 1,i \right\rrbracket,$ on a
		\begin{align*}
			\left<u_{i+1}  \mid v_j \right> &= \left<e_{i+1}  \mid v_j \right> - \sum_{k=1}^i \left<e_{i+1} \mid v_k \right> \left<v_k \mid v_j \right> \\
			&= \left<e_{i+1} \mid v_j \right> - \left<e_{i+1} \mid v_j \right> \|v_j\|^2 \\
			&= 0. \\
		\end{align*}
		On pose $v_{i+1} = \frac{u_{i+1}}{\|u_{i+1}\|}$.
	\end{itemize}
\end{algo}

\begin{exm}
	Avec $E = \R_3[X]$, $\left<P \mid Q \right> = \int_{0}^{1} P(t)\,Q(t)~\mathrm{d}t$ et $\mathcal{B} = (1, X, X^2, X^3)$.
	\begin{enumerate}
		\item $\|1\|^2 = \left<1 \mid 1 \right> = \int_{0}^{1} 1~\mathrm{d}t = 1$ et donc $v_1 = 1$.
		\item $u_2 = X - \left<X  \mid v_1 \right>v_1$. Or, $\left<X \mid v_1 \right> = \int_{0}^{1} t~\mathrm{d}t = \frac{1}{2}$. D'où $u_2 = X - \frac{1}{2}$.
			\begin{align*}
				\|u_2\|^2 &= \int_{0}^{1} \left( t - \frac{1}{2} \right)^2~\mathrm{d}t \\
				&= \int_{0}^{1} \left( t^2 - t + \frac{1}{4} \right)~\mathrm{d}t \\
				&= \frac{1}{3} - \frac{1}{2} + \frac{1}{4} \\
				&= \frac{1}{12} \\
			\end{align*} On en déduit que $v_2 = \sqrt{12}\left( X - \frac{1}{2} \right)$.
		\item $u_3 = X^2 - \left<X^2 \mid v_1 \right>v_1 - \left<X^2 \mid v_2 \right>v_2$.
			On a \[
				\left<X^2 \mid v_1 \right> = \int_{0}^{1} t^2~\mathrm{d}t = \frac{1}{3}
			\] et
			\begin{align*}
				\left<X^2 \mid v_2 \right> &= \sqrt{12} \int_{0}^{1} t^2\left( t - \frac{1}{2} \right)~\mathrm{d}t \\
				&= \frac{\sqrt{12}}{12}. \\
			\end{align*}
			D'où
			\begin{align*}
				u_3 &= X^2 - \frac{1}{3} - \frac{\sqrt{12}}{12}\sqrt{12} \left( X - \frac{1}{2} \right)\\
				&= X^2 - \frac{1}{3} - X + \frac{1}{2} \\
				&= X^2 - X + \frac{1}{6}. \\
			\end{align*}
			\begin{align*}
				\|u_3\|^2 &= \int_{0}^{1} \left( t^2 - t + \frac{1}{6} \right)~\mathrm{d}t\\
				&= \int_{0}^{1} \left( t^4 + t^2 + \frac{1}{36} - 2t^3 + \frac{t^2}{3} - \frac{t}{3} \right) ~\mathrm{d}t \\
				&= \frac{1}{5} + \frac{1}{3} + \frac{1}{36} - \frac{1}{2} + \frac{1}{9} - \frac{1}{6} \\
				&= \frac{36 + 60 + 5 - 90 + 20 - 30}{180} \\
				&= \frac{1}{180} \\
			\end{align*}
			On en déduit que \[
				v_3 = 6\sqrt{5}\left( X^2 - X + \frac{1}{6} \right).
			\]
		\item Exercice : calculer $v_4$.
	\end{enumerate}
\end{exm}

\begin{prop}
	Soit $\mathcal{B} = (e_1, \ldots, e_n)$ une base de $E$ et $\mathcal{C}$ la base obtenue par le procédé d'orthonormalisation de Gram--Schmidt. Alors, \[
		\forall i \in \left\llbracket 1,n \right\rrbracket,\,\Vect(e_1,\ldots, e_i) = \Vect(v_1, \ldots, v_i)
	.\]\qed
\end{prop}

\begin{exm}[orthogonalisation]
	\begin{itemize}
		\item $u_1 = 1$.
		\item
			\begin{align*}
				\begin{rcases*}
					u_2 \in \Vect(e_1, e_2)\\
					u_2 \perp u_1
				\end{rcases*}
				\iff& \begin{cases}
					u_2 = ae_1 + be_2\quad (a,b) \in \R^2\\
					\left<u_1 \mid u_2 \right> = 0
				\end{cases}\\
				\iff& \begin{cases}
					u_2 = a + bX\\
					\int_{0}^{1} (a+bt)~\mathrm{d}t = 0.
				\end{cases}\\
			\end{align*}
			\begin{align*}
				\int_{0}^{1} (a+bt)~\mathrm{d}t = 0 \iff& a + \frac{b}{2} = 0\\
				\iff& a = -\frac{b}{2}\\
				\iff& u_2 = -\frac{b}{2} + bX.
			\end{align*}
			Par exemple, $u_2 = -1 + 2X$.
		\item $\begin{cases}
				u_3 \in \Vect(e_1, e_2, e_3)\\
				u_3 \perp u_1\\
				u_3 \perp u_2
			\end{cases}$

			On pose $u_3 = a + bX + cX^2$ avec $(a,b,c)\in \R^3$.
			\begin{align*}
				\begin{rcases*}
					\int_{0}^{1} \left( a+bt + ct^2 \right)~\mathrm{d}t = 0\\
					\int_{0}^{1} \left(a + bt+ct^2\right)(2t - 1)~\mathrm{d}t = 0
				\end{rcases*} \iff& \begin{cases}
					a + \frac{b}{2} + \frac{c}{3} = 0\\
					\int_{0}^{1} \left( 2ct^3 + (-c + 2b)t^2 + (2a - b)t - a \right) ~\mathrm{d}t = 0
				\end{cases}\\
				\iff& \begin{cases}
					a + \frac{b}{2} + \frac{c}{3} = 0\\
					\frac{c}{2} + \frac{2b - c}{3} + \frac{2\cancel{a} - b}{2} - \cancel{a} = 0
				\end{cases}\\
				\iff&  \begin{cases}
					a = -\frac{b}{2} - \frac{c}{3} = \frac{c}{2} - \frac{c}{3} = \frac{c}{6}\\
					b = -c.
				\end{cases}
			\end{align*}
			On en déduit que \[
				u_3 = 1 - 6X + 6X^2
			.\]
	\end{itemize}
\end{exm}

\begin{crlr}[théorème de la base orthonormée incomplète] Soit $(e_1, \ldots, e_k)$ une base orthonormée d'un espace euclidien. On peut trouver $e_{k+1},\ldots,e_n$ tels que $(e_1, \ldots, e_k, e_{k+1},\ldots,e_n)$ soit une base orthonormée de $E$.
\end{crlr}

\begin{prv}
	On sait que $(e_1, \ldots, e_k)$ est libre. On complète $(e_1, \ldots, e_k)$ en une base $\mathcal{B}$ de $E$. On orthonormalise $\mathcal{B}$ : on obtient une base orthonormée $\mathcal{C}$ de $E$. En détaillant l'algorithme de Gram--Schmidt, on s'aper\c coit que les $k$ premiers vecteurs de $\mathcal{C}$ sont ceux de $\mathcal{B}$.
\end{prv}

\begin{thm}
	Soit $E$ un espace euclidien et $\mathcal{B} = (e_1, \ldots, e_n)$ une base orthonormée de $E$. Soit $(x,y) \in E^2$. On pose $(x_1, \ldots, x_n) \in \R^n$ et $(y_1, \ldots, y_n) \in \R^n$ tels que \[
		x = \sum_{i=1}^n x_i e_i \qquad\qquad y = \sum_{i=1}^n y_i e_i
	.\] Alors \[
		\left<x \mid y \right> = \sum_{i=1}^n x_i y_i
	.\]
	\vspace{3mm}
	Soit $X = \mat{x_1\\\vdots\\x_n}$ et $Y = \mat{y_1\\ \vdots \\ y_n}$. Alors, \[
		\left<x \mid y \right> = X^\T\,Y
	.\]
\end{thm}

\begin{prv}
	\begin{align*}
		\left<x \mid y \right> &= \left<\sum_{i=1}^n x_ie_i  \mid y \right>\\
		&= \sum_{i=1}^n x_i \left<e_i  \mid y \right> \\
		&= \sum_{i=1}^n x_i \left<e_i  \mid \sum_{j=1}^n y_j e_j \right> \\
		&= \sum_{i=1}^n x_i \sum_{j=1}^n y_j \underbrace{\left<e_i \mid e_j \right>}_{\delta_i^j} \\
		&= \sum_{i=1}^n x_i y_i. \\
	\end{align*}
\end{prv}

\begin{prop}
	Soit $E$ un espace euclidien et $\mathcal{B} = (e_1, \ldots, e_n)$ une base orthonormée de $E$. Alors, \[
		\forall x \in E,\,x = \sum_{i=1}^n \left<x \mid e_i \right>e_i
	.\]
\end{prop}

\begin{prv}
	Soit $x \in E$. On pose \[
		x = \sum_{i=1}^n x_i e_i
	\] avec $(x_1, \ldots, x_n) \in \R^n$. Soit $j \in \left\llbracket 1,n \right\rrbracket$. On a
	\begin{align*}
		\left<x \mid e_j \right> &= \left<\sum_{i=1}^n x_i e_i  \mid e_j \right>\\
		&= \sum_{i=1}^n x_i \left<e_i \mid e_j \right> \\
		&= x_j. \\
	\end{align*}
\end{prv}


	\chap[16]{Dérivation}
	\renewcommand{\cwd}{../chap16}
	\part{Topologie de $\R^2$}

\begin{defn}
	La \underline{norme (euclidienne)} de $\R^2$ est l'application définie par \[
		\forall (x,y) \in \R^2, \|(x,y)\| = \sqrt{x^2 + y^2}.
	\]

	\begin{figure}[H]
		\centering
		\begin{asy}
			import graph;
			axes(EndArrow);
			size(4cm);
			pair A = (3,2);
			dot(A);
			draw((3,0)--A, dashed);
			draw((0,2)--A, dashed);
			label("$x$", (3,0), align=S);
			label("$y$", (0,2), align=W);
			draw((0,0)--A);
			dot((4,3), white+0);
		\end{asy}
	\end{figure}
	\index{norme (de $\R^2$)}
	\index{norme euclidienne (de $\R^2$)}
\end{defn}

\begin{prop}
	La norme euclidienne vérifie:
	\begin{enumerate}
		\item (séparation) \[
			\forall (x,y) \in \R^2, \|(x,y)\| = 0 \iff x = y = 0,
			\]
		\item (homogénéité positive) \[
				\forall \lambda \in \R, \forall (x,y) \in \R^2, \|\lambda(x,y)\|= \left| \lambda \right| \|(x,y)\|
			\]
		\item (inégalité triangulaire) \[
			\forall (x,y), (a,b) \in \R^2,
			\|(x,y)+(a,b)\|\le \|(x,y)\|+\|(a,b)\|.
		\]
	\end{enumerate}
\end{prop}

\begin{prv}
	Déjà vue en replaçant $(x,y)$ par $x+iy \in \C$ et $\|(x,y)\|$ par |x+iy|
\end{prv}

\begin{defn}
	Soit $(a,b) \in \R^2$ et $r \in \R_+$.

	La \underline{boule ouverte} (ou \underline{disque ouvert}) de centre $(a,b)$ et de rayon $r$ est \[
		B_{(a,b)}(r) = \big\{ (x,y) \in \R^2  \mid \|(x,y) - (a,b)\| < r \big\}.
	\]

	La \underline{boule fermée} (ou \underline{disque fermé}) de centre $(a,b)$ et de rayon $r$ est \[
		\overline{B_{(a,b)}}(r) = \big\{ (x,y)\in \R^2  \mid \|(x,y) - (a,b)\| \le r \big\}.
	\]

	La \underline{sphère} (ou \underline{boule}) de centre $(a,b)$ et de rayon $r$ est \[
		S_{(a,b)}(r) = \partial \overline{B_{(a,b)}}(r) = \big\{ (x,y) \in \R^2  \mid \|(x,y) - (a,b)\| = r \big\}.
	\]
	\index{boule ouverte (de $\R^2$)}
	\index{disque ouverte (de $\R^2$)}
	\index{boule fermée (de $\R^2$)}
	\index{disque fermée (de $\R^2$)}
	\index{boule (de $\R^2$)}
	\index{sphère (de $\R^2$)}
\end{defn}

\begin{figure}[H]
		\centering
		\incfig{boule}
\end{figure}

\begin{rmk}
	On parle de boule en dimension quelconque.
\end{rmk}

\begin{defn}
	Une \underline{partie ouverte} $O$ de $\R^2$ (ou \underline{un ouvert}) si \[
		\forall (x,y) \in O, \exists r > 0, B_{(a,b)}(r) \subset O.
	\]
	Une partie $F$ est \underline{fermée} su $\R^2\setminus F$ est ouverte.
	\index{partie ouverte (de $\R^2$)}
	\index{ouvert (de $\R^2$)}
	\index{partie fermée (de $\R^2$)}
\end{defn}

\begin{figure}[H]
	\centering
	\incfig{partie-ouverte}
\end{figure}

\begin{prop}
	Une boule ouverte est ouverte. Une boule fermée est fermée.
\end{prop}

\begin{figure}[H]
	\centering
	\begin{subfigure}{4cm}
		\centering
		\begin{asy}
			import patterns;

			pair n(pair a) {return a / length(a);}

			add("hatch",hatch(2mm, SW, red));
			size(4cm);

			draw(circle((0,0), 1));
			dot('$(a_0, b_0)$', (0,0), align=S);

			draw((0,0) -- n((-1, 1)), dashed);
			label("$r$", n((-1, 1)) / 2, align=1.5S);

			pair A = n((1,3)) * (2/3);
			real rho = (1 - length(A)) * (2 / 3);

			dot("$(a,b)$", A, red, align=3SE);
			filldraw(circle(A, rho), pattern("hatch"), red);

			label("$O$", n((1,-1))*2.5/3);
		\end{asy}
	\end{subfigure}
	\begin{subfigure}{1cm}
		\centering~\\
	\end{subfigure}
	\begin{subfigure}{5cm}
		\centering
		\begin{asy}
			import patterns;

			pair n(pair a) {return a / length(a);}

			add("hatch",hatch(1mm, SW, red));
			add("hatch2",hatch(3mm, SE, blue));
			size(5cm);

			guide around = (-1.5, -1.5) -- (-1.5, 1.5) -- (2.5, 1.5) -- (2.5, -1.5) -- cycle;

			pair A = n((3, 1)) * 5/3; 
			real rho = 2 / 9;

			picture inter;
			fill(inter, around, pattern("hatch2"));
			fill(inter, circle((0,0), 1), white);
			add(inter);

			draw(circle((0,0), 1));
			dot('$(a_0, b_0)$', (0,0), align=S);

			draw((0,0) -- n((-1, 1)), dashed);
			label("$r$", n((-1, 1)) / 2, align=1.5S);

			dot("$(a,b)$", A, red, align=2SE);
			filldraw(circle(A, rho), pattern("hatch"), red);

			label("$F$", n((1,-1))*2.5/3);
		\end{asy}
	\end{subfigure}
\end{figure}

\begin{prv}
	$\O$ est un ouvert.

	Soit $B$ la boule ouverte de centre $(a_0, b_0) \in \R^2$ et de rayon $r > 0$.

	On pose $\rho = \frac{1}{2}\big(r - \|(a,b) - (a_0,b_0)\|\big)$.
	Montrons que \[
		B_{(a,b)}(\rho) \subset  B_{(a,b)}(r).
	\]

	Soit $(x,y) \in B_{(a,b)}(\rho)$.
	\begin{align*}
		\|(x,y) - (a_0,b_0)\|&= \|(x,y)- (a,b) + (a,b) - (a_0,b_0)\| \\
		&\le \|(x,y) - (a,b)\| + \|(a,b) - (a_0, b_0)\|\\
		&< \rho + \|(a,b) - (a_0, b_0)\| = \frac{1}{2}r + \frac{1}{2} \|(a,b) - (a_0, b_0)\|\\
		&< r
	\end{align*}
	
	Soit $F$ la boule fermée de centre $(a_0, b_0)$ et de rayon $r \ge 0$.

	Soit $(a,b) \not\in F$. On pose \[
		\rho = \frac{1}{2}\big(\|(a,b) - (a_0, b_0)\| - r\big) > 0.
	\]

	Montrons que $B_{(a,b)}(\rho) \subset \R^2\setminus F$.

	Soit $(x,y) \in B_{(a,b)}(\rho)$.

	\begin{align*}
		\|(x,y) - (a_0, b_0)\| &= \|(x,y) - (a,b) + (a,b) - (a_0, b_0)\| \\
		&\ge \big| \underbrace{\|(x,y) - (a,b)\|}_{\le \rho} - \underbrace{\|(a,b) - (a_0, b_0)\|}_{> r} \big|\\
		&\ge \|(a,b) - (a_0, b_0)\|- \|(x,y) - (a,b)\|\\
		&> \|(a,b) - (a_0, b_0)\|- \rho\\
		&> \frac{1}{2} \|(a,b) - (a_0, b_0)\| + \frac{1}{2}r\\
		&> r
	\end{align*}

	donc $(x,y) \not\in F$.
\end{prv}

\begin{exm}
	\begin{enumerate}
		\item $\O$ est ouvert.\\
			$\R^2$ est ouvert.
		\item $\O$ est fermé.\\
			$\R^2$ est fermé.\\
		\item $\big\{(x,0)  \mid x > 0\big\}$ n'est ni ouverte ni fermé.
	\end{enumerate}
\end{exm}

\begin{figure}[H]
	\centering
	\begin{asy}
		size(3cm);

		draw((0, -1) -- (0, 3), Arrow(TeXHead));
		draw((-1, 0) -- (3, 0), Arrow(TeXHead));
		
		draw((0,0) -- (0, 2.97), red);
		draw(circle((0,1.5), 0.5), deepred);
		draw(circle((0,0.5), 0.1), deepred);
	\end{asy}
\end{figure}

\begin{defn}
	Soit $(a,b) \in \R^2$ et $V \in \mathcal{P}(\R^2)$.

	On dit que $V$ est un voisinage de $(a,b)$ s'il existe $r > 0$ tel que \[
		B_{(a,b)}(r) \subset V.
	\]
	\index{voisinage (dans $\R^2$)}
\end{defn}

\begin{prop}
	Un ouvert non vide est un voisinage en chacun de ces points. \qed
\end{prop}

\begin{defn}
	Soit $D \subset \R^2$. Un \underline{point intérieur} de $D$ est un couple $(a,b) \in D$ tel que \[
		\exists r > 0, B_{(a,b)}(r) \subset D.
	\] en d'autres termes, si $D$ est un voisinage de $(a,b)$.

	On note $\mathring D$ l'ensemble des points intérieurs à $D$. C'est \underline{l'intérieur} de $D$.
	\index{point intérieur (dans $\R^2$)}
	\index{intérieur (dans $\R^2$)}
\end{defn}

\begin{prop}
	$\mathring D$ est le plus grand ouvert $O$ de $\R^2$ tel que $O \subset D$.
\end{prop}

\begin{figure}[H]
	\centering
	\incfig{interieur-plus-grand-ouvert}
\end{figure}


\begin{prv}
	Soit $(a,b) \in \mathring D$.

	Par définition, il existe $r > 0$ tel que \[
		B_{(a,b)}(r) \subset D.
	\] Montrons que $B_{(a,b)}(r) \subset \mathring D$.

	Soit $(x,y) \in B_{(a,b)}(r)$. Comme $B_{(a,b)}(r)$ est un ouvert de $\R^2$, il existe $\rho > 0$ tel que \[
		B_{(x,y)}(\rho) \subset B_{(a,b)}(r)
	\] donc $(x,y) \in \mathring D$.

	Donc $\mathring D$ est ouvert, $\mathring D \subset D$.

	Soit $O$ un ouvert de $\R^2$ tel que $O \subset D$. Montrons que $O \subset \mathring D$.

	Soit $(x,y) \in O$. Soit $r > 0$ tel que \[
		B_{(x,y)}(r) \subset O \subset D
	\] donc $(x,y) \in \mathring D$.
\end{prv}

\begin{defn}
	Soit $f: D \subset \R^2 \to \R$, $\ell \in \R$, $(a,b) \in \mathring D$.

	On dit que \underline{$f(x,y)$ tend vers $\ell$ quand $(x,y)$ tend vers $(a,b)$} ou que $\ell$ est \underline{une limite} de $f$ en $(a,b)$ si \[
		\forall \varepsilon > 0, \exists r > 0, \forall (x,y) \in D, \|(x,y) - (a,b)\| < r \implies \left| f(x,y) - \ell \right| \le \varepsilon.
	\] en d'autres termes si \[
		\forall V \in \mathcal{V}_{\ell}, \exists W \in \mathcal{V}_{(a,b)}, \forall (x,y) \in W \cap D, f(x,y) \in V.
	\]
	\index{limite (dans $\R^2$)}
	\index{tendre vers (dans $\R^2$)}
\end{defn}

\begin{prop}
	[unicité de la limite]
	Soit $f: D \to \R$, $(a,b) \in \mathring D$, $\ell_1, \ell_2 \in \R$ telles que $\ell_1$ et $\ell_2$ sont des limites de $f$ en $(a,b)$.

	Alors $\ell_1 = \ell_2$.
\end{prop}

\begin{figure}[H]
	\centering
	\incfig{preuve-unicité-de-la-limite}
\end{figure}

\begin{prv}
	On suppose $\ell_1 < \ell_2$. On pose $\varepsilon = \frac{\ell_2 - \ell_1}{2} > 0$.

	Soit $r_1 > 0$ tel que \[
		f\big(B_{(a,b)}(r_1)\big) \subset ]\ell_1 - \varepsilon, \ell_1 + \varepsilon[.
	\] Soit $r_2 > 0$ tel que \[
		f\big(B_{(a,b)}(r_2)\big) \subset ]\ell_2 - \varepsilon, \ell_2 + \varepsilon [.
	\] On pose $r = \min(r_1, r_2)$ donc \[
		B_{(a,b)}(r_1) \cap B_{(a,b)}(r_2) = B_{(a,b)}(r) \neq \O.
	\] Soit $(x,y) \in B_{(a,b)}(r)$. Alors, \[
		f(x,y) \in ]\ell_1 - \varepsilon, \ell_1 + \varepsilon[ \cap ]\ell_2 - \varepsilon, \ell_2 + \varepsilon[ = \O.
	\] $\lightning$
\end{prv}

\begin{defn}
	Soit $f : D \to \R$, $(a,b) \in \mathring D$.

	On dit que $f$ est \underline{continue} en $(a,b)$ si \[
		f(x,y) \tendsto{(x,y) \to (a,b)}f(a,b).
	\]
	\index{continuité (dans $\R^2$)}
\end{defn}

\begin{prop}
	\underline{Si} $f(x,y) \tendsto{(x,y) \to (a,b)} \ell$ \\
	\underline{alors} $\begin{cases}
		f(x,b) \tendsto{x \to a} \ell\\
		f(a,y) \tendsto{y \to b} \ell.\\
	\end{cases}$
\end{prop}

\begin{prv}~\\
	\begin{figure}[H]
		\centering
		\incfig{limite-x-en-a-et-y-en-b}
	\end{figure}
\end{prv}

\underline{Contre-exemple} : exercice 3.

\begin{exm}
	\begin{enumerate}
		\item $f : \begin{array}{rcl}
				\R^2 &\longrightarrow& \R \\
				(x,y) &\longmapsto& x
			\end{array}$ limite en $(0,0)$ ?

			Soit $\varepsilon > 0$. On pose $r = \varepsilon$. \[
				\forall (x,y) \in B_{(0,0)}(r),
				\left| f(x,y) \right| = \left| x \right| \le \|(x,y)\| < r = \varepsilon
			\] Donc $f(x,y) \tendsto{(x,y) \to (a,b)} 0$.
		\item limite $f : \begin{array}{rcl}
				\R^2 &\longrightarrow& \R \\
				(x,y) &\longmapsto& x^3
			\end{array}$ en $(0,0)$ ?

			Soit $\varepsilon > 0$. On pose $r = \sqrt[3]{r} > 0$. \[
				\forall (x,y) \in B_{(0,0)}(r),
				\left| f(x,y) \right| = \left| x^3 \right| \le \|(x,y)\|^3 < r^3 = \varepsilon.
			\]
		\item limite de $f : \begin{array}{rcl}
			\R^2 &\longrightarrow& \R \\
			(x,y) &\longmapsto& x^3y^2
		\end{array}$ en $(0,0)$ ?

		Soit $\varepsilon > 0$. On pose $r = \sqrt[5]{\varepsilon} > 0$. \[
			\forall (x,y) \in B_{(0,0)}(r), \left| f(x,y) \right| = \left| x^3 y^2 \right| \le \|(x,y)\|^3 \|(x,y)\|^2 < r^5 = \varepsilon.
		\]
	\end{enumerate}
\end{exm}

\begin{defn}
	Soient $D \subset \R^2$ et $(x,y) \in \R^2$.

	\begin{figure}[H]
    \centering
    \incfig{point-adhérent}
	\end{figure}
	
	On dit que $(x,y)$ est \underline{adhérent} à $D$ si \[
		\forall r > 0, B_{(x,y)}(r) \cap D \neq \O.
	\] L'ensemble des points adhérents à $D$ est noté $\overline{D}$. On dit que $\overline{D}$ est \underline{l'adhérence} de $D$.
	\index{point adhérent (dans $\R^2$)}
	\index{adhérent (dans $\R^2$)}
\end{defn}

\begin{defn}
	Soit $f: D \subset \R^2 \to \R$ et $(a,b) \in \overline{D}$, $\ell \in \R$. On dit que $f$ tend vers $\ell$ quand $(x,y)$ tend vers $(a,b)$ si \[
		\forall \varepsilon > 0, \exists r > 0, \forall (x,y) \in B_{(a,b)}(r) \cap D,
		\left| f(x,y) - \ell \right| \le \varepsilon.
	\]
	\index{limite (dans $\R^2$)}
	\index{tendre vers (dans $\R^2$)}
\end{defn}

\begin{prop}
	\begin{enumerate}
		\item Dans ce contexte, il y a unicité de la limite
		\item La limite d'une somme, d'un produit, d'un quotien, d'une composée se comporte comme dans le cas d'une seule variable.
		\item Soit $f: D \to \R$ continue. Soient $g: I \to \R$ et $h: I \to \R$ continues telles que \[
			\forall t \in I, \big(g(t), h(t)\big) \in D.
		\] Alors \[
			t \in I \mapsto f\big(g(t), h(t)\big) \in \R
		\] est continue.
	\end{enumerate}
\end{prop}

\begin{figure}[H]
	\centering
	\begin{asy}
		import three;
		import graph3;
		size(5cm);

		settings.render = 0;
		settings.prc = false;
		currentprojection = obliqueX;

		draw(O -- X, Arrow3(TeXHead2));
		draw(O -- Y, Arrow3(TeXHead2));
		draw(O -- Z, Arrow3(TeXHead2));

		triple f(real x, real y, real z = 0) { return (x,y,cos(x - 0.5) * cos(y - 0.5)/1.2 + 0.15); }

		real inc = 1 / 5;

		for(real x = 0; x <= 1; x += inc) {
			draw(graph(
				new real(real t) { return x; }, // x
				new real(real y) { return y; }, // y
				new real(real y) { return f(x,y).z; }, // z
				0, 1
			), gray);
		}

		for(real y = 0; y <= 1; y += inc) {
			draw(graph(
				new real(real x) { return x; }, // x
				new real(real t) { return y; }, // y
				new real(real x) { return f(x,y).z; }, // z
				0, 1
			), gray);
		}

		path3 path1 = (0.3, 0.2, 0) .. (0.5, 0.5, 0) .. (0.6, 0.7, 0) .. (0.9, 0.8, 0);
		path3 path2 = (0.3, 0.8, 0) .. (0.5, 0.5, 0) .. (0.6, 0.3, 0) .. (0.9, 0.2, 0);
		path3 pathA = f(0.3, 0.2, 0) .. f(0.5, 0.5, 0) .. f(0.6, 0.7, 0) .. f(0.9, 0.8, 0);
		path3 pathB = f(0.3, 0.8, 0) .. f(0.5, 0.5, 0) .. f(0.6, 0.3, 0) .. f(0.9, 0.2, 0);

		draw(path1, red, Arrow3(TeXHead2, position=0.5));
		draw(pathA, red, Arrow3(TeXHead2, position=0.5));
		draw(path2, deepcyan, Arrow3(TeXHead2, position=0.3));
		draw(pathB, deepcyan, Arrow3(TeXHead2, position=0.3));

		dot((0.5, 0.5, 0));
		dot(f(0.5, 0.5, 0));
		draw((0.5, 0.5, 0) -- f(0.5, 0.5, 0), dashed);
	\end{asy}
\end{figure}


	\part{Transpositions}

\begin{defn}
	Une \underline{transposition} est un cycle de longueur 2 : $\begin{pmatrix}
		a&b
	\end{pmatrix}$ avec $a \neq b$.
	\index{transposition (permutation)}
\end{defn}

\begin{exm}
	Avec $n = 5$ et $\gamma = \begin{pmatrix}
		2&4&1
	\end{pmatrix}$.

	\begin{figure}[H]
		\centering

		\begin{asy}
			size(5cm);

			real rho = 0.15; // circles

			void draw_cycle(pair O, real r ...int[] nums) {
				int n = nums.length;
				real eps = (15 / r) * 0.8;

				for(int i = 0; i < n; ++i) {
					real theta_1 = (360/n) * (i+1);
					real theta_2 = (360/n) * i;

					pair C = O + dir(theta_2) * r;

					draw(circle(C, rho));
					label("$" + string(nums[i]) + "$", C);
					draw(arc(O, r, theta_2+eps, theta_1-eps), Arrow(TeXHead));
				}
			}

			draw_cycle((-1,0), 0.8, 1, 2, 4);
			draw_cycle((1,0), 0.3, 3);
			draw_cycle((2,0), 0.3, 5);
		\end{asy}
	\end{figure}

	\[
		\gamma = \begin{pmatrix}
			1&4
		\end{pmatrix} \begin{pmatrix}
			1&2
		\end{pmatrix}
	\]

	Avec $n = 6$ et $\gamma = \begin{pmatrix}
		1&3&5&6&2
	\end{pmatrix} = \begin{pmatrix}
		1&2&3&4&5&6\\
		3&1&5&4&6&2
	\end{pmatrix}$.

	Donc, \[
		\gamma = \begin{pmatrix}
			1&2
		\end{pmatrix} \begin{pmatrix}
			1&6
		\end{pmatrix} \begin{pmatrix}
			1&5
		\end{pmatrix} \begin{pmatrix}
			1&3
		\end{pmatrix}
	\] 
	\[
		\begin{pmatrix}
			1&2&3&4&5&6\\
			3&2&1&4&5&6\\
			3&2&5&4&1&6\\
			3&2&5&4&6&1\\
			3&1&5&4&6&2\\
		\end{pmatrix}
	\]

	Et, \[
		\gamma = \begin{pmatrix}
			1&3
		\end{pmatrix} \begin{pmatrix}
			2&3
		\end{pmatrix} \begin{pmatrix}
			3&5
		\end{pmatrix} \begin{pmatrix}
			5&6
		\end{pmatrix} 
	\]

	\[
		\begin{pmatrix}
			1&2&3&4&5&6\\
			1&2&3&4&6&5\\
			1&2&5&4&6&3\\
			1&3&5&4&6&2\\
			3&1&5&4&6&2\\
		\end{pmatrix} 
	\] 
\end{exm}

\begin{exm}
	\[
		\begin{pmatrix}
			1&4
		\end{pmatrix} = \begin{pmatrix}
			1&2
		\end{pmatrix} \begin{pmatrix}
			2&3
		\end{pmatrix} \begin{pmatrix}
			3&4
		\end{pmatrix} \begin{pmatrix}
			2&3
		\end{pmatrix} \begin{pmatrix}
			1&2
		\end{pmatrix}
	\]
	On n'a pas toujours le même nombre de transpositions mais la parité du nombre reste la même (proposition plus loin).
\end{exm}

\begin{thm}
	Toute permutation se décompose en produit de transpositions.
\end{thm}

\begin{prv}
	Soit $\gamma = \begin{pmatrix}
		a_1&\cdots&a_k
	\end{pmatrix}$ un $k$-cycle.

	On remarque que
	\[
		\gamma = \begin{pmatrix}
			a_1&a_k
		\end{pmatrix} \cdots \begin{pmatrix}
			a_1&a_4
		\end{pmatrix} \begin{pmatrix}
			a_1&a_3
		\end{pmatrix} \begin{pmatrix}
			a_1&a_2
		\end{pmatrix}
	\] C'est un produit de transpositions.
\end{prv}

\begin{exm}
	Avec $n = 10$ et $\sigma = \begin{pmatrix}
		1&2&3&4&5&6&7&8&9&10\\
		9&8&1&7&2&3&4&5&10&6
	\end{pmatrix}$.

	On a
	\begin{align*}
		\sigma &= \begin{pmatrix}
			1&9&10&6&3
		\end{pmatrix} \begin{pmatrix}
			2&8&5
		\end{pmatrix} \begin{pmatrix}
			4&7
		\end{pmatrix}\\
		&= \begin{pmatrix}
			1&3
		\end{pmatrix} \begin{pmatrix}
			1&6
		\end{pmatrix} \begin{pmatrix}
			1&10
		\end{pmatrix} \begin{pmatrix}
			1&9
		\end{pmatrix} \begin{pmatrix}
			2&5
		\end{pmatrix} \begin{pmatrix}
			2&8
		\end{pmatrix} \begin{pmatrix}
			4&7
		\end{pmatrix} \\
	\end{align*}

	Vérification : \[
		\begin{pmatrix}
			1&2&3&4&5&6&7&8&9&10\\
			1&2&3&7&5&6&4&8&9&10\\
			1&8&3&7&5&6&4&2&9&10\\
			1&8&3&7&2&6&4&5&9&10\\
			9&8&3&7&2&6&4&5&1&10\\
			9&8&3&7&2&6&4&5&10&1\\
			9&8&3&7&2&1&4&5&10&6\\
			9&8&1&7&2&3&4&5&10&6\\
		\end{pmatrix} 
	\] 
\end{exm}

	\part{Familles orthogonales}

\begin{thm}[Pythagore]
	Soit $(x,y) \in E^2$. \[
		\|x+y\|^2 = \|x\|^2 + \|y\|^2 \iff x \perp y
	.\]
	\begin{figure}[H]
		\centering
		\begin{asy}
			size(4cm);
			pair u = (1, 0.5);
			pair v = 1.5 * (0, 1) * u;
			draw((0,0)--u, Arrow(TeXHead));
			label("$x$", u/2, align=S);
			draw(u--v+u, Arrow(TeXHead));
			label("$y$", u + v/2, align=NE);
			draw((0,0) -- u + v, Arrow(TeXHead));
			draw(u + v / 7.5 -- u + v / 7.5 - u / 5 -- u - u / 5 -- u -- cycle);
		\end{asy}
	\end{figure}
\end{thm}

\begin{prv}
	\[
		\|x + y\|^2 = \|x\|^2 + \|y\|^2 \iff 2\left<x \mid y \right> = 0 \iff x \perp y
	.\]
\end{prv}

\begin{defn}
	Soit $(e_i)_{i\in I}$ une famille de vecteurs. On dit que cette famille est \underline{orthogonale} si \[
		\forall i \neq j\, e_i \perp e_j
	.\] Si, en plus, on a \[
		\forall i \in I,\,\|e_i\| = 1,
	\] alors on dit que la famille est \underline{orthonormale} ou \underline{orthonormée}.
	\index{famille orthogonale}
	\index{famille orthonormale}
	\index{famille orthonormée}
\end{defn}

\begin{prop}[Pythagore]
	Soit $(e_1, \ldots, e_n)$ une famille orthogonale. Alors \[
		\left\| \sum_{i=1}^n e_i \right\|^2 = \sum_{i=1}^n \|e_i\|^2
	.\]
\end{prop}

\begin{thm}
	Toute famille orthogonale de vecteurs non nuls est libre.
\end{thm}

\begin{prv}
	Soit $(e_i)_{i\in I}$ une famille orthogonale telle que \[
		\forall i \in I,\,e_i \neq 0_E
	.\] Soit $n \in \N^*$, $(\lambda_1, \ldots, \lambda_n) \in \R^n$. On suppose \[
		\sum_{k=1}^n \lambda_k e_{i_k} = 0_E
	.\] Soit $j \in \left\llbracket 1,n \right\rrbracket$.
	\begin{align*}
		0 &= \left<\sum_{k=1}^n \lambda_k e_{i_k}  \mid e_{i_j} \right>\\
		&= \sum_{k=1}^n \lambda_k \left<e_{i_k}  \mid e_{i_j} \right> \\
		&= \lambda_j \underbrace{\|e_{i_j}\|^2}_{\neq 0} \\
	\end{align*}
	donc $\lambda_j = 0$.
\end{prv}

\begin{algo}[Orthonormalisation de Gran--Schmidt]
	On suppose $E$ de dimension finie. Soit $\mathcal{B} = (e_1, \ldots, e_n)$ une base de $E$.

	\begin{itemize}
		\item\underline{\it Étape 1}: On pose $v_1 = \frac{e_1}{\|e_1\|}$ de sorte que $\|v_1\| = 1$.
		\item\underline{\it Étape 2} : On pose \[
				u_2 = e_2 - \left<e_2  \mid v_1 \right> v_1
			.\] Ainsi,
			\begin{align*}
				\left<u_2 \mid v_1 \right> &= \big<e_2 - \left<e_2 \mid v_1 \right> v_1  \mid v_1 \big>\\
				&= \left<e_2 \mid v_1 \right> - \left<e_2 \mid v_1 \right> \left<v_1 \mid v_1 \right> \\
				&= 0. \\
			\end{align*}
			On pose $v_2 = \frac{u_2}{\|u_2\|}$ donc $v_2 \perp v_1$ et $\|v_2\| = 1$.
		\item\underline{\it Étape 3} : On pose \[
				u_2 = e_3 - \left<e_3 \mid v_1 \right>v_1 - \left<e_3 \mid v_2 \right>v_2
			.\] Ainsi,
			\begin{align*}
				\left<u_3  \mid v_1 \right> &= \left<e_3  \mid v_1 \right> - \left<e_3 \mid v_1 \right>\underbrace{\left<v_1 \mid v_1 \right>}_{=1} - \left<e_3 \mid v_2 \right>\underbrace{\left<v_2 \mid v_1 \right>}_{=0} \\
				&= 0 \\
			\end{align*}
			et 
			\begin{align*}
				\left<u_3 \mid v_2 \right> &= \left<e_3  \mid  v_2 \right> - \left<e_3 \mid v_1 \right> \underbrace{\left<v_1 \mid v_2 \right>}_{=0} - \left<e_3 \mid v_2 \right> \underbrace{\left<v_2 \mid v_2 \right>}_{=1}\\
				&= 0. \\
			\end{align*}
			On pose $v_3 = \frac{u_3}{\|u_3\|}$ de sorte que $v_3 \perp v_1$, $v_3 \perp v_2$ et $\|v_3\| = 1$.
		\item\underline{\it Étape $i+1$}: On pose \[
			u_{i+1} = e_{i+1} - \sum_{k=1}^i \left<e_{i+1}  \mid v_k \right> v_k
		.\] Ainsi, pour tout $j \in \left\llbracket 1,i \right\rrbracket,$ on a
		\begin{align*}
			\left<u_{i+1}  \mid v_j \right> &= \left<e_{i+1}  \mid v_j \right> - \sum_{k=1}^i \left<e_{i+1} \mid v_k \right> \left<v_k \mid v_j \right> \\
			&= \left<e_{i+1} \mid v_j \right> - \left<e_{i+1} \mid v_j \right> \|v_j\|^2 \\
			&= 0. \\
		\end{align*}
		On pose $v_{i+1} = \frac{u_{i+1}}{\|u_{i+1}\|}$.
	\end{itemize}
\end{algo}

\begin{exm}
	Avec $E = \R_3[X]$, $\left<P \mid Q \right> = \int_{0}^{1} P(t)\,Q(t)~\mathrm{d}t$ et $\mathcal{B} = (1, X, X^2, X^3)$.
	\begin{enumerate}
		\item $\|1\|^2 = \left<1 \mid 1 \right> = \int_{0}^{1} 1~\mathrm{d}t = 1$ et donc $v_1 = 1$.
		\item $u_2 = X - \left<X  \mid v_1 \right>v_1$. Or, $\left<X \mid v_1 \right> = \int_{0}^{1} t~\mathrm{d}t = \frac{1}{2}$. D'où $u_2 = X - \frac{1}{2}$.
			\begin{align*}
				\|u_2\|^2 &= \int_{0}^{1} \left( t - \frac{1}{2} \right)^2~\mathrm{d}t \\
				&= \int_{0}^{1} \left( t^2 - t + \frac{1}{4} \right)~\mathrm{d}t \\
				&= \frac{1}{3} - \frac{1}{2} + \frac{1}{4} \\
				&= \frac{1}{12} \\
			\end{align*} On en déduit que $v_2 = \sqrt{12}\left( X - \frac{1}{2} \right)$.
		\item $u_3 = X^2 - \left<X^2 \mid v_1 \right>v_1 - \left<X^2 \mid v_2 \right>v_2$.
			On a \[
				\left<X^2 \mid v_1 \right> = \int_{0}^{1} t^2~\mathrm{d}t = \frac{1}{3}
			\] et
			\begin{align*}
				\left<X^2 \mid v_2 \right> &= \sqrt{12} \int_{0}^{1} t^2\left( t - \frac{1}{2} \right)~\mathrm{d}t \\
				&= \frac{\sqrt{12}}{12}. \\
			\end{align*}
			D'où
			\begin{align*}
				u_3 &= X^2 - \frac{1}{3} - \frac{\sqrt{12}}{12}\sqrt{12} \left( X - \frac{1}{2} \right)\\
				&= X^2 - \frac{1}{3} - X + \frac{1}{2} \\
				&= X^2 - X + \frac{1}{6}. \\
			\end{align*}
			\begin{align*}
				\|u_3\|^2 &= \int_{0}^{1} \left( t^2 - t + \frac{1}{6} \right)~\mathrm{d}t\\
				&= \int_{0}^{1} \left( t^4 + t^2 + \frac{1}{36} - 2t^3 + \frac{t^2}{3} - \frac{t}{3} \right) ~\mathrm{d}t \\
				&= \frac{1}{5} + \frac{1}{3} + \frac{1}{36} - \frac{1}{2} + \frac{1}{9} - \frac{1}{6} \\
				&= \frac{36 + 60 + 5 - 90 + 20 - 30}{180} \\
				&= \frac{1}{180} \\
			\end{align*}
			On en déduit que \[
				v_3 = 6\sqrt{5}\left( X^2 - X + \frac{1}{6} \right).
			\]
		\item Exercice : calculer $v_4$.
	\end{enumerate}
\end{exm}

\begin{prop}
	Soit $\mathcal{B} = (e_1, \ldots, e_n)$ une base de $E$ et $\mathcal{C}$ la base obtenue par le procédé d'orthonormalisation de Gram--Schmidt. Alors, \[
		\forall i \in \left\llbracket 1,n \right\rrbracket,\,\Vect(e_1,\ldots, e_i) = \Vect(v_1, \ldots, v_i)
	.\]\qed
\end{prop}

\begin{exm}[orthogonalisation]
	\begin{itemize}
		\item $u_1 = 1$.
		\item
			\begin{align*}
				\begin{rcases*}
					u_2 \in \Vect(e_1, e_2)\\
					u_2 \perp u_1
				\end{rcases*}
				\iff& \begin{cases}
					u_2 = ae_1 + be_2\quad (a,b) \in \R^2\\
					\left<u_1 \mid u_2 \right> = 0
				\end{cases}\\
				\iff& \begin{cases}
					u_2 = a + bX\\
					\int_{0}^{1} (a+bt)~\mathrm{d}t = 0.
				\end{cases}\\
			\end{align*}
			\begin{align*}
				\int_{0}^{1} (a+bt)~\mathrm{d}t = 0 \iff& a + \frac{b}{2} = 0\\
				\iff& a = -\frac{b}{2}\\
				\iff& u_2 = -\frac{b}{2} + bX.
			\end{align*}
			Par exemple, $u_2 = -1 + 2X$.
		\item $\begin{cases}
				u_3 \in \Vect(e_1, e_2, e_3)\\
				u_3 \perp u_1\\
				u_3 \perp u_2
			\end{cases}$

			On pose $u_3 = a + bX + cX^2$ avec $(a,b,c)\in \R^3$.
			\begin{align*}
				\begin{rcases*}
					\int_{0}^{1} \left( a+bt + ct^2 \right)~\mathrm{d}t = 0\\
					\int_{0}^{1} \left(a + bt+ct^2\right)(2t - 1)~\mathrm{d}t = 0
				\end{rcases*} \iff& \begin{cases}
					a + \frac{b}{2} + \frac{c}{3} = 0\\
					\int_{0}^{1} \left( 2ct^3 + (-c + 2b)t^2 + (2a - b)t - a \right) ~\mathrm{d}t = 0
				\end{cases}\\
				\iff& \begin{cases}
					a + \frac{b}{2} + \frac{c}{3} = 0\\
					\frac{c}{2} + \frac{2b - c}{3} + \frac{2\cancel{a} - b}{2} - \cancel{a} = 0
				\end{cases}\\
				\iff&  \begin{cases}
					a = -\frac{b}{2} - \frac{c}{3} = \frac{c}{2} - \frac{c}{3} = \frac{c}{6}\\
					b = -c.
				\end{cases}
			\end{align*}
			On en déduit que \[
				u_3 = 1 - 6X + 6X^2
			.\]
	\end{itemize}
\end{exm}

\begin{crlr}[théorème de la base orthonormée incomplète] Soit $(e_1, \ldots, e_k)$ une base orthonormée d'un espace euclidien. On peut trouver $e_{k+1},\ldots,e_n$ tels que $(e_1, \ldots, e_k, e_{k+1},\ldots,e_n)$ soit une base orthonormée de $E$.
\end{crlr}

\begin{prv}
	On sait que $(e_1, \ldots, e_k)$ est libre. On complète $(e_1, \ldots, e_k)$ en une base $\mathcal{B}$ de $E$. On orthonormalise $\mathcal{B}$ : on obtient une base orthonormée $\mathcal{C}$ de $E$. En détaillant l'algorithme de Gram--Schmidt, on s'aper\c coit que les $k$ premiers vecteurs de $\mathcal{C}$ sont ceux de $\mathcal{B}$.
\end{prv}

\begin{thm}
	Soit $E$ un espace euclidien et $\mathcal{B} = (e_1, \ldots, e_n)$ une base orthonormée de $E$. Soit $(x,y) \in E^2$. On pose $(x_1, \ldots, x_n) \in \R^n$ et $(y_1, \ldots, y_n) \in \R^n$ tels que \[
		x = \sum_{i=1}^n x_i e_i \qquad\qquad y = \sum_{i=1}^n y_i e_i
	.\] Alors \[
		\left<x \mid y \right> = \sum_{i=1}^n x_i y_i
	.\]
	\vspace{3mm}
	Soit $X = \mat{x_1\\\vdots\\x_n}$ et $Y = \mat{y_1\\ \vdots \\ y_n}$. Alors, \[
		\left<x \mid y \right> = X^\T\,Y
	.\]
\end{thm}

\begin{prv}
	\begin{align*}
		\left<x \mid y \right> &= \left<\sum_{i=1}^n x_ie_i  \mid y \right>\\
		&= \sum_{i=1}^n x_i \left<e_i  \mid y \right> \\
		&= \sum_{i=1}^n x_i \left<e_i  \mid \sum_{j=1}^n y_j e_j \right> \\
		&= \sum_{i=1}^n x_i \sum_{j=1}^n y_j \underbrace{\left<e_i \mid e_j \right>}_{\delta_i^j} \\
		&= \sum_{i=1}^n x_i y_i. \\
	\end{align*}
\end{prv}

\begin{prop}
	Soit $E$ un espace euclidien et $\mathcal{B} = (e_1, \ldots, e_n)$ une base orthonormée de $E$. Alors, \[
		\forall x \in E,\,x = \sum_{i=1}^n \left<x \mid e_i \right>e_i
	.\]
\end{prop}

\begin{prv}
	Soit $x \in E$. On pose \[
		x = \sum_{i=1}^n x_i e_i
	\] avec $(x_1, \ldots, x_n) \in \R^n$. Soit $j \in \left\llbracket 1,n \right\rrbracket$. On a
	\begin{align*}
		\left<x \mid e_j \right> &= \left<\sum_{i=1}^n x_i e_i  \mid e_j \right>\\
		&= \sum_{i=1}^n x_i \left<e_i \mid e_j \right> \\
		&= x_j. \\
	\end{align*}
\end{prv}

	\part{Lois de composition}

\begin{defn}
	Une \underline{loi de composition interne} \index{loi de composition interne} est une application $f$ de $E \times E$ dans $E$.
	
	On la note $x * y$ au lieu de $f(x,y)$ (on est libre de choisir le symbôle).
\end{defn}

\begin{defn}
	Soit $E$ un ensemble muni d'une loi de composition interne $\boxtimes$.

	On dit que $\boxtimes$ est \underline{associative} \index{associativité (loi de composition interne)} si \[
		\forall (x,y,z) \in E^3,\;(x\boxtimes y)\boxtimes z = x \boxtimes (y \boxtimes z).
	\] Dans ce cas, on écrit plutôt $x \boxtimes y \boxtimes z$.
\end{defn}

\begin{exm}
	\begin{itemize}
		\item $+$ et $\times $ dans $\C$ sont associatives;
		\item $ \circ$ est associative;
		\item  la multiplication matricielle est aussi associative.
	\end{itemize}
\end{exm}

\begin{defn}
	On dit que $\boxtimes$ est \underline{commutative} \index{commutativité (loi de composition interne)} si \[
		\forall (x,y) \in E^2, x\boxtimes y = y\boxtimes x.
	\]
\end{defn}

\begin{exm}
	\begin{itemize}
		\item $+$ et $\times $ dans $\C$ sont commuatives;
		\item $ \circ $ n'est pas commutative;
		\item  la multiplication matricielle n'est pas commutative.
	\end{itemize}
\end{exm}

\begin{defn}
	Soit $e \in E$. On dit que $e$ est un
	\begin{itemize}
		\item \underline{élément neutre à gauche}\index{élément neutre à gauche (loi de composition interne)} si  \[
				\forall x \in E,\; e\boxtimes x = x;
			\]
		\item \underline{élément neutre à droite}\index{élément neutre à droite (loi de composition interne)} si  \[
				\forall x \in E,\; x\boxtimes e = x;
			\]
		\item \underline{élément neutre}\index{élément neutre (loi de composition interne)} si  \[
				\forall x \in E,\; e\boxtimes x = x\boxtimes e = x.
			\]
	\end{itemize}
\end{defn}

\begin{prop}
	Sous reserve d'existence, il y a unicité de l'élément neutre.
\end{prop}

\begin{prv}
	Soient $e$ et $e'$ deux éléments neutre.
	\begin{itemize}
		\item $e \boxtimes e' = e'$ car $e$ est neutre,
		\item $e \boxtimes e' = e$ car $e'$ est neutre.
	\end{itemize} On a donc $e = e'$.
\end{prv}

\begin{axm}[axiome du choix]
	Soit $E$ un ensemble non vide. Il existe $f : \mathcal{P}(E) \setminus \{\O\} \to E$ telle que \[
		\forall A \in \mathcal{P}(E) \setminus \{\O\},\; f(A) \in A.
	\]
\end{axm}

\begin{defn}
	Soit $f: E \to F$. Le \underline{graphe} \index{graphe (application)} de $f$ est \[
		\Big\{\big(x,f(x)\big)  \mid x \in E\Big\} \subset E \times F.
	\]
\end{defn}

\begin{prop}
	Soit $G \subset E\times F$. $G$ est le graphe d'une application si et seulement si \[
		\forall x \in E,\,\exists! y \in F,\, (x,y) \in G.
	\]
\end{prop}

\begin{prv}
	\begin{itemize}
		\item[``$\implies$''] par définition d'une application
		\item[``$\impliedby$''] On pose $f(x)$ le seul élément $y$ de $F$ qui vérifie $(x,y) \in G$. Alors $f \in F^E$ et son graphe vaut $G$.
	\end{itemize}
\end{prv}

\begin{defn}
	Soit $A \in \mathcal{P}(E)$. L'\underline{indicatrice}\index{indicatrice (ensemble)} de $A$ est \begin{align*}
		\mathbbm{1}_A: E &\longrightarrow \{0,1\} \\
		x &\longmapsto \begin{cases}
			1 &\text{ si } x \in A,\\
			0 & \text{ si } x \not\in A.
		\end{cases}
	\end{align*}
\end{defn}

\begin{exm}
	\begin{enumerate}
		\item Dans $\C$, le neutre de $+$ est $0$ et le neutre de $\times$ est $1$.
		\item Dans $E^E$, le neutre de $ \circ $ est $\id_E$.
		\item Dans $\mathcal{M}_n(\C)$ (l'ensemble des matrices carrées $n \times n$ à valeurs dans $\C$), le neutre de $\times $ est $I_n$ : \[
				I_n =
				\begin{pNiceMatrix}
					1&&(0)\\
					&\Ddots&\\
					(0)&&1
				\end{pNiceMatrix}
			\] 
	\end{enumerate}
\end{exm}

\begin{defn}
	Soit $E$ un ensemble muni d'une loi de composition interne $\boxtimes$ et $x \in E$.

	\begin{enumerate}
		\item On dit que $x$ est \underline{simplifiable à gauche}\index{simplifiabilité à gauche} si \[
				\forall (y,z) \in E^2,\,(x\boxtimes y = x \boxtimes z) \implies x = z.
			\] et que $x$ est \underline{simplifiable à droite}\index{simplifiabilité à droite} si \[
				\forall (y,z) \in E^2,\,(y\boxtimes x = z \boxtimes y) \implies x = z.
			\]
		\item On dit que $x$ est \underline{symétrisable à gauche}\index{symétrisabilité à gauche} s'il exiiste $y \in E$ tel que $y\boxtimes x = e$ où $e$ est l'élément neutre de $\boxtimes$.

			De même, on dit que $x$ est \underline{symétrisable à droite}\index{symétrisabilité à droite} s'il existe $y \in E$ tel que $x \boxtimes y = e$.

			On dit que $x$ est \underline{symétrisable}\index{symétrisabilité} s'il est symétrisable à gauche et à droite, donc s'il existe $y \in E$ tel que $x \boxtimes y = y \boxtimes x = e$.
	\end{enumerate}
\end{defn}

\begin{exm}
	$E = \N$ muni de la loi $+$, tous les éléments de $E$ sont simplifiables. $0$ est le seuele élément de $E$ symétrisable.
\end{exm}

\begin{prop}
	Avec les notations précédentes, si $\boxtimes$ est associative, et $x$ est symétrisable, alors $x$ est simplifiable.
\end{prop}

\begin{prv}
	Soient $y, z \in E$.
	\begin{itemize}
		\item On suppose $x \boxtimes y = x \boxtimes z$. Soit $a \in E$ tel que $a\in E$ tel que $a \boxtimes x = e$. Alors \[
				a \boxtimes (x\boxtimes y) = a \boxtimes (x \boxtimes z).
			\] Or,
			\begin{align*}
				a \boxtimes (x \boxtimes y) &= (a \boxtimes x) \boxtimes y \\
				&= e \boxtimes y \\
				&= y. \\
			\end{align*}

			De même, $a \boxtimes (x \boxtimes z) = z$.

			Donc $y = z$.
		\item De même, si $y \boxtimes x = z \boxtimes x$, on ``multiplie'' $x$ à droite par $a$ et on obtient $y = z$.
	\end{itemize}
\end{prv}

\begin{prop-defn}
	On suppose $\boxtimes$ associative. Soit $x \in E$ symétrisable. Alors \[
		\exists ! y \in E,\; x \boxtimes y = y \boxtimes x = e.
	\] On dit que $y$ est le \underline{symétrique}\index{symétrique (loi de composition interne)} de $x$ et on le note $y = x^*$.
\end{prop-defn}

\begin{prv}
	Soeint $x,y,z \in E$ tels que \[
		\begin{cases}
			 x \boxtimes y = y \boxtimes x = e\\
			 x \boxtimes z = z \boxtimes x = e\\
		\end{cases}
	\] Alors, $x \boxtimes y = x \boxtimes z$ et, en simplifiant par $x$, on a $y = z$.
\end{prv}

\begin{exm}
	Les fonctions symétrisables de $(E^E,  \circ)$ sont les bijections et le symétrique d'une bijection est sa réciproque.
\end{exm}

\begin{rmk}
	\begin{enumerate}
		\item Si la loi est notée $+$, on parle d'\underline{opposé}\index{opposé (loi de composition interne)} plutôt que de symétrique et on le note $-x$ au lieu de $x^*$.
			L'élément neutre est noté $0_E$.
		\item Si la loi est notée $\times$, on parle d'élément \underline{inversible}\index{inversibilité (loi de composition interne)} au lieu de symétrisable, d'\underline{inverse}\index{inverse (loi de composition interne)} au lieu de symétrique et on note $x^{-1}$ au lieu de $x^*$. On note le neutre $1_E$.
	\end{enumerate}
\end{rmk}

\begin{exo}
	Soient $x,y \in E = \R^+_*$. On définit la loi de composition interne $\oplus$ : \[
		x \oplus y = \frac{1}{\frac{1}{x}\oplus \frac{1}{y}}.
	\] Cette loi peut-être utile en physique pour le calcul de résistances équivalentes en parallèles.
	\begin{itemize}
		\item {\sc Associativité} : soient $x,y,z \in E$.

			D'une part, on a \[
				x \oplus (y \oplus z) = \frac{1}{\frac{1}{x} + \frac{1}{\frac{1}{\frac{1}{x}+ \frac{1}{y}}}} = \frac{1}{\frac{1}{x}+\frac{1}{y}+\frac{1}{z}}.
			\] D'autre part, on a \[
			(x \oplus y) \oplus z = \frac{1}{\frac{1}{\frac{1}{\frac{1}{x}+\frac{1}{y}}}+\frac{1}{z}} = \frac{1}{\frac{1}{x}+ \frac{1}{y}+\frac{1}{z}}.
			\] La loi $\oplus$ est associative.
		\item {\sc Commutativité} : soient $x, y \in E$. \[
				x \oplus y = \frac{1}{\frac{1}{x}+\frac{1}{y}} = \frac{1}{\frac{1}{y}+\frac{1}{x}} = y\oplus x.
			\] Donc la loi $\oplus$ est commutative.
		\item {\sc Élément neutre} : soit $e$ l'élément neutre de $\oplus$. \[
				\forall x \in E,\; x \oplus e = e \oplus x = x.
			\] Comme la loi est commutative, seul l'égalité $x \oplus e = x$ est utile.

			Soit $x \in E$. On a donc $\frac{1}{\frac{1}{x}+\frac{1}{e}}=x$ donc $\frac{ex}{e+x}=x$ donc $ex = x(e+x)$ et donc $\cancel{ex} = \cancel{ex} + x^2$. On en déduit que $x^2 = 0$, ce qui n'est pas possible car $x \in \R^+_*$. Donc, il n'y a pas d'élément neutre pour $\oplus$.
	\end{itemize}
\end{exo}


	\chap[17]{Dimension finie}
	\renewcommand{\cwd}{../chap17}
	\part{Topologie de $\R^2$}

\begin{defn}
	La \underline{norme (euclidienne)} de $\R^2$ est l'application définie par \[
		\forall (x,y) \in \R^2, \|(x,y)\| = \sqrt{x^2 + y^2}.
	\]

	\begin{figure}[H]
		\centering
		\begin{asy}
			import graph;
			axes(EndArrow);
			size(4cm);
			pair A = (3,2);
			dot(A);
			draw((3,0)--A, dashed);
			draw((0,2)--A, dashed);
			label("$x$", (3,0), align=S);
			label("$y$", (0,2), align=W);
			draw((0,0)--A);
			dot((4,3), white+0);
		\end{asy}
	\end{figure}
	\index{norme (de $\R^2$)}
	\index{norme euclidienne (de $\R^2$)}
\end{defn}

\begin{prop}
	La norme euclidienne vérifie:
	\begin{enumerate}
		\item (séparation) \[
			\forall (x,y) \in \R^2, \|(x,y)\| = 0 \iff x = y = 0,
			\]
		\item (homogénéité positive) \[
				\forall \lambda \in \R, \forall (x,y) \in \R^2, \|\lambda(x,y)\|= \left| \lambda \right| \|(x,y)\|
			\]
		\item (inégalité triangulaire) \[
			\forall (x,y), (a,b) \in \R^2,
			\|(x,y)+(a,b)\|\le \|(x,y)\|+\|(a,b)\|.
		\]
	\end{enumerate}
\end{prop}

\begin{prv}
	Déjà vue en replaçant $(x,y)$ par $x+iy \in \C$ et $\|(x,y)\|$ par |x+iy|
\end{prv}

\begin{defn}
	Soit $(a,b) \in \R^2$ et $r \in \R_+$.

	La \underline{boule ouverte} (ou \underline{disque ouvert}) de centre $(a,b)$ et de rayon $r$ est \[
		B_{(a,b)}(r) = \big\{ (x,y) \in \R^2  \mid \|(x,y) - (a,b)\| < r \big\}.
	\]

	La \underline{boule fermée} (ou \underline{disque fermé}) de centre $(a,b)$ et de rayon $r$ est \[
		\overline{B_{(a,b)}}(r) = \big\{ (x,y)\in \R^2  \mid \|(x,y) - (a,b)\| \le r \big\}.
	\]

	La \underline{sphère} (ou \underline{boule}) de centre $(a,b)$ et de rayon $r$ est \[
		S_{(a,b)}(r) = \partial \overline{B_{(a,b)}}(r) = \big\{ (x,y) \in \R^2  \mid \|(x,y) - (a,b)\| = r \big\}.
	\]
	\index{boule ouverte (de $\R^2$)}
	\index{disque ouverte (de $\R^2$)}
	\index{boule fermée (de $\R^2$)}
	\index{disque fermée (de $\R^2$)}
	\index{boule (de $\R^2$)}
	\index{sphère (de $\R^2$)}
\end{defn}

\begin{figure}[H]
		\centering
		\incfig{boule}
\end{figure}

\begin{rmk}
	On parle de boule en dimension quelconque.
\end{rmk}

\begin{defn}
	Une \underline{partie ouverte} $O$ de $\R^2$ (ou \underline{un ouvert}) si \[
		\forall (x,y) \in O, \exists r > 0, B_{(a,b)}(r) \subset O.
	\]
	Une partie $F$ est \underline{fermée} su $\R^2\setminus F$ est ouverte.
	\index{partie ouverte (de $\R^2$)}
	\index{ouvert (de $\R^2$)}
	\index{partie fermée (de $\R^2$)}
\end{defn}

\begin{figure}[H]
	\centering
	\incfig{partie-ouverte}
\end{figure}

\begin{prop}
	Une boule ouverte est ouverte. Une boule fermée est fermée.
\end{prop}

\begin{figure}[H]
	\centering
	\begin{subfigure}{4cm}
		\centering
		\begin{asy}
			import patterns;

			pair n(pair a) {return a / length(a);}

			add("hatch",hatch(2mm, SW, red));
			size(4cm);

			draw(circle((0,0), 1));
			dot('$(a_0, b_0)$', (0,0), align=S);

			draw((0,0) -- n((-1, 1)), dashed);
			label("$r$", n((-1, 1)) / 2, align=1.5S);

			pair A = n((1,3)) * (2/3);
			real rho = (1 - length(A)) * (2 / 3);

			dot("$(a,b)$", A, red, align=3SE);
			filldraw(circle(A, rho), pattern("hatch"), red);

			label("$O$", n((1,-1))*2.5/3);
		\end{asy}
	\end{subfigure}
	\begin{subfigure}{1cm}
		\centering~\\
	\end{subfigure}
	\begin{subfigure}{5cm}
		\centering
		\begin{asy}
			import patterns;

			pair n(pair a) {return a / length(a);}

			add("hatch",hatch(1mm, SW, red));
			add("hatch2",hatch(3mm, SE, blue));
			size(5cm);

			guide around = (-1.5, -1.5) -- (-1.5, 1.5) -- (2.5, 1.5) -- (2.5, -1.5) -- cycle;

			pair A = n((3, 1)) * 5/3; 
			real rho = 2 / 9;

			picture inter;
			fill(inter, around, pattern("hatch2"));
			fill(inter, circle((0,0), 1), white);
			add(inter);

			draw(circle((0,0), 1));
			dot('$(a_0, b_0)$', (0,0), align=S);

			draw((0,0) -- n((-1, 1)), dashed);
			label("$r$", n((-1, 1)) / 2, align=1.5S);

			dot("$(a,b)$", A, red, align=2SE);
			filldraw(circle(A, rho), pattern("hatch"), red);

			label("$F$", n((1,-1))*2.5/3);
		\end{asy}
	\end{subfigure}
\end{figure}

\begin{prv}
	$\O$ est un ouvert.

	Soit $B$ la boule ouverte de centre $(a_0, b_0) \in \R^2$ et de rayon $r > 0$.

	On pose $\rho = \frac{1}{2}\big(r - \|(a,b) - (a_0,b_0)\|\big)$.
	Montrons que \[
		B_{(a,b)}(\rho) \subset  B_{(a,b)}(r).
	\]

	Soit $(x,y) \in B_{(a,b)}(\rho)$.
	\begin{align*}
		\|(x,y) - (a_0,b_0)\|&= \|(x,y)- (a,b) + (a,b) - (a_0,b_0)\| \\
		&\le \|(x,y) - (a,b)\| + \|(a,b) - (a_0, b_0)\|\\
		&< \rho + \|(a,b) - (a_0, b_0)\| = \frac{1}{2}r + \frac{1}{2} \|(a,b) - (a_0, b_0)\|\\
		&< r
	\end{align*}
	
	Soit $F$ la boule fermée de centre $(a_0, b_0)$ et de rayon $r \ge 0$.

	Soit $(a,b) \not\in F$. On pose \[
		\rho = \frac{1}{2}\big(\|(a,b) - (a_0, b_0)\| - r\big) > 0.
	\]

	Montrons que $B_{(a,b)}(\rho) \subset \R^2\setminus F$.

	Soit $(x,y) \in B_{(a,b)}(\rho)$.

	\begin{align*}
		\|(x,y) - (a_0, b_0)\| &= \|(x,y) - (a,b) + (a,b) - (a_0, b_0)\| \\
		&\ge \big| \underbrace{\|(x,y) - (a,b)\|}_{\le \rho} - \underbrace{\|(a,b) - (a_0, b_0)\|}_{> r} \big|\\
		&\ge \|(a,b) - (a_0, b_0)\|- \|(x,y) - (a,b)\|\\
		&> \|(a,b) - (a_0, b_0)\|- \rho\\
		&> \frac{1}{2} \|(a,b) - (a_0, b_0)\| + \frac{1}{2}r\\
		&> r
	\end{align*}

	donc $(x,y) \not\in F$.
\end{prv}

\begin{exm}
	\begin{enumerate}
		\item $\O$ est ouvert.\\
			$\R^2$ est ouvert.
		\item $\O$ est fermé.\\
			$\R^2$ est fermé.\\
		\item $\big\{(x,0)  \mid x > 0\big\}$ n'est ni ouverte ni fermé.
	\end{enumerate}
\end{exm}

\begin{figure}[H]
	\centering
	\begin{asy}
		size(3cm);

		draw((0, -1) -- (0, 3), Arrow(TeXHead));
		draw((-1, 0) -- (3, 0), Arrow(TeXHead));
		
		draw((0,0) -- (0, 2.97), red);
		draw(circle((0,1.5), 0.5), deepred);
		draw(circle((0,0.5), 0.1), deepred);
	\end{asy}
\end{figure}

\begin{defn}
	Soit $(a,b) \in \R^2$ et $V \in \mathcal{P}(\R^2)$.

	On dit que $V$ est un voisinage de $(a,b)$ s'il existe $r > 0$ tel que \[
		B_{(a,b)}(r) \subset V.
	\]
	\index{voisinage (dans $\R^2$)}
\end{defn}

\begin{prop}
	Un ouvert non vide est un voisinage en chacun de ces points. \qed
\end{prop}

\begin{defn}
	Soit $D \subset \R^2$. Un \underline{point intérieur} de $D$ est un couple $(a,b) \in D$ tel que \[
		\exists r > 0, B_{(a,b)}(r) \subset D.
	\] en d'autres termes, si $D$ est un voisinage de $(a,b)$.

	On note $\mathring D$ l'ensemble des points intérieurs à $D$. C'est \underline{l'intérieur} de $D$.
	\index{point intérieur (dans $\R^2$)}
	\index{intérieur (dans $\R^2$)}
\end{defn}

\begin{prop}
	$\mathring D$ est le plus grand ouvert $O$ de $\R^2$ tel que $O \subset D$.
\end{prop}

\begin{figure}[H]
	\centering
	\incfig{interieur-plus-grand-ouvert}
\end{figure}


\begin{prv}
	Soit $(a,b) \in \mathring D$.

	Par définition, il existe $r > 0$ tel que \[
		B_{(a,b)}(r) \subset D.
	\] Montrons que $B_{(a,b)}(r) \subset \mathring D$.

	Soit $(x,y) \in B_{(a,b)}(r)$. Comme $B_{(a,b)}(r)$ est un ouvert de $\R^2$, il existe $\rho > 0$ tel que \[
		B_{(x,y)}(\rho) \subset B_{(a,b)}(r)
	\] donc $(x,y) \in \mathring D$.

	Donc $\mathring D$ est ouvert, $\mathring D \subset D$.

	Soit $O$ un ouvert de $\R^2$ tel que $O \subset D$. Montrons que $O \subset \mathring D$.

	Soit $(x,y) \in O$. Soit $r > 0$ tel que \[
		B_{(x,y)}(r) \subset O \subset D
	\] donc $(x,y) \in \mathring D$.
\end{prv}

\begin{defn}
	Soit $f: D \subset \R^2 \to \R$, $\ell \in \R$, $(a,b) \in \mathring D$.

	On dit que \underline{$f(x,y)$ tend vers $\ell$ quand $(x,y)$ tend vers $(a,b)$} ou que $\ell$ est \underline{une limite} de $f$ en $(a,b)$ si \[
		\forall \varepsilon > 0, \exists r > 0, \forall (x,y) \in D, \|(x,y) - (a,b)\| < r \implies \left| f(x,y) - \ell \right| \le \varepsilon.
	\] en d'autres termes si \[
		\forall V \in \mathcal{V}_{\ell}, \exists W \in \mathcal{V}_{(a,b)}, \forall (x,y) \in W \cap D, f(x,y) \in V.
	\]
	\index{limite (dans $\R^2$)}
	\index{tendre vers (dans $\R^2$)}
\end{defn}

\begin{prop}
	[unicité de la limite]
	Soit $f: D \to \R$, $(a,b) \in \mathring D$, $\ell_1, \ell_2 \in \R$ telles que $\ell_1$ et $\ell_2$ sont des limites de $f$ en $(a,b)$.

	Alors $\ell_1 = \ell_2$.
\end{prop}

\begin{figure}[H]
	\centering
	\incfig{preuve-unicité-de-la-limite}
\end{figure}

\begin{prv}
	On suppose $\ell_1 < \ell_2$. On pose $\varepsilon = \frac{\ell_2 - \ell_1}{2} > 0$.

	Soit $r_1 > 0$ tel que \[
		f\big(B_{(a,b)}(r_1)\big) \subset ]\ell_1 - \varepsilon, \ell_1 + \varepsilon[.
	\] Soit $r_2 > 0$ tel que \[
		f\big(B_{(a,b)}(r_2)\big) \subset ]\ell_2 - \varepsilon, \ell_2 + \varepsilon [.
	\] On pose $r = \min(r_1, r_2)$ donc \[
		B_{(a,b)}(r_1) \cap B_{(a,b)}(r_2) = B_{(a,b)}(r) \neq \O.
	\] Soit $(x,y) \in B_{(a,b)}(r)$. Alors, \[
		f(x,y) \in ]\ell_1 - \varepsilon, \ell_1 + \varepsilon[ \cap ]\ell_2 - \varepsilon, \ell_2 + \varepsilon[ = \O.
	\] $\lightning$
\end{prv}

\begin{defn}
	Soit $f : D \to \R$, $(a,b) \in \mathring D$.

	On dit que $f$ est \underline{continue} en $(a,b)$ si \[
		f(x,y) \tendsto{(x,y) \to (a,b)}f(a,b).
	\]
	\index{continuité (dans $\R^2$)}
\end{defn}

\begin{prop}
	\underline{Si} $f(x,y) \tendsto{(x,y) \to (a,b)} \ell$ \\
	\underline{alors} $\begin{cases}
		f(x,b) \tendsto{x \to a} \ell\\
		f(a,y) \tendsto{y \to b} \ell.\\
	\end{cases}$
\end{prop}

\begin{prv}~\\
	\begin{figure}[H]
		\centering
		\incfig{limite-x-en-a-et-y-en-b}
	\end{figure}
\end{prv}

\underline{Contre-exemple} : exercice 3.

\begin{exm}
	\begin{enumerate}
		\item $f : \begin{array}{rcl}
				\R^2 &\longrightarrow& \R \\
				(x,y) &\longmapsto& x
			\end{array}$ limite en $(0,0)$ ?

			Soit $\varepsilon > 0$. On pose $r = \varepsilon$. \[
				\forall (x,y) \in B_{(0,0)}(r),
				\left| f(x,y) \right| = \left| x \right| \le \|(x,y)\| < r = \varepsilon
			\] Donc $f(x,y) \tendsto{(x,y) \to (a,b)} 0$.
		\item limite $f : \begin{array}{rcl}
				\R^2 &\longrightarrow& \R \\
				(x,y) &\longmapsto& x^3
			\end{array}$ en $(0,0)$ ?

			Soit $\varepsilon > 0$. On pose $r = \sqrt[3]{r} > 0$. \[
				\forall (x,y) \in B_{(0,0)}(r),
				\left| f(x,y) \right| = \left| x^3 \right| \le \|(x,y)\|^3 < r^3 = \varepsilon.
			\]
		\item limite de $f : \begin{array}{rcl}
			\R^2 &\longrightarrow& \R \\
			(x,y) &\longmapsto& x^3y^2
		\end{array}$ en $(0,0)$ ?

		Soit $\varepsilon > 0$. On pose $r = \sqrt[5]{\varepsilon} > 0$. \[
			\forall (x,y) \in B_{(0,0)}(r), \left| f(x,y) \right| = \left| x^3 y^2 \right| \le \|(x,y)\|^3 \|(x,y)\|^2 < r^5 = \varepsilon.
		\]
	\end{enumerate}
\end{exm}

\begin{defn}
	Soient $D \subset \R^2$ et $(x,y) \in \R^2$.

	\begin{figure}[H]
    \centering
    \incfig{point-adhérent}
	\end{figure}
	
	On dit que $(x,y)$ est \underline{adhérent} à $D$ si \[
		\forall r > 0, B_{(x,y)}(r) \cap D \neq \O.
	\] L'ensemble des points adhérents à $D$ est noté $\overline{D}$. On dit que $\overline{D}$ est \underline{l'adhérence} de $D$.
	\index{point adhérent (dans $\R^2$)}
	\index{adhérent (dans $\R^2$)}
\end{defn}

\begin{defn}
	Soit $f: D \subset \R^2 \to \R$ et $(a,b) \in \overline{D}$, $\ell \in \R$. On dit que $f$ tend vers $\ell$ quand $(x,y)$ tend vers $(a,b)$ si \[
		\forall \varepsilon > 0, \exists r > 0, \forall (x,y) \in B_{(a,b)}(r) \cap D,
		\left| f(x,y) - \ell \right| \le \varepsilon.
	\]
	\index{limite (dans $\R^2$)}
	\index{tendre vers (dans $\R^2$)}
\end{defn}

\begin{prop}
	\begin{enumerate}
		\item Dans ce contexte, il y a unicité de la limite
		\item La limite d'une somme, d'un produit, d'un quotien, d'une composée se comporte comme dans le cas d'une seule variable.
		\item Soit $f: D \to \R$ continue. Soient $g: I \to \R$ et $h: I \to \R$ continues telles que \[
			\forall t \in I, \big(g(t), h(t)\big) \in D.
		\] Alors \[
			t \in I \mapsto f\big(g(t), h(t)\big) \in \R
		\] est continue.
	\end{enumerate}
\end{prop}

\begin{figure}[H]
	\centering
	\begin{asy}
		import three;
		import graph3;
		size(5cm);

		settings.render = 0;
		settings.prc = false;
		currentprojection = obliqueX;

		draw(O -- X, Arrow3(TeXHead2));
		draw(O -- Y, Arrow3(TeXHead2));
		draw(O -- Z, Arrow3(TeXHead2));

		triple f(real x, real y, real z = 0) { return (x,y,cos(x - 0.5) * cos(y - 0.5)/1.2 + 0.15); }

		real inc = 1 / 5;

		for(real x = 0; x <= 1; x += inc) {
			draw(graph(
				new real(real t) { return x; }, // x
				new real(real y) { return y; }, // y
				new real(real y) { return f(x,y).z; }, // z
				0, 1
			), gray);
		}

		for(real y = 0; y <= 1; y += inc) {
			draw(graph(
				new real(real x) { return x; }, // x
				new real(real t) { return y; }, // y
				new real(real x) { return f(x,y).z; }, // z
				0, 1
			), gray);
		}

		path3 path1 = (0.3, 0.2, 0) .. (0.5, 0.5, 0) .. (0.6, 0.7, 0) .. (0.9, 0.8, 0);
		path3 path2 = (0.3, 0.8, 0) .. (0.5, 0.5, 0) .. (0.6, 0.3, 0) .. (0.9, 0.2, 0);
		path3 pathA = f(0.3, 0.2, 0) .. f(0.5, 0.5, 0) .. f(0.6, 0.7, 0) .. f(0.9, 0.8, 0);
		path3 pathB = f(0.3, 0.8, 0) .. f(0.5, 0.5, 0) .. f(0.6, 0.3, 0) .. f(0.9, 0.2, 0);

		draw(path1, red, Arrow3(TeXHead2, position=0.5));
		draw(pathA, red, Arrow3(TeXHead2, position=0.5));
		draw(path2, deepcyan, Arrow3(TeXHead2, position=0.3));
		draw(pathB, deepcyan, Arrow3(TeXHead2, position=0.3));

		dot((0.5, 0.5, 0));
		dot(f(0.5, 0.5, 0));
		draw((0.5, 0.5, 0) -- f(0.5, 0.5, 0), dashed);
	\end{asy}
\end{figure}



	\chap[18]{Polynômes formels}
	\renewcommand{\cwd}{../chap18}
	\part{Topologie de $\R^2$}

\begin{defn}
	La \underline{norme (euclidienne)} de $\R^2$ est l'application définie par \[
		\forall (x,y) \in \R^2, \|(x,y)\| = \sqrt{x^2 + y^2}.
	\]

	\begin{figure}[H]
		\centering
		\begin{asy}
			import graph;
			axes(EndArrow);
			size(4cm);
			pair A = (3,2);
			dot(A);
			draw((3,0)--A, dashed);
			draw((0,2)--A, dashed);
			label("$x$", (3,0), align=S);
			label("$y$", (0,2), align=W);
			draw((0,0)--A);
			dot((4,3), white+0);
		\end{asy}
	\end{figure}
	\index{norme (de $\R^2$)}
	\index{norme euclidienne (de $\R^2$)}
\end{defn}

\begin{prop}
	La norme euclidienne vérifie:
	\begin{enumerate}
		\item (séparation) \[
			\forall (x,y) \in \R^2, \|(x,y)\| = 0 \iff x = y = 0,
			\]
		\item (homogénéité positive) \[
				\forall \lambda \in \R, \forall (x,y) \in \R^2, \|\lambda(x,y)\|= \left| \lambda \right| \|(x,y)\|
			\]
		\item (inégalité triangulaire) \[
			\forall (x,y), (a,b) \in \R^2,
			\|(x,y)+(a,b)\|\le \|(x,y)\|+\|(a,b)\|.
		\]
	\end{enumerate}
\end{prop}

\begin{prv}
	Déjà vue en replaçant $(x,y)$ par $x+iy \in \C$ et $\|(x,y)\|$ par |x+iy|
\end{prv}

\begin{defn}
	Soit $(a,b) \in \R^2$ et $r \in \R_+$.

	La \underline{boule ouverte} (ou \underline{disque ouvert}) de centre $(a,b)$ et de rayon $r$ est \[
		B_{(a,b)}(r) = \big\{ (x,y) \in \R^2  \mid \|(x,y) - (a,b)\| < r \big\}.
	\]

	La \underline{boule fermée} (ou \underline{disque fermé}) de centre $(a,b)$ et de rayon $r$ est \[
		\overline{B_{(a,b)}}(r) = \big\{ (x,y)\in \R^2  \mid \|(x,y) - (a,b)\| \le r \big\}.
	\]

	La \underline{sphère} (ou \underline{boule}) de centre $(a,b)$ et de rayon $r$ est \[
		S_{(a,b)}(r) = \partial \overline{B_{(a,b)}}(r) = \big\{ (x,y) \in \R^2  \mid \|(x,y) - (a,b)\| = r \big\}.
	\]
	\index{boule ouverte (de $\R^2$)}
	\index{disque ouverte (de $\R^2$)}
	\index{boule fermée (de $\R^2$)}
	\index{disque fermée (de $\R^2$)}
	\index{boule (de $\R^2$)}
	\index{sphère (de $\R^2$)}
\end{defn}

\begin{figure}[H]
		\centering
		\incfig{boule}
\end{figure}

\begin{rmk}
	On parle de boule en dimension quelconque.
\end{rmk}

\begin{defn}
	Une \underline{partie ouverte} $O$ de $\R^2$ (ou \underline{un ouvert}) si \[
		\forall (x,y) \in O, \exists r > 0, B_{(a,b)}(r) \subset O.
	\]
	Une partie $F$ est \underline{fermée} su $\R^2\setminus F$ est ouverte.
	\index{partie ouverte (de $\R^2$)}
	\index{ouvert (de $\R^2$)}
	\index{partie fermée (de $\R^2$)}
\end{defn}

\begin{figure}[H]
	\centering
	\incfig{partie-ouverte}
\end{figure}

\begin{prop}
	Une boule ouverte est ouverte. Une boule fermée est fermée.
\end{prop}

\begin{figure}[H]
	\centering
	\begin{subfigure}{4cm}
		\centering
		\begin{asy}
			import patterns;

			pair n(pair a) {return a / length(a);}

			add("hatch",hatch(2mm, SW, red));
			size(4cm);

			draw(circle((0,0), 1));
			dot('$(a_0, b_0)$', (0,0), align=S);

			draw((0,0) -- n((-1, 1)), dashed);
			label("$r$", n((-1, 1)) / 2, align=1.5S);

			pair A = n((1,3)) * (2/3);
			real rho = (1 - length(A)) * (2 / 3);

			dot("$(a,b)$", A, red, align=3SE);
			filldraw(circle(A, rho), pattern("hatch"), red);

			label("$O$", n((1,-1))*2.5/3);
		\end{asy}
	\end{subfigure}
	\begin{subfigure}{1cm}
		\centering~\\
	\end{subfigure}
	\begin{subfigure}{5cm}
		\centering
		\begin{asy}
			import patterns;

			pair n(pair a) {return a / length(a);}

			add("hatch",hatch(1mm, SW, red));
			add("hatch2",hatch(3mm, SE, blue));
			size(5cm);

			guide around = (-1.5, -1.5) -- (-1.5, 1.5) -- (2.5, 1.5) -- (2.5, -1.5) -- cycle;

			pair A = n((3, 1)) * 5/3; 
			real rho = 2 / 9;

			picture inter;
			fill(inter, around, pattern("hatch2"));
			fill(inter, circle((0,0), 1), white);
			add(inter);

			draw(circle((0,0), 1));
			dot('$(a_0, b_0)$', (0,0), align=S);

			draw((0,0) -- n((-1, 1)), dashed);
			label("$r$", n((-1, 1)) / 2, align=1.5S);

			dot("$(a,b)$", A, red, align=2SE);
			filldraw(circle(A, rho), pattern("hatch"), red);

			label("$F$", n((1,-1))*2.5/3);
		\end{asy}
	\end{subfigure}
\end{figure}

\begin{prv}
	$\O$ est un ouvert.

	Soit $B$ la boule ouverte de centre $(a_0, b_0) \in \R^2$ et de rayon $r > 0$.

	On pose $\rho = \frac{1}{2}\big(r - \|(a,b) - (a_0,b_0)\|\big)$.
	Montrons que \[
		B_{(a,b)}(\rho) \subset  B_{(a,b)}(r).
	\]

	Soit $(x,y) \in B_{(a,b)}(\rho)$.
	\begin{align*}
		\|(x,y) - (a_0,b_0)\|&= \|(x,y)- (a,b) + (a,b) - (a_0,b_0)\| \\
		&\le \|(x,y) - (a,b)\| + \|(a,b) - (a_0, b_0)\|\\
		&< \rho + \|(a,b) - (a_0, b_0)\| = \frac{1}{2}r + \frac{1}{2} \|(a,b) - (a_0, b_0)\|\\
		&< r
	\end{align*}
	
	Soit $F$ la boule fermée de centre $(a_0, b_0)$ et de rayon $r \ge 0$.

	Soit $(a,b) \not\in F$. On pose \[
		\rho = \frac{1}{2}\big(\|(a,b) - (a_0, b_0)\| - r\big) > 0.
	\]

	Montrons que $B_{(a,b)}(\rho) \subset \R^2\setminus F$.

	Soit $(x,y) \in B_{(a,b)}(\rho)$.

	\begin{align*}
		\|(x,y) - (a_0, b_0)\| &= \|(x,y) - (a,b) + (a,b) - (a_0, b_0)\| \\
		&\ge \big| \underbrace{\|(x,y) - (a,b)\|}_{\le \rho} - \underbrace{\|(a,b) - (a_0, b_0)\|}_{> r} \big|\\
		&\ge \|(a,b) - (a_0, b_0)\|- \|(x,y) - (a,b)\|\\
		&> \|(a,b) - (a_0, b_0)\|- \rho\\
		&> \frac{1}{2} \|(a,b) - (a_0, b_0)\| + \frac{1}{2}r\\
		&> r
	\end{align*}

	donc $(x,y) \not\in F$.
\end{prv}

\begin{exm}
	\begin{enumerate}
		\item $\O$ est ouvert.\\
			$\R^2$ est ouvert.
		\item $\O$ est fermé.\\
			$\R^2$ est fermé.\\
		\item $\big\{(x,0)  \mid x > 0\big\}$ n'est ni ouverte ni fermé.
	\end{enumerate}
\end{exm}

\begin{figure}[H]
	\centering
	\begin{asy}
		size(3cm);

		draw((0, -1) -- (0, 3), Arrow(TeXHead));
		draw((-1, 0) -- (3, 0), Arrow(TeXHead));
		
		draw((0,0) -- (0, 2.97), red);
		draw(circle((0,1.5), 0.5), deepred);
		draw(circle((0,0.5), 0.1), deepred);
	\end{asy}
\end{figure}

\begin{defn}
	Soit $(a,b) \in \R^2$ et $V \in \mathcal{P}(\R^2)$.

	On dit que $V$ est un voisinage de $(a,b)$ s'il existe $r > 0$ tel que \[
		B_{(a,b)}(r) \subset V.
	\]
	\index{voisinage (dans $\R^2$)}
\end{defn}

\begin{prop}
	Un ouvert non vide est un voisinage en chacun de ces points. \qed
\end{prop}

\begin{defn}
	Soit $D \subset \R^2$. Un \underline{point intérieur} de $D$ est un couple $(a,b) \in D$ tel que \[
		\exists r > 0, B_{(a,b)}(r) \subset D.
	\] en d'autres termes, si $D$ est un voisinage de $(a,b)$.

	On note $\mathring D$ l'ensemble des points intérieurs à $D$. C'est \underline{l'intérieur} de $D$.
	\index{point intérieur (dans $\R^2$)}
	\index{intérieur (dans $\R^2$)}
\end{defn}

\begin{prop}
	$\mathring D$ est le plus grand ouvert $O$ de $\R^2$ tel que $O \subset D$.
\end{prop}

\begin{figure}[H]
	\centering
	\incfig{interieur-plus-grand-ouvert}
\end{figure}


\begin{prv}
	Soit $(a,b) \in \mathring D$.

	Par définition, il existe $r > 0$ tel que \[
		B_{(a,b)}(r) \subset D.
	\] Montrons que $B_{(a,b)}(r) \subset \mathring D$.

	Soit $(x,y) \in B_{(a,b)}(r)$. Comme $B_{(a,b)}(r)$ est un ouvert de $\R^2$, il existe $\rho > 0$ tel que \[
		B_{(x,y)}(\rho) \subset B_{(a,b)}(r)
	\] donc $(x,y) \in \mathring D$.

	Donc $\mathring D$ est ouvert, $\mathring D \subset D$.

	Soit $O$ un ouvert de $\R^2$ tel que $O \subset D$. Montrons que $O \subset \mathring D$.

	Soit $(x,y) \in O$. Soit $r > 0$ tel que \[
		B_{(x,y)}(r) \subset O \subset D
	\] donc $(x,y) \in \mathring D$.
\end{prv}

\begin{defn}
	Soit $f: D \subset \R^2 \to \R$, $\ell \in \R$, $(a,b) \in \mathring D$.

	On dit que \underline{$f(x,y)$ tend vers $\ell$ quand $(x,y)$ tend vers $(a,b)$} ou que $\ell$ est \underline{une limite} de $f$ en $(a,b)$ si \[
		\forall \varepsilon > 0, \exists r > 0, \forall (x,y) \in D, \|(x,y) - (a,b)\| < r \implies \left| f(x,y) - \ell \right| \le \varepsilon.
	\] en d'autres termes si \[
		\forall V \in \mathcal{V}_{\ell}, \exists W \in \mathcal{V}_{(a,b)}, \forall (x,y) \in W \cap D, f(x,y) \in V.
	\]
	\index{limite (dans $\R^2$)}
	\index{tendre vers (dans $\R^2$)}
\end{defn}

\begin{prop}
	[unicité de la limite]
	Soit $f: D \to \R$, $(a,b) \in \mathring D$, $\ell_1, \ell_2 \in \R$ telles que $\ell_1$ et $\ell_2$ sont des limites de $f$ en $(a,b)$.

	Alors $\ell_1 = \ell_2$.
\end{prop}

\begin{figure}[H]
	\centering
	\incfig{preuve-unicité-de-la-limite}
\end{figure}

\begin{prv}
	On suppose $\ell_1 < \ell_2$. On pose $\varepsilon = \frac{\ell_2 - \ell_1}{2} > 0$.

	Soit $r_1 > 0$ tel que \[
		f\big(B_{(a,b)}(r_1)\big) \subset ]\ell_1 - \varepsilon, \ell_1 + \varepsilon[.
	\] Soit $r_2 > 0$ tel que \[
		f\big(B_{(a,b)}(r_2)\big) \subset ]\ell_2 - \varepsilon, \ell_2 + \varepsilon [.
	\] On pose $r = \min(r_1, r_2)$ donc \[
		B_{(a,b)}(r_1) \cap B_{(a,b)}(r_2) = B_{(a,b)}(r) \neq \O.
	\] Soit $(x,y) \in B_{(a,b)}(r)$. Alors, \[
		f(x,y) \in ]\ell_1 - \varepsilon, \ell_1 + \varepsilon[ \cap ]\ell_2 - \varepsilon, \ell_2 + \varepsilon[ = \O.
	\] $\lightning$
\end{prv}

\begin{defn}
	Soit $f : D \to \R$, $(a,b) \in \mathring D$.

	On dit que $f$ est \underline{continue} en $(a,b)$ si \[
		f(x,y) \tendsto{(x,y) \to (a,b)}f(a,b).
	\]
	\index{continuité (dans $\R^2$)}
\end{defn}

\begin{prop}
	\underline{Si} $f(x,y) \tendsto{(x,y) \to (a,b)} \ell$ \\
	\underline{alors} $\begin{cases}
		f(x,b) \tendsto{x \to a} \ell\\
		f(a,y) \tendsto{y \to b} \ell.\\
	\end{cases}$
\end{prop}

\begin{prv}~\\
	\begin{figure}[H]
		\centering
		\incfig{limite-x-en-a-et-y-en-b}
	\end{figure}
\end{prv}

\underline{Contre-exemple} : exercice 3.

\begin{exm}
	\begin{enumerate}
		\item $f : \begin{array}{rcl}
				\R^2 &\longrightarrow& \R \\
				(x,y) &\longmapsto& x
			\end{array}$ limite en $(0,0)$ ?

			Soit $\varepsilon > 0$. On pose $r = \varepsilon$. \[
				\forall (x,y) \in B_{(0,0)}(r),
				\left| f(x,y) \right| = \left| x \right| \le \|(x,y)\| < r = \varepsilon
			\] Donc $f(x,y) \tendsto{(x,y) \to (a,b)} 0$.
		\item limite $f : \begin{array}{rcl}
				\R^2 &\longrightarrow& \R \\
				(x,y) &\longmapsto& x^3
			\end{array}$ en $(0,0)$ ?

			Soit $\varepsilon > 0$. On pose $r = \sqrt[3]{r} > 0$. \[
				\forall (x,y) \in B_{(0,0)}(r),
				\left| f(x,y) \right| = \left| x^3 \right| \le \|(x,y)\|^3 < r^3 = \varepsilon.
			\]
		\item limite de $f : \begin{array}{rcl}
			\R^2 &\longrightarrow& \R \\
			(x,y) &\longmapsto& x^3y^2
		\end{array}$ en $(0,0)$ ?

		Soit $\varepsilon > 0$. On pose $r = \sqrt[5]{\varepsilon} > 0$. \[
			\forall (x,y) \in B_{(0,0)}(r), \left| f(x,y) \right| = \left| x^3 y^2 \right| \le \|(x,y)\|^3 \|(x,y)\|^2 < r^5 = \varepsilon.
		\]
	\end{enumerate}
\end{exm}

\begin{defn}
	Soient $D \subset \R^2$ et $(x,y) \in \R^2$.

	\begin{figure}[H]
    \centering
    \incfig{point-adhérent}
	\end{figure}
	
	On dit que $(x,y)$ est \underline{adhérent} à $D$ si \[
		\forall r > 0, B_{(x,y)}(r) \cap D \neq \O.
	\] L'ensemble des points adhérents à $D$ est noté $\overline{D}$. On dit que $\overline{D}$ est \underline{l'adhérence} de $D$.
	\index{point adhérent (dans $\R^2$)}
	\index{adhérent (dans $\R^2$)}
\end{defn}

\begin{defn}
	Soit $f: D \subset \R^2 \to \R$ et $(a,b) \in \overline{D}$, $\ell \in \R$. On dit que $f$ tend vers $\ell$ quand $(x,y)$ tend vers $(a,b)$ si \[
		\forall \varepsilon > 0, \exists r > 0, \forall (x,y) \in B_{(a,b)}(r) \cap D,
		\left| f(x,y) - \ell \right| \le \varepsilon.
	\]
	\index{limite (dans $\R^2$)}
	\index{tendre vers (dans $\R^2$)}
\end{defn}

\begin{prop}
	\begin{enumerate}
		\item Dans ce contexte, il y a unicité de la limite
		\item La limite d'une somme, d'un produit, d'un quotien, d'une composée se comporte comme dans le cas d'une seule variable.
		\item Soit $f: D \to \R$ continue. Soient $g: I \to \R$ et $h: I \to \R$ continues telles que \[
			\forall t \in I, \big(g(t), h(t)\big) \in D.
		\] Alors \[
			t \in I \mapsto f\big(g(t), h(t)\big) \in \R
		\] est continue.
	\end{enumerate}
\end{prop}

\begin{figure}[H]
	\centering
	\begin{asy}
		import three;
		import graph3;
		size(5cm);

		settings.render = 0;
		settings.prc = false;
		currentprojection = obliqueX;

		draw(O -- X, Arrow3(TeXHead2));
		draw(O -- Y, Arrow3(TeXHead2));
		draw(O -- Z, Arrow3(TeXHead2));

		triple f(real x, real y, real z = 0) { return (x,y,cos(x - 0.5) * cos(y - 0.5)/1.2 + 0.15); }

		real inc = 1 / 5;

		for(real x = 0; x <= 1; x += inc) {
			draw(graph(
				new real(real t) { return x; }, // x
				new real(real y) { return y; }, // y
				new real(real y) { return f(x,y).z; }, // z
				0, 1
			), gray);
		}

		for(real y = 0; y <= 1; y += inc) {
			draw(graph(
				new real(real x) { return x; }, // x
				new real(real t) { return y; }, // y
				new real(real x) { return f(x,y).z; }, // z
				0, 1
			), gray);
		}

		path3 path1 = (0.3, 0.2, 0) .. (0.5, 0.5, 0) .. (0.6, 0.7, 0) .. (0.9, 0.8, 0);
		path3 path2 = (0.3, 0.8, 0) .. (0.5, 0.5, 0) .. (0.6, 0.3, 0) .. (0.9, 0.2, 0);
		path3 pathA = f(0.3, 0.2, 0) .. f(0.5, 0.5, 0) .. f(0.6, 0.7, 0) .. f(0.9, 0.8, 0);
		path3 pathB = f(0.3, 0.8, 0) .. f(0.5, 0.5, 0) .. f(0.6, 0.3, 0) .. f(0.9, 0.2, 0);

		draw(path1, red, Arrow3(TeXHead2, position=0.5));
		draw(pathA, red, Arrow3(TeXHead2, position=0.5));
		draw(path2, deepcyan, Arrow3(TeXHead2, position=0.3));
		draw(pathB, deepcyan, Arrow3(TeXHead2, position=0.3));

		dot((0.5, 0.5, 0));
		dot(f(0.5, 0.5, 0));
		draw((0.5, 0.5, 0) -- f(0.5, 0.5, 0), dashed);
	\end{asy}
\end{figure}


	\part{Transpositions}

\begin{defn}
	Une \underline{transposition} est un cycle de longueur 2 : $\begin{pmatrix}
		a&b
	\end{pmatrix}$ avec $a \neq b$.
	\index{transposition (permutation)}
\end{defn}

\begin{exm}
	Avec $n = 5$ et $\gamma = \begin{pmatrix}
		2&4&1
	\end{pmatrix}$.

	\begin{figure}[H]
		\centering

		\begin{asy}
			size(5cm);

			real rho = 0.15; // circles

			void draw_cycle(pair O, real r ...int[] nums) {
				int n = nums.length;
				real eps = (15 / r) * 0.8;

				for(int i = 0; i < n; ++i) {
					real theta_1 = (360/n) * (i+1);
					real theta_2 = (360/n) * i;

					pair C = O + dir(theta_2) * r;

					draw(circle(C, rho));
					label("$" + string(nums[i]) + "$", C);
					draw(arc(O, r, theta_2+eps, theta_1-eps), Arrow(TeXHead));
				}
			}

			draw_cycle((-1,0), 0.8, 1, 2, 4);
			draw_cycle((1,0), 0.3, 3);
			draw_cycle((2,0), 0.3, 5);
		\end{asy}
	\end{figure}

	\[
		\gamma = \begin{pmatrix}
			1&4
		\end{pmatrix} \begin{pmatrix}
			1&2
		\end{pmatrix}
	\]

	Avec $n = 6$ et $\gamma = \begin{pmatrix}
		1&3&5&6&2
	\end{pmatrix} = \begin{pmatrix}
		1&2&3&4&5&6\\
		3&1&5&4&6&2
	\end{pmatrix}$.

	Donc, \[
		\gamma = \begin{pmatrix}
			1&2
		\end{pmatrix} \begin{pmatrix}
			1&6
		\end{pmatrix} \begin{pmatrix}
			1&5
		\end{pmatrix} \begin{pmatrix}
			1&3
		\end{pmatrix}
	\] 
	\[
		\begin{pmatrix}
			1&2&3&4&5&6\\
			3&2&1&4&5&6\\
			3&2&5&4&1&6\\
			3&2&5&4&6&1\\
			3&1&5&4&6&2\\
		\end{pmatrix}
	\]

	Et, \[
		\gamma = \begin{pmatrix}
			1&3
		\end{pmatrix} \begin{pmatrix}
			2&3
		\end{pmatrix} \begin{pmatrix}
			3&5
		\end{pmatrix} \begin{pmatrix}
			5&6
		\end{pmatrix} 
	\]

	\[
		\begin{pmatrix}
			1&2&3&4&5&6\\
			1&2&3&4&6&5\\
			1&2&5&4&6&3\\
			1&3&5&4&6&2\\
			3&1&5&4&6&2\\
		\end{pmatrix} 
	\] 
\end{exm}

\begin{exm}
	\[
		\begin{pmatrix}
			1&4
		\end{pmatrix} = \begin{pmatrix}
			1&2
		\end{pmatrix} \begin{pmatrix}
			2&3
		\end{pmatrix} \begin{pmatrix}
			3&4
		\end{pmatrix} \begin{pmatrix}
			2&3
		\end{pmatrix} \begin{pmatrix}
			1&2
		\end{pmatrix}
	\]
	On n'a pas toujours le même nombre de transpositions mais la parité du nombre reste la même (proposition plus loin).
\end{exm}

\begin{thm}
	Toute permutation se décompose en produit de transpositions.
\end{thm}

\begin{prv}
	Soit $\gamma = \begin{pmatrix}
		a_1&\cdots&a_k
	\end{pmatrix}$ un $k$-cycle.

	On remarque que
	\[
		\gamma = \begin{pmatrix}
			a_1&a_k
		\end{pmatrix} \cdots \begin{pmatrix}
			a_1&a_4
		\end{pmatrix} \begin{pmatrix}
			a_1&a_3
		\end{pmatrix} \begin{pmatrix}
			a_1&a_2
		\end{pmatrix}
	\] C'est un produit de transpositions.
\end{prv}

\begin{exm}
	Avec $n = 10$ et $\sigma = \begin{pmatrix}
		1&2&3&4&5&6&7&8&9&10\\
		9&8&1&7&2&3&4&5&10&6
	\end{pmatrix}$.

	On a
	\begin{align*}
		\sigma &= \begin{pmatrix}
			1&9&10&6&3
		\end{pmatrix} \begin{pmatrix}
			2&8&5
		\end{pmatrix} \begin{pmatrix}
			4&7
		\end{pmatrix}\\
		&= \begin{pmatrix}
			1&3
		\end{pmatrix} \begin{pmatrix}
			1&6
		\end{pmatrix} \begin{pmatrix}
			1&10
		\end{pmatrix} \begin{pmatrix}
			1&9
		\end{pmatrix} \begin{pmatrix}
			2&5
		\end{pmatrix} \begin{pmatrix}
			2&8
		\end{pmatrix} \begin{pmatrix}
			4&7
		\end{pmatrix} \\
	\end{align*}

	Vérification : \[
		\begin{pmatrix}
			1&2&3&4&5&6&7&8&9&10\\
			1&2&3&7&5&6&4&8&9&10\\
			1&8&3&7&5&6&4&2&9&10\\
			1&8&3&7&2&6&4&5&9&10\\
			9&8&3&7&2&6&4&5&1&10\\
			9&8&3&7&2&6&4&5&10&1\\
			9&8&3&7&2&1&4&5&10&6\\
			9&8&1&7&2&3&4&5&10&6\\
		\end{pmatrix} 
	\] 
\end{exm}

	\part{Familles orthogonales}

\begin{thm}[Pythagore]
	Soit $(x,y) \in E^2$. \[
		\|x+y\|^2 = \|x\|^2 + \|y\|^2 \iff x \perp y
	.\]
	\begin{figure}[H]
		\centering
		\begin{asy}
			size(4cm);
			pair u = (1, 0.5);
			pair v = 1.5 * (0, 1) * u;
			draw((0,0)--u, Arrow(TeXHead));
			label("$x$", u/2, align=S);
			draw(u--v+u, Arrow(TeXHead));
			label("$y$", u + v/2, align=NE);
			draw((0,0) -- u + v, Arrow(TeXHead));
			draw(u + v / 7.5 -- u + v / 7.5 - u / 5 -- u - u / 5 -- u -- cycle);
		\end{asy}
	\end{figure}
\end{thm}

\begin{prv}
	\[
		\|x + y\|^2 = \|x\|^2 + \|y\|^2 \iff 2\left<x \mid y \right> = 0 \iff x \perp y
	.\]
\end{prv}

\begin{defn}
	Soit $(e_i)_{i\in I}$ une famille de vecteurs. On dit que cette famille est \underline{orthogonale} si \[
		\forall i \neq j\, e_i \perp e_j
	.\] Si, en plus, on a \[
		\forall i \in I,\,\|e_i\| = 1,
	\] alors on dit que la famille est \underline{orthonormale} ou \underline{orthonormée}.
	\index{famille orthogonale}
	\index{famille orthonormale}
	\index{famille orthonormée}
\end{defn}

\begin{prop}[Pythagore]
	Soit $(e_1, \ldots, e_n)$ une famille orthogonale. Alors \[
		\left\| \sum_{i=1}^n e_i \right\|^2 = \sum_{i=1}^n \|e_i\|^2
	.\]
\end{prop}

\begin{thm}
	Toute famille orthogonale de vecteurs non nuls est libre.
\end{thm}

\begin{prv}
	Soit $(e_i)_{i\in I}$ une famille orthogonale telle que \[
		\forall i \in I,\,e_i \neq 0_E
	.\] Soit $n \in \N^*$, $(\lambda_1, \ldots, \lambda_n) \in \R^n$. On suppose \[
		\sum_{k=1}^n \lambda_k e_{i_k} = 0_E
	.\] Soit $j \in \left\llbracket 1,n \right\rrbracket$.
	\begin{align*}
		0 &= \left<\sum_{k=1}^n \lambda_k e_{i_k}  \mid e_{i_j} \right>\\
		&= \sum_{k=1}^n \lambda_k \left<e_{i_k}  \mid e_{i_j} \right> \\
		&= \lambda_j \underbrace{\|e_{i_j}\|^2}_{\neq 0} \\
	\end{align*}
	donc $\lambda_j = 0$.
\end{prv}

\begin{algo}[Orthonormalisation de Gran--Schmidt]
	On suppose $E$ de dimension finie. Soit $\mathcal{B} = (e_1, \ldots, e_n)$ une base de $E$.

	\begin{itemize}
		\item\underline{\it Étape 1}: On pose $v_1 = \frac{e_1}{\|e_1\|}$ de sorte que $\|v_1\| = 1$.
		\item\underline{\it Étape 2} : On pose \[
				u_2 = e_2 - \left<e_2  \mid v_1 \right> v_1
			.\] Ainsi,
			\begin{align*}
				\left<u_2 \mid v_1 \right> &= \big<e_2 - \left<e_2 \mid v_1 \right> v_1  \mid v_1 \big>\\
				&= \left<e_2 \mid v_1 \right> - \left<e_2 \mid v_1 \right> \left<v_1 \mid v_1 \right> \\
				&= 0. \\
			\end{align*}
			On pose $v_2 = \frac{u_2}{\|u_2\|}$ donc $v_2 \perp v_1$ et $\|v_2\| = 1$.
		\item\underline{\it Étape 3} : On pose \[
				u_2 = e_3 - \left<e_3 \mid v_1 \right>v_1 - \left<e_3 \mid v_2 \right>v_2
			.\] Ainsi,
			\begin{align*}
				\left<u_3  \mid v_1 \right> &= \left<e_3  \mid v_1 \right> - \left<e_3 \mid v_1 \right>\underbrace{\left<v_1 \mid v_1 \right>}_{=1} - \left<e_3 \mid v_2 \right>\underbrace{\left<v_2 \mid v_1 \right>}_{=0} \\
				&= 0 \\
			\end{align*}
			et 
			\begin{align*}
				\left<u_3 \mid v_2 \right> &= \left<e_3  \mid  v_2 \right> - \left<e_3 \mid v_1 \right> \underbrace{\left<v_1 \mid v_2 \right>}_{=0} - \left<e_3 \mid v_2 \right> \underbrace{\left<v_2 \mid v_2 \right>}_{=1}\\
				&= 0. \\
			\end{align*}
			On pose $v_3 = \frac{u_3}{\|u_3\|}$ de sorte que $v_3 \perp v_1$, $v_3 \perp v_2$ et $\|v_3\| = 1$.
		\item\underline{\it Étape $i+1$}: On pose \[
			u_{i+1} = e_{i+1} - \sum_{k=1}^i \left<e_{i+1}  \mid v_k \right> v_k
		.\] Ainsi, pour tout $j \in \left\llbracket 1,i \right\rrbracket,$ on a
		\begin{align*}
			\left<u_{i+1}  \mid v_j \right> &= \left<e_{i+1}  \mid v_j \right> - \sum_{k=1}^i \left<e_{i+1} \mid v_k \right> \left<v_k \mid v_j \right> \\
			&= \left<e_{i+1} \mid v_j \right> - \left<e_{i+1} \mid v_j \right> \|v_j\|^2 \\
			&= 0. \\
		\end{align*}
		On pose $v_{i+1} = \frac{u_{i+1}}{\|u_{i+1}\|}$.
	\end{itemize}
\end{algo}

\begin{exm}
	Avec $E = \R_3[X]$, $\left<P \mid Q \right> = \int_{0}^{1} P(t)\,Q(t)~\mathrm{d}t$ et $\mathcal{B} = (1, X, X^2, X^3)$.
	\begin{enumerate}
		\item $\|1\|^2 = \left<1 \mid 1 \right> = \int_{0}^{1} 1~\mathrm{d}t = 1$ et donc $v_1 = 1$.
		\item $u_2 = X - \left<X  \mid v_1 \right>v_1$. Or, $\left<X \mid v_1 \right> = \int_{0}^{1} t~\mathrm{d}t = \frac{1}{2}$. D'où $u_2 = X - \frac{1}{2}$.
			\begin{align*}
				\|u_2\|^2 &= \int_{0}^{1} \left( t - \frac{1}{2} \right)^2~\mathrm{d}t \\
				&= \int_{0}^{1} \left( t^2 - t + \frac{1}{4} \right)~\mathrm{d}t \\
				&= \frac{1}{3} - \frac{1}{2} + \frac{1}{4} \\
				&= \frac{1}{12} \\
			\end{align*} On en déduit que $v_2 = \sqrt{12}\left( X - \frac{1}{2} \right)$.
		\item $u_3 = X^2 - \left<X^2 \mid v_1 \right>v_1 - \left<X^2 \mid v_2 \right>v_2$.
			On a \[
				\left<X^2 \mid v_1 \right> = \int_{0}^{1} t^2~\mathrm{d}t = \frac{1}{3}
			\] et
			\begin{align*}
				\left<X^2 \mid v_2 \right> &= \sqrt{12} \int_{0}^{1} t^2\left( t - \frac{1}{2} \right)~\mathrm{d}t \\
				&= \frac{\sqrt{12}}{12}. \\
			\end{align*}
			D'où
			\begin{align*}
				u_3 &= X^2 - \frac{1}{3} - \frac{\sqrt{12}}{12}\sqrt{12} \left( X - \frac{1}{2} \right)\\
				&= X^2 - \frac{1}{3} - X + \frac{1}{2} \\
				&= X^2 - X + \frac{1}{6}. \\
			\end{align*}
			\begin{align*}
				\|u_3\|^2 &= \int_{0}^{1} \left( t^2 - t + \frac{1}{6} \right)~\mathrm{d}t\\
				&= \int_{0}^{1} \left( t^4 + t^2 + \frac{1}{36} - 2t^3 + \frac{t^2}{3} - \frac{t}{3} \right) ~\mathrm{d}t \\
				&= \frac{1}{5} + \frac{1}{3} + \frac{1}{36} - \frac{1}{2} + \frac{1}{9} - \frac{1}{6} \\
				&= \frac{36 + 60 + 5 - 90 + 20 - 30}{180} \\
				&= \frac{1}{180} \\
			\end{align*}
			On en déduit que \[
				v_3 = 6\sqrt{5}\left( X^2 - X + \frac{1}{6} \right).
			\]
		\item Exercice : calculer $v_4$.
	\end{enumerate}
\end{exm}

\begin{prop}
	Soit $\mathcal{B} = (e_1, \ldots, e_n)$ une base de $E$ et $\mathcal{C}$ la base obtenue par le procédé d'orthonormalisation de Gram--Schmidt. Alors, \[
		\forall i \in \left\llbracket 1,n \right\rrbracket,\,\Vect(e_1,\ldots, e_i) = \Vect(v_1, \ldots, v_i)
	.\]\qed
\end{prop}

\begin{exm}[orthogonalisation]
	\begin{itemize}
		\item $u_1 = 1$.
		\item
			\begin{align*}
				\begin{rcases*}
					u_2 \in \Vect(e_1, e_2)\\
					u_2 \perp u_1
				\end{rcases*}
				\iff& \begin{cases}
					u_2 = ae_1 + be_2\quad (a,b) \in \R^2\\
					\left<u_1 \mid u_2 \right> = 0
				\end{cases}\\
				\iff& \begin{cases}
					u_2 = a + bX\\
					\int_{0}^{1} (a+bt)~\mathrm{d}t = 0.
				\end{cases}\\
			\end{align*}
			\begin{align*}
				\int_{0}^{1} (a+bt)~\mathrm{d}t = 0 \iff& a + \frac{b}{2} = 0\\
				\iff& a = -\frac{b}{2}\\
				\iff& u_2 = -\frac{b}{2} + bX.
			\end{align*}
			Par exemple, $u_2 = -1 + 2X$.
		\item $\begin{cases}
				u_3 \in \Vect(e_1, e_2, e_3)\\
				u_3 \perp u_1\\
				u_3 \perp u_2
			\end{cases}$

			On pose $u_3 = a + bX + cX^2$ avec $(a,b,c)\in \R^3$.
			\begin{align*}
				\begin{rcases*}
					\int_{0}^{1} \left( a+bt + ct^2 \right)~\mathrm{d}t = 0\\
					\int_{0}^{1} \left(a + bt+ct^2\right)(2t - 1)~\mathrm{d}t = 0
				\end{rcases*} \iff& \begin{cases}
					a + \frac{b}{2} + \frac{c}{3} = 0\\
					\int_{0}^{1} \left( 2ct^3 + (-c + 2b)t^2 + (2a - b)t - a \right) ~\mathrm{d}t = 0
				\end{cases}\\
				\iff& \begin{cases}
					a + \frac{b}{2} + \frac{c}{3} = 0\\
					\frac{c}{2} + \frac{2b - c}{3} + \frac{2\cancel{a} - b}{2} - \cancel{a} = 0
				\end{cases}\\
				\iff&  \begin{cases}
					a = -\frac{b}{2} - \frac{c}{3} = \frac{c}{2} - \frac{c}{3} = \frac{c}{6}\\
					b = -c.
				\end{cases}
			\end{align*}
			On en déduit que \[
				u_3 = 1 - 6X + 6X^2
			.\]
	\end{itemize}
\end{exm}

\begin{crlr}[théorème de la base orthonormée incomplète] Soit $(e_1, \ldots, e_k)$ une base orthonormée d'un espace euclidien. On peut trouver $e_{k+1},\ldots,e_n$ tels que $(e_1, \ldots, e_k, e_{k+1},\ldots,e_n)$ soit une base orthonormée de $E$.
\end{crlr}

\begin{prv}
	On sait que $(e_1, \ldots, e_k)$ est libre. On complète $(e_1, \ldots, e_k)$ en une base $\mathcal{B}$ de $E$. On orthonormalise $\mathcal{B}$ : on obtient une base orthonormée $\mathcal{C}$ de $E$. En détaillant l'algorithme de Gram--Schmidt, on s'aper\c coit que les $k$ premiers vecteurs de $\mathcal{C}$ sont ceux de $\mathcal{B}$.
\end{prv}

\begin{thm}
	Soit $E$ un espace euclidien et $\mathcal{B} = (e_1, \ldots, e_n)$ une base orthonormée de $E$. Soit $(x,y) \in E^2$. On pose $(x_1, \ldots, x_n) \in \R^n$ et $(y_1, \ldots, y_n) \in \R^n$ tels que \[
		x = \sum_{i=1}^n x_i e_i \qquad\qquad y = \sum_{i=1}^n y_i e_i
	.\] Alors \[
		\left<x \mid y \right> = \sum_{i=1}^n x_i y_i
	.\]
	\vspace{3mm}
	Soit $X = \mat{x_1\\\vdots\\x_n}$ et $Y = \mat{y_1\\ \vdots \\ y_n}$. Alors, \[
		\left<x \mid y \right> = X^\T\,Y
	.\]
\end{thm}

\begin{prv}
	\begin{align*}
		\left<x \mid y \right> &= \left<\sum_{i=1}^n x_ie_i  \mid y \right>\\
		&= \sum_{i=1}^n x_i \left<e_i  \mid y \right> \\
		&= \sum_{i=1}^n x_i \left<e_i  \mid \sum_{j=1}^n y_j e_j \right> \\
		&= \sum_{i=1}^n x_i \sum_{j=1}^n y_j \underbrace{\left<e_i \mid e_j \right>}_{\delta_i^j} \\
		&= \sum_{i=1}^n x_i y_i. \\
	\end{align*}
\end{prv}

\begin{prop}
	Soit $E$ un espace euclidien et $\mathcal{B} = (e_1, \ldots, e_n)$ une base orthonormée de $E$. Alors, \[
		\forall x \in E,\,x = \sum_{i=1}^n \left<x \mid e_i \right>e_i
	.\]
\end{prop}

\begin{prv}
	Soit $x \in E$. On pose \[
		x = \sum_{i=1}^n x_i e_i
	\] avec $(x_1, \ldots, x_n) \in \R^n$. Soit $j \in \left\llbracket 1,n \right\rrbracket$. On a
	\begin{align*}
		\left<x \mid e_j \right> &= \left<\sum_{i=1}^n x_i e_i  \mid e_j \right>\\
		&= \sum_{i=1}^n x_i \left<e_i \mid e_j \right> \\
		&= x_j. \\
	\end{align*}
\end{prv}

	\part{Lois de composition}

\begin{defn}
	Une \underline{loi de composition interne} \index{loi de composition interne} est une application $f$ de $E \times E$ dans $E$.
	
	On la note $x * y$ au lieu de $f(x,y)$ (on est libre de choisir le symbôle).
\end{defn}

\begin{defn}
	Soit $E$ un ensemble muni d'une loi de composition interne $\boxtimes$.

	On dit que $\boxtimes$ est \underline{associative} \index{associativité (loi de composition interne)} si \[
		\forall (x,y,z) \in E^3,\;(x\boxtimes y)\boxtimes z = x \boxtimes (y \boxtimes z).
	\] Dans ce cas, on écrit plutôt $x \boxtimes y \boxtimes z$.
\end{defn}

\begin{exm}
	\begin{itemize}
		\item $+$ et $\times $ dans $\C$ sont associatives;
		\item $ \circ$ est associative;
		\item  la multiplication matricielle est aussi associative.
	\end{itemize}
\end{exm}

\begin{defn}
	On dit que $\boxtimes$ est \underline{commutative} \index{commutativité (loi de composition interne)} si \[
		\forall (x,y) \in E^2, x\boxtimes y = y\boxtimes x.
	\]
\end{defn}

\begin{exm}
	\begin{itemize}
		\item $+$ et $\times $ dans $\C$ sont commuatives;
		\item $ \circ $ n'est pas commutative;
		\item  la multiplication matricielle n'est pas commutative.
	\end{itemize}
\end{exm}

\begin{defn}
	Soit $e \in E$. On dit que $e$ est un
	\begin{itemize}
		\item \underline{élément neutre à gauche}\index{élément neutre à gauche (loi de composition interne)} si  \[
				\forall x \in E,\; e\boxtimes x = x;
			\]
		\item \underline{élément neutre à droite}\index{élément neutre à droite (loi de composition interne)} si  \[
				\forall x \in E,\; x\boxtimes e = x;
			\]
		\item \underline{élément neutre}\index{élément neutre (loi de composition interne)} si  \[
				\forall x \in E,\; e\boxtimes x = x\boxtimes e = x.
			\]
	\end{itemize}
\end{defn}

\begin{prop}
	Sous reserve d'existence, il y a unicité de l'élément neutre.
\end{prop}

\begin{prv}
	Soient $e$ et $e'$ deux éléments neutre.
	\begin{itemize}
		\item $e \boxtimes e' = e'$ car $e$ est neutre,
		\item $e \boxtimes e' = e$ car $e'$ est neutre.
	\end{itemize} On a donc $e = e'$.
\end{prv}

\begin{axm}[axiome du choix]
	Soit $E$ un ensemble non vide. Il existe $f : \mathcal{P}(E) \setminus \{\O\} \to E$ telle que \[
		\forall A \in \mathcal{P}(E) \setminus \{\O\},\; f(A) \in A.
	\]
\end{axm}

\begin{defn}
	Soit $f: E \to F$. Le \underline{graphe} \index{graphe (application)} de $f$ est \[
		\Big\{\big(x,f(x)\big)  \mid x \in E\Big\} \subset E \times F.
	\]
\end{defn}

\begin{prop}
	Soit $G \subset E\times F$. $G$ est le graphe d'une application si et seulement si \[
		\forall x \in E,\,\exists! y \in F,\, (x,y) \in G.
	\]
\end{prop}

\begin{prv}
	\begin{itemize}
		\item[``$\implies$''] par définition d'une application
		\item[``$\impliedby$''] On pose $f(x)$ le seul élément $y$ de $F$ qui vérifie $(x,y) \in G$. Alors $f \in F^E$ et son graphe vaut $G$.
	\end{itemize}
\end{prv}

\begin{defn}
	Soit $A \in \mathcal{P}(E)$. L'\underline{indicatrice}\index{indicatrice (ensemble)} de $A$ est \begin{align*}
		\mathbbm{1}_A: E &\longrightarrow \{0,1\} \\
		x &\longmapsto \begin{cases}
			1 &\text{ si } x \in A,\\
			0 & \text{ si } x \not\in A.
		\end{cases}
	\end{align*}
\end{defn}

\begin{exm}
	\begin{enumerate}
		\item Dans $\C$, le neutre de $+$ est $0$ et le neutre de $\times$ est $1$.
		\item Dans $E^E$, le neutre de $ \circ $ est $\id_E$.
		\item Dans $\mathcal{M}_n(\C)$ (l'ensemble des matrices carrées $n \times n$ à valeurs dans $\C$), le neutre de $\times $ est $I_n$ : \[
				I_n =
				\begin{pNiceMatrix}
					1&&(0)\\
					&\Ddots&\\
					(0)&&1
				\end{pNiceMatrix}
			\] 
	\end{enumerate}
\end{exm}

\begin{defn}
	Soit $E$ un ensemble muni d'une loi de composition interne $\boxtimes$ et $x \in E$.

	\begin{enumerate}
		\item On dit que $x$ est \underline{simplifiable à gauche}\index{simplifiabilité à gauche} si \[
				\forall (y,z) \in E^2,\,(x\boxtimes y = x \boxtimes z) \implies x = z.
			\] et que $x$ est \underline{simplifiable à droite}\index{simplifiabilité à droite} si \[
				\forall (y,z) \in E^2,\,(y\boxtimes x = z \boxtimes y) \implies x = z.
			\]
		\item On dit que $x$ est \underline{symétrisable à gauche}\index{symétrisabilité à gauche} s'il exiiste $y \in E$ tel que $y\boxtimes x = e$ où $e$ est l'élément neutre de $\boxtimes$.

			De même, on dit que $x$ est \underline{symétrisable à droite}\index{symétrisabilité à droite} s'il existe $y \in E$ tel que $x \boxtimes y = e$.

			On dit que $x$ est \underline{symétrisable}\index{symétrisabilité} s'il est symétrisable à gauche et à droite, donc s'il existe $y \in E$ tel que $x \boxtimes y = y \boxtimes x = e$.
	\end{enumerate}
\end{defn}

\begin{exm}
	$E = \N$ muni de la loi $+$, tous les éléments de $E$ sont simplifiables. $0$ est le seuele élément de $E$ symétrisable.
\end{exm}

\begin{prop}
	Avec les notations précédentes, si $\boxtimes$ est associative, et $x$ est symétrisable, alors $x$ est simplifiable.
\end{prop}

\begin{prv}
	Soient $y, z \in E$.
	\begin{itemize}
		\item On suppose $x \boxtimes y = x \boxtimes z$. Soit $a \in E$ tel que $a\in E$ tel que $a \boxtimes x = e$. Alors \[
				a \boxtimes (x\boxtimes y) = a \boxtimes (x \boxtimes z).
			\] Or,
			\begin{align*}
				a \boxtimes (x \boxtimes y) &= (a \boxtimes x) \boxtimes y \\
				&= e \boxtimes y \\
				&= y. \\
			\end{align*}

			De même, $a \boxtimes (x \boxtimes z) = z$.

			Donc $y = z$.
		\item De même, si $y \boxtimes x = z \boxtimes x$, on ``multiplie'' $x$ à droite par $a$ et on obtient $y = z$.
	\end{itemize}
\end{prv}

\begin{prop-defn}
	On suppose $\boxtimes$ associative. Soit $x \in E$ symétrisable. Alors \[
		\exists ! y \in E,\; x \boxtimes y = y \boxtimes x = e.
	\] On dit que $y$ est le \underline{symétrique}\index{symétrique (loi de composition interne)} de $x$ et on le note $y = x^*$.
\end{prop-defn}

\begin{prv}
	Soeint $x,y,z \in E$ tels que \[
		\begin{cases}
			 x \boxtimes y = y \boxtimes x = e\\
			 x \boxtimes z = z \boxtimes x = e\\
		\end{cases}
	\] Alors, $x \boxtimes y = x \boxtimes z$ et, en simplifiant par $x$, on a $y = z$.
\end{prv}

\begin{exm}
	Les fonctions symétrisables de $(E^E,  \circ)$ sont les bijections et le symétrique d'une bijection est sa réciproque.
\end{exm}

\begin{rmk}
	\begin{enumerate}
		\item Si la loi est notée $+$, on parle d'\underline{opposé}\index{opposé (loi de composition interne)} plutôt que de symétrique et on le note $-x$ au lieu de $x^*$.
			L'élément neutre est noté $0_E$.
		\item Si la loi est notée $\times$, on parle d'élément \underline{inversible}\index{inversibilité (loi de composition interne)} au lieu de symétrisable, d'\underline{inverse}\index{inverse (loi de composition interne)} au lieu de symétrique et on note $x^{-1}$ au lieu de $x^*$. On note le neutre $1_E$.
	\end{enumerate}
\end{rmk}

\begin{exo}
	Soient $x,y \in E = \R^+_*$. On définit la loi de composition interne $\oplus$ : \[
		x \oplus y = \frac{1}{\frac{1}{x}\oplus \frac{1}{y}}.
	\] Cette loi peut-être utile en physique pour le calcul de résistances équivalentes en parallèles.
	\begin{itemize}
		\item {\sc Associativité} : soient $x,y,z \in E$.

			D'une part, on a \[
				x \oplus (y \oplus z) = \frac{1}{\frac{1}{x} + \frac{1}{\frac{1}{\frac{1}{x}+ \frac{1}{y}}}} = \frac{1}{\frac{1}{x}+\frac{1}{y}+\frac{1}{z}}.
			\] D'autre part, on a \[
			(x \oplus y) \oplus z = \frac{1}{\frac{1}{\frac{1}{\frac{1}{x}+\frac{1}{y}}}+\frac{1}{z}} = \frac{1}{\frac{1}{x}+ \frac{1}{y}+\frac{1}{z}}.
			\] La loi $\oplus$ est associative.
		\item {\sc Commutativité} : soient $x, y \in E$. \[
				x \oplus y = \frac{1}{\frac{1}{x}+\frac{1}{y}} = \frac{1}{\frac{1}{y}+\frac{1}{x}} = y\oplus x.
			\] Donc la loi $\oplus$ est commutative.
		\item {\sc Élément neutre} : soit $e$ l'élément neutre de $\oplus$. \[
				\forall x \in E,\; x \oplus e = e \oplus x = x.
			\] Comme la loi est commutative, seul l'égalité $x \oplus e = x$ est utile.

			Soit $x \in E$. On a donc $\frac{1}{\frac{1}{x}+\frac{1}{e}}=x$ donc $\frac{ex}{e+x}=x$ donc $ex = x(e+x)$ et donc $\cancel{ex} = \cancel{ex} + x^2$. On en déduit que $x^2 = 0$, ce qui n'est pas possible car $x \in \R^+_*$. Donc, il n'y a pas d'élément neutre pour $\oplus$.
	\end{itemize}
\end{exo}


	\chap[19]{Applications linéaires}
	\renewcommand{\cwd}{../chap19}
	\part{Topologie de $\R^2$}

\begin{defn}
	La \underline{norme (euclidienne)} de $\R^2$ est l'application définie par \[
		\forall (x,y) \in \R^2, \|(x,y)\| = \sqrt{x^2 + y^2}.
	\]

	\begin{figure}[H]
		\centering
		\begin{asy}
			import graph;
			axes(EndArrow);
			size(4cm);
			pair A = (3,2);
			dot(A);
			draw((3,0)--A, dashed);
			draw((0,2)--A, dashed);
			label("$x$", (3,0), align=S);
			label("$y$", (0,2), align=W);
			draw((0,0)--A);
			dot((4,3), white+0);
		\end{asy}
	\end{figure}
	\index{norme (de $\R^2$)}
	\index{norme euclidienne (de $\R^2$)}
\end{defn}

\begin{prop}
	La norme euclidienne vérifie:
	\begin{enumerate}
		\item (séparation) \[
			\forall (x,y) \in \R^2, \|(x,y)\| = 0 \iff x = y = 0,
			\]
		\item (homogénéité positive) \[
				\forall \lambda \in \R, \forall (x,y) \in \R^2, \|\lambda(x,y)\|= \left| \lambda \right| \|(x,y)\|
			\]
		\item (inégalité triangulaire) \[
			\forall (x,y), (a,b) \in \R^2,
			\|(x,y)+(a,b)\|\le \|(x,y)\|+\|(a,b)\|.
		\]
	\end{enumerate}
\end{prop}

\begin{prv}
	Déjà vue en replaçant $(x,y)$ par $x+iy \in \C$ et $\|(x,y)\|$ par |x+iy|
\end{prv}

\begin{defn}
	Soit $(a,b) \in \R^2$ et $r \in \R_+$.

	La \underline{boule ouverte} (ou \underline{disque ouvert}) de centre $(a,b)$ et de rayon $r$ est \[
		B_{(a,b)}(r) = \big\{ (x,y) \in \R^2  \mid \|(x,y) - (a,b)\| < r \big\}.
	\]

	La \underline{boule fermée} (ou \underline{disque fermé}) de centre $(a,b)$ et de rayon $r$ est \[
		\overline{B_{(a,b)}}(r) = \big\{ (x,y)\in \R^2  \mid \|(x,y) - (a,b)\| \le r \big\}.
	\]

	La \underline{sphère} (ou \underline{boule}) de centre $(a,b)$ et de rayon $r$ est \[
		S_{(a,b)}(r) = \partial \overline{B_{(a,b)}}(r) = \big\{ (x,y) \in \R^2  \mid \|(x,y) - (a,b)\| = r \big\}.
	\]
	\index{boule ouverte (de $\R^2$)}
	\index{disque ouverte (de $\R^2$)}
	\index{boule fermée (de $\R^2$)}
	\index{disque fermée (de $\R^2$)}
	\index{boule (de $\R^2$)}
	\index{sphère (de $\R^2$)}
\end{defn}

\begin{figure}[H]
		\centering
		\incfig{boule}
\end{figure}

\begin{rmk}
	On parle de boule en dimension quelconque.
\end{rmk}

\begin{defn}
	Une \underline{partie ouverte} $O$ de $\R^2$ (ou \underline{un ouvert}) si \[
		\forall (x,y) \in O, \exists r > 0, B_{(a,b)}(r) \subset O.
	\]
	Une partie $F$ est \underline{fermée} su $\R^2\setminus F$ est ouverte.
	\index{partie ouverte (de $\R^2$)}
	\index{ouvert (de $\R^2$)}
	\index{partie fermée (de $\R^2$)}
\end{defn}

\begin{figure}[H]
	\centering
	\incfig{partie-ouverte}
\end{figure}

\begin{prop}
	Une boule ouverte est ouverte. Une boule fermée est fermée.
\end{prop}

\begin{figure}[H]
	\centering
	\begin{subfigure}{4cm}
		\centering
		\begin{asy}
			import patterns;

			pair n(pair a) {return a / length(a);}

			add("hatch",hatch(2mm, SW, red));
			size(4cm);

			draw(circle((0,0), 1));
			dot('$(a_0, b_0)$', (0,0), align=S);

			draw((0,0) -- n((-1, 1)), dashed);
			label("$r$", n((-1, 1)) / 2, align=1.5S);

			pair A = n((1,3)) * (2/3);
			real rho = (1 - length(A)) * (2 / 3);

			dot("$(a,b)$", A, red, align=3SE);
			filldraw(circle(A, rho), pattern("hatch"), red);

			label("$O$", n((1,-1))*2.5/3);
		\end{asy}
	\end{subfigure}
	\begin{subfigure}{1cm}
		\centering~\\
	\end{subfigure}
	\begin{subfigure}{5cm}
		\centering
		\begin{asy}
			import patterns;

			pair n(pair a) {return a / length(a);}

			add("hatch",hatch(1mm, SW, red));
			add("hatch2",hatch(3mm, SE, blue));
			size(5cm);

			guide around = (-1.5, -1.5) -- (-1.5, 1.5) -- (2.5, 1.5) -- (2.5, -1.5) -- cycle;

			pair A = n((3, 1)) * 5/3; 
			real rho = 2 / 9;

			picture inter;
			fill(inter, around, pattern("hatch2"));
			fill(inter, circle((0,0), 1), white);
			add(inter);

			draw(circle((0,0), 1));
			dot('$(a_0, b_0)$', (0,0), align=S);

			draw((0,0) -- n((-1, 1)), dashed);
			label("$r$", n((-1, 1)) / 2, align=1.5S);

			dot("$(a,b)$", A, red, align=2SE);
			filldraw(circle(A, rho), pattern("hatch"), red);

			label("$F$", n((1,-1))*2.5/3);
		\end{asy}
	\end{subfigure}
\end{figure}

\begin{prv}
	$\O$ est un ouvert.

	Soit $B$ la boule ouverte de centre $(a_0, b_0) \in \R^2$ et de rayon $r > 0$.

	On pose $\rho = \frac{1}{2}\big(r - \|(a,b) - (a_0,b_0)\|\big)$.
	Montrons que \[
		B_{(a,b)}(\rho) \subset  B_{(a,b)}(r).
	\]

	Soit $(x,y) \in B_{(a,b)}(\rho)$.
	\begin{align*}
		\|(x,y) - (a_0,b_0)\|&= \|(x,y)- (a,b) + (a,b) - (a_0,b_0)\| \\
		&\le \|(x,y) - (a,b)\| + \|(a,b) - (a_0, b_0)\|\\
		&< \rho + \|(a,b) - (a_0, b_0)\| = \frac{1}{2}r + \frac{1}{2} \|(a,b) - (a_0, b_0)\|\\
		&< r
	\end{align*}
	
	Soit $F$ la boule fermée de centre $(a_0, b_0)$ et de rayon $r \ge 0$.

	Soit $(a,b) \not\in F$. On pose \[
		\rho = \frac{1}{2}\big(\|(a,b) - (a_0, b_0)\| - r\big) > 0.
	\]

	Montrons que $B_{(a,b)}(\rho) \subset \R^2\setminus F$.

	Soit $(x,y) \in B_{(a,b)}(\rho)$.

	\begin{align*}
		\|(x,y) - (a_0, b_0)\| &= \|(x,y) - (a,b) + (a,b) - (a_0, b_0)\| \\
		&\ge \big| \underbrace{\|(x,y) - (a,b)\|}_{\le \rho} - \underbrace{\|(a,b) - (a_0, b_0)\|}_{> r} \big|\\
		&\ge \|(a,b) - (a_0, b_0)\|- \|(x,y) - (a,b)\|\\
		&> \|(a,b) - (a_0, b_0)\|- \rho\\
		&> \frac{1}{2} \|(a,b) - (a_0, b_0)\| + \frac{1}{2}r\\
		&> r
	\end{align*}

	donc $(x,y) \not\in F$.
\end{prv}

\begin{exm}
	\begin{enumerate}
		\item $\O$ est ouvert.\\
			$\R^2$ est ouvert.
		\item $\O$ est fermé.\\
			$\R^2$ est fermé.\\
		\item $\big\{(x,0)  \mid x > 0\big\}$ n'est ni ouverte ni fermé.
	\end{enumerate}
\end{exm}

\begin{figure}[H]
	\centering
	\begin{asy}
		size(3cm);

		draw((0, -1) -- (0, 3), Arrow(TeXHead));
		draw((-1, 0) -- (3, 0), Arrow(TeXHead));
		
		draw((0,0) -- (0, 2.97), red);
		draw(circle((0,1.5), 0.5), deepred);
		draw(circle((0,0.5), 0.1), deepred);
	\end{asy}
\end{figure}

\begin{defn}
	Soit $(a,b) \in \R^2$ et $V \in \mathcal{P}(\R^2)$.

	On dit que $V$ est un voisinage de $(a,b)$ s'il existe $r > 0$ tel que \[
		B_{(a,b)}(r) \subset V.
	\]
	\index{voisinage (dans $\R^2$)}
\end{defn}

\begin{prop}
	Un ouvert non vide est un voisinage en chacun de ces points. \qed
\end{prop}

\begin{defn}
	Soit $D \subset \R^2$. Un \underline{point intérieur} de $D$ est un couple $(a,b) \in D$ tel que \[
		\exists r > 0, B_{(a,b)}(r) \subset D.
	\] en d'autres termes, si $D$ est un voisinage de $(a,b)$.

	On note $\mathring D$ l'ensemble des points intérieurs à $D$. C'est \underline{l'intérieur} de $D$.
	\index{point intérieur (dans $\R^2$)}
	\index{intérieur (dans $\R^2$)}
\end{defn}

\begin{prop}
	$\mathring D$ est le plus grand ouvert $O$ de $\R^2$ tel que $O \subset D$.
\end{prop}

\begin{figure}[H]
	\centering
	\incfig{interieur-plus-grand-ouvert}
\end{figure}


\begin{prv}
	Soit $(a,b) \in \mathring D$.

	Par définition, il existe $r > 0$ tel que \[
		B_{(a,b)}(r) \subset D.
	\] Montrons que $B_{(a,b)}(r) \subset \mathring D$.

	Soit $(x,y) \in B_{(a,b)}(r)$. Comme $B_{(a,b)}(r)$ est un ouvert de $\R^2$, il existe $\rho > 0$ tel que \[
		B_{(x,y)}(\rho) \subset B_{(a,b)}(r)
	\] donc $(x,y) \in \mathring D$.

	Donc $\mathring D$ est ouvert, $\mathring D \subset D$.

	Soit $O$ un ouvert de $\R^2$ tel que $O \subset D$. Montrons que $O \subset \mathring D$.

	Soit $(x,y) \in O$. Soit $r > 0$ tel que \[
		B_{(x,y)}(r) \subset O \subset D
	\] donc $(x,y) \in \mathring D$.
\end{prv}

\begin{defn}
	Soit $f: D \subset \R^2 \to \R$, $\ell \in \R$, $(a,b) \in \mathring D$.

	On dit que \underline{$f(x,y)$ tend vers $\ell$ quand $(x,y)$ tend vers $(a,b)$} ou que $\ell$ est \underline{une limite} de $f$ en $(a,b)$ si \[
		\forall \varepsilon > 0, \exists r > 0, \forall (x,y) \in D, \|(x,y) - (a,b)\| < r \implies \left| f(x,y) - \ell \right| \le \varepsilon.
	\] en d'autres termes si \[
		\forall V \in \mathcal{V}_{\ell}, \exists W \in \mathcal{V}_{(a,b)}, \forall (x,y) \in W \cap D, f(x,y) \in V.
	\]
	\index{limite (dans $\R^2$)}
	\index{tendre vers (dans $\R^2$)}
\end{defn}

\begin{prop}
	[unicité de la limite]
	Soit $f: D \to \R$, $(a,b) \in \mathring D$, $\ell_1, \ell_2 \in \R$ telles que $\ell_1$ et $\ell_2$ sont des limites de $f$ en $(a,b)$.

	Alors $\ell_1 = \ell_2$.
\end{prop}

\begin{figure}[H]
	\centering
	\incfig{preuve-unicité-de-la-limite}
\end{figure}

\begin{prv}
	On suppose $\ell_1 < \ell_2$. On pose $\varepsilon = \frac{\ell_2 - \ell_1}{2} > 0$.

	Soit $r_1 > 0$ tel que \[
		f\big(B_{(a,b)}(r_1)\big) \subset ]\ell_1 - \varepsilon, \ell_1 + \varepsilon[.
	\] Soit $r_2 > 0$ tel que \[
		f\big(B_{(a,b)}(r_2)\big) \subset ]\ell_2 - \varepsilon, \ell_2 + \varepsilon [.
	\] On pose $r = \min(r_1, r_2)$ donc \[
		B_{(a,b)}(r_1) \cap B_{(a,b)}(r_2) = B_{(a,b)}(r) \neq \O.
	\] Soit $(x,y) \in B_{(a,b)}(r)$. Alors, \[
		f(x,y) \in ]\ell_1 - \varepsilon, \ell_1 + \varepsilon[ \cap ]\ell_2 - \varepsilon, \ell_2 + \varepsilon[ = \O.
	\] $\lightning$
\end{prv}

\begin{defn}
	Soit $f : D \to \R$, $(a,b) \in \mathring D$.

	On dit que $f$ est \underline{continue} en $(a,b)$ si \[
		f(x,y) \tendsto{(x,y) \to (a,b)}f(a,b).
	\]
	\index{continuité (dans $\R^2$)}
\end{defn}

\begin{prop}
	\underline{Si} $f(x,y) \tendsto{(x,y) \to (a,b)} \ell$ \\
	\underline{alors} $\begin{cases}
		f(x,b) \tendsto{x \to a} \ell\\
		f(a,y) \tendsto{y \to b} \ell.\\
	\end{cases}$
\end{prop}

\begin{prv}~\\
	\begin{figure}[H]
		\centering
		\incfig{limite-x-en-a-et-y-en-b}
	\end{figure}
\end{prv}

\underline{Contre-exemple} : exercice 3.

\begin{exm}
	\begin{enumerate}
		\item $f : \begin{array}{rcl}
				\R^2 &\longrightarrow& \R \\
				(x,y) &\longmapsto& x
			\end{array}$ limite en $(0,0)$ ?

			Soit $\varepsilon > 0$. On pose $r = \varepsilon$. \[
				\forall (x,y) \in B_{(0,0)}(r),
				\left| f(x,y) \right| = \left| x \right| \le \|(x,y)\| < r = \varepsilon
			\] Donc $f(x,y) \tendsto{(x,y) \to (a,b)} 0$.
		\item limite $f : \begin{array}{rcl}
				\R^2 &\longrightarrow& \R \\
				(x,y) &\longmapsto& x^3
			\end{array}$ en $(0,0)$ ?

			Soit $\varepsilon > 0$. On pose $r = \sqrt[3]{r} > 0$. \[
				\forall (x,y) \in B_{(0,0)}(r),
				\left| f(x,y) \right| = \left| x^3 \right| \le \|(x,y)\|^3 < r^3 = \varepsilon.
			\]
		\item limite de $f : \begin{array}{rcl}
			\R^2 &\longrightarrow& \R \\
			(x,y) &\longmapsto& x^3y^2
		\end{array}$ en $(0,0)$ ?

		Soit $\varepsilon > 0$. On pose $r = \sqrt[5]{\varepsilon} > 0$. \[
			\forall (x,y) \in B_{(0,0)}(r), \left| f(x,y) \right| = \left| x^3 y^2 \right| \le \|(x,y)\|^3 \|(x,y)\|^2 < r^5 = \varepsilon.
		\]
	\end{enumerate}
\end{exm}

\begin{defn}
	Soient $D \subset \R^2$ et $(x,y) \in \R^2$.

	\begin{figure}[H]
    \centering
    \incfig{point-adhérent}
	\end{figure}
	
	On dit que $(x,y)$ est \underline{adhérent} à $D$ si \[
		\forall r > 0, B_{(x,y)}(r) \cap D \neq \O.
	\] L'ensemble des points adhérents à $D$ est noté $\overline{D}$. On dit que $\overline{D}$ est \underline{l'adhérence} de $D$.
	\index{point adhérent (dans $\R^2$)}
	\index{adhérent (dans $\R^2$)}
\end{defn}

\begin{defn}
	Soit $f: D \subset \R^2 \to \R$ et $(a,b) \in \overline{D}$, $\ell \in \R$. On dit que $f$ tend vers $\ell$ quand $(x,y)$ tend vers $(a,b)$ si \[
		\forall \varepsilon > 0, \exists r > 0, \forall (x,y) \in B_{(a,b)}(r) \cap D,
		\left| f(x,y) - \ell \right| \le \varepsilon.
	\]
	\index{limite (dans $\R^2$)}
	\index{tendre vers (dans $\R^2$)}
\end{defn}

\begin{prop}
	\begin{enumerate}
		\item Dans ce contexte, il y a unicité de la limite
		\item La limite d'une somme, d'un produit, d'un quotien, d'une composée se comporte comme dans le cas d'une seule variable.
		\item Soit $f: D \to \R$ continue. Soient $g: I \to \R$ et $h: I \to \R$ continues telles que \[
			\forall t \in I, \big(g(t), h(t)\big) \in D.
		\] Alors \[
			t \in I \mapsto f\big(g(t), h(t)\big) \in \R
		\] est continue.
	\end{enumerate}
\end{prop}

\begin{figure}[H]
	\centering
	\begin{asy}
		import three;
		import graph3;
		size(5cm);

		settings.render = 0;
		settings.prc = false;
		currentprojection = obliqueX;

		draw(O -- X, Arrow3(TeXHead2));
		draw(O -- Y, Arrow3(TeXHead2));
		draw(O -- Z, Arrow3(TeXHead2));

		triple f(real x, real y, real z = 0) { return (x,y,cos(x - 0.5) * cos(y - 0.5)/1.2 + 0.15); }

		real inc = 1 / 5;

		for(real x = 0; x <= 1; x += inc) {
			draw(graph(
				new real(real t) { return x; }, // x
				new real(real y) { return y; }, // y
				new real(real y) { return f(x,y).z; }, // z
				0, 1
			), gray);
		}

		for(real y = 0; y <= 1; y += inc) {
			draw(graph(
				new real(real x) { return x; }, // x
				new real(real t) { return y; }, // y
				new real(real x) { return f(x,y).z; }, // z
				0, 1
			), gray);
		}

		path3 path1 = (0.3, 0.2, 0) .. (0.5, 0.5, 0) .. (0.6, 0.7, 0) .. (0.9, 0.8, 0);
		path3 path2 = (0.3, 0.8, 0) .. (0.5, 0.5, 0) .. (0.6, 0.3, 0) .. (0.9, 0.2, 0);
		path3 pathA = f(0.3, 0.2, 0) .. f(0.5, 0.5, 0) .. f(0.6, 0.7, 0) .. f(0.9, 0.8, 0);
		path3 pathB = f(0.3, 0.8, 0) .. f(0.5, 0.5, 0) .. f(0.6, 0.3, 0) .. f(0.9, 0.2, 0);

		draw(path1, red, Arrow3(TeXHead2, position=0.5));
		draw(pathA, red, Arrow3(TeXHead2, position=0.5));
		draw(path2, deepcyan, Arrow3(TeXHead2, position=0.3));
		draw(pathB, deepcyan, Arrow3(TeXHead2, position=0.3));

		dot((0.5, 0.5, 0));
		dot(f(0.5, 0.5, 0));
		draw((0.5, 0.5, 0) -- f(0.5, 0.5, 0), dashed);
	\end{asy}
\end{figure}


	\part{Transpositions}

\begin{defn}
	Une \underline{transposition} est un cycle de longueur 2 : $\begin{pmatrix}
		a&b
	\end{pmatrix}$ avec $a \neq b$.
	\index{transposition (permutation)}
\end{defn}

\begin{exm}
	Avec $n = 5$ et $\gamma = \begin{pmatrix}
		2&4&1
	\end{pmatrix}$.

	\begin{figure}[H]
		\centering

		\begin{asy}
			size(5cm);

			real rho = 0.15; // circles

			void draw_cycle(pair O, real r ...int[] nums) {
				int n = nums.length;
				real eps = (15 / r) * 0.8;

				for(int i = 0; i < n; ++i) {
					real theta_1 = (360/n) * (i+1);
					real theta_2 = (360/n) * i;

					pair C = O + dir(theta_2) * r;

					draw(circle(C, rho));
					label("$" + string(nums[i]) + "$", C);
					draw(arc(O, r, theta_2+eps, theta_1-eps), Arrow(TeXHead));
				}
			}

			draw_cycle((-1,0), 0.8, 1, 2, 4);
			draw_cycle((1,0), 0.3, 3);
			draw_cycle((2,0), 0.3, 5);
		\end{asy}
	\end{figure}

	\[
		\gamma = \begin{pmatrix}
			1&4
		\end{pmatrix} \begin{pmatrix}
			1&2
		\end{pmatrix}
	\]

	Avec $n = 6$ et $\gamma = \begin{pmatrix}
		1&3&5&6&2
	\end{pmatrix} = \begin{pmatrix}
		1&2&3&4&5&6\\
		3&1&5&4&6&2
	\end{pmatrix}$.

	Donc, \[
		\gamma = \begin{pmatrix}
			1&2
		\end{pmatrix} \begin{pmatrix}
			1&6
		\end{pmatrix} \begin{pmatrix}
			1&5
		\end{pmatrix} \begin{pmatrix}
			1&3
		\end{pmatrix}
	\] 
	\[
		\begin{pmatrix}
			1&2&3&4&5&6\\
			3&2&1&4&5&6\\
			3&2&5&4&1&6\\
			3&2&5&4&6&1\\
			3&1&5&4&6&2\\
		\end{pmatrix}
	\]

	Et, \[
		\gamma = \begin{pmatrix}
			1&3
		\end{pmatrix} \begin{pmatrix}
			2&3
		\end{pmatrix} \begin{pmatrix}
			3&5
		\end{pmatrix} \begin{pmatrix}
			5&6
		\end{pmatrix} 
	\]

	\[
		\begin{pmatrix}
			1&2&3&4&5&6\\
			1&2&3&4&6&5\\
			1&2&5&4&6&3\\
			1&3&5&4&6&2\\
			3&1&5&4&6&2\\
		\end{pmatrix} 
	\] 
\end{exm}

\begin{exm}
	\[
		\begin{pmatrix}
			1&4
		\end{pmatrix} = \begin{pmatrix}
			1&2
		\end{pmatrix} \begin{pmatrix}
			2&3
		\end{pmatrix} \begin{pmatrix}
			3&4
		\end{pmatrix} \begin{pmatrix}
			2&3
		\end{pmatrix} \begin{pmatrix}
			1&2
		\end{pmatrix}
	\]
	On n'a pas toujours le même nombre de transpositions mais la parité du nombre reste la même (proposition plus loin).
\end{exm}

\begin{thm}
	Toute permutation se décompose en produit de transpositions.
\end{thm}

\begin{prv}
	Soit $\gamma = \begin{pmatrix}
		a_1&\cdots&a_k
	\end{pmatrix}$ un $k$-cycle.

	On remarque que
	\[
		\gamma = \begin{pmatrix}
			a_1&a_k
		\end{pmatrix} \cdots \begin{pmatrix}
			a_1&a_4
		\end{pmatrix} \begin{pmatrix}
			a_1&a_3
		\end{pmatrix} \begin{pmatrix}
			a_1&a_2
		\end{pmatrix}
	\] C'est un produit de transpositions.
\end{prv}

\begin{exm}
	Avec $n = 10$ et $\sigma = \begin{pmatrix}
		1&2&3&4&5&6&7&8&9&10\\
		9&8&1&7&2&3&4&5&10&6
	\end{pmatrix}$.

	On a
	\begin{align*}
		\sigma &= \begin{pmatrix}
			1&9&10&6&3
		\end{pmatrix} \begin{pmatrix}
			2&8&5
		\end{pmatrix} \begin{pmatrix}
			4&7
		\end{pmatrix}\\
		&= \begin{pmatrix}
			1&3
		\end{pmatrix} \begin{pmatrix}
			1&6
		\end{pmatrix} \begin{pmatrix}
			1&10
		\end{pmatrix} \begin{pmatrix}
			1&9
		\end{pmatrix} \begin{pmatrix}
			2&5
		\end{pmatrix} \begin{pmatrix}
			2&8
		\end{pmatrix} \begin{pmatrix}
			4&7
		\end{pmatrix} \\
	\end{align*}

	Vérification : \[
		\begin{pmatrix}
			1&2&3&4&5&6&7&8&9&10\\
			1&2&3&7&5&6&4&8&9&10\\
			1&8&3&7&5&6&4&2&9&10\\
			1&8&3&7&2&6&4&5&9&10\\
			9&8&3&7&2&6&4&5&1&10\\
			9&8&3&7&2&6&4&5&10&1\\
			9&8&3&7&2&1&4&5&10&6\\
			9&8&1&7&2&3&4&5&10&6\\
		\end{pmatrix} 
	\] 
\end{exm}

	\part{Familles orthogonales}

\begin{thm}[Pythagore]
	Soit $(x,y) \in E^2$. \[
		\|x+y\|^2 = \|x\|^2 + \|y\|^2 \iff x \perp y
	.\]
	\begin{figure}[H]
		\centering
		\begin{asy}
			size(4cm);
			pair u = (1, 0.5);
			pair v = 1.5 * (0, 1) * u;
			draw((0,0)--u, Arrow(TeXHead));
			label("$x$", u/2, align=S);
			draw(u--v+u, Arrow(TeXHead));
			label("$y$", u + v/2, align=NE);
			draw((0,0) -- u + v, Arrow(TeXHead));
			draw(u + v / 7.5 -- u + v / 7.5 - u / 5 -- u - u / 5 -- u -- cycle);
		\end{asy}
	\end{figure}
\end{thm}

\begin{prv}
	\[
		\|x + y\|^2 = \|x\|^2 + \|y\|^2 \iff 2\left<x \mid y \right> = 0 \iff x \perp y
	.\]
\end{prv}

\begin{defn}
	Soit $(e_i)_{i\in I}$ une famille de vecteurs. On dit que cette famille est \underline{orthogonale} si \[
		\forall i \neq j\, e_i \perp e_j
	.\] Si, en plus, on a \[
		\forall i \in I,\,\|e_i\| = 1,
	\] alors on dit que la famille est \underline{orthonormale} ou \underline{orthonormée}.
	\index{famille orthogonale}
	\index{famille orthonormale}
	\index{famille orthonormée}
\end{defn}

\begin{prop}[Pythagore]
	Soit $(e_1, \ldots, e_n)$ une famille orthogonale. Alors \[
		\left\| \sum_{i=1}^n e_i \right\|^2 = \sum_{i=1}^n \|e_i\|^2
	.\]
\end{prop}

\begin{thm}
	Toute famille orthogonale de vecteurs non nuls est libre.
\end{thm}

\begin{prv}
	Soit $(e_i)_{i\in I}$ une famille orthogonale telle que \[
		\forall i \in I,\,e_i \neq 0_E
	.\] Soit $n \in \N^*$, $(\lambda_1, \ldots, \lambda_n) \in \R^n$. On suppose \[
		\sum_{k=1}^n \lambda_k e_{i_k} = 0_E
	.\] Soit $j \in \left\llbracket 1,n \right\rrbracket$.
	\begin{align*}
		0 &= \left<\sum_{k=1}^n \lambda_k e_{i_k}  \mid e_{i_j} \right>\\
		&= \sum_{k=1}^n \lambda_k \left<e_{i_k}  \mid e_{i_j} \right> \\
		&= \lambda_j \underbrace{\|e_{i_j}\|^2}_{\neq 0} \\
	\end{align*}
	donc $\lambda_j = 0$.
\end{prv}

\begin{algo}[Orthonormalisation de Gran--Schmidt]
	On suppose $E$ de dimension finie. Soit $\mathcal{B} = (e_1, \ldots, e_n)$ une base de $E$.

	\begin{itemize}
		\item\underline{\it Étape 1}: On pose $v_1 = \frac{e_1}{\|e_1\|}$ de sorte que $\|v_1\| = 1$.
		\item\underline{\it Étape 2} : On pose \[
				u_2 = e_2 - \left<e_2  \mid v_1 \right> v_1
			.\] Ainsi,
			\begin{align*}
				\left<u_2 \mid v_1 \right> &= \big<e_2 - \left<e_2 \mid v_1 \right> v_1  \mid v_1 \big>\\
				&= \left<e_2 \mid v_1 \right> - \left<e_2 \mid v_1 \right> \left<v_1 \mid v_1 \right> \\
				&= 0. \\
			\end{align*}
			On pose $v_2 = \frac{u_2}{\|u_2\|}$ donc $v_2 \perp v_1$ et $\|v_2\| = 1$.
		\item\underline{\it Étape 3} : On pose \[
				u_2 = e_3 - \left<e_3 \mid v_1 \right>v_1 - \left<e_3 \mid v_2 \right>v_2
			.\] Ainsi,
			\begin{align*}
				\left<u_3  \mid v_1 \right> &= \left<e_3  \mid v_1 \right> - \left<e_3 \mid v_1 \right>\underbrace{\left<v_1 \mid v_1 \right>}_{=1} - \left<e_3 \mid v_2 \right>\underbrace{\left<v_2 \mid v_1 \right>}_{=0} \\
				&= 0 \\
			\end{align*}
			et 
			\begin{align*}
				\left<u_3 \mid v_2 \right> &= \left<e_3  \mid  v_2 \right> - \left<e_3 \mid v_1 \right> \underbrace{\left<v_1 \mid v_2 \right>}_{=0} - \left<e_3 \mid v_2 \right> \underbrace{\left<v_2 \mid v_2 \right>}_{=1}\\
				&= 0. \\
			\end{align*}
			On pose $v_3 = \frac{u_3}{\|u_3\|}$ de sorte que $v_3 \perp v_1$, $v_3 \perp v_2$ et $\|v_3\| = 1$.
		\item\underline{\it Étape $i+1$}: On pose \[
			u_{i+1} = e_{i+1} - \sum_{k=1}^i \left<e_{i+1}  \mid v_k \right> v_k
		.\] Ainsi, pour tout $j \in \left\llbracket 1,i \right\rrbracket,$ on a
		\begin{align*}
			\left<u_{i+1}  \mid v_j \right> &= \left<e_{i+1}  \mid v_j \right> - \sum_{k=1}^i \left<e_{i+1} \mid v_k \right> \left<v_k \mid v_j \right> \\
			&= \left<e_{i+1} \mid v_j \right> - \left<e_{i+1} \mid v_j \right> \|v_j\|^2 \\
			&= 0. \\
		\end{align*}
		On pose $v_{i+1} = \frac{u_{i+1}}{\|u_{i+1}\|}$.
	\end{itemize}
\end{algo}

\begin{exm}
	Avec $E = \R_3[X]$, $\left<P \mid Q \right> = \int_{0}^{1} P(t)\,Q(t)~\mathrm{d}t$ et $\mathcal{B} = (1, X, X^2, X^3)$.
	\begin{enumerate}
		\item $\|1\|^2 = \left<1 \mid 1 \right> = \int_{0}^{1} 1~\mathrm{d}t = 1$ et donc $v_1 = 1$.
		\item $u_2 = X - \left<X  \mid v_1 \right>v_1$. Or, $\left<X \mid v_1 \right> = \int_{0}^{1} t~\mathrm{d}t = \frac{1}{2}$. D'où $u_2 = X - \frac{1}{2}$.
			\begin{align*}
				\|u_2\|^2 &= \int_{0}^{1} \left( t - \frac{1}{2} \right)^2~\mathrm{d}t \\
				&= \int_{0}^{1} \left( t^2 - t + \frac{1}{4} \right)~\mathrm{d}t \\
				&= \frac{1}{3} - \frac{1}{2} + \frac{1}{4} \\
				&= \frac{1}{12} \\
			\end{align*} On en déduit que $v_2 = \sqrt{12}\left( X - \frac{1}{2} \right)$.
		\item $u_3 = X^2 - \left<X^2 \mid v_1 \right>v_1 - \left<X^2 \mid v_2 \right>v_2$.
			On a \[
				\left<X^2 \mid v_1 \right> = \int_{0}^{1} t^2~\mathrm{d}t = \frac{1}{3}
			\] et
			\begin{align*}
				\left<X^2 \mid v_2 \right> &= \sqrt{12} \int_{0}^{1} t^2\left( t - \frac{1}{2} \right)~\mathrm{d}t \\
				&= \frac{\sqrt{12}}{12}. \\
			\end{align*}
			D'où
			\begin{align*}
				u_3 &= X^2 - \frac{1}{3} - \frac{\sqrt{12}}{12}\sqrt{12} \left( X - \frac{1}{2} \right)\\
				&= X^2 - \frac{1}{3} - X + \frac{1}{2} \\
				&= X^2 - X + \frac{1}{6}. \\
			\end{align*}
			\begin{align*}
				\|u_3\|^2 &= \int_{0}^{1} \left( t^2 - t + \frac{1}{6} \right)~\mathrm{d}t\\
				&= \int_{0}^{1} \left( t^4 + t^2 + \frac{1}{36} - 2t^3 + \frac{t^2}{3} - \frac{t}{3} \right) ~\mathrm{d}t \\
				&= \frac{1}{5} + \frac{1}{3} + \frac{1}{36} - \frac{1}{2} + \frac{1}{9} - \frac{1}{6} \\
				&= \frac{36 + 60 + 5 - 90 + 20 - 30}{180} \\
				&= \frac{1}{180} \\
			\end{align*}
			On en déduit que \[
				v_3 = 6\sqrt{5}\left( X^2 - X + \frac{1}{6} \right).
			\]
		\item Exercice : calculer $v_4$.
	\end{enumerate}
\end{exm}

\begin{prop}
	Soit $\mathcal{B} = (e_1, \ldots, e_n)$ une base de $E$ et $\mathcal{C}$ la base obtenue par le procédé d'orthonormalisation de Gram--Schmidt. Alors, \[
		\forall i \in \left\llbracket 1,n \right\rrbracket,\,\Vect(e_1,\ldots, e_i) = \Vect(v_1, \ldots, v_i)
	.\]\qed
\end{prop}

\begin{exm}[orthogonalisation]
	\begin{itemize}
		\item $u_1 = 1$.
		\item
			\begin{align*}
				\begin{rcases*}
					u_2 \in \Vect(e_1, e_2)\\
					u_2 \perp u_1
				\end{rcases*}
				\iff& \begin{cases}
					u_2 = ae_1 + be_2\quad (a,b) \in \R^2\\
					\left<u_1 \mid u_2 \right> = 0
				\end{cases}\\
				\iff& \begin{cases}
					u_2 = a + bX\\
					\int_{0}^{1} (a+bt)~\mathrm{d}t = 0.
				\end{cases}\\
			\end{align*}
			\begin{align*}
				\int_{0}^{1} (a+bt)~\mathrm{d}t = 0 \iff& a + \frac{b}{2} = 0\\
				\iff& a = -\frac{b}{2}\\
				\iff& u_2 = -\frac{b}{2} + bX.
			\end{align*}
			Par exemple, $u_2 = -1 + 2X$.
		\item $\begin{cases}
				u_3 \in \Vect(e_1, e_2, e_3)\\
				u_3 \perp u_1\\
				u_3 \perp u_2
			\end{cases}$

			On pose $u_3 = a + bX + cX^2$ avec $(a,b,c)\in \R^3$.
			\begin{align*}
				\begin{rcases*}
					\int_{0}^{1} \left( a+bt + ct^2 \right)~\mathrm{d}t = 0\\
					\int_{0}^{1} \left(a + bt+ct^2\right)(2t - 1)~\mathrm{d}t = 0
				\end{rcases*} \iff& \begin{cases}
					a + \frac{b}{2} + \frac{c}{3} = 0\\
					\int_{0}^{1} \left( 2ct^3 + (-c + 2b)t^2 + (2a - b)t - a \right) ~\mathrm{d}t = 0
				\end{cases}\\
				\iff& \begin{cases}
					a + \frac{b}{2} + \frac{c}{3} = 0\\
					\frac{c}{2} + \frac{2b - c}{3} + \frac{2\cancel{a} - b}{2} - \cancel{a} = 0
				\end{cases}\\
				\iff&  \begin{cases}
					a = -\frac{b}{2} - \frac{c}{3} = \frac{c}{2} - \frac{c}{3} = \frac{c}{6}\\
					b = -c.
				\end{cases}
			\end{align*}
			On en déduit que \[
				u_3 = 1 - 6X + 6X^2
			.\]
	\end{itemize}
\end{exm}

\begin{crlr}[théorème de la base orthonormée incomplète] Soit $(e_1, \ldots, e_k)$ une base orthonormée d'un espace euclidien. On peut trouver $e_{k+1},\ldots,e_n$ tels que $(e_1, \ldots, e_k, e_{k+1},\ldots,e_n)$ soit une base orthonormée de $E$.
\end{crlr}

\begin{prv}
	On sait que $(e_1, \ldots, e_k)$ est libre. On complète $(e_1, \ldots, e_k)$ en une base $\mathcal{B}$ de $E$. On orthonormalise $\mathcal{B}$ : on obtient une base orthonormée $\mathcal{C}$ de $E$. En détaillant l'algorithme de Gram--Schmidt, on s'aper\c coit que les $k$ premiers vecteurs de $\mathcal{C}$ sont ceux de $\mathcal{B}$.
\end{prv}

\begin{thm}
	Soit $E$ un espace euclidien et $\mathcal{B} = (e_1, \ldots, e_n)$ une base orthonormée de $E$. Soit $(x,y) \in E^2$. On pose $(x_1, \ldots, x_n) \in \R^n$ et $(y_1, \ldots, y_n) \in \R^n$ tels que \[
		x = \sum_{i=1}^n x_i e_i \qquad\qquad y = \sum_{i=1}^n y_i e_i
	.\] Alors \[
		\left<x \mid y \right> = \sum_{i=1}^n x_i y_i
	.\]
	\vspace{3mm}
	Soit $X = \mat{x_1\\\vdots\\x_n}$ et $Y = \mat{y_1\\ \vdots \\ y_n}$. Alors, \[
		\left<x \mid y \right> = X^\T\,Y
	.\]
\end{thm}

\begin{prv}
	\begin{align*}
		\left<x \mid y \right> &= \left<\sum_{i=1}^n x_ie_i  \mid y \right>\\
		&= \sum_{i=1}^n x_i \left<e_i  \mid y \right> \\
		&= \sum_{i=1}^n x_i \left<e_i  \mid \sum_{j=1}^n y_j e_j \right> \\
		&= \sum_{i=1}^n x_i \sum_{j=1}^n y_j \underbrace{\left<e_i \mid e_j \right>}_{\delta_i^j} \\
		&= \sum_{i=1}^n x_i y_i. \\
	\end{align*}
\end{prv}

\begin{prop}
	Soit $E$ un espace euclidien et $\mathcal{B} = (e_1, \ldots, e_n)$ une base orthonormée de $E$. Alors, \[
		\forall x \in E,\,x = \sum_{i=1}^n \left<x \mid e_i \right>e_i
	.\]
\end{prop}

\begin{prv}
	Soit $x \in E$. On pose \[
		x = \sum_{i=1}^n x_i e_i
	\] avec $(x_1, \ldots, x_n) \in \R^n$. Soit $j \in \left\llbracket 1,n \right\rrbracket$. On a
	\begin{align*}
		\left<x \mid e_j \right> &= \left<\sum_{i=1}^n x_i e_i  \mid e_j \right>\\
		&= \sum_{i=1}^n x_i \left<e_i \mid e_j \right> \\
		&= x_j. \\
	\end{align*}
\end{prv}

	\part{Lois de composition}

\begin{defn}
	Une \underline{loi de composition interne} \index{loi de composition interne} est une application $f$ de $E \times E$ dans $E$.
	
	On la note $x * y$ au lieu de $f(x,y)$ (on est libre de choisir le symbôle).
\end{defn}

\begin{defn}
	Soit $E$ un ensemble muni d'une loi de composition interne $\boxtimes$.

	On dit que $\boxtimes$ est \underline{associative} \index{associativité (loi de composition interne)} si \[
		\forall (x,y,z) \in E^3,\;(x\boxtimes y)\boxtimes z = x \boxtimes (y \boxtimes z).
	\] Dans ce cas, on écrit plutôt $x \boxtimes y \boxtimes z$.
\end{defn}

\begin{exm}
	\begin{itemize}
		\item $+$ et $\times $ dans $\C$ sont associatives;
		\item $ \circ$ est associative;
		\item  la multiplication matricielle est aussi associative.
	\end{itemize}
\end{exm}

\begin{defn}
	On dit que $\boxtimes$ est \underline{commutative} \index{commutativité (loi de composition interne)} si \[
		\forall (x,y) \in E^2, x\boxtimes y = y\boxtimes x.
	\]
\end{defn}

\begin{exm}
	\begin{itemize}
		\item $+$ et $\times $ dans $\C$ sont commuatives;
		\item $ \circ $ n'est pas commutative;
		\item  la multiplication matricielle n'est pas commutative.
	\end{itemize}
\end{exm}

\begin{defn}
	Soit $e \in E$. On dit que $e$ est un
	\begin{itemize}
		\item \underline{élément neutre à gauche}\index{élément neutre à gauche (loi de composition interne)} si  \[
				\forall x \in E,\; e\boxtimes x = x;
			\]
		\item \underline{élément neutre à droite}\index{élément neutre à droite (loi de composition interne)} si  \[
				\forall x \in E,\; x\boxtimes e = x;
			\]
		\item \underline{élément neutre}\index{élément neutre (loi de composition interne)} si  \[
				\forall x \in E,\; e\boxtimes x = x\boxtimes e = x.
			\]
	\end{itemize}
\end{defn}

\begin{prop}
	Sous reserve d'existence, il y a unicité de l'élément neutre.
\end{prop}

\begin{prv}
	Soient $e$ et $e'$ deux éléments neutre.
	\begin{itemize}
		\item $e \boxtimes e' = e'$ car $e$ est neutre,
		\item $e \boxtimes e' = e$ car $e'$ est neutre.
	\end{itemize} On a donc $e = e'$.
\end{prv}

\begin{axm}[axiome du choix]
	Soit $E$ un ensemble non vide. Il existe $f : \mathcal{P}(E) \setminus \{\O\} \to E$ telle que \[
		\forall A \in \mathcal{P}(E) \setminus \{\O\},\; f(A) \in A.
	\]
\end{axm}

\begin{defn}
	Soit $f: E \to F$. Le \underline{graphe} \index{graphe (application)} de $f$ est \[
		\Big\{\big(x,f(x)\big)  \mid x \in E\Big\} \subset E \times F.
	\]
\end{defn}

\begin{prop}
	Soit $G \subset E\times F$. $G$ est le graphe d'une application si et seulement si \[
		\forall x \in E,\,\exists! y \in F,\, (x,y) \in G.
	\]
\end{prop}

\begin{prv}
	\begin{itemize}
		\item[``$\implies$''] par définition d'une application
		\item[``$\impliedby$''] On pose $f(x)$ le seul élément $y$ de $F$ qui vérifie $(x,y) \in G$. Alors $f \in F^E$ et son graphe vaut $G$.
	\end{itemize}
\end{prv}

\begin{defn}
	Soit $A \in \mathcal{P}(E)$. L'\underline{indicatrice}\index{indicatrice (ensemble)} de $A$ est \begin{align*}
		\mathbbm{1}_A: E &\longrightarrow \{0,1\} \\
		x &\longmapsto \begin{cases}
			1 &\text{ si } x \in A,\\
			0 & \text{ si } x \not\in A.
		\end{cases}
	\end{align*}
\end{defn}

\begin{exm}
	\begin{enumerate}
		\item Dans $\C$, le neutre de $+$ est $0$ et le neutre de $\times$ est $1$.
		\item Dans $E^E$, le neutre de $ \circ $ est $\id_E$.
		\item Dans $\mathcal{M}_n(\C)$ (l'ensemble des matrices carrées $n \times n$ à valeurs dans $\C$), le neutre de $\times $ est $I_n$ : \[
				I_n =
				\begin{pNiceMatrix}
					1&&(0)\\
					&\Ddots&\\
					(0)&&1
				\end{pNiceMatrix}
			\] 
	\end{enumerate}
\end{exm}

\begin{defn}
	Soit $E$ un ensemble muni d'une loi de composition interne $\boxtimes$ et $x \in E$.

	\begin{enumerate}
		\item On dit que $x$ est \underline{simplifiable à gauche}\index{simplifiabilité à gauche} si \[
				\forall (y,z) \in E^2,\,(x\boxtimes y = x \boxtimes z) \implies x = z.
			\] et que $x$ est \underline{simplifiable à droite}\index{simplifiabilité à droite} si \[
				\forall (y,z) \in E^2,\,(y\boxtimes x = z \boxtimes y) \implies x = z.
			\]
		\item On dit que $x$ est \underline{symétrisable à gauche}\index{symétrisabilité à gauche} s'il exiiste $y \in E$ tel que $y\boxtimes x = e$ où $e$ est l'élément neutre de $\boxtimes$.

			De même, on dit que $x$ est \underline{symétrisable à droite}\index{symétrisabilité à droite} s'il existe $y \in E$ tel que $x \boxtimes y = e$.

			On dit que $x$ est \underline{symétrisable}\index{symétrisabilité} s'il est symétrisable à gauche et à droite, donc s'il existe $y \in E$ tel que $x \boxtimes y = y \boxtimes x = e$.
	\end{enumerate}
\end{defn}

\begin{exm}
	$E = \N$ muni de la loi $+$, tous les éléments de $E$ sont simplifiables. $0$ est le seuele élément de $E$ symétrisable.
\end{exm}

\begin{prop}
	Avec les notations précédentes, si $\boxtimes$ est associative, et $x$ est symétrisable, alors $x$ est simplifiable.
\end{prop}

\begin{prv}
	Soient $y, z \in E$.
	\begin{itemize}
		\item On suppose $x \boxtimes y = x \boxtimes z$. Soit $a \in E$ tel que $a\in E$ tel que $a \boxtimes x = e$. Alors \[
				a \boxtimes (x\boxtimes y) = a \boxtimes (x \boxtimes z).
			\] Or,
			\begin{align*}
				a \boxtimes (x \boxtimes y) &= (a \boxtimes x) \boxtimes y \\
				&= e \boxtimes y \\
				&= y. \\
			\end{align*}

			De même, $a \boxtimes (x \boxtimes z) = z$.

			Donc $y = z$.
		\item De même, si $y \boxtimes x = z \boxtimes x$, on ``multiplie'' $x$ à droite par $a$ et on obtient $y = z$.
	\end{itemize}
\end{prv}

\begin{prop-defn}
	On suppose $\boxtimes$ associative. Soit $x \in E$ symétrisable. Alors \[
		\exists ! y \in E,\; x \boxtimes y = y \boxtimes x = e.
	\] On dit que $y$ est le \underline{symétrique}\index{symétrique (loi de composition interne)} de $x$ et on le note $y = x^*$.
\end{prop-defn}

\begin{prv}
	Soeint $x,y,z \in E$ tels que \[
		\begin{cases}
			 x \boxtimes y = y \boxtimes x = e\\
			 x \boxtimes z = z \boxtimes x = e\\
		\end{cases}
	\] Alors, $x \boxtimes y = x \boxtimes z$ et, en simplifiant par $x$, on a $y = z$.
\end{prv}

\begin{exm}
	Les fonctions symétrisables de $(E^E,  \circ)$ sont les bijections et le symétrique d'une bijection est sa réciproque.
\end{exm}

\begin{rmk}
	\begin{enumerate}
		\item Si la loi est notée $+$, on parle d'\underline{opposé}\index{opposé (loi de composition interne)} plutôt que de symétrique et on le note $-x$ au lieu de $x^*$.
			L'élément neutre est noté $0_E$.
		\item Si la loi est notée $\times$, on parle d'élément \underline{inversible}\index{inversibilité (loi de composition interne)} au lieu de symétrisable, d'\underline{inverse}\index{inverse (loi de composition interne)} au lieu de symétrique et on note $x^{-1}$ au lieu de $x^*$. On note le neutre $1_E$.
	\end{enumerate}
\end{rmk}

\begin{exo}
	Soient $x,y \in E = \R^+_*$. On définit la loi de composition interne $\oplus$ : \[
		x \oplus y = \frac{1}{\frac{1}{x}\oplus \frac{1}{y}}.
	\] Cette loi peut-être utile en physique pour le calcul de résistances équivalentes en parallèles.
	\begin{itemize}
		\item {\sc Associativité} : soient $x,y,z \in E$.

			D'une part, on a \[
				x \oplus (y \oplus z) = \frac{1}{\frac{1}{x} + \frac{1}{\frac{1}{\frac{1}{x}+ \frac{1}{y}}}} = \frac{1}{\frac{1}{x}+\frac{1}{y}+\frac{1}{z}}.
			\] D'autre part, on a \[
			(x \oplus y) \oplus z = \frac{1}{\frac{1}{\frac{1}{\frac{1}{x}+\frac{1}{y}}}+\frac{1}{z}} = \frac{1}{\frac{1}{x}+ \frac{1}{y}+\frac{1}{z}}.
			\] La loi $\oplus$ est associative.
		\item {\sc Commutativité} : soient $x, y \in E$. \[
				x \oplus y = \frac{1}{\frac{1}{x}+\frac{1}{y}} = \frac{1}{\frac{1}{y}+\frac{1}{x}} = y\oplus x.
			\] Donc la loi $\oplus$ est commutative.
		\item {\sc Élément neutre} : soit $e$ l'élément neutre de $\oplus$. \[
				\forall x \in E,\; x \oplus e = e \oplus x = x.
			\] Comme la loi est commutative, seul l'égalité $x \oplus e = x$ est utile.

			Soit $x \in E$. On a donc $\frac{1}{\frac{1}{x}+\frac{1}{e}}=x$ donc $\frac{ex}{e+x}=x$ donc $ex = x(e+x)$ et donc $\cancel{ex} = \cancel{ex} + x^2$. On en déduit que $x^2 = 0$, ce qui n'est pas possible car $x \in \R^+_*$. Donc, il n'y a pas d'élément neutre pour $\oplus$.
	\end{itemize}
\end{exo}

	\part{Divers}

\begin{defn}
	Soient $E$ et $F$ deux ensembles. Un \underline{couple}\index{couple} $(x,y)$ est la donnée d'un élément $x$ de $E$ et d'un élément $y$ de $F$ où \[
		\forall x,x' \in E,\,\forall y,y' \in F,\qquad (x,y) = (x',y') \iff \begin{cases}
			x=x',\\
			y=y'.
		\end{cases}
	\] On note $E \times F$ l'ensemble des couples; c'est le \underline{produit cartésien}\index{produit cartésion (ensembles)} de $E$ et $F$.
\end{defn}

\begin{exm}
	$D \times [0,1]$ est un cylindre plein où $D$ est le disque unité fermé i.e. \[
		D = \Big\{(x,y) \in \R^2 \mid x^2+y^2 \le 1\Big\}.
	\]
	\begin{figure}[H]
		\centering
		\begin{subfigure}[b]{3cm}
			\centering
			\begin{asy}
				size(3cm);
				draw(unitcircle);
				draw((0,0)--(1,0), red);
				label("$1$",(0.5,0), red, align=S);
			\end{asy}
		\end{subfigure}
		\begin{subfigure}[b]{3cm}
			\centering
			\begin{asy}
				size(3cm);
				label("$\times\; [0,1]\; =$", (0,0), fontsize(10));
				draw(unitcircle, white+0);
			\end{asy}
		\end{subfigure}
		\begin{subfigure}[b]{3cm}
			\centering
			\begin{asy}
				import solids;
				size(3cm);
				draw(shift((0, 0.5)) * unitcircle, white+0);
				revolution r = cylinder(O, 1, 1.5, Z);
				draw(r);
				triple M = (-1/2, sqrt(3)/2, 0);
				draw((0,0,0) -- M, red);
				label("$1$", M/2, red, align=S);
				draw(M*1.1--M*1.1+(0,0,1.5), magenta, Arrows3(TeXHead2));
				label("$1$", M*1.1+(0,0,0.75), magenta, align=E);
			\end{asy}
		\end{subfigure}
	\end{figure}

	$C \times C$ où $C = \Big\{(x,y) \in \R^2  \mid x^2 + y^2 = 1\Big\}$ est un tore (creu).

	\begin{figure}[H]
		\centering
		\begin{subfigure}[b]{3cm}
			\centering
			\begin{asy}
				size(3cm);
				draw(unitcircle);
				draw((0,0)--(1,0), red);
				label("$1$",(0.5,0), red, align=S);
			\end{asy}
		\end{subfigure}
		\begin{subfigure}[b]{1cm}
			\centering
			\begin{asy}
				size(3cm);
				label("$\times$", (0,0), fontsize(10));
				dot((0.1, 1), white+0);
				dot((-0.1, -1), white+0);
			\end{asy}
		\end{subfigure}
		\begin{subfigure}[b]{3cm}
			\centering
			\begin{asy}
				size(3cm);
				draw(unitcircle);
				draw((0,0)--(1,0), red);
				label("$1$",(0.5,0), red, align=S);
			\end{asy}
		\end{subfigure}
		\begin{subfigure}[b]{1cm}
			\centering
			\begin{asy}
				size(3cm);
				label("$=$", (0,0), fontsize(10));
				dot((0.1, 1), white+0);
				dot((-0.1, -1), white+0);
			\end{asy}
		\end{subfigure}
		\begin{subfigure}[b]{3cm}
			\centering
			\begin{asy}
				import three;
				import graph3;

				size(3cm,3cm);
				surface torus = surface(Circle(c=2Y,normal=X,r=0.5,n=32), c=O, axis=Z, n=32);

				draw(torus, white + opacity(0), meshpen=black + 0.2pt, nolight, render(merge=true));
			\end{asy}
			\vspace{0.7cm}
		\end{subfigure}
	\end{figure}
\end{exm}

\begin{defn}
	Soient $E$ et $F$ deux ensembles. On dit que $E$ et $F$ sont \underline{équipotents} s'il existe une bijection de $E$ dans $F$.
	\index{équipotence (ensembles)}
\end{defn}

\begin{exm}
	\begin{enumerate}
		\item $\N$ et $\N^*$ sont équipotents car  $f : \begin{array}{rcl}
				\N &\longrightarrow& \N^* \\
				k &\longmapsto& k + 1
			\end{array}$ est bijective.
		\item $P = \{n \in \N  \mid n \text{ pair}\}$ et $I= \{n \in \N \mid n \text{ impair}\}$ sont équipotents car $f : \begin{array}{rcl}
				P &\longrightarrow& I \\
				x &\longmapsto& x+1
			\end{array}$ est bijective.
		\item $\N$ et $P$ sont équipotents car $f : \begin{array}{rcl}
				\N &\longrightarrow& P \\
				k &\longmapsto& 2k
			\end{array}$ est bijective.
		\item $[0,1]$ et $[0,1[$ sont équipotents car \begin{align*}
			f: [0,1] &\longrightarrow [0,1[ \\
			x &\longmapsto \begin{cases}
				\frac{1}{n+1} &\text{ si } x = \frac{1}{n} \text{ avec } n \in \N^*\\
				x &\text{ sinon}
			\end{cases}
		\end{align*} est bijective.
		\item De même, $]0,1[$ et $]0,1]$ sont équipotents.
		\item $]0,1[$ et $[0,1[$ sont équipotents : $f : \begin{array}{rcl}
					]0,1] &\longrightarrow& [0,1[ \\
				x &\longmapsto& 1-x
			\end{array}$ est bijective.
		\item $\forall a < b$, $[a,b]$ et $[0,1]$ sont équipotents : \begin{align*}
				f: [0,1] &\longrightarrow [a,b] \\
				\alpha &\longmapsto \alpha b + (1 - \alpha) a
			\end{align*} est bijective (interpolation linéaire).
		\item $\R$ et $]0,1[$ sont équipotents : \begin{align*}
				f: \R &\longrightarrow ]0,1[ \\
				x &\longmapsto \frac{1}{2} + \frac{\Arctan x}{\pi}
			\end{align*} est bijective.
		\item $[0,1[$ et $\N$ ne sont pas équipotents (argument de Cantor). Soit $f: \N \to [0,1[$ une bijection :
			\[
				\begin{array}{c|l}
					k&\hfill f(k)\hfill~ \\ \hline
					0&0,\hfill \!0\hfill 0\hfill 0\hfill 0\hfill\ldots\\
					1&0,\hfill a_1\hfill a_2\hfill a_3\hfill a_4\hfill\ldots\\
					2&0,\hfill b_1\hfill b_2\hfill b_3\hfill b_4\hfill\ldots\\
					\vdots&\hfill\vdots\hfill\ddots
				\end{array}
			\] On considère le nombre \[
				x = 0,\,(a_0+1)(b_1+1)(c_2+1)\cdots
			\] $f(1) \neq x$ car ils n'ont pas le même chiffre des dizaines.\\
			$f(2) \neq x$ car ils n'ont pas le même chiffre des centaines.

			Par le même raisonement, on en déduit que \[
				\forall n \in \N, f(n) \neq x
			\] donc $x$ n'a pas d'antécédant : une contradiction.
		\item On verra en exercice que $E$ et $\mathcal{P}(E)$ ne sont pas équipotents. $\R$ et $\mathcal{P}(\R)$ ne sont pas équipotents mais $\R$ et $\mathcal{P}(\N)$ le sont (développement dyadique).
		\item $\R^2$ et $\R$ sont équipotents; $\C$ et $\R$ sont équipotents.
	\end{enumerate}
\end{exm}

\begin{exo}
	Soit $E$ un ensemble. L'application \begin{align*}
		f: \mathcal{P}(E) &\longrightarrow {0,1}^E \\
		A &\longmapsto \mathbbm{1}_A
	\end{align*} est bijective.

	Soit $g : E \to \{0,1\}$.
	\begin{itemize}
		\item[\underline{\sc Analyse}] Soit $A \in \mathcal{P}(E)$ tel que $f(A) = g$. Alors $g = \mathbbm{1}_A$.
			donc  \[
				\forall x \in E,\; g(x) = \mathbbm{1}_A(x)
			\] et donc \[
				\begin{cases}
					\forall x \in A,\, g(x) = 1\\
					\forall x \in E \setminus A,\,g(x) = 0
				\end{cases}
			\] On en déduit que \[
				A = \{ x \in E  \mid  g(x) = 1\}  = g^{-1}\big(\{1\}\big).
			\]
		\item[\underline{\sc Synthèse}] On pose $A = g^{-1}\big(\{1\}\big)$. Montrons que $f(A) = g$.
			\[
				\forall x \in E,\,g(x) = \begin{cases}
					1 &\text{ si } x \in A\\
					0 &\text{ si } x \not\in A
				\end{cases} = \mathbbm{1}_A
			\] donc $g = \mathbbm{1}_A$.
	\end{itemize}

	On aurait aussi pu rédiger de la fa\c con suivante : on pose \begin{align*}
		u: \{0,1\}^E &\longrightarrow \mathcal{P}(E) \\
		g &\longmapsto g^{-1}\big(\{1\}\big).
	\end{align*} On montre que $u$ est la réciproque de $f$ : \[
		\begin{cases}
			f \circ u = \id_{\{0,1\}^E},\\
			u \circ f = \id_{\mathcal{P}(E)}.
		\end{cases}
	\]
\end{exo}

\begin{defn}
	Soit $f : E \to F$. L'\underline{image de $f$}\index{image (application)} est \[
		\mathrm{Im}(f) = f(E) = \big\{f(x) \mid x \in E\big\}.
	\]
\end{defn}

\begin{prop}
	Soit $f: E \to F$. \[
		f \text{ est surjective } \iff f(E) = F.
	\]
\end{prop}

\begin{defn}
	Une \underline{suite de $E$}\index{suite (ensemble)} est une application de $\N$ dans $E$.
\end{defn}

\begin{rmk}[Notation]
	Soit $u \in E^\N$. Pour $n \in \N$, on écrit $u_n$ à la place de $u(n)$.
\end{rmk}

\begin{defn}
	Soient $E$ et $I$ deux ensembles. Une \underline{famille de $E$ indéxée par $I$}\index{famille (ensemble)} est une application de $I$ dans $E$.

	À la place de $u(i)$ (avec $i \in I$), on écrit $u_i$.
\end{defn}

\begin{defn}
	Soit $E$ un ensemble et $(A_i)_{i \in I}$ une famille de parties de $E$. On suppose $I \neq \O$. On pose \[
		\bigcup_{i \in  I} A_i = \{x \in E  \mid \exists i \in I,\, x \in A_i\}
	\] et \[
		\bigcap_{i \in  I} A_i = \{x \in E  \mid \forall i \in I,\, x \in A_i\}.
	\] On pose aussi $\bigcup_{i \in \O} A_i = \O$ et $\bigcap_{i \in \O}  A_i = E$.
\end{defn}

\begin{rmk}
	De même que pour les sommes et produits de complexes, on peut intervertir des réunions doubles.
\end{rmk}

\begin{prop}
	Soit $E$ un ensemble, $(A,B) \in \mathcal{P}(E)^2$. \[
		A \subset (E \setminus B) \iff A \cap B = \O.
	\]
\end{prop}

\begin{figure}[H]
	\centering
	\begin{asy}
		import patterns;
		add("hatch",hatch(1mm, deepcyan));
		add("hatch2",hatch(1mm, heavygreen));
		size(3cm);

		guide main_set = scale(1.3) * ((-1,1)..(-0.8,-0.8)..(0,-0.9)..(0.7,-1.2)..(0.8, 0.9)..cycle);
		guide set_a = shift((-0.5, -0.2)) * ((-0.6, 0.6)..(0.2,-0.2)..(0.2,-0.4)..(-0.6,-0.2)..cycle);
		guide set_b = shift((0.3, 0.4)) * ((0.8, -0.6)..(1.1,-0.2)..(0.2,0.5)..(0.2,-0.8)..cycle);

		draw(main_set, magenta); label("$E$", 1.3*(0.8,0.9),magenta, align=NE);
		draw(set_a, deepcyan); label("$A$", (-0.6,0.6), deepcyan, align=NW);
		draw(set_b, heavygreen); label("$B$", (0.8,-0.6), heavygreen, align=SE);

		fill(set_a, pattern("hatch"));
		fill(set_b, pattern("hatch2"));
	\end{asy}
\end{figure}

\begin{prv}
	\begin{itemize}
		\item[``$\implies$''] Soit $x \in A \cap B$. Alors $x \in A$ et $x \in B$. Comme $x \in A \subset (E \setminus B)$, alors $x \in E \setminus B$ i.e. $x \not\in B$ : une contradiction. Donc $A \cap B = \O$.
		\item[``$\impliedby$''] On suppose $A \cap B = \O$. Soit $x \in A$. Si $x \in B$, alors $x \in A \cap B = \O$ : faux.
			Donc $x \not\in B$ et donc $x \in E \setminus B$.
	\end{itemize}
\end{prv}

\begin{prop}
	Si $f: E\to F$ et $g: F \to G$ sont bijectives, alors $g \circ f$ est bijective et \[
		(g \circ f)^{-1} = f^{-1} \circ g^{-1}.
	\] \qed
\end{prop}

\begin{rmk}[\danger Attention]
	$g \circ f$ peut-être bijective alors que $f$ et $g$ ne le sont pas.
\end{rmk}



	\chap[20]{Fractions rationnelles}
	\renewcommand{\cwd}{../chap20}
	\part{Topologie de $\R^2$}

\begin{defn}
	La \underline{norme (euclidienne)} de $\R^2$ est l'application définie par \[
		\forall (x,y) \in \R^2, \|(x,y)\| = \sqrt{x^2 + y^2}.
	\]

	\begin{figure}[H]
		\centering
		\begin{asy}
			import graph;
			axes(EndArrow);
			size(4cm);
			pair A = (3,2);
			dot(A);
			draw((3,0)--A, dashed);
			draw((0,2)--A, dashed);
			label("$x$", (3,0), align=S);
			label("$y$", (0,2), align=W);
			draw((0,0)--A);
			dot((4,3), white+0);
		\end{asy}
	\end{figure}
	\index{norme (de $\R^2$)}
	\index{norme euclidienne (de $\R^2$)}
\end{defn}

\begin{prop}
	La norme euclidienne vérifie:
	\begin{enumerate}
		\item (séparation) \[
			\forall (x,y) \in \R^2, \|(x,y)\| = 0 \iff x = y = 0,
			\]
		\item (homogénéité positive) \[
				\forall \lambda \in \R, \forall (x,y) \in \R^2, \|\lambda(x,y)\|= \left| \lambda \right| \|(x,y)\|
			\]
		\item (inégalité triangulaire) \[
			\forall (x,y), (a,b) \in \R^2,
			\|(x,y)+(a,b)\|\le \|(x,y)\|+\|(a,b)\|.
		\]
	\end{enumerate}
\end{prop}

\begin{prv}
	Déjà vue en replaçant $(x,y)$ par $x+iy \in \C$ et $\|(x,y)\|$ par |x+iy|
\end{prv}

\begin{defn}
	Soit $(a,b) \in \R^2$ et $r \in \R_+$.

	La \underline{boule ouverte} (ou \underline{disque ouvert}) de centre $(a,b)$ et de rayon $r$ est \[
		B_{(a,b)}(r) = \big\{ (x,y) \in \R^2  \mid \|(x,y) - (a,b)\| < r \big\}.
	\]

	La \underline{boule fermée} (ou \underline{disque fermé}) de centre $(a,b)$ et de rayon $r$ est \[
		\overline{B_{(a,b)}}(r) = \big\{ (x,y)\in \R^2  \mid \|(x,y) - (a,b)\| \le r \big\}.
	\]

	La \underline{sphère} (ou \underline{boule}) de centre $(a,b)$ et de rayon $r$ est \[
		S_{(a,b)}(r) = \partial \overline{B_{(a,b)}}(r) = \big\{ (x,y) \in \R^2  \mid \|(x,y) - (a,b)\| = r \big\}.
	\]
	\index{boule ouverte (de $\R^2$)}
	\index{disque ouverte (de $\R^2$)}
	\index{boule fermée (de $\R^2$)}
	\index{disque fermée (de $\R^2$)}
	\index{boule (de $\R^2$)}
	\index{sphère (de $\R^2$)}
\end{defn}

\begin{figure}[H]
		\centering
		\incfig{boule}
\end{figure}

\begin{rmk}
	On parle de boule en dimension quelconque.
\end{rmk}

\begin{defn}
	Une \underline{partie ouverte} $O$ de $\R^2$ (ou \underline{un ouvert}) si \[
		\forall (x,y) \in O, \exists r > 0, B_{(a,b)}(r) \subset O.
	\]
	Une partie $F$ est \underline{fermée} su $\R^2\setminus F$ est ouverte.
	\index{partie ouverte (de $\R^2$)}
	\index{ouvert (de $\R^2$)}
	\index{partie fermée (de $\R^2$)}
\end{defn}

\begin{figure}[H]
	\centering
	\incfig{partie-ouverte}
\end{figure}

\begin{prop}
	Une boule ouverte est ouverte. Une boule fermée est fermée.
\end{prop}

\begin{figure}[H]
	\centering
	\begin{subfigure}{4cm}
		\centering
		\begin{asy}
			import patterns;

			pair n(pair a) {return a / length(a);}

			add("hatch",hatch(2mm, SW, red));
			size(4cm);

			draw(circle((0,0), 1));
			dot('$(a_0, b_0)$', (0,0), align=S);

			draw((0,0) -- n((-1, 1)), dashed);
			label("$r$", n((-1, 1)) / 2, align=1.5S);

			pair A = n((1,3)) * (2/3);
			real rho = (1 - length(A)) * (2 / 3);

			dot("$(a,b)$", A, red, align=3SE);
			filldraw(circle(A, rho), pattern("hatch"), red);

			label("$O$", n((1,-1))*2.5/3);
		\end{asy}
	\end{subfigure}
	\begin{subfigure}{1cm}
		\centering~\\
	\end{subfigure}
	\begin{subfigure}{5cm}
		\centering
		\begin{asy}
			import patterns;

			pair n(pair a) {return a / length(a);}

			add("hatch",hatch(1mm, SW, red));
			add("hatch2",hatch(3mm, SE, blue));
			size(5cm);

			guide around = (-1.5, -1.5) -- (-1.5, 1.5) -- (2.5, 1.5) -- (2.5, -1.5) -- cycle;

			pair A = n((3, 1)) * 5/3; 
			real rho = 2 / 9;

			picture inter;
			fill(inter, around, pattern("hatch2"));
			fill(inter, circle((0,0), 1), white);
			add(inter);

			draw(circle((0,0), 1));
			dot('$(a_0, b_0)$', (0,0), align=S);

			draw((0,0) -- n((-1, 1)), dashed);
			label("$r$", n((-1, 1)) / 2, align=1.5S);

			dot("$(a,b)$", A, red, align=2SE);
			filldraw(circle(A, rho), pattern("hatch"), red);

			label("$F$", n((1,-1))*2.5/3);
		\end{asy}
	\end{subfigure}
\end{figure}

\begin{prv}
	$\O$ est un ouvert.

	Soit $B$ la boule ouverte de centre $(a_0, b_0) \in \R^2$ et de rayon $r > 0$.

	On pose $\rho = \frac{1}{2}\big(r - \|(a,b) - (a_0,b_0)\|\big)$.
	Montrons que \[
		B_{(a,b)}(\rho) \subset  B_{(a,b)}(r).
	\]

	Soit $(x,y) \in B_{(a,b)}(\rho)$.
	\begin{align*}
		\|(x,y) - (a_0,b_0)\|&= \|(x,y)- (a,b) + (a,b) - (a_0,b_0)\| \\
		&\le \|(x,y) - (a,b)\| + \|(a,b) - (a_0, b_0)\|\\
		&< \rho + \|(a,b) - (a_0, b_0)\| = \frac{1}{2}r + \frac{1}{2} \|(a,b) - (a_0, b_0)\|\\
		&< r
	\end{align*}
	
	Soit $F$ la boule fermée de centre $(a_0, b_0)$ et de rayon $r \ge 0$.

	Soit $(a,b) \not\in F$. On pose \[
		\rho = \frac{1}{2}\big(\|(a,b) - (a_0, b_0)\| - r\big) > 0.
	\]

	Montrons que $B_{(a,b)}(\rho) \subset \R^2\setminus F$.

	Soit $(x,y) \in B_{(a,b)}(\rho)$.

	\begin{align*}
		\|(x,y) - (a_0, b_0)\| &= \|(x,y) - (a,b) + (a,b) - (a_0, b_0)\| \\
		&\ge \big| \underbrace{\|(x,y) - (a,b)\|}_{\le \rho} - \underbrace{\|(a,b) - (a_0, b_0)\|}_{> r} \big|\\
		&\ge \|(a,b) - (a_0, b_0)\|- \|(x,y) - (a,b)\|\\
		&> \|(a,b) - (a_0, b_0)\|- \rho\\
		&> \frac{1}{2} \|(a,b) - (a_0, b_0)\| + \frac{1}{2}r\\
		&> r
	\end{align*}

	donc $(x,y) \not\in F$.
\end{prv}

\begin{exm}
	\begin{enumerate}
		\item $\O$ est ouvert.\\
			$\R^2$ est ouvert.
		\item $\O$ est fermé.\\
			$\R^2$ est fermé.\\
		\item $\big\{(x,0)  \mid x > 0\big\}$ n'est ni ouverte ni fermé.
	\end{enumerate}
\end{exm}

\begin{figure}[H]
	\centering
	\begin{asy}
		size(3cm);

		draw((0, -1) -- (0, 3), Arrow(TeXHead));
		draw((-1, 0) -- (3, 0), Arrow(TeXHead));
		
		draw((0,0) -- (0, 2.97), red);
		draw(circle((0,1.5), 0.5), deepred);
		draw(circle((0,0.5), 0.1), deepred);
	\end{asy}
\end{figure}

\begin{defn}
	Soit $(a,b) \in \R^2$ et $V \in \mathcal{P}(\R^2)$.

	On dit que $V$ est un voisinage de $(a,b)$ s'il existe $r > 0$ tel que \[
		B_{(a,b)}(r) \subset V.
	\]
	\index{voisinage (dans $\R^2$)}
\end{defn}

\begin{prop}
	Un ouvert non vide est un voisinage en chacun de ces points. \qed
\end{prop}

\begin{defn}
	Soit $D \subset \R^2$. Un \underline{point intérieur} de $D$ est un couple $(a,b) \in D$ tel que \[
		\exists r > 0, B_{(a,b)}(r) \subset D.
	\] en d'autres termes, si $D$ est un voisinage de $(a,b)$.

	On note $\mathring D$ l'ensemble des points intérieurs à $D$. C'est \underline{l'intérieur} de $D$.
	\index{point intérieur (dans $\R^2$)}
	\index{intérieur (dans $\R^2$)}
\end{defn}

\begin{prop}
	$\mathring D$ est le plus grand ouvert $O$ de $\R^2$ tel que $O \subset D$.
\end{prop}

\begin{figure}[H]
	\centering
	\incfig{interieur-plus-grand-ouvert}
\end{figure}


\begin{prv}
	Soit $(a,b) \in \mathring D$.

	Par définition, il existe $r > 0$ tel que \[
		B_{(a,b)}(r) \subset D.
	\] Montrons que $B_{(a,b)}(r) \subset \mathring D$.

	Soit $(x,y) \in B_{(a,b)}(r)$. Comme $B_{(a,b)}(r)$ est un ouvert de $\R^2$, il existe $\rho > 0$ tel que \[
		B_{(x,y)}(\rho) \subset B_{(a,b)}(r)
	\] donc $(x,y) \in \mathring D$.

	Donc $\mathring D$ est ouvert, $\mathring D \subset D$.

	Soit $O$ un ouvert de $\R^2$ tel que $O \subset D$. Montrons que $O \subset \mathring D$.

	Soit $(x,y) \in O$. Soit $r > 0$ tel que \[
		B_{(x,y)}(r) \subset O \subset D
	\] donc $(x,y) \in \mathring D$.
\end{prv}

\begin{defn}
	Soit $f: D \subset \R^2 \to \R$, $\ell \in \R$, $(a,b) \in \mathring D$.

	On dit que \underline{$f(x,y)$ tend vers $\ell$ quand $(x,y)$ tend vers $(a,b)$} ou que $\ell$ est \underline{une limite} de $f$ en $(a,b)$ si \[
		\forall \varepsilon > 0, \exists r > 0, \forall (x,y) \in D, \|(x,y) - (a,b)\| < r \implies \left| f(x,y) - \ell \right| \le \varepsilon.
	\] en d'autres termes si \[
		\forall V \in \mathcal{V}_{\ell}, \exists W \in \mathcal{V}_{(a,b)}, \forall (x,y) \in W \cap D, f(x,y) \in V.
	\]
	\index{limite (dans $\R^2$)}
	\index{tendre vers (dans $\R^2$)}
\end{defn}

\begin{prop}
	[unicité de la limite]
	Soit $f: D \to \R$, $(a,b) \in \mathring D$, $\ell_1, \ell_2 \in \R$ telles que $\ell_1$ et $\ell_2$ sont des limites de $f$ en $(a,b)$.

	Alors $\ell_1 = \ell_2$.
\end{prop}

\begin{figure}[H]
	\centering
	\incfig{preuve-unicité-de-la-limite}
\end{figure}

\begin{prv}
	On suppose $\ell_1 < \ell_2$. On pose $\varepsilon = \frac{\ell_2 - \ell_1}{2} > 0$.

	Soit $r_1 > 0$ tel que \[
		f\big(B_{(a,b)}(r_1)\big) \subset ]\ell_1 - \varepsilon, \ell_1 + \varepsilon[.
	\] Soit $r_2 > 0$ tel que \[
		f\big(B_{(a,b)}(r_2)\big) \subset ]\ell_2 - \varepsilon, \ell_2 + \varepsilon [.
	\] On pose $r = \min(r_1, r_2)$ donc \[
		B_{(a,b)}(r_1) \cap B_{(a,b)}(r_2) = B_{(a,b)}(r) \neq \O.
	\] Soit $(x,y) \in B_{(a,b)}(r)$. Alors, \[
		f(x,y) \in ]\ell_1 - \varepsilon, \ell_1 + \varepsilon[ \cap ]\ell_2 - \varepsilon, \ell_2 + \varepsilon[ = \O.
	\] $\lightning$
\end{prv}

\begin{defn}
	Soit $f : D \to \R$, $(a,b) \in \mathring D$.

	On dit que $f$ est \underline{continue} en $(a,b)$ si \[
		f(x,y) \tendsto{(x,y) \to (a,b)}f(a,b).
	\]
	\index{continuité (dans $\R^2$)}
\end{defn}

\begin{prop}
	\underline{Si} $f(x,y) \tendsto{(x,y) \to (a,b)} \ell$ \\
	\underline{alors} $\begin{cases}
		f(x,b) \tendsto{x \to a} \ell\\
		f(a,y) \tendsto{y \to b} \ell.\\
	\end{cases}$
\end{prop}

\begin{prv}~\\
	\begin{figure}[H]
		\centering
		\incfig{limite-x-en-a-et-y-en-b}
	\end{figure}
\end{prv}

\underline{Contre-exemple} : exercice 3.

\begin{exm}
	\begin{enumerate}
		\item $f : \begin{array}{rcl}
				\R^2 &\longrightarrow& \R \\
				(x,y) &\longmapsto& x
			\end{array}$ limite en $(0,0)$ ?

			Soit $\varepsilon > 0$. On pose $r = \varepsilon$. \[
				\forall (x,y) \in B_{(0,0)}(r),
				\left| f(x,y) \right| = \left| x \right| \le \|(x,y)\| < r = \varepsilon
			\] Donc $f(x,y) \tendsto{(x,y) \to (a,b)} 0$.
		\item limite $f : \begin{array}{rcl}
				\R^2 &\longrightarrow& \R \\
				(x,y) &\longmapsto& x^3
			\end{array}$ en $(0,0)$ ?

			Soit $\varepsilon > 0$. On pose $r = \sqrt[3]{r} > 0$. \[
				\forall (x,y) \in B_{(0,0)}(r),
				\left| f(x,y) \right| = \left| x^3 \right| \le \|(x,y)\|^3 < r^3 = \varepsilon.
			\]
		\item limite de $f : \begin{array}{rcl}
			\R^2 &\longrightarrow& \R \\
			(x,y) &\longmapsto& x^3y^2
		\end{array}$ en $(0,0)$ ?

		Soit $\varepsilon > 0$. On pose $r = \sqrt[5]{\varepsilon} > 0$. \[
			\forall (x,y) \in B_{(0,0)}(r), \left| f(x,y) \right| = \left| x^3 y^2 \right| \le \|(x,y)\|^3 \|(x,y)\|^2 < r^5 = \varepsilon.
		\]
	\end{enumerate}
\end{exm}

\begin{defn}
	Soient $D \subset \R^2$ et $(x,y) \in \R^2$.

	\begin{figure}[H]
    \centering
    \incfig{point-adhérent}
	\end{figure}
	
	On dit que $(x,y)$ est \underline{adhérent} à $D$ si \[
		\forall r > 0, B_{(x,y)}(r) \cap D \neq \O.
	\] L'ensemble des points adhérents à $D$ est noté $\overline{D}$. On dit que $\overline{D}$ est \underline{l'adhérence} de $D$.
	\index{point adhérent (dans $\R^2$)}
	\index{adhérent (dans $\R^2$)}
\end{defn}

\begin{defn}
	Soit $f: D \subset \R^2 \to \R$ et $(a,b) \in \overline{D}$, $\ell \in \R$. On dit que $f$ tend vers $\ell$ quand $(x,y)$ tend vers $(a,b)$ si \[
		\forall \varepsilon > 0, \exists r > 0, \forall (x,y) \in B_{(a,b)}(r) \cap D,
		\left| f(x,y) - \ell \right| \le \varepsilon.
	\]
	\index{limite (dans $\R^2$)}
	\index{tendre vers (dans $\R^2$)}
\end{defn}

\begin{prop}
	\begin{enumerate}
		\item Dans ce contexte, il y a unicité de la limite
		\item La limite d'une somme, d'un produit, d'un quotien, d'une composée se comporte comme dans le cas d'une seule variable.
		\item Soit $f: D \to \R$ continue. Soient $g: I \to \R$ et $h: I \to \R$ continues telles que \[
			\forall t \in I, \big(g(t), h(t)\big) \in D.
		\] Alors \[
			t \in I \mapsto f\big(g(t), h(t)\big) \in \R
		\] est continue.
	\end{enumerate}
\end{prop}

\begin{figure}[H]
	\centering
	\begin{asy}
		import three;
		import graph3;
		size(5cm);

		settings.render = 0;
		settings.prc = false;
		currentprojection = obliqueX;

		draw(O -- X, Arrow3(TeXHead2));
		draw(O -- Y, Arrow3(TeXHead2));
		draw(O -- Z, Arrow3(TeXHead2));

		triple f(real x, real y, real z = 0) { return (x,y,cos(x - 0.5) * cos(y - 0.5)/1.2 + 0.15); }

		real inc = 1 / 5;

		for(real x = 0; x <= 1; x += inc) {
			draw(graph(
				new real(real t) { return x; }, // x
				new real(real y) { return y; }, // y
				new real(real y) { return f(x,y).z; }, // z
				0, 1
			), gray);
		}

		for(real y = 0; y <= 1; y += inc) {
			draw(graph(
				new real(real x) { return x; }, // x
				new real(real t) { return y; }, // y
				new real(real x) { return f(x,y).z; }, // z
				0, 1
			), gray);
		}

		path3 path1 = (0.3, 0.2, 0) .. (0.5, 0.5, 0) .. (0.6, 0.7, 0) .. (0.9, 0.8, 0);
		path3 path2 = (0.3, 0.8, 0) .. (0.5, 0.5, 0) .. (0.6, 0.3, 0) .. (0.9, 0.2, 0);
		path3 pathA = f(0.3, 0.2, 0) .. f(0.5, 0.5, 0) .. f(0.6, 0.7, 0) .. f(0.9, 0.8, 0);
		path3 pathB = f(0.3, 0.8, 0) .. f(0.5, 0.5, 0) .. f(0.6, 0.3, 0) .. f(0.9, 0.2, 0);

		draw(path1, red, Arrow3(TeXHead2, position=0.5));
		draw(pathA, red, Arrow3(TeXHead2, position=0.5));
		draw(path2, deepcyan, Arrow3(TeXHead2, position=0.3));
		draw(pathB, deepcyan, Arrow3(TeXHead2, position=0.3));

		dot((0.5, 0.5, 0));
		dot(f(0.5, 0.5, 0));
		draw((0.5, 0.5, 0) -- f(0.5, 0.5, 0), dashed);
	\end{asy}
\end{figure}


	\part{Transpositions}

\begin{defn}
	Une \underline{transposition} est un cycle de longueur 2 : $\begin{pmatrix}
		a&b
	\end{pmatrix}$ avec $a \neq b$.
	\index{transposition (permutation)}
\end{defn}

\begin{exm}
	Avec $n = 5$ et $\gamma = \begin{pmatrix}
		2&4&1
	\end{pmatrix}$.

	\begin{figure}[H]
		\centering

		\begin{asy}
			size(5cm);

			real rho = 0.15; // circles

			void draw_cycle(pair O, real r ...int[] nums) {
				int n = nums.length;
				real eps = (15 / r) * 0.8;

				for(int i = 0; i < n; ++i) {
					real theta_1 = (360/n) * (i+1);
					real theta_2 = (360/n) * i;

					pair C = O + dir(theta_2) * r;

					draw(circle(C, rho));
					label("$" + string(nums[i]) + "$", C);
					draw(arc(O, r, theta_2+eps, theta_1-eps), Arrow(TeXHead));
				}
			}

			draw_cycle((-1,0), 0.8, 1, 2, 4);
			draw_cycle((1,0), 0.3, 3);
			draw_cycle((2,0), 0.3, 5);
		\end{asy}
	\end{figure}

	\[
		\gamma = \begin{pmatrix}
			1&4
		\end{pmatrix} \begin{pmatrix}
			1&2
		\end{pmatrix}
	\]

	Avec $n = 6$ et $\gamma = \begin{pmatrix}
		1&3&5&6&2
	\end{pmatrix} = \begin{pmatrix}
		1&2&3&4&5&6\\
		3&1&5&4&6&2
	\end{pmatrix}$.

	Donc, \[
		\gamma = \begin{pmatrix}
			1&2
		\end{pmatrix} \begin{pmatrix}
			1&6
		\end{pmatrix} \begin{pmatrix}
			1&5
		\end{pmatrix} \begin{pmatrix}
			1&3
		\end{pmatrix}
	\] 
	\[
		\begin{pmatrix}
			1&2&3&4&5&6\\
			3&2&1&4&5&6\\
			3&2&5&4&1&6\\
			3&2&5&4&6&1\\
			3&1&5&4&6&2\\
		\end{pmatrix}
	\]

	Et, \[
		\gamma = \begin{pmatrix}
			1&3
		\end{pmatrix} \begin{pmatrix}
			2&3
		\end{pmatrix} \begin{pmatrix}
			3&5
		\end{pmatrix} \begin{pmatrix}
			5&6
		\end{pmatrix} 
	\]

	\[
		\begin{pmatrix}
			1&2&3&4&5&6\\
			1&2&3&4&6&5\\
			1&2&5&4&6&3\\
			1&3&5&4&6&2\\
			3&1&5&4&6&2\\
		\end{pmatrix} 
	\] 
\end{exm}

\begin{exm}
	\[
		\begin{pmatrix}
			1&4
		\end{pmatrix} = \begin{pmatrix}
			1&2
		\end{pmatrix} \begin{pmatrix}
			2&3
		\end{pmatrix} \begin{pmatrix}
			3&4
		\end{pmatrix} \begin{pmatrix}
			2&3
		\end{pmatrix} \begin{pmatrix}
			1&2
		\end{pmatrix}
	\]
	On n'a pas toujours le même nombre de transpositions mais la parité du nombre reste la même (proposition plus loin).
\end{exm}

\begin{thm}
	Toute permutation se décompose en produit de transpositions.
\end{thm}

\begin{prv}
	Soit $\gamma = \begin{pmatrix}
		a_1&\cdots&a_k
	\end{pmatrix}$ un $k$-cycle.

	On remarque que
	\[
		\gamma = \begin{pmatrix}
			a_1&a_k
		\end{pmatrix} \cdots \begin{pmatrix}
			a_1&a_4
		\end{pmatrix} \begin{pmatrix}
			a_1&a_3
		\end{pmatrix} \begin{pmatrix}
			a_1&a_2
		\end{pmatrix}
	\] C'est un produit de transpositions.
\end{prv}

\begin{exm}
	Avec $n = 10$ et $\sigma = \begin{pmatrix}
		1&2&3&4&5&6&7&8&9&10\\
		9&8&1&7&2&3&4&5&10&6
	\end{pmatrix}$.

	On a
	\begin{align*}
		\sigma &= \begin{pmatrix}
			1&9&10&6&3
		\end{pmatrix} \begin{pmatrix}
			2&8&5
		\end{pmatrix} \begin{pmatrix}
			4&7
		\end{pmatrix}\\
		&= \begin{pmatrix}
			1&3
		\end{pmatrix} \begin{pmatrix}
			1&6
		\end{pmatrix} \begin{pmatrix}
			1&10
		\end{pmatrix} \begin{pmatrix}
			1&9
		\end{pmatrix} \begin{pmatrix}
			2&5
		\end{pmatrix} \begin{pmatrix}
			2&8
		\end{pmatrix} \begin{pmatrix}
			4&7
		\end{pmatrix} \\
	\end{align*}

	Vérification : \[
		\begin{pmatrix}
			1&2&3&4&5&6&7&8&9&10\\
			1&2&3&7&5&6&4&8&9&10\\
			1&8&3&7&5&6&4&2&9&10\\
			1&8&3&7&2&6&4&5&9&10\\
			9&8&3&7&2&6&4&5&1&10\\
			9&8&3&7&2&6&4&5&10&1\\
			9&8&3&7&2&1&4&5&10&6\\
			9&8&1&7&2&3&4&5&10&6\\
		\end{pmatrix} 
	\] 
\end{exm}


	\chap[21]{Matrices et applications linéaires}
	\renewcommand{\cwd}{../chap21}
	\part{Topologie de $\R^2$}

\begin{defn}
	La \underline{norme (euclidienne)} de $\R^2$ est l'application définie par \[
		\forall (x,y) \in \R^2, \|(x,y)\| = \sqrt{x^2 + y^2}.
	\]

	\begin{figure}[H]
		\centering
		\begin{asy}
			import graph;
			axes(EndArrow);
			size(4cm);
			pair A = (3,2);
			dot(A);
			draw((3,0)--A, dashed);
			draw((0,2)--A, dashed);
			label("$x$", (3,0), align=S);
			label("$y$", (0,2), align=W);
			draw((0,0)--A);
			dot((4,3), white+0);
		\end{asy}
	\end{figure}
	\index{norme (de $\R^2$)}
	\index{norme euclidienne (de $\R^2$)}
\end{defn}

\begin{prop}
	La norme euclidienne vérifie:
	\begin{enumerate}
		\item (séparation) \[
			\forall (x,y) \in \R^2, \|(x,y)\| = 0 \iff x = y = 0,
			\]
		\item (homogénéité positive) \[
				\forall \lambda \in \R, \forall (x,y) \in \R^2, \|\lambda(x,y)\|= \left| \lambda \right| \|(x,y)\|
			\]
		\item (inégalité triangulaire) \[
			\forall (x,y), (a,b) \in \R^2,
			\|(x,y)+(a,b)\|\le \|(x,y)\|+\|(a,b)\|.
		\]
	\end{enumerate}
\end{prop}

\begin{prv}
	Déjà vue en replaçant $(x,y)$ par $x+iy \in \C$ et $\|(x,y)\|$ par |x+iy|
\end{prv}

\begin{defn}
	Soit $(a,b) \in \R^2$ et $r \in \R_+$.

	La \underline{boule ouverte} (ou \underline{disque ouvert}) de centre $(a,b)$ et de rayon $r$ est \[
		B_{(a,b)}(r) = \big\{ (x,y) \in \R^2  \mid \|(x,y) - (a,b)\| < r \big\}.
	\]

	La \underline{boule fermée} (ou \underline{disque fermé}) de centre $(a,b)$ et de rayon $r$ est \[
		\overline{B_{(a,b)}}(r) = \big\{ (x,y)\in \R^2  \mid \|(x,y) - (a,b)\| \le r \big\}.
	\]

	La \underline{sphère} (ou \underline{boule}) de centre $(a,b)$ et de rayon $r$ est \[
		S_{(a,b)}(r) = \partial \overline{B_{(a,b)}}(r) = \big\{ (x,y) \in \R^2  \mid \|(x,y) - (a,b)\| = r \big\}.
	\]
	\index{boule ouverte (de $\R^2$)}
	\index{disque ouverte (de $\R^2$)}
	\index{boule fermée (de $\R^2$)}
	\index{disque fermée (de $\R^2$)}
	\index{boule (de $\R^2$)}
	\index{sphère (de $\R^2$)}
\end{defn}

\begin{figure}[H]
		\centering
		\incfig{boule}
\end{figure}

\begin{rmk}
	On parle de boule en dimension quelconque.
\end{rmk}

\begin{defn}
	Une \underline{partie ouverte} $O$ de $\R^2$ (ou \underline{un ouvert}) si \[
		\forall (x,y) \in O, \exists r > 0, B_{(a,b)}(r) \subset O.
	\]
	Une partie $F$ est \underline{fermée} su $\R^2\setminus F$ est ouverte.
	\index{partie ouverte (de $\R^2$)}
	\index{ouvert (de $\R^2$)}
	\index{partie fermée (de $\R^2$)}
\end{defn}

\begin{figure}[H]
	\centering
	\incfig{partie-ouverte}
\end{figure}

\begin{prop}
	Une boule ouverte est ouverte. Une boule fermée est fermée.
\end{prop}

\begin{figure}[H]
	\centering
	\begin{subfigure}{4cm}
		\centering
		\begin{asy}
			import patterns;

			pair n(pair a) {return a / length(a);}

			add("hatch",hatch(2mm, SW, red));
			size(4cm);

			draw(circle((0,0), 1));
			dot('$(a_0, b_0)$', (0,0), align=S);

			draw((0,0) -- n((-1, 1)), dashed);
			label("$r$", n((-1, 1)) / 2, align=1.5S);

			pair A = n((1,3)) * (2/3);
			real rho = (1 - length(A)) * (2 / 3);

			dot("$(a,b)$", A, red, align=3SE);
			filldraw(circle(A, rho), pattern("hatch"), red);

			label("$O$", n((1,-1))*2.5/3);
		\end{asy}
	\end{subfigure}
	\begin{subfigure}{1cm}
		\centering~\\
	\end{subfigure}
	\begin{subfigure}{5cm}
		\centering
		\begin{asy}
			import patterns;

			pair n(pair a) {return a / length(a);}

			add("hatch",hatch(1mm, SW, red));
			add("hatch2",hatch(3mm, SE, blue));
			size(5cm);

			guide around = (-1.5, -1.5) -- (-1.5, 1.5) -- (2.5, 1.5) -- (2.5, -1.5) -- cycle;

			pair A = n((3, 1)) * 5/3; 
			real rho = 2 / 9;

			picture inter;
			fill(inter, around, pattern("hatch2"));
			fill(inter, circle((0,0), 1), white);
			add(inter);

			draw(circle((0,0), 1));
			dot('$(a_0, b_0)$', (0,0), align=S);

			draw((0,0) -- n((-1, 1)), dashed);
			label("$r$", n((-1, 1)) / 2, align=1.5S);

			dot("$(a,b)$", A, red, align=2SE);
			filldraw(circle(A, rho), pattern("hatch"), red);

			label("$F$", n((1,-1))*2.5/3);
		\end{asy}
	\end{subfigure}
\end{figure}

\begin{prv}
	$\O$ est un ouvert.

	Soit $B$ la boule ouverte de centre $(a_0, b_0) \in \R^2$ et de rayon $r > 0$.

	On pose $\rho = \frac{1}{2}\big(r - \|(a,b) - (a_0,b_0)\|\big)$.
	Montrons que \[
		B_{(a,b)}(\rho) \subset  B_{(a,b)}(r).
	\]

	Soit $(x,y) \in B_{(a,b)}(\rho)$.
	\begin{align*}
		\|(x,y) - (a_0,b_0)\|&= \|(x,y)- (a,b) + (a,b) - (a_0,b_0)\| \\
		&\le \|(x,y) - (a,b)\| + \|(a,b) - (a_0, b_0)\|\\
		&< \rho + \|(a,b) - (a_0, b_0)\| = \frac{1}{2}r + \frac{1}{2} \|(a,b) - (a_0, b_0)\|\\
		&< r
	\end{align*}
	
	Soit $F$ la boule fermée de centre $(a_0, b_0)$ et de rayon $r \ge 0$.

	Soit $(a,b) \not\in F$. On pose \[
		\rho = \frac{1}{2}\big(\|(a,b) - (a_0, b_0)\| - r\big) > 0.
	\]

	Montrons que $B_{(a,b)}(\rho) \subset \R^2\setminus F$.

	Soit $(x,y) \in B_{(a,b)}(\rho)$.

	\begin{align*}
		\|(x,y) - (a_0, b_0)\| &= \|(x,y) - (a,b) + (a,b) - (a_0, b_0)\| \\
		&\ge \big| \underbrace{\|(x,y) - (a,b)\|}_{\le \rho} - \underbrace{\|(a,b) - (a_0, b_0)\|}_{> r} \big|\\
		&\ge \|(a,b) - (a_0, b_0)\|- \|(x,y) - (a,b)\|\\
		&> \|(a,b) - (a_0, b_0)\|- \rho\\
		&> \frac{1}{2} \|(a,b) - (a_0, b_0)\| + \frac{1}{2}r\\
		&> r
	\end{align*}

	donc $(x,y) \not\in F$.
\end{prv}

\begin{exm}
	\begin{enumerate}
		\item $\O$ est ouvert.\\
			$\R^2$ est ouvert.
		\item $\O$ est fermé.\\
			$\R^2$ est fermé.\\
		\item $\big\{(x,0)  \mid x > 0\big\}$ n'est ni ouverte ni fermé.
	\end{enumerate}
\end{exm}

\begin{figure}[H]
	\centering
	\begin{asy}
		size(3cm);

		draw((0, -1) -- (0, 3), Arrow(TeXHead));
		draw((-1, 0) -- (3, 0), Arrow(TeXHead));
		
		draw((0,0) -- (0, 2.97), red);
		draw(circle((0,1.5), 0.5), deepred);
		draw(circle((0,0.5), 0.1), deepred);
	\end{asy}
\end{figure}

\begin{defn}
	Soit $(a,b) \in \R^2$ et $V \in \mathcal{P}(\R^2)$.

	On dit que $V$ est un voisinage de $(a,b)$ s'il existe $r > 0$ tel que \[
		B_{(a,b)}(r) \subset V.
	\]
	\index{voisinage (dans $\R^2$)}
\end{defn}

\begin{prop}
	Un ouvert non vide est un voisinage en chacun de ces points. \qed
\end{prop}

\begin{defn}
	Soit $D \subset \R^2$. Un \underline{point intérieur} de $D$ est un couple $(a,b) \in D$ tel que \[
		\exists r > 0, B_{(a,b)}(r) \subset D.
	\] en d'autres termes, si $D$ est un voisinage de $(a,b)$.

	On note $\mathring D$ l'ensemble des points intérieurs à $D$. C'est \underline{l'intérieur} de $D$.
	\index{point intérieur (dans $\R^2$)}
	\index{intérieur (dans $\R^2$)}
\end{defn}

\begin{prop}
	$\mathring D$ est le plus grand ouvert $O$ de $\R^2$ tel que $O \subset D$.
\end{prop}

\begin{figure}[H]
	\centering
	\incfig{interieur-plus-grand-ouvert}
\end{figure}


\begin{prv}
	Soit $(a,b) \in \mathring D$.

	Par définition, il existe $r > 0$ tel que \[
		B_{(a,b)}(r) \subset D.
	\] Montrons que $B_{(a,b)}(r) \subset \mathring D$.

	Soit $(x,y) \in B_{(a,b)}(r)$. Comme $B_{(a,b)}(r)$ est un ouvert de $\R^2$, il existe $\rho > 0$ tel que \[
		B_{(x,y)}(\rho) \subset B_{(a,b)}(r)
	\] donc $(x,y) \in \mathring D$.

	Donc $\mathring D$ est ouvert, $\mathring D \subset D$.

	Soit $O$ un ouvert de $\R^2$ tel que $O \subset D$. Montrons que $O \subset \mathring D$.

	Soit $(x,y) \in O$. Soit $r > 0$ tel que \[
		B_{(x,y)}(r) \subset O \subset D
	\] donc $(x,y) \in \mathring D$.
\end{prv}

\begin{defn}
	Soit $f: D \subset \R^2 \to \R$, $\ell \in \R$, $(a,b) \in \mathring D$.

	On dit que \underline{$f(x,y)$ tend vers $\ell$ quand $(x,y)$ tend vers $(a,b)$} ou que $\ell$ est \underline{une limite} de $f$ en $(a,b)$ si \[
		\forall \varepsilon > 0, \exists r > 0, \forall (x,y) \in D, \|(x,y) - (a,b)\| < r \implies \left| f(x,y) - \ell \right| \le \varepsilon.
	\] en d'autres termes si \[
		\forall V \in \mathcal{V}_{\ell}, \exists W \in \mathcal{V}_{(a,b)}, \forall (x,y) \in W \cap D, f(x,y) \in V.
	\]
	\index{limite (dans $\R^2$)}
	\index{tendre vers (dans $\R^2$)}
\end{defn}

\begin{prop}
	[unicité de la limite]
	Soit $f: D \to \R$, $(a,b) \in \mathring D$, $\ell_1, \ell_2 \in \R$ telles que $\ell_1$ et $\ell_2$ sont des limites de $f$ en $(a,b)$.

	Alors $\ell_1 = \ell_2$.
\end{prop}

\begin{figure}[H]
	\centering
	\incfig{preuve-unicité-de-la-limite}
\end{figure}

\begin{prv}
	On suppose $\ell_1 < \ell_2$. On pose $\varepsilon = \frac{\ell_2 - \ell_1}{2} > 0$.

	Soit $r_1 > 0$ tel que \[
		f\big(B_{(a,b)}(r_1)\big) \subset ]\ell_1 - \varepsilon, \ell_1 + \varepsilon[.
	\] Soit $r_2 > 0$ tel que \[
		f\big(B_{(a,b)}(r_2)\big) \subset ]\ell_2 - \varepsilon, \ell_2 + \varepsilon [.
	\] On pose $r = \min(r_1, r_2)$ donc \[
		B_{(a,b)}(r_1) \cap B_{(a,b)}(r_2) = B_{(a,b)}(r) \neq \O.
	\] Soit $(x,y) \in B_{(a,b)}(r)$. Alors, \[
		f(x,y) \in ]\ell_1 - \varepsilon, \ell_1 + \varepsilon[ \cap ]\ell_2 - \varepsilon, \ell_2 + \varepsilon[ = \O.
	\] $\lightning$
\end{prv}

\begin{defn}
	Soit $f : D \to \R$, $(a,b) \in \mathring D$.

	On dit que $f$ est \underline{continue} en $(a,b)$ si \[
		f(x,y) \tendsto{(x,y) \to (a,b)}f(a,b).
	\]
	\index{continuité (dans $\R^2$)}
\end{defn}

\begin{prop}
	\underline{Si} $f(x,y) \tendsto{(x,y) \to (a,b)} \ell$ \\
	\underline{alors} $\begin{cases}
		f(x,b) \tendsto{x \to a} \ell\\
		f(a,y) \tendsto{y \to b} \ell.\\
	\end{cases}$
\end{prop}

\begin{prv}~\\
	\begin{figure}[H]
		\centering
		\incfig{limite-x-en-a-et-y-en-b}
	\end{figure}
\end{prv}

\underline{Contre-exemple} : exercice 3.

\begin{exm}
	\begin{enumerate}
		\item $f : \begin{array}{rcl}
				\R^2 &\longrightarrow& \R \\
				(x,y) &\longmapsto& x
			\end{array}$ limite en $(0,0)$ ?

			Soit $\varepsilon > 0$. On pose $r = \varepsilon$. \[
				\forall (x,y) \in B_{(0,0)}(r),
				\left| f(x,y) \right| = \left| x \right| \le \|(x,y)\| < r = \varepsilon
			\] Donc $f(x,y) \tendsto{(x,y) \to (a,b)} 0$.
		\item limite $f : \begin{array}{rcl}
				\R^2 &\longrightarrow& \R \\
				(x,y) &\longmapsto& x^3
			\end{array}$ en $(0,0)$ ?

			Soit $\varepsilon > 0$. On pose $r = \sqrt[3]{r} > 0$. \[
				\forall (x,y) \in B_{(0,0)}(r),
				\left| f(x,y) \right| = \left| x^3 \right| \le \|(x,y)\|^3 < r^3 = \varepsilon.
			\]
		\item limite de $f : \begin{array}{rcl}
			\R^2 &\longrightarrow& \R \\
			(x,y) &\longmapsto& x^3y^2
		\end{array}$ en $(0,0)$ ?

		Soit $\varepsilon > 0$. On pose $r = \sqrt[5]{\varepsilon} > 0$. \[
			\forall (x,y) \in B_{(0,0)}(r), \left| f(x,y) \right| = \left| x^3 y^2 \right| \le \|(x,y)\|^3 \|(x,y)\|^2 < r^5 = \varepsilon.
		\]
	\end{enumerate}
\end{exm}

\begin{defn}
	Soient $D \subset \R^2$ et $(x,y) \in \R^2$.

	\begin{figure}[H]
    \centering
    \incfig{point-adhérent}
	\end{figure}
	
	On dit que $(x,y)$ est \underline{adhérent} à $D$ si \[
		\forall r > 0, B_{(x,y)}(r) \cap D \neq \O.
	\] L'ensemble des points adhérents à $D$ est noté $\overline{D}$. On dit que $\overline{D}$ est \underline{l'adhérence} de $D$.
	\index{point adhérent (dans $\R^2$)}
	\index{adhérent (dans $\R^2$)}
\end{defn}

\begin{defn}
	Soit $f: D \subset \R^2 \to \R$ et $(a,b) \in \overline{D}$, $\ell \in \R$. On dit que $f$ tend vers $\ell$ quand $(x,y)$ tend vers $(a,b)$ si \[
		\forall \varepsilon > 0, \exists r > 0, \forall (x,y) \in B_{(a,b)}(r) \cap D,
		\left| f(x,y) - \ell \right| \le \varepsilon.
	\]
	\index{limite (dans $\R^2$)}
	\index{tendre vers (dans $\R^2$)}
\end{defn}

\begin{prop}
	\begin{enumerate}
		\item Dans ce contexte, il y a unicité de la limite
		\item La limite d'une somme, d'un produit, d'un quotien, d'une composée se comporte comme dans le cas d'une seule variable.
		\item Soit $f: D \to \R$ continue. Soient $g: I \to \R$ et $h: I \to \R$ continues telles que \[
			\forall t \in I, \big(g(t), h(t)\big) \in D.
		\] Alors \[
			t \in I \mapsto f\big(g(t), h(t)\big) \in \R
		\] est continue.
	\end{enumerate}
\end{prop}

\begin{figure}[H]
	\centering
	\begin{asy}
		import three;
		import graph3;
		size(5cm);

		settings.render = 0;
		settings.prc = false;
		currentprojection = obliqueX;

		draw(O -- X, Arrow3(TeXHead2));
		draw(O -- Y, Arrow3(TeXHead2));
		draw(O -- Z, Arrow3(TeXHead2));

		triple f(real x, real y, real z = 0) { return (x,y,cos(x - 0.5) * cos(y - 0.5)/1.2 + 0.15); }

		real inc = 1 / 5;

		for(real x = 0; x <= 1; x += inc) {
			draw(graph(
				new real(real t) { return x; }, // x
				new real(real y) { return y; }, // y
				new real(real y) { return f(x,y).z; }, // z
				0, 1
			), gray);
		}

		for(real y = 0; y <= 1; y += inc) {
			draw(graph(
				new real(real x) { return x; }, // x
				new real(real t) { return y; }, // y
				new real(real x) { return f(x,y).z; }, // z
				0, 1
			), gray);
		}

		path3 path1 = (0.3, 0.2, 0) .. (0.5, 0.5, 0) .. (0.6, 0.7, 0) .. (0.9, 0.8, 0);
		path3 path2 = (0.3, 0.8, 0) .. (0.5, 0.5, 0) .. (0.6, 0.3, 0) .. (0.9, 0.2, 0);
		path3 pathA = f(0.3, 0.2, 0) .. f(0.5, 0.5, 0) .. f(0.6, 0.7, 0) .. f(0.9, 0.8, 0);
		path3 pathB = f(0.3, 0.8, 0) .. f(0.5, 0.5, 0) .. f(0.6, 0.3, 0) .. f(0.9, 0.2, 0);

		draw(path1, red, Arrow3(TeXHead2, position=0.5));
		draw(pathA, red, Arrow3(TeXHead2, position=0.5));
		draw(path2, deepcyan, Arrow3(TeXHead2, position=0.3));
		draw(pathB, deepcyan, Arrow3(TeXHead2, position=0.3));

		dot((0.5, 0.5, 0));
		dot(f(0.5, 0.5, 0));
		draw((0.5, 0.5, 0) -- f(0.5, 0.5, 0), dashed);
	\end{asy}
\end{figure}


	\part{Transpositions}

\begin{defn}
	Une \underline{transposition} est un cycle de longueur 2 : $\begin{pmatrix}
		a&b
	\end{pmatrix}$ avec $a \neq b$.
	\index{transposition (permutation)}
\end{defn}

\begin{exm}
	Avec $n = 5$ et $\gamma = \begin{pmatrix}
		2&4&1
	\end{pmatrix}$.

	\begin{figure}[H]
		\centering

		\begin{asy}
			size(5cm);

			real rho = 0.15; // circles

			void draw_cycle(pair O, real r ...int[] nums) {
				int n = nums.length;
				real eps = (15 / r) * 0.8;

				for(int i = 0; i < n; ++i) {
					real theta_1 = (360/n) * (i+1);
					real theta_2 = (360/n) * i;

					pair C = O + dir(theta_2) * r;

					draw(circle(C, rho));
					label("$" + string(nums[i]) + "$", C);
					draw(arc(O, r, theta_2+eps, theta_1-eps), Arrow(TeXHead));
				}
			}

			draw_cycle((-1,0), 0.8, 1, 2, 4);
			draw_cycle((1,0), 0.3, 3);
			draw_cycle((2,0), 0.3, 5);
		\end{asy}
	\end{figure}

	\[
		\gamma = \begin{pmatrix}
			1&4
		\end{pmatrix} \begin{pmatrix}
			1&2
		\end{pmatrix}
	\]

	Avec $n = 6$ et $\gamma = \begin{pmatrix}
		1&3&5&6&2
	\end{pmatrix} = \begin{pmatrix}
		1&2&3&4&5&6\\
		3&1&5&4&6&2
	\end{pmatrix}$.

	Donc, \[
		\gamma = \begin{pmatrix}
			1&2
		\end{pmatrix} \begin{pmatrix}
			1&6
		\end{pmatrix} \begin{pmatrix}
			1&5
		\end{pmatrix} \begin{pmatrix}
			1&3
		\end{pmatrix}
	\] 
	\[
		\begin{pmatrix}
			1&2&3&4&5&6\\
			3&2&1&4&5&6\\
			3&2&5&4&1&6\\
			3&2&5&4&6&1\\
			3&1&5&4&6&2\\
		\end{pmatrix}
	\]

	Et, \[
		\gamma = \begin{pmatrix}
			1&3
		\end{pmatrix} \begin{pmatrix}
			2&3
		\end{pmatrix} \begin{pmatrix}
			3&5
		\end{pmatrix} \begin{pmatrix}
			5&6
		\end{pmatrix} 
	\]

	\[
		\begin{pmatrix}
			1&2&3&4&5&6\\
			1&2&3&4&6&5\\
			1&2&5&4&6&3\\
			1&3&5&4&6&2\\
			3&1&5&4&6&2\\
		\end{pmatrix} 
	\] 
\end{exm}

\begin{exm}
	\[
		\begin{pmatrix}
			1&4
		\end{pmatrix} = \begin{pmatrix}
			1&2
		\end{pmatrix} \begin{pmatrix}
			2&3
		\end{pmatrix} \begin{pmatrix}
			3&4
		\end{pmatrix} \begin{pmatrix}
			2&3
		\end{pmatrix} \begin{pmatrix}
			1&2
		\end{pmatrix}
	\]
	On n'a pas toujours le même nombre de transpositions mais la parité du nombre reste la même (proposition plus loin).
\end{exm}

\begin{thm}
	Toute permutation se décompose en produit de transpositions.
\end{thm}

\begin{prv}
	Soit $\gamma = \begin{pmatrix}
		a_1&\cdots&a_k
	\end{pmatrix}$ un $k$-cycle.

	On remarque que
	\[
		\gamma = \begin{pmatrix}
			a_1&a_k
		\end{pmatrix} \cdots \begin{pmatrix}
			a_1&a_4
		\end{pmatrix} \begin{pmatrix}
			a_1&a_3
		\end{pmatrix} \begin{pmatrix}
			a_1&a_2
		\end{pmatrix}
	\] C'est un produit de transpositions.
\end{prv}

\begin{exm}
	Avec $n = 10$ et $\sigma = \begin{pmatrix}
		1&2&3&4&5&6&7&8&9&10\\
		9&8&1&7&2&3&4&5&10&6
	\end{pmatrix}$.

	On a
	\begin{align*}
		\sigma &= \begin{pmatrix}
			1&9&10&6&3
		\end{pmatrix} \begin{pmatrix}
			2&8&5
		\end{pmatrix} \begin{pmatrix}
			4&7
		\end{pmatrix}\\
		&= \begin{pmatrix}
			1&3
		\end{pmatrix} \begin{pmatrix}
			1&6
		\end{pmatrix} \begin{pmatrix}
			1&10
		\end{pmatrix} \begin{pmatrix}
			1&9
		\end{pmatrix} \begin{pmatrix}
			2&5
		\end{pmatrix} \begin{pmatrix}
			2&8
		\end{pmatrix} \begin{pmatrix}
			4&7
		\end{pmatrix} \\
	\end{align*}

	Vérification : \[
		\begin{pmatrix}
			1&2&3&4&5&6&7&8&9&10\\
			1&2&3&7&5&6&4&8&9&10\\
			1&8&3&7&5&6&4&2&9&10\\
			1&8&3&7&2&6&4&5&9&10\\
			9&8&3&7&2&6&4&5&1&10\\
			9&8&3&7&2&6&4&5&10&1\\
			9&8&3&7&2&1&4&5&10&6\\
			9&8&1&7&2&3&4&5&10&6\\
		\end{pmatrix} 
	\] 
\end{exm}

	\part{Familles orthogonales}

\begin{thm}[Pythagore]
	Soit $(x,y) \in E^2$. \[
		\|x+y\|^2 = \|x\|^2 + \|y\|^2 \iff x \perp y
	.\]
	\begin{figure}[H]
		\centering
		\begin{asy}
			size(4cm);
			pair u = (1, 0.5);
			pair v = 1.5 * (0, 1) * u;
			draw((0,0)--u, Arrow(TeXHead));
			label("$x$", u/2, align=S);
			draw(u--v+u, Arrow(TeXHead));
			label("$y$", u + v/2, align=NE);
			draw((0,0) -- u + v, Arrow(TeXHead));
			draw(u + v / 7.5 -- u + v / 7.5 - u / 5 -- u - u / 5 -- u -- cycle);
		\end{asy}
	\end{figure}
\end{thm}

\begin{prv}
	\[
		\|x + y\|^2 = \|x\|^2 + \|y\|^2 \iff 2\left<x \mid y \right> = 0 \iff x \perp y
	.\]
\end{prv}

\begin{defn}
	Soit $(e_i)_{i\in I}$ une famille de vecteurs. On dit que cette famille est \underline{orthogonale} si \[
		\forall i \neq j\, e_i \perp e_j
	.\] Si, en plus, on a \[
		\forall i \in I,\,\|e_i\| = 1,
	\] alors on dit que la famille est \underline{orthonormale} ou \underline{orthonormée}.
	\index{famille orthogonale}
	\index{famille orthonormale}
	\index{famille orthonormée}
\end{defn}

\begin{prop}[Pythagore]
	Soit $(e_1, \ldots, e_n)$ une famille orthogonale. Alors \[
		\left\| \sum_{i=1}^n e_i \right\|^2 = \sum_{i=1}^n \|e_i\|^2
	.\]
\end{prop}

\begin{thm}
	Toute famille orthogonale de vecteurs non nuls est libre.
\end{thm}

\begin{prv}
	Soit $(e_i)_{i\in I}$ une famille orthogonale telle que \[
		\forall i \in I,\,e_i \neq 0_E
	.\] Soit $n \in \N^*$, $(\lambda_1, \ldots, \lambda_n) \in \R^n$. On suppose \[
		\sum_{k=1}^n \lambda_k e_{i_k} = 0_E
	.\] Soit $j \in \left\llbracket 1,n \right\rrbracket$.
	\begin{align*}
		0 &= \left<\sum_{k=1}^n \lambda_k e_{i_k}  \mid e_{i_j} \right>\\
		&= \sum_{k=1}^n \lambda_k \left<e_{i_k}  \mid e_{i_j} \right> \\
		&= \lambda_j \underbrace{\|e_{i_j}\|^2}_{\neq 0} \\
	\end{align*}
	donc $\lambda_j = 0$.
\end{prv}

\begin{algo}[Orthonormalisation de Gran--Schmidt]
	On suppose $E$ de dimension finie. Soit $\mathcal{B} = (e_1, \ldots, e_n)$ une base de $E$.

	\begin{itemize}
		\item\underline{\it Étape 1}: On pose $v_1 = \frac{e_1}{\|e_1\|}$ de sorte que $\|v_1\| = 1$.
		\item\underline{\it Étape 2} : On pose \[
				u_2 = e_2 - \left<e_2  \mid v_1 \right> v_1
			.\] Ainsi,
			\begin{align*}
				\left<u_2 \mid v_1 \right> &= \big<e_2 - \left<e_2 \mid v_1 \right> v_1  \mid v_1 \big>\\
				&= \left<e_2 \mid v_1 \right> - \left<e_2 \mid v_1 \right> \left<v_1 \mid v_1 \right> \\
				&= 0. \\
			\end{align*}
			On pose $v_2 = \frac{u_2}{\|u_2\|}$ donc $v_2 \perp v_1$ et $\|v_2\| = 1$.
		\item\underline{\it Étape 3} : On pose \[
				u_2 = e_3 - \left<e_3 \mid v_1 \right>v_1 - \left<e_3 \mid v_2 \right>v_2
			.\] Ainsi,
			\begin{align*}
				\left<u_3  \mid v_1 \right> &= \left<e_3  \mid v_1 \right> - \left<e_3 \mid v_1 \right>\underbrace{\left<v_1 \mid v_1 \right>}_{=1} - \left<e_3 \mid v_2 \right>\underbrace{\left<v_2 \mid v_1 \right>}_{=0} \\
				&= 0 \\
			\end{align*}
			et 
			\begin{align*}
				\left<u_3 \mid v_2 \right> &= \left<e_3  \mid  v_2 \right> - \left<e_3 \mid v_1 \right> \underbrace{\left<v_1 \mid v_2 \right>}_{=0} - \left<e_3 \mid v_2 \right> \underbrace{\left<v_2 \mid v_2 \right>}_{=1}\\
				&= 0. \\
			\end{align*}
			On pose $v_3 = \frac{u_3}{\|u_3\|}$ de sorte que $v_3 \perp v_1$, $v_3 \perp v_2$ et $\|v_3\| = 1$.
		\item\underline{\it Étape $i+1$}: On pose \[
			u_{i+1} = e_{i+1} - \sum_{k=1}^i \left<e_{i+1}  \mid v_k \right> v_k
		.\] Ainsi, pour tout $j \in \left\llbracket 1,i \right\rrbracket,$ on a
		\begin{align*}
			\left<u_{i+1}  \mid v_j \right> &= \left<e_{i+1}  \mid v_j \right> - \sum_{k=1}^i \left<e_{i+1} \mid v_k \right> \left<v_k \mid v_j \right> \\
			&= \left<e_{i+1} \mid v_j \right> - \left<e_{i+1} \mid v_j \right> \|v_j\|^2 \\
			&= 0. \\
		\end{align*}
		On pose $v_{i+1} = \frac{u_{i+1}}{\|u_{i+1}\|}$.
	\end{itemize}
\end{algo}

\begin{exm}
	Avec $E = \R_3[X]$, $\left<P \mid Q \right> = \int_{0}^{1} P(t)\,Q(t)~\mathrm{d}t$ et $\mathcal{B} = (1, X, X^2, X^3)$.
	\begin{enumerate}
		\item $\|1\|^2 = \left<1 \mid 1 \right> = \int_{0}^{1} 1~\mathrm{d}t = 1$ et donc $v_1 = 1$.
		\item $u_2 = X - \left<X  \mid v_1 \right>v_1$. Or, $\left<X \mid v_1 \right> = \int_{0}^{1} t~\mathrm{d}t = \frac{1}{2}$. D'où $u_2 = X - \frac{1}{2}$.
			\begin{align*}
				\|u_2\|^2 &= \int_{0}^{1} \left( t - \frac{1}{2} \right)^2~\mathrm{d}t \\
				&= \int_{0}^{1} \left( t^2 - t + \frac{1}{4} \right)~\mathrm{d}t \\
				&= \frac{1}{3} - \frac{1}{2} + \frac{1}{4} \\
				&= \frac{1}{12} \\
			\end{align*} On en déduit que $v_2 = \sqrt{12}\left( X - \frac{1}{2} \right)$.
		\item $u_3 = X^2 - \left<X^2 \mid v_1 \right>v_1 - \left<X^2 \mid v_2 \right>v_2$.
			On a \[
				\left<X^2 \mid v_1 \right> = \int_{0}^{1} t^2~\mathrm{d}t = \frac{1}{3}
			\] et
			\begin{align*}
				\left<X^2 \mid v_2 \right> &= \sqrt{12} \int_{0}^{1} t^2\left( t - \frac{1}{2} \right)~\mathrm{d}t \\
				&= \frac{\sqrt{12}}{12}. \\
			\end{align*}
			D'où
			\begin{align*}
				u_3 &= X^2 - \frac{1}{3} - \frac{\sqrt{12}}{12}\sqrt{12} \left( X - \frac{1}{2} \right)\\
				&= X^2 - \frac{1}{3} - X + \frac{1}{2} \\
				&= X^2 - X + \frac{1}{6}. \\
			\end{align*}
			\begin{align*}
				\|u_3\|^2 &= \int_{0}^{1} \left( t^2 - t + \frac{1}{6} \right)~\mathrm{d}t\\
				&= \int_{0}^{1} \left( t^4 + t^2 + \frac{1}{36} - 2t^3 + \frac{t^2}{3} - \frac{t}{3} \right) ~\mathrm{d}t \\
				&= \frac{1}{5} + \frac{1}{3} + \frac{1}{36} - \frac{1}{2} + \frac{1}{9} - \frac{1}{6} \\
				&= \frac{36 + 60 + 5 - 90 + 20 - 30}{180} \\
				&= \frac{1}{180} \\
			\end{align*}
			On en déduit que \[
				v_3 = 6\sqrt{5}\left( X^2 - X + \frac{1}{6} \right).
			\]
		\item Exercice : calculer $v_4$.
	\end{enumerate}
\end{exm}

\begin{prop}
	Soit $\mathcal{B} = (e_1, \ldots, e_n)$ une base de $E$ et $\mathcal{C}$ la base obtenue par le procédé d'orthonormalisation de Gram--Schmidt. Alors, \[
		\forall i \in \left\llbracket 1,n \right\rrbracket,\,\Vect(e_1,\ldots, e_i) = \Vect(v_1, \ldots, v_i)
	.\]\qed
\end{prop}

\begin{exm}[orthogonalisation]
	\begin{itemize}
		\item $u_1 = 1$.
		\item
			\begin{align*}
				\begin{rcases*}
					u_2 \in \Vect(e_1, e_2)\\
					u_2 \perp u_1
				\end{rcases*}
				\iff& \begin{cases}
					u_2 = ae_1 + be_2\quad (a,b) \in \R^2\\
					\left<u_1 \mid u_2 \right> = 0
				\end{cases}\\
				\iff& \begin{cases}
					u_2 = a + bX\\
					\int_{0}^{1} (a+bt)~\mathrm{d}t = 0.
				\end{cases}\\
			\end{align*}
			\begin{align*}
				\int_{0}^{1} (a+bt)~\mathrm{d}t = 0 \iff& a + \frac{b}{2} = 0\\
				\iff& a = -\frac{b}{2}\\
				\iff& u_2 = -\frac{b}{2} + bX.
			\end{align*}
			Par exemple, $u_2 = -1 + 2X$.
		\item $\begin{cases}
				u_3 \in \Vect(e_1, e_2, e_3)\\
				u_3 \perp u_1\\
				u_3 \perp u_2
			\end{cases}$

			On pose $u_3 = a + bX + cX^2$ avec $(a,b,c)\in \R^3$.
			\begin{align*}
				\begin{rcases*}
					\int_{0}^{1} \left( a+bt + ct^2 \right)~\mathrm{d}t = 0\\
					\int_{0}^{1} \left(a + bt+ct^2\right)(2t - 1)~\mathrm{d}t = 0
				\end{rcases*} \iff& \begin{cases}
					a + \frac{b}{2} + \frac{c}{3} = 0\\
					\int_{0}^{1} \left( 2ct^3 + (-c + 2b)t^2 + (2a - b)t - a \right) ~\mathrm{d}t = 0
				\end{cases}\\
				\iff& \begin{cases}
					a + \frac{b}{2} + \frac{c}{3} = 0\\
					\frac{c}{2} + \frac{2b - c}{3} + \frac{2\cancel{a} - b}{2} - \cancel{a} = 0
				\end{cases}\\
				\iff&  \begin{cases}
					a = -\frac{b}{2} - \frac{c}{3} = \frac{c}{2} - \frac{c}{3} = \frac{c}{6}\\
					b = -c.
				\end{cases}
			\end{align*}
			On en déduit que \[
				u_3 = 1 - 6X + 6X^2
			.\]
	\end{itemize}
\end{exm}

\begin{crlr}[théorème de la base orthonormée incomplète] Soit $(e_1, \ldots, e_k)$ une base orthonormée d'un espace euclidien. On peut trouver $e_{k+1},\ldots,e_n$ tels que $(e_1, \ldots, e_k, e_{k+1},\ldots,e_n)$ soit une base orthonormée de $E$.
\end{crlr}

\begin{prv}
	On sait que $(e_1, \ldots, e_k)$ est libre. On complète $(e_1, \ldots, e_k)$ en une base $\mathcal{B}$ de $E$. On orthonormalise $\mathcal{B}$ : on obtient une base orthonormée $\mathcal{C}$ de $E$. En détaillant l'algorithme de Gram--Schmidt, on s'aper\c coit que les $k$ premiers vecteurs de $\mathcal{C}$ sont ceux de $\mathcal{B}$.
\end{prv}

\begin{thm}
	Soit $E$ un espace euclidien et $\mathcal{B} = (e_1, \ldots, e_n)$ une base orthonormée de $E$. Soit $(x,y) \in E^2$. On pose $(x_1, \ldots, x_n) \in \R^n$ et $(y_1, \ldots, y_n) \in \R^n$ tels que \[
		x = \sum_{i=1}^n x_i e_i \qquad\qquad y = \sum_{i=1}^n y_i e_i
	.\] Alors \[
		\left<x \mid y \right> = \sum_{i=1}^n x_i y_i
	.\]
	\vspace{3mm}
	Soit $X = \mat{x_1\\\vdots\\x_n}$ et $Y = \mat{y_1\\ \vdots \\ y_n}$. Alors, \[
		\left<x \mid y \right> = X^\T\,Y
	.\]
\end{thm}

\begin{prv}
	\begin{align*}
		\left<x \mid y \right> &= \left<\sum_{i=1}^n x_ie_i  \mid y \right>\\
		&= \sum_{i=1}^n x_i \left<e_i  \mid y \right> \\
		&= \sum_{i=1}^n x_i \left<e_i  \mid \sum_{j=1}^n y_j e_j \right> \\
		&= \sum_{i=1}^n x_i \sum_{j=1}^n y_j \underbrace{\left<e_i \mid e_j \right>}_{\delta_i^j} \\
		&= \sum_{i=1}^n x_i y_i. \\
	\end{align*}
\end{prv}

\begin{prop}
	Soit $E$ un espace euclidien et $\mathcal{B} = (e_1, \ldots, e_n)$ une base orthonormée de $E$. Alors, \[
		\forall x \in E,\,x = \sum_{i=1}^n \left<x \mid e_i \right>e_i
	.\]
\end{prop}

\begin{prv}
	Soit $x \in E$. On pose \[
		x = \sum_{i=1}^n x_i e_i
	\] avec $(x_1, \ldots, x_n) \in \R^n$. Soit $j \in \left\llbracket 1,n \right\rrbracket$. On a
	\begin{align*}
		\left<x \mid e_j \right> &= \left<\sum_{i=1}^n x_i e_i  \mid e_j \right>\\
		&= \sum_{i=1}^n x_i \left<e_i \mid e_j \right> \\
		&= x_j. \\
	\end{align*}
\end{prv}

	\part{Lois de composition}

\begin{defn}
	Une \underline{loi de composition interne} \index{loi de composition interne} est une application $f$ de $E \times E$ dans $E$.
	
	On la note $x * y$ au lieu de $f(x,y)$ (on est libre de choisir le symbôle).
\end{defn}

\begin{defn}
	Soit $E$ un ensemble muni d'une loi de composition interne $\boxtimes$.

	On dit que $\boxtimes$ est \underline{associative} \index{associativité (loi de composition interne)} si \[
		\forall (x,y,z) \in E^3,\;(x\boxtimes y)\boxtimes z = x \boxtimes (y \boxtimes z).
	\] Dans ce cas, on écrit plutôt $x \boxtimes y \boxtimes z$.
\end{defn}

\begin{exm}
	\begin{itemize}
		\item $+$ et $\times $ dans $\C$ sont associatives;
		\item $ \circ$ est associative;
		\item  la multiplication matricielle est aussi associative.
	\end{itemize}
\end{exm}

\begin{defn}
	On dit que $\boxtimes$ est \underline{commutative} \index{commutativité (loi de composition interne)} si \[
		\forall (x,y) \in E^2, x\boxtimes y = y\boxtimes x.
	\]
\end{defn}

\begin{exm}
	\begin{itemize}
		\item $+$ et $\times $ dans $\C$ sont commuatives;
		\item $ \circ $ n'est pas commutative;
		\item  la multiplication matricielle n'est pas commutative.
	\end{itemize}
\end{exm}

\begin{defn}
	Soit $e \in E$. On dit que $e$ est un
	\begin{itemize}
		\item \underline{élément neutre à gauche}\index{élément neutre à gauche (loi de composition interne)} si  \[
				\forall x \in E,\; e\boxtimes x = x;
			\]
		\item \underline{élément neutre à droite}\index{élément neutre à droite (loi de composition interne)} si  \[
				\forall x \in E,\; x\boxtimes e = x;
			\]
		\item \underline{élément neutre}\index{élément neutre (loi de composition interne)} si  \[
				\forall x \in E,\; e\boxtimes x = x\boxtimes e = x.
			\]
	\end{itemize}
\end{defn}

\begin{prop}
	Sous reserve d'existence, il y a unicité de l'élément neutre.
\end{prop}

\begin{prv}
	Soient $e$ et $e'$ deux éléments neutre.
	\begin{itemize}
		\item $e \boxtimes e' = e'$ car $e$ est neutre,
		\item $e \boxtimes e' = e$ car $e'$ est neutre.
	\end{itemize} On a donc $e = e'$.
\end{prv}

\begin{axm}[axiome du choix]
	Soit $E$ un ensemble non vide. Il existe $f : \mathcal{P}(E) \setminus \{\O\} \to E$ telle que \[
		\forall A \in \mathcal{P}(E) \setminus \{\O\},\; f(A) \in A.
	\]
\end{axm}

\begin{defn}
	Soit $f: E \to F$. Le \underline{graphe} \index{graphe (application)} de $f$ est \[
		\Big\{\big(x,f(x)\big)  \mid x \in E\Big\} \subset E \times F.
	\]
\end{defn}

\begin{prop}
	Soit $G \subset E\times F$. $G$ est le graphe d'une application si et seulement si \[
		\forall x \in E,\,\exists! y \in F,\, (x,y) \in G.
	\]
\end{prop}

\begin{prv}
	\begin{itemize}
		\item[``$\implies$''] par définition d'une application
		\item[``$\impliedby$''] On pose $f(x)$ le seul élément $y$ de $F$ qui vérifie $(x,y) \in G$. Alors $f \in F^E$ et son graphe vaut $G$.
	\end{itemize}
\end{prv}

\begin{defn}
	Soit $A \in \mathcal{P}(E)$. L'\underline{indicatrice}\index{indicatrice (ensemble)} de $A$ est \begin{align*}
		\mathbbm{1}_A: E &\longrightarrow \{0,1\} \\
		x &\longmapsto \begin{cases}
			1 &\text{ si } x \in A,\\
			0 & \text{ si } x \not\in A.
		\end{cases}
	\end{align*}
\end{defn}

\begin{exm}
	\begin{enumerate}
		\item Dans $\C$, le neutre de $+$ est $0$ et le neutre de $\times$ est $1$.
		\item Dans $E^E$, le neutre de $ \circ $ est $\id_E$.
		\item Dans $\mathcal{M}_n(\C)$ (l'ensemble des matrices carrées $n \times n$ à valeurs dans $\C$), le neutre de $\times $ est $I_n$ : \[
				I_n =
				\begin{pNiceMatrix}
					1&&(0)\\
					&\Ddots&\\
					(0)&&1
				\end{pNiceMatrix}
			\] 
	\end{enumerate}
\end{exm}

\begin{defn}
	Soit $E$ un ensemble muni d'une loi de composition interne $\boxtimes$ et $x \in E$.

	\begin{enumerate}
		\item On dit que $x$ est \underline{simplifiable à gauche}\index{simplifiabilité à gauche} si \[
				\forall (y,z) \in E^2,\,(x\boxtimes y = x \boxtimes z) \implies x = z.
			\] et que $x$ est \underline{simplifiable à droite}\index{simplifiabilité à droite} si \[
				\forall (y,z) \in E^2,\,(y\boxtimes x = z \boxtimes y) \implies x = z.
			\]
		\item On dit que $x$ est \underline{symétrisable à gauche}\index{symétrisabilité à gauche} s'il exiiste $y \in E$ tel que $y\boxtimes x = e$ où $e$ est l'élément neutre de $\boxtimes$.

			De même, on dit que $x$ est \underline{symétrisable à droite}\index{symétrisabilité à droite} s'il existe $y \in E$ tel que $x \boxtimes y = e$.

			On dit que $x$ est \underline{symétrisable}\index{symétrisabilité} s'il est symétrisable à gauche et à droite, donc s'il existe $y \in E$ tel que $x \boxtimes y = y \boxtimes x = e$.
	\end{enumerate}
\end{defn}

\begin{exm}
	$E = \N$ muni de la loi $+$, tous les éléments de $E$ sont simplifiables. $0$ est le seuele élément de $E$ symétrisable.
\end{exm}

\begin{prop}
	Avec les notations précédentes, si $\boxtimes$ est associative, et $x$ est symétrisable, alors $x$ est simplifiable.
\end{prop}

\begin{prv}
	Soient $y, z \in E$.
	\begin{itemize}
		\item On suppose $x \boxtimes y = x \boxtimes z$. Soit $a \in E$ tel que $a\in E$ tel que $a \boxtimes x = e$. Alors \[
				a \boxtimes (x\boxtimes y) = a \boxtimes (x \boxtimes z).
			\] Or,
			\begin{align*}
				a \boxtimes (x \boxtimes y) &= (a \boxtimes x) \boxtimes y \\
				&= e \boxtimes y \\
				&= y. \\
			\end{align*}

			De même, $a \boxtimes (x \boxtimes z) = z$.

			Donc $y = z$.
		\item De même, si $y \boxtimes x = z \boxtimes x$, on ``multiplie'' $x$ à droite par $a$ et on obtient $y = z$.
	\end{itemize}
\end{prv}

\begin{prop-defn}
	On suppose $\boxtimes$ associative. Soit $x \in E$ symétrisable. Alors \[
		\exists ! y \in E,\; x \boxtimes y = y \boxtimes x = e.
	\] On dit que $y$ est le \underline{symétrique}\index{symétrique (loi de composition interne)} de $x$ et on le note $y = x^*$.
\end{prop-defn}

\begin{prv}
	Soeint $x,y,z \in E$ tels que \[
		\begin{cases}
			 x \boxtimes y = y \boxtimes x = e\\
			 x \boxtimes z = z \boxtimes x = e\\
		\end{cases}
	\] Alors, $x \boxtimes y = x \boxtimes z$ et, en simplifiant par $x$, on a $y = z$.
\end{prv}

\begin{exm}
	Les fonctions symétrisables de $(E^E,  \circ)$ sont les bijections et le symétrique d'une bijection est sa réciproque.
\end{exm}

\begin{rmk}
	\begin{enumerate}
		\item Si la loi est notée $+$, on parle d'\underline{opposé}\index{opposé (loi de composition interne)} plutôt que de symétrique et on le note $-x$ au lieu de $x^*$.
			L'élément neutre est noté $0_E$.
		\item Si la loi est notée $\times$, on parle d'élément \underline{inversible}\index{inversibilité (loi de composition interne)} au lieu de symétrisable, d'\underline{inverse}\index{inverse (loi de composition interne)} au lieu de symétrique et on note $x^{-1}$ au lieu de $x^*$. On note le neutre $1_E$.
	\end{enumerate}
\end{rmk}

\begin{exo}
	Soient $x,y \in E = \R^+_*$. On définit la loi de composition interne $\oplus$ : \[
		x \oplus y = \frac{1}{\frac{1}{x}\oplus \frac{1}{y}}.
	\] Cette loi peut-être utile en physique pour le calcul de résistances équivalentes en parallèles.
	\begin{itemize}
		\item {\sc Associativité} : soient $x,y,z \in E$.

			D'une part, on a \[
				x \oplus (y \oplus z) = \frac{1}{\frac{1}{x} + \frac{1}{\frac{1}{\frac{1}{x}+ \frac{1}{y}}}} = \frac{1}{\frac{1}{x}+\frac{1}{y}+\frac{1}{z}}.
			\] D'autre part, on a \[
			(x \oplus y) \oplus z = \frac{1}{\frac{1}{\frac{1}{\frac{1}{x}+\frac{1}{y}}}+\frac{1}{z}} = \frac{1}{\frac{1}{x}+ \frac{1}{y}+\frac{1}{z}}.
			\] La loi $\oplus$ est associative.
		\item {\sc Commutativité} : soient $x, y \in E$. \[
				x \oplus y = \frac{1}{\frac{1}{x}+\frac{1}{y}} = \frac{1}{\frac{1}{y}+\frac{1}{x}} = y\oplus x.
			\] Donc la loi $\oplus$ est commutative.
		\item {\sc Élément neutre} : soit $e$ l'élément neutre de $\oplus$. \[
				\forall x \in E,\; x \oplus e = e \oplus x = x.
			\] Comme la loi est commutative, seul l'égalité $x \oplus e = x$ est utile.

			Soit $x \in E$. On a donc $\frac{1}{\frac{1}{x}+\frac{1}{e}}=x$ donc $\frac{ex}{e+x}=x$ donc $ex = x(e+x)$ et donc $\cancel{ex} = \cancel{ex} + x^2$. On en déduit que $x^2 = 0$, ce qui n'est pas possible car $x \in \R^+_*$. Donc, il n'y a pas d'élément neutre pour $\oplus$.
	\end{itemize}
\end{exo}

	\part{Divers}

\begin{defn}
	Soient $E$ et $F$ deux ensembles. Un \underline{couple}\index{couple} $(x,y)$ est la donnée d'un élément $x$ de $E$ et d'un élément $y$ de $F$ où \[
		\forall x,x' \in E,\,\forall y,y' \in F,\qquad (x,y) = (x',y') \iff \begin{cases}
			x=x',\\
			y=y'.
		\end{cases}
	\] On note $E \times F$ l'ensemble des couples; c'est le \underline{produit cartésien}\index{produit cartésion (ensembles)} de $E$ et $F$.
\end{defn}

\begin{exm}
	$D \times [0,1]$ est un cylindre plein où $D$ est le disque unité fermé i.e. \[
		D = \Big\{(x,y) \in \R^2 \mid x^2+y^2 \le 1\Big\}.
	\]
	\begin{figure}[H]
		\centering
		\begin{subfigure}[b]{3cm}
			\centering
			\begin{asy}
				size(3cm);
				draw(unitcircle);
				draw((0,0)--(1,0), red);
				label("$1$",(0.5,0), red, align=S);
			\end{asy}
		\end{subfigure}
		\begin{subfigure}[b]{3cm}
			\centering
			\begin{asy}
				size(3cm);
				label("$\times\; [0,1]\; =$", (0,0), fontsize(10));
				draw(unitcircle, white+0);
			\end{asy}
		\end{subfigure}
		\begin{subfigure}[b]{3cm}
			\centering
			\begin{asy}
				import solids;
				size(3cm);
				draw(shift((0, 0.5)) * unitcircle, white+0);
				revolution r = cylinder(O, 1, 1.5, Z);
				draw(r);
				triple M = (-1/2, sqrt(3)/2, 0);
				draw((0,0,0) -- M, red);
				label("$1$", M/2, red, align=S);
				draw(M*1.1--M*1.1+(0,0,1.5), magenta, Arrows3(TeXHead2));
				label("$1$", M*1.1+(0,0,0.75), magenta, align=E);
			\end{asy}
		\end{subfigure}
	\end{figure}

	$C \times C$ où $C = \Big\{(x,y) \in \R^2  \mid x^2 + y^2 = 1\Big\}$ est un tore (creu).

	\begin{figure}[H]
		\centering
		\begin{subfigure}[b]{3cm}
			\centering
			\begin{asy}
				size(3cm);
				draw(unitcircle);
				draw((0,0)--(1,0), red);
				label("$1$",(0.5,0), red, align=S);
			\end{asy}
		\end{subfigure}
		\begin{subfigure}[b]{1cm}
			\centering
			\begin{asy}
				size(3cm);
				label("$\times$", (0,0), fontsize(10));
				dot((0.1, 1), white+0);
				dot((-0.1, -1), white+0);
			\end{asy}
		\end{subfigure}
		\begin{subfigure}[b]{3cm}
			\centering
			\begin{asy}
				size(3cm);
				draw(unitcircle);
				draw((0,0)--(1,0), red);
				label("$1$",(0.5,0), red, align=S);
			\end{asy}
		\end{subfigure}
		\begin{subfigure}[b]{1cm}
			\centering
			\begin{asy}
				size(3cm);
				label("$=$", (0,0), fontsize(10));
				dot((0.1, 1), white+0);
				dot((-0.1, -1), white+0);
			\end{asy}
		\end{subfigure}
		\begin{subfigure}[b]{3cm}
			\centering
			\begin{asy}
				import three;
				import graph3;

				size(3cm,3cm);
				surface torus = surface(Circle(c=2Y,normal=X,r=0.5,n=32), c=O, axis=Z, n=32);

				draw(torus, white + opacity(0), meshpen=black + 0.2pt, nolight, render(merge=true));
			\end{asy}
			\vspace{0.7cm}
		\end{subfigure}
	\end{figure}
\end{exm}

\begin{defn}
	Soient $E$ et $F$ deux ensembles. On dit que $E$ et $F$ sont \underline{équipotents} s'il existe une bijection de $E$ dans $F$.
	\index{équipotence (ensembles)}
\end{defn}

\begin{exm}
	\begin{enumerate}
		\item $\N$ et $\N^*$ sont équipotents car  $f : \begin{array}{rcl}
				\N &\longrightarrow& \N^* \\
				k &\longmapsto& k + 1
			\end{array}$ est bijective.
		\item $P = \{n \in \N  \mid n \text{ pair}\}$ et $I= \{n \in \N \mid n \text{ impair}\}$ sont équipotents car $f : \begin{array}{rcl}
				P &\longrightarrow& I \\
				x &\longmapsto& x+1
			\end{array}$ est bijective.
		\item $\N$ et $P$ sont équipotents car $f : \begin{array}{rcl}
				\N &\longrightarrow& P \\
				k &\longmapsto& 2k
			\end{array}$ est bijective.
		\item $[0,1]$ et $[0,1[$ sont équipotents car \begin{align*}
			f: [0,1] &\longrightarrow [0,1[ \\
			x &\longmapsto \begin{cases}
				\frac{1}{n+1} &\text{ si } x = \frac{1}{n} \text{ avec } n \in \N^*\\
				x &\text{ sinon}
			\end{cases}
		\end{align*} est bijective.
		\item De même, $]0,1[$ et $]0,1]$ sont équipotents.
		\item $]0,1[$ et $[0,1[$ sont équipotents : $f : \begin{array}{rcl}
					]0,1] &\longrightarrow& [0,1[ \\
				x &\longmapsto& 1-x
			\end{array}$ est bijective.
		\item $\forall a < b$, $[a,b]$ et $[0,1]$ sont équipotents : \begin{align*}
				f: [0,1] &\longrightarrow [a,b] \\
				\alpha &\longmapsto \alpha b + (1 - \alpha) a
			\end{align*} est bijective (interpolation linéaire).
		\item $\R$ et $]0,1[$ sont équipotents : \begin{align*}
				f: \R &\longrightarrow ]0,1[ \\
				x &\longmapsto \frac{1}{2} + \frac{\Arctan x}{\pi}
			\end{align*} est bijective.
		\item $[0,1[$ et $\N$ ne sont pas équipotents (argument de Cantor). Soit $f: \N \to [0,1[$ une bijection :
			\[
				\begin{array}{c|l}
					k&\hfill f(k)\hfill~ \\ \hline
					0&0,\hfill \!0\hfill 0\hfill 0\hfill 0\hfill\ldots\\
					1&0,\hfill a_1\hfill a_2\hfill a_3\hfill a_4\hfill\ldots\\
					2&0,\hfill b_1\hfill b_2\hfill b_3\hfill b_4\hfill\ldots\\
					\vdots&\hfill\vdots\hfill\ddots
				\end{array}
			\] On considère le nombre \[
				x = 0,\,(a_0+1)(b_1+1)(c_2+1)\cdots
			\] $f(1) \neq x$ car ils n'ont pas le même chiffre des dizaines.\\
			$f(2) \neq x$ car ils n'ont pas le même chiffre des centaines.

			Par le même raisonement, on en déduit que \[
				\forall n \in \N, f(n) \neq x
			\] donc $x$ n'a pas d'antécédant : une contradiction.
		\item On verra en exercice que $E$ et $\mathcal{P}(E)$ ne sont pas équipotents. $\R$ et $\mathcal{P}(\R)$ ne sont pas équipotents mais $\R$ et $\mathcal{P}(\N)$ le sont (développement dyadique).
		\item $\R^2$ et $\R$ sont équipotents; $\C$ et $\R$ sont équipotents.
	\end{enumerate}
\end{exm}

\begin{exo}
	Soit $E$ un ensemble. L'application \begin{align*}
		f: \mathcal{P}(E) &\longrightarrow {0,1}^E \\
		A &\longmapsto \mathbbm{1}_A
	\end{align*} est bijective.

	Soit $g : E \to \{0,1\}$.
	\begin{itemize}
		\item[\underline{\sc Analyse}] Soit $A \in \mathcal{P}(E)$ tel que $f(A) = g$. Alors $g = \mathbbm{1}_A$.
			donc  \[
				\forall x \in E,\; g(x) = \mathbbm{1}_A(x)
			\] et donc \[
				\begin{cases}
					\forall x \in A,\, g(x) = 1\\
					\forall x \in E \setminus A,\,g(x) = 0
				\end{cases}
			\] On en déduit que \[
				A = \{ x \in E  \mid  g(x) = 1\}  = g^{-1}\big(\{1\}\big).
			\]
		\item[\underline{\sc Synthèse}] On pose $A = g^{-1}\big(\{1\}\big)$. Montrons que $f(A) = g$.
			\[
				\forall x \in E,\,g(x) = \begin{cases}
					1 &\text{ si } x \in A\\
					0 &\text{ si } x \not\in A
				\end{cases} = \mathbbm{1}_A
			\] donc $g = \mathbbm{1}_A$.
	\end{itemize}

	On aurait aussi pu rédiger de la fa\c con suivante : on pose \begin{align*}
		u: \{0,1\}^E &\longrightarrow \mathcal{P}(E) \\
		g &\longmapsto g^{-1}\big(\{1\}\big).
	\end{align*} On montre que $u$ est la réciproque de $f$ : \[
		\begin{cases}
			f \circ u = \id_{\{0,1\}^E},\\
			u \circ f = \id_{\mathcal{P}(E)}.
		\end{cases}
	\]
\end{exo}

\begin{defn}
	Soit $f : E \to F$. L'\underline{image de $f$}\index{image (application)} est \[
		\mathrm{Im}(f) = f(E) = \big\{f(x) \mid x \in E\big\}.
	\]
\end{defn}

\begin{prop}
	Soit $f: E \to F$. \[
		f \text{ est surjective } \iff f(E) = F.
	\]
\end{prop}

\begin{defn}
	Une \underline{suite de $E$}\index{suite (ensemble)} est une application de $\N$ dans $E$.
\end{defn}

\begin{rmk}[Notation]
	Soit $u \in E^\N$. Pour $n \in \N$, on écrit $u_n$ à la place de $u(n)$.
\end{rmk}

\begin{defn}
	Soient $E$ et $I$ deux ensembles. Une \underline{famille de $E$ indéxée par $I$}\index{famille (ensemble)} est une application de $I$ dans $E$.

	À la place de $u(i)$ (avec $i \in I$), on écrit $u_i$.
\end{defn}

\begin{defn}
	Soit $E$ un ensemble et $(A_i)_{i \in I}$ une famille de parties de $E$. On suppose $I \neq \O$. On pose \[
		\bigcup_{i \in  I} A_i = \{x \in E  \mid \exists i \in I,\, x \in A_i\}
	\] et \[
		\bigcap_{i \in  I} A_i = \{x \in E  \mid \forall i \in I,\, x \in A_i\}.
	\] On pose aussi $\bigcup_{i \in \O} A_i = \O$ et $\bigcap_{i \in \O}  A_i = E$.
\end{defn}

\begin{rmk}
	De même que pour les sommes et produits de complexes, on peut intervertir des réunions doubles.
\end{rmk}

\begin{prop}
	Soit $E$ un ensemble, $(A,B) \in \mathcal{P}(E)^2$. \[
		A \subset (E \setminus B) \iff A \cap B = \O.
	\]
\end{prop}

\begin{figure}[H]
	\centering
	\begin{asy}
		import patterns;
		add("hatch",hatch(1mm, deepcyan));
		add("hatch2",hatch(1mm, heavygreen));
		size(3cm);

		guide main_set = scale(1.3) * ((-1,1)..(-0.8,-0.8)..(0,-0.9)..(0.7,-1.2)..(0.8, 0.9)..cycle);
		guide set_a = shift((-0.5, -0.2)) * ((-0.6, 0.6)..(0.2,-0.2)..(0.2,-0.4)..(-0.6,-0.2)..cycle);
		guide set_b = shift((0.3, 0.4)) * ((0.8, -0.6)..(1.1,-0.2)..(0.2,0.5)..(0.2,-0.8)..cycle);

		draw(main_set, magenta); label("$E$", 1.3*(0.8,0.9),magenta, align=NE);
		draw(set_a, deepcyan); label("$A$", (-0.6,0.6), deepcyan, align=NW);
		draw(set_b, heavygreen); label("$B$", (0.8,-0.6), heavygreen, align=SE);

		fill(set_a, pattern("hatch"));
		fill(set_b, pattern("hatch2"));
	\end{asy}
\end{figure}

\begin{prv}
	\begin{itemize}
		\item[``$\implies$''] Soit $x \in A \cap B$. Alors $x \in A$ et $x \in B$. Comme $x \in A \subset (E \setminus B)$, alors $x \in E \setminus B$ i.e. $x \not\in B$ : une contradiction. Donc $A \cap B = \O$.
		\item[``$\impliedby$''] On suppose $A \cap B = \O$. Soit $x \in A$. Si $x \in B$, alors $x \in A \cap B = \O$ : faux.
			Donc $x \not\in B$ et donc $x \in E \setminus B$.
	\end{itemize}
\end{prv}

\begin{prop}
	Si $f: E\to F$ et $g: F \to G$ sont bijectives, alors $g \circ f$ est bijective et \[
		(g \circ f)^{-1} = f^{-1} \circ g^{-1}.
	\] \qed
\end{prop}

\begin{rmk}[\danger Attention]
	$g \circ f$ peut-être bijective alors que $f$ et $g$ ne le sont pas.
\end{rmk}


	\part{Rappels sur $\ln$ et $\exp$}

\begin{prop}~\\
	\begin{itemize}
		\item Soit $(a_i)_{i\in I}$ une famille finie de réels strictement positifs. Alors,
			\[
				\ln\left( \prod_{i \in I} a_i \right) = \sum_{i \in I} \ln a_i.
			\]
		\item Soit $(b_i)_{i\in I}$ une famille de réels. Alors \[
			\exp\left( \sum_{i \in I} b_i \right) = \prod_{i \in I} \exp(b_i).
		\]
	\end{itemize}
\end{prop}

\begin{rmk}
	Soit $f: I \to \R^*$ dérivable. On pose $g: x \mapsto \ln \left| f(x) \right|$.

	Alors $g$ est dérivable sur $I$ et \[
		\forall x \in I, g'(x) = \frac{f'(x)}{f(x)}
	\]

	On dit que $\frac{f'}{f}$ est la \underline{dérivée logarithmique} de $f$.

	Soient $f_1,f_2: I\to \R^*$ dérivables. Alors \[
		\frac{(f_1\,f_2)'}{f_1\,f_2} = \frac{f_1'}{f_1} + \frac{f_2'}{f_2}.
	\]
\end{rmk}

\begin{rmk}
	Soit $a \in \R$.
	\begin{itemize}
		\item Soit $n \in \N^*$. Alors, $a^n = \overbrace{a\times a\times a\times \cdots \times a}^{n \text{ fois }}$.
		\item Soit $n \in \Z^-_*$. Si $a \neq 0$, alors $a^n = \frac{1}{a^{-n}}$.
		\item Si $a \neq 0$, $a^{0} = 1$ et \[
				\forall p,q \in \Z, a^{p}\times a^{q} = a^{p+q}.
			\]
		\item Soit $p \in \Z$ et $a > 0$. \[
			a^{p} = \exp(\ln a^{p}) = \exp(p \ln a) = e^{p \ln a}.
		\]
	\end{itemize}
\end{rmk}

\begin{defn}
	Soit $a \in \R^+_*$ et $p \in \R$. On pose $a^p = e^{p \ln a}$.
\end{defn}


	\chap[22]{Fonctions de deux variables}
	\renewcommand{\cwd}{../chap22}
	\part{Modes de définition}

\begin{defn}
	Une suite peut être définie
	\begin{itemize}
		\item \underline{Explicitement}
			On dispose pour tout $n \in \N$ de l'expression de $u_n$ en fonction de $n$.\\
			\ex $\forall n \in \N_*, u_n = \frac{\ln(n)}{n}e^{-n}$\\
		\item \underline{Par récurrence}
			On connait $u_{n+1}$ en fonction de  $u_0, u_1, \ldots, u_n$\\
			\ex $\begin{cases}
				u_0=1\\
				\forall n \in \N, u_{n+1} = \sin(u_n)
			\end{cases}$\\
		\item \underline{Implicitement}
			$\forall n \in \N, u_n$ est le seul nombre verifiant une certaine propriété\\
			\ex $u_n$ est le seul réel vérifiant  $x^5 + nx - 1 = 0$
	\end{itemize}
\end{defn}

	\part{Topologie de $\R^2$}

\begin{defn}
	La \underline{norme (euclidienne)} de $\R^2$ est l'application définie par \[
		\forall (x,y) \in \R^2, \|(x,y)\| = \sqrt{x^2 + y^2}.
	\]

	\begin{figure}[H]
		\centering
		\begin{asy}
			import graph;
			axes(EndArrow);
			size(4cm);
			pair A = (3,2);
			dot(A);
			draw((3,0)--A, dashed);
			draw((0,2)--A, dashed);
			label("$x$", (3,0), align=S);
			label("$y$", (0,2), align=W);
			draw((0,0)--A);
			dot((4,3), white+0);
		\end{asy}
	\end{figure}
	\index{norme (de $\R^2$)}
	\index{norme euclidienne (de $\R^2$)}
\end{defn}

\begin{prop}
	La norme euclidienne vérifie:
	\begin{enumerate}
		\item (séparation) \[
			\forall (x,y) \in \R^2, \|(x,y)\| = 0 \iff x = y = 0,
			\]
		\item (homogénéité positive) \[
				\forall \lambda \in \R, \forall (x,y) \in \R^2, \|\lambda(x,y)\|= \left| \lambda \right| \|(x,y)\|
			\]
		\item (inégalité triangulaire) \[
			\forall (x,y), (a,b) \in \R^2,
			\|(x,y)+(a,b)\|\le \|(x,y)\|+\|(a,b)\|.
		\]
	\end{enumerate}
\end{prop}

\begin{prv}
	Déjà vue en replaçant $(x,y)$ par $x+iy \in \C$ et $\|(x,y)\|$ par |x+iy|
\end{prv}

\begin{defn}
	Soit $(a,b) \in \R^2$ et $r \in \R_+$.

	La \underline{boule ouverte} (ou \underline{disque ouvert}) de centre $(a,b)$ et de rayon $r$ est \[
		B_{(a,b)}(r) = \big\{ (x,y) \in \R^2  \mid \|(x,y) - (a,b)\| < r \big\}.
	\]

	La \underline{boule fermée} (ou \underline{disque fermé}) de centre $(a,b)$ et de rayon $r$ est \[
		\overline{B_{(a,b)}}(r) = \big\{ (x,y)\in \R^2  \mid \|(x,y) - (a,b)\| \le r \big\}.
	\]

	La \underline{sphère} (ou \underline{boule}) de centre $(a,b)$ et de rayon $r$ est \[
		S_{(a,b)}(r) = \partial \overline{B_{(a,b)}}(r) = \big\{ (x,y) \in \R^2  \mid \|(x,y) - (a,b)\| = r \big\}.
	\]
	\index{boule ouverte (de $\R^2$)}
	\index{disque ouverte (de $\R^2$)}
	\index{boule fermée (de $\R^2$)}
	\index{disque fermée (de $\R^2$)}
	\index{boule (de $\R^2$)}
	\index{sphère (de $\R^2$)}
\end{defn}

\begin{figure}[H]
		\centering
		\incfig{boule}
\end{figure}

\begin{rmk}
	On parle de boule en dimension quelconque.
\end{rmk}

\begin{defn}
	Une \underline{partie ouverte} $O$ de $\R^2$ (ou \underline{un ouvert}) si \[
		\forall (x,y) \in O, \exists r > 0, B_{(a,b)}(r) \subset O.
	\]
	Une partie $F$ est \underline{fermée} su $\R^2\setminus F$ est ouverte.
	\index{partie ouverte (de $\R^2$)}
	\index{ouvert (de $\R^2$)}
	\index{partie fermée (de $\R^2$)}
\end{defn}

\begin{figure}[H]
	\centering
	\incfig{partie-ouverte}
\end{figure}

\begin{prop}
	Une boule ouverte est ouverte. Une boule fermée est fermée.
\end{prop}

\begin{figure}[H]
	\centering
	\begin{subfigure}{4cm}
		\centering
		\begin{asy}
			import patterns;

			pair n(pair a) {return a / length(a);}

			add("hatch",hatch(2mm, SW, red));
			size(4cm);

			draw(circle((0,0), 1));
			dot('$(a_0, b_0)$', (0,0), align=S);

			draw((0,0) -- n((-1, 1)), dashed);
			label("$r$", n((-1, 1)) / 2, align=1.5S);

			pair A = n((1,3)) * (2/3);
			real rho = (1 - length(A)) * (2 / 3);

			dot("$(a,b)$", A, red, align=3SE);
			filldraw(circle(A, rho), pattern("hatch"), red);

			label("$O$", n((1,-1))*2.5/3);
		\end{asy}
	\end{subfigure}
	\begin{subfigure}{1cm}
		\centering~\\
	\end{subfigure}
	\begin{subfigure}{5cm}
		\centering
		\begin{asy}
			import patterns;

			pair n(pair a) {return a / length(a);}

			add("hatch",hatch(1mm, SW, red));
			add("hatch2",hatch(3mm, SE, blue));
			size(5cm);

			guide around = (-1.5, -1.5) -- (-1.5, 1.5) -- (2.5, 1.5) -- (2.5, -1.5) -- cycle;

			pair A = n((3, 1)) * 5/3; 
			real rho = 2 / 9;

			picture inter;
			fill(inter, around, pattern("hatch2"));
			fill(inter, circle((0,0), 1), white);
			add(inter);

			draw(circle((0,0), 1));
			dot('$(a_0, b_0)$', (0,0), align=S);

			draw((0,0) -- n((-1, 1)), dashed);
			label("$r$", n((-1, 1)) / 2, align=1.5S);

			dot("$(a,b)$", A, red, align=2SE);
			filldraw(circle(A, rho), pattern("hatch"), red);

			label("$F$", n((1,-1))*2.5/3);
		\end{asy}
	\end{subfigure}
\end{figure}

\begin{prv}
	$\O$ est un ouvert.

	Soit $B$ la boule ouverte de centre $(a_0, b_0) \in \R^2$ et de rayon $r > 0$.

	On pose $\rho = \frac{1}{2}\big(r - \|(a,b) - (a_0,b_0)\|\big)$.
	Montrons que \[
		B_{(a,b)}(\rho) \subset  B_{(a,b)}(r).
	\]

	Soit $(x,y) \in B_{(a,b)}(\rho)$.
	\begin{align*}
		\|(x,y) - (a_0,b_0)\|&= \|(x,y)- (a,b) + (a,b) - (a_0,b_0)\| \\
		&\le \|(x,y) - (a,b)\| + \|(a,b) - (a_0, b_0)\|\\
		&< \rho + \|(a,b) - (a_0, b_0)\| = \frac{1}{2}r + \frac{1}{2} \|(a,b) - (a_0, b_0)\|\\
		&< r
	\end{align*}
	
	Soit $F$ la boule fermée de centre $(a_0, b_0)$ et de rayon $r \ge 0$.

	Soit $(a,b) \not\in F$. On pose \[
		\rho = \frac{1}{2}\big(\|(a,b) - (a_0, b_0)\| - r\big) > 0.
	\]

	Montrons que $B_{(a,b)}(\rho) \subset \R^2\setminus F$.

	Soit $(x,y) \in B_{(a,b)}(\rho)$.

	\begin{align*}
		\|(x,y) - (a_0, b_0)\| &= \|(x,y) - (a,b) + (a,b) - (a_0, b_0)\| \\
		&\ge \big| \underbrace{\|(x,y) - (a,b)\|}_{\le \rho} - \underbrace{\|(a,b) - (a_0, b_0)\|}_{> r} \big|\\
		&\ge \|(a,b) - (a_0, b_0)\|- \|(x,y) - (a,b)\|\\
		&> \|(a,b) - (a_0, b_0)\|- \rho\\
		&> \frac{1}{2} \|(a,b) - (a_0, b_0)\| + \frac{1}{2}r\\
		&> r
	\end{align*}

	donc $(x,y) \not\in F$.
\end{prv}

\begin{exm}
	\begin{enumerate}
		\item $\O$ est ouvert.\\
			$\R^2$ est ouvert.
		\item $\O$ est fermé.\\
			$\R^2$ est fermé.\\
		\item $\big\{(x,0)  \mid x > 0\big\}$ n'est ni ouverte ni fermé.
	\end{enumerate}
\end{exm}

\begin{figure}[H]
	\centering
	\begin{asy}
		size(3cm);

		draw((0, -1) -- (0, 3), Arrow(TeXHead));
		draw((-1, 0) -- (3, 0), Arrow(TeXHead));
		
		draw((0,0) -- (0, 2.97), red);
		draw(circle((0,1.5), 0.5), deepred);
		draw(circle((0,0.5), 0.1), deepred);
	\end{asy}
\end{figure}

\begin{defn}
	Soit $(a,b) \in \R^2$ et $V \in \mathcal{P}(\R^2)$.

	On dit que $V$ est un voisinage de $(a,b)$ s'il existe $r > 0$ tel que \[
		B_{(a,b)}(r) \subset V.
	\]
	\index{voisinage (dans $\R^2$)}
\end{defn}

\begin{prop}
	Un ouvert non vide est un voisinage en chacun de ces points. \qed
\end{prop}

\begin{defn}
	Soit $D \subset \R^2$. Un \underline{point intérieur} de $D$ est un couple $(a,b) \in D$ tel que \[
		\exists r > 0, B_{(a,b)}(r) \subset D.
	\] en d'autres termes, si $D$ est un voisinage de $(a,b)$.

	On note $\mathring D$ l'ensemble des points intérieurs à $D$. C'est \underline{l'intérieur} de $D$.
	\index{point intérieur (dans $\R^2$)}
	\index{intérieur (dans $\R^2$)}
\end{defn}

\begin{prop}
	$\mathring D$ est le plus grand ouvert $O$ de $\R^2$ tel que $O \subset D$.
\end{prop}

\begin{figure}[H]
	\centering
	\incfig{interieur-plus-grand-ouvert}
\end{figure}


\begin{prv}
	Soit $(a,b) \in \mathring D$.

	Par définition, il existe $r > 0$ tel que \[
		B_{(a,b)}(r) \subset D.
	\] Montrons que $B_{(a,b)}(r) \subset \mathring D$.

	Soit $(x,y) \in B_{(a,b)}(r)$. Comme $B_{(a,b)}(r)$ est un ouvert de $\R^2$, il existe $\rho > 0$ tel que \[
		B_{(x,y)}(\rho) \subset B_{(a,b)}(r)
	\] donc $(x,y) \in \mathring D$.

	Donc $\mathring D$ est ouvert, $\mathring D \subset D$.

	Soit $O$ un ouvert de $\R^2$ tel que $O \subset D$. Montrons que $O \subset \mathring D$.

	Soit $(x,y) \in O$. Soit $r > 0$ tel que \[
		B_{(x,y)}(r) \subset O \subset D
	\] donc $(x,y) \in \mathring D$.
\end{prv}

\begin{defn}
	Soit $f: D \subset \R^2 \to \R$, $\ell \in \R$, $(a,b) \in \mathring D$.

	On dit que \underline{$f(x,y)$ tend vers $\ell$ quand $(x,y)$ tend vers $(a,b)$} ou que $\ell$ est \underline{une limite} de $f$ en $(a,b)$ si \[
		\forall \varepsilon > 0, \exists r > 0, \forall (x,y) \in D, \|(x,y) - (a,b)\| < r \implies \left| f(x,y) - \ell \right| \le \varepsilon.
	\] en d'autres termes si \[
		\forall V \in \mathcal{V}_{\ell}, \exists W \in \mathcal{V}_{(a,b)}, \forall (x,y) \in W \cap D, f(x,y) \in V.
	\]
	\index{limite (dans $\R^2$)}
	\index{tendre vers (dans $\R^2$)}
\end{defn}

\begin{prop}
	[unicité de la limite]
	Soit $f: D \to \R$, $(a,b) \in \mathring D$, $\ell_1, \ell_2 \in \R$ telles que $\ell_1$ et $\ell_2$ sont des limites de $f$ en $(a,b)$.

	Alors $\ell_1 = \ell_2$.
\end{prop}

\begin{figure}[H]
	\centering
	\incfig{preuve-unicité-de-la-limite}
\end{figure}

\begin{prv}
	On suppose $\ell_1 < \ell_2$. On pose $\varepsilon = \frac{\ell_2 - \ell_1}{2} > 0$.

	Soit $r_1 > 0$ tel que \[
		f\big(B_{(a,b)}(r_1)\big) \subset ]\ell_1 - \varepsilon, \ell_1 + \varepsilon[.
	\] Soit $r_2 > 0$ tel que \[
		f\big(B_{(a,b)}(r_2)\big) \subset ]\ell_2 - \varepsilon, \ell_2 + \varepsilon [.
	\] On pose $r = \min(r_1, r_2)$ donc \[
		B_{(a,b)}(r_1) \cap B_{(a,b)}(r_2) = B_{(a,b)}(r) \neq \O.
	\] Soit $(x,y) \in B_{(a,b)}(r)$. Alors, \[
		f(x,y) \in ]\ell_1 - \varepsilon, \ell_1 + \varepsilon[ \cap ]\ell_2 - \varepsilon, \ell_2 + \varepsilon[ = \O.
	\] $\lightning$
\end{prv}

\begin{defn}
	Soit $f : D \to \R$, $(a,b) \in \mathring D$.

	On dit que $f$ est \underline{continue} en $(a,b)$ si \[
		f(x,y) \tendsto{(x,y) \to (a,b)}f(a,b).
	\]
	\index{continuité (dans $\R^2$)}
\end{defn}

\begin{prop}
	\underline{Si} $f(x,y) \tendsto{(x,y) \to (a,b)} \ell$ \\
	\underline{alors} $\begin{cases}
		f(x,b) \tendsto{x \to a} \ell\\
		f(a,y) \tendsto{y \to b} \ell.\\
	\end{cases}$
\end{prop}

\begin{prv}~\\
	\begin{figure}[H]
		\centering
		\incfig{limite-x-en-a-et-y-en-b}
	\end{figure}
\end{prv}

\underline{Contre-exemple} : exercice 3.

\begin{exm}
	\begin{enumerate}
		\item $f : \begin{array}{rcl}
				\R^2 &\longrightarrow& \R \\
				(x,y) &\longmapsto& x
			\end{array}$ limite en $(0,0)$ ?

			Soit $\varepsilon > 0$. On pose $r = \varepsilon$. \[
				\forall (x,y) \in B_{(0,0)}(r),
				\left| f(x,y) \right| = \left| x \right| \le \|(x,y)\| < r = \varepsilon
			\] Donc $f(x,y) \tendsto{(x,y) \to (a,b)} 0$.
		\item limite $f : \begin{array}{rcl}
				\R^2 &\longrightarrow& \R \\
				(x,y) &\longmapsto& x^3
			\end{array}$ en $(0,0)$ ?

			Soit $\varepsilon > 0$. On pose $r = \sqrt[3]{r} > 0$. \[
				\forall (x,y) \in B_{(0,0)}(r),
				\left| f(x,y) \right| = \left| x^3 \right| \le \|(x,y)\|^3 < r^3 = \varepsilon.
			\]
		\item limite de $f : \begin{array}{rcl}
			\R^2 &\longrightarrow& \R \\
			(x,y) &\longmapsto& x^3y^2
		\end{array}$ en $(0,0)$ ?

		Soit $\varepsilon > 0$. On pose $r = \sqrt[5]{\varepsilon} > 0$. \[
			\forall (x,y) \in B_{(0,0)}(r), \left| f(x,y) \right| = \left| x^3 y^2 \right| \le \|(x,y)\|^3 \|(x,y)\|^2 < r^5 = \varepsilon.
		\]
	\end{enumerate}
\end{exm}

\begin{defn}
	Soient $D \subset \R^2$ et $(x,y) \in \R^2$.

	\begin{figure}[H]
    \centering
    \incfig{point-adhérent}
	\end{figure}
	
	On dit que $(x,y)$ est \underline{adhérent} à $D$ si \[
		\forall r > 0, B_{(x,y)}(r) \cap D \neq \O.
	\] L'ensemble des points adhérents à $D$ est noté $\overline{D}$. On dit que $\overline{D}$ est \underline{l'adhérence} de $D$.
	\index{point adhérent (dans $\R^2$)}
	\index{adhérent (dans $\R^2$)}
\end{defn}

\begin{defn}
	Soit $f: D \subset \R^2 \to \R$ et $(a,b) \in \overline{D}$, $\ell \in \R$. On dit que $f$ tend vers $\ell$ quand $(x,y)$ tend vers $(a,b)$ si \[
		\forall \varepsilon > 0, \exists r > 0, \forall (x,y) \in B_{(a,b)}(r) \cap D,
		\left| f(x,y) - \ell \right| \le \varepsilon.
	\]
	\index{limite (dans $\R^2$)}
	\index{tendre vers (dans $\R^2$)}
\end{defn}

\begin{prop}
	\begin{enumerate}
		\item Dans ce contexte, il y a unicité de la limite
		\item La limite d'une somme, d'un produit, d'un quotien, d'une composée se comporte comme dans le cas d'une seule variable.
		\item Soit $f: D \to \R$ continue. Soient $g: I \to \R$ et $h: I \to \R$ continues telles que \[
			\forall t \in I, \big(g(t), h(t)\big) \in D.
		\] Alors \[
			t \in I \mapsto f\big(g(t), h(t)\big) \in \R
		\] est continue.
	\end{enumerate}
\end{prop}

\begin{figure}[H]
	\centering
	\begin{asy}
		import three;
		import graph3;
		size(5cm);

		settings.render = 0;
		settings.prc = false;
		currentprojection = obliqueX;

		draw(O -- X, Arrow3(TeXHead2));
		draw(O -- Y, Arrow3(TeXHead2));
		draw(O -- Z, Arrow3(TeXHead2));

		triple f(real x, real y, real z = 0) { return (x,y,cos(x - 0.5) * cos(y - 0.5)/1.2 + 0.15); }

		real inc = 1 / 5;

		for(real x = 0; x <= 1; x += inc) {
			draw(graph(
				new real(real t) { return x; }, // x
				new real(real y) { return y; }, // y
				new real(real y) { return f(x,y).z; }, // z
				0, 1
			), gray);
		}

		for(real y = 0; y <= 1; y += inc) {
			draw(graph(
				new real(real x) { return x; }, // x
				new real(real t) { return y; }, // y
				new real(real x) { return f(x,y).z; }, // z
				0, 1
			), gray);
		}

		path3 path1 = (0.3, 0.2, 0) .. (0.5, 0.5, 0) .. (0.6, 0.7, 0) .. (0.9, 0.8, 0);
		path3 path2 = (0.3, 0.8, 0) .. (0.5, 0.5, 0) .. (0.6, 0.3, 0) .. (0.9, 0.2, 0);
		path3 pathA = f(0.3, 0.2, 0) .. f(0.5, 0.5, 0) .. f(0.6, 0.7, 0) .. f(0.9, 0.8, 0);
		path3 pathB = f(0.3, 0.8, 0) .. f(0.5, 0.5, 0) .. f(0.6, 0.3, 0) .. f(0.9, 0.2, 0);

		draw(path1, red, Arrow3(TeXHead2, position=0.5));
		draw(pathA, red, Arrow3(TeXHead2, position=0.5));
		draw(path2, deepcyan, Arrow3(TeXHead2, position=0.3));
		draw(pathB, deepcyan, Arrow3(TeXHead2, position=0.3));

		dot((0.5, 0.5, 0));
		dot(f(0.5, 0.5, 0));
		draw((0.5, 0.5, 0) -- f(0.5, 0.5, 0), dashed);
	\end{asy}
\end{figure}


	\part{Transpositions}

\begin{defn}
	Une \underline{transposition} est un cycle de longueur 2 : $\begin{pmatrix}
		a&b
	\end{pmatrix}$ avec $a \neq b$.
	\index{transposition (permutation)}
\end{defn}

\begin{exm}
	Avec $n = 5$ et $\gamma = \begin{pmatrix}
		2&4&1
	\end{pmatrix}$.

	\begin{figure}[H]
		\centering

		\begin{asy}
			size(5cm);

			real rho = 0.15; // circles

			void draw_cycle(pair O, real r ...int[] nums) {
				int n = nums.length;
				real eps = (15 / r) * 0.8;

				for(int i = 0; i < n; ++i) {
					real theta_1 = (360/n) * (i+1);
					real theta_2 = (360/n) * i;

					pair C = O + dir(theta_2) * r;

					draw(circle(C, rho));
					label("$" + string(nums[i]) + "$", C);
					draw(arc(O, r, theta_2+eps, theta_1-eps), Arrow(TeXHead));
				}
			}

			draw_cycle((-1,0), 0.8, 1, 2, 4);
			draw_cycle((1,0), 0.3, 3);
			draw_cycle((2,0), 0.3, 5);
		\end{asy}
	\end{figure}

	\[
		\gamma = \begin{pmatrix}
			1&4
		\end{pmatrix} \begin{pmatrix}
			1&2
		\end{pmatrix}
	\]

	Avec $n = 6$ et $\gamma = \begin{pmatrix}
		1&3&5&6&2
	\end{pmatrix} = \begin{pmatrix}
		1&2&3&4&5&6\\
		3&1&5&4&6&2
	\end{pmatrix}$.

	Donc, \[
		\gamma = \begin{pmatrix}
			1&2
		\end{pmatrix} \begin{pmatrix}
			1&6
		\end{pmatrix} \begin{pmatrix}
			1&5
		\end{pmatrix} \begin{pmatrix}
			1&3
		\end{pmatrix}
	\] 
	\[
		\begin{pmatrix}
			1&2&3&4&5&6\\
			3&2&1&4&5&6\\
			3&2&5&4&1&6\\
			3&2&5&4&6&1\\
			3&1&5&4&6&2\\
		\end{pmatrix}
	\]

	Et, \[
		\gamma = \begin{pmatrix}
			1&3
		\end{pmatrix} \begin{pmatrix}
			2&3
		\end{pmatrix} \begin{pmatrix}
			3&5
		\end{pmatrix} \begin{pmatrix}
			5&6
		\end{pmatrix} 
	\]

	\[
		\begin{pmatrix}
			1&2&3&4&5&6\\
			1&2&3&4&6&5\\
			1&2&5&4&6&3\\
			1&3&5&4&6&2\\
			3&1&5&4&6&2\\
		\end{pmatrix} 
	\] 
\end{exm}

\begin{exm}
	\[
		\begin{pmatrix}
			1&4
		\end{pmatrix} = \begin{pmatrix}
			1&2
		\end{pmatrix} \begin{pmatrix}
			2&3
		\end{pmatrix} \begin{pmatrix}
			3&4
		\end{pmatrix} \begin{pmatrix}
			2&3
		\end{pmatrix} \begin{pmatrix}
			1&2
		\end{pmatrix}
	\]
	On n'a pas toujours le même nombre de transpositions mais la parité du nombre reste la même (proposition plus loin).
\end{exm}

\begin{thm}
	Toute permutation se décompose en produit de transpositions.
\end{thm}

\begin{prv}
	Soit $\gamma = \begin{pmatrix}
		a_1&\cdots&a_k
	\end{pmatrix}$ un $k$-cycle.

	On remarque que
	\[
		\gamma = \begin{pmatrix}
			a_1&a_k
		\end{pmatrix} \cdots \begin{pmatrix}
			a_1&a_4
		\end{pmatrix} \begin{pmatrix}
			a_1&a_3
		\end{pmatrix} \begin{pmatrix}
			a_1&a_2
		\end{pmatrix}
	\] C'est un produit de transpositions.
\end{prv}

\begin{exm}
	Avec $n = 10$ et $\sigma = \begin{pmatrix}
		1&2&3&4&5&6&7&8&9&10\\
		9&8&1&7&2&3&4&5&10&6
	\end{pmatrix}$.

	On a
	\begin{align*}
		\sigma &= \begin{pmatrix}
			1&9&10&6&3
		\end{pmatrix} \begin{pmatrix}
			2&8&5
		\end{pmatrix} \begin{pmatrix}
			4&7
		\end{pmatrix}\\
		&= \begin{pmatrix}
			1&3
		\end{pmatrix} \begin{pmatrix}
			1&6
		\end{pmatrix} \begin{pmatrix}
			1&10
		\end{pmatrix} \begin{pmatrix}
			1&9
		\end{pmatrix} \begin{pmatrix}
			2&5
		\end{pmatrix} \begin{pmatrix}
			2&8
		\end{pmatrix} \begin{pmatrix}
			4&7
		\end{pmatrix} \\
	\end{align*}

	Vérification : \[
		\begin{pmatrix}
			1&2&3&4&5&6&7&8&9&10\\
			1&2&3&7&5&6&4&8&9&10\\
			1&8&3&7&5&6&4&2&9&10\\
			1&8&3&7&2&6&4&5&9&10\\
			9&8&3&7&2&6&4&5&1&10\\
			9&8&3&7&2&6&4&5&10&1\\
			9&8&3&7&2&1&4&5&10&6\\
			9&8&1&7&2&3&4&5&10&6\\
		\end{pmatrix} 
	\] 
\end{exm}


	\chap[23]{Dénombrement}
	\renewcommand{\cwd}{../chap23}
	\part{Topologie de $\R^2$}

\begin{defn}
	La \underline{norme (euclidienne)} de $\R^2$ est l'application définie par \[
		\forall (x,y) \in \R^2, \|(x,y)\| = \sqrt{x^2 + y^2}.
	\]

	\begin{figure}[H]
		\centering
		\begin{asy}
			import graph;
			axes(EndArrow);
			size(4cm);
			pair A = (3,2);
			dot(A);
			draw((3,0)--A, dashed);
			draw((0,2)--A, dashed);
			label("$x$", (3,0), align=S);
			label("$y$", (0,2), align=W);
			draw((0,0)--A);
			dot((4,3), white+0);
		\end{asy}
	\end{figure}
	\index{norme (de $\R^2$)}
	\index{norme euclidienne (de $\R^2$)}
\end{defn}

\begin{prop}
	La norme euclidienne vérifie:
	\begin{enumerate}
		\item (séparation) \[
			\forall (x,y) \in \R^2, \|(x,y)\| = 0 \iff x = y = 0,
			\]
		\item (homogénéité positive) \[
				\forall \lambda \in \R, \forall (x,y) \in \R^2, \|\lambda(x,y)\|= \left| \lambda \right| \|(x,y)\|
			\]
		\item (inégalité triangulaire) \[
			\forall (x,y), (a,b) \in \R^2,
			\|(x,y)+(a,b)\|\le \|(x,y)\|+\|(a,b)\|.
		\]
	\end{enumerate}
\end{prop}

\begin{prv}
	Déjà vue en replaçant $(x,y)$ par $x+iy \in \C$ et $\|(x,y)\|$ par |x+iy|
\end{prv}

\begin{defn}
	Soit $(a,b) \in \R^2$ et $r \in \R_+$.

	La \underline{boule ouverte} (ou \underline{disque ouvert}) de centre $(a,b)$ et de rayon $r$ est \[
		B_{(a,b)}(r) = \big\{ (x,y) \in \R^2  \mid \|(x,y) - (a,b)\| < r \big\}.
	\]

	La \underline{boule fermée} (ou \underline{disque fermé}) de centre $(a,b)$ et de rayon $r$ est \[
		\overline{B_{(a,b)}}(r) = \big\{ (x,y)\in \R^2  \mid \|(x,y) - (a,b)\| \le r \big\}.
	\]

	La \underline{sphère} (ou \underline{boule}) de centre $(a,b)$ et de rayon $r$ est \[
		S_{(a,b)}(r) = \partial \overline{B_{(a,b)}}(r) = \big\{ (x,y) \in \R^2  \mid \|(x,y) - (a,b)\| = r \big\}.
	\]
	\index{boule ouverte (de $\R^2$)}
	\index{disque ouverte (de $\R^2$)}
	\index{boule fermée (de $\R^2$)}
	\index{disque fermée (de $\R^2$)}
	\index{boule (de $\R^2$)}
	\index{sphère (de $\R^2$)}
\end{defn}

\begin{figure}[H]
		\centering
		\incfig{boule}
\end{figure}

\begin{rmk}
	On parle de boule en dimension quelconque.
\end{rmk}

\begin{defn}
	Une \underline{partie ouverte} $O$ de $\R^2$ (ou \underline{un ouvert}) si \[
		\forall (x,y) \in O, \exists r > 0, B_{(a,b)}(r) \subset O.
	\]
	Une partie $F$ est \underline{fermée} su $\R^2\setminus F$ est ouverte.
	\index{partie ouverte (de $\R^2$)}
	\index{ouvert (de $\R^2$)}
	\index{partie fermée (de $\R^2$)}
\end{defn}

\begin{figure}[H]
	\centering
	\incfig{partie-ouverte}
\end{figure}

\begin{prop}
	Une boule ouverte est ouverte. Une boule fermée est fermée.
\end{prop}

\begin{figure}[H]
	\centering
	\begin{subfigure}{4cm}
		\centering
		\begin{asy}
			import patterns;

			pair n(pair a) {return a / length(a);}

			add("hatch",hatch(2mm, SW, red));
			size(4cm);

			draw(circle((0,0), 1));
			dot('$(a_0, b_0)$', (0,0), align=S);

			draw((0,0) -- n((-1, 1)), dashed);
			label("$r$", n((-1, 1)) / 2, align=1.5S);

			pair A = n((1,3)) * (2/3);
			real rho = (1 - length(A)) * (2 / 3);

			dot("$(a,b)$", A, red, align=3SE);
			filldraw(circle(A, rho), pattern("hatch"), red);

			label("$O$", n((1,-1))*2.5/3);
		\end{asy}
	\end{subfigure}
	\begin{subfigure}{1cm}
		\centering~\\
	\end{subfigure}
	\begin{subfigure}{5cm}
		\centering
		\begin{asy}
			import patterns;

			pair n(pair a) {return a / length(a);}

			add("hatch",hatch(1mm, SW, red));
			add("hatch2",hatch(3mm, SE, blue));
			size(5cm);

			guide around = (-1.5, -1.5) -- (-1.5, 1.5) -- (2.5, 1.5) -- (2.5, -1.5) -- cycle;

			pair A = n((3, 1)) * 5/3; 
			real rho = 2 / 9;

			picture inter;
			fill(inter, around, pattern("hatch2"));
			fill(inter, circle((0,0), 1), white);
			add(inter);

			draw(circle((0,0), 1));
			dot('$(a_0, b_0)$', (0,0), align=S);

			draw((0,0) -- n((-1, 1)), dashed);
			label("$r$", n((-1, 1)) / 2, align=1.5S);

			dot("$(a,b)$", A, red, align=2SE);
			filldraw(circle(A, rho), pattern("hatch"), red);

			label("$F$", n((1,-1))*2.5/3);
		\end{asy}
	\end{subfigure}
\end{figure}

\begin{prv}
	$\O$ est un ouvert.

	Soit $B$ la boule ouverte de centre $(a_0, b_0) \in \R^2$ et de rayon $r > 0$.

	On pose $\rho = \frac{1}{2}\big(r - \|(a,b) - (a_0,b_0)\|\big)$.
	Montrons que \[
		B_{(a,b)}(\rho) \subset  B_{(a,b)}(r).
	\]

	Soit $(x,y) \in B_{(a,b)}(\rho)$.
	\begin{align*}
		\|(x,y) - (a_0,b_0)\|&= \|(x,y)- (a,b) + (a,b) - (a_0,b_0)\| \\
		&\le \|(x,y) - (a,b)\| + \|(a,b) - (a_0, b_0)\|\\
		&< \rho + \|(a,b) - (a_0, b_0)\| = \frac{1}{2}r + \frac{1}{2} \|(a,b) - (a_0, b_0)\|\\
		&< r
	\end{align*}
	
	Soit $F$ la boule fermée de centre $(a_0, b_0)$ et de rayon $r \ge 0$.

	Soit $(a,b) \not\in F$. On pose \[
		\rho = \frac{1}{2}\big(\|(a,b) - (a_0, b_0)\| - r\big) > 0.
	\]

	Montrons que $B_{(a,b)}(\rho) \subset \R^2\setminus F$.

	Soit $(x,y) \in B_{(a,b)}(\rho)$.

	\begin{align*}
		\|(x,y) - (a_0, b_0)\| &= \|(x,y) - (a,b) + (a,b) - (a_0, b_0)\| \\
		&\ge \big| \underbrace{\|(x,y) - (a,b)\|}_{\le \rho} - \underbrace{\|(a,b) - (a_0, b_0)\|}_{> r} \big|\\
		&\ge \|(a,b) - (a_0, b_0)\|- \|(x,y) - (a,b)\|\\
		&> \|(a,b) - (a_0, b_0)\|- \rho\\
		&> \frac{1}{2} \|(a,b) - (a_0, b_0)\| + \frac{1}{2}r\\
		&> r
	\end{align*}

	donc $(x,y) \not\in F$.
\end{prv}

\begin{exm}
	\begin{enumerate}
		\item $\O$ est ouvert.\\
			$\R^2$ est ouvert.
		\item $\O$ est fermé.\\
			$\R^2$ est fermé.\\
		\item $\big\{(x,0)  \mid x > 0\big\}$ n'est ni ouverte ni fermé.
	\end{enumerate}
\end{exm}

\begin{figure}[H]
	\centering
	\begin{asy}
		size(3cm);

		draw((0, -1) -- (0, 3), Arrow(TeXHead));
		draw((-1, 0) -- (3, 0), Arrow(TeXHead));
		
		draw((0,0) -- (0, 2.97), red);
		draw(circle((0,1.5), 0.5), deepred);
		draw(circle((0,0.5), 0.1), deepred);
	\end{asy}
\end{figure}

\begin{defn}
	Soit $(a,b) \in \R^2$ et $V \in \mathcal{P}(\R^2)$.

	On dit que $V$ est un voisinage de $(a,b)$ s'il existe $r > 0$ tel que \[
		B_{(a,b)}(r) \subset V.
	\]
	\index{voisinage (dans $\R^2$)}
\end{defn}

\begin{prop}
	Un ouvert non vide est un voisinage en chacun de ces points. \qed
\end{prop}

\begin{defn}
	Soit $D \subset \R^2$. Un \underline{point intérieur} de $D$ est un couple $(a,b) \in D$ tel que \[
		\exists r > 0, B_{(a,b)}(r) \subset D.
	\] en d'autres termes, si $D$ est un voisinage de $(a,b)$.

	On note $\mathring D$ l'ensemble des points intérieurs à $D$. C'est \underline{l'intérieur} de $D$.
	\index{point intérieur (dans $\R^2$)}
	\index{intérieur (dans $\R^2$)}
\end{defn}

\begin{prop}
	$\mathring D$ est le plus grand ouvert $O$ de $\R^2$ tel que $O \subset D$.
\end{prop}

\begin{figure}[H]
	\centering
	\incfig{interieur-plus-grand-ouvert}
\end{figure}


\begin{prv}
	Soit $(a,b) \in \mathring D$.

	Par définition, il existe $r > 0$ tel que \[
		B_{(a,b)}(r) \subset D.
	\] Montrons que $B_{(a,b)}(r) \subset \mathring D$.

	Soit $(x,y) \in B_{(a,b)}(r)$. Comme $B_{(a,b)}(r)$ est un ouvert de $\R^2$, il existe $\rho > 0$ tel que \[
		B_{(x,y)}(\rho) \subset B_{(a,b)}(r)
	\] donc $(x,y) \in \mathring D$.

	Donc $\mathring D$ est ouvert, $\mathring D \subset D$.

	Soit $O$ un ouvert de $\R^2$ tel que $O \subset D$. Montrons que $O \subset \mathring D$.

	Soit $(x,y) \in O$. Soit $r > 0$ tel que \[
		B_{(x,y)}(r) \subset O \subset D
	\] donc $(x,y) \in \mathring D$.
\end{prv}

\begin{defn}
	Soit $f: D \subset \R^2 \to \R$, $\ell \in \R$, $(a,b) \in \mathring D$.

	On dit que \underline{$f(x,y)$ tend vers $\ell$ quand $(x,y)$ tend vers $(a,b)$} ou que $\ell$ est \underline{une limite} de $f$ en $(a,b)$ si \[
		\forall \varepsilon > 0, \exists r > 0, \forall (x,y) \in D, \|(x,y) - (a,b)\| < r \implies \left| f(x,y) - \ell \right| \le \varepsilon.
	\] en d'autres termes si \[
		\forall V \in \mathcal{V}_{\ell}, \exists W \in \mathcal{V}_{(a,b)}, \forall (x,y) \in W \cap D, f(x,y) \in V.
	\]
	\index{limite (dans $\R^2$)}
	\index{tendre vers (dans $\R^2$)}
\end{defn}

\begin{prop}
	[unicité de la limite]
	Soit $f: D \to \R$, $(a,b) \in \mathring D$, $\ell_1, \ell_2 \in \R$ telles que $\ell_1$ et $\ell_2$ sont des limites de $f$ en $(a,b)$.

	Alors $\ell_1 = \ell_2$.
\end{prop}

\begin{figure}[H]
	\centering
	\incfig{preuve-unicité-de-la-limite}
\end{figure}

\begin{prv}
	On suppose $\ell_1 < \ell_2$. On pose $\varepsilon = \frac{\ell_2 - \ell_1}{2} > 0$.

	Soit $r_1 > 0$ tel que \[
		f\big(B_{(a,b)}(r_1)\big) \subset ]\ell_1 - \varepsilon, \ell_1 + \varepsilon[.
	\] Soit $r_2 > 0$ tel que \[
		f\big(B_{(a,b)}(r_2)\big) \subset ]\ell_2 - \varepsilon, \ell_2 + \varepsilon [.
	\] On pose $r = \min(r_1, r_2)$ donc \[
		B_{(a,b)}(r_1) \cap B_{(a,b)}(r_2) = B_{(a,b)}(r) \neq \O.
	\] Soit $(x,y) \in B_{(a,b)}(r)$. Alors, \[
		f(x,y) \in ]\ell_1 - \varepsilon, \ell_1 + \varepsilon[ \cap ]\ell_2 - \varepsilon, \ell_2 + \varepsilon[ = \O.
	\] $\lightning$
\end{prv}

\begin{defn}
	Soit $f : D \to \R$, $(a,b) \in \mathring D$.

	On dit que $f$ est \underline{continue} en $(a,b)$ si \[
		f(x,y) \tendsto{(x,y) \to (a,b)}f(a,b).
	\]
	\index{continuité (dans $\R^2$)}
\end{defn}

\begin{prop}
	\underline{Si} $f(x,y) \tendsto{(x,y) \to (a,b)} \ell$ \\
	\underline{alors} $\begin{cases}
		f(x,b) \tendsto{x \to a} \ell\\
		f(a,y) \tendsto{y \to b} \ell.\\
	\end{cases}$
\end{prop}

\begin{prv}~\\
	\begin{figure}[H]
		\centering
		\incfig{limite-x-en-a-et-y-en-b}
	\end{figure}
\end{prv}

\underline{Contre-exemple} : exercice 3.

\begin{exm}
	\begin{enumerate}
		\item $f : \begin{array}{rcl}
				\R^2 &\longrightarrow& \R \\
				(x,y) &\longmapsto& x
			\end{array}$ limite en $(0,0)$ ?

			Soit $\varepsilon > 0$. On pose $r = \varepsilon$. \[
				\forall (x,y) \in B_{(0,0)}(r),
				\left| f(x,y) \right| = \left| x \right| \le \|(x,y)\| < r = \varepsilon
			\] Donc $f(x,y) \tendsto{(x,y) \to (a,b)} 0$.
		\item limite $f : \begin{array}{rcl}
				\R^2 &\longrightarrow& \R \\
				(x,y) &\longmapsto& x^3
			\end{array}$ en $(0,0)$ ?

			Soit $\varepsilon > 0$. On pose $r = \sqrt[3]{r} > 0$. \[
				\forall (x,y) \in B_{(0,0)}(r),
				\left| f(x,y) \right| = \left| x^3 \right| \le \|(x,y)\|^3 < r^3 = \varepsilon.
			\]
		\item limite de $f : \begin{array}{rcl}
			\R^2 &\longrightarrow& \R \\
			(x,y) &\longmapsto& x^3y^2
		\end{array}$ en $(0,0)$ ?

		Soit $\varepsilon > 0$. On pose $r = \sqrt[5]{\varepsilon} > 0$. \[
			\forall (x,y) \in B_{(0,0)}(r), \left| f(x,y) \right| = \left| x^3 y^2 \right| \le \|(x,y)\|^3 \|(x,y)\|^2 < r^5 = \varepsilon.
		\]
	\end{enumerate}
\end{exm}

\begin{defn}
	Soient $D \subset \R^2$ et $(x,y) \in \R^2$.

	\begin{figure}[H]
    \centering
    \incfig{point-adhérent}
	\end{figure}
	
	On dit que $(x,y)$ est \underline{adhérent} à $D$ si \[
		\forall r > 0, B_{(x,y)}(r) \cap D \neq \O.
	\] L'ensemble des points adhérents à $D$ est noté $\overline{D}$. On dit que $\overline{D}$ est \underline{l'adhérence} de $D$.
	\index{point adhérent (dans $\R^2$)}
	\index{adhérent (dans $\R^2$)}
\end{defn}

\begin{defn}
	Soit $f: D \subset \R^2 \to \R$ et $(a,b) \in \overline{D}$, $\ell \in \R$. On dit que $f$ tend vers $\ell$ quand $(x,y)$ tend vers $(a,b)$ si \[
		\forall \varepsilon > 0, \exists r > 0, \forall (x,y) \in B_{(a,b)}(r) \cap D,
		\left| f(x,y) - \ell \right| \le \varepsilon.
	\]
	\index{limite (dans $\R^2$)}
	\index{tendre vers (dans $\R^2$)}
\end{defn}

\begin{prop}
	\begin{enumerate}
		\item Dans ce contexte, il y a unicité de la limite
		\item La limite d'une somme, d'un produit, d'un quotien, d'une composée se comporte comme dans le cas d'une seule variable.
		\item Soit $f: D \to \R$ continue. Soient $g: I \to \R$ et $h: I \to \R$ continues telles que \[
			\forall t \in I, \big(g(t), h(t)\big) \in D.
		\] Alors \[
			t \in I \mapsto f\big(g(t), h(t)\big) \in \R
		\] est continue.
	\end{enumerate}
\end{prop}

\begin{figure}[H]
	\centering
	\begin{asy}
		import three;
		import graph3;
		size(5cm);

		settings.render = 0;
		settings.prc = false;
		currentprojection = obliqueX;

		draw(O -- X, Arrow3(TeXHead2));
		draw(O -- Y, Arrow3(TeXHead2));
		draw(O -- Z, Arrow3(TeXHead2));

		triple f(real x, real y, real z = 0) { return (x,y,cos(x - 0.5) * cos(y - 0.5)/1.2 + 0.15); }

		real inc = 1 / 5;

		for(real x = 0; x <= 1; x += inc) {
			draw(graph(
				new real(real t) { return x; }, // x
				new real(real y) { return y; }, // y
				new real(real y) { return f(x,y).z; }, // z
				0, 1
			), gray);
		}

		for(real y = 0; y <= 1; y += inc) {
			draw(graph(
				new real(real x) { return x; }, // x
				new real(real t) { return y; }, // y
				new real(real x) { return f(x,y).z; }, // z
				0, 1
			), gray);
		}

		path3 path1 = (0.3, 0.2, 0) .. (0.5, 0.5, 0) .. (0.6, 0.7, 0) .. (0.9, 0.8, 0);
		path3 path2 = (0.3, 0.8, 0) .. (0.5, 0.5, 0) .. (0.6, 0.3, 0) .. (0.9, 0.2, 0);
		path3 pathA = f(0.3, 0.2, 0) .. f(0.5, 0.5, 0) .. f(0.6, 0.7, 0) .. f(0.9, 0.8, 0);
		path3 pathB = f(0.3, 0.8, 0) .. f(0.5, 0.5, 0) .. f(0.6, 0.3, 0) .. f(0.9, 0.2, 0);

		draw(path1, red, Arrow3(TeXHead2, position=0.5));
		draw(pathA, red, Arrow3(TeXHead2, position=0.5));
		draw(path2, deepcyan, Arrow3(TeXHead2, position=0.3));
		draw(pathB, deepcyan, Arrow3(TeXHead2, position=0.3));

		dot((0.5, 0.5, 0));
		dot(f(0.5, 0.5, 0));
		draw((0.5, 0.5, 0) -- f(0.5, 0.5, 0), dashed);
	\end{asy}
\end{figure}


	\part{Transpositions}

\begin{defn}
	Une \underline{transposition} est un cycle de longueur 2 : $\begin{pmatrix}
		a&b
	\end{pmatrix}$ avec $a \neq b$.
	\index{transposition (permutation)}
\end{defn}

\begin{exm}
	Avec $n = 5$ et $\gamma = \begin{pmatrix}
		2&4&1
	\end{pmatrix}$.

	\begin{figure}[H]
		\centering

		\begin{asy}
			size(5cm);

			real rho = 0.15; // circles

			void draw_cycle(pair O, real r ...int[] nums) {
				int n = nums.length;
				real eps = (15 / r) * 0.8;

				for(int i = 0; i < n; ++i) {
					real theta_1 = (360/n) * (i+1);
					real theta_2 = (360/n) * i;

					pair C = O + dir(theta_2) * r;

					draw(circle(C, rho));
					label("$" + string(nums[i]) + "$", C);
					draw(arc(O, r, theta_2+eps, theta_1-eps), Arrow(TeXHead));
				}
			}

			draw_cycle((-1,0), 0.8, 1, 2, 4);
			draw_cycle((1,0), 0.3, 3);
			draw_cycle((2,0), 0.3, 5);
		\end{asy}
	\end{figure}

	\[
		\gamma = \begin{pmatrix}
			1&4
		\end{pmatrix} \begin{pmatrix}
			1&2
		\end{pmatrix}
	\]

	Avec $n = 6$ et $\gamma = \begin{pmatrix}
		1&3&5&6&2
	\end{pmatrix} = \begin{pmatrix}
		1&2&3&4&5&6\\
		3&1&5&4&6&2
	\end{pmatrix}$.

	Donc, \[
		\gamma = \begin{pmatrix}
			1&2
		\end{pmatrix} \begin{pmatrix}
			1&6
		\end{pmatrix} \begin{pmatrix}
			1&5
		\end{pmatrix} \begin{pmatrix}
			1&3
		\end{pmatrix}
	\] 
	\[
		\begin{pmatrix}
			1&2&3&4&5&6\\
			3&2&1&4&5&6\\
			3&2&5&4&1&6\\
			3&2&5&4&6&1\\
			3&1&5&4&6&2\\
		\end{pmatrix}
	\]

	Et, \[
		\gamma = \begin{pmatrix}
			1&3
		\end{pmatrix} \begin{pmatrix}
			2&3
		\end{pmatrix} \begin{pmatrix}
			3&5
		\end{pmatrix} \begin{pmatrix}
			5&6
		\end{pmatrix} 
	\]

	\[
		\begin{pmatrix}
			1&2&3&4&5&6\\
			1&2&3&4&6&5\\
			1&2&5&4&6&3\\
			1&3&5&4&6&2\\
			3&1&5&4&6&2\\
		\end{pmatrix} 
	\] 
\end{exm}

\begin{exm}
	\[
		\begin{pmatrix}
			1&4
		\end{pmatrix} = \begin{pmatrix}
			1&2
		\end{pmatrix} \begin{pmatrix}
			2&3
		\end{pmatrix} \begin{pmatrix}
			3&4
		\end{pmatrix} \begin{pmatrix}
			2&3
		\end{pmatrix} \begin{pmatrix}
			1&2
		\end{pmatrix}
	\]
	On n'a pas toujours le même nombre de transpositions mais la parité du nombre reste la même (proposition plus loin).
\end{exm}

\begin{thm}
	Toute permutation se décompose en produit de transpositions.
\end{thm}

\begin{prv}
	Soit $\gamma = \begin{pmatrix}
		a_1&\cdots&a_k
	\end{pmatrix}$ un $k$-cycle.

	On remarque que
	\[
		\gamma = \begin{pmatrix}
			a_1&a_k
		\end{pmatrix} \cdots \begin{pmatrix}
			a_1&a_4
		\end{pmatrix} \begin{pmatrix}
			a_1&a_3
		\end{pmatrix} \begin{pmatrix}
			a_1&a_2
		\end{pmatrix}
	\] C'est un produit de transpositions.
\end{prv}

\begin{exm}
	Avec $n = 10$ et $\sigma = \begin{pmatrix}
		1&2&3&4&5&6&7&8&9&10\\
		9&8&1&7&2&3&4&5&10&6
	\end{pmatrix}$.

	On a
	\begin{align*}
		\sigma &= \begin{pmatrix}
			1&9&10&6&3
		\end{pmatrix} \begin{pmatrix}
			2&8&5
		\end{pmatrix} \begin{pmatrix}
			4&7
		\end{pmatrix}\\
		&= \begin{pmatrix}
			1&3
		\end{pmatrix} \begin{pmatrix}
			1&6
		\end{pmatrix} \begin{pmatrix}
			1&10
		\end{pmatrix} \begin{pmatrix}
			1&9
		\end{pmatrix} \begin{pmatrix}
			2&5
		\end{pmatrix} \begin{pmatrix}
			2&8
		\end{pmatrix} \begin{pmatrix}
			4&7
		\end{pmatrix} \\
	\end{align*}

	Vérification : \[
		\begin{pmatrix}
			1&2&3&4&5&6&7&8&9&10\\
			1&2&3&7&5&6&4&8&9&10\\
			1&8&3&7&5&6&4&2&9&10\\
			1&8&3&7&2&6&4&5&9&10\\
			9&8&3&7&2&6&4&5&1&10\\
			9&8&3&7&2&6&4&5&10&1\\
			9&8&3&7&2&1&4&5&10&6\\
			9&8&1&7&2&3&4&5&10&6\\
		\end{pmatrix} 
	\] 
\end{exm}

	\part{Familles orthogonales}

\begin{thm}[Pythagore]
	Soit $(x,y) \in E^2$. \[
		\|x+y\|^2 = \|x\|^2 + \|y\|^2 \iff x \perp y
	.\]
	\begin{figure}[H]
		\centering
		\begin{asy}
			size(4cm);
			pair u = (1, 0.5);
			pair v = 1.5 * (0, 1) * u;
			draw((0,0)--u, Arrow(TeXHead));
			label("$x$", u/2, align=S);
			draw(u--v+u, Arrow(TeXHead));
			label("$y$", u + v/2, align=NE);
			draw((0,0) -- u + v, Arrow(TeXHead));
			draw(u + v / 7.5 -- u + v / 7.5 - u / 5 -- u - u / 5 -- u -- cycle);
		\end{asy}
	\end{figure}
\end{thm}

\begin{prv}
	\[
		\|x + y\|^2 = \|x\|^2 + \|y\|^2 \iff 2\left<x \mid y \right> = 0 \iff x \perp y
	.\]
\end{prv}

\begin{defn}
	Soit $(e_i)_{i\in I}$ une famille de vecteurs. On dit que cette famille est \underline{orthogonale} si \[
		\forall i \neq j\, e_i \perp e_j
	.\] Si, en plus, on a \[
		\forall i \in I,\,\|e_i\| = 1,
	\] alors on dit que la famille est \underline{orthonormale} ou \underline{orthonormée}.
	\index{famille orthogonale}
	\index{famille orthonormale}
	\index{famille orthonormée}
\end{defn}

\begin{prop}[Pythagore]
	Soit $(e_1, \ldots, e_n)$ une famille orthogonale. Alors \[
		\left\| \sum_{i=1}^n e_i \right\|^2 = \sum_{i=1}^n \|e_i\|^2
	.\]
\end{prop}

\begin{thm}
	Toute famille orthogonale de vecteurs non nuls est libre.
\end{thm}

\begin{prv}
	Soit $(e_i)_{i\in I}$ une famille orthogonale telle que \[
		\forall i \in I,\,e_i \neq 0_E
	.\] Soit $n \in \N^*$, $(\lambda_1, \ldots, \lambda_n) \in \R^n$. On suppose \[
		\sum_{k=1}^n \lambda_k e_{i_k} = 0_E
	.\] Soit $j \in \left\llbracket 1,n \right\rrbracket$.
	\begin{align*}
		0 &= \left<\sum_{k=1}^n \lambda_k e_{i_k}  \mid e_{i_j} \right>\\
		&= \sum_{k=1}^n \lambda_k \left<e_{i_k}  \mid e_{i_j} \right> \\
		&= \lambda_j \underbrace{\|e_{i_j}\|^2}_{\neq 0} \\
	\end{align*}
	donc $\lambda_j = 0$.
\end{prv}

\begin{algo}[Orthonormalisation de Gran--Schmidt]
	On suppose $E$ de dimension finie. Soit $\mathcal{B} = (e_1, \ldots, e_n)$ une base de $E$.

	\begin{itemize}
		\item\underline{\it Étape 1}: On pose $v_1 = \frac{e_1}{\|e_1\|}$ de sorte que $\|v_1\| = 1$.
		\item\underline{\it Étape 2} : On pose \[
				u_2 = e_2 - \left<e_2  \mid v_1 \right> v_1
			.\] Ainsi,
			\begin{align*}
				\left<u_2 \mid v_1 \right> &= \big<e_2 - \left<e_2 \mid v_1 \right> v_1  \mid v_1 \big>\\
				&= \left<e_2 \mid v_1 \right> - \left<e_2 \mid v_1 \right> \left<v_1 \mid v_1 \right> \\
				&= 0. \\
			\end{align*}
			On pose $v_2 = \frac{u_2}{\|u_2\|}$ donc $v_2 \perp v_1$ et $\|v_2\| = 1$.
		\item\underline{\it Étape 3} : On pose \[
				u_2 = e_3 - \left<e_3 \mid v_1 \right>v_1 - \left<e_3 \mid v_2 \right>v_2
			.\] Ainsi,
			\begin{align*}
				\left<u_3  \mid v_1 \right> &= \left<e_3  \mid v_1 \right> - \left<e_3 \mid v_1 \right>\underbrace{\left<v_1 \mid v_1 \right>}_{=1} - \left<e_3 \mid v_2 \right>\underbrace{\left<v_2 \mid v_1 \right>}_{=0} \\
				&= 0 \\
			\end{align*}
			et 
			\begin{align*}
				\left<u_3 \mid v_2 \right> &= \left<e_3  \mid  v_2 \right> - \left<e_3 \mid v_1 \right> \underbrace{\left<v_1 \mid v_2 \right>}_{=0} - \left<e_3 \mid v_2 \right> \underbrace{\left<v_2 \mid v_2 \right>}_{=1}\\
				&= 0. \\
			\end{align*}
			On pose $v_3 = \frac{u_3}{\|u_3\|}$ de sorte que $v_3 \perp v_1$, $v_3 \perp v_2$ et $\|v_3\| = 1$.
		\item\underline{\it Étape $i+1$}: On pose \[
			u_{i+1} = e_{i+1} - \sum_{k=1}^i \left<e_{i+1}  \mid v_k \right> v_k
		.\] Ainsi, pour tout $j \in \left\llbracket 1,i \right\rrbracket,$ on a
		\begin{align*}
			\left<u_{i+1}  \mid v_j \right> &= \left<e_{i+1}  \mid v_j \right> - \sum_{k=1}^i \left<e_{i+1} \mid v_k \right> \left<v_k \mid v_j \right> \\
			&= \left<e_{i+1} \mid v_j \right> - \left<e_{i+1} \mid v_j \right> \|v_j\|^2 \\
			&= 0. \\
		\end{align*}
		On pose $v_{i+1} = \frac{u_{i+1}}{\|u_{i+1}\|}$.
	\end{itemize}
\end{algo}

\begin{exm}
	Avec $E = \R_3[X]$, $\left<P \mid Q \right> = \int_{0}^{1} P(t)\,Q(t)~\mathrm{d}t$ et $\mathcal{B} = (1, X, X^2, X^3)$.
	\begin{enumerate}
		\item $\|1\|^2 = \left<1 \mid 1 \right> = \int_{0}^{1} 1~\mathrm{d}t = 1$ et donc $v_1 = 1$.
		\item $u_2 = X - \left<X  \mid v_1 \right>v_1$. Or, $\left<X \mid v_1 \right> = \int_{0}^{1} t~\mathrm{d}t = \frac{1}{2}$. D'où $u_2 = X - \frac{1}{2}$.
			\begin{align*}
				\|u_2\|^2 &= \int_{0}^{1} \left( t - \frac{1}{2} \right)^2~\mathrm{d}t \\
				&= \int_{0}^{1} \left( t^2 - t + \frac{1}{4} \right)~\mathrm{d}t \\
				&= \frac{1}{3} - \frac{1}{2} + \frac{1}{4} \\
				&= \frac{1}{12} \\
			\end{align*} On en déduit que $v_2 = \sqrt{12}\left( X - \frac{1}{2} \right)$.
		\item $u_3 = X^2 - \left<X^2 \mid v_1 \right>v_1 - \left<X^2 \mid v_2 \right>v_2$.
			On a \[
				\left<X^2 \mid v_1 \right> = \int_{0}^{1} t^2~\mathrm{d}t = \frac{1}{3}
			\] et
			\begin{align*}
				\left<X^2 \mid v_2 \right> &= \sqrt{12} \int_{0}^{1} t^2\left( t - \frac{1}{2} \right)~\mathrm{d}t \\
				&= \frac{\sqrt{12}}{12}. \\
			\end{align*}
			D'où
			\begin{align*}
				u_3 &= X^2 - \frac{1}{3} - \frac{\sqrt{12}}{12}\sqrt{12} \left( X - \frac{1}{2} \right)\\
				&= X^2 - \frac{1}{3} - X + \frac{1}{2} \\
				&= X^2 - X + \frac{1}{6}. \\
			\end{align*}
			\begin{align*}
				\|u_3\|^2 &= \int_{0}^{1} \left( t^2 - t + \frac{1}{6} \right)~\mathrm{d}t\\
				&= \int_{0}^{1} \left( t^4 + t^2 + \frac{1}{36} - 2t^3 + \frac{t^2}{3} - \frac{t}{3} \right) ~\mathrm{d}t \\
				&= \frac{1}{5} + \frac{1}{3} + \frac{1}{36} - \frac{1}{2} + \frac{1}{9} - \frac{1}{6} \\
				&= \frac{36 + 60 + 5 - 90 + 20 - 30}{180} \\
				&= \frac{1}{180} \\
			\end{align*}
			On en déduit que \[
				v_3 = 6\sqrt{5}\left( X^2 - X + \frac{1}{6} \right).
			\]
		\item Exercice : calculer $v_4$.
	\end{enumerate}
\end{exm}

\begin{prop}
	Soit $\mathcal{B} = (e_1, \ldots, e_n)$ une base de $E$ et $\mathcal{C}$ la base obtenue par le procédé d'orthonormalisation de Gram--Schmidt. Alors, \[
		\forall i \in \left\llbracket 1,n \right\rrbracket,\,\Vect(e_1,\ldots, e_i) = \Vect(v_1, \ldots, v_i)
	.\]\qed
\end{prop}

\begin{exm}[orthogonalisation]
	\begin{itemize}
		\item $u_1 = 1$.
		\item
			\begin{align*}
				\begin{rcases*}
					u_2 \in \Vect(e_1, e_2)\\
					u_2 \perp u_1
				\end{rcases*}
				\iff& \begin{cases}
					u_2 = ae_1 + be_2\quad (a,b) \in \R^2\\
					\left<u_1 \mid u_2 \right> = 0
				\end{cases}\\
				\iff& \begin{cases}
					u_2 = a + bX\\
					\int_{0}^{1} (a+bt)~\mathrm{d}t = 0.
				\end{cases}\\
			\end{align*}
			\begin{align*}
				\int_{0}^{1} (a+bt)~\mathrm{d}t = 0 \iff& a + \frac{b}{2} = 0\\
				\iff& a = -\frac{b}{2}\\
				\iff& u_2 = -\frac{b}{2} + bX.
			\end{align*}
			Par exemple, $u_2 = -1 + 2X$.
		\item $\begin{cases}
				u_3 \in \Vect(e_1, e_2, e_3)\\
				u_3 \perp u_1\\
				u_3 \perp u_2
			\end{cases}$

			On pose $u_3 = a + bX + cX^2$ avec $(a,b,c)\in \R^3$.
			\begin{align*}
				\begin{rcases*}
					\int_{0}^{1} \left( a+bt + ct^2 \right)~\mathrm{d}t = 0\\
					\int_{0}^{1} \left(a + bt+ct^2\right)(2t - 1)~\mathrm{d}t = 0
				\end{rcases*} \iff& \begin{cases}
					a + \frac{b}{2} + \frac{c}{3} = 0\\
					\int_{0}^{1} \left( 2ct^3 + (-c + 2b)t^2 + (2a - b)t - a \right) ~\mathrm{d}t = 0
				\end{cases}\\
				\iff& \begin{cases}
					a + \frac{b}{2} + \frac{c}{3} = 0\\
					\frac{c}{2} + \frac{2b - c}{3} + \frac{2\cancel{a} - b}{2} - \cancel{a} = 0
				\end{cases}\\
				\iff&  \begin{cases}
					a = -\frac{b}{2} - \frac{c}{3} = \frac{c}{2} - \frac{c}{3} = \frac{c}{6}\\
					b = -c.
				\end{cases}
			\end{align*}
			On en déduit que \[
				u_3 = 1 - 6X + 6X^2
			.\]
	\end{itemize}
\end{exm}

\begin{crlr}[théorème de la base orthonormée incomplète] Soit $(e_1, \ldots, e_k)$ une base orthonormée d'un espace euclidien. On peut trouver $e_{k+1},\ldots,e_n$ tels que $(e_1, \ldots, e_k, e_{k+1},\ldots,e_n)$ soit une base orthonormée de $E$.
\end{crlr}

\begin{prv}
	On sait que $(e_1, \ldots, e_k)$ est libre. On complète $(e_1, \ldots, e_k)$ en une base $\mathcal{B}$ de $E$. On orthonormalise $\mathcal{B}$ : on obtient une base orthonormée $\mathcal{C}$ de $E$. En détaillant l'algorithme de Gram--Schmidt, on s'aper\c coit que les $k$ premiers vecteurs de $\mathcal{C}$ sont ceux de $\mathcal{B}$.
\end{prv}

\begin{thm}
	Soit $E$ un espace euclidien et $\mathcal{B} = (e_1, \ldots, e_n)$ une base orthonormée de $E$. Soit $(x,y) \in E^2$. On pose $(x_1, \ldots, x_n) \in \R^n$ et $(y_1, \ldots, y_n) \in \R^n$ tels que \[
		x = \sum_{i=1}^n x_i e_i \qquad\qquad y = \sum_{i=1}^n y_i e_i
	.\] Alors \[
		\left<x \mid y \right> = \sum_{i=1}^n x_i y_i
	.\]
	\vspace{3mm}
	Soit $X = \mat{x_1\\\vdots\\x_n}$ et $Y = \mat{y_1\\ \vdots \\ y_n}$. Alors, \[
		\left<x \mid y \right> = X^\T\,Y
	.\]
\end{thm}

\begin{prv}
	\begin{align*}
		\left<x \mid y \right> &= \left<\sum_{i=1}^n x_ie_i  \mid y \right>\\
		&= \sum_{i=1}^n x_i \left<e_i  \mid y \right> \\
		&= \sum_{i=1}^n x_i \left<e_i  \mid \sum_{j=1}^n y_j e_j \right> \\
		&= \sum_{i=1}^n x_i \sum_{j=1}^n y_j \underbrace{\left<e_i \mid e_j \right>}_{\delta_i^j} \\
		&= \sum_{i=1}^n x_i y_i. \\
	\end{align*}
\end{prv}

\begin{prop}
	Soit $E$ un espace euclidien et $\mathcal{B} = (e_1, \ldots, e_n)$ une base orthonormée de $E$. Alors, \[
		\forall x \in E,\,x = \sum_{i=1}^n \left<x \mid e_i \right>e_i
	.\]
\end{prop}

\begin{prv}
	Soit $x \in E$. On pose \[
		x = \sum_{i=1}^n x_i e_i
	\] avec $(x_1, \ldots, x_n) \in \R^n$. Soit $j \in \left\llbracket 1,n \right\rrbracket$. On a
	\begin{align*}
		\left<x \mid e_j \right> &= \left<\sum_{i=1}^n x_i e_i  \mid e_j \right>\\
		&= \sum_{i=1}^n x_i \left<e_i \mid e_j \right> \\
		&= x_j. \\
	\end{align*}
\end{prv}


	\chap[24]{Groupe symétrique}
	\renewcommand{\cwd}{../chap24}
	\part{Modes de définition}

\begin{defn}
	Une suite peut être définie
	\begin{itemize}
		\item \underline{Explicitement}
			On dispose pour tout $n \in \N$ de l'expression de $u_n$ en fonction de $n$.\\
			\ex $\forall n \in \N_*, u_n = \frac{\ln(n)}{n}e^{-n}$\\
		\item \underline{Par récurrence}
			On connait $u_{n+1}$ en fonction de  $u_0, u_1, \ldots, u_n$\\
			\ex $\begin{cases}
				u_0=1\\
				\forall n \in \N, u_{n+1} = \sin(u_n)
			\end{cases}$\\
		\item \underline{Implicitement}
			$\forall n \in \N, u_n$ est le seul nombre verifiant une certaine propriété\\
			\ex $u_n$ est le seul réel vérifiant  $x^5 + nx - 1 = 0$
	\end{itemize}
\end{defn}

	\part{Topologie de $\R^2$}

\begin{defn}
	La \underline{norme (euclidienne)} de $\R^2$ est l'application définie par \[
		\forall (x,y) \in \R^2, \|(x,y)\| = \sqrt{x^2 + y^2}.
	\]

	\begin{figure}[H]
		\centering
		\begin{asy}
			import graph;
			axes(EndArrow);
			size(4cm);
			pair A = (3,2);
			dot(A);
			draw((3,0)--A, dashed);
			draw((0,2)--A, dashed);
			label("$x$", (3,0), align=S);
			label("$y$", (0,2), align=W);
			draw((0,0)--A);
			dot((4,3), white+0);
		\end{asy}
	\end{figure}
	\index{norme (de $\R^2$)}
	\index{norme euclidienne (de $\R^2$)}
\end{defn}

\begin{prop}
	La norme euclidienne vérifie:
	\begin{enumerate}
		\item (séparation) \[
			\forall (x,y) \in \R^2, \|(x,y)\| = 0 \iff x = y = 0,
			\]
		\item (homogénéité positive) \[
				\forall \lambda \in \R, \forall (x,y) \in \R^2, \|\lambda(x,y)\|= \left| \lambda \right| \|(x,y)\|
			\]
		\item (inégalité triangulaire) \[
			\forall (x,y), (a,b) \in \R^2,
			\|(x,y)+(a,b)\|\le \|(x,y)\|+\|(a,b)\|.
		\]
	\end{enumerate}
\end{prop}

\begin{prv}
	Déjà vue en replaçant $(x,y)$ par $x+iy \in \C$ et $\|(x,y)\|$ par |x+iy|
\end{prv}

\begin{defn}
	Soit $(a,b) \in \R^2$ et $r \in \R_+$.

	La \underline{boule ouverte} (ou \underline{disque ouvert}) de centre $(a,b)$ et de rayon $r$ est \[
		B_{(a,b)}(r) = \big\{ (x,y) \in \R^2  \mid \|(x,y) - (a,b)\| < r \big\}.
	\]

	La \underline{boule fermée} (ou \underline{disque fermé}) de centre $(a,b)$ et de rayon $r$ est \[
		\overline{B_{(a,b)}}(r) = \big\{ (x,y)\in \R^2  \mid \|(x,y) - (a,b)\| \le r \big\}.
	\]

	La \underline{sphère} (ou \underline{boule}) de centre $(a,b)$ et de rayon $r$ est \[
		S_{(a,b)}(r) = \partial \overline{B_{(a,b)}}(r) = \big\{ (x,y) \in \R^2  \mid \|(x,y) - (a,b)\| = r \big\}.
	\]
	\index{boule ouverte (de $\R^2$)}
	\index{disque ouverte (de $\R^2$)}
	\index{boule fermée (de $\R^2$)}
	\index{disque fermée (de $\R^2$)}
	\index{boule (de $\R^2$)}
	\index{sphère (de $\R^2$)}
\end{defn}

\begin{figure}[H]
		\centering
		\incfig{boule}
\end{figure}

\begin{rmk}
	On parle de boule en dimension quelconque.
\end{rmk}

\begin{defn}
	Une \underline{partie ouverte} $O$ de $\R^2$ (ou \underline{un ouvert}) si \[
		\forall (x,y) \in O, \exists r > 0, B_{(a,b)}(r) \subset O.
	\]
	Une partie $F$ est \underline{fermée} su $\R^2\setminus F$ est ouverte.
	\index{partie ouverte (de $\R^2$)}
	\index{ouvert (de $\R^2$)}
	\index{partie fermée (de $\R^2$)}
\end{defn}

\begin{figure}[H]
	\centering
	\incfig{partie-ouverte}
\end{figure}

\begin{prop}
	Une boule ouverte est ouverte. Une boule fermée est fermée.
\end{prop}

\begin{figure}[H]
	\centering
	\begin{subfigure}{4cm}
		\centering
		\begin{asy}
			import patterns;

			pair n(pair a) {return a / length(a);}

			add("hatch",hatch(2mm, SW, red));
			size(4cm);

			draw(circle((0,0), 1));
			dot('$(a_0, b_0)$', (0,0), align=S);

			draw((0,0) -- n((-1, 1)), dashed);
			label("$r$", n((-1, 1)) / 2, align=1.5S);

			pair A = n((1,3)) * (2/3);
			real rho = (1 - length(A)) * (2 / 3);

			dot("$(a,b)$", A, red, align=3SE);
			filldraw(circle(A, rho), pattern("hatch"), red);

			label("$O$", n((1,-1))*2.5/3);
		\end{asy}
	\end{subfigure}
	\begin{subfigure}{1cm}
		\centering~\\
	\end{subfigure}
	\begin{subfigure}{5cm}
		\centering
		\begin{asy}
			import patterns;

			pair n(pair a) {return a / length(a);}

			add("hatch",hatch(1mm, SW, red));
			add("hatch2",hatch(3mm, SE, blue));
			size(5cm);

			guide around = (-1.5, -1.5) -- (-1.5, 1.5) -- (2.5, 1.5) -- (2.5, -1.5) -- cycle;

			pair A = n((3, 1)) * 5/3; 
			real rho = 2 / 9;

			picture inter;
			fill(inter, around, pattern("hatch2"));
			fill(inter, circle((0,0), 1), white);
			add(inter);

			draw(circle((0,0), 1));
			dot('$(a_0, b_0)$', (0,0), align=S);

			draw((0,0) -- n((-1, 1)), dashed);
			label("$r$", n((-1, 1)) / 2, align=1.5S);

			dot("$(a,b)$", A, red, align=2SE);
			filldraw(circle(A, rho), pattern("hatch"), red);

			label("$F$", n((1,-1))*2.5/3);
		\end{asy}
	\end{subfigure}
\end{figure}

\begin{prv}
	$\O$ est un ouvert.

	Soit $B$ la boule ouverte de centre $(a_0, b_0) \in \R^2$ et de rayon $r > 0$.

	On pose $\rho = \frac{1}{2}\big(r - \|(a,b) - (a_0,b_0)\|\big)$.
	Montrons que \[
		B_{(a,b)}(\rho) \subset  B_{(a,b)}(r).
	\]

	Soit $(x,y) \in B_{(a,b)}(\rho)$.
	\begin{align*}
		\|(x,y) - (a_0,b_0)\|&= \|(x,y)- (a,b) + (a,b) - (a_0,b_0)\| \\
		&\le \|(x,y) - (a,b)\| + \|(a,b) - (a_0, b_0)\|\\
		&< \rho + \|(a,b) - (a_0, b_0)\| = \frac{1}{2}r + \frac{1}{2} \|(a,b) - (a_0, b_0)\|\\
		&< r
	\end{align*}
	
	Soit $F$ la boule fermée de centre $(a_0, b_0)$ et de rayon $r \ge 0$.

	Soit $(a,b) \not\in F$. On pose \[
		\rho = \frac{1}{2}\big(\|(a,b) - (a_0, b_0)\| - r\big) > 0.
	\]

	Montrons que $B_{(a,b)}(\rho) \subset \R^2\setminus F$.

	Soit $(x,y) \in B_{(a,b)}(\rho)$.

	\begin{align*}
		\|(x,y) - (a_0, b_0)\| &= \|(x,y) - (a,b) + (a,b) - (a_0, b_0)\| \\
		&\ge \big| \underbrace{\|(x,y) - (a,b)\|}_{\le \rho} - \underbrace{\|(a,b) - (a_0, b_0)\|}_{> r} \big|\\
		&\ge \|(a,b) - (a_0, b_0)\|- \|(x,y) - (a,b)\|\\
		&> \|(a,b) - (a_0, b_0)\|- \rho\\
		&> \frac{1}{2} \|(a,b) - (a_0, b_0)\| + \frac{1}{2}r\\
		&> r
	\end{align*}

	donc $(x,y) \not\in F$.
\end{prv}

\begin{exm}
	\begin{enumerate}
		\item $\O$ est ouvert.\\
			$\R^2$ est ouvert.
		\item $\O$ est fermé.\\
			$\R^2$ est fermé.\\
		\item $\big\{(x,0)  \mid x > 0\big\}$ n'est ni ouverte ni fermé.
	\end{enumerate}
\end{exm}

\begin{figure}[H]
	\centering
	\begin{asy}
		size(3cm);

		draw((0, -1) -- (0, 3), Arrow(TeXHead));
		draw((-1, 0) -- (3, 0), Arrow(TeXHead));
		
		draw((0,0) -- (0, 2.97), red);
		draw(circle((0,1.5), 0.5), deepred);
		draw(circle((0,0.5), 0.1), deepred);
	\end{asy}
\end{figure}

\begin{defn}
	Soit $(a,b) \in \R^2$ et $V \in \mathcal{P}(\R^2)$.

	On dit que $V$ est un voisinage de $(a,b)$ s'il existe $r > 0$ tel que \[
		B_{(a,b)}(r) \subset V.
	\]
	\index{voisinage (dans $\R^2$)}
\end{defn}

\begin{prop}
	Un ouvert non vide est un voisinage en chacun de ces points. \qed
\end{prop}

\begin{defn}
	Soit $D \subset \R^2$. Un \underline{point intérieur} de $D$ est un couple $(a,b) \in D$ tel que \[
		\exists r > 0, B_{(a,b)}(r) \subset D.
	\] en d'autres termes, si $D$ est un voisinage de $(a,b)$.

	On note $\mathring D$ l'ensemble des points intérieurs à $D$. C'est \underline{l'intérieur} de $D$.
	\index{point intérieur (dans $\R^2$)}
	\index{intérieur (dans $\R^2$)}
\end{defn}

\begin{prop}
	$\mathring D$ est le plus grand ouvert $O$ de $\R^2$ tel que $O \subset D$.
\end{prop}

\begin{figure}[H]
	\centering
	\incfig{interieur-plus-grand-ouvert}
\end{figure}


\begin{prv}
	Soit $(a,b) \in \mathring D$.

	Par définition, il existe $r > 0$ tel que \[
		B_{(a,b)}(r) \subset D.
	\] Montrons que $B_{(a,b)}(r) \subset \mathring D$.

	Soit $(x,y) \in B_{(a,b)}(r)$. Comme $B_{(a,b)}(r)$ est un ouvert de $\R^2$, il existe $\rho > 0$ tel que \[
		B_{(x,y)}(\rho) \subset B_{(a,b)}(r)
	\] donc $(x,y) \in \mathring D$.

	Donc $\mathring D$ est ouvert, $\mathring D \subset D$.

	Soit $O$ un ouvert de $\R^2$ tel que $O \subset D$. Montrons que $O \subset \mathring D$.

	Soit $(x,y) \in O$. Soit $r > 0$ tel que \[
		B_{(x,y)}(r) \subset O \subset D
	\] donc $(x,y) \in \mathring D$.
\end{prv}

\begin{defn}
	Soit $f: D \subset \R^2 \to \R$, $\ell \in \R$, $(a,b) \in \mathring D$.

	On dit que \underline{$f(x,y)$ tend vers $\ell$ quand $(x,y)$ tend vers $(a,b)$} ou que $\ell$ est \underline{une limite} de $f$ en $(a,b)$ si \[
		\forall \varepsilon > 0, \exists r > 0, \forall (x,y) \in D, \|(x,y) - (a,b)\| < r \implies \left| f(x,y) - \ell \right| \le \varepsilon.
	\] en d'autres termes si \[
		\forall V \in \mathcal{V}_{\ell}, \exists W \in \mathcal{V}_{(a,b)}, \forall (x,y) \in W \cap D, f(x,y) \in V.
	\]
	\index{limite (dans $\R^2$)}
	\index{tendre vers (dans $\R^2$)}
\end{defn}

\begin{prop}
	[unicité de la limite]
	Soit $f: D \to \R$, $(a,b) \in \mathring D$, $\ell_1, \ell_2 \in \R$ telles que $\ell_1$ et $\ell_2$ sont des limites de $f$ en $(a,b)$.

	Alors $\ell_1 = \ell_2$.
\end{prop}

\begin{figure}[H]
	\centering
	\incfig{preuve-unicité-de-la-limite}
\end{figure}

\begin{prv}
	On suppose $\ell_1 < \ell_2$. On pose $\varepsilon = \frac{\ell_2 - \ell_1}{2} > 0$.

	Soit $r_1 > 0$ tel que \[
		f\big(B_{(a,b)}(r_1)\big) \subset ]\ell_1 - \varepsilon, \ell_1 + \varepsilon[.
	\] Soit $r_2 > 0$ tel que \[
		f\big(B_{(a,b)}(r_2)\big) \subset ]\ell_2 - \varepsilon, \ell_2 + \varepsilon [.
	\] On pose $r = \min(r_1, r_2)$ donc \[
		B_{(a,b)}(r_1) \cap B_{(a,b)}(r_2) = B_{(a,b)}(r) \neq \O.
	\] Soit $(x,y) \in B_{(a,b)}(r)$. Alors, \[
		f(x,y) \in ]\ell_1 - \varepsilon, \ell_1 + \varepsilon[ \cap ]\ell_2 - \varepsilon, \ell_2 + \varepsilon[ = \O.
	\] $\lightning$
\end{prv}

\begin{defn}
	Soit $f : D \to \R$, $(a,b) \in \mathring D$.

	On dit que $f$ est \underline{continue} en $(a,b)$ si \[
		f(x,y) \tendsto{(x,y) \to (a,b)}f(a,b).
	\]
	\index{continuité (dans $\R^2$)}
\end{defn}

\begin{prop}
	\underline{Si} $f(x,y) \tendsto{(x,y) \to (a,b)} \ell$ \\
	\underline{alors} $\begin{cases}
		f(x,b) \tendsto{x \to a} \ell\\
		f(a,y) \tendsto{y \to b} \ell.\\
	\end{cases}$
\end{prop}

\begin{prv}~\\
	\begin{figure}[H]
		\centering
		\incfig{limite-x-en-a-et-y-en-b}
	\end{figure}
\end{prv}

\underline{Contre-exemple} : exercice 3.

\begin{exm}
	\begin{enumerate}
		\item $f : \begin{array}{rcl}
				\R^2 &\longrightarrow& \R \\
				(x,y) &\longmapsto& x
			\end{array}$ limite en $(0,0)$ ?

			Soit $\varepsilon > 0$. On pose $r = \varepsilon$. \[
				\forall (x,y) \in B_{(0,0)}(r),
				\left| f(x,y) \right| = \left| x \right| \le \|(x,y)\| < r = \varepsilon
			\] Donc $f(x,y) \tendsto{(x,y) \to (a,b)} 0$.
		\item limite $f : \begin{array}{rcl}
				\R^2 &\longrightarrow& \R \\
				(x,y) &\longmapsto& x^3
			\end{array}$ en $(0,0)$ ?

			Soit $\varepsilon > 0$. On pose $r = \sqrt[3]{r} > 0$. \[
				\forall (x,y) \in B_{(0,0)}(r),
				\left| f(x,y) \right| = \left| x^3 \right| \le \|(x,y)\|^3 < r^3 = \varepsilon.
			\]
		\item limite de $f : \begin{array}{rcl}
			\R^2 &\longrightarrow& \R \\
			(x,y) &\longmapsto& x^3y^2
		\end{array}$ en $(0,0)$ ?

		Soit $\varepsilon > 0$. On pose $r = \sqrt[5]{\varepsilon} > 0$. \[
			\forall (x,y) \in B_{(0,0)}(r), \left| f(x,y) \right| = \left| x^3 y^2 \right| \le \|(x,y)\|^3 \|(x,y)\|^2 < r^5 = \varepsilon.
		\]
	\end{enumerate}
\end{exm}

\begin{defn}
	Soient $D \subset \R^2$ et $(x,y) \in \R^2$.

	\begin{figure}[H]
    \centering
    \incfig{point-adhérent}
	\end{figure}
	
	On dit que $(x,y)$ est \underline{adhérent} à $D$ si \[
		\forall r > 0, B_{(x,y)}(r) \cap D \neq \O.
	\] L'ensemble des points adhérents à $D$ est noté $\overline{D}$. On dit que $\overline{D}$ est \underline{l'adhérence} de $D$.
	\index{point adhérent (dans $\R^2$)}
	\index{adhérent (dans $\R^2$)}
\end{defn}

\begin{defn}
	Soit $f: D \subset \R^2 \to \R$ et $(a,b) \in \overline{D}$, $\ell \in \R$. On dit que $f$ tend vers $\ell$ quand $(x,y)$ tend vers $(a,b)$ si \[
		\forall \varepsilon > 0, \exists r > 0, \forall (x,y) \in B_{(a,b)}(r) \cap D,
		\left| f(x,y) - \ell \right| \le \varepsilon.
	\]
	\index{limite (dans $\R^2$)}
	\index{tendre vers (dans $\R^2$)}
\end{defn}

\begin{prop}
	\begin{enumerate}
		\item Dans ce contexte, il y a unicité de la limite
		\item La limite d'une somme, d'un produit, d'un quotien, d'une composée se comporte comme dans le cas d'une seule variable.
		\item Soit $f: D \to \R$ continue. Soient $g: I \to \R$ et $h: I \to \R$ continues telles que \[
			\forall t \in I, \big(g(t), h(t)\big) \in D.
		\] Alors \[
			t \in I \mapsto f\big(g(t), h(t)\big) \in \R
		\] est continue.
	\end{enumerate}
\end{prop}

\begin{figure}[H]
	\centering
	\begin{asy}
		import three;
		import graph3;
		size(5cm);

		settings.render = 0;
		settings.prc = false;
		currentprojection = obliqueX;

		draw(O -- X, Arrow3(TeXHead2));
		draw(O -- Y, Arrow3(TeXHead2));
		draw(O -- Z, Arrow3(TeXHead2));

		triple f(real x, real y, real z = 0) { return (x,y,cos(x - 0.5) * cos(y - 0.5)/1.2 + 0.15); }

		real inc = 1 / 5;

		for(real x = 0; x <= 1; x += inc) {
			draw(graph(
				new real(real t) { return x; }, // x
				new real(real y) { return y; }, // y
				new real(real y) { return f(x,y).z; }, // z
				0, 1
			), gray);
		}

		for(real y = 0; y <= 1; y += inc) {
			draw(graph(
				new real(real x) { return x; }, // x
				new real(real t) { return y; }, // y
				new real(real x) { return f(x,y).z; }, // z
				0, 1
			), gray);
		}

		path3 path1 = (0.3, 0.2, 0) .. (0.5, 0.5, 0) .. (0.6, 0.7, 0) .. (0.9, 0.8, 0);
		path3 path2 = (0.3, 0.8, 0) .. (0.5, 0.5, 0) .. (0.6, 0.3, 0) .. (0.9, 0.2, 0);
		path3 pathA = f(0.3, 0.2, 0) .. f(0.5, 0.5, 0) .. f(0.6, 0.7, 0) .. f(0.9, 0.8, 0);
		path3 pathB = f(0.3, 0.8, 0) .. f(0.5, 0.5, 0) .. f(0.6, 0.3, 0) .. f(0.9, 0.2, 0);

		draw(path1, red, Arrow3(TeXHead2, position=0.5));
		draw(pathA, red, Arrow3(TeXHead2, position=0.5));
		draw(path2, deepcyan, Arrow3(TeXHead2, position=0.3));
		draw(pathB, deepcyan, Arrow3(TeXHead2, position=0.3));

		dot((0.5, 0.5, 0));
		dot(f(0.5, 0.5, 0));
		draw((0.5, 0.5, 0) -- f(0.5, 0.5, 0), dashed);
	\end{asy}
\end{figure}


	\part{Transpositions}

\begin{defn}
	Une \underline{transposition} est un cycle de longueur 2 : $\begin{pmatrix}
		a&b
	\end{pmatrix}$ avec $a \neq b$.
	\index{transposition (permutation)}
\end{defn}

\begin{exm}
	Avec $n = 5$ et $\gamma = \begin{pmatrix}
		2&4&1
	\end{pmatrix}$.

	\begin{figure}[H]
		\centering

		\begin{asy}
			size(5cm);

			real rho = 0.15; // circles

			void draw_cycle(pair O, real r ...int[] nums) {
				int n = nums.length;
				real eps = (15 / r) * 0.8;

				for(int i = 0; i < n; ++i) {
					real theta_1 = (360/n) * (i+1);
					real theta_2 = (360/n) * i;

					pair C = O + dir(theta_2) * r;

					draw(circle(C, rho));
					label("$" + string(nums[i]) + "$", C);
					draw(arc(O, r, theta_2+eps, theta_1-eps), Arrow(TeXHead));
				}
			}

			draw_cycle((-1,0), 0.8, 1, 2, 4);
			draw_cycle((1,0), 0.3, 3);
			draw_cycle((2,0), 0.3, 5);
		\end{asy}
	\end{figure}

	\[
		\gamma = \begin{pmatrix}
			1&4
		\end{pmatrix} \begin{pmatrix}
			1&2
		\end{pmatrix}
	\]

	Avec $n = 6$ et $\gamma = \begin{pmatrix}
		1&3&5&6&2
	\end{pmatrix} = \begin{pmatrix}
		1&2&3&4&5&6\\
		3&1&5&4&6&2
	\end{pmatrix}$.

	Donc, \[
		\gamma = \begin{pmatrix}
			1&2
		\end{pmatrix} \begin{pmatrix}
			1&6
		\end{pmatrix} \begin{pmatrix}
			1&5
		\end{pmatrix} \begin{pmatrix}
			1&3
		\end{pmatrix}
	\] 
	\[
		\begin{pmatrix}
			1&2&3&4&5&6\\
			3&2&1&4&5&6\\
			3&2&5&4&1&6\\
			3&2&5&4&6&1\\
			3&1&5&4&6&2\\
		\end{pmatrix}
	\]

	Et, \[
		\gamma = \begin{pmatrix}
			1&3
		\end{pmatrix} \begin{pmatrix}
			2&3
		\end{pmatrix} \begin{pmatrix}
			3&5
		\end{pmatrix} \begin{pmatrix}
			5&6
		\end{pmatrix} 
	\]

	\[
		\begin{pmatrix}
			1&2&3&4&5&6\\
			1&2&3&4&6&5\\
			1&2&5&4&6&3\\
			1&3&5&4&6&2\\
			3&1&5&4&6&2\\
		\end{pmatrix} 
	\] 
\end{exm}

\begin{exm}
	\[
		\begin{pmatrix}
			1&4
		\end{pmatrix} = \begin{pmatrix}
			1&2
		\end{pmatrix} \begin{pmatrix}
			2&3
		\end{pmatrix} \begin{pmatrix}
			3&4
		\end{pmatrix} \begin{pmatrix}
			2&3
		\end{pmatrix} \begin{pmatrix}
			1&2
		\end{pmatrix}
	\]
	On n'a pas toujours le même nombre de transpositions mais la parité du nombre reste la même (proposition plus loin).
\end{exm}

\begin{thm}
	Toute permutation se décompose en produit de transpositions.
\end{thm}

\begin{prv}
	Soit $\gamma = \begin{pmatrix}
		a_1&\cdots&a_k
	\end{pmatrix}$ un $k$-cycle.

	On remarque que
	\[
		\gamma = \begin{pmatrix}
			a_1&a_k
		\end{pmatrix} \cdots \begin{pmatrix}
			a_1&a_4
		\end{pmatrix} \begin{pmatrix}
			a_1&a_3
		\end{pmatrix} \begin{pmatrix}
			a_1&a_2
		\end{pmatrix}
	\] C'est un produit de transpositions.
\end{prv}

\begin{exm}
	Avec $n = 10$ et $\sigma = \begin{pmatrix}
		1&2&3&4&5&6&7&8&9&10\\
		9&8&1&7&2&3&4&5&10&6
	\end{pmatrix}$.

	On a
	\begin{align*}
		\sigma &= \begin{pmatrix}
			1&9&10&6&3
		\end{pmatrix} \begin{pmatrix}
			2&8&5
		\end{pmatrix} \begin{pmatrix}
			4&7
		\end{pmatrix}\\
		&= \begin{pmatrix}
			1&3
		\end{pmatrix} \begin{pmatrix}
			1&6
		\end{pmatrix} \begin{pmatrix}
			1&10
		\end{pmatrix} \begin{pmatrix}
			1&9
		\end{pmatrix} \begin{pmatrix}
			2&5
		\end{pmatrix} \begin{pmatrix}
			2&8
		\end{pmatrix} \begin{pmatrix}
			4&7
		\end{pmatrix} \\
	\end{align*}

	Vérification : \[
		\begin{pmatrix}
			1&2&3&4&5&6&7&8&9&10\\
			1&2&3&7&5&6&4&8&9&10\\
			1&8&3&7&5&6&4&2&9&10\\
			1&8&3&7&2&6&4&5&9&10\\
			9&8&3&7&2&6&4&5&1&10\\
			9&8&3&7&2&6&4&5&10&1\\
			9&8&3&7&2&1&4&5&10&6\\
			9&8&1&7&2&3&4&5&10&6\\
		\end{pmatrix} 
	\] 
\end{exm}

	\part{Familles orthogonales}

\begin{thm}[Pythagore]
	Soit $(x,y) \in E^2$. \[
		\|x+y\|^2 = \|x\|^2 + \|y\|^2 \iff x \perp y
	.\]
	\begin{figure}[H]
		\centering
		\begin{asy}
			size(4cm);
			pair u = (1, 0.5);
			pair v = 1.5 * (0, 1) * u;
			draw((0,0)--u, Arrow(TeXHead));
			label("$x$", u/2, align=S);
			draw(u--v+u, Arrow(TeXHead));
			label("$y$", u + v/2, align=NE);
			draw((0,0) -- u + v, Arrow(TeXHead));
			draw(u + v / 7.5 -- u + v / 7.5 - u / 5 -- u - u / 5 -- u -- cycle);
		\end{asy}
	\end{figure}
\end{thm}

\begin{prv}
	\[
		\|x + y\|^2 = \|x\|^2 + \|y\|^2 \iff 2\left<x \mid y \right> = 0 \iff x \perp y
	.\]
\end{prv}

\begin{defn}
	Soit $(e_i)_{i\in I}$ une famille de vecteurs. On dit que cette famille est \underline{orthogonale} si \[
		\forall i \neq j\, e_i \perp e_j
	.\] Si, en plus, on a \[
		\forall i \in I,\,\|e_i\| = 1,
	\] alors on dit que la famille est \underline{orthonormale} ou \underline{orthonormée}.
	\index{famille orthogonale}
	\index{famille orthonormale}
	\index{famille orthonormée}
\end{defn}

\begin{prop}[Pythagore]
	Soit $(e_1, \ldots, e_n)$ une famille orthogonale. Alors \[
		\left\| \sum_{i=1}^n e_i \right\|^2 = \sum_{i=1}^n \|e_i\|^2
	.\]
\end{prop}

\begin{thm}
	Toute famille orthogonale de vecteurs non nuls est libre.
\end{thm}

\begin{prv}
	Soit $(e_i)_{i\in I}$ une famille orthogonale telle que \[
		\forall i \in I,\,e_i \neq 0_E
	.\] Soit $n \in \N^*$, $(\lambda_1, \ldots, \lambda_n) \in \R^n$. On suppose \[
		\sum_{k=1}^n \lambda_k e_{i_k} = 0_E
	.\] Soit $j \in \left\llbracket 1,n \right\rrbracket$.
	\begin{align*}
		0 &= \left<\sum_{k=1}^n \lambda_k e_{i_k}  \mid e_{i_j} \right>\\
		&= \sum_{k=1}^n \lambda_k \left<e_{i_k}  \mid e_{i_j} \right> \\
		&= \lambda_j \underbrace{\|e_{i_j}\|^2}_{\neq 0} \\
	\end{align*}
	donc $\lambda_j = 0$.
\end{prv}

\begin{algo}[Orthonormalisation de Gran--Schmidt]
	On suppose $E$ de dimension finie. Soit $\mathcal{B} = (e_1, \ldots, e_n)$ une base de $E$.

	\begin{itemize}
		\item\underline{\it Étape 1}: On pose $v_1 = \frac{e_1}{\|e_1\|}$ de sorte que $\|v_1\| = 1$.
		\item\underline{\it Étape 2} : On pose \[
				u_2 = e_2 - \left<e_2  \mid v_1 \right> v_1
			.\] Ainsi,
			\begin{align*}
				\left<u_2 \mid v_1 \right> &= \big<e_2 - \left<e_2 \mid v_1 \right> v_1  \mid v_1 \big>\\
				&= \left<e_2 \mid v_1 \right> - \left<e_2 \mid v_1 \right> \left<v_1 \mid v_1 \right> \\
				&= 0. \\
			\end{align*}
			On pose $v_2 = \frac{u_2}{\|u_2\|}$ donc $v_2 \perp v_1$ et $\|v_2\| = 1$.
		\item\underline{\it Étape 3} : On pose \[
				u_2 = e_3 - \left<e_3 \mid v_1 \right>v_1 - \left<e_3 \mid v_2 \right>v_2
			.\] Ainsi,
			\begin{align*}
				\left<u_3  \mid v_1 \right> &= \left<e_3  \mid v_1 \right> - \left<e_3 \mid v_1 \right>\underbrace{\left<v_1 \mid v_1 \right>}_{=1} - \left<e_3 \mid v_2 \right>\underbrace{\left<v_2 \mid v_1 \right>}_{=0} \\
				&= 0 \\
			\end{align*}
			et 
			\begin{align*}
				\left<u_3 \mid v_2 \right> &= \left<e_3  \mid  v_2 \right> - \left<e_3 \mid v_1 \right> \underbrace{\left<v_1 \mid v_2 \right>}_{=0} - \left<e_3 \mid v_2 \right> \underbrace{\left<v_2 \mid v_2 \right>}_{=1}\\
				&= 0. \\
			\end{align*}
			On pose $v_3 = \frac{u_3}{\|u_3\|}$ de sorte que $v_3 \perp v_1$, $v_3 \perp v_2$ et $\|v_3\| = 1$.
		\item\underline{\it Étape $i+1$}: On pose \[
			u_{i+1} = e_{i+1} - \sum_{k=1}^i \left<e_{i+1}  \mid v_k \right> v_k
		.\] Ainsi, pour tout $j \in \left\llbracket 1,i \right\rrbracket,$ on a
		\begin{align*}
			\left<u_{i+1}  \mid v_j \right> &= \left<e_{i+1}  \mid v_j \right> - \sum_{k=1}^i \left<e_{i+1} \mid v_k \right> \left<v_k \mid v_j \right> \\
			&= \left<e_{i+1} \mid v_j \right> - \left<e_{i+1} \mid v_j \right> \|v_j\|^2 \\
			&= 0. \\
		\end{align*}
		On pose $v_{i+1} = \frac{u_{i+1}}{\|u_{i+1}\|}$.
	\end{itemize}
\end{algo}

\begin{exm}
	Avec $E = \R_3[X]$, $\left<P \mid Q \right> = \int_{0}^{1} P(t)\,Q(t)~\mathrm{d}t$ et $\mathcal{B} = (1, X, X^2, X^3)$.
	\begin{enumerate}
		\item $\|1\|^2 = \left<1 \mid 1 \right> = \int_{0}^{1} 1~\mathrm{d}t = 1$ et donc $v_1 = 1$.
		\item $u_2 = X - \left<X  \mid v_1 \right>v_1$. Or, $\left<X \mid v_1 \right> = \int_{0}^{1} t~\mathrm{d}t = \frac{1}{2}$. D'où $u_2 = X - \frac{1}{2}$.
			\begin{align*}
				\|u_2\|^2 &= \int_{0}^{1} \left( t - \frac{1}{2} \right)^2~\mathrm{d}t \\
				&= \int_{0}^{1} \left( t^2 - t + \frac{1}{4} \right)~\mathrm{d}t \\
				&= \frac{1}{3} - \frac{1}{2} + \frac{1}{4} \\
				&= \frac{1}{12} \\
			\end{align*} On en déduit que $v_2 = \sqrt{12}\left( X - \frac{1}{2} \right)$.
		\item $u_3 = X^2 - \left<X^2 \mid v_1 \right>v_1 - \left<X^2 \mid v_2 \right>v_2$.
			On a \[
				\left<X^2 \mid v_1 \right> = \int_{0}^{1} t^2~\mathrm{d}t = \frac{1}{3}
			\] et
			\begin{align*}
				\left<X^2 \mid v_2 \right> &= \sqrt{12} \int_{0}^{1} t^2\left( t - \frac{1}{2} \right)~\mathrm{d}t \\
				&= \frac{\sqrt{12}}{12}. \\
			\end{align*}
			D'où
			\begin{align*}
				u_3 &= X^2 - \frac{1}{3} - \frac{\sqrt{12}}{12}\sqrt{12} \left( X - \frac{1}{2} \right)\\
				&= X^2 - \frac{1}{3} - X + \frac{1}{2} \\
				&= X^2 - X + \frac{1}{6}. \\
			\end{align*}
			\begin{align*}
				\|u_3\|^2 &= \int_{0}^{1} \left( t^2 - t + \frac{1}{6} \right)~\mathrm{d}t\\
				&= \int_{0}^{1} \left( t^4 + t^2 + \frac{1}{36} - 2t^3 + \frac{t^2}{3} - \frac{t}{3} \right) ~\mathrm{d}t \\
				&= \frac{1}{5} + \frac{1}{3} + \frac{1}{36} - \frac{1}{2} + \frac{1}{9} - \frac{1}{6} \\
				&= \frac{36 + 60 + 5 - 90 + 20 - 30}{180} \\
				&= \frac{1}{180} \\
			\end{align*}
			On en déduit que \[
				v_3 = 6\sqrt{5}\left( X^2 - X + \frac{1}{6} \right).
			\]
		\item Exercice : calculer $v_4$.
	\end{enumerate}
\end{exm}

\begin{prop}
	Soit $\mathcal{B} = (e_1, \ldots, e_n)$ une base de $E$ et $\mathcal{C}$ la base obtenue par le procédé d'orthonormalisation de Gram--Schmidt. Alors, \[
		\forall i \in \left\llbracket 1,n \right\rrbracket,\,\Vect(e_1,\ldots, e_i) = \Vect(v_1, \ldots, v_i)
	.\]\qed
\end{prop}

\begin{exm}[orthogonalisation]
	\begin{itemize}
		\item $u_1 = 1$.
		\item
			\begin{align*}
				\begin{rcases*}
					u_2 \in \Vect(e_1, e_2)\\
					u_2 \perp u_1
				\end{rcases*}
				\iff& \begin{cases}
					u_2 = ae_1 + be_2\quad (a,b) \in \R^2\\
					\left<u_1 \mid u_2 \right> = 0
				\end{cases}\\
				\iff& \begin{cases}
					u_2 = a + bX\\
					\int_{0}^{1} (a+bt)~\mathrm{d}t = 0.
				\end{cases}\\
			\end{align*}
			\begin{align*}
				\int_{0}^{1} (a+bt)~\mathrm{d}t = 0 \iff& a + \frac{b}{2} = 0\\
				\iff& a = -\frac{b}{2}\\
				\iff& u_2 = -\frac{b}{2} + bX.
			\end{align*}
			Par exemple, $u_2 = -1 + 2X$.
		\item $\begin{cases}
				u_3 \in \Vect(e_1, e_2, e_3)\\
				u_3 \perp u_1\\
				u_3 \perp u_2
			\end{cases}$

			On pose $u_3 = a + bX + cX^2$ avec $(a,b,c)\in \R^3$.
			\begin{align*}
				\begin{rcases*}
					\int_{0}^{1} \left( a+bt + ct^2 \right)~\mathrm{d}t = 0\\
					\int_{0}^{1} \left(a + bt+ct^2\right)(2t - 1)~\mathrm{d}t = 0
				\end{rcases*} \iff& \begin{cases}
					a + \frac{b}{2} + \frac{c}{3} = 0\\
					\int_{0}^{1} \left( 2ct^3 + (-c + 2b)t^2 + (2a - b)t - a \right) ~\mathrm{d}t = 0
				\end{cases}\\
				\iff& \begin{cases}
					a + \frac{b}{2} + \frac{c}{3} = 0\\
					\frac{c}{2} + \frac{2b - c}{3} + \frac{2\cancel{a} - b}{2} - \cancel{a} = 0
				\end{cases}\\
				\iff&  \begin{cases}
					a = -\frac{b}{2} - \frac{c}{3} = \frac{c}{2} - \frac{c}{3} = \frac{c}{6}\\
					b = -c.
				\end{cases}
			\end{align*}
			On en déduit que \[
				u_3 = 1 - 6X + 6X^2
			.\]
	\end{itemize}
\end{exm}

\begin{crlr}[théorème de la base orthonormée incomplète] Soit $(e_1, \ldots, e_k)$ une base orthonormée d'un espace euclidien. On peut trouver $e_{k+1},\ldots,e_n$ tels que $(e_1, \ldots, e_k, e_{k+1},\ldots,e_n)$ soit une base orthonormée de $E$.
\end{crlr}

\begin{prv}
	On sait que $(e_1, \ldots, e_k)$ est libre. On complète $(e_1, \ldots, e_k)$ en une base $\mathcal{B}$ de $E$. On orthonormalise $\mathcal{B}$ : on obtient une base orthonormée $\mathcal{C}$ de $E$. En détaillant l'algorithme de Gram--Schmidt, on s'aper\c coit que les $k$ premiers vecteurs de $\mathcal{C}$ sont ceux de $\mathcal{B}$.
\end{prv}

\begin{thm}
	Soit $E$ un espace euclidien et $\mathcal{B} = (e_1, \ldots, e_n)$ une base orthonormée de $E$. Soit $(x,y) \in E^2$. On pose $(x_1, \ldots, x_n) \in \R^n$ et $(y_1, \ldots, y_n) \in \R^n$ tels que \[
		x = \sum_{i=1}^n x_i e_i \qquad\qquad y = \sum_{i=1}^n y_i e_i
	.\] Alors \[
		\left<x \mid y \right> = \sum_{i=1}^n x_i y_i
	.\]
	\vspace{3mm}
	Soit $X = \mat{x_1\\\vdots\\x_n}$ et $Y = \mat{y_1\\ \vdots \\ y_n}$. Alors, \[
		\left<x \mid y \right> = X^\T\,Y
	.\]
\end{thm}

\begin{prv}
	\begin{align*}
		\left<x \mid y \right> &= \left<\sum_{i=1}^n x_ie_i  \mid y \right>\\
		&= \sum_{i=1}^n x_i \left<e_i  \mid y \right> \\
		&= \sum_{i=1}^n x_i \left<e_i  \mid \sum_{j=1}^n y_j e_j \right> \\
		&= \sum_{i=1}^n x_i \sum_{j=1}^n y_j \underbrace{\left<e_i \mid e_j \right>}_{\delta_i^j} \\
		&= \sum_{i=1}^n x_i y_i. \\
	\end{align*}
\end{prv}

\begin{prop}
	Soit $E$ un espace euclidien et $\mathcal{B} = (e_1, \ldots, e_n)$ une base orthonormée de $E$. Alors, \[
		\forall x \in E,\,x = \sum_{i=1}^n \left<x \mid e_i \right>e_i
	.\]
\end{prop}

\begin{prv}
	Soit $x \in E$. On pose \[
		x = \sum_{i=1}^n x_i e_i
	\] avec $(x_1, \ldots, x_n) \in \R^n$. Soit $j \in \left\llbracket 1,n \right\rrbracket$. On a
	\begin{align*}
		\left<x \mid e_j \right> &= \left<\sum_{i=1}^n x_i e_i  \mid e_j \right>\\
		&= \sum_{i=1}^n x_i \left<e_i \mid e_j \right> \\
		&= x_j. \\
	\end{align*}
\end{prv}

\end{document}