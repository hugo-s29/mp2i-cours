\part{Définitions}

\begin{defn}
	Soit $\Omega$ un ensemble fini. Une \underline{probabilité} \index{probabilité} sur $\Omega$ est une application \[
		P : \mathcal{P}(\Omega) \longrightarrow [0, 1]
	\] telle que
	\begin{enumerate}
		\item $P(\Omega) = 1$,
		\item $\forall A,B \in \mathcal{P}(W),\; A\cap B = \O \implies P(A\cup B) = P(A) + P(B).$
	\end{enumerate}
	
	Dans ce cas, on dit que $(\Omega, P)$ est un \underline{espace probabilisé}.
	\index{espace probabilisé}
\end{defn}

\begin{exm}[équiprobabilité]
	Soit $\Omega$ un ensemble fini non vide. Soit \begin{align*}
		P: \mathcal{P}(\Omega) &\longrightarrow [0,1] \\
		A &\longmapsto \frac{\#A}{\#\Omega}.
	\end{align*}
	En effet,
	\begin{enumerate}
		\item $P(\Omega) = \frac{\#\Omega}{\#\Omega} = 1$.
		\item Soient $A, B \in \mathcal{P}(\Omega)$ avec $A \cap B = \O$.
			\begin{align*}
				P(A \cup B) &= \frac{\#(A\cup B)}{\#\Omega} \\
				&= \frac{\#A + \#B}{\#\Omega} \\
				&= P(A) + P(B) \\
			\end{align*}
	\end{enumerate}

	De plus, on pose $n = \#\Omega$. \[
		\forall \omega \in \Omega, P\big(\{\omega\}\big) = \frac{1}{n} \text{ et ne dépend pas de } \omega.
	\]
\end{exm}

\begin{defn}
	Soit $(\Omega, P)$ un espace probabilisé.

	L'ensemble $\Omega$ est appelé \underline{univers}\index{univers (probabilités)}, les singletons $\{\omega\}$ avec $\omega \in \Omega$ sont appelés \underline{événements élémentaires}\index{événement élémentaire (probabilités)}, les parties de $\Omega$ sont appelées \underline{événements}\index{événement (probabilités)}, $\O$ est appelé \underline{événement impossible}\index{événement impossible (probabilités)} et $\Omega$ est appelé \underline{événement certain}\index{événement certain (probabilités)}.

	Soient $A,B \in \mathcal{P}(\Omega)$. On dit que $A$ et $B$ sont \underline{incompatibles}\index{événements incompatibles (probabilités)} si $A \cap B = \O$.
\end{defn}

\begin{prop}
	Soit $\Omega = \{\omega_1,\ldots,\omega_n\}$ un ensemble de cardinal $n$ et $(p_1, \ldots, p_n) \in [0,1]^n$ tel que $\sum_{i=1}^n p_i = 1$. Il existe une unique probabilité $P$ sur $\Omega$ que \[
		\forall i \in \left\llbracket 1,n \right\rrbracket,\, P\big(\{\omega_i\} \big) = p_i.
	\]
\end{prop}

\begin{prv}
	\begin{itemize}
		\item[Éxistence] On pose \begin{align*}
				P: \mathcal{P}(\Omega) &\longrightarrow [0,1] \\
				A &\longmapsto \sum_{i \in I_A} p_i
			\end{align*} où $I_A = \big\{i \in \left\llbracket 1,n \right\rrbracket  \mid \omega_i \in A\big\}$.
			\begin{itemize}
				\item Soit $a \in \mathcal{P}(\Omega)$. \[
						0 \le P(A) = \sum_{i \in I_A} p_i \le \sum_{i = 1}^n p_i = 1
					\] 
				\item $P(\Omega) = \sum_{i \in I_\Omega} p_i = \sum_{i=1}^n p_i = 1$.
				\item Soient $A, B \in \mathcal{P}(\Omega)$ incompatibles.
					\begin{align*}
						I_{A\cup B} &= \{ i \in \left\llbracket 1,n \right\rrbracket, \omega_i \in A \cup B\}\\
						&= I_A \cupdot I_B \\
					\end{align*} \[
						P(A \cup B) = \sum_{i \in I_{A\cup B}} p_i = \sum_{i \in I_A} p_i + \sum_{i \in I_B} p_i = P(A) + P(B).
					\]
			\end{itemize}
		\item[Unicité] Soit $Q$ une probabilité sur $\Omega$ tel que \[
				\forall i \in \left\llbracket 1,n \right\rrbracket, Q\big(\{\omega_i\}\big) = p_i.
			\] Soit $A \in \mathcal{P}(\Omega)$. \[
				A = \bigcupdot_{\omega \in A} \{\omega\} = \bigcupdot_{i \in I_A} \{\omega_i\}.
			\] En utilisant le lemme suivant,
			\begin{align*}
				Q(A) &= Q\left( \bigcupdot_{i \in I_A} \{\omega_i\} \right) = \sum_{i \in I_A} Q\big(\{\omega_i\}\big)\\
				&= \sum_{i \in I_A} p_i = P(A). \\
			\end{align*}
	\end{itemize}
\end{prv}

\begin{lem}
	Soit $P$ une probabilité sur $\Omega$ et $(A_i)_{1\le i \le n}$ une famille d'événements 2 à 2 incompatibles. Alors \[
		P\left( \bigcup_{i=1}^n A_i \right) = \sum_{i=1}^n P(A_i).
	\]
\end{lem}

\begin{prv}
	par récurrence sur $n$.
\end{prv}

\begin{prop}
	Soit $P$ une probabilitésur $\Omega$.
	\begin{enumerate}
		\item $P(\O) = 0$;
		\item $\forall A,B \in \mathcal{P}(\Omega),\; A \subset B \implies P(A) \le P(B)$;
		\item $\forall A,B \in \mathcal{P}(\Omega),\; P(A \cup B) = P(A) + P(B) - P(A \cap B)$.
	\end{enumerate}
\end{prop}

\begin{prv}
	\begin{enumerate}
		\item $\Omega = \O \cupdot \Omega$ donc $P(\Omega) = P(\O) + P(\Omega)$ donc $P(\O) = 0$.
		\item Soient $A, B \in \mathcal{P}(\Omega)$ avec $A \subset  B$.
			On pose $C = B \setminus A$ : \[
				\begin{cases}
					 B = A \cup C;\\
					 A \cap C = \O.
				\end{cases}
			\] Donc, $P(B) = P(A) + \underbrace{P(C)}_{\ge 0} \ge P(A)$.
		\item On pose $\begin{cases}
			A' = A \setminus (A \cap B),\\
			B' = B \setminus (A \cap B),\\
			C = A \cap B.
		\end{cases}$ \[
			\begin{cases}
				A' \cup B' \cup C = A \cup B\\
				A' \cap C = B' \cap C = A' \cap B' = \O
			\end{cases}
		\] donc \[
			P(A \cup B) = P(A') + P(B') + P(C).
		\] De plus, \[
			\begin{cases}
				A = A' \cupdot C \text{ donc } P(A) = P(A') + P(C);\\
				B = B' \cupdot C \text{ donc } P(B) = P(B') + P(C).\\
			\end{cases}
		\] D'où
		\begin{align*}
			P(A \cup B) &= P(A) - P(C) + P(B) - \cancel{P(C)} + \cancel{P(C)} \\
			&= P(A) + P(B) - P(A \cap B). \\
		\end{align*}
	\end{enumerate}
\end{prv}

