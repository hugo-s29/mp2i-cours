\part{Anneaux}

\begin{defn}
	Un \underline{anneau} $(A,+,\times)$ est un ensemble $A$ muni de deux lois de compositions \underline{internes} notées $+$ et $\times$ vérifiant
	\begin{enumerate}
		\item $(A,+)$ est un groupe commutatif (son neutre est noté $0_A$)
		\item $(A,\times)$ est un \underline{monoïde}
			\begin{enumerate}
				\item $\times$ est associative
				\item  $\times$ a un neutre $1_A \in A$
			\end{enumerate}
		\item distributivité à gauche et à droite: \[
			 \forall (a,b,c) \in A^3, 
			 \begin{cases}
				 a\times (b + c) = (a\times b) + (a\times c)\\
				 (b+c) \times a = (b \times a) + (c \times a)
			 \end{cases}
		\]
	\end{enumerate}
	\index{monoïde (groupe)}
	\index{anneau}
\end{defn}

\begin{rmk}
	[Convention]

	Soit $(A, +, \times)$ un anneau.\\
	On convient que la multiplication est prioritaire sur l'addition. \[
		(a\times b) + (a\times c) = a\times b + a \times c
	\]
	et l'exponentiation est prioritaire sur la multiplication ($n \in \N$)
	\begin{align*}
		a \times b^n &= a \times (\underbrace{b \times b \times \cdots \times b}_{n \text{ fois}})\\
								 &\neq (a\times b)^n 
	\end{align*}
\end{rmk}

\begin{prop}
	Soit $(A, +, \times)$ un anneau. Alors, $0_A$ est absorbant \[
		\forall  a \in A, a \times 0_A = 0_A \times a = 0_A
	\]
\end{prop}

\begin{prv}
	Soit $a \in A$. On pose $b = a \times 0_A \in A$.
	\begin{align*}
		b = a \times 0_A &= a \times (0_A + 0_A) = a \times 0_A + a \times 0_A \\
		&= b + b~(= 2b) \\
	\end{align*}
	Donc, \[
		-b + b = -b + b + b
	\] donc $0_A = b$\\
	De même, $0_A \times  a = 0_A$.
\end{prv}

\begin{rmk}
	On peut imaginer  $\begin{cases}
		a \times b = 0_A\\
		a \neq 0_A\\
		b \neq 0_A
	\end{cases}$
\end{rmk}

\begin{exm}
	\begin{itemize}
		\item $(\Z / 4\Z, +, \times)$ est un anneau \[
				\begin{cases}
					\overline{2} \times \overline{2} = \overline{0} &\text{ car } 4 \equiv 0~[4]\\
					\overline{2} \neq \overline{0} &\text{ car } 2 \not\equiv 0 ~ [4]
				\end{cases}
			\] 
		\item $(\mathcal{M}_2(\C), +, \times)$ est un anneau (non commutatif)
			\begin{align*}
				A &= \begin{pmatrix} 0&1\\0&0 \end{pmatrix} \neq \begin{pmatrix} 0&0\\0&0 \end{pmatrix} = 0_A \\
				A^2 &= \begin{pmatrix} 0&0\\0&0 \end{pmatrix} \\
			\end{align*}
	\end{itemize}
\end{exm}

\begin{defn}
	On dit qu'un anneau $(A, +, \times)$ est \underline{intègre} si \[
		\forall (a,b) \in A^2, \left( a \times b = 0_A \implies a = 0_A \text{ ou } b = 0_A \right) 
	\]\\
	\index{intègrité (anneau)}
\end{defn}

\begin{exm}
	\begin{itemize}
		\item $(\Z,+,\times)$ est intègre\\
		\item $\forall p$ premier, $(\Z / p\Z, +, \times)$ est intègre (car tout élément non nul de $\Z / p\Z$ est inversible donc simplifiable)
	\end{itemize}
\end{exm}

\begin{exm}
	Soit $(A, +, \times)$ un anneau et $(a,b) \in A^2$.
	\begin{align*}
		(a+b)^2 &= (a+b)\times (a+b) \\
		&= (a+b) \times a + (a+b) \times b \\
		&= a^2 + b\times a + a\times b + b^2 \\
	\end{align*}
	Si $a$ et $b$ commutent, alors, $a\times b = b\times a$ et donc $(a+b)^2 = a^2+b^2+2ab$\\
	\vspace{3mm}
	\begin{align*}
		(a+b)^3 &= (a+b)\times (a+b)\times (a+b)\\
		&= a^3 + a^2\times b  + a \times b \times a + b\times a^2\\
		& + b^2\times a+ b\times a\times b + a\times b^2 + b^3\\
	\end{align*}
	Si $a$ et $b$ commutent, \[
		(a+b)^3 = a^3+3a^2b+3ab^2+b^3
	\] 
\end{exm}

\begin{prop}
	Soient $(A, +, \times)$ un anneau, $(a,b) \in A^2$, $n \in \Z$. Alors, \[
		n(a\times b) = (na)\times b = a\times (nb)
	\]
\end{prop}

\begin{prv}
	\begin{itemize}
		\item \'Evident si $n = 0$
		\item On suppose $n > 0$.
			\begin{align*}
				(n (a\times b) &= \underbrace{a\times b + \cdots + a\times b}_{n \text{ fois}} \\
				&= \sum_{k=1}^{n} (a\times b) \\
				&= a\times \sum_{k=1}^n = a \times (nb) \\
				&= \left( \sum_{k=1}^n a \right) \times b = (na) \times b \\
			\end{align*}
		\item On suppose $n < 0$. On pose $n = -p$ avec $p = \N_*$.
			\begin{align*}
				n(a\times b) &= (-p)(a\times b) = -\left( p(a\times b) \right) \\
				&= -\left( (pa)\times b \right) = (-p)a \times b =  (na)\times b\\
				&= -\left( a\times (pb) \right) = a \times (-pb) =  a\times (nb)\\
			\end{align*}
			En effet,  \[\forall (a',b') \in A^2
				(-a')\times b' + a'\times b' = (-a'+a')\times b' = 0_A \times b' = 0_A
			\] donc $-\left( a'\times b' \right) = \left( -a' \right) \times b'$
	\end{itemize}
\end{prv}

\begin{thm}
	[Formule du binôme de Newton]
	Soient $(A,+,\times)$ un anneau, $(a,b) \in A^2$, $n \in \N$.\\
	Si $a$ et $b$ commutent alors \[
		(a + b)^n = \sum_{k=0}^n {n \choose k} a^k b^{n-k}
	\]
\end{thm}

\begin{prv}
	[par récurrence sur $n$]
\end{prv}

\begin{prop}
	Soient $(A,+,\times)$ un anneau, $(a,b) \in A^2$ et $n \in \N_*$.\\
	Si $a$ et $b$ commutent, alors \[
		a^n - b^n = (a-b) \sum_{k=0}^{n-1} a^kb^{n-1-k}
	\] \qed
\end{prop}

\begin{prop}
	On note $A^\times$ l'ensemble des éléments inversibles d'un anneau $(A,+,\times)$.\\
	$(A^\times,\times)$ est un groupe.
	\qed
\end{prop}

\begin{exm}
	\begin{itemize}
		\item $\Z^\times = \{-1,1\}$ 
		\item $\mathcal{M}_n(\C)^\times = GL_n(\C)$ 
		\item $\left( \Z / 4\Z \right) ^\times = \left\{ \overline{1}, \overline{3} \right\}$
	\end{itemize}
\end{exm}

\begin{defn}
	Soit $(A,+,\times)$ un anneau \underline{commutatif}.
	\begin{enumerate}
		\item Soient $(a,b) \in A^2$. On dit que $a$ \underline{divise} $b$ s'il existe $k \in A$ tel que $b = a\times k$. On dit aussi que $a$ est un \underline{diviseur} de $b$ et que $b$ est un \underline{multiple} de $a$.
		\item On dit que $a$ et $b$ sont \underline{associés} s'il existe $k \in A^\times$ tel que $ak = b$ (dans ce cas, $a \mid b$ et $b \mid a$)
	\end{enumerate}
	\index{division (anneau)}
	\index{multiple (anneau)}
	\index{diviseur (anneau)}
	\index{association (anneau)}
\end{defn}

\begin{rmk}
	Le théorème des deux carrés peut se démontrer en exploitant les propriétés arithmétiques de l'anneau $(Z[i],+,\times)$ où $Z[i] = \{a+ib  \mid a\in \Z, b\in \Z\}$.\\
	$\Z[i]^\times = \{1,-1,i,-i\}$

	\begin{center}
		\underline{Théorème des deux carrés:}
	\end{center}
	\begin{enumerate}
		\item Soit $p$ un nombre premier. \[
				\exists (a,b)\in \N^2, p = a^2+b^2 \iff p \equiv 1 ~[4]
			\]
		\item Soit $n \in \N_*$, $n = \prod_{p \in \mathcal{P}} p^{\alpha(p)}$ \[
				\exists (a,b)\in \N^2, n = a^2 + b^2 \iff\forall p \in \mathcal{P} \text{ tel que }\alpha(p) \neq 0, p \equiv 1~[4]
			\] 
	\end{enumerate}
\end{rmk}

\begin{defn}
	Soit $(A,+,\times)$ un anneau et $B \subset A$. On dit que $B$ est un \underline{sous anneau} de $A$ si
	\begin{enumerate}
		\item $B$ est un sous groupe de $(A,+)$ 
		\item $\forall (a,b) \in B^2, a\times b \in B$
		\item $1_A \in B$
	\end{enumerate}
	\index{sous anneau}
\end{defn}

\begin{exm}
	$Z[i]$ est un sous anneau de $(\C, +, \times)$
\end{exm}

\begin{prop}
	Soit $(A, +, \times)$ un anneau et $B$ un sous anneau de $A$. Alors, $(B,+,\times)$ est un anneau.
	\qed
\end{prop}

\begin{exo}[Exercice à connaître]
	Soit $(A, +, \times)$ un anneau. Le \underline{centre} de $A$ est \[
		Z(A) = \left\{x \in A  \mid \forall a \in A, a\times x = x \times a \right\} 
	\] $Z(A)$ est un sous anneau de $A$.
\end{exo}

\begin{prop}
	Soit $(A,+,\times)$ un anneau.\\
	Si $0_A=1_A$ alors $A = \{0_A\}$. On dit alors que $A$ est l'anneau nul.
\end{prop}

\begin{prv}
	Soit $a \in A$. \[
		a = a \times 1_A = a \times 0_A = 0_A
	\] 
\end{prv}

\newcommand{\red}[1]{{\color{red} #1}}
\newcommand{\blue}[1]{{\color{blue} #1}}

\begin{defn}
	Soient $(A, \blue+, \blue\times)$ et $(B, \red+, \red\times)$ deux anneaux (les lois notés de la même façon mais ne sont pas forcément les mêmes!).\\
	Soit $f: A \to B$. On dit que $f$ est un \underline{(homo)morphisme d'anneaux} si
	\begin{enumerate}
		\item $\forall (a,b) \in A^2, f(a ~\blue+~ b) = f(a) ~\red+~ f(b)$
		\item $\forall (a,b) \in A^2, f(a ~\blue\times~ b) = f(a) ~\red\times~ f(b)$
		\item $f(1_A) = 1_B$
	\end{enumerate}
	\index{morphisme (d'anneaux)}
	\index{homomorphisme (d'anneaux)}
\end{defn}

\begin{prop}
	Avec les notations précédentes, {\bf si}  $a \in A^\times$ alors $f(a) \in B^\times$ et dans ce cas,  \[
		f(a)^{-1} = f\left( a^{-1} \right) 
	\] 
\end{prop}

\begin{prv}
	On suppose $a \in A^\times$. \[
	\begin{cases}
		f\left( a^{-1} \right) ~\red\times~ f(a) = f\left( a^{-1} ~\blue\times~ a \right) = f(1_A) = 1_B\\
		f(a) ~\red\times~ f\left( a^{-1} \right)= f\left(a ~\blue\times~  a^{-1} \right) = f(1_A) = 1_B\\
	\end{cases}
	\]
	Donc, $f(a) \in B^\times$ et $f(a)^{-1}= f\left( a^{-1} \right)$
\end{prv}

\begin{defn}
	Soient $(A,\blue+, \blue\times)$ et $(B, \red+, \red\times)$ deux anneaux et $f: A \to B$ un morphisme d'anneaux.\\
	On dit que $f$ est un
	\begin{itemize}
		\item \underline{isomorphisme d'anneaux} si $f$ est bijective
		\item \underline{endomorphisme d'anneaux} si $\begin{cases}
				A = B\\
				\red+ = \blue+\\
				\red\times = \blue\times 
			\end{cases}$ 
		\item \underline{automorphisme d'anneaux} si $f$ est à la fois un isomorphisme et un endomorphisme d'anneaux
	\end{itemize}
	\index{isomorphisme (d'anneaux)}
	\index{endomorphisme (d'anneaux)}
	\index{automorphisme (d'anneaux)}
\end{defn}

\begin{exm}
	\begin{enumerate}
		\item Soit $a \in \Z$ et \begin{align*}
				f: \Z &\longrightarrow \Z \\
				x &\longmapsto ax
			\end{align*}
			$f$ endomorphisme d'anneaux $\iff a = 1$ 
		\item \begin{align*}
				f: \mathcal{M}_n(\C) &\longrightarrow \mathcal{M}_n(\C) \\
				A &\longmapsto A^2
			\end{align*}
			$f$ n'est pas un morphisme d'anneaux car \[
				(A+B)^2 \neq A^2 + B^2
			\] 
		\item \begin{align*}
				f: \C &\longrightarrow \C \\
				z &\longmapsto \overline{z}
			\end{align*} est un automorphisme d'anneaux
		\item  \begin{align*}
				f: \Z &\longrightarrow \R \\
				x &\longmapsto x
			\end{align*}
			$f$ est un morphisme d'anneaux mais ce n'est pas un endomorphisme.
		\item  \begin{align*}
				f: \Z &\longrightarrow \Z / n\Z \\
				k &\longmapsto \overline{k}
			\end{align*}
			$f$ est un morphisme d'anneaux surjectif.
	\end{enumerate}
\end{exm}

\begin{prop}
	La composée de deux morphismes d'anneaux est un morphisme d'anneaux.
	\qed
\end{prop}

\begin{prop}
	La réciproque d'un isomorphisme d'anneaux est un isomorphisme d'anneaux.
	\qed
\end{prop}

\begin{prop}
	L'ensemble des automorphismes d'anneaux de $A$ est un sous groupe de $(S(A), \circ)$.
	\qed
\end{prop}

\begin{prop}
	L'image directe ou réciproque d'un sous anneau par un morphisme d'anneaux est un sous anneaux.
\end{prop}

\begin{defn}
	Soi $f: A \to B$ un morphisme d'anneaux. Le \underline{noyau} de $f$ est \[
		\Ker(f) = \{a \in A \mid f(a) = 0_B\} 
	\]  
	\index{noyau (d'une application)}
\end{defn}

\begin{prop}
	Avec les notations précédents, \[
		f \text{ injective } \iff \Ker(f) = \{0_A\} 
	\] 
	\qed
\end{prop}

\begin{rmk}
	$\Ker(f)$ n'est pas un sous anneau en général (car $1_A \not\in \Ker(f)$ sauf si $A = \{0_A\}$)
\end{rmk}

\begin{defn}
	Soit $(A,+,\times)$ un anneau et $a \in A\setminus \{0_A\}$.\\
	On dit que $a$ est un \underline{diviseur de zéro} s'il existe $b \in A \setminus \{0_A\} $ tel que $a\times b = b\times a = 0_A$
	\index{diviseur de zéro (anneau)}
\end{defn}

\begin{prop}
	Les diviseurs de zéro ne sont pas inversibles.
	\qed
\end{prop}

\begin{exm}
	$A = \mathcal{M}_2(\C)$ \\
	$M=\begin{pmatrix} 0&1\\0&0 \end{pmatrix}$ est un diviseur de zéro\\
	car $M \times M = \begin{pmatrix} 0&0\\0&0 \end{pmatrix} $
\end{exm}
