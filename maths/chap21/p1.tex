\part{Matrices d'un vecteur}

Soit $E$ un $\mathbbm{K}$-espace vectoriel de dimension finie $n$. Soit $\mathcal{B} = (e_1, \ldots, e_n)$ une base de $E$.\\
\vspace{5mm}

\begin{defn}
	Soit $x \in E$. On sait qu'il existe un unique $n$-uplet $(x_1, \ldots, x_n) \in \mathbbm{K}^n$ tel que \[
		x = \sum_{i=1}^n x_i e_i
	\]
	La \underlin{matrice de $x$ dans la base $\mathcal{B}$} est la colonne \[
		\Mat_\mathcal{B}(x) = \begin{pmatrix}
			x_1\\
			x_2\\
			\vdots\\
			x_n
		\end{pmatrix}.
	\]
\end{defn}

\begin{rmk}
	En général, si  $\mathcal{B}$ et $\mathcal{B}'$ sont 2 bases différentes, alors $\Mat_{\mathcal{B}}(x) \neq \Mat_{\mathcal{B}'}(x)$.
\end{rmk}

\begin{exm}
	$E = \R_3[X]$, $\mathbbm{K} = \R$, $\mathcal{B} = (1, X, X^2, X^3)$ et \\
	\begin{align*}
		\mathcal{C} &= \left( \frac{X(X-1)(X-2)}{6}, -\frac{X(X-1)(X-3)}{2},\right.\\
			&\left.\frac{X(X-2)(X-3)}{2}, -\frac{(X-1)(X-2)(X-3)}{6} \right)
	\end{align*}
	
	\[
		P = X^2-X+1
	\]

	\[
		\Mat_\mathcal{B}(P) = \begin{pmatrix}
			1\\
			-1\\
			1\\
			0
		\end{pmatrix}
	\]
	\begin{align*}
		P &= P(3) \frac{X(X-1)(X-2)}{6} + P(2) \frac{-X(X-1)(X-3)}{2}\\
			&+ P(1) \frac{X(X-2)(X-3)}{2} + P(0) \frac{-(X-1)(X-2)(X-3)}{6}
	\end{align*}

	\[
		\Mat_{\mathcal{B}'}(P) = \begin{pmatrix}
			7\\
			3\\
			1\\
			1\\
		\end{pmatrix}
	\]
\end{exm}
