\part{Exercice 4}

\[
	\forall P \in \R_n[X],
	\int_{0}^{1} P(t)\,dt = \sum_{i=0}^n \lambda_i P(a_i)
\]
On pose \begin{align*}
	f: \R_n[X] &\longrightarrow \R \\
	 P &\longmapsto \int_{0}^{1} P(t)\, dt.
\end{align*}

\[
	\forall P \in \R_n[X],\, f(P) = \sum_{i=0}^n \lambda_i P(a_i)
\]

On pose \begin{align*}
	g_i: \R_n[X] &\longrightarrow \R \\
	P &\longmapsto P(a_i).
\end{align*}


On doit montrer : $\boxed{\forall P \in \R_n[X], f(P) = \sum_{i=0}^n \lambda_i\,g_i(P).}$
\\[3mm]

$\mathcal{B} = (g_0, \ldots, g_n)$ est une base de $\R_n[X]^*$. Soit $\mathcal{C} = (1, X, \ldots, X^n)$ la base canonique de $\mathbbm{K}_n[X]$.

On a $g_1(1) = 1$, $g_1(X) = a_1$, $\ldots$, $g_1(X^n) = {a_1}^n$ 

Donc, \[
	A = \begin{pmatrix}
		1&1&\cdots&1\\
		a_0&a_1&\cdots&a_n\\
		\vdots&\vdots&\ddots&\vdots\\
		{a_0}^n&{a_1}^n&\cdots&{a_n}^n
	\end{pmatrix}
\] 

$\t A = \Mat_\mathcal{C}(D)$ où $D$ est la base des polynômes interpolateurs de Lagrange.

Donc $\t A$ inversible et donc $A$ inversible.
