\part{Suites complexes}

\begin{defn}
	Soit $(u_n) \in \C^\N$ et $\ell\in \C$. On dit que $(u_n)$ converge vers $\ell$ si \[
		\forall \varepsilon>0, \exists N \in \N, \forall n \ge  N, \left| u_n-\ell \right| \le \varepsilon
	\]
	\index{limite finie (suites complexes)}
	\index{convergence (suites complexes)}
	
	\begin{figure}[H]
		\begin{center}
			\begin{asy}
				import graph;

				size(8cm); srand(0);

				axes("$\Re$", "$\Im$", EndArrow);

				draw(circle((3,2), 1), dashed);

				dot("$\ell$", (3,2), N+E);

				draw(brace((4,2), (3,2), .1));
				label("$\varepsilon$", (3.5, 2), 2 * S);

				for(int i = 0; i < 15; ++i) {
					real x = cos(i * 360 / 7) * exp(-i / 3) * 4 + 3;
					real y = sin(i * 360 / 7) * exp(-i / 3) * 4.2 + 2;

					dot((x,y), magenta);
				}

			\end{asy}
		\end{center}
		\caption{Suite complexe convergente}
	 	\label{suite-complexe-convergente}
	\end{figure}
\end{defn}

\begin{prop}
	Si $\ell_1$ et $\ell_2$ sont deux limites de $u$ alors $\ell_1= \ell_2$ 
	\begin{figure}[H]
		\begin{center}
			\begin{asy}
				import graph;

				size(4cm); srand(0);

				axes("$\Re$", "$\Im$", EndArrow);

				draw(circle((1,3), 0.5), dashed);
				dot("$\ell_1$", (1,3), N+E);

				draw(circle((3,1), 0.5), dashed);
				dot("$\ell_2$", (3,1), N+E);
				
				draw((1,3)--(3,1));
			\end{asy}
		\end{center}
		\caption{Unicité de la limite de suites complexes}
	 	\label{unicite-limite-complexe}
	\end{figure}
	\qed
\end{prop}

\begin{prop}
	Les limites de somme, produit, quotient de suites complexes respectent les mêmes lois que pour les suites réelles.
	\qed
\end{prop}

\begin{thm}
	Soit $u \in \C^\N$ et $\ell\in \C$. \[
	u_n \tendsto{n \to +\infty} \ell
	\iff
	\begin{cases}
		\Re(u_n) \tendsto{n \to  +\infty} \Re(\ell)\\
		\Im(u_n) \tendsto{n \to  +\infty} \Im(\ell)\\
	\end{cases}
	\] 
\end{thm}

\begin{prv}
	\begin{itemize}
		\item[$\implies$] On suppose $u_n \tendsto{} \ell$.\\
			Soit $\varepsilon >0$. Soit $N \in \N$ tel que \[
			\forall  n \ge  N, \left| u_n-\ell \right| \le  \varepsilon
			\] Or, \[
			\forall  n \ge  N, \begin{cases}
				\Re(u_n) - \Re(\ell) = \Re(u_n - \ell) \le \left| u_n-\ell \right| \le \varepsilon\\
				\Im(u_n) - \Im(\ell) = \Im(u_n - \ell) \le \left| u_n-\ell \right| \le \varepsilon
			\end{cases}
			\] donc \[
			\begin{cases}
				\Re(u_n) \tendsto{} \Re(\ell)\\
				\Im(u_n) \tendsto{} \Im(\ell)\\
			\end{cases}
			\] 
		\item[$\impliedby$] On suppose $\begin{cases}
			\Re(u_n) \tendsto{} \Re(\ell)\\
			\Im(u_n) \tendsto{} \Im(\ell)\\
		\end{cases}$\\
		Alors, \[
		\forall n \in \N, u_n = \Re(u_n) + i \Im(u_n) \tendsto{} \Re(\ell) + i\Im(\ell) = \ell
		\] 
	\end{itemize}
\end{prv}

\begin{prop}
	Soit $u \in \C^\N$ et $\ell\in \C$.\\
	Si $u_n \tendsto{} \ell$ alors $\left| u_n \right| \tendsto{} \left| \ell \right|$
\end{prop}

\begin{prv}
	On suppose $u_n \tendsto{} \ell$\\
	\[
	\forall n \in \N, \left| u_n \right| = \sqrt{\Re^2(u_n) + \Im^2(u_n)} \tendsto{} \sqrt{\Re^2(\ell) + \Im^2(\ell)}  = \left| \ell \right| 
	\] 
\end{prv}

\begin{prop}
	Tous les résultats (sauf ceux avec des limites infinies !) concernant les suites extraites sont encore valables dans $\C$ y compris le théorème de Bolzano-Weierstrass (mais avec une autre preuve).
\end{prop}

\begin{defn}
	Soit $u \in \C^\N$. On dit que $u$ est \underline{bornée} s'il existe $M \in \R^+$ tel que \[
	\forall n \in \N, \left| u_n \right| \le M
	\]
	\index{bornée (suite complexe)}

	\begin{figure}[H]
		\begin{center}
			\begin{asy}
				import graph;

				size(8cm); srand(0);

				axes("$\Re$", "$\Im$", EndArrow);

				draw(circle((0,0), 1), dashed);

				real k = 0.02;

				for(int i = 0; i < 10; ++i) {
					real r = unitrand();
					real a = unitrand() * pi * 2;

					pair coords = (r * cos(a), r * sin(a));
					draw(coords - (k,0) -- coords -- coords + (k,0));
					draw(coords - (0,k) -- coords -- coords + (0,k));
				}

			\end{asy}
		\end{center}
		\caption{Suite complexe bornée}
	 	\label{suite-complexe-bornée}
	\end{figure}
\end{defn}

\begin{thm}
	[Bolzano Weierstrass]
	Soit $u \in \C^\N$ bornée. Il existe $\varphi: \N \to \N$ strictement croissante telle que $\left( u_{\varphi(n)} \right) $ converge.
\end{thm}

\begin{prv}
	Soit $M \in \R^+$ tel que \[
	\forall n \in \N, \left| u_n \right| \le M
	\] Donc, $\forall n \in \N, \left| \Re(u_n) \right| \le \left| u_n \right| \le M$ Donc $\left( \Re(u_n) \right) _{n \in \N}$ est bornée. Donc, il existe $\varphi: \N \to \N$ strictement croissante telle que $\left( \Re\left( u_{\varphi(n)} \right)  \right) $ converge.
	\[
	\forall n \in \N, \left| \Im \left( u_{\varphi(n)} \right)  \right| \le  \left| u_{\varphi(n)} \right| \le M
	\] donc $\left( \Im(u_{\varphi(n)} \right)$ est bornée. Soit $\psi: \N\to\N$ strictement croissante telle que $\left( \Im\left( u_{\varphi(\psi(n))} \right)  \right) $ converge.
	Or, $\left( \Re\left( u_{\varphi(\psi(n))} \right)  \right) $ est une sous suite de la suite convergente $\left( \Re\left( u_{\varphi(n)} \right)  \right) $ donc $\left( \Re\left( u_{\varphi(\psi(n))} \right)  \right) $ converge. \\
	Donc, $\left( u_{\varphi(\psi(n))} \right) $ converge.\\
	Comme  $\varphi \circ \psi$ est strictement croissante, $\left( u_{\varphi(\psi(n))} \right) $ est une sous suite de $(u_n)$
\end{prv}

