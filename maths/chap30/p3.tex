\part{Propriétés de l'intégrale}

% Chasles, inégalité triangulaire, croissance, linéarité

\begin{prop}
	Soient $f$ et $g$ deux fonctions définies de $[a,b]$ à valeurs dans $\R$.
	Si $f \le g$, alors \[
		I^-_{[a,b]}(f) \le I^-_{[a,b]}(g)
	.\]
\end{prop}

\begin{prv}
	On suppose $f \le g$.
	\begin{align*}
		I^-(f) \le I^-(g) \iff& \sup_{\sigma \in \mathfrak{S}} S_\sigma^-(f) \le \sup_{\sigma \in \mathfrak{S}} S^-_{\sigma}(g)\\
		\iff& \forall \sigma \in \mathfrak{S},\,S^-_{\sigma}(f) \le \sup_{\sigma' \in \mathcal{S}}I^-_{\sigma'}(g)\\
	\end{align*}

	Soit $\sigma \in \mathfrak{S}$. On pose $\sigma = (x_0, x_1, \ldots, x_n)$ et, pour tout $i \in \left\llbracket 0,n-1 \right\rrbracket$, $m_i = \inf_{]x_i,x_{i+1}[}f)$. Alors, \[
		S_\sigma^-(f) = \sum_{i=1}^{n-1}(x_{i+1}-x_i) m_i
	.\] On pose aussi, pour tout $i \in \left\llbracket 0,n-1 \right\rrbracket$, $m_i' = \inf_{]x_i,x_{i+1}[}(g)$.

	Soit $i \in \left\llbracket 0,n-1 \right\rrbracket$.
	\begin{align*}
		m_i \le m_i' \iff& \inf_{]x_i,x_{i+1}[}(f) \le \inf_{]x_i,x_{i+1}[}(g)\\
		\iff& \forall x \in \;]x_i,x_{i+1}[,\,\inf_{]x_i,x_{i+1}[}(f) \le g(x)
	\end{align*}
	Soit $x \in \;]x_i,x_{i+1}[$. On sait que \[
		\inf_{]x_i,x_{i+1}[}(f) \le f(g) \le g(x)
	.\] On en déduit que \[
	S^-_\sigma(f) \le \sum_{i=0}^{n-1} (x_i - x_{i+1})m_i' = S_\sigma^-(g) \le \sup_{\sigma' \in \mathfrak{S}} S_{\sigma'}^-(g)
	.\]
\end{prv}

\begin{exo}
	Démontrer le même résultat avec les sommes de Darboux supérieures : \[
		I^+_{[a,b]}(f) \le I^+_{[a,b]}(g)
	.\]
\end{exo}

\begin{crlr}
	Soient $f$ et $g$ deux fonctions intégrables définies sur $[a,b]$ à valeurs dans $\R$.
	Si $f \le g$, alors \[
		\int_{[a,b]} f \le \int_{[a,b]} g
	.\] \qed
\end{crlr}

\begin{prop}[Chasles]
	Soit $f$ une fonction définie sur $[a,b]$ à valeurs dans $\R$.
	Soit $c \in\; ]a,b[$. Alors, \[
		\begin{cases}
			I_{[a,b]}^-(f) = I_{[a,c]}^-(f) + I_{[c,b]}^-(f)\\
			I_{[a,b]}^+(f) = I_{[a,c]}^+(f) + I_{[c,b]}^+(f).
		\end{cases}
	\]
\end{prop}

\begin{prv}~
	\begin{itemize}
		\item $I_{[a,b]}^-(f) \le I_{[a,c]}^-(f) + I_{[c,b]}^-(f) \iff \forall \sigma \in \mathfrak{S}_{[a,b]},\,S_{\sigma}^-(f) \le I_{[a,c]}^-(f) + I_{[c,b]}^-(f).$

			Soit $\sigma \in \mathfrak{S}_{[a,b]}$. On pose $\sigma = (x_0, x_1, \ldots, x_n)$. On note $k \in \left\llbracket 0,n-1 \right\rrbracket$ tel que $x_k \le c \le x_{k+1}$.

			On pose aussi $\sigma_1 = (x_0, x_1, \ldots, x_k, c) \in \mathfrak{S}_{[a,c]}$ et $\sigma_2 = (c, x_{k+1}, \ldots, x_n) \in \mathfrak{S}_{[c,b]}$.

			$\sigma_1 \cup \sigma_2$ est une subdivision plus fine que $\sigma$. On en déduit que \[
				S_\sigma^-(f) \le S_{\sigma_1 \cup \sigma_2}^-(f) = \underbrace{S_{\sigma_1}^-(f)}_{\substack{\vrt\le\\I^-_{[a,c]}(f)}} + \underbrace{S^-_{\sigma_2}(f)}_{\substack{\vrt\le\\I^-_{[c,b]}(f)}}
			.\]
		\item
			\begin{align*}
				I_{[a,b]}^-(f) \ge I^-_{[a,c]}(f) + I^-_{[c,b]}(f)
				\iff& I^-_{[a,c]}(f) \ge I^-_{[a,b]}(f) - I^-_{[c,b]}(f)\\
				\iff&\forall \sigma_1 \in \mathfrak{S}_{[a,c]},\,S^-_{\sigma_1} \le I^-_{[a,b]}(f) - I^-_{[c,b]}(f)\\
				\iff& \forall \sigma_1 \in \mathfrak{S}_{[a,c]},\,I^-_{[c,b]}(f) \le I^-_{[a,b]}(f) - S^-_{\sigma_1}(f)\\
				\iff& \forall \sigma_1 \in \mathfrak{S}_{[a,c]},\,\forall \sigma_2 \in \mathfrak{S}_{[c,b]},\,S_{\sigma_2}^-(f) \le I^-_{[a,b]}(f) - S^-_{\sigma_1}(f).
			\end{align*}
			Soient $\sigma \in \mathcal{S}_{[a,c]}$ et $\sigma_2 \in \mathfrak{S}_{[c,b]}$. \[
				S_{\sigma_1}^-(f) + S_{\sigma_2}^-(f)  = S_{\sigma_1 \cup \sigma_2}^-(f) \le I^-_{[a,b]}(f)
			.\]
	\end{itemize}
\end{prv}

\begin{crlr}
	Soit $f$ une fonction définie sur $[a,b]$ à valeurs dans $\R$.
	Soit $c \in\; ]a,b[$. Alors, \[
		\int_{[a,b]}f = \int_{[a,c]}f + \int_{[c,b]}f
	.\]\qed
\end{crlr}

\begin{prop}
	Soient $f$ et $g$ deux fonctions intégrables définies sur $[a,b]$ à valeurs dans $\R$.
	Alors, \[
		\begin{cases}
			I^-_{[a,b]}(f+g) \ge I^-_{[a,b]}(f) + I^-_{[a,b]}(f)\\
			I^+_{[a,b]}(f+g) \le I^+_{[a,b]}(f) + I^+_{[a,b]}(f)
		\end{cases}
	\]
\end{prop}

\begin{prv}
	\begin{align*}
		I^-(f+g) \ge I^-(f) + I^-(g)
		\iff& I^-(f) \le I^-(f+g) - I^-(g)\\
		\iff& \forall \sigma \in \mathfrak{S}_{[a,b]},S_{\sigma}^-(f) \le I^-(f+g) - I^-(g)\\
		\iff& \forall \sigma \in \mathfrak{S},\,I^-(g) \le I^-(f+g) - S^-_\sigma(f)\\
		\iff& \forall \sigma \in \mathfrak{S},\forall \sigma' \in \mathfrak{S},\,S_{\sigma'}^-(g) \le I^-(f+g) - I^-_\sigma(f)\\
		\iff& \forall \sigma \in \mathfrak{S},\forall \sigma' \in \mathfrak{S},S_{\sigma}^-(f) + S^-_{\sigma'}(g) \le I^-(f+g)
	\end{align*}
	Soit $\sigma, \sigma' \in \mathfrak{S}$. On considère $\sigma'' \in \mathfrak{S}$ telle que $\sigma'' \prec \sigma'$ et $\sigma'' \prec \sigma$. On a \[
		\begin{cases}
			S_\sigma^-(f) \ge S_{\sigma''}^-(f)\\
			S^-_{\sigma'}(g) \le S^-_{\sigma''}(g)
		\end{cases}
	\] donc \[
		S^-_{\sigma}(f) + S_{\sigma'}^-(g) \le S_{\sigma''}^-(f) + S_{\sigma''}^-(g)
	.\] On pose $\sigma'' = (x_0, \ldots, x_n)$ et, pour $i \in \left\llbracket 0,n-1 \right\rrbracket,$ \[
		\begin{cases}
			m_i(f) = \inf_{]x_i,x_{i+1}[}(f)\\
			m_i(g) = \inf_{]x_i,x_{i+1}[}(g)\\
			m_i(f+g) = \inf_{]x_i,x_{i+1}[}(f+g).
		\end{cases}
	\]
	Alors,
	\begin{align*}
		S^-_{\sigma''}(f) + S^-_{\sigma''}(g) &= \sum_{i=0}^{n-1}(x_{i+1}- x_i)m_i(f) + \sum_{i=0}^{n-1}(x_{i+1}-x_i)m_i(g) \\
		&= \sum_{i=0}^{n-1}(x_{i+1} - x_i)\big(m_i(f) + m_i(g)\big). \\
	\end{align*}

	Soit $i \in \left\llbracket 0,n-1 \right\rrbracket$.
	\begin{align*}
		m_i(f) + m_i(g) \le m_i(f+g)
		\iff&\inf_{]x_i,x_{i+1}[}(f) + \inf_{]x_i,x_{i+1}[}(g) \le \inf_{]x_i,x_{i+1}[}(f+g)\\
		\iff& \forall x \in\; ]x_i,x_{i+1}[,\inf_{]x_i,x_{i+1}[}(f) + \inf_{]x_i,x_{i+1}[}(g) \le (f+g)(x)
	\end{align*}
	Soit $x \in\;]x_i,x_{i+1}[$.
	\begin{align*}
		(f+g)(x) &= f(x) + g(x) \\
		&\ge \inf_{]x_i,x_{i+1}[}(f) + \inf_{]x_i,x_{i+1}[}(g).
	\end{align*}
	D'où
	\begin{align*}
		S^-_{\sigma''}(f) + S^-_{\sigma''}(g) \le& \sum_{i=0}^{n-1}(x_{i+1}-x_i)m_i(f+g)\\
		\le& S^-_{\sigma''}(f+g) \le I^-(f+g).
	\end{align*}
\end{prv}

\begin{prop}
	Soit $\lambda \in \R$. On a \[
		I^-_{[a,b]}(\lambda f) = \begin{cases}
			\lambda I^-_{[a,b]}(f) &\text{ si } \lambda \ge 0\\
			\lambda I^+_{[a,b]}(f) &\text{ si } \lambda \le 0\\
		\end{cases}
	\] et \[
		I^+_{[a,b]}(\lambda f) = \begin{cases}
			\lambda I^+_{[a,b]}(f) &\text{ si } \lambda \ge 0\\
			\lambda I^-_{[a,b]}(f) &\text{ si } \lambda \le 0\\
		\end{cases}
	.\]
\end{prop}

\begin{prv}
	\begin{itemize}
		\item On suppose $\lambda > 0$.
			\[
				I^-(\lambda f) \le \lambda I^-(f) \iff
				\forall \sigma \in \mathfrak{S},\,S_\sigma^-(\lambda f) \le \lambda I^-(f)
			.\]
			Soit $\sigma = (x_0, \ldots, x_n) \in \mathfrak{S}$. On pose, pour $i \in \left\llbracket 0,n-1 \right\rrbracket$, $m_i(f) = \inf_{]x_i,x_{i+1}[}(f)$ et $m_i(\lambda f) = \inf_{]x_i,x_{i+1}[}(\lambda f)$.

			Soit $i \in \left\llbracket 0,n-1 \right\rrbracket$.
			\begin{align*}
				m_i(\lambda f) \le \lambda m_i(f) \iff& m_i(f) \ge m_i(f) \ge \frac{1}{\lambda}m_i(\lambda f)\\
				\iff& \forall x \in\;]x_i,x_{i+1}[,\,\frac{1}{\lambda} m_i(\lambda f) \le f(x)
			\end{align*}
			Soit $x \in \,]x_i,x_{i+1}[$. On a \[
				f(x) = \frac{1}{\lambda}\big(\lambda f(x)\big) \ge \frac{1}{\lambda}m_i(\lambda f)
			.\] Donc,
			\begin{align*}
				S^-_{\sigma}(\lambda f) &= \sum_{i=0}^{n-1}(x_{i+1}-x_i) m_i(\lambda f) \\
				&\le \sum_{i=0}^{n-1}\lambda(x_{i+1}-x_i)m_i(f)\\
				&\le \lambda S^-_\sigma(f)\\
				&\le \lambda I^-(f).
			\end{align*}
			En outre,
			\begin{align*}
				I^-(f) &= I^- \left( \frac{1}{\lambda}\,\lambda f \right) \\
				&\le \frac{1}{\lambda}I^-(\lambda f)
			\end{align*}
			donc \[
				\lambda I^-(f) \le I^-(\lambda f)
			.\]
		\item On suppose $\lambda < 0$.
			\[
				I^-(\lambda f) \le \lambda I^-(f) \iff
				\forall \sigma \in \mathfrak{S},\,S^-_\sigma(\lambda f) \le \lambda I^+(f)
			.\]

			Soit $\sigma \in \mathfrak{S}$. On pose $\sigma = (x_0, \ldots, x_n)$. Alors, \[
				S^-_{\sigma}(\lambda f) = \sum_{i=0}^{n-1}(x_{i+1}-x_i) m_i(\lambda f)
			.\] Soit $i \in \left\llbracket 0,n-1 \right\rrbracket$. Soit $x \in \,]x_i,x_{i+1}[$.
%
			\begin{align*}
				~&f(x) = \frac{1}{\lambda}\big(\underbrace{\lambda f(x)}_{\ge \inf(\lambda f)}\big) \le \frac{1}{\lambda} m_i(\lambda f)\\
				\text{donc }& \frac{1}{\lambda}m_i(\lambda f) \text{ majore } f \text{ sur } ]x_i,x_{i+1}[\\
				\text{ donc }& \frac{1}{\lambda}m_i(\lambda f) \ge \sup_{]x_i,x_{i+1}[}(f) = M_i(f)\\
				\text{ et donc }& m_i(\lambda f) \le \lambda M_i(f).
			\end{align*}

			D'où
			\begin{align*}
				S^-_{\sigma}(\lambda f) &\le \sum_{i=0}^{n-1}\lambda (x_{i+1}-x_i) M_i(f)\\
				&\le \lambda S^+_\sigma(f)\\
				&\le \lambda I^+(f)
			\end{align*}
			car $\lambda < 0$ et $I^+(f) \le S_\sigma^+(f)$.

			De plus, \[
				I^-\left( \frac{1}{\lambda}\,\lambda f \right) \le \frac{1}{\lambda} I^+(\lambda f)
			\] donc \[
				\lambda I^-(f) \ge I^+(\lambda f)
			.\]
			\vspace{1cm}

			Il reste à prouver (pour $\lambda < 0$) que \[
				\begin{cases}
					I^-(\lambda f) \ge \lambda I^+(f)\\
					I^+(\lambda f) \ge \lambda I^-(f).
				\end{cases}
			\]
	\end{itemize}
\end{prv}

