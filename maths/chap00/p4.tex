\part{Prédicat}

\begin{defn}
	Un \underline{prédicat} $P(x)$ est un énoncé dont la valeur de vérité dépend de l'objet $x$, élément d'un ensemble $E$.\\
	Le \underline{domaine de validité} de $P$ est l'ensemble des valeurs $x$ de $E$ pour lequelles $P(x)$ est vraie: \[
		\{x \in E  \mid P(x)\} 
	\]
	\index{prédicat (logique)}\index{domaine de validité (prédicat, logique)}
\end{defn}

\begin{rmk}
	[Notation]
	On écrit \[
		\forall x \in E, P(x)
	\] pour dire que $P(x)$ est vraie pour tous les $x$ de $E$.\\

	On écrit \[
		\exists x \in E, P(x)
	\] pour dire qu'il existe (au moints) un élément $x \in E$ pour lequels $P(x)$ est vraie.\\

	On écrit \[
		\exists! x \in E, P(x)
	\] pour dire qu'il existe un \underline{unique} élément $x \in E$ tel que $P(x)$ est vraie.
\end{rmk}

\vspace{5mm}
~~\underlin{Comment démontrer $\forall x \in E, P(x)$ ?}
\marginpar{\fbox{$\forall x \in E, P(x)$}}\\
Soit $x \in E$ (fixé quelconque). Montrons $P(x)$.\\

\vspace{2mm}
\underlin{Comment utiliser $\forall x \in E, P(x)$ ?}\\
On choisit (spécialise) une ou plusieurs (voir toutes) valeurs de $x$ et on exploite $P(x)$.\\

\vspace{8mm}

\begin{exm}
	Soient $a,b,c \in \R$.
	On suppose que \[
		\forall n \in \N, a + b \times 2^n + c \times 3^n
	\] Montrons que $a = b= c = 0$.\\
	On sait que $(S): \begin{cases}
		a+b+c=0 &~~(n=0)\\
		a+2b+3c=0 &~~(n=1)\\
		a+4b+9c=0 &~~(n=2)\\
	\end{cases}$\\
	\[
		(S) \iff
		\begin{cases}
			a + b + c = 0\\
			b+2c = 0\\
			3b + 8c=0\\
		\end{cases} \iff
		\begin{cases}
			a+b+c = 0\\
			b+2c = 0\\
			2c = 0
		\end{cases}\iff
		\begin{cases}
			c=0\\
			b=0\\
			a=0
		\end{cases}
	\] 
\end{exm}
