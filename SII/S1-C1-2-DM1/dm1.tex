\documentclass{article}


\usepackage[french]{varioref}
\usepackage{lmodern}
\usepackage{amssymb}
\usepackage{amsmath}
\usepackage{mathtools}
\usepackage[utf8]{inputenc}
\usepackage[T1]{fontenc}
\usepackage{blindtext}
\usepackage[a4paper, margin=2cm]{geometry}
\usepackage{fancyhdr}
\usepackage{textcomp}
\usepackage{gensymb}
\usepackage{xcolor}
\usepackage[dvips]{graphicx}
\usepackage{epstopdf}

\newcommand{\question}[1]{\vspace{7mm}{\quad\quad\quad \textbf{Question #1:}}\\}
\newcommand{\nocapt}[0]{\vspace{5mm}}
\newcommand{\graph}[2][12.5cm]{
  \makebox[\textwidth]{
    \includegraphics[width=#1]{#2.eps}
  }
}
\newcommand{\capt}[1]{
  \centerline{
    \textit{#1}
  }
  \vspace{2mm}
}

\usepackage{xparse}
\usepackage{expl3}

\setlength{\parindent}{0cm}

\ExplSyntaxOn
\NewDocumentCommand{\biglittlecap}{m}
{
\sheljohn_biglittecap:nn { #1 }
}
\tl_new:N \l__sheljohn_biglittecap_input_tl
\cs_new_protected:Npn \sheljohn_biglittecap:nn #1
{
% store the string in a variable   
\tl_set:Nn \l__sheljohn_biglittecap_input_tl { #1 }
\regex_replace_all:nnN
% search a capital letter (or more)
{ ([A-Z]+ | \cC.\{?[A-Z]+\}?) }
% replace the match with \huge{match}
{ \cB\{\c{Large}\1\cE\} }   % <=== could use large, Large (or some other command...)
\l__sheljohn_biglittecap_input_tl
\tl_use:N \MakeUppercase{\l__sheljohn_biglittecap_input_tl}
}
\ExplSyntaxOff


\def\labelitemi{--}

\pagestyle{fancy}
\fancyhf{}
\rhead{SII}
\lhead{MP2I}
\chead{DM1}
\rfoot{Page \thepage}

\begin{document}
  \begin{titlepage}
    \begin{center}
      \vspace*{3.5cm}
      {
        \Huge
        \textbf{SII}
        \vspace{0.5cm}
        \(\text{DM}_\text{1}\)
      }
          
      \vspace{3cm}

      \textbf{{\Large H}ugo \biglittlecap{Salou}}\vspace{2mm}

      \textbf{{\Large A}ntoine \biglittlecap{Villoteau}}\vspace{2mm}

      \textbf{{\Large T}homas \biglittlecap{Leguéré}}

      \vfill
      
      MP2I - Novembre 2021
            
    \end{center}
  \end{titlepage}

  \question{4}
  \graph{q4}
  
  \question{5}
  \graph{q5}
  \capt{\(y=\theta_p(t)\)}

  \question{6}
  En environ 2 secondes, on a \(\varepsilon(2) = -0.156\degree\).
  Pour cette situation, c'est très précis.\\
  Non, le gain du correcteur ne modifie pas la précision du système.\\

  \question{7}
  \(5\% \times 90\degree = 4,5\degree\)\\
  \`A \(t = 0.57\text{s}\), on a \(\theta_p = 85.30\degree\).\\
  \`A \(t = 0.58\text{s}\), on a \(\theta_p = 85.56\degree\).\\
  Le temps de réponse est d'environ \(\Delta t = 0.58\text{s}\).\\
  Plus le gain du correcteur est grand, plus le temps de réponse diminue.\\

  \(\left(
    {\color{yellow} K_r=10}\;\;\;
    {\color{red} K_r=5}\;\;\;
    {\color{green} K_r=1}\;\;\;
    {\color{black} K_r=0.5}
  \right)\).

  \graph{q7}
  \capt{\(y=\theta_p(t)\)}

  \graph{q7b}
  \capt{\(y=\theta_p(t)\)}

  Ici, \(\tau \simeq 0.195s\).

  \question{8}
  \graph{q8}
  \capt{\(y=U_m(t)\)}
  \vspace{2mm}

  \`A \(t=2.94\text{s, } U_m = 100\text{V}\).\,
  \`A \(t=2.95\text{s, } U_m = 97.6\text{V}\).\;
  On a donc \(t_s \simeq 2.95\text{s}\).\\
  
  \graph[1.3\textwidth]{q8b}
  \capt{
    \(y_1=\varepsilon(t)\)\\
    \(y_2=U_m(t)\)\\
    \(y_3=\theta_p(t)\)\\
  }

  \question{9}
  \graph{q9}
  \capt{\(
    y = \frac{d\theta_p}{dt}(t)\\
  \)}

  \`A \(t=0\text{s}\), la vitesse \(\omega_m\) passe directement à \(200 \text{ rad}/\text{s}\).
  L'acceleration n'est pas progressive.

  On peut peu à peu augmenter la consigne pour avoir une accélération plus progressive.
  Au lieu d'utiliser un échelon, on peut utiliser une rampe.

  \question{10}
  
  On a \(t_r=2.99\text{s}\).\\
  On utilise donc
  \begin{itemize}
    \item une rampe infinie avec une pente \(\displaystyle\frac{\theta_{\text{max}}}{t_r}\) et un démarrage à \(t=0\text{ s}\)
    \item une rampe infinie avec une pente \(\displaystyle-\frac{\theta_{\text{max}}}{t_r}\) et un démarrage à \(t=t_r\)
  \end{itemize}
  \graph{q10}
  \capt{\(y_1=\theta_c(t)\) \\ \(y_2=\frac{d\theta_p}{dt}(t)\)}

  \question{11}
  \graph[1\textwidth]{q11}
  \capt{
    \(
      y_1 = \theta_p(t)\\
      y_2 = \frac{d\theta_p}{dt}(t)\\
      y_3 = \frac{d^2\theta_p}{dt^2}(t)\\
    \)
  }

  Pas de saturation, la tension du moteur reste en dessous de \(100\text{ V}\).

  \graph{q11b}
  \capt{\(y=U_m(t)\)}

  La tangente à l'origine du déplacement est horizontale, l'accélération est donc progressive (donc confortable).

  \question{12}
  \graph{q12}
  \capt{\(y_1=\theta_p(t) \\ y_2 = \frac{\theta_{\text{max}}}{tr}(t - \tau)\)}

  \question{13}
  \graph{q13}
  Au maximum, on a \(\varepsilon_t=9\text{ rad}\) pour \(t=2.5\text{s}\).\\
  \graph{q13b}
  
  \(\left(
    {\color{green} K_r=2}\;\;\;
    {\color{black} K_r=1}\;\;\;
    {\color{red} K_r=0.5}
  \right)\).\\

  En augmentant \(K_r\), on diminue \(\varepsilon_t\).

  \question{14}

  En régime permanent, le système est très précis: 
  en \(\Delta t = 5.0s\), on atteint \(\theta_p=89.999 \degree\).
  Pour son utilisation, c'est largement suffisant.

  \question{15}

  On a \(t_r = 3.01s\) avec \(K_r = 1\). Mais, augmenter \(K_r\) permet d'atteindre ce 
  critère à 5\% plus rapidement.

  \graph{q15}

  \(\left(
    {\color{green} K_r=4}\;\;\;
    {\color{black} K_r=1}
  \right)\).

\end{document}
