\part{Limites et inégalités}

\begin{prop}
	Soient $u$ et $v$ deux suites réelles convergentes de limites respectives $\ell_1$ et $\ell_2$. On suppose que \[
	\forall n \in \N, u_n \le  v_n
	\]
	Alors, $\ell_1\le\ell_2$
\end{prop}

\begin{prv}
	On suppose $\ell_1<\ell_2$. On pose $\varepsilon = \frac{\ell_1 -\ell_2}{3} > 0$.\\
	Il existe $N_1 \in \N$ tel que \[
	\forall n \ge  N_1, u_n \ge  \ell_1-\varepsilon
	\] Il existe $N_2\in \N$ tel que \[
	\forall n \ge  N_2, v_n \le  \ell_2 + \varepsilon
	\]\\
	Ainsi, \[
	\forall n \ge  \max(N_1,N_2), \ell_1 - \varepsilon \le  u_n \le  v_n\le  \ell_2 + \varepsilon
	\] et donc \[
	\ell_1-\varepsilon\le \ell_2+\varepsilon
	\] donc, $\ell_1 - \ell_2 \le 2 \varepsilon$\\
	donc, $1 \le \frac{2}{3}$ une contradiction
\end{prv}

\begin{rmk}
	Si $\begin{cases}
		u_n \to  \ell_1 \in \R\\
		v_n \to  \ell_2 \in \R\\
		\forall n \in \N, u_n < v_n
	\end{cases}$\\ on n'a pas forcément $\ell_1 < \ell_2$\\
	\ex $\forall n \in \N_*, \frac{1}{n+1}< \frac{1}{n}$ mais les deux convergent vers $0$
\end{rmk}

\begin{prop}
	Soient $u$ et $v$ deux suites réelles telles que \[
	\forall n\in \N, u_n < v_n
	\]
	\begin{enumerate}
		\item si $u_n\to  +\infty$,$v_n \to +\infty$
		\item si $v_n\to  -\infty$,$u_n \to -\infty$
	\end{enumerate}
\end{prop}

\begin{prv}
	\begin{enumerate}
		\item On suppose $u_n \to +\infty$. Soit $M \in \R$, il existe $N \in \N$ tel que \[
		\forall n \ge  N, u_n \ge M
		\] Donc \[
		\forall n \ge N, v_n \ge u_n \ge  M
		\] Donc $v_n \to  +\infty$
	\end{enumerate}
\end{prv}

\begin{thm}
	[Théorème des "gendarmes"]
	Soient $u$, $v$ et $w$ trois suites réelles telles que \[
	\forall n \in \N, u_n \le v_n \le w_n
	\]
	On suppose que $u$ et $w$ convergent vers la même limite $\ell\in \R$. Alors, $v$ converge vers $\ell$
\end{thm}

\begin{prv}
	Soit $\varepsilon>0$. Il existe $N_1 \in \N$ tel que \[
	\forall n \ge  N_1, w_n \le  \ell+ \varepsilon
	\] Il existe $N_2 \in \N$ tel que \[
	\forall n \ge N_2, u_n \le  \ell - \varepsilon
	\] On pose $N = \max(N_1,N_2)$. D'où, \[
	\forall n \ge  N, \ell- \varepsilon \le  u_n \le  v_n \le  w_n \le \ell+\varepsilon
	\]

	Donc, $v_n \tendsto{n \to +\infty} \ell$
\end{prv}

\begin{thm}
	[Limite monotone]
	\begin{enumerate}
		\item Soit $u$ une suite croissante majorée par $M$.\\ Alors, $u$ converge et $\lim u_n \le M$
		\item Soit $u$ une suite croissante non majorée.\\Alors, $u_n \tendsto{n \to +\infty}+\infty$
		\item Soit $u$ une suite décroissante minorée par $m$.\\Alors, $u$ converge et $\lim u_n \ge m$
		\item Soit $u$ une suite décroissante non minorée.\\Alors, $u_n \tendsto{n \to +\infty} -\infty$
	\end{enumerate}
\end{thm}

\begin{prv}
	\begin{enumerate}
		\item $\{u_n  \mid n \in \N\}  \neq \O$ ($u_0$ y est) majorée (par hypothèse) par $M$.\\
			On pose  $\ell = \sup_{n \in \N} (u_n)$. Soit $\varepsilon>0$ quelconque\\
			$\ell-\varepsilon<\ell$ donc, $\exists N\in \N, u_N>\ell-\varepsilon$\\
			$u$ est croissante donc \[
			\forall n \ge N, u_n\ge u_N>\ell-\varepsilon
			\] donc, \[
			\forall n\ge N, \ell-\varepsilon\le u_n \le \ell \le \ell+\varepsilon
			\]\\
			Donc, $u_n\to \ell$ 
		\item Soit $M\in \R$. $M$ n'est pas un majorant de l'ensemble $\{u_n \mid n \in \N\}$ donc \[
		\exists N \in \N, u_N > M
		\] Comme $u$ est croissante \[
		\forall n \ge N, u_n \ge u_N \ge M
		\] donc \[
		u_n \tendsto{n \to +\infty} +\infty
		\]
	\end{enumerate}
\end{prv}

\begin{exm}
	\[
	\begin{cases}
		u_0 = a \in \left]0, 1\right[\\
		\forall n \in \N, u_{n+1} = u_n(1-u_n)
	\end{cases}
	\] 
	\begin{center}
		(suite logistique)
	\end{center}
	\[
	 u_{n+1} = f(u_n) \text{ avec } f: x \mapsto x(1-x)
	 \]

	 \begin{figure}[H]
		\begin{center}
			\begin{asy}
				import graph;

				size(8cm);

				real f(real x) {return x * (1 - x); }
				real g(real x) {return x;}

				draw(graph(g, 0, .3));
				draw(graph(f, 0, 1), magenta);
				axes("$x$", "$y$", EndArrow);

				real a = 0.7;
				real u = a;

				draw((a,0)--(a,f(a)));

				for (int i = 0; i < 20; ++i) {
					real v = f(u);

					draw((u,v)--(v,v)--(v,f(v)));

					draw((v,v)--(v,0), dashed);
					draw((v,v)--(0,v), dashed);
					u = v;
				}
			\end{asy}
		\end{center}
		\caption{Courbe logistique}
	 	\label{logistique-curve}
	\end{figure}

		\begin{itemize}
			\item Soit $n\in \N$, 
				\begin{align*}
					u_{n+1} - u_n &= u_n (1-u_n) - u_n\\
					&= -{u_n}^2 \le 0 \\
				\end{align*}
				Donc, $u$ est décroissante.
			\item Montrons par récurrence que \[
				\forall n\in \N, u_n \in [0,1]
				\]
				\begin{itemize}
					\item $u_0 = a \in ]0,1[$ donc $u_0\in [0,1]$
					\item Soit $n\in N$. On suppose $u_n \in [0,1]$ \[
					\begin{cases}
						0 \le  u_n \le 1\\
						0 \le 1-u_n\le 1
					\end{cases}
					\] donc \[
					0 \le u_{n+1} \le 1
					\] 
				\end{itemize}
				donc $u$ minoré par $0$
			\item D'après le théorème de la limite monotone, $u$ converge. On pose $\ell$ sa limite : \[
			\ell=\lim_{n \to +\infty} u_n
			\] Alors, $u_{n+1} \tendsto{n\to +\infty} \ell$
			\[
				u_n(1-u_n) \tendsto{n\to +\infty} \ell(1-\ell)
			\]
			Par unicité de la limite, 
			\begin{align*}
				&\ell = \ell(1-\ell)\\
				\iff& 1= 1-\ell \\
				\iff& 0 = -\ell \iff \ell = 0
			\end{align*}
		\end{itemize}
\end{exm}

\begin{exm}
	\[
	\begin{cases}
		u_0 = a \in ]0,1[\\
		u_{n+1} = 2u_n(1-u_n)
	\end{cases}
	\]
	\begin{figure}[H]
		\begin{center}
			\begin{asy}
				import graph;

				size(8cm);

				real f(real x) {return 2 * x * (1 - x); }
				real g(real x) {return x;}

				draw(graph(g, 0, .6));
				draw(graph(f, 0, 1), magenta);
				axes("$x$", "$y$", EndArrow);

				real a = 0.7;
				real u = a;

				draw((a,0)--(a,f(a)));

				for (int i = 0; i < 20; ++i) {
					real v = f(u);

					draw((u,v)--(v,v)--(v,f(v)));

					draw((v,v)--(v,0), dashed);
					draw((v,v)--(0,v), dashed);
					u = v;
				}
			\end{asy}
		\end{center}
		\caption{Courbe logistique (2)}
	 	\label{logistique-curve2}
	\end{figure}
\end{exm}

\begin{exm}
	\[
	\begin{cases}
		u_0 = a \in ]0,1[\\
		u_{n+1} = 3u_n(1-u_n)
	\end{cases}
	\]
	\begin{figure}[H]
		\begin{center}
			\begin{asy}
				import graph;

				size(8cm);

				real f(real x) {return 3 * x * (1 - x); }
				real g(real x) {return x;}

				draw(graph(g, 0, .9));
				draw(graph(f, 0, 1), magenta);
				axes("$x$", "$y$", EndArrow);

				real a = 0.3;
				real u = a;

				draw((a,0)--(a,f(a)));

				for (int i = 0; i < 20; ++i) {
					real v = f(u);

					draw((u,v)--(v,v)--(v,f(v)));

					draw((v,v)--(v,0), dashed);
					draw((v,v)--(0,v), dashed);
					u = v;
				}
			\end{asy}
		\end{center}
		\caption{Courbe logistique (3)}
	 	\label{logistique-curve3}
	\end{figure}
\end{exm}

\begin{exm}
	\[
	\begin{cases}
		u_0 = a \in ]0,1[\\
		u_{n+1} = 4u_n(1-u_n)
	\end{cases}
	\]
	\begin{figure}[H]
		\begin{center}
			\begin{asy}
				import graph;

				size(8cm);

				real f(real x) {return 4 * x * (1 - x); }
				real g(real x) {return x;}

				draw(graph(g, 0, 1.2));
				draw(graph(f, 0, 1), magenta);
				axes("$x$", "$y$", EndArrow);
			\end{asy}
		\end{center}
		\caption{Courbe logistique (4)}
	 	\label{logistique-curve4}
	\end{figure}
\end{exm}

\begin{defn}
	Soient $u$ et $v$ deux suites réelles. On dit que $u$ et $v $ sont adjacentes si
	\begin{itemize}
		\item $u$ est croissante
		\item $v$ est décroissante
		\item $u_n-v_n\tendsto{n \to +\infty} 0$
	\end{itemize}
	\index{adjacence (suites)}
\end{defn}

\begin{thm}
	Soient $u$ et $v$ deux suites adjacentes.
	Alors, $u$ et $v$ convergent vers la même limite.
\end{thm}

\begin{prv}
	$u-v$ est croissante donc, \[
	\forall n \in \N, u_n - v_n \le  0
	\] $v$ décroissante donc \[
	\forall n\in \N, v_n \le v_0
	\] donc $u$ majorée par $v_0$ donc $u$ converge.\\
	$u$ est croissante donc, \[
	\forall n\in \N, u_n \ge u_0
	\] donc $v$ est minorée par $u_0$ donc $v$ converge.\\
	Donc, $u_n-v_n\to\lim(u_n)-\lim(v_n)$
	Par unicité de la limite,
	\begin{align*}
		&\lim(u_n)-\lim(v_n) = 0\\
		\iff& \lim(u_n) = \lim(v_n)
	\end{align*}
\end{prv}

\begin{thm}[Théorème des segments emboîtés]
	Soit $(I_n)$ une suite de segments (non vide) décroissante  \[
	\forall n\in \N, I_{n+1} \subset I_n
	\] On note $\ell(I)$ la longueur d'un intervalle $I$.\\
	Si $\ell(I_n)\tendsto{n\to+\infty}0$ alors $\bigcap_{n \in \N} I_n$ est un singleton.
\end{thm}


\begin{prv}
	On pose $\forall n \in \N, I_n = [a_n,b_n]$ avec $\forall n \in \N, a_n \le  b_n$.
	Soit $n\in \N$.\\
	$I_{n+1} \subset I_n$ donc $a_{n+1} \in I_{n+1} \subset I_n$\\
	donc $a_{n+1} \ge  a_n$. De même, $b_{n+1} \in I_{n+1}$ donc $b_{n+1} \in I_n$ donc $b_{n+1} \le  b_n$.\[
	\forall n\in \N, b_n-a_n = \ell(I_n) \tendsto{n \to  +\infty} 0
	\] donc $(a_n)$ et $(b_n)$ sont adjacentes, elles convergent donc vers la même limite $\ell\in \R$.\\
	$(a_n)$ croissante de limite $\ell$ donc \[
	\forall n \in \N, a_n \le \ell
	\] $(b_n)$ est décroissante de limite $\ell$ donc \[
	\forall n\in \N, b_n \ge  \ell
	\] Donc, $\forall n\in \N, \ell\in I_n$ donc $\ell\in \bigcap_{n\in \N}  I_n$.\\
	Soit $\ell' \neq \ell$.\\
	\begin{itemize}
		\item Si $\ell' < \ell = \sup_{n \in \N} (a_n)$ donc $\ell'$ ne majore pas $(a_n)$ \[
			\exists N\in \N, a_N > \ell'
			\] donc $\ell' \not\in I_N$ donc $\ell'\not\in \bigcap_{n \in \N} I_n$
		\item Si $\ell' > \ell = \inf_{n \in \N}(b_n)$ donc $\ell'$ ne minore pas $(b_n)$ \[
		\exists N' \in \N, b_{N'} < \ell'
		\] et donc $\ell'\not\in I_{N'}$ donc $\ell'\not\in \bigcap_{n \in  \N}  I_n$
	\end{itemize}
\end{prv}
