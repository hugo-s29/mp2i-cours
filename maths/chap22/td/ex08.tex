\part{Exercice 8}

On a \[
	\begin{cases}
		\frac{\partial g}{\partial x}(x,y) = \frac{1}{y}\left( e^{-x}-xe^{-x} \right) = \frac{e^{-x}}{y}(1-x)\\[3mm]
		\frac{\partial g}{\partial y}(x,y) = \frac{-xe^{-x}}{y^2} + \frac{1}{e}
	\end{cases}
\]

\begin{align*}
	\begin{rcases*}
		\frac{\partial g}{\partial x}(x,y) = 0\\[1mm]
		\frac{\partial g}{\partial y}(x,y) = 0
	\end{rcases*} \iff& \begin{cases}
		x = 1\\
		y = 1
	\end{cases}\\
	\iff& \begin{cases}
		x = 1\\
		y = 1
	\end{cases}
\end{align*}

On a \[
	g(1,1) = \frac{1}{e} + \frac{1}{e} = \frac{2}{e}.
\]
Or, \[
	g(1,2) = \frac{1}{2e} + \frac{2}{e} > \frac{2}{e}
\] et \[
	g\left( \frac{1}{2}, 1 \right) = \frac{1}{2}e^{-\frac{1}{2}} + \frac{1}{e} < \frac{2}{e}
\] car $\frac{1}{2}e^{-\frac{1}{2}} < e^{-1} \iff \frac{1}{2} < e^{-\frac{1}{2}} \iff 2 > e^{\frac{1}{2}} \iff 4 > e$.

Donc, $(1,1)$ n'est ni un minimum, ni un maximum pour $g$.
