\part{Exercice 11}

\begin{enumerate}
	\item 
		Soient $E$ un $\mathbbm{K}$-espace vectoriel de dimension finie $n$ et $f,g \in \mathcal{L}(E)$.

		\begin{align*}
			f_{|\mathrm{Im}(g)}: \mathrm{Im}(g) &\longrightarrow E \\
			x &\longmapsto f(x)
		\end{align*}

		Soit $x \in \mathrm{Im}(g)$.
		\begin{align*}
			x \in \Ker(f_{|\mathrm{Im}(g)}) \iff& f_{|\mathrm{Im}(g)}(x) = 0\\
			\iff& \begin{cases}
				f(x) = 0\\
				x \in \mathrm{Im}(g)
			\end{cases}\\
			\iff& x \in \Ker(f) \cap \mathrm{Im}(g)
		\end{align*}

		\[
			\Ker(f_{|\mathrm{Im}(g)}) = \Ker(f) \cap \mathrm{Im}(g)
		\]\[
			\dim(\mathrm{Im}\ g) = \dim\big(f(\mathrm{Im}\ g)\big) + \dim(\Ker f \cap \mathrm{Im}\ g)
		\] \[
			\dim\big(f(\mathrm{Im}\ g)\big) = \dim\big(\mathrm{Im}(f \circ g)\big) = \rg(f \circ g)
		\] \[
			\rg(f \circ g) = \rg(g) - \dim(\mathrm{Im}\ g \cap \Ker f)
		\] \[
			\dim(E) = n = \rg(g) + \dim(\Ker f)
		\] donc \[
			-n + \rg(f) = \dim(\Ker f) \le -\dim(\mathrm{Im}\ g \cap \Ker f)
		\] \[
			\rg(f \circ g) \ge \rg(g) + \rg(f) - n
		\]
	\item
		\begin{itemize}
			\item[\underline{\sc Analyse}]
				Soit $f \in \mathcal{L}(E)$ et $\dim(E) = 3$. On suppose $f \circ f = 0$.
				\[
					\rg(0) = \rg(f \circ f) \ge \rg(f) = \rg(f) + \rg(f) - n
				\] Donc \[
					0 \le 2\rg (f) - n
				\] \[
					3 = n \ge 2\rg(f)
				\] Donc, \[
					\rg(f) \in \{0, 1\} 
				\] et donc \[
					\dim(\mathrm{Im}\ f) \in \{1,0\}
				\] 
				Soit $u \in \mathrm{Im}(f) \setminus \{0\}$ donc $\Vect\ u = \mathrm{Im}\ f$. \[
					\forall x \in K, \exists \lambda(x) \in \mathbbm{K}, f(x) = \lambda(x) \cdot u
				\]
			\item[\underline{\sc Synthèse}]~\\
				\begin{itemize}
					\item $f = 0$ 
					\item On suppose $f \neq 0$. Soit $u \in E$, $\lambda \in \mathcal{L}(E, \mathbbm{K}) = E^*$.
						\begin{align*}
							f: E &\longrightarrow E \\
							x &\longmapsto \lambda(x) \cdot u
						\end{align*}
						Soit $x \in E$.
						\begin{align*}
							(f \circ f)(x) &= f\big(\lambda(x) \cdot u\big) \\
							&= \lambda\big(\lambda(x) \cdot u\big) \\
							&= \lambda(x) \lambda(u) \cdot u \\
						\end{align*}
						Soient $(x,y) \in E^2$ et $\alpha, \beta \in \mathbbm{K}^2$.
						\begin{align*}
							f(\alpha a\,&+ \beta y) = \lambda (\alpha x + \beta y) \cdot u\\
												 &\vrt=\\
							\alpha f(x) \,&+ \beta f(y) = \alpha \lambda(x) \cdot u + \beta \lambda(y) \cdot u
						\end{align*}
						Donc, \[
							\lambda(\alpha x + \beta y) \cdot u = \big(\alpha \lambda(x) + \beta \lambda(y)\big) \cdot u
						\] et donc \[
							\lambda (\alpha x + \beta y) = \alpha \lambda(x) + \beta \lambda(y)
						\]
						\[
							f(u) = u
						\] Soit $v \in E$ tel que $f(v) = u$. \[
							f(u) = f\big(f(v)\big) = 0 = \lambda (u) \cdot u
						\] Donc $\lambda(u) = 0$ et donc $u \in \Ker(\lambda)$.
				\end{itemize}
		\end{itemize}
\end{enumerate}

