\part{Exercice 16}

On note $\mathcal{P}$ l'ensemble des nombres premiers.
On pose \[
\begin{cases}
	a = \prod_{p \in \mathcal{P}} p^{\alpha_p}  \\
	b = \prod_{p \in \mathcal{P}} p^{\beta_p}  \\
\end{cases}
\] où $(\alpha_p)$ et $(\beta_p)$ sont presque nulles.\\


\begin{align*}
	a^2 \wedge ab \wedge b^2 &= \prod_{p \in \mathcal{P}} p^{\max(2\alpha_p, \alpha_p + \beta_p, 2\beta_p)}\\
	(*) &= \prod_{p \in \mathcal{P}} p^{2\max(\alpha_p, \beta_b)} \\
	&= \left( \prod_{p \in \mathcal{P}} p^{\max(\alpha_p, \beta_p)} \right)  ^2\\
	&= (a \wedge b)^2 \\
\end{align*}

(*): Soit $p \in \mathcal{P}$.\\
\begin{itemize}
	\item Si $\alpha_p \le \beta_p$ alors $\begin{cases}
			2\alpha_p \le 2\beta_p\\
			\alpha_p + \beta_p \le 2\beta_p
		\end{cases}$ 
		donc $\max(2\alpha_p, \alpha_p+\beta_p, 2\beta_p) = 2\beta_p = 2\max(\alpha_p,\beta_p)$
	\item Si $\beta_p < \alpha_p$, $\max(2\alpha_p,\alpha_p+\beta_p,2\beta_p) = 2\alpha_p = 2\max(\alpha_p,\beta_p)$
\end{itemize}
