\part{Trigonométrie hyperbolique}

\begin{defn}
	Pour tout $x \in \R$, on pose \[
		\begin{cases}
			\ch x = \frac{e^x + e^{-x}}{2},\\
			\sh x = \frac{e^x - e^{-x}}{2},\\
			\th x = \frac{\sh x}{\ch x}.
		\end{cases}
	\]

	$\ch$ est appelé \underline{cosinus hyperbolique}, $\sh$ est appelé \underline{sinus hyperbolique} et $\th$ est appelé \underline{tangeante hyperbolique}.
	\index{cosinus hyperbolique}
	\index{sinus hyperbolique}
	\index{tangente hyperbolique}
\end{defn}

\begin{rmk}
	Ces formules rappèlent les formules d'Euler : pour tout $x \in \R$,
	\begin{align*}
		\cos x = \frac{e^{ix} + e^{-ix}}{2}\quad\longleftrightarrow\quad\ch x = \frac{e^x + e^{-x}}{2}\\
		\sin x = \frac{e^{ix} - e^{-ix}}{2i}\quad\longleftrightarrow\quad\sh x = \frac{e^x - e^{-x}}{2}\\
	\end{align*}
\end{rmk}

\begin{figure}[H]
	\centering
	\begin{asy}
		import graph;

		size(12cm);

		pair A = (-2, 0);
		pair B = (2, 0);

		real eps = 0.05;

		draw(shift(A) * ((0, -1.3) -- (0, 1.3)), Arrow(TeXHead));
		draw(shift(A) * ((-1.3, 0) -- (1.3, 0)), Arrow(TeXHead));

		draw(circle(A, 1), magenta);
		
		real theta = 38;
		pair M = dir(theta) + A;
		draw(A -- M, red);
		draw(arc(A, 0.35, 0, theta), red, Arrow(TeXHead));
		draw(M -- (A.x-eps, M.y), dashed);
		draw(M -- (M.x, A.y-eps), dashed);
		label("\small$\theta$", 0.5dir(theta/2) + A, red);
		label("\small$\cos\theta$", (M.x, A.y), align=S);
		label("\small$\sin\theta$", (A.x, M.y), align=1.2W);
		dot("\small$M$", M);

		label("\small$x^2 + y^2 = 1$", A + 1.5dir(45+180));

		draw(shift(B) * ((0, -1.3) -- (0, 1.3)), Arrow(TeXHead));
		draw(shift(B) * ((-1.3, 0) -- (1.3, 0)), Arrow(TeXHead));

		real ch(real x) { return (exp(x) + exp(-x)) / 2.; }
		real sh(real x) { return (exp(x) - exp(-x)) / 2.; }
		real nch(real x) { return -ch(x); }

		real k = 1.9; real r = 1.2;
		real t = 1.4;

		draw(shift(B) * scale(0.35) * graph(ch, sh, -k, k), magenta);
		draw(shift(B) * scale(0.35) * graph(nch, sh, -k, k), magenta);

		label("\small$x^2 - y^2 = 1$", B + 1.5dir(45+180) + (0, -0.2));

		M = B + 0.35(ch(t), sh(t));

		draw(M -- (B.x-eps, M.y), dashed);
		draw(M -- (M.x, B.y-eps), dashed);
		dot("\small$M$", M);
		label("\small$\ch x$", (M.x, B.y), align=S);
		label("\small$\sh x$", (B.x, M.y), align=1.2W);

		draw(shift(B) * ((-r, -r)--(r,r)), gray + dashed);
		draw(shift(B) * ((r, -r)--(-r,r)), gray + dashed);
	\end{asy}
\end{figure}

